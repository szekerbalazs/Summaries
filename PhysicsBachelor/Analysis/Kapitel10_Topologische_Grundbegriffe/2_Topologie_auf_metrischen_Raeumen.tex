\subsection{Topologie auf metrischen Räumen}

\vspace{1\baselineskip}

\Definition{

    Sei $(X,d)$ ein metrischer Raum, \xinX und $r>0$ eine reelle Zahl, so schreiben wir
    \begin{align*}
        B(x,r) = \geschwungeneklammer{y \in X \ | \ d(x,y) < r}
    \end{align*}
    und nennen die Menge $B(x,r)$ \fat{offener Ball} oder \fat{offene Kreisscheibe}
    mit Zentrum $x$ und Radius $r$.
}

\vspace{1\baselineskip}

\Definition{

    Sei $(X,d)$ ein metrischer Raum. Eine Teilmenge $U \subseteq \X$ heisst \fat{offen},
    falls für jedes $x \in U$ ein $\epsilon>0$ existiert, derart, dass
    $B(x,\epsilon) \subseteq U$ gilt. Die damit definierte Topologie $\tau_d$ auf der
    Menge $X$ wird die von der Metrik \fat{induzierte Topologie} genannt.
}

\vspace{1\baselineskip}

\Lemma{

    Sei $(X,d)$ ein metrischer Raum.
    \begin{enumerate}
        \item Eine Teilmenge $U \subset X$ ist genau dann offen, wenn für jede
                konvergente Folge in $X$ mit Grenzwert in $U$ fast alle
                Folgenglieder in $U$ liegen.
        \item Eine Teilmenge $A \subset X$ ist genau dann abgeschlossen, wenn für
                jede konvergente Folge $\xFolge$ in $A$ mit $x_n \in A$ für alle
                \ninN auch der Grenzwert in $A$ liegt.
    \end{enumerate}
}

\vspace{1\baselineskip}

\Proposition{

    Sei $d \in \N$. Jede abgeschlossene Teilmenge von $\R^d$ ist vollständig.
}

\vspace{1\baselineskip}

\Definition{

    Seien $(X,d_X)$ und $(Y,d_Y)$ metrische Räume, und sei $f: X \rightarrow Y$ eine
    Funktion.
    \begin{enumerate}
        \item Wir sagen, $f$ ist \fat{$\epsilon-\delta-$stetig}, falls für alle \xinX
                und alle $\epsilon >0$ ein $\delta>0$ existiert, so dass
                $d_X (x,y) < \delta \Longrightarrow d_Y (f(x),f(y)) < \epsilon$ für alle
                \yinX gilt.
        \item Wir sagen, dass $f$ \fat{folgenstetig} ist, falls für jede konvergente
                Folge $\xFolge$ in $X$ mit Grenzwert $x = \limesninf x_n$ die Folge
                $(f(x_n))_{n}$ konvergiert, mit $\limesninf f(x_n) = f(x)$.
        \item Wir sagen, $f$ ist \fat{topologisch stetig}, falls für jede offene
                Teilmenge $U \subseteq Y$ das Urbild $f^{-1} (U)$ offen in $X$ ist.
    \end{enumerate}
}

\vspace{1\baselineskip}

\Proposition{

    Seien $X$ und $Y$ metrische Räume und sei $f: X \rightarrow Y$ eine Funktion. Dann
    sind folgende Aussagen äquivalent:

    (1) Die Funktion $f$ ist $\epsilon-\delta-$stetig.

    (2) Die Funktion $f$ ist folgenstetig.

    (3) Funktion $f$ ist topologisch stetig.
}

\vspace{1\baselineskip}

\Definition{

    Seien $(X,d_X)$ und $(Y,d_Y)$ metrische Räume und $f:X \rightarrow Y$ eine Funktion.
    Die Funktion $f$ heisst \fat{gleichmässig stetig}, falls es zu jedem $\epsilon>0$
    ein $\delta>0$ gibt, so dass für alle $x_1,x_2 \in X$ mit $d_X (x_1,x_2) < \delta$
    auch $d_Y (f(x_1),f(x_2)) < \epsilon$ gilt.
}

\vspace{1\baselineskip}

\Definition{

    Seien $(X,d_X)$ und $(Y,d_Y)$ metrische Räume und $f:X \rightarrow Y$ eine Funktion.
    Die Funktion $f$ heisst \fat{Lipschitz-stetig}, falls es eine reelle Zahl $L \geq 0$
    gibt, genannt \fat{Lipschitz-Konstante}, mit
    \begin{align*}
        d_Y (f(x_1),f(x_2)) \leq L d_X (x_1,x_2)
    \end{align*}
    für alle $x_1 , x_2 \in X$. Es gelten die Implikationen
    \begin{align*}
        f \text{ ist Lipschitz-stetig } \Longrightarrow
        f \text{ ist gleichmässig stetig } \Longrightarrow
        f \text{ ist stetig} 
    \end{align*}
    und im Allgemeinen sind die umgekehrten Implikationen falsch.
}

\vspace{1\baselineskip}

\Satz{ (Banach'sche Fixpunktsatz)

    Sei $(X,d)$ ein nicht-leerer, vollständiger metrischer Raum. Sei $T: X \rightarrow X$
    eine Abbildung und $0 \leq \lambda < 1$ eine reelle Zahl mit der Eigenschaft, dass
    \begin{align*}
        d(T(x_1),T(x_2)) \leq \lambda d(x_1 , x_2)
    \end{align*}
    für alle $x_1 , x_2 \in X$ gilt. Dann existiert genau ein Element $a \in X$ mit
    $T(a) = a$. Die Zahl $\lambda$ nennt man \fat{Lipschitz-Konstante} und die
    Funktion $T$ eine \fat{Lipschitz-Kontraktion}. Ein Punkt \xinX mit $T(x) = x$
    heisst \fat{Fixpunkt} der Abbildung $T$, und der Satz besagt also, dass eine
    Lipschitz-Kontraktion genau einen Fixpunkt besitzt.
}
