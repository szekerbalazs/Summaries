\subsection{Wegintegrale}

\vspace{1\baselineskip}

\Definition{

    Sei $U \subseteq \R^n$ eine offene Teilmenge und $\gamma: [a,b] \rightarrow U$
    ein stetig differenzierbarer Weg. Wir definieren die \fat{Länge} von $\gamma$ als
    \begin{align*}
        L(\gamma) = \intab \Norm{\gamma'(t)} dt
    \end{align*}
    Hier soll $\Norm{\gamma'(t)}$ als \fat{Geschwindigkeit} des Weges zur Zeit $t$
    gelesen werden.
}

\vspace{1\baselineskip}

\Definition{

    Eine stetige Funktion $\gamma:[a,b] \rightarrow \R^n$ heisst \fat{stückweise
    stetig differenzierbar}, falls eine Zerlegung von $[a,b]$: $a=s_0 < s_1 < \dots < s_N = b$
    existiert, so, dass $\gamma |_{[s_{k-1},s_k]}$ für alle
    $k \in \geschwungeneklammer{1,\dots,N}$ stetig differenzierbar ist. Wir sagen
    in dem Fall $\gamma$ sei \fat{stückweise stetig differenzierbar bezüglich}
    dieser Zerlegung, und definieren die \fat{Länge} von $\gamma$ durch
    \begin{align*}
        L(\gamma) = \sum_{k=1}^{\N} \int_{S_{k-1}}^{s_k} \Norm{(\gamma |_{[s_{k-1},s_k]})'(t)} dt
    \end{align*}
}

\vspace{1\baselineskip}

\Definition{

    Sei $f: U \rightarrow \R$ eine stetige reellwertige Funktion. Entlang eines
    stetigen differenzierbaren Weges $\gamma:[a,b] \rightarrow U$ definieren wir
    das \fat{skalare Wegintegral} als
    \begin{align*}
        \intab f(\gamma(t)) \Norm{\gamma'(t)} dt
    \end{align*}
}

\vspace{1\baselineskip}

\Definition{

    Wie benutzen den Term \fat{Gebiet} für offene zusammenhängende Teilmengen
    von $\R^n$.
}

\vspace{1\baselineskip}

\Definition{

    Sei $U \subseteq \R^n$ offen. Ein \fat{Vektorfeld} auf $U$ ist eine Funktion
    $F: U \rightarrow \R^n$. Stetige, stetig differenzierbare oder glatte
    Vektorfelder sind dementsprechend Funktionen $F: U \rightarrow \R^n$.
}

\vspace{1\baselineskip}

\Definition{

    Sei $U \subseteq \R^n$ ein offene Teilmenge und sei $F: U \rightarrow \R^n$ ein
    stetiges Vektorfeld. Wir definieren das \fat{Arbeitsintegral/Wegintegral des
    Vektorfeldes} $F$ entlang eines stückweise stetig differenzierbaren Weges
    $\gamma: [a,b] \rightarrow U$ durch
    \begin{align*}
        \int_{\gamma} F dt = \intab \scalprod{F(\gamma(t))}{\gamma'(t)} dt
    \end{align*}
    Ist $\gamma$ stückweise stetig differenzierbar bezüglich einer Zerlegung
    $a = t_0 < t_1 < \dots < t_N = b$, so ist das Integral als Summe von
    Integralen über Intervalle $[t_{k-1} , t_k]$ zu lesen.
}

\vspace{1\baselineskip}

\Definition{

    Eine stetige, orientierungserhaltende Reparametrisierung von $\gamma:[a,b]
    \rightarrow U$ ist eine Verknüpfung $\gamma \circ \psi$ mit
    \begin{align*}
        [c,d] \stackrel{\psi}{\longrightarrow} [a,b] \stackrel{\gamma}{\longrightarrow} U
    \end{align*}
    wobei $\psi$ stetig ist und $\psi(c) = a$ und $\psi(d) = b$.
}

\vspace{1\baselineskip}

\Lemma{

    Sei $U \subseteq \R^n$ eine offene Teilmenge, $f: U \rightarrow \R^n$ ein
    stetiges Vektorfeld und sei $\gamma:[a,b] \rightarrow \R^d$ ein stetig
    differenzierbarer Weg. Dann ändert sich der Wert des Wegintegrals
    $\int_{\gamma} F dt$ nicht unter orientierungserhaltenden Reparametrisierungen
    von $\gamma$.
}

\vspace{1\baselineskip}

\Lemma{

    Sei $U \subset \R^n$ offen und sei $\gamma:[0,1] \rightarrow U$ ein stückweise
    stetig differenzierbarer Weg. Dann gibt es eine stetig differenzierbare
    Reparametrisierung $\varphi = \gamma \circ \psi$. Darüberhinaus kann man
    $\varphi'(0) = \varphi'(1)=0$ einrichten.

}

\vspace{1\baselineskip}

\Definition{

    Sei $U \subseteq \R^n$ offen und $F: U \rightarrow \R^n$ ein stetiges
    Vektorfeld. Eine stetige differenzierbare Funktion $f: U \rightarrow \R$
    heisst \fat{Potential} für $F$, falls $F = \grad (f)$ gilt.
    Ist $f$ ein Potential, so ist auch $f+c$ für $c \in \R$ ein Potential. Ist $U$
    zusammenhängend, so ist jedes weitere Potential von dieser Form.
}

\vspace{1\baselineskip}

\Proposition{

    Sei $U \subseteq \R^n$ offen und sei $F:U \rightarrow \R^n$ ein stetiges
    Vektorfeld. Angenommen es existiert ein Potential $f:U \rightarrow \R$ für $F$.
    Dann gilt:
    \begin{align*}
        \int_{\gamma} F dt = f(\gamma(1)) - f(\gamma(0))
    \end{align*}
    für jeden stückweise stetig differenzierbaren Pfad $\gamma:[0,1] \rightarrow U$.
}

\vspace{1\baselineskip}

\Definition{

    Sei $U \subseteq \R^n$ offen und sei $F:U \rightarrow \R^n$ ein stetiges
    Vektorfeld. Dann heisst $F$ \fat{konservativ}, falls für alle stückweise
    stetig differenzierbaren Wege $\gamma:[0,1] \rightarrow U$ und
    $\eta:[0,1] \rightarrow U$ die Implikation
    \begin{align*}
        \gamma(0) = \eta(0)
        \quad \text{   und   } \quad
        \gamma(1) = \eta(1)
        \quad \Longrightarrow \quad
        \int_{\gamma} F dt = \int_{\eta} F dt
    \end{align*}
    gilt.
}

\vspace{1\baselineskip}

\Satz{

    Sei $U \subseteq \R^n$ ein Gebiet und $F: U \rightarrow \R^n$ ein stetiges
    Vektorfeld. Dann ist $F$ genau dann konservativ, wenn $F$ ein Potential besitzt.
}

\vspace{1\baselineskip}

\Korollar{

    Sei $U \subseteq \R^n$ offen, und sei $F$ ein stetig differenzierbares
    konservatives Vektorfeld auf $U$, mit Komponenten $F_1 , \dots , F_n$.
    Dann gilt
    \begin{align*}
        \partial_j F_k = \partial_k F_j
    \end{align*}
    für alle $j,k \in \geschwungeneklammer{1,\dots,n}$. Man nennt diese Gleichung
    auch \fat{Integrabilitätsbedingung}. Ausgeschrieben gilt dann
    \begin{align*}
        \partial_j F_k = \partial_j \partial_k f
        = \partial_k \partial_j f = \partial_k F_j
    \end{align*}
}

\vspace{1\baselineskip}

\Satz{

    Sei $U \subseteq \R^n$ offe, und sei $F: U \rightarrow \R^n$ ein stetig
    differenzierbares Vektorfeld, das den Integrabilitätsbedingung
    \begin{align*}
        \partial_k F_j = \partial_k F_j
    \end{align*}
    für alle $j,k \in \geschwungeneklammer{1,\dots,n}$ genügt. Seien
    $\gamma_0 :[0,1] \rightarrow U$ und $\gamma_1 : [0,1] \rightarrow U$ stückweise
    stetig differenzierbare Pfade mti dem selben Anfangspunkt $x_0$ und dem selben
    Endpunkt $x_1$. Sind $\gamma_0$ und $\gamma_1$ homotop, so gilt
    \begin{align*}
        \int_{\gamma_0} F dt = \int_{\gamma_1} F dt
    \end{align*}
}

\pagebreak

\Korollar{

    Sei $U \subseteq \R^n$ offen und einfach zusammenhängend. Ein stetig differenzierbares
    Vektorfeld auf $U$ ist genau dann konservativ, falls es den Integrabilitätsbedingungen
    genügt.
}

\vspace{1\baselineskip}

\Lemma{

    Sei $U \subseteq \R^n$ offen und konvex, und sei $F: U \rightarrow \R^n$ ein
    stetig differenzierbar Vektorfeld, das den Integrabilitätsbedingungen genügt.
    Dann ist $F$ konservativ.
}

\vspace{1\baselineskip}

\Definition{

    Sei $U \subseteq \R^n$ offen und $F:U \rightarrow \R^n$ ein stetiges Vektorfeld.
    Sei $V \subseteq \R^m$ offen und $\varphi:V \rightarrow U$ eine stetig
    differenzierbare Funktion. Das Vektorfeld $\varphi^{*} F$ auf $V$, gegeben durch
    \begin{align*}
        \varphi^{*}F : x \mapsto \sum_{k=1}^{m} \scalprod{\partial_k \varphi (x)}{F(\varphi(x))} e_k
    \end{align*}
    heisst \fat{Pullback} von $F$, oder das entlang $\varphi$ \fat{zurückgezogene}
    Vektorfeld. Die dadurch entstehende Abbildung
    \begin{align*}
        \varphi^{*}: \geschwungeneklammer{\text{Stetige Vektorfelder auf } U}
        \rightarrow \geschwungeneklammer{ \text{Stetige Vektorfelder auf } V}
    \end{align*}
    nennen wir \fat{Zurückziehen} von Vektorfeldern.
}

\vspace{1\baselineskip}

\Proposition{

    Seien $U \subseteq \R^n$ und $V \subseteq \R^m$ und $\varphi: V \rightarrow U$
    offen und sei $\varphi:V \rightarrow U$ eine stetig differenzierbare Funktion.
    \begin{enumerate}[{(1)}]
        \item Die vorher betrachtete Abbildung $\varphi^{*}$ ist linear.
        \item Sei $W \subseteq \R^p$ offen und $\psi:W \rightarrow V$ stetig differenzierbar.
                Dann gilt $\psi^{*} \varphi^{*} F = (\varphi \circ \psi)^{*} F$ für jedes
                stetige Vektorfeld $F$ auf $U$. Ausserdem gilt $\text{id}_U^{*} F = F$.
        \item Sei $\gamma:[0,1] \rightarrow V$ ein stückweise stetig differenzierbarer Weg.
                Dann gilt für jedes stetige Vektorfeld $F$ auf $U$
                \begin{align*}
                    \int_{\gamma} \varphi^{*} F dt = \int_{\varphi \circ \gamma} F dt
                \end{align*}
        \item Ist $F$ ein Vektorfeld von Klasse $C^k$ auf $U$ und $\varphi:V \rightarrow U$
                von Klasse $C^{k+1}$, dann ist $\varphi F$ von Klasse $C^k$.
        \item Sei $F$ ein Vektorfeld von Klasse $C^1$ auf $U$ und $\varphi:V \rightarrow U$
                von Klasse $C^2$. Erfüllt $F$ die Integrabilitätsbedingungen, dann erfüllt
                $\varphi^{*}F$ ebenfalls die Integrabilitätsbedingungen.
    \end{enumerate}
}

\vspace{1\baselineskip}

\Lemma{

    Sei $f:\R^n \rightarrow \R^m$ stetig mit kompaktem Träger $K =
    \geschwungeneklammer{x \in \R \ | \ f(x) \neq 0}$, sei $\epsilon>0$ und $\delta>0$.
    Es existiert eine glatte Funktion $\tilde{f}:\R^n \rightarrow \R^m$ derart, dass
    für alle $x \in \R^n$ folgendes gilt
    \begin{align*}
        \Norm{f(x)-\tilde{f}(x)} < \epsilon
        \quad \quad \text{     und     } \quad \quad
        B(x,\delta) \cap K = \emptyset \Longrightarrow \tilde{f}(x) = 0
    \end{align*}
}

\vspace{1\baselineskip}

\Lemma{

    Seien $\gamma_0 , \gamma_1 \in \Omega$. Sind die Pfade $\gamma_0$ und $\gamma_1$
    homotop, so existiert ein Pfad $\varpi: [0,1] \rightarrow \Omega$ mit
    $\varphi(0) = \gamma_0$ und $\varphi(1) = \gamma_1$.
}
