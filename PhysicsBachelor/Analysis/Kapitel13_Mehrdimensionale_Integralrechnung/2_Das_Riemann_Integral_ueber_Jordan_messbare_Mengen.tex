\subsection{Das Riemann-Integral über Jordan-messbare Mengen}

\vspace{1\baselineskip}

\Definition{

    Eine Teilmenge $B$ von $\R^n$ heisst \fat{Jordan-messbar}, falls es einen
    abgeschlossenen Quader $Q$ in $\R^n$ mit $B \subseteq Q$ gibt, so dass die
    charakteristische Funktion $\mathds{1}_B$ auf $Q$ Riemann-integrierbar ist. Das
    \fat{Volumen} oder \fat{Jordan-Mass} von $B$ ist in diesem Fall wie folgt definiert
    \begin{align*}
        \vol (B) = \int_Q \mathds{1}_B dx
    \end{align*}
    Dieses ist unabhängig von der Wahl von $Q$, solange $B \subseteq Q$ gilt.
}

\vspace{1\baselineskip}

\Bemerkung{

    Jordan-Messbarkeit $\neq$ Lebesque-Messbarkeit.
}

\vspace{1\baselineskip}

\Korollar{ (zum Lebesgue-Kriterium)

    Eine Teilmenge $B \subset \R^n$ ist genau dann Jordan-messbar, wenn $B$ beschränkt
    ist und der Rand $\partial B$ eine Nullmenge ist. Sind $B_1 , B_2 \subset \R^n$
    Jordan-messbar, so sind auch $B_1 \cup B_2 , B_1 \cap B_2$ und $B_1 \backslash B_2$
    Jordan-messbar.
}

\vspace{1\baselineskip}

\Proposition{

    Sei $Q \subset R^{n-1}$ ein abgeschlossener Quader und seien $f_- , f_+ :
    Q \rightarrow \R$ Riemann integrierbar, und sei $D \subseteq \Q$ Jordan-messbar.
    Dann ist die folgende Menge Jordan-messbar
    \begin{align*}
        B = \geschwungeneklammer{(x,y) \in \R^n \ | \ x \in D , f_- (x) \leq y \leq f_+ (x)}
    \end{align*}
}

\vspace{1\baselineskip}

\Definition{

    Sei $B \subseteq \R^n$ eine Jordan-messbare Teilmenge und sei $f$ eine reellwertige
    Funktion auf $B$. Dann heisst $f$ \fat{Riemann-integrierbar}, falls es einen
    abgeschlossenen Quader $Q \subseteq \R^n$ mit $B \subseteq Q$ gibt, so dass die
    durch
    \begin{align*}
        f_{!}(x) = \begin{cases}
            f(x) \quad &\text{ falls } x \in B \\
            0 \quad &\text{ falls } x \in Q \backslash B
        \end{cases}
    \end{align*}
    gegebene Funktion $f_{!}:Q \rightarrow \R$ Riemann-integrierbar ist. Wir
    schreiben in diesem Fall
    \begin{align*}
        \int_B f \ dx \ = \ \int_Q f_{!} \ dx
    \end{align*}
    und nennen diese Zahl das \fat{Riemann-Integral} von $f$ über $B$.
    Dieses ist unabhängig von der Wahl von $Q$ und es gilt Linearität, Monotonie
    und Dreiecksungleichung für $\int_B (-) \ dx$.
}

\vspace{1\baselineskip}

\Korollar{ (Lebesgue-Kriterium)

    Sei $B \subseteq \R^n$ Jordan-messbar und sei $f:B \rightarrow \R$ beschränkt.
    Dann ist $f$ genau dann Riemann-integrierbar, wenn $f$ auf $B$ fast überall
    stetig ist, das heisst, wenn die Menge der Unstetigkeitsstellen von $f$ eine
    Nullmenge ist. Insbesondere ist jede beschränkte Funktion auf einer
    Jordan-messbaren Menge Riemann-integrierbar.
}

\vspace{1\baselineskip}

\Proposition{

    Sind $B_1 , B_2 \subseteq \R^n$ Jordan-messbar und $f: B_1 \cup B_2 \rightarrow \R$
    eine Riemann-integrierbare Funktion. Dann sind $f |_{B_1}$ und $f |_{B_2}$
    Riemann-integrierbar und es gilt
    \begin{align*}
        \int_{B_1 \cup B_2} f \ dx = \int_{B_1} f \ dx + \int_{B_2} f\ dx - \int_{B_1 \cap B_2} f \ dx
    \end{align*}
}

\vspace{1\baselineskip}

\Satz{ (Fubini)

    Seien $P \subseteq \R^n$ und $Q \subseteq \R^m$ abgeschlossene Quader, und
    $f: P \times Q \rightarrow \R$ eine Riemann-integrierbare Funktion. Für
    $x \in P$, schreibe $f_x : Q \rightarrow \R$ für die Funktion
    $f_x (y) = f(x,y)$, und definiere
    \begin{align*}
        F_- (x) = \sup \mathcal{U} (f_x)
        \quad \quad \text{   und   } \quad \quad
        F_+ (x) = \inf \mathcal{O} (f_x)
    \end{align*}
    Es existiert eine Nullmenge $N \subseteq P$ so, dass für alle $x \notin N$ die
    Funktion $f_x$ Riemann-integrierbar ist, also
    \begin{align*}
        F_- (x) = F_+ (x) = \int_Q f_x (y) dy = \int_Q f(x,y) dy
    \end{align*}
    gilt. Die Funktion $F_-$ und $F_+$ auf $P$ sind beide Riemann-integrierbar, und
    es gilt
    \begin{align*}
        \int_{P \times Q} f(x,y) d(x,y) = \int_P F_- (x) dx = \int_Q F_+ (x) dx
        = \int_P \int_Q f(x,y) dy dx
    \end{align*}
}

\vspace{1\baselineskip}

\Korollar{

    Sei $Q = [a_1,b_1] \times \dots \times [a_n , b_n]$ ein Quader und sei
    $f:Q \rightarrow \R$ Riemann-integrierbar. Dann gilt
    \begin{align*}
        \int_Q f dx = \int_{a_1}^{b_1} \dots \int_{a_n}^{b_n} f(x_1 , \dots , x_n) dx_n \dots dx_1
    \end{align*}
    falls alle Parameterintegrale existieren. Andernfalls können die Parameterintegrale
    durch Suprema von Untersummen oder auch durch Infima von Obersummen ersetzt werden.
}

\vspace{1\baselineskip}

\Korollar{

    Sei $D \subseteq \R^{n-1}$ eine Jordan-messbare Menge, seien
    $\varphi_- , \varphi_+ : D \rightarrow \R$ Riemann-integrierbar und sei
    $B \subseteq \R^n$ die Jordan-messbare Teilmenge
    \begin{align*}
        B = \geschwungeneklammer{(x,y) \in D \times \R \ | \ \varphi_- (x) \leq y \leq \varphi_+ (x)}
    \end{align*}
    Für jede Riemann-integrierbare Funktion $f$ auf $B$ gilt
    \begin{align*}
        \int_B f(x,y) d(x,y) = \int_D \klammer{\int_{\varphi_- (x)}^{\varphi_+ (x)} f(x,y) dy} dx
    \end{align*}
}

\vspace{1\baselineskip}

\Korollar{ (Prinzip von Cavalieri)

    Sei $B \subseteq [a,b] \times \R^{n-1}$ eine beschränkte und Jordan-messbare Menge.
    Dann gilt
    \begin{align*}
        \vol (B) = \intab \vol (B_t) dt
    \end{align*}
    wobei für $t \in [a,b]$ die Teilmenge $B_t \subseteq \R^{n-1}$ durch
    $B_t = \geschwungeneklammer{y \in \R^{n-1} \ | \ (t,y) \in B}$
    gegeben ist, und für fast alle $t \in [a,b]$ Jordan-messbar ist.
}

\vspace{1\baselineskip}

\Bemerkung{

    Ist $C \subseteq \R^n$ eine Jordan-messbare Teilmenge, $\lambda \in \R$, dann ist
    $\lambda C = \geschwungeneklammer{\lambda x \ | \ x \in C}$ auch messbar und es
    gilt $\vol (\lambda C) = \abs{\lambda}^n \cdot \vol (C)$.
}
