\subsection{Mehrdimensionale Substitutionsregel}

\vspace{1\baselineskip}

\Definition{

    Sei $n \geq 1$ und $U \subseteq \R^n$ offen. Der \fat{Träger} oder \fat{Support}
    einer Funktion $f:U \rightarrow \R$ ist die Menge
    \begin{align*}
        \supp (f) = \overline{\geschwungeneklammer{x \in U \ | \ f(x) \neq 0}} \subseteq U
    \end{align*}
    Achtung! Der Abschluss liegt in $U$. Im Allgemeinen ist dies nicht das selbe wie
    der Abschluss davon in $\R^n$. Wir sagen, dass $f$ \fat{kompakten Träger} hat,
    falls $\supp (f)$ eine kompakte Teilmenge von $U$ ist. Hat $f$ kompakten Träger,
    so ist $\supp (f)$ beschränkt und damit ist $f(x) = 0$ für alle $x \in U \backslash Q$
    für ein grosses Quader $Q$. Wir definieren weiter das Integral
    \begin{align*}
        \int_U f dx = \int_{Q \cap U} f |_{Q \cap U} dx
    \end{align*}
    als Integral über $f$ über einen Quader, der $\supp (f)$ enthält, falls es existiert,
    und nennen in dem Fall $f$ Riemann-integrierbar. Dabei braucht $U$ nicht unbedingt
    beschränkt oder Jordan-messbar zu sein.
}

\vspace{1\baselineskip}

\Satz{

    Seien $X,Y \subseteq \R^n$ offene Teilmengen, sei $\Phi: X \rightarrow Y$ ein
    $C^1$-Diffeomorphismus und sie $f: Y \rightarrow \R$ eine Riemann-integrierbare
    Funktion mit kompaktem Träger. Dann ist die Funktion $\Phi^{*} f: X \rightarrow \R$
    gegeben durch $(\Phi^{*} f)(x) = (f \circ \Phi(x)) \cdot \abs{\det (D \Phi (x))}$
    Riemann-integrierbar, hat kompakten Träger, und es gilt
    \begin{align*}
        \int_Y f(y) dy = \int_X (f \circ \Phi (x)) \cdot \abs{\det (D \Phi (x))} dx
    \end{align*}
}

\vspace{1\baselineskip}

\Proposition{

    Seien $X$ und $Y$ offene Teilmengen von $\R^n$ und sei $\Phi: X \rightarrow Y$
    ein Homeomorphismus. Sei $f:Y \rightarrow \R$ eine Funktion und schreibe
    $g = f \circ \Phi$.
    \begin{enumerate}
        \item Ist $f$ Riemann-integrierbar, so ist auch $g$ Riemann-integrierbar.
        \item Es gilt $\supp(g) = \Phi^{-1} (\supp (f))$.
    \end{enumerate}
}

\vspace{1\baselineskip}

\Lemma{

    Sei $T: \R^n \rightarrow \R^n$ eine invertierbare lineare Abbildung die durch eine
    obere Dreiecksmatrix gegeben ist, und sei $f: \R^n \rightarrow \R$ eine stetige
    Funktion mit kompaktem Träger. Dann gilt
    \begin{align*}
        \int_{\R^n} f(x) dx = \abs{\det (T)} \int_{\R^n} f(T(x)) dx
    \end{align*}
}

\vspace{1\baselineskip}

\Lemma{

    Sei $L: \R^n \rightarrow \R^n$ eine invertierbare lineare Abbildung und
    $f: \R^n \rightarrow \R$ eine Riemann-integrierbare Funktion. Dann ist
    $f \circ L$ Riemann-integrierbar, und es gilt
    \begin{align*}
        \int_{\R^n} f(x) dx = \abs{\det (L)} \int_{\R^n} f(L(x)) dx
    \end{align*}
}

\Korollar{

    Sei $L: \R^n \rightarrow \R^n$ eine lineare Abbildung und $B \subseteq \R^n$ eine
    Jordan-messbare Teilmenge. Dann ist $L(B)$ Jordan-messbar, und es gilt
    $\vol(L(B)) = \abs{\det L} \vol (B)$.
}

\pagebreak

\Korollar{

    Sei $L \in \Mat_n (\R)$ mit Spalten $v_1 , \dots , v_n \in \R^n$. Dann ist das
    \fat{Parallelotop}
    \begin{align*}
        P = L([0,1]^n) = \geschwungeneklammer{\sum_{i=1}^{n} s_i v_i \ | \ 0 \leq s_i \leq 1}
    \end{align*}
    Jordan-messbar und es gilt $\vol (P) = \abs{\det (L)} = \sqrt{\text{gram}(v_1 ,\dots, v_n)}$.
    Hierbei ist gram die \fat{Gramsche Determinante}.
}

\vspace{1\baselineskip}

\Lemma{

    Seien $X \subseteq \R^n$ und $Y \subseteq \R^n$ offen, und sei $\Phi:X \rightarrow Y$
    ein $C^1$-Diffeomorphismus. Sei $Q_0 \subseteq X$ ein achsenparalleler abgeschlossener
    Würfel mit Kantenlänge $2r>0$ und Mittelpunkt $x_0 \in X$. Wir setzen
    $y_0 = \Phi(x_0)$, $L = D \Phi (x_0)$ und
    \begin{align*}
        \sigma = \max \geschwungeneklammer{\Norm{D \Phi (x) - L}_{\text{op}} \ | \ x \in Q_0}
    \end{align*}
    Für jede reelle Zahl $s$ mit $\sigma \Norm{L^{-1}}_{\text{op}} \sqrt{n} \leq s < 1$
    gilt
    \begin{align*}
        y_0 + (1-s)L(Q_0 - x_0) \subseteq \Phi (Q_0) \subseteq y_0 + (1+s)L(Q_0-x_0)
    \end{align*}
}

\vspace{1\baselineskip}

\Lemma{

    Seien $X,Y \subseteq \R^n$ offen, sei $\Phi:X \rightarrow Y$ ein $C^1$-Diffeomorphismus
    und sei $K_0 \subseteq X$ eine kompakte Teilmenge. Dann existiert für jedes
    $\epsilon \in (0,1)$ ein $\delta > 0$ mit folgender Eigenschaft. Für jeden Würfel
    $Q_0$ mit Mittelpunkt $x_0$, Kantenlänge kleiner als $\delta$ und $Q_0 \cap K_0
    \neq \emptyset$ gilt
    \begin{align*}
        \frac{\vol (\Phi (Q_0))}{1 + \epsilon} \leq \abs{\det D \Phi (x_0)} \vol (Q_0)
        \leq \frac{\vol (\Phi (Q_0))}{1- \epsilon}
    \end{align*}
}
