\subsection{Der Divergenzsatz und der Satz von Green in der Ebene}

\vspace{1\baselineskip}

\Definition{

    Sei $U \subseteq \R^n$ offen und sei $F: U \rightarrow \R^n$ ein differenzierbares
    Vektorfeld auf $U$. Die \fat{Divergenz} (Quellenstärke) von $F$ ist die Funktion
    div$(F) : U \rightarrow \R$ gegeben durch
    \begin{align*}
        \text{div} (F) = \text{tr} (Df) = \partial_1 F_1 + \partial_2 F_2 + \dots + \partial_n F_n
    \end{align*}
}

\vspace{1\baselineskip}

\Definition{

    Sei $\gamma = (\gamma_1 , \gamma_2) : [a,b] \rightarrow \R^2$ ein stückweise stetig
    differenzierbarer Pfad. Sei $t \in [a,b]$ so, dass $\gamma$ bei $t$ differenzierbar
    ist. Wir schreiben
    \begin{align*}
        n_{\gamma} (t) = \klammer{\gamma_2' (t) , - \gamma_1' (t)}
    \end{align*}
    und bezeichnen diesen Vektor als \fat{Aussennormale} oder \fat{Normalenvektor}
    an $\gamma$ im Punkt $\gamma (t)$. Dieser Vektor zeigt bezüglich der Laufrichtung
    nach rechts. Wichtig ist, dass die Laufrichtung immer im gegenuhrzeigersinn
    ist, damit die Aussennormale tatsächlich nach aussen zeigt.
    Für $v = (v_1 , v_2) \in \R^2$ ist gilt:
    \begin{align*}
        \scalprod{v}{n_{\gamma'(t)}} = v_1 \gamma_2' (t) - v_2 \gamma_1' (t)
        = \det \begin{pmatrix}
            v_1 & \gamma_1' (t) \\
            v_2 & \gamma_2' (t)
        \end{pmatrix}
        = v \times \gamma' (t)
        = v \wedge \gamma' (t)
    \end{align*}
}

\vspace{1\baselineskip}

\Definition{

    Sei $U \subseteq \R^n$ offen, sei $F: U \rightarrow \R^n$ ein stetiges Vektorfeld
    auf $U$ und $\gamma: [a,b] \rightarrow \R^2$ ein stückweise stetig differenzierbarer
    Pfad. Das \fat{Flussintegral} von $F$ entlang $\gamma$ ist das Integral
    \begin{align*}
        \int_{\gamma} F \ dn_{\gamma} = \intab \scalprod{F (\gamma (t))}{n_{\gamma} (t)} dt
    \end{align*}
}

\vspace{1\baselineskip}

\Definition{

    Sei $U \subseteq \R^2$ offen und $B \subseteq U$ beschränkt und zusammenhängend
    mit glattem Rand. Sei $F: U \rightarrow \R^2$ von Klasse $C^1$. Dann gilt
    \begin{align*}
        \int_B \text{div} F(x) \ dx = \int_{\partial B} F \ dn = \int_{\gamma} F \ dn_{\gamma}
    \end{align*}
    für eine Kurve $\gamma: [a,b] \rightarrow U$ die $\partial B$ parametrisiert.
}

\vspace{1\baselineskip}

\Bemerkung{

    Das Flussintegral bleibt unter Reparametrisierung eines Pfades unverändert.
}

\vspace{1\baselineskip}

\Proposition{

    Sei $U \subseteq \R^2$ offen und sei $F: U \rightarrow \R^2$ ein stetig differenzierbares
    Vektorfeld und $B = [a,b] \times [c,d] \subseteq U$ ein Rechteck. Dann gilt
    \begin{align*}
        \int_B \div (F) \ dx = \int_{\partial B} F \ dn
    \end{align*}
}

\pagebreak

\Korollar{

    Sei $U \subseteq \R^2$ offen und sei $F: U \rightarrow \R^2$ ein stetig
    differenzierbares Vektorfeld. Dann gilt für alle $x_0 \in U$
    \begin{align*}
        \div (F)(x_0) = \limes{h \rightarrow 0} \frac{1}{4 h^2} \int_{\partial \klammer{x_0 + [-h,h]^2}} F \ dn
    \end{align*}
}

\vspace{1\baselineskip}

\Proposition{

    Sei $U \subseteq \R^2$ offen und $F$ ein stetig differenzierbares Vektorfeld
    auf $U$. Seien $a<b$ und $c<d$ reelle Zahlen, so, dass das Rechteck
    $[a,b] \times [c,d]$ in $U$ enthalten ist, und sei $\varphi: [a,b] \rightarrow [c,d]$
    stetig und stückweise stetig differenzierbar. Dann gilt
    für den Bereich $B = \geschwungeneklammer{(x,y) \in U \ | \ x \in [a,b] , \ c \leq y \leq \varphi (x)}$
    \begin{align*}
        \int_B \div (F) \ dx = \int_{\partial B} F \ dn
    \end{align*}
}

\vspace{1\baselineskip}

\Definition{

    Eine abgeschlossene Teilmenge $B \subseteq \R^n$ heisst ein \fat{glatt berandeter
    Bereich}, falls es für jeden Punkt $p \in B$ eine offene Umgebung $U_p$ in $\R^n$
    von $p$ gibt und ein glatter Diffeomorphismus $\varphi_p : U_p \rightarrow V_p
    = \varphi_p (U_p)$ auf eine weitere offene Teilmenge $V_p \subseteq \R^n$ existiert,
    so dass folgendes gilt
    \begin{align*}
        \varphi_p (U_p \cap B) = \geschwungeneklammer{y \in V_p \ | \ y_n \leq 0}
    \end{align*}
}

\Korollar{

    Aus der obigen Definition folgt insbesondere, dass der Rand $\partial B$ eines
    glatt berandeten Bereichs $B$ eine $n-1$-dimensionale Teilmannigfaltigkeit von
    $\R^n$ ist.
}

\vspace{1\baselineskip}

\Korollar{

    Wenn ein Bereich $B$ lokal um jeden Punkt aussieht wie das Gebiet unter dem Graphen einer
    glatten reellwertigen Funktion in $(n-1)$ Variablen, geeignet rotiert, dann ist
    der Bereich $B$ glatt berandet.
}

\vspace{1\baselineskip}

\Lemma{

    Sei $F: \R^n \rightarrow \R$ eine glatte Funktion mit Null als regulären Wert.
    Dann ist die abgeschlossene Teilmenge
    \begin{align*}
        B = \geschwungeneklammer{x \in \R^n \ | \ F(x) \geq 0}
    \end{align*}
    glatt berandet und $\partial B = \geschwungeneklammer{u \in \R^n \ | \ F(u) = 0}$
}

\vspace{1\baselineskip}

\Satz{

    Sei $K \subseteq \R^n$ kompakt und sei $U_1 , \dots , U_N$ eine (endliche) offene
    Überdeckung von $K$. Dann existieren glatte Funktionen $\eta_0 , \dots , \eta_N :
    \R^n \rightarrow [0,1]$ mit:
    \begin{enumerate}[{1)}]
        \item $\supp (\eta_0) \subseteq \R^n \backslash K$
        \item $\supp (\eta_i) \subseteq U_i$ für $i = 1, \dots , N$
        \item $\eta_0 + \dots + \eta_N = 1$
    \end{enumerate}
}

\vspace{1\baselineskip}

\Lemma{

    Sei $f: \R^n \rightarrow \R$ eine stetige Funktion. Sei $\psi : \R^n \rightarrow \R$
    eine glatte Funktion mit kompaktem Träger. Dann ist die durch
    \begin{align*}
        (\psi * f)(x) = \int_{\R^n} \psi (x-y) f(y) dy
    \end{align*}
    definierte Funktion $\psi * f : \R^n \rightarrow \R$ glatt, mit Träger in
    $\supp (\psi) + \supp(f)$.
}

\pagebreak

\Lemma{

    Sei $f: \R^n \rightarrow \R$ stetig, $K \subseteq \R^n$ kompakt und $\epsilon>0$.
    Dann existieren $\delta \in (0,\epsilon)$ und eine glatte Funktion $\psi:
    \R^n \rightarrow [0,\infty)$ so dass
    \begin{align*}
        \abs{x-y} < \delta \Longrightarrow \abs{f(x) - f(y)} < \epsilon
        \quad , \quad 
        \supp (\psi) \subseteq B(0,\delta)
        \quad \text{ und } \quad
        \int_{\R^n} \psi \ dx \ = 1
    \end{align*}
    für alle $x \in K$ und $y \in \R^n$ gilt. Für jede solche Funktion $\psi$
    gilt $\abs{(\psi * f)(x) - f(x)} \leq \epsilon$ für alle $x \in K$.
}

\vspace{1\baselineskip}

\Definition{

    Sei $B \subseteq \R^2$ abgeschlossen. Eine \fat{Parametrisierung des Randes} von
    $B$ ist eine endliche Kollektion stetig differenzierbarer Wege
    $\gamma_k : [a_k,b_k] \rightarrow \partial B$ mit folgenden Eigenschaften:
    \begin{enumerate}
        \item (Überdeckend) Es gilt $\partial B = \bigcup_{k=1}^{K} \gamma_k \klammer{[a_k,b_k]}$.
        \item (Keine Überschneidungen ausser an den Enden) Falls $\gamma_j (s) = \gamma_k (t)$
                für $(j,s) \neq (k,t)$ gilt, so gilt $s \in \geschwungeneklammer{a_j , b_j}$
                und $t \in \geschwungeneklammer{a_k , b_k}$.
        \item (Aufeinanderfolgend) Für jedes $j \in \geschwungeneklammer{1,\dots,K}$ existiert
                genau ein $k \in \geschwungeneklammer{1,\dots,K}$ mit $\gamma_j (b_j)
                = \gamma_k (a_k)$.
        \item (Regularität) Es gilt $\gamma_k'(t) \neq 0$ für alle $k$ und $t \in (a_k,b_k)$.
    \end{enumerate}
    Die Parametrisierung $\gamma_1, \dots ,  \gamma_K$ heisst \fat{positiv orientiert},
    wenn für jedes $k$ und jedes $t \in (a_k,b_k)$ ein $\epsilon>0$ existiert, so dass
    $\gamma(t) + \lambda n_{\gamma} (t) \in B^{\circ}$ für alle $\lambda \in (0,\epsilon)$.
    Dabei ist $n_{\gamma} (t) = \klammer{- \gamma_2'(t),\gamma_1'(t)}$
    die Aussennormale an $\gamma$.
}

\vspace{1\baselineskip}

\Definition{

    Sei $U \subseteq \R^2$ offen und $B \subseteq \R^2$ eine kompakte Teilmenge,
    deren Rand eine positiv orientiert Parametrisierung $\gamma_1 : [a_1 , b_1]
    \rightarrow \partial B \ , \dots , \ \gamma_K : [a_k,b_k] \rightarrow \partial B$
    besitzt. Sei $F: U \rightarrow \R^2$ ein stetig differenzierbares Vektorfeld
    definiert auf einer offenen Umgebung $U$ von $B$. Das \fat{Wegintegral} oder
    \fat{Arbeitsintegral} von $F$ entlang $\partial B$ ist durch
    \begin{align*}
        \int_{\partial B} F \ dt \ = \sum_{k=1}^{K} \int_{a_k}^{b_k} \scalprod{F(\gamma_k (t))}{\gamma_k' (t)} dt
    \end{align*}
    definiert. Das \fat{Flussintegral} von $F$ durch den Rand $\partial B$ ist durch
    \begin{align*}
        \int_{\partial B} F \ dn \ = \sum_{k=1}^{K} \int_{a_k}^{b_k} \scalprod{F(\gamma_k (t))}{n_{\gamma} (t)} dt
    \end{align*}
    definiert. Diese Integrale sind unabhängig von der Wahl der positiv orientierten
    Parametrisierung.
}

\vspace{1\baselineskip}

\Satz{ (Divergenzsatz in der Ebene)

    Sei $U \subseteq \R^2$ offen und $B \subseteq U$ glatt berandet und kompakt.
    Sei $F: U \rightarrow \R^2$ ein Vektorfeld der Klasse $C^1$. Dann gilt
    \begin{align*}
        \int_B \div(F) dx = \int_{\partial B} F \ dn
    \end{align*}
}

\vspace{1\baselineskip}

\Bemerkung{

    Der Satz gilt auch für Bereiche $B \subseteq \R^2$ die durch das Innere von
    Rechtecken $R_i$ abgedeckt werden kann, so dass $B \cap R_i$ geeignet rotiert
    das Gebiet unter einem Graphen einer stückweise stetig differenzierbaren Funktion
    ist. Salopp heisst die, das $B$ "endlich viele Ecken" haben kann.
}

\vspace{1\baselineskip}

\Satz{ (Satz von Green)

    Sei $f: U \rightarrow \R^2$ ein stetig differenzierbares Vektorfeld auf einer
    offenen Menge $U \subseteq \R^2$. Dann gilt für jeden glatt berandeten, kompakten
    Bereich $B \subseteq U$
    \begin{align*}
        \int_B \rot (F) dx = \int_{\partial B} F \ dt
    \end{align*}
    Wobei $\rot (F) = \partial_2 F_1 (x) - \partial_1 F_2 (x)$
}

\pagebreak

\Definition{

    Wir sagen, dass ein stetig differenzierbares Vektorfeld $F:U \rightarrow \R^2$
    \fat{rotationsfrei} ist, falls $\rot (F) = 0$.
}

\vspace{1\baselineskip}

\Korollar{

    Dass $F$ rotationsfrei ist, bedeutet in anderen Worten gerade, dass $F$ die
    Integrabilitätsbedingung erfüllt.
}

\vspace{1\baselineskip}

\Satz{ (Jordanscher Kurvensatz)

    Sei $\gamma: [0,1] \rightarrow \R^2$ ein glatter, regulärer, einfacher,
    geschlossener Weg. Dann kann man das Komplement von $\Gamma = \gamma([0,1])$
    eindeutig als disjunkte Vereinigung
    \begin{align*}
        \R^2 \backslash \Gamma = \text{Inn}(\Gamma) \cup \text{Auss}(\Gamma)
        \quad , \quad
        \text{Inn} (\Gamma) \cap \text{Auss}(\Gamma) = \emptyset
    \end{align*}
    schreiben, wobei dasd Innere $\text{Inn}(\Gamma)$ eine offene, beschränkte,
    zusammenhängende Teilmenge und das Äussere $\text{Auss}(\Gamma)$ eine offene,
    unbeschränkte, zusammenhängende Teilmenge ist. Des Weiteren gilt
    $\partial \text{Inn} (\gamma) = \partial \text{Auss} (\gamma) = \Gamma$.
}
