\subsection{Der Divergenzsatz im dreidimensionalen Raum}

\vspace{1\baselineskip}

\Proposition{

    Sei $U \subseteq \R^3$ offen und $F:U \rightarrow \R^3$ ein stetig differenzierbares
    Vektorfeld. Sei $Q \subseteq \R^2$ ein abgeschlossenes Rechteck, $z_0 \in \R$ und
    $\varphi: Q \rightarrow [z_0,\infty)$ eine stetig differenzierbare Funktion, so
    dass der abgeschlossene Bereich under dem Graphen von $\varphi$
    \begin{align*}
        B = \geschwungeneklammer{(x,y,z) \in \R^3 \ | \ (x,y) \in \Q \ , \ z_0 \leq z \leq \varphi(x,y)}
    \end{align*}
    in $U$ liegt. Dann gilt
    \begin{align*}
        \int_B \div (F) \ dx \ = \ \int_{\partial B} F \ dn
    \end{align*}
}

\vspace{1\baselineskip}

\Satz{ (Divergenzsatz/ Satz von Gauss)

    Sei $B \subseteq \R^3$ ein kompakter, glatt berandeter Bereich und $F$ ein stetig
    differenzierbares Vektorfeld, definiert auf einer offenen Umgebung von $B$. Dann
    gilt
    \begin{align*}
        \int_B \div (F) \ dx \ = \ \int_{\partial B} F \ dn
    \end{align*}
}
