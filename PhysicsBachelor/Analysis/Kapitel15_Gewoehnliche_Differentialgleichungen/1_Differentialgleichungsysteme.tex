\subsection{Differentialgleichungssysteme}

\vspace{1\baselineskip}

Wir fixieren ein offenes, nichtleeres Intervall $I \subseteq \R$ und $x_0 \in I$.
Wir schreiben $C^{\infty} (I)$ für den $\C$-Vektorraum aller glatten Funktionen
$I \rightarrow \C$.

\vspace{1\baselineskip}

\Definition{

    Wir schreiben $D: C^{\infty} (I) \rightarrow C^{\infty} (I)$ für die Ableitung
    $Df = f'$, aufgefasst als $\C$-linearer Endomorphismus des Vektorraums $C^{\infty} (I)$.
    Ein linearer \fat{Differentialoperator} der Ordnung $d \geq 0$ ist eine lineare
    Abbildung
    \begin{align*}
        L: C^{\infty} (I) \rightarrow C^{\infty} (I)
        \quad , \quad \quad
        \sum_{i=0}^{d} L = a_i D^{i}
    \end{align*}
    die sich als Linearkombination von $D^0 =$ id, $D^1 = D$, $D^2 = D \circ D$, \dots \
    mit Koeffizienten $a_i \in C^{\infty} (I)$ schreibt. Als lineare, gewöhnliche
    \fat{Differentialgleichung} der Ordnung $d$ bezeichnen wir die Gleichung $Lu=g$
    für einen vorgegebenen Differentialoperator $L$ und eine vorgegebene
    \fat{Störfunktion} $g \in C^{\infty} (I)$. Als \fat{Anfangswertproblem} bezeichnet
    man das lineare Gleichungssystem bestehend aus dieser Differentialgleichung und
    $d$ Gleichungen
    \begin{align*}
        &Lu = g \\
        &u(x_0) = w_0 \quad , \quad Du(x_0) = w_1 \quad , \dots , \quad D^{d-1} u(x_0) = w_{d-1}
    \end{align*}
    für vorgegebene \fat{Anfangswerte} $w_0 , w_1 , \dots , w_{d-1} \in \R$.
}

\vspace{1\baselineskip}

\Korollar{

    Um die Lösung der homogenen Differentialgleichung $Lu = 0$ zu finden, setzen wir
    $u(x) = \exp (\alpha x)$ für ein $\alpha \in \C$. Damit gilt $D^{i} u = \alpha^{i} u$
    und
    \begin{align*}
        Lu \ = \ \sum_{i=0}^{d} a_i D^{i} u \ = \ \sum_{i=1}^{d} a_i \alpha^{i} u
    \end{align*}
    Daher ist $u(x) = \exp (\alpha x)$ genau dann eine Lösung der Differentialgleichung
    $Lu=0$, wenn $\alpha \in \C$ eine Nullstelle des sogenannten \fat{charakteristischen
    Polynoms} des Operators $L$
    \begin{align*}
        p(T) = T^d + a_{d-1} T^{d-1} + \dots + a_0 \in \C [T]
    \end{align*}
    ist. Ist $\alpha$ eine reelle Nullstelle von $p(T)$, dann ist durch $u(x) = \exp (\alpha x)$
    eine reellwertige Lösung gegeben. Bei reellen Koeffizienten und einer komplexen
    Nullstelle $\alpha = \beta + \gamma i \ \in \C$ mit $\beta , \gamma \in \R$ ist
    $\overline{\alpha} = \beta - \gamma i$ ebenso eine Nullstelle von $p(T)$. Dann
    sind alle Linearkombinationen $s \exp (\alpha) + t \exp(\overline{\alpha} x)$
    und insbesondere
    \begin{align*}
        \frac{\exp (\alpha x) + \exp (\overline{\alpha} x)}{2} = \exp (\beta x) \cos (\gamma x)
        \quad , \quad
        \frac{\exp (\alpha x) + \exp (\overline{\alpha} x)}{2i} = \exp (\beta x) \sin (\gamma x)
    \end{align*}
    Lösungen von $Lu=0$. Man beachte, dass $\exp (\alpha x)$ und $\exp (\overline{\alpha} x)$
    gleichen $\C$-Unterraum von Lösungen zu $Lu=0$ aufspannen wie $\exp (\beta x) \cos(\gamma x)$
    und $\exp(\beta x) \sin (\gamma x)$.
}

\vspace{1\baselineskip}

\Korollar{

    Wir betrachten nun den Fall, in dem das charakteristische Polynom des
    Differentialoperators $L$ eine mehrfache Nullstelle hat. Dazu untersuchen wir
    Linearkombinationen von Funktionen des Types $f(x) = x^n \exp (\alpha x)$.

    Hierbei ist $n$ die Vielfachheit einer Nullstelle $\alpha$. Als Lösungsansatz
    summieren wir über all die Nullstellen und alle Vielfachheiten. Angenommen
    wir haben zwei Nullstellen $\lambda_1$ und $\lambda_2$ mit Vielfachheiten
    $\mu_1$ bzw $\mu_2$. Dann ist der Ansatz von $u$ gegeben als
    \begin{align*}
        u = \sum_{i=0}^{\mu_1 -1} c_i x^{i} e^{\lambda_1 x} \ + \ \sum_{j=0}^{\mu_2 -1} c_j x^{j} e^{\lambda_2 x}
    \end{align*}
    wobei $c_i$ und $c_j$ Koeffizienten sind, welche mit den Anfangsbedingungen
    herausgefungen werden müssen. Wichtig ist, dass die Koeffizienten in den beiden
    Summen verschieden sind. Die Notation ist etwas ungünstig, da mehrere Koeffizienten
    den gleichen Intex haben könnten.
}

\pagebreak

\Definition{

    Wir bezeichnen den $\C$-Vektorraum der von diesen Funktionen aufgespannt wird,
    also
    \begin{align*}
        \text{PE}_{\C} (I) = \langle x^n \exp (\alpha x) \ | \ n \in \N \ , \ \alpha \in \C \rangle
    \end{align*}
    als Raum der \fat{Exponentialpolynome}. Zusammen mit der üblichen Multiplikation
    bilden Exponentialpolynome eine $\C$-Algebra. Die Ableitung definiert eine
    $\C$-lineare Abbildung
    \begin{align*}
        D: \text{PE}_{\C} (I) \rightarrow \text{PE}_{\C} (I)
        \quad , \quad \quad
        D(x^n e^{\alpha x}) = n x^{n-1} e^{\alpha x} + \alpha x^n e^{\alpha x}
    \end{align*}
    die die Leibnitz-Regel $D(fg) = D(f) g + f D(g)$ erfüllt.
}

\vspace{1\baselineskip}

\Proposition{

    Die Funktion $\geschwungeneklammer{x \mapsto x^n \exp (\alpha x) \ | \ n \in N , \alpha \in \C}$
    bilden eine Basis des $\C$-Vektorraums $\text{PE}_{\C} (I)$.
}

\vspace{1\baselineskip}

\Proposition{

    Sei $p \in \C [T]$ ein Polynom von Grad $d \geq 1$, das mittels
    \begin{align*}
        p(T) = a \prod_{j=1}^{k} (T-\alpha_j)^{d_j}
    \end{align*}
    in $d$ Linearfaktoren zerfällt, wobei wie annehmen, dass $\alpha_i \neq \alpha_j$
    für $i \neq j$ gilt. Sei $L=p(D)$ der Differentialoperator mit charakteristischem
    Polynom $p$. Dann hat die zugehörige homogene Differentialgleichung $Lu =0$
    die $d$ linear unabhängigen Lösungen $u(x) = x^n \exp(\alpha_j x)$ für
    $0 \leq n \leq n_j$ und $1 \leq j \leq k$.
}

\vspace{1\baselineskip}

\Bemerkung{

    Wir wenden uns nun den inhomogenen Problem zu. Für ein Polynom $p(T) \in \C [T]$
    und eine Störfunktion $g \in \text{PE}_{\C} (I)$ gibt es ein einfaches Verfahren,
    mit dem man eine Lösung $u_0$ der inhomogenen Differentialgleichung $p(D)u = g$
    finden kann.
    \begin{enumerate}
        \item Falls $g(t) = q(t) e^{\alpha t}$ für ein Polynom $q$ vom Grad $n$ und
                $\alpha \in \C$ mit $p(\alpha) \neq 0$, dann definiert man $u_0 (t)
                =Q(t)e^{\alpha t}$, wobei $Q(T)$ ein Polynom vom Grad $n$ mit noch zu
                bestimmenden Koeffizienten ist. Nun berechnet man $p(D)u_0$ und
                bestimmt Koeffizienten so dass $p(D)u_0 = g$ gilt.
        \item Falls $g(x) = q(x) e^{\alpha x}$ für ein Polynom $q(T)$ vom Grad $n$ und
                $\alpha \in \C$ mit $p(\alpha) = 0$, dann wiederholt man obiges Verfahren,
                allerdings mit dem Ansatz $u_0 (t) = Q(t) t^m e^{\alpha t}$, wobei $m$
                die Vielfachheit der Nullstelle $\alpha$ von $p(T)$ angibt.
        \item Ein allgemeines $g \in \text{PE}_{\C} (I)$ lässt sich als Linearkombination
                von Ausdrücken wie oben darstellen. Auf Grund der Linearität von $p(D)$
                kann man also obiges Verfahren für Summanden der Form $q(x) e^{\alpha x}$
                in $g$ anwenden und dann die resultierenden Lösungsfunktionen addieren.
    \end{enumerate}
}

\vspace{1\baselineskip}

\Definition{

    Sei $I \subseteq \R$ ein Intervall, $U \subseteq I \times \R^d$ offen und
    $F: U \rightarrow \R^d$ eine stetige Funktion. Eine Differentialgleichung
    der Form
    \begin{align*}
        u'(t) = F(t,u(t))
    \end{align*}
    für eine unbekannte Funktion $u$ bezeichnen wir als $d$-\fat{dimensionale
    Differentialgleichungssysteme erster Ordnung}. Dabei ist der Definitionsbereich
    einer Lösung $u$ möglicherweise nur ein Teilintervall von $I$. Damit die obere
    Gleichung überhaupt Sinn ergibt, muss $(t,u(t)) \in U$ für alle $t$ im
    Definitionsbereich von $u$ gelten. Zu $(t_0 , x_0) \in U$ nennen wir die
    Gleichung
    \begin{align*}
        u'(t) = F(t,u(t))
        \quad , \quad \quad
        u(t_0) = x_0
    \end{align*}
    ein \fat{Anfangswertproblem} zum \fat{Anfangswert} $x_0$ bei $t_0 \in I$.
    Die Differentialgleichung heisst \fat{autonom}, falls $F$ konstant bezüglich
    der Variablen $t$ ist.
}

\vspace{1\baselineskip}

\Proposition{

    Sei $A \in \Mat_d (\R)$, $x_0 \in \R^d$ und $t_0 \in \R$. Das Anfangswertproblem
    für differenzierbare Funktionen $u: \R \rightarrow \C^d$
    \begin{align*}
        u'(t) = Au
        \quad , \quad \quad
        u(t_0) = x_0
    \end{align*}
    hat die eindeutig bestimmte Lösung $u(t) = \exp(A(t-t_0))x_0$.
}

\vspace{1\baselineskip}

\Definition{ (\fat{Trennung der Variablen / Variablenseparation})

    Gegeben sei eine Differentialgleichung erster Ordnung in der Form
    \begin{align*}
        u'(t) = f(t) g(u(t))
    \end{align*}
    für Funktionen $f$ und $g$. Solch eine Differentialgleichung bezeichnet man als
    \fat{separierbar}, da auf der rechten Seite ein Produkt von zwei Funktionen steht,
    von denen eine nur von der Variablen $t$, und die andere nur von der "Variablen"
    $u=u(t)$ abhängt. Wir teilen durch $g(u(t))$ und integrieren:
    \begin{align*}
        \int \frac{u'(t)}{g(u(t))} \ dt \ = \ \int f(t) \ dt \ + C
    \end{align*}
    Im Integral linkerhand können wir die Substitution $s=u(t)$ vornehmen. Ist also
    $G$ eine Stammfunktion von $\frac{1}{g}$ und $F$ eine Stammfunktion von $f$, so
    erhalten wir $G(u(t)) = F(t) + C$ für eine Integrationskonstante $C$. Falls
    $G$ invertierbar ist, ergibt sich damit die Lösung
    \begin{align*}
        u(t) = G^{-1}(F(t) + C)
    \end{align*}
    der Differentialgleichung.
}
