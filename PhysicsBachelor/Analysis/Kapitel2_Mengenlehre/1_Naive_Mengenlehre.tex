\subsection{Naive Mengenlehre}

\vspace{1\baselineskip}

\Definition{

    (1) Eiene \fat{Menge} besteht aus beliebigen unterscheidbaren Elementen.

    (2) Eine Menge ist unverwechselbar dursch ihre \fat{Elemente} bestimmt.

    (3) Eine Menge ist nicht Element ihrer selbst.

    (4) Jede Aussage $A$ über Elemente einer Menge $X$ definiert die 
        Menge der Elemente in $X$ fpr die die Aussage $A$ \hspace*{13pt} wahr ist, 
        notiert $\geschwungeneklammer{x \in X | \text{A gilt für x}}$
}

\vspace{1\baselineskip}


\Definition{

    Mengen von Mengen wird \fat{Familien} von Mengen genannt.
    Seltener braucht man auch die Bezeichnung \fat{Kollektion}
    oder \fat{Ansammlung}.
}

\vspace{1\baselineskip}

\Definition{

    Seien $P$ und $Q$ Mengen. Wir sagen, dass $P$ \fat{Teilmenge} von $Q$
    ist, und schreiben $P \subset Q$, falls für alle $x \in P$ auch 
    $x \in Q$ gilt. Wir sagen, dass P eine \fat{echte Teilmenge} von $Q$
    ist und schreiben $P \subsetneq Q$, falls $P$ eine Teilmenge von $Q$, 
    aber nicht gleich $Q$ ist. Wir schreiben $P \not\subset Q$, falls $P$
    keine Teilmenge von $Q$ ist.
}

\vspace{1\baselineskip}

\Definition{

    Seien $P$ und $Q$ Mengen. Der \fat{Durchschnitt} $P \cap Q$, die
    \fat{Vereinigung} $P \cup Q$, das \fat{relative Komplement} $P \backslash Q$
    und die symmetrische Differenz $P \Delta Q$ sind durch 
    \begin{align*}
        &P \cap Q = \geschwungeneklammer{x \ | \ x \in P \land x \in Q}
        \\
        &P \cup Q = \geschwungeneklammer{x \ | \ x \in P \lor x \in Q}
        \\
        &P \backslash Q = \geschwungeneklammer{x \ | \ x \in P \land x \not\in Q}
        \\
        &P \Delta Q = (P \cup Q) \backslash (P \cap Q) = \geschwungeneklammer{x \ | \ x \in P \text{ XOR } x \in Q}
    \end{align*}
}

\vspace{1\baselineskip}

\Definition{

    Sei $\mathcal{A}$ eine Familie von Mengen, also eine Menge deren Elemente
    selbst Mengen sind. Dann definieren wir die \fat{Vereinigung} als
    \begin{align*}
        \bigcup_{A \in \mathcal{A}} A = \geschwungeneklammer{x \ | \ \exists A \in \mathcal{A}: x \in A}
    \end{align*}
    respektive den \fat{Durchschnitt} der Menge $\mathcal{A}$ als
    \begin{align*}
        \bigcap_{A \in \mathcal{A}} A = \geschwungeneklammer{x \ | \ \forall A \in \mathcal{A}: x \in A}
    \end{align*}
    Falls $\mathcal{A} = \geschwungeneklammer{A_1,A_2, \dots}$, dann 
    schreiben wir auch
    \begin{align*}
        \bigcup_{n=1}^{\infty} A_n = \geschwungeneklammer{x \ | \ \exists n \in \N : x \in A_n}
        \\
        \bigcap_{n=1}^{\infty} A_n = \geschwungeneklammer{x \ | \ \forall n \in \N : x \in A_n}
    \end{align*}
    für die Vereinigung, und den Durchschnitt der Menge in $\mathcal{A}$.
}

\pagebreak

\Definition{

    Zwei Mengen $A,B$ heissen \fat{disjunkt}, falls $A \cap B = \emptyset$ gilt.
    Für eine Kollektion $\mathcal{A}$ von Mengen, sagen wir, dass die Menge in 
    $\mathcal{A}$ \fat{paarweise disjunkt} sind, falls für alle $A_1,A_2 \in \mathcal{A}$
    mit $A_1 \neq A_2$ gilt $A_1 \cap A_2 = \emptyset$.
}

\vspace{1\baselineskip}

\Definition{

    Sei $X$ eine Menge. Die \fat{Potenzmenge} $\mathcal{P}(X)$ von $X$
    ist die Menge aller Teilmengen von $X$, das heisst
    \begin{align*}
        \mathcal{P}(X) = \geschwungeneklammer{Q \ | \ Q \text{ ist eine Menge und } Q \subset X}
    \end{align*}
}

\Definition{

    Für zwei Mengen $X$ und $Y$ ist das \fat{kartesische Produkt}
    $X \times Y$ die Menge aller geordneten Paare $(x,y)$ wobei
    $x \in X$ und $y \in Y$. In Symbolen,
    \begin{align*}
        X \times Y = \geschwungeneklammer{(x,y) \ | \ x \in X \land y \in Y}
    \end{align*}

    Allgemeiner, sei $I$ eine Indexmenge, und für jedes $i \in I$
    sei $X_i$ eine Menge. Das \fat{Produkt} der Familien von Mengen
    $\geschwungeneklammer{X_i | i \in I}$ definieren wir als
    \begin{align*}
        \prod_{i \in I} X_i = \geschwungeneklammer{(x_i)_{i \in I} \ | \ \forall i \in I \ : \ x_i \in X_i}
    \end{align*}
    Für eine Zahl $n \leq 1$ und eine Menge $X$ definieren wir $X^n$
    als das $n$-fache kartesische Produkt von $X$ mit sich selbst.
}