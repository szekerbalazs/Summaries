\subsection{Funktionen und Abbildungen}

\vspace{1\baselineskip}

\Definition{

    Seien \X und \Y Mengen. Eine \fat{Funktion} von \X nach \Y
    ist eine Teilmenge $F$ des karesischen Produktes $X \times Y$
    mit der Eigenschaft, dass es für jedes \xinX genau ein \yinY
    mit $(x,y) \in F$ gibt:
    \begin{align*}
        \forall x \in X \ \exists! \ y \in Y : (x,y) \in F
    \end{align*}
    Wir bezeichnen die Menge \X als \fat{Definitionsbereich} und 
    die Menge \Y als \fat{Wertebereich} oder auch als \fat{Zielbereich}.
}

\Definition{

    Die Menge $F$ ist gegeben durch
    \begin{align*}
        F = \geschwungeneklammer{(x,f(x)) | x \in X}
    \end{align*}
    gegeben und wird \fat{Graph} von $f$ bezeichnet.
}

\vspace{1\baselineskip}

\Definition{

    Sei \X eine Menge und sei $A$ eine Teilmenge von \X. Die
    Funktion $\iota_A : A \rightarrow X$ die durch $\iota_A(a) = a \ \forall a \in A$
    gegeben ist, nennt man \fat{Inklusionsabbildung} von $A$ nach
    \X. Die Inklusionsabbildung $\iota_X : X \rightarrow X$ wird
    \fat{Identität} von \X genannt, und als $id_X : X \rightarrow X$
    geschrieben. Die Funktion $\mathds{1}_A : X \rightarrow \geschwungeneklammer{0,1}$
    gegeben durch
    \begin{align*}
        \mathds{1}_A (x) = \begin{cases}
            0 \quad \text{ für } x \not\in A \\
            1 \quad \text{ für } x \in A
        \end{cases}
    \end{align*}
    wird \fat{charakteristische Funktion} von $A$ genannt. Für 
    zwei Mengen \X und \Y wird die \fat{Menge aller Abbildungen} von 
    \X nach \Y als $Y^X$ geschrieben, formaler
    \begin{align*}
        Y^X = \geschwungeneklammer{f \ | \ f: X \rightarrow X \text{ ist eine Funktion}}
    \end{align*}
    Grund der Notation ist unter anderem die folgende Behauptug. 
    Falls \X und \Y endliche Mengen sind mit $m$, respektive $n$
    Elementen für zwei natürliche Zahlen $m,n \in \N$, so hat 
    $Y^X$ genau $n^m$ Elemente.
}

\vspace{1\baselineskip}

\Definition{

    Sei $f: X \rightarrow Y$ eine Funktion, und sei $A$ eine
    Teilmenge von \X. Die Verknüpfung $f \circ \iota_A$ von $f$
    mit der Inklusionsabbildung $\iota_A : A \rightarrow X$ nennt
    man \fat{Einschränkung} von $f$ auf $A$. Man notiert diese 
    Funktion als 
    \begin{align*}
        f|_A : A \rightarrow Y
    \end{align*}
    Es gilt $f|_A (a) = f(a) \ \forall a \in A$. Dennoch betrachten
    wir $f|_A$ und $f$ als verschiedene Funktionen, da ihre 
    Definitionsbereich nicht derselbe ist - ausgenommen natürlich
    man hätte $A = X$ und also $\iota_A = id_X$
}

\vspace{1\baselineskip}

\Definition{

    Sei $f: X \rightarrow Y$ eine Funktion. Wir nennen $f$

    \vspace{1\baselineskip}

    1. \fat{injektiv} oder eine \fat{Injektion} falls $f(x_1) = f(x_2) \Rightarrow x_1 = x_2 \ \forall x_1,x_2 \in X$ gilt.

    2. \fat{surjektiv} oder eine \fat{Surjektion} falls $\forall \ y \in Y \ \exists x \in X$ mit $f(x) = y$

    3. \fat{bijektiv} oder eine \fat{Bijektion}, falls sie suejektiv und injektiv ist.

    \vspace{1\baselineskip}

    Ist $f$ bijektiv, so wird die Funktion $g: Y \rightarrow X$, 
    die eindeutig durch 
    \begin{align*}
        g \circ f = id_X 
        \quad
        und 
        \quad
        f \circ g = id_X
    \end{align*}
    bestimmt ist, \fat{Umkehrabbildung} von $f$, oder zu $f$
    \fat{inverse Funktion} genannt. Es ist also $g(y)$ das 
    eindeutig bestimmte Element \xinX mit $f(x) = y$.
}

\vspace{1\baselineskip}

\Lemma{

    Seien $f: X \rightarrow Y$ und $g: Y \rightarrow Z$ Funktionen.

    1. Falls $f$ und $g$ injektiv sind, dann ist auch $g \circ f$ injektiv.
    
    2. Falls $g \circ f$ injektiv ist, dann ist auch $f$ injektiv.

    3. Falls $f$ und $g$ surjektiv sind, dann ist auch $g \circ f$ surjektiv.
    
    4. Falls $g \circ f$ surjektiv ist, dann ist auch $g$ surjektiv.

    5. Falls $f$ und $g$ bijektiv sind, dann ist auch $g \circ f$
        bijektiv un des gilt $(g \circ f)^{-1} = f^{-1} \circ g^{-1}$.    
}

\vspace{1\baselineskip}

\Definition{

    Für eine Funktion $f: X \rightarrow Y$ und eine Teilmenge $A \subset X$
    schreiben wir
    \begin{align*}
        f(A) = \geschwungeneklammer{y \in Y \ | \ \exists x \in A : f(x)=y}
    \end{align*}
    und nennen diese Teilmenge von \Y das \fat{Bild} von $A$ 
    bezüglich der Funktion $f$. Für eine Teilmenge $B \subset Y$
    schreiben wir 
    \begin{align*}
        f^{-1}(B) = \geschwungeneklammer{x \in X \ | \ \exists y \in B : f(x) = y}
    \end{align*}
    und nennen diese Teilmenge von \X das \fat{Urbild} von $B$
    bezüglich der Funktion $f$.
}

