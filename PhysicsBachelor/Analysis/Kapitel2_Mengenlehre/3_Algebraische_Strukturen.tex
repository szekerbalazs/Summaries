\subsection{Algebraische Strukturen}

\vspace{1\baselineskip}

\Definition{

    Ein \fat{kommutatives Monoid} ist ein Tripel $(M,m_0,f)$ bestehend aus 
    einer Menge $M$, einem Element $m_0 \in M$ und einer Abbildung
    $f: M \times M \rightarrow M$ die folgende Eigenschaften erfüllen.

    \vspace{1\baselineskip}

    1. Neutrales Element: $f(x,m_0) = f(m_0,x) = x \ \forall x \in M$

    2. Kommutativität: $f(x,f(y,z)) = f(f(x,y),z) \ \forall x,y,z \in M$

    3. Assiziativität: $f(x,y) = f(y,x) \ \forall x,y \in M$

    \vspace{1\baselineskip}

    Wir nennen die Funktion $f$ \fat{Addition} und $m_0$ \fat{neutrales Element}
    oder \fat{Null}, und schreiben üblicherweise $x+y$ anstelle
    von $f(x,y)$ und $0$ anstelle von $m_0$.
}

\vspace{1\baselineskip}

\Definition{

    Sei $X$ eine Menge. Eine \fat{Relation} auf \X ist eine Teilmenge
    $R \subset X \times X$. Wir schreiben auch $xRy$ falls $(x,y) \in R$.
    Wenn $\sim$ eine Relation ist, dann schreiben wir auch 
    $x \not\sim y$ für $\neg (x \sim y)$. Eine Relation $\sim$ heisst:

    \vspace{1\baselineskip}

    1. \fat{Reflexiv}: Falls $\forall x \in X : x \sim x$

    2. \fat{Transitiv}: Falls $\forall x,y,z \in X: ((x \sim y) \land (y \sim z)) \Rightarrow x \sim z$

    3. \fat{Symmetrisch}: Falls $\forall x,y \in X : x \sim y \Rightarrow y \sim x$

    4. \fat{Antisymmetrisch}: Falls $\forall x,y \in X: ((x \sim y) \land (y \sim x)) \Rightarrow x = y$

    \vspace{1\baselineskip}

    Eine Relation heisst \fat{Äquivalenzrelation}, falls sie reflexiv, transitiv 
    und symmetrisch ist. Eine Relation heisst \fat{Ordnungsrelation}, falls sie 
    reflexiv, transitiv und antisymmetrisch ist.
}

\vspace{1\baselineskip}

\Definition{

    Sei $\sim$ ein Äquivalenzrelation auf einer Menge $X$. Dann wird für \xinX
    die Menge
    \begin{align*}
        [x]_\sim = \geschwungeneklammer{y \in X \ | \ y \sim x}
    \end{align*}
    die \fat{Äquivalenzklasse} von $x$ genannt. Weiter heisst die Menge aller
    Äquivalenzklassen
    \begin{align*}
        X /_\sim = \geschwungeneklammer{[x]_\sim \ | \ x \in X}
    \end{align*}
    \fat{Quotient} oder die \fat{Quotientenmenge} von $X$ modulo $\sim$. 
    Ein Element \xinX wird auch \fat{Repräsentant} seiner Äquivalenzklasse
    $[x]_\sim$ genannt.
}

\vspace{1\baselineskip}

\Definition{

    Sei \X eine Menge. Eine \fat{Partition} von $X$ ist eine Familie $\mathcal{P}$ 
    von nicht leeren, paarweise disjunkten Teilmengen von $X$, so dass
    \begin{align*}
        X = \bigcup_{P \in \mathcal{P}} P
    \end{align*}
    gilt. Mit anderen Worten: Mengen $P \in \mathcal{P}$ sind nicht leer, und jedes 
    Element von $X$ ist Element von genau einem $P \in \mathcal{P}$.
}

\vspace{1\baselineskip}

\Proposition{

    Sei $X$ eine nicht leere Menge. Äquivalenzrelationen auf $X$ und 
    Partitionen von $X$ entsprechen einander im folgenden Sinne:
    Für eine gegebene Äquivalenzrelation $\sim$ auf $X$ ist die Menge
    \begin{align*}
        P = \geschwungeneklammer{[x]_\sim \ | \ x \in X}
    \end{align*}
    eine Partition von $X$. Umgekehrt definiert für eine gegebene Partition
    $P$ von $X$
    \begin{align*}
        x \sim y \ \Leftrightarrow \ \exists p \in P : x \in p  \land y \in p
    \end{align*}
    für \xyinX eine Äquivalenzrelation auf $X$. Wir erhalten eine kanonische
    Bijektion zwischen der Menge aller Äquivalenzrelationen auf \X und der Menge
    aller Partitionen von $X$.
}

\vspace{1\baselineskip}

\Definition{

    Eine Funktion $f$ heisst \fat{wohldefiniert}, falls $f$ nicht von der Wahl
    der Repräsentanten $x$ der Äquivalenzklasse $[x]_\sim$ abhängt und 
    jedem Element $[x]_\sim$ des Definitionsbereichs $X/_\sim$ das eindeutig
    bestimmte Element $f([x]_\sim)$ zuordnet.
}

\vspace{1\baselineskip}