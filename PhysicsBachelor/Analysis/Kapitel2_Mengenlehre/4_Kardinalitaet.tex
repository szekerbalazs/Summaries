\subsection{Kardinalität}

\vspace{1\baselineskip}

\Definition{

    Seien \X und \Y zwei Mengen. Wir sagen, dass $X$ und $Y$ \fat{gleichmächtig}
    sind, geschrieben $X \sim Y$ oder $\abs{X} = \abs{Y}$, falls es eine
    Bijektion $f: X \rightarrow Y$ gibt. Wir sagen dass $Y$ \fat{mächtiger}
    als $X$ ist, und schreiben $X \underset{\sim}{<} Y$, falls es eine 
    Injektion $f: X \rightarrow Y$ gibt. Wir sagen in dem Fall auch $X$
    sei \fat{schmächtiger} als $Y$.
}

\vspace{1\baselineskip}

\Definition{

    Wir sagen $X$ ist \fat{abzählbar} (unendlich), falls $\abs{X} = \abs{\N}$.

    Wir sagen, dass $X$ \fat{unendlich} ist, falls $\abs{\N} \leq \abs{X}$.

    Wir sagen, dass $X$ \fat{überabzählbar}  ist, falls $\abs{\N} < \abs{X}$
}

\vspace{1\baselineskip}

\Satz{
    (Cantors Diagonalargument)

    Sei $X$ eine Menge. Dann ist die $\mathcal{P}(X)$ mächtiger als $X$ und nicht gleichmächtig
    zu $X$.
}

\vspace{1\baselineskip}

\Satz{
    (Cantor, Schröder, Bernstein)

    Seien $X$ und $Y$ Mengen, so dass $X \underset{\sim}{<} Y$ und 
    $Y \underset{\sim}{<} X$. Dann gilt $X \sim Y$.
}

\vspace{1\baselineskip}

\Definition{

    Wir sagen, dass die Kardinalität der leeren Menge Null ist, und schreiben
    $\abs{\emptyset} = 0$. Sei $X$ eine Menge und $n \geq 1$ eine natürliche
    Zahl. Wir sagen die Menge $X$ habe Kardinalität $n$, und schreiben 
    $\abs{X} = n$, falls $X$ gleichmächtig zu $\geschwungeneklammer{1, \dots ,n}$
    ist. In diesem Fall nennen wir $X$ eine \fat{endliche Menge} und schreiben
    $\abs{X} < \infty$. Ist $X$ nicht endlich, so nennen wir $X$ eine 
    \fat{unendliche Menge}. Die Menge heisst \fat{abzählbar unendlich}, falls 
    sie gleichmächtig zu $\N$ ist. Die Kardinalität von $\N$ wird auch $\aleph_0$, 
    gesprochen \fat{Aleph-0}, genannt.
}

\vspace{1\baselineskip}

\textbf{Auswahlaxiom}

\vspace{1\baselineskip}

Variante (1)

Seien $X$ und $Y$ Mengen, und sei $f:X \rightarrow Y$ eine surjektive
Funktion. Dann existiert eine Funktion $g:Y \rightarrow X$ mit der 
Eigenschaft $f \circ g = id_Y$

Die Funktion $g$ in dieser Verion des Auswahlaxioms nennt man einen
\fat{Schnitt} von $f$. 

\vspace{1\baselineskip}

Variante (2)

Sei $Y$ eine Menge, und $\mathcal{X}$ eine Familie von nichtleeren Teilmengen von $Y$.
Dann gibt es eine Funktion $\alpha : X \rightarrow Y$ mit der Eigenschaft
dass $\alpha(X) \in X$ für alle $X \in \mathcal{X}$ gilt.
Die Funktio $\alpha$ in dieser Version nennt man \fat{Auswahlfunktion}, 
da sie der Auswahl eines Elementes $\alpha(X)$ in jeder der nichtleeren 
Mengen $X \in \mathcal{X}$ gleichkommt. 

\pagebreak

Variante (3)

Sei $\mathcal{X} = \geschwungeneklammer{X_i \ | \ i \in I}$ einer Familie von 
nichtleeren Mengen. Dann ist das Produkt
\begin{align*}
    \prod_{i \in I} X_i
\end{align*}
nicht leer.

\vspace{1\baselineskip}

\Definition{

    Sei $(X,\leq)$ eine geordnete Menge. Ein Element \xinX heisst \fat{maximal}
    falls für alle \yinX gilt: $x \leq y \Rightarrow x = y$. Ein Element 
    $m \in X$, so dass $x \leq m$ für alle \xinX gilt, dann heisst 
    $m \in X$ \fat{Maximum} von $X$.
}

\vspace{1\baselineskip}

\Definition{

    Sei $(X, \leq)$ eine geordnete Menge, und sei $A \subseteq X$ eine 
    Teilmenge. Ein Element \xinX heiss \fat{obere Schranke} von $A$ falls
    $a \leq x$ für alle $a\in A$ gilt. Ein Element \xinX heisst 
    \fat{untere Schranke} von A falls $x \leq a$ für alle $a \in A$ gilt.
}

\vspace{1\baselineskip}

\Definition{

    Sei $(X,\leq)$ eine geordnete Menge. Eine Teilmenge $K \subseteq X$
    heisst \fat{Kette}, falls für alle $x, y \in K$ gilt:
    $x \leq y$ oder $y \leq x$. Wir sagen $(X,\leq)$ sei \fat{induktiv}
    geordnet, falls jede Kette in $X$ eine obere Schranke besitzt.
}

\vspace{1\baselineskip}

\Lemma{
    (Zorn's Lemma)

    Sei $(X,\leq)$ eine induktiv geordnete Menge. Dann existiert ein maximales
    Element in $X$.
}


\vspace{1\baselineskip}

\Satz{
    (Hausdorff'sches Maximumprinzip)

    Sei $(X, \leq)$ eine geordnete Menge. Dann existiert eine maximale Kette 
    in $X$. Das heisst, es existiert eine Kette $M \subseteq X$, so dass 
    \begin{align*}
        M \subseteq L \ \Rightarrow \ M = L
    \end{align*}
    für jede Kette $L \subseteq X$ gilt.
}


