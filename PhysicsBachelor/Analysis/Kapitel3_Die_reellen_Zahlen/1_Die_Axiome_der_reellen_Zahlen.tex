In diesem Kapitel werden Körper thematisiert. Es wird vorausgesetzt,
dass der Leser, bzw Leserin, mit diesem Begriff vertraut ist.
Falls dies nicht der Fall ist, verweise ich auf die 
Unterlagen der Linearen Algebra \uproman{1} und \uproman{2}. 
Insbesondere möchte ich auf die Zusammenfassung 
\textit{Linearen Algebra \uproman{1} und \uproman{2}}
von Paul Sander verweisen.

Weitere Begriffe die bekannt sein sollten, sind \textit{ganze Zahlen}
und \textit{rationale Zahlen}.


\subsection{Die Axiome der reellen Zahlen}

\vspace{1\baselineskip}

\Definition{

    Sei $K$ ein Körper, und sei $\leq$ eine Ordnungsrelation auf 
    der Menge $K$. Wir nennen $(K,\leq)$, oder kurz $K$, einen
    \fat{angeordneten Körper} falls die folgenden Bedingungen
    erfüllt sind.

    \vspace{1\baselineskip}

    1. Linearität der Ordnung: Für alle $x,y, \in K$ gilt $x \leq y$ oder $y \leq x$.

    2. Kompatibilität von Ordnung und Addition: Für alle $x,y,z \in K$ gilt 
    \begin{align*}
        x \leq y \Rightarrow x + y \leq y + z
    \end{align*}
    3. Kompatibilität von Ordnung und Multiplikation: Für alle $x,y \in K$ gilt
    \begin{align*}
        (x \geq 0) \land (y \geq 0) \Rightarrow x \cdot y \geq 0
    \end{align*}
}

\vspace{1\baselineskip}

\Bemerkung{

    Im Folgenden sei $(K,\leq)$ ein angeordneter Körper, und 
    $x,y,z,w$ bezeichnen Elemente aus $K$.

    \vspace{1\baselineskip}

    (a) (Trichotomie) Es gilt entweder $x < y$ oder $x = y$ oder
    $x > y$.
    
    (b) Falls $x < y$ und $y \leq z$ ist, dann gilt auch $x < z$.

    (c) (Addition von Ungleichungen) Gilt $x \leq y$ und $z \leq w$,
    dann gilt auch $x + z \leq y + w$. 

    (d) Es gilt $x \leq y$ genau dann, wenn $0 \leq y - x$ gilt.

    (e) Es gilt $x \leq 0 \Leftrightarrow 0 \leq -x$.
    
    (f) Es gilt $x^2 \geq 0$, und $x > 0$, falls $x \neq 0$.

    (g) Es gilt $0 < 1$.

    (h) Falls $0 \leq x$ und $y \leq z$, dann gilt $xy \leq xz$.

    (i) Falls $x \leq 0$ und $y \leq z$, dann gilt $xy \geq xz$.

    (j) Aus $0 < x \leq y$ folgt $0 < y^{-1} \leq x^{-1}$.

    (k) Aus $0 \leq x \leq y$ und $0 \leq z \leq w$ folgt $0 \leq xz \leq yw$.

    (l) Aus $x+y \leq x + z$ folgt $y \leq z$.

    (m) Aus $xy \leq xz$ und $x > 0$ folgt $y \leq z$.
}

\vspace{1\baselineskip}

\Definition{

    Sei $(k,\leq)$ ein geordneter Körper. Der \fat{Absolutbetrag}
    auf $K$ ist die Funktion $\abs{ \hspace*{2pt} . \hspace*{2pt}}:K \rightarrow K$ die durch
    \begin{align*}
        \abs{x} = \begin{cases}
            x \quad \text{ falls } x \geq 0 
            \\
            -x \quad \text{ falls } x < 0
        \end{cases}
    \end{align*}
    für alle $x \in K$ definiert ist. Das \fat{Signum} ist die Funktion
    sgn: $K \rightarrow \geschwungeneklammer{-1,0,1}$ die durch
    \begin{align*}
        \text{sgn}(x) = \begin{cases}
            -1 \quad \text{ falls } x < 0 \\
            0 \quad \text{ falls } x = 0 \\
            1 \quad \text{ falls } x > 0
        \end{cases}
    \end{align*}
    für alle $x \in K$ definiert ist.
}

\pagebreak

\Bemerkung{

    Im Skript (Seite 46) gibt es Punkte \textit{(a)} bis \textit{(h)}. Hier werden nur
    \textit{(g)} und \textit{(h)} erwähnt.

    \vspace{1\baselineskip}

    (g) \fat{Dreiecksungleichung}: Es gilt $\abs{x + y} \leq \abs{x} + \abs{y}$.

    (h) Umgekehrte Dreiecksungleichung: Es gilt $\abs{y} - \abs{x} \leq \abs{x - y}$.
}

\vspace{1\baselineskip}

\Definition{
    (Vollständigkeitsaxiom)

    Sei \angeordneterK ein angeordneter Körper. Wir sagen \angeordneterK
    sei \fat{vollständig} oder \fat{vollständig angeordneter Körper} falls
    Aussage (\uproman{5}) wahr ist.

    \vspace{1\baselineskip}

    (\uproman{5}) 
    \indent 
    Seien $X,Y$ nicht-leere Teilmengen von $K$ derart, dass
    für alle \xinX und \yinY die Ungleichung $x \leq y$ gilt,
    dann \hspace*{7mm}gibt es ein
    $c \in K$, das zwischen $X$ und $Y$ liegt, in dem Sinn, dass für alle
    \xinX und \yinY die 
    Ungleichung \hspace*{7mm}$x \leq c \leq y$ gilt.

    \vspace{1\baselineskip}

    Die Aussage (\uproman{5}) bezeichnen wir als \fat{Vollständigkeitsaxiom}.
}

\vspace{1\baselineskip}

\Definition{

    Wir nennen \fat{Körper der reellen Zahlen} jeden vollständigen
    angeordneten Körper. Solge Körper notieren wir mit dem Symbol $\R$.
}
