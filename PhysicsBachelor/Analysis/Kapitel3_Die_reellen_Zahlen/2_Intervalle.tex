\subsection{Intervalle}

\vspace{1\baselineskip}

\Definition{

    Seien $a,b \in \R$. Dann ist das \fat{abgeschlossene Intervall} $[a,b]$
    definiert durch
    \begin{align*}
        [a,b] = \geschwungeneklammer{x \in \R \ | \ a \leq x \leq b}
    \end{align*} 
    und das \fat{offene Intervall} $(a,b)$ definiert durch
    \begin{align*}
        (a,b) = \geschwungeneklammer{x \in \R \ | \ a < x < b}
    \end{align*}
    Die \fat{halboffenen Intervalle} $[a,b)$ und $(a,b]$ sind durch
    \begin{align*}
        [a,b) = \geschwungeneklammer{x \in \R \ | \ a \leq x < b}
        \quad
        \text{  und     }
        \quad
        (a,b] = \geschwungeneklammer{x \in \R \ | \ a < x \leq b}
    \end{align*}
    definiert. Wir definieren die \fat{unbeschränkten abgeschlossenen
    Intervalle}
    \begin{align*}
        [a,\infty) = \R_{\geq a} =  \geschwungeneklammer{x \in \R \ | \ a \leq x}
        \quad
        \text{  und     }
        \quad
        (-\infty,b] = \R_{\leq b} = \geschwungeneklammer{x \in \R \ | \ x \leq b}
    \end{align*}
    sowie die \fat{unbeschränkten offenen Intervalle}
    \begin{align*}
        (a,\infty) = \R_{> a} =  \geschwungeneklammer{x \in \R \ | \ a < x}
        \quad
        \text{  und     }
        \quad
        (-\infty,b) = \R_{< b} = \geschwungeneklammer{x \in \R \ | \ x < b}
    \end{align*}
    und schliesslich $(-\infty,\infty) = \R$.
}

\vspace{1\baselineskip}

\Definition{

    Sei $x \in \R$. Eine Menge, die ein offenes Intervall enthält, in 
    dem $x$ liegt, wird auch eine \fat{Umgebung} oder \fat{Nachbarschaftne}
    von $x$ genannt. Für ein $\delta > 0$ wird das offene Intervall
    $(x-\delta,x+\delta)$ die \fat{$\delta$-Umgebung} von $x$ genannt.
}

\vspace{1\baselineskip}

\Definition{

    Eine Teilmenge $U \subseteq \R$ heisst \fat{offen} in $\R$, wenn für jedes
    $x \in U$ ein offenes Intervall $I$ mit $x \in I$ und $I \subset U$ 
    existert. Eine Teilmenge $F \subseteq \R$ heisst \fat{abgeschlossen} 
    in $\R$, wenn ihr Komplement $\R \backslash F$ offen ist.
}








