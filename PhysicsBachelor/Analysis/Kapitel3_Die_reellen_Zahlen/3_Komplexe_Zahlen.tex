\subsection{Komplexe Zahlen}

\vspace{1\baselineskip}

Es wird vorausgesetzt, dass der Leser, bzw die Leserin, bereits eine
Vorstellung hat was komplexe Zahlen sind und wie die Notation funktioniert. 
Falls dies nicht der Fall ist, muss sich mittels anderer Literatur 
geholfen werden. Hier werden nur Operationen und mathematische Tools
des Körpers der komplexen Zahlen thematisiert.

Kurze erläuterung der Notation: Wir schreiben $z = (x,y)$ wobei
$x$ der Realteil von $z$ und $y$ der Imaginärteil von $z$ ist.
Die gängigere Schreibweise ist jedoch $z = x + yi$.

\vspace{1\baselineskip}

\Definition{

    Wir nennen \fat{Addition} und \fat{Multiplikation} auf der Menge
    $\C$ die folgenden Operationen.

    \vspace{1\baselineskip}

    Addition: $(x_1,y_1) + (x_2,y_2) = (x_1 + x_2, y_1 + y_2)$  

    Multiplikation: $(x_1,y_1) \cdot (x_2,y_2) = (x_1 x_2 - y_1 y_2, x_1 y_2 + x_2 y_1)$
}

\vspace{1\baselineskip}

\Proposition{

    Die Menge $\C$, zusammen mit dem Nullelement $(0,0)$, dem Einselement
    $(1,0)$ und den vorher definierten Operationen, ist ein Körper.
}

\vspace{1\baselineskip}

\Definition{

    Sei $z = x + yi$ eine komplexe Zahl. Wir nennen $\overline{z} = x - yi$
    die zu $z$ \fat{konjugierte} komplexe Zahl. Die Abbildung $\C \rightarrow \C$
    gegeben durch $z \rightmapsto \overline{z}$ heisst \fat{komplexe Konjugation}.
}

\vspace{1\baselineskip}

\Lemma{

    Die komplexe Konjugation erfüllt folgende Eigenschaften:

    \vspace{1\baselineskip}

    1. Für alle \zinC ist $z \overline{z} \in \R$ und $z \overline{z} \geq 0$. 
    Des Weiteren gilt für alle $z \in \C$, dass $z \overline{z} = 0$ genau dann wenn $z=0$.

    2. Für alle $z,w \in \C$ gilt $\overline{z + w} = \overline{z} + \overline{w}$.

    3. Für alle $z,w \in \C$ gilt $\overline{z \cdot q} = \overline{z} \cdot \overline{w}$.
}

\vspace{1\baselineskip}

\Bemerkung{

    Es lässt sich keine Ordnung auf $\C$ definierten, die zur Addition
    und zur Multiplikation kompatibel ist.
}

\vspace{1\baselineskip}

\Definition{

    Der \fat{Absolutbetrag} oder die \fat{Norm} auf $\C$ ist die
    Funktion $\abs{ \hspace*{2pt} . \hspace*{2pt} } : \C \rightarrow \R$ gegeben durch
    \begin{align*}
        \abs{z} = \sqrt{z \overline{z}} = \sqrt{x^2 + y^2}
    \end{align*}
    für $z = x + yi \in \C$.
}

\vspace{1\baselineskip}

\Bemerkung{

    Der Ausolutbetrag ist positiv definit und multiplikativ, dh.
    \begin{align*}
        \abs{z w} = \sqrt{z w \overline{z w}} = \sqrt{z \overline{z}} \sqrt{w \overline{w}} = \abs{z} \abs{w}
        \quad \forall z,w \in \C
    \end{align*}
}

\vspace{1\baselineskip}

\Proposition{
    (Dreiecksungleichung)

    Für alle $z,w \in \C$ gilt $\abs{z + w} \leq \abs{z} + \abs{w}$.
}

\pagebreak

\Definition{
    (\fat{Cauchy-Schwarz Ungleichung})

    Für komplexe Zahlen $z = x_1 + y_1 i$ und $w = x_2 + y_2 i$ gilt
    \begin{align*}
        x_1 x_2 + y_1 y_2 \leq \abs{z} \abs{w}
    \end{align*}
}

\vspace{1\baselineskip}

\Definition{

    Die \fat{offene Kreisscheibe} mit Radius $r > 0$ um einen Punkt
    \zinC ist die Menge 
    \begin{align*}
        B(z,r) = \geschwungeneklammer{w \in \C \ | \ \abs{z-w} < r}
    \end{align*}
    Die \fat{abgeschlossene Kreisscheibe} mit Radius $r > 0$ um 
    \zinC ist die Menge
    \begin{align*}
        \overline{B(z,r)} = \geschwungeneklammer{w \in \C \ | \ \abs{z-w} \leq r}
    \end{align*}
}

\vspace{1\baselineskip}

\Definition{

    Eine Teilmenge $U \subseteq \C$ heisst \fat{offen} in $\C$, 
    wenn zu jedem Punkt in $U$ eine offene Kreisscheibe um diesen Punkt
    existert, die in $U$ enthalten ist. Formaler: Für alle $z \in U$ 
    existert ein Radius $r > 0$, so dass $B(z,r) \subseteq U$. 
    Eine Teilmenge $A \subseteq \C$ heisst \fat{abgeschlossen} in 
    $\C$, falls ihr Komplement $\C \backslash A$ offen ist.
}
