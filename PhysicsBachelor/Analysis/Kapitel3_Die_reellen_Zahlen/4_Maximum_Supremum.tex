\subsection{Maximum und Supremum}

\vspace{1\baselineskip}

\Definition{

    Eine Teilmenge $X \in \R$ ist \fat{von oben beschränkt}, falls es ein
    $s \in \R$ gibt mit $x \leq s$ für alle $x \in X$. Ein solches $s \in \R$
    nennt man eine \fat{obere Schranke} von $X$.
}

\vspace{1\baselineskip}

\Definition{

    Sei $X \subseteq \R$ eine Teilmenge. Ein Element $x_1 \in X$ heisst
    \fat{Maximum} von $X$ falls für alle \xinX die Ungleichung
    $x \leq x_1$ gilt. Wir schreiben
    \begin{align*}
        x_1 = \text{max}(X)
    \end{align*}
    falls das Maximum von $X$ existert und gleich $x_1$ ist.
}

\vspace{1\baselineskip}

\Bemerkung{

    Das Maximum ist eindeutig. Das bedeutet wenn $x_1$ und $x_2$
    beide Maxima einer Teilmenge sind, folgt 

    $x_1 \leq x_2 \land x_1 \geq x_2 \Rightarrow x_1 = x_2$.
}

\vspace{1\baselineskip}

\Definition{

    Die Begriffe \fat{von unten beschränkt} und \fat{untere Schranke}
    sowie \fat{Minimum} sind analog definiert. Wir schreiben
    \begin{align*}
        x_0 = \text{min}(X)
    \end{align*}
    falls das Minimum von $X$ existert und gleich $x_0$ ist. Eine
    Teilmenge $X \subseteq \R$ heisst \fat{beschränkt}, falls sie 
    von oben und von unten beschränkt ist.
}

\vspace{1\baselineskip}

\Definition{

    Sei \XsubeqR eine Teilmenge und sei $A := \geschwungeneklammer{a \in \R \ | \ x \leq a \ \forall x \in X}$
    die Menge aller oberen Schranken von $X$. Falls das Minimum $x_0 =$ min$(A)$
    existert, dann nennen wir dieses Minimum \fat{Supremum} von $X$,
    und wir schreiben $x_0 =$ sup$(X)$.
}

\pagebreak

\Satz{

    Sei \XsubR eine von oben beschränkte, nicht leere Teilmenge.
    Dann existert ein Supremum von $X$.    
}

\vspace{1\baselineskip}

\Proposition{

    Seien $X$ und $Y$ von oben beschränkte, nicht leere Teilmengen von 
    $\R$, und schreibe
    \begin{align*}
        X + Y := \geschwungeneklammer{x + y \ | \ x \in X, y \in Y}
        \quad
        \text{  und  }
        \quad
        X Y := \geschwungeneklammer{x y \ | \ x \in X , y \in Y}
    \end{align*}
    Die Mengen $X \cup Y$, $X \cap Y$ und $X + Y$ sind von oben beschränkt,
    und falls $x \geq 0$ für alle \xinX und $y \geq 0$ für alle \yinY gilt,
    so ist auch $X Y$ von oben beschränkt.

    \vspace{1\baselineskip}
    
    (1) Es gilt sup$(X \cup Y)$ = max$\geschwungeneklammer{\text{sup}(X),\text{sup}(Y)}$.

    (2) Falls $X \cap Y$ nicht leer ist, so gilt sup$( X \cap Y) \leq$ min$\geschwungeneklammer{\text{sup}(X),\text{sup}(Y)}$.

    (3) Es gilt sup$(X + Y) =$ sup$(X)$ $+$ sup$(Y)$.

    (4) Falls $x \geq 0$ für alle \xinX und $y \geq 0$ für alle $y \in Y$,
    so gilt sup$(X Y)$ $=$ sup$(X)$sup$(Y)$
}

\vspace{1\baselineskip}

\Definition{

    Für eine von unten beschränkte, nicht leere Teilmenge \XsubeqR wird
    die grösste untere Schranke von $X$ auch als \fat{Infimum} inf$(X)$
    von $X$ genannt. Es gilt die zum vorherigen Satz analoge 
    Existenzaussage für das Infimum. Alternativ kann man das Infimum von
    $X$ als
    \begin{align*}
        - \text{sup}\geschwungeneklammer{-x \ | \ x \in X}
    \end{align*}
    definieren. Überhaupt kann man dadurch praktisch alle Aussagen über 
    Infima auf Aussagen über Suprema 
    
    zurückführen.
}

\vspace{1\baselineskip}

\Definition{

    Sei $X$ eine Teilmenge von $\R$. Falls \XsubR nicht von oben beschränkt
    ist, dann definieren wir sup$(X) = \infty$. Falls $X$ leer ist, setzen
    wir sup$(\emptyset) = - \infty$. Wir nennen in diesem Zusammenhang
    $\infty$ und $- \infty$ \fat{uneigentliche Werte}.
}
