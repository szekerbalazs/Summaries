\subsection{Konsequenz der Vollständigkeit}

\vspace{1\baselineskip}

\Satz{
    (Archimedisches Prinzip)

    Für jedes \xinR existeret genau ein $n \in \Z$ mit $n \leq x < n+1$.
}

\vspace{1\baselineskip}

\Korollar{

    Für jedes $\epsilon > 0$ existert eine ganze Zahl $n \geq 1$ so,
    dass $\frac{1}{n} < \varepsilon$ gilt.
}

\vspace{1\baselineskip}

\Korollar{

    Sei \xinR mit $x >0$. Dann existert eine $n \in \Z , n > 0$ mit
    $0 < \frac{1}{n} < x$.
}

\vspace{1\baselineskip}

\Definition{

    Der \fat{ganzzahlige Anteil} $\lfloor x \rfloor$ einer Zahl 
    \xinR ist die eindeutig bestimmte ganze Zahl $n \in \Z$ mit 
    $n_1 \leq x < n_1 + 1$. Die durch $x \rightmapsto \floor{x}$
    gegebene Funktion von $\R$ nach $\Z$ heisst \fat{Abrundungsfunktion}.
    Der \fat{gebrochene Anteil} einer reellen Zahl $x$ ist 
    $\geschwungeneklammer{x} = x - \floor{x} \in [0,1)$.
}

\pagebreak

\Definition{

    Eine Teilmenge $X$ von $\R$ heisst \fat{dicht} in $\R$ falls es 
    für jedes \xinR und jedes $\delta > 0$ einen Ball $B(x,\delta)$ gibt
    sodass $B(x,\delta) \cap X \neq \emptyset$.
}

\vspace{1\baselineskip}

\Korollar{

    $\Q$ ist dicht in $\R$.
}

\vspace{1\baselineskip}

\Korollar{

    Zu je zwei reellen Zahlen $a,b \in \R$ mit $a < b$ gibt es ein 
    $r \in \Q$ mit $a < r < b$.
}

\vspace{1\baselineskip}

\Proposition{

    Die Menge $\R$ ist nicht abzählbar.
}

\vspace{1\baselineskip}

\Definition{

    Sei $A \subset \R$ und $x \in \R$. Wir sagen, dass $x$ ein \fat{Häufungspunkt}
    der Menge $A$ ist, falls es für jedes $\varepsilon > 0$ ein $a \in A$ mit
    $0 < \abs{a - x} < \varepsilon$ gibt.

    Der Häufungspunkt muss nicht in $A$ liegen.
}

\vspace{1\baselineskip}

\Bemerkung{

    $A \subseteq \R$ ist dicht $\Leftrightarrow$ Jedes \xinR ist ein Häufungspunkt von $A$.
}

\vspace{1\baselineskip}

\Satz{

    Sei $A \subset \R$ eine beschränkte unendliche Teilmenge. Dann
    existert ein Häufungspunkt von $A$ in $\R$.
}

\vspace{1\baselineskip}

\Bemerkung{
    \begin{align*}
        \bigcap_{n=1}^{\infty} [n,\infty) = \emptyset
        \quad
        \text{  und  }
        \quad
        \bigcap_{n=1}^{\infty} (0,\frac{1}{n}) = \emptyset
    \end{align*}
}

\vspace{1\baselineskip}

\Satz{

    Sei $\mathcal{F}$ eine nichtleere Familie von beschränkten und abgeschlossenen
    Teilmengen von $\R$, mit folgenden Eigenschaften:

    1. $\emptyset \not\in \mathcal{F}$

    2. $(F_1 \in \mathcal{F}) \land (F_2 \in \mathcal{F}) \Rightarrow F_1 \cap F_2 \in \mathcal{F}$

    Dann ist der Durchschnitt 
    \begin{align*}
        \bigcap_{F \in \mathcal{F}} F
    \end{align*}
    nicht leer.
}

\vspace{1\baselineskip}

\Korollar{
    (Intervallschachtlungsprinzip)

    Sei für jedes \ninN ein nicht-leeres, abgeschlossenes, beschränktes
    Intervall $I_n = [a_n,b_n]$ gegeben, so dass für alle natürlichen
    Zahlen $m \leq n$ die Inklusion $I_m \supset I_n$ gilt. 
    Dann ist der Durchschnitt
    \begin{align*}
        \bigcap_{n=1}^{\infty} I_n = [\text{sup}\geschwungeneklammer{a_n \ | \ n \in \N}, \text{inf}\geschwungeneklammer{b_n \ | \ n \in \N}]
    \end{align*}
    nicht-leer.
}
