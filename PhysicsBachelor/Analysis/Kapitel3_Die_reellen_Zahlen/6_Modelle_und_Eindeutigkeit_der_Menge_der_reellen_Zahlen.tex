\subsection{Modelle und Eindeutigkeit der Menge der reellen Zahlen}

\vspace{1\baselineskip}

\Definition{

    Wir schreiben $C(0) = \geschwungeneklammer{x \in \Q \ | \ x > 0}$, und nennen
    \fat{Dedekind-Schnitt} jede nicht leere, von unten beschränkte Teilmenge
    $C$ von $\Q$ mit der Eigenschaft, dass $C = C(0) + C$, also
    \begin{align*}
        C = \geschwungeneklammer{x + c \ | \ x \in C(0), c \in C}
    \end{align*}
    gilt. Wir schreiben $\mathcal{D}$ für die Menge aller Dedekind-Schnitte.
}

\vspace{1\baselineskip}

\Bemerkung{

    Für jede rationale Zahl $q$ ist die Menge $C(q) = \geschwungeneklammer{x \in \Q \ | \ q < x}$
    ein Dedekindschnitt.
}

\vspace{1\baselineskip}

\Satz{
    (Dedekind)

    Mit der Addition, Multipliation und der Ordnungsrelation, dem 
    Nullelement $C(0)$ und dem Einselement $C(1)$ bildet die Menge
    der Dedekind-Schnitte $\mathcal{D}$ einen vollständigen,
    angeordneten Körper. Die Rechenoperationen sind wir folgt gegeben:

    \vspace{1\baselineskip}

    Addition: $C(p+q) = C(p) + C(q)$

    Multipliation: $C(pq) = C(p)C(q)$

    Ordnungsrelation: $p \leq q \Rightarrow C(p) \leq C(q)$
}

\vspace{1\baselineskip}

\Definition{

    Seien $C$ und $D$ Dedekind-Schnitte. Die \fat{Summe} von $C$ und $D$
    ist der Dedekind-Schnitt 
    \begin{align*}
        C + D = \geschwungeneklammer{c + d \ | \ c \in C , d \in D}
    \end{align*}
    Wir sagen $C$ ist \fat{kleiner} als $D$ und schreiben $C \leq D$
    falls $C \supseteq D$ gilt.
}

\vspace{1\baselineskip}

\Lemma{

    Mit der Operation $+$ ist die Menge der Dedekind-Schnitte $\mathcal{D}$
    dine kommutative Gruppe mit neutralem Element $C(0)$. Ausserdem gilt

    1. $C \leq D$ oder $D \leq C$ für alle $C,D \in \mathcal{D}$

    2. $C \leq D \Rightarrow C + E \leq D + E$ für alle $C,D,E \in \mathcal{D}$
}

\vspace{1\baselineskip}

\Definition{

    Seien $C$ und $D$ Dedekind-Schnitte. Das \fat{Produkt} von $C$ und $D$
    ist der Dedekind-Schnitt
    \begin{align*}
        C D = \begin{cases}
            \geschwungeneklammer{c d \ | \ c \in C , d \in D} \ &\text{ falls } C(0) \leq C \text{ und } C(0) \leq D
            \\
            - \geschwungeneklammer{c d \ | \ c \in -C , d \in D} \ &\text{ falls } C(0) \geq C \text{ und } C(0) \leq D
            \\
            - \geschwungeneklammer{c d \ | \ c \in C , d \in -D} \ &\text{ falls } C(0) \leq C \text{ und } C(0) \geq D
            \\
            \geschwungeneklammer{c d \ | \ c \in -C , d \in -D} \ &\text{ falls } C(0) \geq C \text{ und } C(0) \geq D
        \end{cases}
    \end{align*}
}

\vspace{1\baselineskip}

\Lemma{

    Mit der vorher eingeführten Addition, Multipliation und Ordnungsrelation,
    dem Nullelement $C(0)$ und dem Einselement $C(1)$ bildet die Menge der 
    Dedekind-Schnitte $\mathcal{D}$ einen angeordneten Körper.
}

\vspace{1\baselineskip}

\Lemma{

    Der angeordnete Körper $(\mathcal{D}, \leq)$ ist vollständig.
}

\pagebreak

\Satz{
    (Eindeutigkeit der reellen Zahlen)

    Seien $(\R, \leq)$ und $(\eS,\leq)$ vollständig angeordnete Körper.
    Dann existiert genau eine Abbildung $\Phi : \R \rightarrow \eS$, die die 
    folgenen Eigenschaften erfüllt:

    \vspace{1\baselineskip}

    1. Additivität: $\Phi (0) = 0$ und $\Phi(x+y) = \Phi (x) + \Phi (y)$ für alle \xyinR

    2. Multiplikativität: $\Phi (1) = 1$ und $\Phi(x y) = \Phi(x) \Phi(y)$ für alle \xyinR

    3. Monotonie: $x \leq y \ \Rightarrow \Phi(x) \leq \Phi(y)$ für alle \xyinR 

    \vspace{1\baselineskip}

    Diese Abbildung $\Phi$ ist bijektiv. Eine solche Abbildung nennt man auch
    \fat{Isomorphismus}.
}