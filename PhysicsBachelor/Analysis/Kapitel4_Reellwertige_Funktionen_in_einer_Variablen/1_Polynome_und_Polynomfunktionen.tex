\subsection{Polynome und Polynomfunktionen}

\vspace{1\baselineskip}

In diesem Kapitel wird vom Leser, bzw von der Leserin, vorausgesetzt, dass er,
bzw sie, bereits vertraut ist mit der \fat{Summennotation} $\sum$ und der 
\fat{Produktnotation} $\prod$ und den dazugerörigen Konventionen der Indexierung.
Eine Sache, die es dennoch wert ist zu erwähnen, ist, dass wir die folgende
Konventionen brauchen werden.
\begin{align*}
    \sum_{j \in \emptyset} a_j = 0
    \quad
    \text{  und  }
    \quad
    \prod_{j \in \emptyset} a_j = 1
\end{align*}

\vspace{1\baselineskip}

\Bemerkung{

    Die Summe ist linear. Das heisst:
    \begin{align*}
        \csum{k=1}{n} (a_k + b_k) = \csum{k=1}{n} a_k + \csum{k=1}{n} b_k
        \quad
        \text{  und  }
        \quad
        \csum{k=1}{n}(c a_k) = c \csum{k=1}{n} a_k
    \end{align*}
}

\vspace{1\baselineskip}

\Bemerkung{

    Für die sogenannte \fat{Teleskopsumme} gilt:
    \begin{align*}
        \csum{k=1}{n-1} (a_k - a_{k+1}) = a_1 - a_n
    \end{align*}
}

\vspace{1\baselineskip}

\Bemerkung{

    Für das Produkt gilt:
    \begin{align*}
        \prod_{k=1}^n (a_k b_k) = \klammer{\prod_{k=1}^n a_k} \klammer{\prod_{k=1}^n b_k}
        \quad
        \text{  und  }
        \quad
        \prod_{k=1}^n (c a_k) = c^n \cdot \prod_{k=1}^n a_k
    \end{align*}
}

\vspace{1\baselineskip}

\Lemma{
    (Bernoulli'sche Ungleichung)

    Für alle reellen Zahlen $a \geq -1$ und \ninN gilt $(1+a)^n \geq 1 + n a$.
}

\vspace{1\baselineskip}

\Proposition{
    (Geometrische Summenformel)

    Sei \ninN und $q \in \C$. Dann gilt:
    \begin{align*}
        \csum{k=0}{n} q^k = \begin{cases}
            n + 1 \quad &\text{ falls } q = 1
            \\
            \frac{q^{n+1} - 1}{q - 1} \quad &\text{ falls } q \neq 1
        \end{cases}
    \end{align*}

}

\vspace{1\baselineskip}

\Definition{

    Sei $K$ ein beliebiger Körper. Ein \fat{Polynom} $f$ in einer Variable
    $T$ und Koeffizienten in $K$ ist ein formaler Ausdruck der Form
    \begin{align*}
        f = a_0 + a_1 T + a_2 T^2 + \dots + a_n T^n = \csum{k=0}{n} a_k T^k
    \end{align*}
    für ein \ninN und Elementen $a_0, \dots a_n \in K$ die wir als 
    \fat{Koeffizienten} bezeichnen. Hierbei ist $T$ die sogenannte
    \fat{Variable}. Wir definieren den \fat{Polynomring} $K[T]$ als 
    die Menge der Polynome mit Koeffizienten in $K$ in der Variablen
    $T$, wobei Addition und Multiplikation formal durch Kommutativität
    und Distributionsgesetz gegeben ist.
}

\pagebreak

\Definition{

    Sei $f \in K[T]$ ein Polynom, gegeben wie in der vorherigen Definition.
    Falls $a_n \neq 0$, so nennen wir $a_n$ den \fat{Leitkoeffizienten}, und 
    die natürliche Zahl $n$ den \fat{Grad} von $f$. Wir definieren den Grad
    des Polynoms $f = 0$ formal als $- \infty$. Ein Polynom von Grad $\leq 0$
    heisst \fat{konstant}, ein Polynom von Grad $\leq 1$ heisst \fat{affin},
    und ein Polynom vom Grad $\leq 2$ heisst \fat{quadratisch.}
}

\vspace{1\baselineskip}

\Bemerkung{

    Ein Polynom $f$ ist keine Funktion.
}

\vspace{1\baselineskip}

\Definition{

    Eine \fat{Polynomfunktion} auf $\C$ ist eine Funktion $\C \rightarrow \C$
    der Form $z \rightmapsto f(z)$ für ein Polynom $f \in \C[T]$. Analog
    dazu definieren wir auch Polynomfunktionen auf $\R$.
}

\vspace{1\baselineskip}

\Proposition{

    Sei $f(T) \in \C [T]$ ein nicht-konstantes Polynom. Dann gibt es zu jeder
    positiven reellen Zahl $M > 0$ eine reelle Zahl $R \geq 1$, so dass für 
    alle \zinC mit $\abs{z} \geq R$ auch $\abs{f(z)} \geq M$ gilt.
}

\vspace{1\baselineskip}

\Korollar{

    Die Zuordnung, die jedem Polynom $f(T) \in \C [T]$ die zugehörige
    Polynomfunktion $z \in \C \rightmapsto f(z) \in \C$ zuweist bijektiv.
}

\vspace{1\baselineskip}

\Definition{

    Eine Nullstelle eines Polynoms $f \in K[T]$ ist eine Zahl $x \in K$
    mit $f(x) = 0$.
}

\vspace{1\baselineskip}

\Satz{
    (Fundamentalsatz der Algebra)

    Jedes nicht-konstante Polynom $f \in \C [T]$ mit komplexen Koeffizienten
    hat eine komplexe Nullstelle.
}

\vspace{1\baselineskip}

\Definition{

    Wir sagen, dass ein Polynom $g$ ein Polynom $f$ \fat{teilt} falls es ein
    Polynom $q$ gibt mit $f = qg$.
}

\vspace{1\baselineskip}

\Definition{

    Wir sagen, dass eine Nullstelle $z \in K$ von $f(T) \in K[T]$ \fat{Vielfachheit}
    $k \in \N$ hat, falls $(T-z)^k$ das Polynom $f$ teilt, aber $(T-z)^{k+1}$
    das Polynom $f$ nicht teilt.
}

\vspace{1\baselineskip}

\Satz{

    Ein Polynom $f(T) \in K[T]$ vom Grad $n \geq 1$ hat $n$ Nullstellen in 
    $K$ auch wenn man diese entsprechend ihrer Vielfachheit mehrfach zählt.
}

\vspace{1\baselineskip}

\Proposition{

    Sei $n \geq 1$ eine ganze Zahl, $z_1 , \dots , z_n \in K$ paarweise verschiedene
    Elemente, und $w_1, \dots , w_n \in K$ beliebige Elemente. Es gibt höchstens
    ein Polynom $f \in K[T]$ von Grad $n-1$ mit der Eigenschaft
    \begin{align*}
        f(z_k) = w_k \ \forall k = 1,2, \dots ,n
    \end{align*} 
    Das Auffinden eines solchen Polynoms wird als \fat{Lagrange Interpolation}
    bezeichnet. Wir beginnen damit, für jedes $k = 1,2, \dots , n$ das Polynom
    \begin{align*}
        Q_k(T) = \prod_{\substack{j=1 \\ j \neq k}}^n \frac{T - z_j}{z_k - z_j}
    \end{align*}
    zu definieren. Die Notation bedeutet hier, dass wir das Produkt über alle
    $j \in \geschwungeneklammer{1,2, \dots , n} \backslash \geschwungeneklammer{k}$
    nehmen. Das Polynom $Q_k (T)$ hat Grad $n-1$, und es gilt
    \begin{align*}
        Q_k (z_j) = \begin{cases}
            1 \quad \text{  falls } j = k
            \\
            0 \quad \text{  falls } j \neq k
        \end{cases}
    \end{align*}
    Das gesuchte Polynom $f \in K[T]$ erhalten wir als Linearkombination
    \begin{align*}
        f(T) = \sum_{k=1}^n w_k Q_k (T)
    \end{align*}
    wobei wir $f(z_k) = w_k$ wiederum direkt durch Einsetzen in die Formel sehen.
    Da alle Polynome $Q_k$ vom Grad $n-1$ sind, ist $f$ höchstens vom Grad $n-1$.
}

\vspace{1\baselineskip}

\Definition{

    Für ein rationales Polynom $\frac{f}{g}$ nennt man die Nullstellen
    von $g$ \fat{Pole}.
}

\vspace{1\baselineskip}

\Definition{

    Eine Zahl $\alpha \in \C$ heisst \fat{algebraisch}, falls es ein von 
    Null verschiedenes Polynom $f \in \Q[x]$ gibt mit $f(\alpha) = 0$.  
}