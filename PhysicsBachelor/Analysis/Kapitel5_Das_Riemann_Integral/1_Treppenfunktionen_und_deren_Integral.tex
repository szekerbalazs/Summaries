\subsection{Treppenfunktionen und deren Integral}

\vspace{1\baselineskip}

\Definition{

    Eine \fat{Zerlegung} von $[a,b]$ ist eine endliche Menge von Punkten
    \begin{align*}
        a = x_0 < x_1 < \dots < x_{n-1} < x_n = b
    \end{align*}
    mit $n \in \N$. Die Punkte $x_0 , \dots , x_n \in [a,b]$ werden Teilungspunkte
    der Zerlegung genannt.
}

\vspace{1\baselineskip}

\Definition{

    Eine Zerlegung $a = y_0 < y_1 < \dots < y_{n-1} < y_m = b$ wird 
    \fat{Verfeinerung} von $a = x_0 < x_1 < \dots < x_{n-1} < x_n = b$
    genannt, falls 
    \begin{align*}
        \geschwungeneklammer{x_0 , x_1 , \dots , x_{n-1} , x_n } \subseteq \geschwungeneklammer{y_0 , y_1 , \dots , y_{m-1}} , y_m
    \end{align*}
    gilt.
}

\vspace{1\baselineskip}

\Definition{

    Eine Funktion $f: [a.b] \rightarrow \R$ heisst \fat{Treppenfunktion},
    falls es eine Zerlegung $a = x_0 < x_1 < \dots < x_{n-1} < x_n = b$
    von $[a,b]$ gibt, so, dass für $k=1,2,\dots,n$ die Einschränkung von 
    $f$ auf das offene Intervall $(x_{k-1},x_k)$ konstant ist. Wir sagen in dem
    Fall auch die Funktion $f$ sei eine Treppenfunktion bezüglich der 
    Zerlegung $a = x_0 < x_1 < \dots < x_{n-1} < x_n = b$.
}

\vspace{1\baselineskip}

\Proposition{

    Seien $f_1$ und $f_2$ Treppenfunktionen auf $[a,b]$ und seien $s_1, s_2 \in \R$.
    Dann sind auch $s_1 f_1 + s_2 f_2$ und $f_1 \cdot f_2$ Treppenfunktionen.
}

\vspace{1\baselineskip}

\Bemerkung{
 
    Die Menge aller Treppenfunktionen auf $[a,b]$
    \begin{align*}
        \mathcal{T F}([a,b]) = \geschwungeneklammer{f \in \mathcal{F}([a,b]) \ | \ f \text{ ist eine Treppenfunktion}}
    \end{align*}
    ist ein Vektorraum, oder genauer, ein Untervektorraum des Vektorraums
    $\mathcal{F}([a,b])$ der reellwertigen Funktionen auf $[a,b]$.
    Da das Produkt zweier Treppenfunktionen wiederum eine Treppenfunktion ist,
    bilden die Treppenfunktionen einen Ring.
    Des Weiteren sind Treppenfunktionen beschränkt, da sie endliche Wertemengen
    haben.
}

\vspace{1\baselineskip}

\Definition{

    Sei $f: [a,b] \rightarrow \R$ eine Treppenfunktion bezüglich einer Zerlegung
    $a = x_0 < x_1 < \dots < x_{n-1} < x_n = b$ von $[a,b]$. Wir definieren
    das \fat{Integral} von $f$ auf $[a,b]$ als die reelle Zahl
    \begin{align*}
        \intab f(x) dx = \csum{k=1}{n} c_k (x_k - x_{k-1})
    \end{align*}
    wobei $c_k$ den Wert von $f$ auf dem Intervall $(x_{k-1},x_k)$ bezeichnet.
}

\vspace{1\baselineskip}

\Bemerkung{

    Ist $a = y0 < y_1 < \dots < y_{m-1} < y_m = b$ eine weitere Zerlegung
    von $[a,b]$ befüglich welcher $f$ eine Treppenfunktion ist, so muss gelten:
    \begin{align*}
        \csum{k=1}{n} c_k (x_k - x_{k-1}) = \csum{k=1}{m} d_k (y_k - y_{k-1})
    \end{align*}
}

\vspace{1\baselineskip}

\Proposition{

    Die Abbildung $\int : \mathcal{T F} ([a,b]) \rightarrow \R$ ist linear.
    Das heisst, für alle $f,g \in \mathcal{T F}([a,b])$ und $\alpha , \beta \in \R$
    ist $\alpha f + \beta g$ eine Treppenfunktion, und es gilt
    \begin{align*}
        \intab (\alpha f + \beta g)(x) dx = \alpha \intab f(x) + \beta \intab g(x) dx
    \end{align*}
}

\vspace{1\baselineskip}

\Proposition{

    Seien $f$ und $g$ Treppenfunktionen auf $[a,b]$ mit $f \leq g$. Dann
    gilt:
    \begin{align*}
        \intab f dx \leq \intab g dx
    \end{align*}
}