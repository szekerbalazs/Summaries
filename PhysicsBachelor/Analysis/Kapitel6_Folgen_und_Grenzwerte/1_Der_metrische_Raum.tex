\subsection{Der metrische Raum}

\vspace{1\baselineskip}

\Definition{

    Ein \fat{metrischer Raum} $(X,d)$ ist eine Menge $X$ gemeinsam mit einer
    Abbildung $d: X \times X \rightarrow \R$, die die \fat{Metrik} auf $X$
    genannt wird und die folgenden drei Eigenschaften erfüllt.

    \vspace{1\baselineskip}

    1. Definitheit: Für alle $x_1,x_1 \in X$ gilt $d(x_1 , x_2) \geq 0$ und $d(x_1 , x_2) = 0 \Leftrightarrow x_1 = x_2$

    2. Symmetrie: Für alle $x_1 , x_2 \in X$ gilt $d(x_1 , x_2) = d(x_2 , x_1)$

    3. Dreiecksungleichung: Für alle $x_1 , x_2 x_3 \in X$ gilt $d(x_1 , x_3) \leq d(x_1 , x_2) + d(x_2 , x_3)$

    \vspace{1\baselineskip}

    Eine Metrik $d$ auf einer Menge $X$ weist je zwei Punkten ihre 
    \fat{Distanz} oder \fat{Abstand} zu.
}

\vspace{1\baselineskip}

\Definition{

    Die \fat{Standardmetrik} $d$ ist definiert durch 
    \begin{align*}
        d(x_1 , x_2) = \abs{x_1 - x_2}
    \end{align*}
    für alle $x_1 , x_2 \in X$.
}

\vspace{1\baselineskip}

\Definition{

    Die \fat{diskrete Metrik} $d$ auf einer Menge $X$ ist definiert durch
    \begin{align*}
        d(x_1 , x_2) = \begin{cases}
            1 \quad \text{ falls } x_1 \neq x_2 \\
            0 \quad \text{ falls } x_1 = x_2
        \end{cases}
    \end{align*}
}

\vspace{1\baselineskip}

\Definition{

    Wir setzen $X = \C$ und definieren die \fat{Manhatten-Metrik} $d_{NY}$ auf $X$ durch
    \begin{align*}
        d_{NY} (z_1 , z_2) = \abs{x_1 - x_2} + \abs{y_1 - y_2}
    \end{align*}
    für komplexe Zahlen $z_1 = x_1 + i y_1$ und $z_2 = x_2 + i y_2$.
}

\vspace{1\baselineskip}

\Definition{

    Eine weitere Metrik auf $\C$ ist die \fat{französische Eisenbahn Metrik}
    $d_{SNFC}$, definiert durch
    \begin{align*}
        d_{SNFC} (z_1, z_2) = \begin{cases}
            \abs{z_1 - z_2} \quad &\text{ falls } z_1 , z_2 \text{ linear abhängig über $\R$ sind}
            \\
            \abs{z_1} + \abs{z_2} \quad &\text{ falls } z_1 , z_2 \text{ linear unabhängig über $\R$ sind}
        \end{cases}
    \end{align*}
    für alle $z_1 , z_2 \in \C$.
}

\vspace{1\baselineskip}

\Definition{

    Sei $(X,d)$ ein metrischer Raum, $x_0 \in X$, $r \geq 0$ reell.
    Wir nennen die Teilmenge
    \begin{align*}
        B(x_0 , r) = \geschwungeneklammer{x \in X \ | \ d(x_0 ,x) < r}
    \end{align*}
    "offener Ball mit Zentrum $x_0$ und Radius $r$".
}

\pagebreak

\Definition{

    Sei $(X,d)$ ein metrischer Raum, $A \subseteq X$ eine Teilmenge.
    Wir sagen $A$ sei beschränkt falls eine reelle Zahl $R \geq 0$ existiert
    mit 
    \begin{align*}
        d(x,y) \leq R \ \forall x,y \in A
    \end{align*}
}

\vspace{1\baselineskip}

\Bemerkung{

    Ist $x_0 \in X$, so ist $A \subseteq X$ beschränkt $\Leftrightarrow$ $\exists R \geq 0$ mit $A \subseteq B(x_0, R)$
}

\vspace{1\baselineskip}

\Definition{

    Sei $(X,d)$ ein metrischer Raum und $A \subseteq X$. Wir sagen $A$ sei
    \fat{offen} in $X$, falls $\forall x_0 \in A \ \exists \delta > 0$ mit
    $B(x_0 , \delta) \subseteq A$. Wir nennen $B \subseteq X$ 
    \fat{abgeschlossen} in $X$, falls $X \backslash B$ offen in $X$ ist.
}

\vspace{1\baselineskip}

\Proposition{

    Sei $(X,d)$ ein metrischer Raum. Jede Vereinigung offener Teilmengen
    von $X$ ist offen. Jeder \underline{endliche} Durchschnitt offener
    Teilmengen von $X$ ist offen.
}

\vspace{1\baselineskip}

\Definition{

    Seien $(X,d_x)$ und $(Y,d_y)$ metrische Räume. Eine Funktion $f: X \rightarrow Y$
    heisst \fat{stetig} im Punkt $x_0 \in X$, falls
    \begin{align*}
        \forall \varepsilon > 0 \ \exists \delta > 0 \text{ mit } d_x(x,x_0) < \delta \Rightarrow d_y ( f(x) , f(x_0)) < \varepsilon \ \forall x \in X
    \end{align*}
    $f$ ist stetig auf $X$ falls $f$ stetig in jedem Punkt $x_0 \in X$ ist.

    \vspace{1\baselineskip}

    Ist $X \subseteq \R$ und $Y \subseteq \R$ mit der Standardmetrik
    $d(x,y) = \abs{x-y}$ auf $X$ und $\R$, so erhalten wir den bekannten
    Stetigkeitsbegriff.
}

\vspace{1\baselineskip}

\Definition{

    Seien $(X,d_x)$ und $(Y,d_y)$ metrische Räume. Eine Funktion $f: X \rightarrow Y$
    heisst \fat{gleichmässig stetig}, falls
    \begin{align*}
        \forall \varepsilon > 0 \ \exists \delta > 0 \text{ mit } d_x (x_1 , x_2) < \delta \Rightarrow d_y (f(x_1) , f(x_2)) < \varepsilon \ \forall x_1 , x_2 \in X
    \end{align*}
}

\vspace{1\baselineskip}

\Definition{

    Seien $(X,d_x)$ und $(Y,d_y)$ metrische Räume. Eine Funktion $f: X \rightarrow Y$
    heisst \fat{Lipschitz stetig}, falls
    \begin{align*}
        \exists L > 0 \text{ (reell) mit } d_y (f(x_1) , f(x_2)) \leq L \cdot d_x (x_1 , x_2)
    \end{align*}
}

\vspace{1\baselineskip}

\Definition{

    Seien $(X,d_x)$ und $(Y,d_y)$ metrische Räume. Eine Funktion $f: X \rightarrow Y$
    heisst \fat{Isometrie}, falls
    \begin{align*}
        d_y (f(x_1) , f(x_2)) = d_x (x_1 , x_2) \ \forall x_1 , x_2 \in X
    \end{align*}
}

\vspace{1\baselineskip}

\Bemerkung{

    $f$ ist eine Isometrie $\Rightarrow$ $f$ ist Lipschitz-stetig
    $\Rightarrow$ $f$ ist gleichmässig stetig $\Rightarrow$ $f$ ist stetig
}

\pagebreak

\Proposition{

    Seien $(X,d)$ und $(Y,d)$ metrische Räume und $f: X \rightarrow Y$
    eine Funktion. Dann sind äquivalent:

    \begin{enumerate}
        \item $f$ ist stetig.
        \item Für jede offene Teilmenge $U \subseteq Y$ ist das Urbild $f^{-1} (U) \subseteq X$
                offen. $f^{-1} = \geschwungeneklammer{x \in X \ | \ f(x) \in U}$.
        \item Ist $(x_n)_{n=0}^{\infty}$ eine konvergente Folge in $X$ mit
                Grenzwert $a \in X$, so ist $(f(x_n))_{n=0}^{\infty}$ konvergent,
                mit Grenzwert $f(a)$.
        \item Für alle \xinX und $\forall \epsilon > 0$ existiert ein $\delta > 0$ so, dass
                $d(x,x_1) < \delta \ \Rightarrow \ d(f(x),f(x_1)) < \epsilon$.
    \end{enumerate}

}