\subsection{Die Exponentialfunktion}

\vspace{1\baselineskip}

\Proposition{

    Sei \xinR eine reelle Zahl. Die Folge reeller Zahlen $\aFolge$ gegeben durch
    \begin{align*}
        a_n = \klammer{1 + \frac{x}{n}}^n
    \end{align*}
    ist konvergent, und ihr Grenzwert ist eine positive reelle Zahl.
}

\vspace{1\baselineskip}

\Definition{

    Die \fat{Exponentialfunktion} $\exp: \R \rightarrow \R_{\geq 0}$ ist die durch
    \begin{align*}
        \exp (x) = \limesninf \klammer{1 + \frac{x}{n}}^n
    \end{align*}
    für alle \xinR definierte Funktion. Die \fat{Euersche Zahl} $e \in \R$ ist definiert als
    \begin{align*}
        e = \exp (1) = \limesninf \klammer{1 + \frac{1}{n}}^n
    \end{align*}
}

\vspace{1\baselineskip}

\Satz{

    Die Exponentialfunktion $\exp : \R \rightarrow \R_{> 0}$ ist bijektiv, streng monoton
    steigend, und stetig. Ausserdem gilt:
    \begin{align*}
        \exp (0) &= 1 \\
        \exp (-x) &= \exp (x)^{-1} \ \ \text{für alle \xinX} \\
        \exp (x + y) &= \exp (x) \exp (y) \ \ \text{für alle } x,y \in \R
    \end{align*}
}

\vspace{1\baselineskip}

\Satz{

    Der Logarithmus $\log : \R_{> 0} \rightarrow \R$ ist eine streng monoton wachsende,
    stetige und bijektive Funktion. Des weiteren gilt:
    \begin{align*}
        \log (1) &= 0 \\
        \log (a^{-1}) &= - \log(a) \ \ \text{für alle } a \in \R_{> 0} \\
        \log (a b) &= \log (a) + log (b) \ \ \text{für alle } a,b \in \R_{> 0}
    \end{align*}
    Diese Funktion ist die eindeutige Umkehrfunktion von $\exp: $.
}

\vspace{1\baselineskip}

\Lemma{ (Bernoulli Ungleichung)

    Sei $a \in \R$ mit $a \geq -1$. Dann gilt für alle \ninN:
    \begin{align*}
        \klammer{1 + a}^n \geq 1 + n \cdot a
    \end{align*}
}

\vspace{1\baselineskip}

\Bemerkung{

    Für eine positive Zahl $a > 0$ und beliebigen Exponenten \xinR schreiben wir
    \begin{align*}
        a^x := \exp (x \log (a))
    \end{align*}
}
