\subsection{Normen und Konvergenzen auf Vektorräumen}

\vspace{1\baselineskip}

In diesem Kapitel legen wir einen Körper $\K$ fest, der entweder $\R$ oder $\C$ ist.

\vspace{1\baselineskip}

\Definition{ 

    Sei $V$ ein Vekorraum über $\K$. Eine \fat{Norm} auf $V$ ist eine Abbildung
    $\Norm{\ \cdot \ }: V \rightarrow \R$, die folgende drei Eigenschaften erfüllt.
    
    (1) (Definitheit) Für alle \vinV gilt $\Norm{v} \geq 0$ und $\Norm{v}=0 \Leftrightarrow v = 0$.

    (2) (Homogenität) Für alle \vinV und alle $\alpha \in \R$ (bzw $\alpha \in \C$) gilt
        $\Norm{\alpha v} = \abs{\alpha} \Norm{v}$.

    (3) (Dreiecksungleichung) Für alle $v_1 , v_2 \in V$ gilt $\Norm{v_1 + v_1} \leq \Norm{v_1} + \Norm{v_2}$.

    Man nennt $V$ gemeinsam mit der Norm $\Norm{\ \cdot \ }$ einen \fat{normierten Vektorraum}.
}

\pagebreak

\Beispiel{

    Sei $n \in \N$. Die \fat{Maximumsnorm} oder \fat{Unendlichnorm} $\Norm{ \ \cdot \ }_{\infty}$
    und die \fat{1-Norm} $\Norm{ \ \cdot \ }_1$ auf $\K^n$ sind definiert durch
    \begin{align*}
        \Norm{v}_{\infty} = \maximum \geschwungeneklammer{\abs{v_1},\abs{v_2},\dots,\abs{v_n}}
        \quad \text{   und   } \quad
        \Norm{v}_1 = \sum_{j = 1}^n \abs{v_j}
    \end{align*}
    für $v = (v_1 , \dots , v_n) \in \K^n$. Allgemeiner kann man für eine reelle Zahl $p \geq 1$
    folgende Norm definieren
    \begin{align*}
        \Norm{v}_p = \klammer{\sum_{i=1}^n \abs{v_i}}^{\frac{1}{p}}
    \end{align*}
    Man nennt diese \fat{p-Norm}. Auf dem Vektorraum der stetigen Funktionen $f: [a,b] \rightarrow \R$
    lässt sich folgende Norm definieren für ein $f \in V$.
    \begin{align*}
        \Norm{f}_1 = \intab \abs{f(x)} dx
    \end{align*}
    Diese Norm wird \fat{$L^1$-Norm} genannt. Allgemeiner kann man für reelle Zahlen $p \geq 1$
    folgende Norm definieren
    \begin{align*}
        \Norm{f}_p = \klammer{\intab \abs{f(x)}^p \ dx}^{\frac{1}{p}}
    \end{align*}
    Eine solche Norm nennt man \fat{p-Norm} oder \fat{$L^p$-Norm}. Das $L$ steht für Lebesgue.
    Die Maximumsnorm auf dem Vektorraum der stetigen Funktionen ist wie folgt definiert
    für ein $f \in V$:
    \begin{align*}
        \Norm{f}_{\infty} = \maximum \geschwungeneklammer{\abs{f(x)} \ | \ x \in [a,b]}
    \end{align*}
    Diese nennt man auch \fat{$\infty$-Norm}, \fat{$L^{\infty}$-Norm} oder auch \fat{sup-Norm}.

    \vspace{1\baselineskip}

    Die \fat{Operatornorm} einer Matrix $A \in \text{Mat}_{m,n} (\R)$ ist durch
    \begin{align*}
        \Norm{A}_{\text{op}} = \sup \geschwungeneklammer{\Norm{Ax}_2 \ | \ x \in \R^n \text{ mit } \Norm{x}_2 \leq 1}
    \end{align*}
    definiert, wobei $\standardNorm_2$ für die Euklidische Norm steht. Des Weiteren gilt
    \begin{align*}
        \Norm{Ax}_2 \leq \Norm{A}_{\text{op}} \Norm{x}_2
        \quad \quad \text{   und   } \quad \quad
        \Norm{AB}_{\text{op}} \leq \Norm{A}_{\text{op}} \Norm{B}_{\text{op}}
    \end{align*}
    für alle $x \in \R^n$ und alle $A,B \in$Mat$_{m,n} (\R)$.
}

\vspace{1\baselineskip}

\Definition{

    Sei $V$ ein Vektorraum über $\K$ und seien $\Norm{ \ \cdot \ }_1$ und $\Norm{ \ \cdot \ }_2$
    zwei Normen auch $V$. Wir nennen $\Norm{ \ \cdot \ }_1$ und $\Norm{ \ \cdot \ }_2$
    \fat{äquivalent}, falls Konstanten $A > 0$ und $B>0$ existieren mit
    \begin{align*}
        \Norm{v}_1 \leq A \cdot \Norm{v}_2
        \quad \text{   und   } \quad
        \Norm{v}_2 \leq B \cdot \Norm{v}_1
        \ \ \ \ \ \ \forall v \in V
    \end{align*} 
}

\vspace{1\baselineskip}

\Lemma{

    Sei $V$ ein Vektorraum über $\R$ und $\Norm{ \ \cdot \ }$ eine Norm auf $V$. Dann ist
    \begin{align*}
        d: V \times V \rightarrow \R
        \ \ \ \ \ \ \ \ \ \ \ \
        d(v,w) = \Norm{v - w}
    \end{align*}
    eine Metrik auf $V$.
}

\vspace{1\baselineskip}

\Definition{

    Wir bezeichnen die im obigen Lemma gegebene Metrik die von der Norm $\Norm{ \ \cdot \ }$
    \fat{induzierte Metrik} auf $V$.
}

\pagebreak

\Satz{

    Sei $V$ ein $\K$ Vektorraum, seien $\Norm{ \ \cdot \ }_1$ und $\Norm{ \ \cdot \ }_2$
    Normen auf $V$. Dann sind folgende Aussagen äquivalent:
    
    (1) $\Norm{ \ \cdot \ }_1$ und $\Norm{ \ \cdot \ }_2$ sind äquivalent.

    (2) Die Identitätsabbildungen $V_{\einsNorm} \stackrel{id}{\longrightarrow} V_{\zweiNorm}$
        und $V_{\zweiNorm} \stackrel{id}{\longrightarrow} V_{\einsNorm}$ sind stetig.
    
    (3) Eine Teilmenge $U \subseteq V$ ist offen bezüglich der von $\einsNorm$ induzierten Metrik,
        genau dann, wenn sie offen bezüglich der von $\zweiNorm$ induzierten Metrik ist.

    (4) Eine Folge $(v_n)_{n=0}^{\infty}$ in $V$ konvergiert nach $w \in V$ bezüglich der
        $\einsNorm$ genau dann, wenn $(v_n)_{n=0}^{\infty}$ gegen $w$ bezüglich der
        $\zweiNorm$ konvergiert.
}

\vspace{1\baselineskip}

\Definition{

    Sei $\K$ ein Körper und $V$ ein Vektorraum über $\K$. Ein \fat{inneres Produkt} oder
    \fat{Skalarprodukt} auf $V$ ist eine Abbildung
    \begin{align*}
        \scalprod{-}{-} : V \times V \rightarrow \K
    \end{align*}
    die folgende Eigenschaften erfüllt

    (1) (Sesquilinearität) Für alle $u,v,w \in V$ und $\alpha , \beta \in \K$ gilt
    \begin{align*}
        &\scalprod{\alpha u + \beta v}{w} = \alpha \scalprod{u}{w} + \beta \scalprod{v}{w}
        \quad \text{   und} \\
        &\scalprod{u}{\alpha v + \beta w} = \overline{\alpha} \scalprod{u}{v} + \overline{\beta} \scalprod{u}{w}
    \end{align*}
    (2) (Symmetrie) Für alle $v,w \in V$ gilt $\scalprod{v}{w} = \overline{\scalprod{v}{w}}$.

    (3) ((positiv)Definitheit) Für alle \vinV ist $\scalprod{v}{v}$ reell und nichtnegativ,
        und $\scalprod{v}{v} = 0 \Leftrightarrow v = 0$.

    Das $\overline{.}$ steht für die komplexe Konjugation.
}

\vspace{1\baselineskip}

\Satz{ (Cauchy-Schwarz Ungleichung)

    Sei $V$ ein Vektorraum über $\K$, sei $\scalprod{-}{-}$ ein Skalarprodukt auf $V$
    und sei $\standardNorm: V \rightarrow \R$ gegeben durch $\Norm{v} = \sqrt{\scalprod{v}{v}}$.
    Dann gilt die Ungleichung
    \begin{align*}
        \abs{\scalprod{v}{w}} \leq \Norm{v} \Norm{w}
    \end{align*}
    für alle $v,w \in V$. Des weiteren gilt Gleichheit genau dann, wenn $v$ und $w$ linear
    unabhängig sind.
}

\vspace{1\baselineskip}

\Proposition{

    Sei $V$ ein $\K$ Vekorraum, $\scalprod{-}{-}$ ein Skalarprodukt auf $V$. Dann definiert
    $v \rightmapsto \Norm{v} := \sqrt{\scalprod{v}{v}}$ eine Norm auf $V$. Diese Norm wird
    \fat{2-Norm} oder \fat{Euklidische Norm} genannt.
}

\vspace{1\baselineskip}

\Lemma{

    Sei $V$ ein $\K$-Vektorraum und sei $\standardNorm$ die Norm auf $V$. Sei $(v_n)_{n=0}^{\infty}$
    eine bezüglich der Norm $\standardNorm$ konvergierende Folge in $V$ mit Grenzwert $w \in V$. Dann
    gilt \begin{align*}
        \limesninf \Norm{v_n} = \Norm{w}
    \end{align*}
}

\vspace{1\baselineskip}

\Lemma{

    Sei $V$ ein $\K$-Vektorraum und seien $\einsNorm$ und $\zweiNorm$ zwei äquivalente Normen
    auf $V$. Sei $(v_n)_{n=0}^{\infty}$ eine Folge in $V$. Falls die Folge $(v_n)_{n=0}^{\infty}$
    bezüglich der Norm $\einsNorm$ konvergiert, dann konvergiert sie auch bezüglich der Norm
    $\zweiNorm$. Die Grenzwerte sind in diesem Fall gleich.
}

\vspace{1\baselineskip}

\Satz{

    Sei $V$ ein endlichdimensionaler $\K$-Vekorraum. Alle Normen auf $V$ sind zueinander
    äquivalent.
}

\vspace{1\baselineskip}

\Lemma{

    Sei $d \in \N$. Eine Folge in $\K^d$ konvergiert genau dann bezüglich der euklidischen
    Norm, wenn sie koordinatenweise konvergiert.
}

\vspace{1\baselineskip}

\Satz{ (Heine-Borell)

    Sei $d \in \N$. Jede bezüglich der euklidischen Norm beschränkte Folge in $\K^d$ besitzt
    eine konvergierende Teilfolge, und einen Häufungspunkt.
}

\vspace{1\baselineskip}

\Korollar{

    Sei $n \in \N$. Bezüglich der euklidischen Norm ist jede Cauchy-Folge in $\K^n$ konvergent.
}
