\subsection{Absolute Konvergenz}

\vspace{1\baselineskip}

Die Definition von absoluter Konvergenz, so wie die erste Proposition, sind im vorherigen
Kapitel zu finden.

\vspace{2\baselineskip}

\Korollar{ (Majorantenkriterium von Weierstrass)

    Sei $\aFolge$ eine komplexe und $(b_n)_n$ eine reelle Folge mit $\abs{a_n} \leq b_n$ für
    alle hinreichend grossen $n \in \N$. Falls $\sum_{n=1}^{\infty} b_n$ konvergiert,
    dann ist $\sum_{n=1}^{\infty} a_n$ absolut konvergent, und daher auch konvergent. 
}

\vspace{1\baselineskip}

\Korollar{ (Cauchy-Wurzelkriterium)

    Sei $\aFolge$ eine Folge komplexer Zahlen und
    \begin{align*}
        \alpha = \limsupninf \sqrt[n]{\abs{a_n}} \in \R \cup \geschwungeneklammer{\infty}
    \end{align*}
    ("Grösster Häufungspunkt"). Dann gilt:

    Falls $\alpha < 1$, so konvergiert $\sum_{n=1}^{\infty} a_n$ absolut.

    Falls $\alpha > 1$, so divergiert $\sum_{n=1}^{\infty} a_n$.
    
}

\vspace{1\baselineskip}

\Korollar{ (D'Alemberts Quotientenkriterium)

    Sei $\aFolge$ eine Folge komplexer Zahlen mit $a_n \neq 0$ für alle n $n \in \N$,
    so dass
    \begin{align*}
        \alpha = \limesninf \frac{\abs{a_{n+1}}}{\abs{a_n}}
    \end{align*}
    existiert. Dann gilt:
    
    Falls $\alpha < 1$, dann konvergiert $\sum_{n=1}^{\infty} a_n$ absolut.

    Falls $\alpha > 1$, dann divergiert $\sum_{n=1}^{\infty} a_n$.
}

\vspace{1\baselineskip}

\Satz{ (Umordnungssatz für absolut konvergente Reihen)

    Sei $\sum_{n=0}^{\infty} a_n$ eine absolut konvergente Reihe komplexer Zahlen.
    Sei $\varphi: \N \rightarrow \N$ eine Bijektion. Dann ist die Reihe
    $\sum_{n=0}^{\infty} a_{\varphi (n)}$ absolut konvergent und es gilt
    \begin{align*}
        \sum_{n=0}^{\infty} a_n = \sum_{n=0}^{\infty} a_{\varphi (n)}
    \end{align*}
}

\vspace{1\baselineskip}

\Satz{ (Produktsatz)

    Seien $\sum_{n=0}^{\infty} a_n$ und $\sum_{n=0}^{\infty} b_n$ absolut konvergente Reihen,
    und sei eine Bijektion $\alpha : \N \rightarrow \N \times \N$ durch
    $\alpha (n) = (\varphi (n) , \psi (n))$ gegeben. Dann gilt
    \begin{align*}
        \klammer{\sum_{n=0}^{\infty} a_n} \klammer{\sum_{n=0}^{\infty} b_n}
        = \sum_{n=0}^{\infty} a_{\varphi (n)} b_{\psi (n)}
    \end{align*}
    und die Reihe auf der rechten Seite konvergiert absolut.
}

\vspace{1\baselineskip}

\Korollar{ (Cauchy-Produkt)

    Falls $\sum_{n=0}^{\infty} a_n$ und $\sum_{n=0}^{\infty} b_n$ absolut konvergente
    Reihen mit komplexen Gliedern sind, dann gilt:
    \begin{align*}
        \sum_{n=0}^{\infty} \klammer{\sum_{k=0}^{n} a_{n-k} b_k}
        = \klammer{\sum_{n=0}^{\infty} a_n} \klammer{\sum_{n=0}^{\infty} b_n}
    \end{align*}
    wobei die Reihe $\sum_{n=0}^{\infty} \klammer{\sum_{k=0}^{n} a_{n-k} b_k}$
    absolut konvergent ist.
}
