\subsection{Konvergenz von Funktonenfolgen}

\vspace{1\baselineskip}

\Definition{

    Sei $D$ eine Menge, sei $\fFolge$ eine Folge von Funktionen $f_n : D \rightarrow \C$,
    und sei $f: D \rightarrow \C$ eine Funktion. Wir sagen die Folge $\fFolge$ konvergiert
    \fat{punktweise} gegen $f$, falls für jedes $x \in D$ die Folge komplexer Zahlen
    $(f_n (x))_{n=0}^{\infty}$ gegen $f(x)$ konvergiert. Wir bezeichnen die Funktion
    $f$ in dem Fall als den \fat{punktweisen Grenzwert} der Folge $\fFolge$.
}

\vspace{1\baselineskip}

\Definition{

    Sei $D$ eine Menge, sei $\fFolge$ eine Folge von Funktionen $f_n : D \rightarrow \C$,
    und sei $f: D \rightarrow \C$ eine Funktion. Wir sagen die Folge $\fFolge$ konvergiert
    \fat{gleichmässig} gegen $f$, falls für jedes $\epsilon > 0$ ein \NinN existiert, so
    dass für alle $n \geq N$ und alle $x \in D$ die folgende Abschätzung gilt.
    \begin{align*}
        \abs{f_n (x) - f(x)} < \epsilon
    \end{align*}
}

\vspace{1\baselineskip}

\Satz{

    Sei $D \subset \C$ und sei $\fFolge$ eine Folge stetiger Funktionen
    $f_n : D \rightarrow \C$ die gleichmässig gegen $f: D \rightarrow \C$ konvergiert.
    Dann ist $f$ stetig.
}

\vspace{1\baselineskip}

\Satz{

    Sei $[a,b]$ ein Intervall und sei $(f_n : [a,b] \rightarrow \C)_{n=0}^\infty$ eine
    Folge integrierbarer Funktionen die gleichmässig gegen eine Funktion
    $f: [a,b] \rightarrow \C$ konvergiert. Dann ist $f$ integrierbar und es gilt:
    \begin{align*}
        \intab f \ dx \ = \ \limesninf \intab f_n \ dx
    \end{align*}
}
