\subsection{Potenzreihen}

\vspace{1\baselineskip}

\Definition{

    Sei $K$ ein Körper. Eine \fat{Potenzreihe} mit Koeffizienten in $\K$ ist eine Folge
    $\aFolge$ in $\K$, suggestiv geschrieben als Reihe
    \begin{align*}
        f(T) = \sum_{n=0}^{\infty} a_n T^n
    \end{align*}
    wobei $T$ als \fat{Variable} bezeichnet wird und $a_n \in \K$ als \fat{Koeffizient}.
    Addition und Multiplikation sind wie folgt definiert.
    \begin{align*}
        \sum_{n=0}^{\infty} a_n T^n + \sum_{n=0}^{\infty} b_n T^n
        &= \sum_{n=0} (a_n + b_n) T^n
        \\
        \klammer{\sum_{n=0}^{\infty} a_n T^n} \klammer{\sum_{n=0}^{\infty} b_n T^n}
        &= \sum_{n=0}^{\infty} \klammer{\sum_{k=0}^{n} a_{n-k} b_k} T^n
    \end{align*}
    Wir schreiben $K$\textlbrackdbl$T$\textrbrackdbl \ für den dadurch entstehenden Ring
    von Potenzreihen mit Koeffizienten in $\K$.
}

\vspace{1\baselineskip}

\Definition{

    Sei $f(T) = \sum_{n=0}^{\infty} a_n T^n \in \C $\textlbrackdbl$ T $\textrbrackdbl \
    eine formale Potenzreihe mit komplexen Koeffizienten. Der \fat{Konverenzradius} von $f$
    ist die Zahl $R \in \R_{\geq 0}$ oder das Symbol $R = \infty$, definiert duch
    \begin{align*}
        \rho = \limsupninf \sqrt[n]{\abs{a_n}}
        \quad \quad \text{    und    } \quad \quad
        R = \begin{cases}
            0 \ \ \text{   falls } \rho = \infty
            \\
            \rho^{-1} \ \ \text{   falls } 0 < \rho < \infty
            \\
            \infty \ \ \text{   falls } \rho = 0
        \end{cases}
    \end{align*}
}

\vspace{1\baselineskip}

\Bemerkung{

    Sei $\sum_{n=0}^{\infty} a_n T^n \in \C$\textlbrackdbl$T$\textrbrackdbl eine Potenzreihe
    mit positivem Konvergenzradius $R$, und sei $r$ eine positive reelle Zahl mit $r < R$.
    Schreibe $D = \overline{B(0,r)}$ und $f_n : D \rightarrow \C$ für die Funktion gegeben
    durch $f_n(z) = \sum_{k=0}^n a_n z^k$.

    (1) Die Reihe $\sum_{n=0}^{\infty} a_n z^n$ konvergiert absolut für alle \zinC
    mit $\abs{z} < R$, und divergiert für alle \zinC mit $\abs{z} > R$.

    (2) Für $z \in B(0,R)$, setze $f(z) = \sum_{n=0}^{\infty} a_n z^n$. Die Folge von
    Funktionen $\fFolge$ konvergiert gleichmässig gegen die Funktion $f|_D$ auf $D$.

    Insbesondere definiert die Potenzreihe die stetige Abbildung $f: B(0,R) \rightarrow \C$.
}

\vspace{1\baselineskip}

\Lemma{

    Seien $D \subseteq \R$, $f_n : D \rightarrow \R$ für alle $n=0,1,2,\dots$
    (Anstatt $\R$ kann man auch $\C$ nehmen) und $f: D \rightarrow \R$.
    Angenommen $f_n$ ist stetig für alle $n \geq 0$ und $\forall \epsilonnull \ \exists N \in \N$
    mit $\Norm{f_n - f}_{\infty} < \epsilonnull \ \forall n \geq N$. (i.e.
    $\limesninf f_n = f$ bezüglich der Norm $\standardNorm_{\infty}$)
    Dann ist $f$ stetig.
}

\vspace{1\baselineskip}

\Lemma{

    Sei $\sum_{n=0}^{\infty} a_n T^n$ eine Potenzreihe mit $a_n \neq 0$ für alle $n \in \N$.
    Der Konvergenzradius $R$ ist gegeben durch
    \begin{align*}
        R = \limesninf \frac{a_n}{a_{n+1}}
    \end{align*}
    falls dieser Grenzwert existiert.
}

\vspace{1\baselineskip}

\Proposition{

    Sei $R \geq 0$, und seien $f(T) = \sum_{n=0}^{\infty}$ und $g(T) = \sum_{n=0}^{\infty} b_n T^n$
    Potenzreihen mit Kovergenzradius mindestens $R$. Dann haben die Summe $f(T) + g(T)$
    und das Produkt $f(T) g(T)$ Konverenzradien mindestens $R$.
}

\vspace{1\baselineskip}

\Lemma{

    Sei $[a,b] \subseteq \R$ ein Intervall mit $a < b$. Sei $\fFolge$ eine Folge
    stetiger Funktionen auf $[a,b]$ und $f: [a,b] \rightarrow \R$ stetig mit
    $\limesninf f_n = f$ bezüglich $\standardNorm_{\infty}$. Dann gilt
    \begin{align*}
        \limesninf \intab f_n \ dx = \intab f \ dx
    \end{align*}
}

\vspace{1\baselineskip}

\Satz{ (Abel'scher Grenzwertsatz)

    Sei $\sum_{n=0}^{\infty} a_n T^n \in \C$\textlbrackdbl$T$\textrbrackdbl eine Potenzreihe
    mit positivem Konvergenzradius $R$, derart, dass die Reihe $\sum_{n=0}^{\infty} a_n R^n$
    konvergiert. Dann gilt
    \begin{align*}
        \limes{\stackrel{t \rightarrow R}{t < R}} \sum_{n=0}^{\infty} a_n t^n = \sum_{n=0}^{\infty} a_n R^n
    \end{align*}
    Anders ausgedrückt: Dann ist die Funktion $f: (-R,R] \rightarrow \C$ gegeben durch
    $f(t) = \sum_{n=0}^{\infty} a_n t^n$ ist stetig bei $R$.
}

\vspace{1\baselineskip}

\Satz{

    Sei $f(T) = \sum_{n=0}^{\infty} a_n T^n \in \C$\textlbrackdbl$T$\textrbrackdbl eine
    Potenzreihe mit Konverenzradius $R>0$. Dann hat die Potenzreihe
    \begin{align*}
        F(T) = \sum_{n=0}^{\infty} \frac{a_n}{n+1} T^{n+1}
    \end{align*}
    denselben Konverenzradius $R$, und es gilt
    \begin{align*}
        \intab f(x) dx = F(b) - F(a) \ \ \ \ \forall a,b \in (-R,R)
    \end{align*}
}

\vspace{1\baselineskip}
