\subsection{Zentrale Sätze der Differentialrechnung}

\vspace{1\baselineskip}

\Definition{
 
    Sei $D \subseteq \R$ eine Teilmenge und $x_0 \in D$. Wir sagen, dass eine Funktion
    $f: D \rightarrow \R$ in $x_0$ ein \fat{lokales Maximum} hat, falls es ein $\delta >0$ gibt,
    so dass für alle $x \in D \cap (x_0 - \delta , x_0 + \delta)$ gilt $f(x) \leq f(x_0)$.
    Falls es sich um eine strikte Ungleichung handelt, also $f(x) < f(x_0)$, dann hat $f$ in
    $x_0$ ein \fat{isoliertes lokales Maximum}. Der Wert $f(x_0)$ wird auch ein \fat{lokaler
    Maximalwert} von $f$ genannt. Ein \fat{lokales Minimum}, ein \fat{isoliertes lokales Minimum}
    und ein \fat{lokaler Minimalwert} von $f$ sind analog definiert. Des Weiteren nennen wir
    $x_0$ ein \fat{lokales Extremum} von $f$ und $f(x_0)$ einen \fat{lokalen Extremwert} von $f$,
    falls $f$ ein lokales Minimum oder ein lokales Maximum in $x_0$ hat.
}

\vspace{1\baselineskip}

\Proposition{

    Sei $D \subseteq \R$ eine Teilmenge und $f$ eine reellwertige Funktion auf $D$.
    Angenommen $f$ ist in einem lokalen Extremum $x_0 \in D$ differenzierbar und $x_0$
    ist sowohl ein rechtsseitiger als auch ein linksseitiger Häufungspunkt von $D$.
    Dann gilt $f'(x_0) = 0$.
}

\vspace{1\baselineskip}

\Korollar{

    Sei $I \subseteq \R$ ein Intervall und $f: I \rightarrow \R$ eine Funktion. Sei $x_0 \in I$
    ein lokales Extremum von $f$. Dann ist mindestens eine der folgenden Aussagen wahr.

    (1) $x_0$ ist ein Randpunkt von $I$.

    (2) $f$ ist nicht ableitbar bei $x_0$.

    (3) $f$ ist bei $x_0$ differenzierbar und $f'(x_0) = 0$. 

    Insbesondere sind alle lokalen Extrema einer differenzierbaren Funktion auf einem offenen
    Intervall Nullstellen der Ableitung.
}

\vspace{1\baselineskip}

\Satz{ (Mittelwertsatz, Rolle)

    Seien $a<b \in \R$ und $f: [a,b] \rightarrow \R$ eine stetige Funktion, die auf dem offenen
    Intervall $(a,b)$ differenzierbar ist. Falls $f(a) = f(b)$ gilt, so existert ein
    $\xi \in (a,b)$ mit $f'(\xi) = 0$.
}

\vspace{1\baselineskip}

\Satz{ (Mittelwertsatz)

    Seien $a<b \in \R$ und $f: [a,b] \rightarrow \R$ eine stetige Funktion, die auf dem offenen
    Intervall $(a,b)$ differenzierbar ist. Dann gibt es ein $\xi \in (a,b)$ mit
    \begin{align*}
        f'(\xi) = \frac{f(b)-f(a)}{b-a}
    \end{align*}
}

\vspace{1\baselineskip}

\Satz{ (Mittelwertsatz nach Cauchy)

    Seien $f$ und $g$ stetige Runktionen auf einem Intervall $[a,b]$ mit $a<b$, so dass $f$
    und $g$ auf $(a,b)$ differenzierbar sind. Dann existert ein $\xi \in (a,b)$ mit
    \begin{align*}
        g'(\xi) \klammer{f(b) - f(a)} = f'(\xi) \klammer{g(b) - g(a)}
        \ \Leftrightarrow \
        \frac{g'(\xi)}{f'(\xi)} = \frac{g(b) - g(a)}{f(b) - f(a)}
    \end{align*}
}

\vspace{1\baselineskip}

\Proposition{

    Sei $I \subseteq \R$ ein Intervall das nicht leer ist und nicht aus einem einzigen Punkt
    besteht. Sei $f: I \rightarrow \R$ eine differenzierbare Funktion. Dann gilt
    \begin{align*}
        f' \geq 0 \ \Leftrightarrow \ f \text{ ist monoton wachsend}
    \end{align*}
    Die Funktion $f$ ist genau dann streng monoton wachsend, wenn es kein nichtleeres, offenes
    Intervall $J \subseteq I$ gibt mit $f'|_J = 0$. Dies ist äquivalent dazu, dass bei der obigen
    Äquivalenz ein striktes ungleich Zeichen steht.
}

\vspace{1\baselineskip}

\Korollar{

    Sei $I \subseteq \R$ ein Intervall mit Endpunkten $a<b$ und $f: I \rightarrow \R$ eine
    Funktion. Dann ist $f$ genau dann konstant, wenn $f$ differenzierbar ist und $f'(x) = 0$
    für alle $x \in I$ gilt.
}

\vspace{1\baselineskip}

\Definition{

    Sei $I \subseteq \R$ ein Interfall und $f: I \rightarrow \R$ eine Funktion. Dann heisst
    $f$ \fat{konvex}, falls für alle $a<b \in I$ und alle $t \in (0,1)$ die Ungleichung
    \begin{align*}
        f \klammer{(1-t) a + t b} \leq (1-t) f(a) + t f(b)
    \end{align*}
    gilt. Wir sagen, dass $f$ \fat{streng konvex} ist, falls in der obigen Ungleichung
    eine strikte Ungleichung gilt. Eine Funktion $g: I \rightarrow \R$ heisst
    \fat{(streng) konkav}, wenn $f = -g$ (streng) konvex ist.

    Merksatz:
    "Hat die Katze Sex, wird der Bauch konvex."
}

\vspace{1\baselineskip}

\Korollar{

    Eine Funktion $f: I \rightarrow \R$ ist genau dann konvex, wenn für alle $a<x<b \in I$
    die folgende Ungleichung gilt:
    \begin{align*}
        \frac{f(x) - f(a)}{x-a} \leq \frac{f(b) - f(x)}{b-x}
    \end{align*}
}

\vspace{1\baselineskip}

\Proposition{

    Sei $I \subseteq \R$ ein Intervall mit Endpunkten $a<b$ und $f: I \rightarrow \R$ eine
    differenzierbare Funktion. Dann ist $f$ genau dann (streng) konvex, wenn $f'$ (streng)
    monoton wachsend ist.
}

\vspace{1\baselineskip}

\Korollar{

    Sei $I \subseteq \R$ ein Intervall mit Endpunkten $a<b$ und $f: I \rightarrow \R$ eine
    zweimal differenzierbare Funktion. Falls $f''(x) \geq 0$ für alle $x \in I$, dann ist $f$
    konvex. Falls $f''(x) > 0$ für alle $x \in I$, dann ist $f$ streng konvex.
}

\vspace{1\baselineskip}

\Satz{ (Regel von de l'Hôpital)

    Seien $a<b$ reelle Zahlen und $f,g : (a,b) \rightarrow \R$ Funktionen. Angenommen
    die folgenden Hypothesen sind erfüllt.

    (1) Die Funktion $f$ und $g$ sind stetig.

    (2) Es gilt $g(x) \neq 0$ und $g'(x) \neq 0$ für alle $x \in (a,b)$

    (3) Es gilt $\limes{x \rightarrow a} g(x) = 0$ und $\limes{x \rightarrow a} = 0$.

    (4) Der Grenzwert $A = \limes{x \rightarrow a} \frac{f'(x)}{g'(x)}$ existert.

    Dann existert auch der Grenzwert $\limes{x \rightarrow a} \frac{f(x)}{g(x)}$ und es gilt
    $\limes{x \rightarrow a} \frac{f(x)}{g(x)} = A$. 
}
