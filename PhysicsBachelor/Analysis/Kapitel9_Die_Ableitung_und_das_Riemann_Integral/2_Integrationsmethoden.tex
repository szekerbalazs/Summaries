\subsection{Integrationsmethoden}

\vspace{1\baselineskip}

Wir legen für dieses Kapitel ein nichtleeres Intervall $I \subseteq \R$ fest, das nicht nur aus
einem isolierten Punkt besteht. Falls nicht ausdrücklich anders erwähnt, so sind alle
Funktionen in diesem Kapitel reellwertige Funktionen mit Definitionsbereich $I$, die auf jedem
kompakten Intervall $[a,b] \subseteq I$ integrierbar sind. Alle Resultate gelten analog für
komplexwertige Funktionen.

\vspace{1\baselineskip}

\Definition{

    Sei $I \subseteq \R$ ein Intervall und $f: I \rightarrow \R$ eine Funktion.
    Dann nennen wir folgendes Integral das \fat{unbestimmte Integral} mit
    \fat{Integrationskonstante} $C$.
    \begin{align*}
        \int f(x) dx = F(x) + C
    \end{align*}    
}

\vspace{1\baselineskip}

\Bemerkung{

    Seien $f$ und $g$ Funktionen, $a$ und $b$ reelle Zahlen. Aus der Linearität der Ableitung folgt
    \begin{align*}
        \int a f(x) + b g(x) \ dx = a \int f(x) dx \ + \ b \int g(x) dx \ + C
    \end{align*}
}

\vspace{1\baselineskip}

\Definition{ (Partielle Integration)

    Seien $f$ und $g$ Funktionen mit Stammfunktion $F$, bzw $G$. Als \fat{partielle Integration}
    bezeichnet man die folgende Identität
    \begin{align*}
        \int F(x) g(x) dx = F(x) G(x) - \int f(x) G(x) dx \ + C
    \end{align*}
}

\vspace{1\baselineskip}

\Definition{ (Substitutionsmethode)

    Sei $J \subseteq \R$ ein weiteres Intervall und sei $f: I \rightarrow J$ eine differenzierbare
    Funktion. Ist $G: J \rightarrow \R$ differenzierbar mit Ableitung $g = G'$, so gilt nach
    der Kettenregel $G(f(x))' = g(f(x))f'(x)$ für alle $x \in I$. Daraus folgt
    \begin{align*}
        \int (g \circ f) (x) f'(x) dx = G(f(x)) + C
    \end{align*}
    Die Funktion $G$ ist eine Stammfunktion von $g$, es gilt also $G(u) = \int g(u) du$.
    Als \fat{Substitutionsregel} bezeichnen wir die Identität
    \begin{align*}
        \int (g \circ f) (x) f'(x) dx = g(u) du + C
    \end{align*}
    wobei $u = f(x)$. Die Substitutionsregel wird auch \fat{Variablenwechsel} genannt, da man
    sozusagen die Variable $u$ in $\int g(u) du$ durch $u=f(x)$ ersetzt hat.
}

\vspace{1\baselineskip}

\Bemerkung{

    Einige wichtige Integrationen \fat{rationaler Funktionen}. Sei dazu $a \in \R$ und $a \neq I$
    und sei $n \geq 2 \in \N$. Dann gilt:
    \begin{align*}
        \int \frac{1}{x-a} dx &= \log \abs{x-a} + C
        \\
        \int \frac{1}{(x-a)^n} dx &= \frac{(x-a)^{1-n}}{1-n} +C
        \\
        \int \frac{1}{a^2 + x^2} dx &= \frac{\arctan \klammer{\frac{x}{a}}}{a} + C
        \\
        \int \frac{x}{a^2 + x^2} dx &= \frac{\log(a^2 + x^2)}{2} + C
        \\
        \int \frac{x}{(a^2 + x^2)^n} dx &= \frac{(a^2 + x^2)^{1-n}}{2 (1-n)} + C 
    \end{align*}
}

\vspace{1\baselineskip}

\Definition{

    Sei $I \subseteq \R$ ein nichtleeres Intervall, und $f: I \rightarrow \R$ eine lokal integrierbare
    Funktion. Setze $a = \inf (I)$ und $b = \sup (I)$ und wähle $x_0 \in I$. Wir definieren das
    \fat{uneigentliche Integral} als die Summe von Grenzwerten
    \begin{align*}
        \intab f(x) dx = \limes{x \rightarrow a} \int_x^{x_0} f(t) dt + \limes{x \rightarrow b} \int_{x_0}^x f(t) dt
    \end{align*}
    falls beide dieser Grenzwerte existieren. In dem Fall sagen wir auch, dass das uneigentliche
    Integral \fat{konvergiert}. Ansonsten nennen wir das uneigentliche Integral \fat{divergent}.
    Wir benutzen die üblichen Konventionen für die Symbole $- \infty$ und $+ \infty$.
}

\vspace{1\baselineskip}

\Lemma{

    Sei $a \in \R$ und $f: [a,\infty) \rightarrow \R_{\geq 0}$ eine nicht-negative, lokal
    integrierbare Funktion. Dann gilt
    \begin{align*}
        \int_a^{\infty} f(x) dx = \sup \geschwungeneklammer{\intab f(x) dx \ | \ b>a}
    \end{align*} 
}

\vspace{1\baselineskip}

\Satz{ (Integralsatz für Reihen)

    Sei $f: [0, \infty) \rightarrow \R_{\geq 0}$ eine monoton fallende Funktion. Dann gilt
    \begin{align*}
        \sum_{n=1}^{\infty} f(n) \geq \int_0^{\infty} f(x) dx \geq \sum_{n=0}^{\infty} f(n)
    \end{align*}
    Insbesondere gilt die folgende Äquivalenz
    \begin{align*}
        \sum_{n=0}^{\infty} f(n) \text{   konvergiert} \Longleftrightarrow \int_0^{\infty} f(x) dx \text{   konvergiert}
    \end{align*}
}
