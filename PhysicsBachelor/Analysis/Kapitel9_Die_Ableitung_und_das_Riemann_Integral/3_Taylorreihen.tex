\subsection{Taylorreihen}

\vspace{1\baselineskip}

\Definition{

    Sei $D \subseteq \R$ ein offenes Intervall, und $f: D \rightarrow \R$ eine $n$-mal
    differenzierbare Funktion. Die $n$-te \fat{Taylor-Approximation} von $f$ um einen
    Punkt $x_0 \in D$ ist die Polynomfunktion
    \begin{align*}
        P_n(x) = \sum_{k=0}^{n} \frac{f^{(k)}(x_0)}{k!} (x-x_0)^k
    \end{align*}
    Die Koeffizienten wurden dabei gerade so gewählt, dass $P^{(k)} (x_0) = f^{(k)} (x_0)$
    für $k \in \geschwungeneklammer{0, \dots , n}$ gilt. Falls $f$ glatt ist, dann ist
    die \fat{Taylorreihe} von $f$ um $x_0 \in D$ definiert als die Potenzreihe
    \begin{align*}
        L(T) = \sum_{k=0}^{\infty} \frac{f^{(k)} (x_0)}{k!} T^k
    \end{align*}
    Ist $R$ der Konvergenzradius dieser Reihe, so konvergiert die Reihe der reellen
    Zahlen
    \begin{align*}
        \sum_{k=0}^{\infty} \frac{f^{(k)} (x_0)}{k!} (x-x_0)^k
    \end{align*}
    für alle $x \in \R$ mit Abstand kleiner $R$ von $x_0$ und divergiert für alle
    $x \in \R$ mit $\abs{x-x_0} > R$. Oft nennt man diese Summe auch \fat{Taylorreihe}.
}

\pagebreak

\Satz{ (Taylor-Approximation)

    Sei $D \subseteq \R$ ein offenes Intervall und $f: D \rightarrow \R$ (oder auch
    $D \rightarrow \C$) eine $(n+1)$-mal stetig differenzierbare Funktion. Dann gilt
    für alle $x \in D$
    \begin{align*}
        f(x) = P_n (x) + \int_{x_0}^x f^{(n+1)} (t) \frac{(x-t)^n}{n!} dt
    \end{align*}
    wobei $P_n$ die $n$-te Taylor-Approximation von $f$ ist.
}

\vspace{1\baselineskip}

\Korollar{ (Taylor-Abschätzung)

    Sei $D \subseteq \R$ ein offenes Intervall, $f: D \rightarrow \R$
    (oder auch $D \rightarrow \C$) eine $(n+1)$-mal stetig differenzierbare Funktion,
    und sei $\delta > 0$ und $x_0 \in D$ und setze
    $M = \sup \geschwungeneklammer{\abs{f^{(n+1)} (t) } \ | \ t \in [x_0 - \delta , x_0 + \delta]}$.
    Dann gilt
    \begin{align*}
        \abs{f(x) - P_n (x)} \leq \frac{M \cdot \abs{x - x_0}^{n+1}}{(n+1)!}
    \end{align*}
    Insbesondere ist $f(x) - P_n (x) = O((x-x_0)^{n+1})$ für $x \longrightarrow x_0$.
}

\vspace{1\baselineskip}

\Definition{

    Sei $I \subseteq \R$ ein Intervall, und $x_0 \in I$ ein Punkt im Inneren von $I$.
    Eine glatte Funktion $f: I \rightarrow \R$ heisst \fat{analytisch} bei $x_0$ falls
    ein $\delta > 0$ existiert, derart, dass die Taylorreihe von $f$ um $x_0$ einen
    Konvergenzradius $R > \delta$ hat, und
    \begin{align*}
        f(x) = \sum_{n=0}^{\infty} \frac{f^{(n)} (x_0)}{n!} (x-x_0)^n
    \end{align*}
    für alle $x \in (x_0 - \delta , x_0 + \delta) \cap I$ gilt. Wir sagen $f$ sei
    analysisch auf $I$ falls $f$ analysisch in jedem Punkt von $I$ ist.
}

\vspace{1\baselineskip}

\Bemerkung{

    Analytische Funktionen $f: I \rightarrow \R$ sind also dadurch charakterisiert,
    dass es zu jedem Punkt $x_0$ im Inneren von $I$ eine Potenzreihe gibt, die in einer
    Umgebung von $x_0$ gegen $f$ konvergiert. Eine Abschätzung die garantiert, dass
    die Taylorreihe von $f$ im Punkt $x_0$ gegen $f$ konvergiert, ist, dass für ein
    $\delta > 0$
    \begin{align*}
        \abs{x-x_0} < \delta \Longrightarrow \abs{f^{(n+1)} (x)} \leq c A^n
    \end{align*}
    für alle \ninN und zwei Konstanten $c,A \geq 1$ gilt. Eine Abschätzung dieser Art
    findet man beispielsweise für $\exp , \sin , \cos$ und Kombinationen davon.
}

\vspace{1\baselineskip}

\Satz{

    Seien $a<b$ reelle Zahlen, $f: [a,b] \rightarrow \R$ eine stetige Funktion,
    $n \in \N$ und $x_k = a + k \frac{b-a}{n}$ für $k \in \geschwungeneklammer{0, \dots , n}$.

    \begin{enumerate}[{(1)}]
        \item (Rechteckregel) Falls $f$ stetig differenzierbar ist, dann gilt
                \begin{align*}
                    \intab f(x) dx = \frac{b-a}{n} (f(x_0) + \dots + f(x_{n-1})) + F_1
                \end{align*}
                wobei der Fehler $F_1$ durch
                $\abs{F_1} \leq \frac{(b-a)^2}{2n} \max \geschwungeneklammer{\abs{f'(x)} \ | \ x \in [a,b]}$
                beschränkt ist.
        \item (Sehnentrapezregel) Falls $f$ zweimal stetig differenzierbar ist, dann
                gilt
                \begin{align*}
                    \intab f(x) dx = \frac{b-a}{2n} (f(x_0) + 2 f(x_1) + \dots + 2 f(x_{n-1}) + f(x_n)) + F_2
                \end{align*}
                wobei der Feher $F_2$ durch
                $\abs{F_2} \leq \frac{(b-a)^3}{6n^2} \max \geschwungeneklammer{\abs{f''(x)} \ | \ x \in [a,b]}$
                beschränkt ist.
        \item (Simpson-Regel) Falls $f$ viermal stetig differenzierbar ist und $n$
                gerade ist, dann gilt
                \begin{align*}
                    \intab f(x) dx = \frac{b-a}{3n} \klammer{f(x_0) + 4 f(x_1) + 2 f(x_2) + 4 f(x_3) + f(x_4) + \dots + 2 f(x_{n-2}) + 4 f(x_{n-1}) + f(x_n)} + F_3
                \end{align*}
                wobei der Fehler $F_3$ durch $\abs{F_3} \leq \frac{(b-a)^5}{45n^4} \max \geschwungeneklammer{\abs{f^{(4)}} \ | \ x \in [a,b]}$
                beschränkt ist.
    \end{enumerate}
}
