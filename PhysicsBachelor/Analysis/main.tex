\documentclass[a4paper,leqno]{article}
% ==== Inputs and Usepackages ====

\usepackage{tablefootnote}
\usepackage{enumerate}
\usepackage{float}
\usepackage{url}
\usepackage{hyperref}
\usepackage{dsfont}
\usepackage{mathrsfs}
\usepackage{amsmath}
\usepackage{amssymb}
\usepackage{amsthm}
\usepackage{amsfonts}
\usepackage{mathtools}
%\usepackage{mathabx}
\usepackage{MnSymbol}
\usepackage{xfrac}
\usepackage{nicefrac}
\usepackage{geometry}
\usepackage{graphicx}
\usepackage{graphics}
\usepackage{latexsym}
\usepackage{setspace}
\usepackage{tikz-cd}
\usepackage{tikz}
 \usetikzlibrary{matrix}
 \usetikzlibrary{calc}
 \usetikzlibrary{circuits.ee.IEC}
\usepackage{circuitikz}

\usepackage{a4wide}
\usepackage{fancybox}
\usepackage{fancyhdr}
\usepackage[utf8]{inputenc}




% ==== Page Settings ====

\hoffset = -1.2 in
\voffset = -0.3 in
\textwidth = 590pt
\textheight = 770pt
\setlength{\headheight}{20pt}
\setlength{\headwidth}{590pt}
\marginparwidth = 0 pt
\topmargin = -0.75 in
\setlength{\parindent}{0cm}


% ==== Presettings for files ====

\pagestyle{fancy}


\cfoot{\thepage}
\lfoot{\href{mailto:szekerb@student.ethz.ch}{szekerb@student.ethz.ch}}
\rfoot{Balázs Szekér, \today}
\lhead{Physics \uproman{3} Summary}

\title{Analysis \uproman{1} und \uproman{2}}
\author{Zusammenfassung \\ \\ Balázs Szekér \\ \href{mailto:szekerb@student.ethz.ch}{szekerb@student.ethz.ch} \\ \\
Zusammenfassung zur Vorlesung vom Herbstsemester 2019 \\ und 
Frühlingssemester 2020 gegeben von Peter Simon Jossen
\\
\\
Eidgenössische Technische Hochschule Zürich}

\renewcommand*\contentsname{Inhaltsverzeichnis}
\renewcommand{\epsilon}{\varepsilon}
\renewcommand{\div}{\text{div}}
\clubpenalty = 10000
\widowpenalty = 10000

\renewcommand{\abstractname}{Vorwort}
\fancypagestyle{myplain}
{
  \fancyhf{}
  \renewcommand\headrulewidth{0pt}
  \renewcommand\footrulewidth{0pt}
  \fancyfoot[C]{\thepage}
}

\begin{document}  
\begin{titlepage}
    \maketitle
    \thispagestyle{empty}
\end{titlepage}  
\newpage
\pagenumbering{Roman}
\thispagestyle{myplain}


\begin{abstract}
    Dies ist eine Sammlung von Definitionen, Sätzen, etc. aus dem Skript
    \textit{Analysis \uproman{1} und \uproman{2}} von Manfred Einsiedler,
    Peter Jossen und Andreas Wieser. Sie dient als Zusammenfassung
    und Begleitung der Vorlesung \textit{Analysis \uproman{1} und \uproman{2}}, gehalten von
    Peter Simon Jossen im Herbstsemester 2019 und Frühlingssemester 2020 an der
    Eidgenössischen Technischen Hochschule Zürich. Diese Publikation ist weder als Ersatz zum Vorlesungsbetrieb
    zu erachten noch als eine ausreichende Vorbereitung für die Prüfung.
    Alle Beweise wurden weggelassen und sind im Skript \textit{Analysis \uproman{1} und \uproman{2}}
    nachzulesen, welches Sie auf der Vorlesunghomepage
    \footnote{\url{https://metaphor.ethz.ch/x/2020/fs/401-1262-07L/}}
    finden.


    \vspace{1\baselineskip}

    Falls Sie Fehler finden, sei es sprachlich oder thematisch, oder
    falls Sie Verbesserungsvorschläge haben,
    wenden Sie sich an \href{mailto:szekerb@student.ethz.ch}{szekerb@student.ethz.ch}

    \vspace{1\baselineskip}

    Besten Dank

    \vspace{1\baselineskip}

    Balázs Szekér
\end{abstract}



\newpage
\thispagestyle{myplain}
\tableofcontents
\thispagestyle{myplain}
\newpage

\pagenumbering{arabic}

\section{Mengenlehre}

\vspace{1\baselineskip}

\subsection{Naive Mengenlehre}

\vspace{1\baselineskip}

\Definition{

    (1) Eiene \fat{Menge} besteht aus beliebigen unterscheidbaren Elementen.

    (2) Eine Menge ist unverwechselbar dursch ihre \fat{Elemente} bestimmt.

    (3) Eine Menge ist nicht Element ihrer selbst.

    (4) Jede Aussage $A$ über Elemente einer Menge $X$ definiert die 
        Menge der Elemente in $X$ fpr die die Aussage $A$ \hspace*{13pt} wahr ist, 
        notiert $\geschwungeneklammer{x \in X | \text{A gilt für x}}$
}

\vspace{1\baselineskip}


\Definition{

    Mengen von Mengen wird \fat{Familien} von Mengen genannt.
    Seltener braucht man auch die Bezeichnung \fat{Kollektion}
    oder \fat{Ansammlung}.
}

\vspace{1\baselineskip}

\Definition{

    Seien $P$ und $Q$ Mengen. Wir sagen, dass $P$ \fat{Teilmenge} von $Q$
    ist, und schreiben $P \subset Q$, falls für alle $x \in P$ auch 
    $x \in Q$ gilt. Wir sagen, dass P eine \fat{echte Teilmenge} von $Q$
    ist und schreiben $P \subsetneq Q$, falls $P$ eine Teilmenge von $Q$, 
    aber nicht gleich $Q$ ist. Wir schreiben $P \not\subset Q$, falls $P$
    keine Teilmenge von $Q$ ist.
}

\vspace{1\baselineskip}

\Definition{

    Seien $P$ und $Q$ Mengen. Der \fat{Durchschnitt} $P \cap Q$, die
    \fat{Vereinigung} $P \cup Q$, das \fat{relative Komplement} $P \backslash Q$
    und die symmetrische Differenz $P \Delta Q$ sind durch 
    \begin{align*}
        &P \cap Q = \geschwungeneklammer{x \ | \ x \in P \land x \in Q}
        \\
        &P \cup Q = \geschwungeneklammer{x \ | \ x \in P \lor x \in Q}
        \\
        &P \backslash Q = \geschwungeneklammer{x \ | \ x \in P \land x \not\in Q}
        \\
        &P \Delta Q = (P \cup Q) \backslash (P \cap Q) = \geschwungeneklammer{x \ | \ x \in P \text{ XOR } x \in Q}
    \end{align*}
}

\vspace{1\baselineskip}

\Definition{

    Sei $\mathcal{A}$ eine Familie von Mengen, also eine Menge deren Elemente
    selbst Mengen sind. Dann definieren wir die \fat{Vereinigung} als
    \begin{align*}
        \bigcup_{A \in \mathcal{A}} A = \geschwungeneklammer{x \ | \ \exists A \in \mathcal{A}: x \in A}
    \end{align*}
    respektive den \fat{Durchschnitt} der Menge $\mathcal{A}$ als
    \begin{align*}
        \bigcap_{A \in \mathcal{A}} A = \geschwungeneklammer{x \ | \ \forall A \in \mathcal{A}: x \in A}
    \end{align*}
    Falls $\mathcal{A} = \geschwungeneklammer{A_1,A_2, \dots}$, dann 
    schreiben wir auch
    \begin{align*}
        \bigcup_{n=1}^{\infty} A_n = \geschwungeneklammer{x \ | \ \exists n \in \N : x \in A_n}
        \\
        \bigcap_{n=1}^{\infty} A_n = \geschwungeneklammer{x \ | \ \forall n \in \N : x \in A_n}
    \end{align*}
    für die Vereinigung, und den Durchschnitt der Menge in $\mathcal{A}$.
}

\pagebreak

\Definition{

    Zwei Mengen $A,B$ heissen \fat{disjunkt}, falls $A \cap B = \emptyset$ gilt.
    Für eine Kollektion $\mathcal{A}$ von Mengen, sagen wir, dass die Menge in 
    $\mathcal{A}$ \fat{paarweise disjunkt} sind, falls für alle $A_1,A_2 \in \mathcal{A}$
    mit $A_1 \neq A_2$ gilt $A_1 \cap A_2 = \emptyset$.
}

\vspace{1\baselineskip}

\Definition{

    Sei $X$ eine Menge. Die \fat{Potenzmenge} $\mathcal{P}(X)$ von $X$
    ist die Menge aller Teilmengen von $X$, das heisst
    \begin{align*}
        \mathcal{P}(X) = \geschwungeneklammer{Q \ | \ Q \text{ ist eine Menge und } Q \subset X}
    \end{align*}
}

\Definition{

    Für zwei Mengen $X$ und $Y$ ist das \fat{kartesische Produkt}
    $X \times Y$ die Menge aller geordneten Paare $(x,y)$ wobei
    $x \in X$ und $y \in Y$. In Symbolen,
    \begin{align*}
        X \times Y = \geschwungeneklammer{(x,y) \ | \ x \in X \land y \in Y}
    \end{align*}

    Allgemeiner, sei $I$ eine Indexmenge, und für jedes $i \in I$
    sei $X_i$ eine Menge. Das \fat{Produkt} der Familien von Mengen
    $\geschwungeneklammer{X_i | i \in I}$ definieren wir als
    \begin{align*}
        \prod_{i \in I} X_i = \geschwungeneklammer{(x_i)_{i \in I} \ | \ \forall i \in I \ : \ x_i \in X_i}
    \end{align*}
    Für eine Zahl $n \leq 1$ und eine Menge $X$ definieren wir $X^n$
    als das $n$-fache kartesische Produkt von $X$ mit sich selbst.
}

\vspace{2\baselineskip}

\subsection{Funktionen und Abbildungen}

\vspace{1\baselineskip}

\Definition{

    Seien \X und \Y Mengen. Eine \fat{Funktion} von \X nach \Y
    ist eine Teilmenge $F$ des karesischen Produktes $X \times Y$
    mit der Eigenschaft, dass es für jedes \xinX genau ein \yinY
    mit $(x,y) \in F$ gibt:
    \begin{align*}
        \forall x \in X \ \exists! \ y \in Y : (x,y) \in F
    \end{align*}
    Wir bezeichnen die Menge \X als \fat{Definitionsbereich} und 
    die Menge \Y als \fat{Wertebereich} oder auch als \fat{Zielbereich}.
}

\Definition{

    Die Menge $F$ ist gegeben durch
    \begin{align*}
        F = \geschwungeneklammer{(x,f(x)) | x \in X}
    \end{align*}
    gegeben und wird \fat{Graph} von $f$ bezeichnet.
}

\vspace{1\baselineskip}

\Definition{

    Sei \X eine Menge und sei $A$ eine Teilmenge von \X. Die
    Funktion $\iota_A : A \rightarrow X$ die durch $\iota_A(a) = a \ \forall a \in A$
    gegeben ist, nennt man \fat{Inklusionsabbildung} von $A$ nach
    \X. Die Inklusionsabbildung $\iota_X : X \rightarrow X$ wird
    \fat{Identität} von \X genannt, und als $id_X : X \rightarrow X$
    geschrieben. Die Funktion $\mathds{1}_A : X \rightarrow \geschwungeneklammer{0,1}$
    gegeben durch
    \begin{align*}
        \mathds{1}_A (x) = \begin{cases}
            0 \quad \text{ für } x \not\in A \\
            1 \quad \text{ für } x \in A
        \end{cases}
    \end{align*}
    wird \fat{charakteristische Funktion} von $A$ genannt. Für 
    zwei Mengen \X und \Y wird die \fat{Menge aller Abbildungen} von 
    \X nach \Y als $Y^X$ geschrieben, formaler
    \begin{align*}
        Y^X = \geschwungeneklammer{f \ | \ f: X \rightarrow X \text{ ist eine Funktion}}
    \end{align*}
    Grund der Notation ist unter anderem die folgende Behauptug. 
    Falls \X und \Y endliche Mengen sind mit $m$, respektive $n$
    Elementen für zwei natürliche Zahlen $m,n \in \N$, so hat 
    $Y^X$ genau $n^m$ Elemente.
}

\vspace{1\baselineskip}

\Definition{

    Sei $f: X \rightarrow Y$ eine Funktion, und sei $A$ eine
    Teilmenge von \X. Die Verknüpfung $f \circ \iota_A$ von $f$
    mit der Inklusionsabbildung $\iota_A : A \rightarrow X$ nennt
    man \fat{Einschränkung} von $f$ auf $A$. Man notiert diese 
    Funktion als 
    \begin{align*}
        f|_A : A \rightarrow Y
    \end{align*}
    Es gilt $f|_A (a) = f(a) \ \forall a \in A$. Dennoch betrachten
    wir $f|_A$ und $f$ als verschiedene Funktionen, da ihre 
    Definitionsbereich nicht derselbe ist - ausgenommen natürlich
    man hätte $A = X$ und also $\iota_A = id_X$
}

\vspace{1\baselineskip}

\Definition{

    Sei $f: X \rightarrow Y$ eine Funktion. Wir nennen $f$

    \vspace{1\baselineskip}

    1. \fat{injektiv} oder eine \fat{Injektion} falls $f(x_1) = f(x_2) \Rightarrow x_1 = x_2 \ \forall x_1,x_2 \in X$ gilt.

    2. \fat{surjektiv} oder eine \fat{Surjektion} falls $\forall \ y \in Y \ \exists x \in X$ mit $f(x) = y$

    3. \fat{bijektiv} oder eine \fat{Bijektion}, falls sie suejektiv und injektiv ist.

    \vspace{1\baselineskip}

    Ist $f$ bijektiv, so wird die Funktion $g: Y \rightarrow X$, 
    die eindeutig durch 
    \begin{align*}
        g \circ f = id_X 
        \quad
        und 
        \quad
        f \circ g = id_X
    \end{align*}
    bestimmt ist, \fat{Umkehrabbildung} von $f$, oder zu $f$
    \fat{inverse Funktion} genannt. Es ist also $g(y)$ das 
    eindeutig bestimmte Element \xinX mit $f(x) = y$.
}

\vspace{1\baselineskip}

\Lemma{

    Seien $f: X \rightarrow Y$ und $g: Y \rightarrow Z$ Funktionen.

    1. Falls $f$ und $g$ injektiv sind, dann ist auch $g \circ f$ injektiv.
    
    2. Falls $g \circ f$ injektiv ist, dann ist auch $f$ injektiv.

    3. Falls $f$ und $g$ surjektiv sind, dann ist auch $g \circ f$ surjektiv.
    
    4. Falls $g \circ f$ surjektiv ist, dann ist auch $g$ surjektiv.

    5. Falls $f$ und $g$ bijektiv sind, dann ist auch $g \circ f$
        bijektiv un des gilt $(g \circ f)^{-1} = f^{-1} \circ g^{-1}$.    
}

\vspace{1\baselineskip}

\Definition{

    Für eine Funktion $f: X \rightarrow Y$ und eine Teilmenge $A \subset X$
    schreiben wir
    \begin{align*}
        f(A) = \geschwungeneklammer{y \in Y \ | \ \exists x \in A : f(x)=y}
    \end{align*}
    und nennen diese Teilmenge von \Y das \fat{Bild} von $A$ 
    bezüglich der Funktion $f$. Für eine Teilmenge $B \subset Y$
    schreiben wir 
    \begin{align*}
        f^{-1}(B) = \geschwungeneklammer{x \in X \ | \ \exists y \in B : f(x) = y}
    \end{align*}
    und nennen diese Teilmenge von \X das \fat{Urbild} von $B$
    bezüglich der Funktion $f$.
}



\pagebreak

\subsection{Algebraische Strukturen}

\vspace{1\baselineskip}

\Definition{

    Ein \fat{kommutatives Monoid} ist ein Tripel $(M,m_0,f)$ bestehend aus 
    einer Menge $M$, einem Element $m_0 \in M$ und einer Abbildung
    $f: M \times M \rightarrow M$ die folgende Eigenschaften erfüllen.

    \vspace{1\baselineskip}

    1. Neutrales Element: $f(x,m_0) = f(m_0,x) = x \ \forall x \in M$

    2. Kommutativität: $f(x,f(y,z)) = f(f(x,y),z) \ \forall x,y,z \in M$

    3. Assiziativität: $f(x,y) = f(y,x) \ \forall x,y \in M$

    \vspace{1\baselineskip}

    Wir nennen die Funktion $f$ \fat{Addition} und $m_0$ \fat{neutrales Element}
    oder \fat{Null}, und schreiben üblicherweise $x+y$ anstelle
    von $f(x,y)$ und $0$ anstelle von $m_0$.
}

\vspace{1\baselineskip}

\Definition{

    Sei $X$ eine Menge. Eine \fat{Relation} auf \X ist eine Teilmenge
    $R \subset X \times X$. Wir schreiben auch $xRy$ falls $(x,y) \in R$.
    Wenn $\sim$ eine Relation ist, dann schreiben wir auch 
    $x \not\sim y$ für $\neg (x \sim y)$. Eine Relation $\sim$ heisst:

    \vspace{1\baselineskip}

    1. \fat{Reflexiv}: Falls $\forall x \in X : x \sim x$

    2. \fat{Transitiv}: Falls $\forall x,y,z \in X: ((x \sim y) \land (y \sim z)) \Rightarrow x \sim z$

    3. \fat{Symmetrisch}: Falls $\forall x,y \in X : x \sim y \Rightarrow y \sim x$

    4. \fat{Antisymmetrisch}: Falls $\forall x,y \in X: ((x \sim y) \land (y \sim x)) \Rightarrow x = y$

    \vspace{1\baselineskip}

    Eine Relation heisst \fat{Äquivalenzrelation}, falls sie reflexiv, transitiv 
    und symmetrisch ist. Eine Relation heisst \fat{Ordnungsrelation}, falls sie 
    reflexiv, transitiv und antisymmetrisch ist.
}

\vspace{1\baselineskip}

\Definition{

    Sei $\sim$ ein Äquivalenzrelation auf einer Menge $X$. Dann wird für \xinX
    die Menge
    \begin{align*}
        [x]_\sim = \geschwungeneklammer{y \in X \ | \ y \sim x}
    \end{align*}
    die \fat{Äquivalenzklasse} von $x$ genannt. Weiter heisst die Menge aller
    Äquivalenzklassen
    \begin{align*}
        X /_\sim = \geschwungeneklammer{[x]_\sim \ | \ x \in X}
    \end{align*}
    \fat{Quotient} oder die \fat{Quotientenmenge} von $X$ modulo $\sim$. 
    Ein Element \xinX wird auch \fat{Repräsentant} seiner Äquivalenzklasse
    $[x]_\sim$ genannt.
}

\vspace{1\baselineskip}

\Definition{

    Sei \X eine Menge. Eine \fat{Partition} von $X$ ist eine Familie $\mathcal{P}$ 
    von nicht leeren, paarweise disjunkten Teilmengen von $X$, so dass
    \begin{align*}
        X = \bigcup_{P \in \mathcal{P}} P
    \end{align*}
    gilt. Mit anderen Worten: Mengen $P \in \mathcal{P}$ sind nicht leer, und jedes 
    Element von $X$ ist Element von genau einem $P \in \mathcal{P}$.
}

\vspace{1\baselineskip}

\Proposition{

    Sei $X$ eine nicht leere Menge. Äquivalenzrelationen auf $X$ und 
    Partitionen von $X$ entsprechen einander im folgenden Sinne:
    Für eine gegebene Äquivalenzrelation $\sim$ auf $X$ ist die Menge
    \begin{align*}
        P = \geschwungeneklammer{[x]_\sim \ | \ x \in X}
    \end{align*}
    eine Partition von $X$. Umgekehrt definiert für eine gegebene Partition
    $P$ von $X$
    \begin{align*}
        x \sim y \ \Leftrightarrow \ \exists p \in P : x \in p  \land y \in p
    \end{align*}
    für \xyinX eine Äquivalenzrelation auf $X$. Wir erhalten eine kanonische
    Bijektion zwischen der Menge aller Äquivalenzrelationen auf \X und der Menge
    aller Partitionen von $X$.
}

\vspace{1\baselineskip}

\Definition{

    Eine Funktion $f$ heisst \fat{wohldefiniert}, falls $f$ nicht von der Wahl
    der Repräsentanten $x$ der Äquivalenzklasse $[x]_\sim$ abhängt und 
    jedem Element $[x]_\sim$ des Definitionsbereichs $X/_\sim$ das eindeutig
    bestimmte Element $f([x]_\sim)$ zuordnet.
}

\vspace{1\baselineskip}

\vspace{2\baselineskip}

\subsection{Kardinalität}

\vspace{1\baselineskip}

\Definition{

    Seien \X und \Y zwei Mengen. Wir sagen, dass $X$ und $Y$ \fat{gleichmächtig}
    sind, geschrieben $X \sim Y$ oder $\abs{X} = \abs{Y}$, falls es eine
    Bijektion $f: X \rightarrow Y$ gibt. Wir sagen dass $Y$ \fat{mächtiger}
    als $X$ ist, und schreiben $X \underset{\sim}{<} Y$, falls es eine 
    Injektion $f: X \rightarrow Y$ gibt. Wir sagen in dem Fall auch $X$
    sei \fat{schmächtiger} als $Y$.
}

\vspace{1\baselineskip}

\Definition{

    Wir sagen $X$ ist \fat{abzählbar} (unendlich), falls $\abs{X} = \abs{\N}$.

    Wir sagen, dass $X$ \fat{unendlich} ist, falls $\abs{\N} \leq \abs{X}$.

    Wir sagen, dass $X$ \fat{überabzählbar}  ist, falls $\abs{\N} < \abs{X}$
}

\vspace{1\baselineskip}

\Satz{
    (Cantors Diagonalargument)

    Sei $X$ eine Menge. Dann ist die $\mathcal{P}(X)$ mächtiger als $X$ und nicht gleichmächtig
    zu $X$.
}

\vspace{1\baselineskip}

\Satz{
    (Cantor, Schröder, Bernstein)

    Seien $X$ und $Y$ Mengen, so dass $X \underset{\sim}{<} Y$ und 
    $Y \underset{\sim}{<} X$. Dann gilt $X \sim Y$.
}

\vspace{1\baselineskip}

\Definition{

    Wir sagen, dass die Kardinalität der leeren Menge Null ist, und schreiben
    $\abs{\emptyset} = 0$. Sei $X$ eine Menge und $n \geq 1$ eine natürliche
    Zahl. Wir sagen die Menge $X$ habe Kardinalität $n$, und schreiben 
    $\abs{X} = n$, falls $X$ gleichmächtig zu $\geschwungeneklammer{1, \dots ,n}$
    ist. In diesem Fall nennen wir $X$ eine \fat{endliche Menge} und schreiben
    $\abs{X} < \infty$. Ist $X$ nicht endlich, so nennen wir $X$ eine 
    \fat{unendliche Menge}. Die Menge heisst \fat{abzählbar unendlich}, falls 
    sie gleichmächtig zu $\N$ ist. Die Kardinalität von $\N$ wird auch $\aleph_0$, 
    gesprochen \fat{Aleph-0}, genannt.
}

\vspace{1\baselineskip}

\textbf{Auswahlaxiom}

\vspace{1\baselineskip}

Variante (1)

Seien $X$ und $Y$ Mengen, und sei $f:X \rightarrow Y$ eine surjektive
Funktion. Dann existiert eine Funktion $g:Y \rightarrow X$ mit der 
Eigenschaft $f \circ g = id_Y$

Die Funktion $g$ in dieser Verion des Auswahlaxioms nennt man einen
\fat{Schnitt} von $f$. 

\vspace{1\baselineskip}

Variante (2)

Sei $Y$ eine Menge, und $\mathcal{X}$ eine Familie von nichtleeren Teilmengen von $Y$.
Dann gibt es eine Funktion $\alpha : X \rightarrow Y$ mit der Eigenschaft
dass $\alpha(X) \in X$ für alle $X \in \mathcal{X}$ gilt.
Die Funktio $\alpha$ in dieser Version nennt man \fat{Auswahlfunktion}, 
da sie der Auswahl eines Elementes $\alpha(X)$ in jeder der nichtleeren 
Mengen $X \in \mathcal{X}$ gleichkommt. 

\pagebreak

Variante (3)

Sei $\mathcal{X} = \geschwungeneklammer{X_i \ | \ i \in I}$ einer Familie von 
nichtleeren Mengen. Dann ist das Produkt
\begin{align*}
    \prod_{i \in I} X_i
\end{align*}
nicht leer.

\vspace{1\baselineskip}

\Definition{

    Sei $(X,\leq)$ eine geordnete Menge. Ein Element \xinX heisst \fat{maximal}
    falls für alle \yinX gilt: $x \leq y \Rightarrow x = y$. Ein Element 
    $m \in X$, so dass $x \leq m$ für alle \xinX gilt, dann heisst 
    $m \in X$ \fat{Maximum} von $X$.
}

\vspace{1\baselineskip}

\Definition{

    Sei $(X, \leq)$ eine geordnete Menge, und sei $A \subseteq X$ eine 
    Teilmenge. Ein Element \xinX heiss \fat{obere Schranke} von $A$ falls
    $a \leq x$ für alle $a\in A$ gilt. Ein Element \xinX heisst 
    \fat{untere Schranke} von A falls $x \leq a$ für alle $a \in A$ gilt.
}

\vspace{1\baselineskip}

\Definition{

    Sei $(X,\leq)$ eine geordnete Menge. Eine Teilmenge $K \subseteq X$
    heisst \fat{Kette}, falls für alle $x, y \in K$ gilt:
    $x \leq y$ oder $y \leq x$. Wir sagen $(X,\leq)$ sei \fat{induktiv}
    geordnet, falls jede Kette in $X$ eine obere Schranke besitzt.
}

\vspace{1\baselineskip}

\Lemma{
    (Zorn's Lemma)

    Sei $(X,\leq)$ eine induktiv geordnete Menge. Dann existiert ein maximales
    Element in $X$.
}


\vspace{1\baselineskip}

\Satz{
    (Hausdorff'sches Maximumprinzip)

    Sei $(X, \leq)$ eine geordnete Menge. Dann existiert eine maximale Kette 
    in $X$. Das heisst, es existiert eine Kette $M \subseteq X$, so dass 
    \begin{align*}
        M \subseteq L \ \Rightarrow \ M = L
    \end{align*}
    für jede Kette $L \subseteq X$ gilt.
}




\pagebreak

\section{Die reellen Zahlen}

\vspace{1\baselineskip}

In diesem Kapitel werden Körper thematisiert. Es wird vorausgesetzt,
dass der Leser, bzw Leserin, mit diesem Begriff vertraut ist.
Falls dies nicht der Fall ist, verweise ich auf die 
Unterlagen der Linearen Algebra \uproman{1} und \uproman{2}. 
Insbesondere möchte ich auf die Zusammenfassung 
\textit{Linearen Algebra \uproman{1} und \uproman{2}}
von Paul Sander verweisen.

Weitere Begriffe die bekannt sein sollten, sind \textit{ganze Zahlen}
und \textit{rationale Zahlen}.


\subsection{Die Axiome der reellen Zahlen}

\vspace{1\baselineskip}

\Definition{

    Sei $K$ ein Körper, und sei $\leq$ eine Ordnungsrelation auf 
    der Menge $K$. Wir nennen $(K,\leq)$, oder kurz $K$, einen
    \fat{angeordneten Körper} falls die folgenden Bedingungen
    erfüllt sind.

    \vspace{1\baselineskip}

    1. Linearität der Ordnung: Für alle $x,y, \in K$ gilt $x \leq y$ oder $y \leq x$.

    2. Kompatibilität von Ordnung und Addition: Für alle $x,y,z \in K$ gilt 
    \begin{align*}
        x \leq y \Rightarrow x + y \leq y + z
    \end{align*}
    3. Kompatibilität von Ordnung und Multiplikation: Für alle $x,y \in K$ gilt
    \begin{align*}
        (x \geq 0) \land (y \geq 0) \Rightarrow x \cdot y \geq 0
    \end{align*}
}

\vspace{1\baselineskip}

\Bemerkung{

    Im Folgenden sei $(K,\leq)$ ein angeordneter Körper, und 
    $x,y,z,w$ bezeichnen Elemente aus $K$.

    \vspace{1\baselineskip}

    (a) (Trichotomie) Es gilt entweder $x < y$ oder $x = y$ oder
    $x > y$.
    
    (b) Falls $x < y$ und $y \leq z$ ist, dann gilt auch $x < z$.

    (c) (Addition von Ungleichungen) Gilt $x \leq y$ und $z \leq w$,
    dann gilt auch $x + z \leq y + w$. 

    (d) Es gilt $x \leq y$ genau dann, wenn $0 \leq y - x$ gilt.

    (e) Es gilt $x \leq 0 \Leftrightarrow 0 \leq -x$.
    
    (f) Es gilt $x^2 \geq 0$, und $x > 0$, falls $x \neq 0$.

    (g) Es gilt $0 < 1$.

    (h) Falls $0 \leq x$ und $y \leq z$, dann gilt $xy \leq xz$.

    (i) Falls $x \leq 0$ und $y \leq z$, dann gilt $xy \geq xz$.

    (j) Aus $0 < x \leq y$ folgt $0 < y^{-1} \leq x^{-1}$.

    (k) Aus $0 \leq x \leq y$ und $0 \leq z \leq w$ folgt $0 \leq xz \leq yw$.

    (l) Aus $x+y \leq x + z$ folgt $y \leq z$.

    (m) Aus $xy \leq xz$ und $x > 0$ folgt $y \leq z$.
}

\vspace{1\baselineskip}

\Definition{

    Sei $(k,\leq)$ ein geordneter Körper. Der \fat{Absolutbetrag}
    auf $K$ ist die Funktion $\abs{ \hspace*{2pt} . \hspace*{2pt}}:K \rightarrow K$ die durch
    \begin{align*}
        \abs{x} = \begin{cases}
            x \quad \text{ falls } x \geq 0 
            \\
            -x \quad \text{ falls } x < 0
        \end{cases}
    \end{align*}
    für alle $x \in K$ definiert ist. Das \fat{Signum} ist die Funktion
    sgn: $K \rightarrow \geschwungeneklammer{-1,0,1}$ die durch
    \begin{align*}
        \text{sgn}(x) = \begin{cases}
            -1 \quad \text{ falls } x < 0 \\
            0 \quad \text{ falls } x = 0 \\
            1 \quad \text{ falls } x > 0
        \end{cases}
    \end{align*}
    für alle $x \in K$ definiert ist.
}

\pagebreak

\Bemerkung{

    Im Skript (Seite 46) gibt es Punkte \textit{(a)} bis \textit{(h)}. Hier werden nur
    \textit{(g)} und \textit{(h)} erwähnt.

    \vspace{1\baselineskip}

    (g) \fat{Dreiecksungleichung}: Es gilt $\abs{x + y} \leq \abs{x} + \abs{y}$.

    (h) Umgekehrte Dreiecksungleichung: Es gilt $\abs{y} - \abs{x} \leq \abs{x - y}$.
}

\vspace{1\baselineskip}

\Definition{
    (Vollständigkeitsaxiom)

    Sei \angeordneterK ein angeordneter Körper. Wir sagen \angeordneterK
    sei \fat{vollständig} oder \fat{vollständig angeordneter Körper} falls
    Aussage (\uproman{5}) wahr ist.

    \vspace{1\baselineskip}

    (\uproman{5}) 
    \indent 
    Seien $X,Y$ nicht-leere Teilmengen von $K$ derart, dass
    für alle \xinX und \yinY die Ungleichung $x \leq y$ gilt,
    dann \hspace*{7mm}gibt es ein
    $c \in K$, das zwischen $X$ und $Y$ liegt, in dem Sinn, dass für alle
    \xinX und \yinY die 
    Ungleichung \hspace*{7mm}$x \leq c \leq y$ gilt.

    \vspace{1\baselineskip}

    Die Aussage (\uproman{5}) bezeichnen wir als \fat{Vollständigkeitsaxiom}.
}

\vspace{1\baselineskip}

\Definition{

    Wir nennen \fat{Körper der reellen Zahlen} jeden vollständigen
    angeordneten Körper. Solge Körper notieren wir mit dem Symbol $\R$.
}


\vspace{2\baselineskip}

\subsection{Intervalle}

\vspace{1\baselineskip}

\Definition{

    Seien $a,b \in \R$. Dann ist das \fat{abgeschlossene Intervall} $[a,b]$
    definiert durch
    \begin{align*}
        [a,b] = \geschwungeneklammer{x \in \R \ | \ a \leq x \leq b}
    \end{align*} 
    und das \fat{offene Intervall} $(a,b)$ definiert durch
    \begin{align*}
        (a,b) = \geschwungeneklammer{x \in \R \ | \ a < x < b}
    \end{align*}
    Die \fat{halboffenen Intervalle} $[a,b)$ und $(a,b]$ sind durch
    \begin{align*}
        [a,b) = \geschwungeneklammer{x \in \R \ | \ a \leq x < b}
        \quad
        \text{  und     }
        \quad
        (a,b] = \geschwungeneklammer{x \in \R \ | \ a < x \leq b}
    \end{align*}
    definiert. Wir definieren die \fat{unbeschränkten abgeschlossenen
    Intervalle}
    \begin{align*}
        [a,\infty) = \R_{\geq a} =  \geschwungeneklammer{x \in \R \ | \ a \leq x}
        \quad
        \text{  und     }
        \quad
        (-\infty,b] = \R_{\leq b} = \geschwungeneklammer{x \in \R \ | \ x \leq b}
    \end{align*}
    sowie die \fat{unbeschränkten offenen Intervalle}
    \begin{align*}
        (a,\infty) = \R_{> a} =  \geschwungeneklammer{x \in \R \ | \ a < x}
        \quad
        \text{  und     }
        \quad
        (-\infty,b) = \R_{< b} = \geschwungeneklammer{x \in \R \ | \ x < b}
    \end{align*}
    und schliesslich $(-\infty,\infty) = \R$.
}

\vspace{1\baselineskip}

\Definition{

    Sei $x \in \R$. Eine Menge, die ein offenes Intervall enthält, in 
    dem $x$ liegt, wird auch eine \fat{Umgebung} oder \fat{Nachbarschaftne}
    von $x$ genannt. Für ein $\delta > 0$ wird das offene Intervall
    $(x-\delta,x+\delta)$ die \fat{$\delta$-Umgebung} von $x$ genannt.
}

\vspace{1\baselineskip}

\Definition{

    Eine Teilmenge $U \subseteq \R$ heisst \fat{offen} in $\R$, wenn für jedes
    $x \in U$ ein offenes Intervall $I$ mit $x \in I$ und $I \subset U$ 
    existert. Eine Teilmenge $F \subseteq \R$ heisst \fat{abgeschlossen} 
    in $\R$, wenn ihr Komplement $\R \backslash F$ offen ist.
}










\pagebreak

\subsection{Komplexe Zahlen}

\vspace{1\baselineskip}

Es wird vorausgesetzt, dass der Leser, bzw die Leserin, bereits eine
Vorstellung hat was komplexe Zahlen sind und wie die Notation funktioniert. 
Falls dies nicht der Fall ist, muss sich mittels anderer Literatur 
geholfen werden. Hier werden nur Operationen und mathematische Tools
des Körpers der komplexen Zahlen thematisiert.

Kurze erläuterung der Notation: Wir schreiben $z = (x,y)$ wobei
$x$ der Realteil von $z$ und $y$ der Imaginärteil von $z$ ist.
Die gängigere Schreibweise ist jedoch $z = x + yi$.

\vspace{1\baselineskip}

\Definition{

    Wir nennen \fat{Addition} und \fat{Multiplikation} auf der Menge
    $\C$ die folgenden Operationen.

    \vspace{1\baselineskip}

    Addition: $(x_1,y_1) + (x_2,y_2) = (x_1 + x_2, y_1 + y_2)$  

    Multiplikation: $(x_1,y_1) \cdot (x_2,y_2) = (x_1 x_2 - y_1 y_2, x_1 y_2 + x_2 y_1)$
}

\vspace{1\baselineskip}

\Proposition{

    Die Menge $\C$, zusammen mit dem Nullelement $(0,0)$, dem Einselement
    $(1,0)$ und den vorher definierten Operationen, ist ein Körper.
}

\vspace{1\baselineskip}

\Definition{

    Sei $z = x + yi$ eine komplexe Zahl. Wir nennen $\overline{z} = x - yi$
    die zu $z$ \fat{konjugierte} komplexe Zahl. Die Abbildung $\C \rightarrow \C$
    gegeben durch $z \rightmapsto \overline{z}$ heisst \fat{komplexe Konjugation}.
}

\vspace{1\baselineskip}

\Lemma{

    Die komplexe Konjugation erfüllt folgende Eigenschaften:

    \vspace{1\baselineskip}

    1. Für alle \zinC ist $z \overline{z} \in \R$ und $z \overline{z} \geq 0$. 
    Des Weiteren gilt für alle $z \in \C$, dass $z \overline{z} = 0$ genau dann wenn $z=0$.

    2. Für alle $z,w \in \C$ gilt $\overline{z + w} = \overline{z} + \overline{w}$.

    3. Für alle $z,w \in \C$ gilt $\overline{z \cdot q} = \overline{z} \cdot \overline{w}$.
}

\vspace{1\baselineskip}

\Bemerkung{

    Es lässt sich keine Ordnung auf $\C$ definierten, die zur Addition
    und zur Multiplikation kompatibel ist.
}

\vspace{1\baselineskip}

\Definition{

    Der \fat{Absolutbetrag} oder die \fat{Norm} auf $\C$ ist die
    Funktion $\abs{ \hspace*{2pt} . \hspace*{2pt} } : \C \rightarrow \R$ gegeben durch
    \begin{align*}
        \abs{z} = \sqrt{z \overline{z}} = \sqrt{x^2 + y^2}
    \end{align*}
    für $z = x + yi \in \C$.
}

\vspace{1\baselineskip}

\Bemerkung{

    Der Ausolutbetrag ist positiv definit und multiplikativ, dh.
    \begin{align*}
        \abs{z w} = \sqrt{z w \overline{z w}} = \sqrt{z \overline{z}} \sqrt{w \overline{w}} = \abs{z} \abs{w}
        \quad \forall z,w \in \C
    \end{align*}
}

\vspace{1\baselineskip}

\Proposition{
    (Dreiecksungleichung)

    Für alle $z,w \in \C$ gilt $\abs{z + w} \leq \abs{z} + \abs{w}$.
}

\pagebreak

\Definition{
    (\fat{Cauchy-Schwarz Ungleichung})

    Für komplexe Zahlen $z = x_1 + y_1 i$ und $w = x_2 + y_2 i$ gilt
    \begin{align*}
        x_1 x_2 + y_1 y_2 \leq \abs{z} \abs{w}
    \end{align*}
}

\vspace{1\baselineskip}

\Definition{

    Die \fat{offene Kreisscheibe} mit Radius $r > 0$ um einen Punkt
    \zinC ist die Menge 
    \begin{align*}
        B(z,r) = \geschwungeneklammer{w \in \C \ | \ \abs{z-w} < r}
    \end{align*}
    Die \fat{abgeschlossene Kreisscheibe} mit Radius $r > 0$ um 
    \zinC ist die Menge
    \begin{align*}
        \overline{B(z,r)} = \geschwungeneklammer{w \in \C \ | \ \abs{z-w} \leq r}
    \end{align*}
}

\vspace{1\baselineskip}

\Definition{

    Eine Teilmenge $U \subseteq \C$ heisst \fat{offen} in $\C$, 
    wenn zu jedem Punkt in $U$ eine offene Kreisscheibe um diesen Punkt
    existert, die in $U$ enthalten ist. Formaler: Für alle $z \in U$ 
    existert ein Radius $r > 0$, so dass $B(z,r) \subseteq U$. 
    Eine Teilmenge $A \subseteq \C$ heisst \fat{abgeschlossen} in 
    $\C$, falls ihr Komplement $\C \backslash A$ offen ist.
}


\vspace{2\baselineskip}

\subsection{Maximum und Supremum}

\vspace{1\baselineskip}

\Definition{

    Eine Teilmenge $X \in \R$ ist \fat{von oben beschränkt}, falls es ein
    $s \in \R$ gibt mit $x \leq s$ für alle $x \in X$. Ein solches $s \in \R$
    nennt man eine \fat{obere Schranke} von $X$.
}

\vspace{1\baselineskip}

\Definition{

    Sei $X \subseteq \R$ eine Teilmenge. Ein Element $x_1 \in X$ heisst
    \fat{Maximum} von $X$ falls für alle \xinX die Ungleichung
    $x \leq x_1$ gilt. Wir schreiben
    \begin{align*}
        x_1 = \text{max}(X)
    \end{align*}
    falls das Maximum von $X$ existert und gleich $x_1$ ist.
}

\vspace{1\baselineskip}

\Bemerkung{

    Das Maximum ist eindeutig. Das bedeutet wenn $x_1$ und $x_2$
    beide Maxima einer Teilmenge sind, folgt 

    $x_1 \leq x_2 \land x_1 \geq x_2 \Rightarrow x_1 = x_2$.
}

\vspace{1\baselineskip}

\Definition{

    Die Begriffe \fat{von unten beschränkt} und \fat{untere Schranke}
    sowie \fat{Minimum} sind analog definiert. Wir schreiben
    \begin{align*}
        x_0 = \text{min}(X)
    \end{align*}
    falls das Minimum von $X$ existert und gleich $x_0$ ist. Eine
    Teilmenge $X \subseteq \R$ heisst \fat{beschränkt}, falls sie 
    von oben und von unten beschränkt ist.
}

\vspace{1\baselineskip}

\Definition{

    Sei \XsubeqR eine Teilmenge und sei $A := \geschwungeneklammer{a \in \R \ | \ x \leq a \ \forall x \in X}$
    die Menge aller oberen Schranken von $X$. Falls das Minimum $x_0 =$ min$(A)$
    existert, dann nennen wir dieses Minimum \fat{Supremum} von $X$,
    und wir schreiben $x_0 =$ sup$(X)$.
}

\pagebreak

\Satz{

    Sei \XsubR eine von oben beschränkte, nicht leere Teilmenge.
    Dann existert ein Supremum von $X$.    
}

\vspace{1\baselineskip}

\Proposition{

    Seien $X$ und $Y$ von oben beschränkte, nicht leere Teilmengen von 
    $\R$, und schreibe
    \begin{align*}
        X + Y := \geschwungeneklammer{x + y \ | \ x \in X, y \in Y}
        \quad
        \text{  und  }
        \quad
        X Y := \geschwungeneklammer{x y \ | \ x \in X , y \in Y}
    \end{align*}
    Die Mengen $X \cup Y$, $X \cap Y$ und $X + Y$ sind von oben beschränkt,
    und falls $x \geq 0$ für alle \xinX und $y \geq 0$ für alle \yinY gilt,
    so ist auch $X Y$ von oben beschränkt.

    \vspace{1\baselineskip}
    
    (1) Es gilt sup$(X \cup Y)$ = max$\geschwungeneklammer{\text{sup}(X),\text{sup}(Y)}$.

    (2) Falls $X \cap Y$ nicht leer ist, so gilt sup$( X \cap Y) \leq$ min$\geschwungeneklammer{\text{sup}(X),\text{sup}(Y)}$.

    (3) Es gilt sup$(X + Y) =$ sup$(X)$ $+$ sup$(Y)$.

    (4) Falls $x \geq 0$ für alle \xinX und $y \geq 0$ für alle $y \in Y$,
    so gilt sup$(X Y)$ $=$ sup$(X)$sup$(Y)$
}

\vspace{1\baselineskip}

\Definition{

    Für eine von unten beschränkte, nicht leere Teilmenge \XsubeqR wird
    die grösste untere Schranke von $X$ auch als \fat{Infimum} inf$(X)$
    von $X$ genannt. Es gilt die zum vorherigen Satz analoge 
    Existenzaussage für das Infimum. Alternativ kann man das Infimum von
    $X$ als
    \begin{align*}
        - \text{sup}\geschwungeneklammer{-x \ | \ x \in X}
    \end{align*}
    definieren. Überhaupt kann man dadurch praktisch alle Aussagen über 
    Infima auf Aussagen über Suprema 
    
    zurückführen.
}

\vspace{1\baselineskip}

\Definition{

    Sei $X$ eine Teilmenge von $\R$. Falls \XsubR nicht von oben beschränkt
    ist, dann definieren wir sup$(X) = \infty$. Falls $X$ leer ist, setzen
    wir sup$(\emptyset) = - \infty$. Wir nennen in diesem Zusammenhang
    $\infty$ und $- \infty$ \fat{uneigentliche Werte}.
}


\vspace{2\baselineskip}

\input{Kapitel3_Die_reellen_Zahlen/5_Konsequenzen_der_Vollständigkeit.tex}

\pagebreak

\subsection{Modelle und Eindeutigkeit der Menge der reellen Zahlen}

\vspace{1\baselineskip}

\Definition{

    Wir schreiben $C(0) = \geschwungeneklammer{x \in \Q \ | \ x > 0}$, und nennen
    \fat{Dedekind-Schnitt} jede nicht leere, von unten beschränkte Teilmenge
    $C$ von $\Q$ mit der Eigenschaft, dass $C = C(0) + C$, also
    \begin{align*}
        C = \geschwungeneklammer{x + c \ | \ x \in C(0), c \in C}
    \end{align*}
    gilt. Wir schreiben $\mathcal{D}$ für die Menge aller Dedekind-Schnitte.
}

\vspace{1\baselineskip}

\Bemerkung{

    Für jede rationale Zahl $q$ ist die Menge $C(q) = \geschwungeneklammer{x \in \Q \ | \ q < x}$
    ein Dedekindschnitt.
}

\vspace{1\baselineskip}

\Satz{
    (Dedekind)

    Mit der Addition, Multipliation und der Ordnungsrelation, dem 
    Nullelement $C(0)$ und dem Einselement $C(1)$ bildet die Menge
    der Dedekind-Schnitte $\mathcal{D}$ einen vollständigen,
    angeordneten Körper. Die Rechenoperationen sind wir folgt gegeben:

    \vspace{1\baselineskip}

    Addition: $C(p+q) = C(p) + C(q)$

    Multipliation: $C(pq) = C(p)C(q)$

    Ordnungsrelation: $p \leq q \Rightarrow C(p) \leq C(q)$
}

\vspace{1\baselineskip}

\Definition{

    Seien $C$ und $D$ Dedekind-Schnitte. Die \fat{Summe} von $C$ und $D$
    ist der Dedekind-Schnitt 
    \begin{align*}
        C + D = \geschwungeneklammer{c + d \ | \ c \in C , d \in D}
    \end{align*}
    Wir sagen $C$ ist \fat{kleiner} als $D$ und schreiben $C \leq D$
    falls $C \supseteq D$ gilt.
}

\vspace{1\baselineskip}

\Lemma{

    Mit der Operation $+$ ist die Menge der Dedekind-Schnitte $\mathcal{D}$
    dine kommutative Gruppe mit neutralem Element $C(0)$. Ausserdem gilt

    1. $C \leq D$ oder $D \leq C$ für alle $C,D \in \mathcal{D}$

    2. $C \leq D \Rightarrow C + E \leq D + E$ für alle $C,D,E \in \mathcal{D}$
}

\vspace{1\baselineskip}

\Definition{

    Seien $C$ und $D$ Dedekind-Schnitte. Das \fat{Produkt} von $C$ und $D$
    ist der Dedekind-Schnitt
    \begin{align*}
        C D = \begin{cases}
            \geschwungeneklammer{c d \ | \ c \in C , d \in D} \ &\text{ falls } C(0) \leq C \text{ und } C(0) \leq D
            \\
            - \geschwungeneklammer{c d \ | \ c \in -C , d \in D} \ &\text{ falls } C(0) \geq C \text{ und } C(0) \leq D
            \\
            - \geschwungeneklammer{c d \ | \ c \in C , d \in -D} \ &\text{ falls } C(0) \leq C \text{ und } C(0) \geq D
            \\
            \geschwungeneklammer{c d \ | \ c \in -C , d \in -D} \ &\text{ falls } C(0) \geq C \text{ und } C(0) \geq D
        \end{cases}
    \end{align*}
}

\vspace{1\baselineskip}

\Lemma{

    Mit der vorher eingeführten Addition, Multipliation und Ordnungsrelation,
    dem Nullelement $C(0)$ und dem Einselement $C(1)$ bildet die Menge der 
    Dedekind-Schnitte $\mathcal{D}$ einen angeordneten Körper.
}

\vspace{1\baselineskip}

\Lemma{

    Der angeordnete Körper $(\mathcal{D}, \leq)$ ist vollständig.
}

\pagebreak

\Satz{
    (Eindeutigkeit der reellen Zahlen)

    Seien $(\R, \leq)$ und $(\eS,\leq)$ vollständig angeordnete Körper.
    Dann existiert genau eine Abbildung $\Phi : \R \rightarrow \eS$, die die 
    folgenen Eigenschaften erfüllt:

    \vspace{1\baselineskip}

    1. Additivität: $\Phi (0) = 0$ und $\Phi(x+y) = \Phi (x) + \Phi (y)$ für alle \xyinR

    2. Multiplikativität: $\Phi (1) = 1$ und $\Phi(x y) = \Phi(x) \Phi(y)$ für alle \xyinR

    3. Monotonie: $x \leq y \ \Rightarrow \Phi(x) \leq \Phi(y)$ für alle \xyinR 

    \vspace{1\baselineskip}

    Diese Abbildung $\Phi$ ist bijektiv. Eine solche Abbildung nennt man auch
    \fat{Isomorphismus}.
}

\pagebreak

\section{Reellwertige Funktionen in einer Variablen}

\vspace{1\baselineskip}

\subsection{Polynome und Polynomfunktionen}

\vspace{1\baselineskip}

In diesem Kapitel wird vom Leser, bzw von der Leserin, vorausgesetzt, dass er,
bzw sie, bereits vertraut ist mit der \fat{Summennotation} $\sum$ und der 
\fat{Produktnotation} $\prod$ und den dazugerörigen Konventionen der Indexierung.
Eine Sache, die es dennoch wert ist zu erwähnen, ist, dass wir die folgende
Konventionen brauchen werden.
\begin{align*}
    \sum_{j \in \emptyset} a_j = 0
    \quad
    \text{  und  }
    \quad
    \prod_{j \in \emptyset} a_j = 1
\end{align*}

\vspace{1\baselineskip}

\Bemerkung{

    Die Summe ist linear. Das heisst:
    \begin{align*}
        \csum{k=1}{n} (a_k + b_k) = \csum{k=1}{n} a_k + \csum{k=1}{n} b_k
        \quad
        \text{  und  }
        \quad
        \csum{k=1}{n}(c a_k) = c \csum{k=1}{n} a_k
    \end{align*}
}

\vspace{1\baselineskip}

\Bemerkung{

    Für die sogenannte \fat{Teleskopsumme} gilt:
    \begin{align*}
        \csum{k=1}{n-1} (a_k - a_{k+1}) = a_1 - a_n
    \end{align*}
}

\vspace{1\baselineskip}

\Bemerkung{

    Für das Produkt gilt:
    \begin{align*}
        \prod_{k=1}^n (a_k b_k) = \klammer{\prod_{k=1}^n a_k} \klammer{\prod_{k=1}^n b_k}
        \quad
        \text{  und  }
        \quad
        \prod_{k=1}^n (c a_k) = c^n \cdot \prod_{k=1}^n a_k
    \end{align*}
}

\vspace{1\baselineskip}

\Lemma{
    (Bernoulli'sche Ungleichung)

    Für alle reellen Zahlen $a \geq -1$ und \ninN gilt $(1+a)^n \geq 1 + n a$.
}

\vspace{1\baselineskip}

\Proposition{
    (Geometrische Summenformel)

    Sei \ninN und $q \in \C$. Dann gilt:
    \begin{align*}
        \csum{k=0}{n} q^k = \begin{cases}
            n + 1 \quad &\text{ falls } q = 1
            \\
            \frac{q^{n+1} - 1}{q - 1} \quad &\text{ falls } q \neq 1
        \end{cases}
    \end{align*}

}

\vspace{1\baselineskip}

\Definition{

    Sei $K$ ein beliebiger Körper. Ein \fat{Polynom} $f$ in einer Variable
    $T$ und Koeffizienten in $K$ ist ein formaler Ausdruck der Form
    \begin{align*}
        f = a_0 + a_1 T + a_2 T^2 + \dots + a_n T^n = \csum{k=0}{n} a_k T^k
    \end{align*}
    für ein \ninN und Elementen $a_0, \dots a_n \in K$ die wir als 
    \fat{Koeffizienten} bezeichnen. Hierbei ist $T$ die sogenannte
    \fat{Variable}. Wir definieren den \fat{Polynomring} $K[T]$ als 
    die Menge der Polynome mit Koeffizienten in $K$ in der Variablen
    $T$, wobei Addition und Multiplikation formal durch Kommutativität
    und Distributionsgesetz gegeben ist.
}

\pagebreak

\Definition{

    Sei $f \in K[T]$ ein Polynom, gegeben wie in der vorherigen Definition.
    Falls $a_n \neq 0$, so nennen wir $a_n$ den \fat{Leitkoeffizienten}, und 
    die natürliche Zahl $n$ den \fat{Grad} von $f$. Wir definieren den Grad
    des Polynoms $f = 0$ formal als $- \infty$. Ein Polynom von Grad $\leq 0$
    heisst \fat{konstant}, ein Polynom von Grad $\leq 1$ heisst \fat{affin},
    und ein Polynom vom Grad $\leq 2$ heisst \fat{quadratisch.}
}

\vspace{1\baselineskip}

\Bemerkung{

    Ein Polynom $f$ ist keine Funktion.
}

\vspace{1\baselineskip}

\Definition{

    Eine \fat{Polynomfunktion} auf $\C$ ist eine Funktion $\C \rightarrow \C$
    der Form $z \rightmapsto f(z)$ für ein Polynom $f \in \C[T]$. Analog
    dazu definieren wir auch Polynomfunktionen auf $\R$.
}

\vspace{1\baselineskip}

\Proposition{

    Sei $f(T) \in \C [T]$ ein nicht-konstantes Polynom. Dann gibt es zu jeder
    positiven reellen Zahl $M > 0$ eine reelle Zahl $R \geq 1$, so dass für 
    alle \zinC mit $\abs{z} \geq R$ auch $\abs{f(z)} \geq M$ gilt.
}

\vspace{1\baselineskip}

\Korollar{

    Die Zuordnung, die jedem Polynom $f(T) \in \C [T]$ die zugehörige
    Polynomfunktion $z \in \C \rightmapsto f(z) \in \C$ zuweist bijektiv.
}

\vspace{1\baselineskip}

\Definition{

    Eine Nullstelle eines Polynoms $f \in K[T]$ ist eine Zahl $x \in K$
    mit $f(x) = 0$.
}

\vspace{1\baselineskip}

\Satz{
    (Fundamentalsatz der Algebra)

    Jedes nicht-konstante Polynom $f \in \C [T]$ mit komplexen Koeffizienten
    hat eine komplexe Nullstelle.
}

\vspace{1\baselineskip}

\Definition{

    Wir sagen, dass ein Polynom $g$ ein Polynom $f$ \fat{teilt} falls es ein
    Polynom $q$ gibt mit $f = qg$.
}

\vspace{1\baselineskip}

\Definition{

    Wir sagen, dass eine Nullstelle $z \in K$ von $f(T) \in K[T]$ \fat{Vielfachheit}
    $k \in \N$ hat, falls $(T-z)^k$ das Polynom $f$ teilt, aber $(T-z)^{k+1}$
    das Polynom $f$ nicht teilt.
}

\vspace{1\baselineskip}

\Satz{

    Ein Polynom $f(T) \in K[T]$ vom Grad $n \geq 1$ hat $n$ Nullstellen in 
    $K$ auch wenn man diese entsprechend ihrer Vielfachheit mehrfach zählt.
}

\vspace{1\baselineskip}

\Proposition{

    Sei $n \geq 1$ eine ganze Zahl, $z_1 , \dots , z_n \in K$ paarweise verschiedene
    Elemente, und $w_1, \dots , w_n \in K$ beliebige Elemente. Es gibt höchstens
    ein Polynom $f \in K[T]$ von Grad $n-1$ mit der Eigenschaft
    \begin{align*}
        f(z_k) = w_k \ \forall k = 1,2, \dots ,n
    \end{align*} 
    Das Auffinden eines solchen Polynoms wird als \fat{Lagrange Interpolation}
    bezeichnet. Wir beginnen damit, für jedes $k = 1,2, \dots , n$ das Polynom
    \begin{align*}
        Q_k(T) = \prod_{\substack{j=1 \\ j \neq k}}^n \frac{T - z_j}{z_k - z_j}
    \end{align*}
    zu definieren. Die Notation bedeutet hier, dass wir das Produkt über alle
    $j \in \geschwungeneklammer{1,2, \dots , n} \backslash \geschwungeneklammer{k}$
    nehmen. Das Polynom $Q_k (T)$ hat Grad $n-1$, und es gilt
    \begin{align*}
        Q_k (z_j) = \begin{cases}
            1 \quad \text{  falls } j = k
            \\
            0 \quad \text{  falls } j \neq k
        \end{cases}
    \end{align*}
    Das gesuchte Polynom $f \in K[T]$ erhalten wir als Linearkombination
    \begin{align*}
        f(T) = \sum_{k=1}^n w_k Q_k (T)
    \end{align*}
    wobei wir $f(z_k) = w_k$ wiederum direkt durch Einsetzen in die Formel sehen.
    Da alle Polynome $Q_k$ vom Grad $n-1$ sind, ist $f$ höchstens vom Grad $n-1$.
}

\vspace{1\baselineskip}

\Definition{

    Für ein rationales Polynom $\frac{f}{g}$ nennt man die Nullstellen
    von $g$ \fat{Pole}.
}

\vspace{1\baselineskip}

\Definition{

    Eine Zahl $\alpha \in \C$ heisst \fat{algebraisch}, falls es ein von 
    Null verschiedenes Polynom $f \in \Q[x]$ gibt mit $f(\alpha) = 0$.  
}

\vspace{2\baselineskip}

\subsection{Reellwertige Funktionen}

\vspace{1\baselineskip}

\Definition{

    Als \fat{reellwertige} Funktion bezeichnet man jede Funktion mit Wertebereich $\R$.
}

\vspace{1\baselineskip}

\Definition{

    Für eine beliebige, nicht-leere Menge $D$ definieren wir die Menge
    der \fat{reellwertigen} Funktionen auf $D$ als
    \begin{align*}
        \mathcal{F}(D) = \R^D = \geschwungeneklammer{f \ | \ f: D \rightarrow \R}
    \end{align*}
    Die Menge $\mathcal{F}(D)$ bildet einen Vektorraum über $\R$, wobei Addition
    und skalare Multiplikation durch
    \begin{align*}
        (f_1 + f_2)(x) = f_1(x) + f_2(x)
        \quad \text{    und    } \quad
        (\alpha f_1)(x) = \alpha f_1 (x)
    \end{align*}
    für alle $f_1 , f_2 \in \mathcal{F}(D)$ und $x \in D$. Die Menge 
    $\mathcal{F}(D)$ bildet mit dieser Operation einen kommutativen Ring.

}

\vspace{1\baselineskip}

\Definition{

    Wir definieren eine Ordnungsrelation auf $\mathcal{F}(D)$ durch
    \begin{align*}
        f_1 \leq f_2 \Leftrightarrow \forall x \in D : f_1(x) \leq f_2 (x)
    \end{align*}
    für $f_1 , f_2 \in \mathcal{F}(D)$. Wir sagen, dass $f \in \mathcal{F}(D)$ 
    \fat{nicht-negativ} ist, falls $f \geq 0$ gilt.
}

\vspace{1\baselineskip}

\Definition{

    Sei $D$ eine nicht-leere Menge und sei $f:D \rightarrow \R$ eine Funktion.
    Wir sagen die Funktion $f$ sei \fat{von oben beschränkt}, falls die 
    Wertemenge $f(D)$ von oben beschränkt ist, und wir sagen $f$ sei
    \fat{von unten beschränkt}, falls die Wertemenge $f(D)$ von 
    unten beschränkt ist. Wir sagen $f$ sei \fat{beschränkt} falls $f$ von
    unten und von oben beschränkt ist.
}

\pagebreak

\Definition{

    Analog zu den reellwertigen Funktionen, können auch die \fat{komplexwertigen Funktionen}
    definiert werden als
    \begin{align*}
        \mathcal{F}(D) = \geschwungeneklammer{f \ | \ f: D \rightarrow \C}
    \end{align*}
    Es gelten die analogen Definitionen für die Beschränktheit.
}

\vspace{1\baselineskip}

\Definition{

    Sei $D$ eine Teilmenge von $\R$. Eine Funktion $f : D \rightarrow \R$ heisst

    \vspace{1\baselineskip}

    \fat{monoton wachsend}, falls
    \begin{align*}
        \forall x,y \in D : x \leq y \Rightarrow f(x) \leq f(y)
    \end{align*} 

    \fat{streng monoton wachsend}, falls 
    \begin{align*}
        \forall x,y \in D : x < y \Rightarrow f(x) < f(y)
    \end{align*}

    \fat{monoton fallend}, falls
    \begin{align*}
        \forall x,y \in D : x \leq y \Rightarrow f(x) \geq f(y)
    \end{align*} 

    \fat{streng monoton fallend}, falls
    \begin{align*}
        \forall x,y \in D : x < y \Rightarrow f(x) > f(y)
    \end{align*} 

    \vspace{1\baselineskip}

    Wir nennen eine Funktion $f: D \rightarrow \R$ \fat{monoton}, falls sie 
    monoton wachsend oder monoton fallend ist, und \fat{streng monoton},
    falls sie streng monoton wachsend oder streng monoton fallend ist.
}

\vspace{1\baselineskip}

\Bemerkung{

    Streng monoton $\Rightarrow$ injektiv.
}

\vspace{1\baselineskip}

\Definition{

    Sei $D \subseteq \R$ eine Teilmenge und sei $f: D \rightarrow \R$ eine 
    Funktion. Wir sagen, dass $f$ \fat{stetig bei einem Punkt} $x_0 \in D$
    ist, falls es für alle $\varepsilon > 0$ ein $\delta > 0$ gibt, so dass 
    für alle $x \in D$ die Implikation
    \begin{align*}
        \abs{x - x_0} < \delta \Rightarrow \abs{f(x)-f(x_0)} < \varepsilon
    \end{align*}
    gilt. Die Funktion $f$ ist \fat{stetig auf} $D$, falls sie bei jedem Punkt
    von $D$ stetig ist. Formal ist Stetigkeit von $f$ auf $D$ also durch
    \begin{align*}
        \forall x_0 \in D : \forall \varepsilon > 0 \ \exists \delta > 0 \ \forall x \in D: \abs{x - x_0} < \delta \Rightarrow \abs{f(x)-f(x_0)} < \varepsilon
    \end{align*}
    definiert.
}

\vspace{1\baselineskip}

\Proposition{

    Sei $D \subseteq \R$, und seien $f_+ , f_2 : D \rightarrow \R$, die bei
    einem Punkt $x_0 \in D$ stetig sind. Dann sind auch die Funktionen 
    $f_1 + f_2, f_1 \cdot f_2$ und $a f_1$ für ein $a \in \R$ stetig bie 
    $x_0$. Insbesondere bildet die Menge der stetigen Funktionen 
    \begin{align*}
        \mathcal{C}(D) = \geschwungeneklammer{f \in \mathcal{F}(D) \ | \ f \text{ ist stetig}}
    \end{align*}
    einen Untervekorraum des Vektorraums $\mathcal{F}(D)$.
}

\vspace{1\baselineskip}

\Proposition{

    Verknüpfungen stetiger Funktionen sind wierderum stetig.
}

\pagebreak

\Korollar{

    Polynomfunktionen sind stetig, das heisst, $\R[x] \subseteq \mathcal{C}(\R)$
}

\vspace{1\baselineskip}

\Satz{
    (\fat{Zwischenwertsatz})

    Sei $D \subseteq \R$ eine Tielmenge, $f: D \rightarrow \R$ eine stetige
    Funktion und $[a,b]$ ein in $D$ enthaltenes Intervall. Für jede reelle Zahl $c$
    mit $f(a) \leq c \leq f(b)$ gibt es ein $x \in [a,b]$, so dass $f(x) = c$ gilt.
}

\vspace{1\baselineskip}

\Satz{
    (Umkehresatz)

    Sei $I$ ein Intervall und $f: I \rightarrow \R$ eine stetige, streng monotone
    Funktion. Dann ist $f(I) \subset \R$ ein Intervall und die Abbildung $f:I \rightarrow f(I)$
    hat eine stetige, streng monotone inverse Abbildung $f^{-1} : f(I) \rightarrow I$. 
}

\vspace{2\baselineskip}

\subsection{Stetige Funktionen auf kompakten Intervallen}

\vspace{1\baselineskip}

\Satz{

    Seien $a,b \in \R$ reelle Zahlen mit $a<b$ und sei $f: [a,b] \rightarrow \R$
    eine stetige Funktion. Dann ist $f$ beschränkt. Das heisst, es existiert
    ein $M \in \R$ mit $\abs{f(x)} \leq M$ für alle $x \in [a,b]$. 
}

\vspace{1\baselineskip}

\Definition{

    Sei $D$ eine Menge und $f: D \rightarrow \R$ eine reellwertige Funktion
    auf $D$. Wir sagen, dass die Funktion $f$ ihr \fat{Maximum} in einem Punkt
    $x_0 \in D$ \fat{annimmt}, wenn $f(x) \leq f(x_0)$ gilt. Wir bezeichnen
    $f(x_0)$ als das \fat{Maximum} von $f$. Analog \fat{nimmt} $f$ ihr 
    \fat{Minimum} in $x_0 \in D$ \fat{an}, falls $f(x) \geq f(x_0)$ für alle
    $x \in D$ gilt. In dem Fall nennen wir $f(x_0)$ das \fat{Minimum} von $f$.
    Maaxima und Minima bezeichnet man summarisch als \fat{Extrama} oder
    \fat{Extremwerte}.
}

\vspace{1\baselineskip}

\Satz{

    Seien $a \leq b \in \R$ reelle Zahlen und sei $f: [a,b] \rightarrow \R$
    eine stetige Funktion. Dann nimmt $f$ sowohl ihr Maximum als auch ihr 
    Minimum an.
}

\vspace{1\baselineskip}

\Definition{

    Sei $D \subseteq \R$ eine Teilmenge. Eine Funktion $f: D \rightarrow \R$
    heisst \fat{gleichmässig stetig}, falls es für alle $\varepsilon > 0$ ein
    $\delta > 0$ gibt, so dass
    \begin{align*}
        \abs{x-y} < \delta \Rightarrow \abs{f(x) - f(y)} < \varepsilon
    \end{align*}
    für alle $x,y \in D$ gilt.
}

\vspace{1\baselineskip}

\Bemerkung{
    (Unterschied stetig und gleichmässig stetig)

    Sei $X \subseteq \R$ eine Teilmenge von $\R$ und sei $x_0 \in X$.
    Dann sind stetig und gleichmässig stetig in Quantoren wie folgt
    definiert:

    \vspace{1\baselineskip}

    Stetig: $\forall x_0 \in X \ \forall \varepsilon > 0 \ \exists \delta > 0$ mit $\abs{x - x_0} < \delta \Rightarrow \abs{f(x)-f(x_0)} < \varepsilon$

    Gleichmässig Stetig: $\forall \varepsilon > 0 \ \exists \delta > 0$ mit $\abs{x-x_0} < \delta \Rightarrow \abs{f(x)-f(x_0)} > \varepsilon$

    \vspace{1\baselineskip}

    Der wesentliche Unterschied besteht darin, dass bei der gleichmässigen 
    Stetigkeit das $\delta$ gleich bleibt für alle $x \in X$, 
    und bei der "normalen" Stetigkeit ändert sich das $\delta$ für 
    jedes $x \in X$.
}

\vspace{1\baselineskip}

\Satz{

    Sei $[a,b]$ ein kompaktes Intervall für $a<b$ und $f: [a,b] \rightarrow \R$
    eine stetige Funktion. Dann ist $f$ gleichmässig stetig.
}

\pagebreak

\Definition{

    Eine Funktion $f: X \rightarrow \R$ heisst \fat{Lipschitz stetig}, falls
    ein $L \geq 0$ existiert mit
    \begin{align*}
        \abs{f(x) - f(y)} \leq L \abs{x-y}
    \end{align*}
    für alle $x,y \in X$. $L$ heisst Lipschitz-Konstante für $f$.
}

\vspace{1\baselineskip}

\Bemerkung{

    Lipschitz stetig $\Rightarrow$ Gleichmässig stetig $\Rightarrow$ Stetige

    \vspace{1\baselineskip}

    \fat{Achtung!} Dies ist nur eine einseitige Implikation und keine Äquivalenz.
}

\pagebreak

\section{Das Riemann Integral}

\vspace{1\baselineskip}

\subsection{Treppenfunktionen und deren Integral}

\vspace{1\baselineskip}

\Definition{

    Eine \fat{Zerlegung} von $[a,b]$ ist eine endliche Menge von Punkten
    \begin{align*}
        a = x_0 < x_1 < \dots < x_{n-1} < x_n = b
    \end{align*}
    mit $n \in \N$. Die Punkte $x_0 , \dots , x_n \in [a,b]$ werden Teilungspunkte
    der Zerlegung genannt.
}

\vspace{1\baselineskip}

\Definition{

    Eine Zerlegung $a = y_0 < y_1 < \dots < y_{n-1} < y_m = b$ wird 
    \fat{Verfeinerung} von $a = x_0 < x_1 < \dots < x_{n-1} < x_n = b$
    genannt, falls 
    \begin{align*}
        \geschwungeneklammer{x_0 , x_1 , \dots , x_{n-1} , x_n } \subseteq \geschwungeneklammer{y_0 , y_1 , \dots , y_{m-1}} , y_m
    \end{align*}
    gilt.
}

\vspace{1\baselineskip}

\Definition{

    Eine Funktion $f: [a.b] \rightarrow \R$ heisst \fat{Treppenfunktion},
    falls es eine Zerlegung $a = x_0 < x_1 < \dots < x_{n-1} < x_n = b$
    von $[a,b]$ gibt, so, dass für $k=1,2,\dots,n$ die Einschränkung von 
    $f$ auf das offene Intervall $(x_{k-1},x_k)$ konstant ist. Wir sagen in dem
    Fall auch die Funktion $f$ sei eine Treppenfunktion bezüglich der 
    Zerlegung $a = x_0 < x_1 < \dots < x_{n-1} < x_n = b$.
}

\vspace{1\baselineskip}

\Proposition{

    Seien $f_1$ und $f_2$ Treppenfunktionen auf $[a,b]$ und seien $s_1, s_2 \in \R$.
    Dann sind auch $s_1 f_1 + s_2 f_2$ und $f_1 \cdot f_2$ Treppenfunktionen.
}

\vspace{1\baselineskip}

\Bemerkung{
 
    Die Menge aller Treppenfunktionen auf $[a,b]$
    \begin{align*}
        \mathcal{T F}([a,b]) = \geschwungeneklammer{f \in \mathcal{F}([a,b]) \ | \ f \text{ ist eine Treppenfunktion}}
    \end{align*}
    ist ein Vektorraum, oder genauer, ein Untervektorraum des Vektorraums
    $\mathcal{F}([a,b])$ der reellwertigen Funktionen auf $[a,b]$.
    Da das Produkt zweier Treppenfunktionen wiederum eine Treppenfunktion ist,
    bilden die Treppenfunktionen einen Ring.
    Des Weiteren sind Treppenfunktionen beschränkt, da sie endliche Wertemengen
    haben.
}

\vspace{1\baselineskip}

\Definition{

    Sei $f: [a,b] \rightarrow \R$ eine Treppenfunktion bezüglich einer Zerlegung
    $a = x_0 < x_1 < \dots < x_{n-1} < x_n = b$ von $[a,b]$. Wir definieren
    das \fat{Integral} von $f$ auf $[a,b]$ als die reelle Zahl
    \begin{align*}
        \intab f(x) dx = \csum{k=1}{n} c_k (x_k - x_{k-1})
    \end{align*}
    wobei $c_k$ den Wert von $f$ auf dem Intervall $(x_{k-1},x_k)$ bezeichnet.
}

\vspace{1\baselineskip}

\Bemerkung{

    Ist $a = y0 < y_1 < \dots < y_{m-1} < y_m = b$ eine weitere Zerlegung
    von $[a,b]$ befüglich welcher $f$ eine Treppenfunktion ist, so muss gelten:
    \begin{align*}
        \csum{k=1}{n} c_k (x_k - x_{k-1}) = \csum{k=1}{m} d_k (y_k - y_{k-1})
    \end{align*}
}

\vspace{1\baselineskip}

\Proposition{

    Die Abbildung $\int : \mathcal{T F} ([a,b]) \rightarrow \R$ ist linear.
    Das heisst, für alle $f,g \in \mathcal{T F}([a,b])$ und $\alpha , \beta \in \R$
    ist $\alpha f + \beta g$ eine Treppenfunktion, und es gilt
    \begin{align*}
        \intab (\alpha f + \beta g)(x) dx = \alpha \intab f(x) + \beta \intab g(x) dx
    \end{align*}
}

\vspace{1\baselineskip}

\Proposition{

    Seien $f$ und $g$ Treppenfunktionen auf $[a,b]$ mit $f \leq g$. Dann
    gilt:
    \begin{align*}
        \intab f dx \leq \intab g dx
    \end{align*}
}

\vspace{2\baselineskip}

\subsection{Definition und erste Eigenschaften des Riemann-Integrals}

\vspace{1\baselineskip}

\Definition{

    Sei $f: [a,b] \rightarrow \R$ eine Funktion. Dann definieren wir die Menge
    der \fat{Untersummen} $\mathcal{U}(f)$ und \fat{Obersummen} $\mathcal{O}(f)$
    von $f$ durch
    \begin{align*}
        \mathcal{F}(f) = \geschwungeneklammer{ \intab u dx \ | \ u \in \mathcal{T F} \text{ und } u \leq f}
        \quad
        \text{      }
        \quad
        \mathcal{O}(f) = \geschwungeneklammer{ \intab o dx \ | \ o \in \mathcal{T F} \text{ und } f \leq o}
    \end{align*}
    Falls $f$ beschränkt ist, so sind diese Mengen nicht leer. Für $u,o \in \mathcal{T F}$
    mit $u \leq f \leq o$ gilt auch 
    \begin{align*}
        \intab u dx \leq \intab o dx
    \end{align*}
    und deshalb gilt $s \leq t$ für alle $s \in \mathcal{U}(F)$ und $t \in \mathcal{O}(f)$.
    Falls $f$ beschränkt ist, gilt insbesondere die Ungleichung
    \begin{align*}
        \text{sup } \mathcal{U} (f) \leq \text{inf } \mathcal{O} (f)
    \end{align*}
}

\vspace{1\baselineskip}

\Definition{

    Eine beschränkte Funktion $f: [a,b] \rightarrow \R$ heisst
    \fat{Riemann-integrierbar}, falls sup $\mathcal{U}(f)$ $=$ inf $\mathcal{O}(f)$
    gilt. In diesem Fall wird dieser gemeinsame Wert das \fat{Riemann-Integral}
    von $f$ genannt, und wird wie folgt geschrieben.
    \begin{align*}
        \intab f dx = \text{sup } \mathcal{U}(f) = \text{inf } \mathcal{O}(f)
    \end{align*} 
}

\vspace{1\baselineskip}

\Proposition{

    Sei $f: [a,b] \rightarrow \R$ beschränkt. Die Funktion $f$ ist Riemann-integrierbar
    genau dann, wenn es zu jedem $\varepsilon > 0$ Treppenfunktionen $u$ und 
    $o$ gibt, die folgendes erfüllen:
    \begin{align*}
        u \leq f \leq o 
        \quad
        \text{     und     }
        \quad
        \intab (o-u)dx < \varepsilon
    \end{align*}
}

\vspace{1\baselineskip}

\Definition{

    Wir schreiben $\mathcal{R}([a,b])$ oder einfach $\mathcal{R}$ falls
    $[a,b]$ klar aus dem Kontext ist, für die Menge der Riemann-integrierbaren
    Funktionen auf $[a,b]$.
    \begin{align*}
        \mathcal{R}([a,b]) = \geschwungeneklammer{ f \in \mathcal{F}([a,b]) \ | \ f \text{ ist Riemann-integrierbar}}
    \end{align*}
}

\pagebreak

\Bemerkung{

    Treppenfunktionen sind integrierbar, es gilt also
    $\mathcal{T F}([a,b]) \subseteq \mathcal{R}([a,b]) \subseteq \mathcal{F}([a,b])$
}

\vspace{1\baselineskip}

\Satz{

    Die Menge der Riemann-integrierbaren Funktionen $\mathcal{R}([a,b])$
    bildet einen linearen Unterraum von $\mathcal{F}([a,b])$ und das
    Integral ist eine lineare Funktion auf $\mathcal{R}([a,b])$. Das heisst,
    für $f,g \in \mathcal{R}([a,b])$ und $\alpha, \beta \in \R$ ist
    $\alpha f + \beta g$ integrierbar, mit Integral
    \begin{align*}
        \intab (\alpha f + \beta g) dx = \alpha \intab f dx + \beta \intab g dx
    \end{align*}
}

\vspace{1\baselineskip}

\Proposition{

    Seien $f: [a,b] \rightarrow \R$ und $g: [a,b] \rightarrow \R$ integrierbare
    Funktionen. Falls $f \leq g$, so gilt auch $\intab f dx = \intab g dx$.
}

\vspace{1\baselineskip}

\Definition{

    Sei $f:[a,b] \rightarrow \R$ eine Funktion. Wir definieren Funktionen
    $f^+ , f^-$ und $\abs{f}$ auf $[a,b]$ durch
    \begin{align*}
        f^+ (x) = \text{max} \geschwungeneklammer{0,f(x)},
        \quad
        f^- (x) = -\text{min} \geschwungeneklammer{0,f(x)},
        \quad
        \abs{f}(x) = \abs{f(x)}
    \end{align*}
    für $x \in [a,b]$. Die Funktion $f^+$ ist der \fat{Positivteil},
    $f^-$ ist der \fat{Negativteil} und $\abs{f}$ ist der \fat{Absolutbetrag}
    der Funktion $f$. Es gilt $f = f^+ + f^-$.
}

\vspace{1\baselineskip}

\Satz{

    Sei $f: [a,b] \rightarrow \R$ eine integrierbare Funktion. Dann sind auch
    $f^+ , f^-$ und $\abs{f}$ integrierbar, und es gilt
    \begin{align*}
        \abs{\intab f dx} \leq \intab \abs{f} dx
    \end{align*}
}

\vspace{2\baselineskip}

\subsection{Integrierbarkeitssätze}

\vspace{1\baselineskip}

\Satz{

    Jede monotone Funktion $f: [a,b] \rightarrow \R$ ist Riemann-integrierbar.
}

\vspace{1\baselineskip}

\Definition{

    Eine Funktion $f: [a,b] \rightarrow \R$ heisst \fat{stückweise monoton},
    falls es eine Zerlegung $a = x_0 < x_1 < \dots < x_{n-1} < x_n = b$ von $[a,b]$
    gibt, so dass $f|_{(x_{k-1} , x_k)}$ monoton ist für alle $k \in \geschwungeneklammer{1, \dots , n}$.
    Jede monotone Funktion ist stückweise monoton.
}

\vspace{1\baselineskip}

\Korollar{

    Jede stückweise monotone, beschränkte Funktion $f: [a,b] \rightarrow \R$
    ist Riemann-integrierbar.
}

\pagebreak

\Lemma{

    Sei $d \geq 0$ eine Ganze Zahl. Es existieren rationale Zahlen
    $c_0 , c_1 , \dots , c_n \in \Q$ mit der Eigenschaft, dass
    \begin{align*}
        \csum{k=1}{n} k^d = \frac{n^{d+1}}{d+1} + c_d n^d + c_{d-1} n^{d-1} + \dots + c_0
    \end{align*}
    für alle $n \geq 1$ gilt.
}

\vspace{1\baselineskip}

\Satz{

    Polynomfunktionen auf $[a,b]$ sind Riemann-integrierbar. Für alle Monome
    $x^d$ mit $d \geq 0$ gilt
    \begin{align*}
        \intab x^d dx = \frac{1}{d + 1} \klammer{b^{d+1} - a^{d+1}}
    \end{align*}
}

\vspace{1\baselineskip}

\Satz{

    Jede stetige Funktion $f: [a,b] \rightarrow \R$ ist integrierbar.
}

\pagebreak

\section{Folgen und Grenzwerte}

\vspace{1\baselineskip}

\subsection{Der metrische Raum}

\vspace{1\baselineskip}

\Definition{

    Ein \fat{metrischer Raum} $(X,d)$ ist eine Menge $X$ gemeinsam mit einer
    Abbildung $d: X \times X \rightarrow \R$, die die \fat{Metrik} auf $X$
    genannt wird und die folgenden drei Eigenschaften erfüllt.

    \vspace{1\baselineskip}

    1. Definitheit: Für alle $x_1,x_1 \in X$ gilt $d(x_1 , x_2) \geq 0$ und $d(x_1 , x_2) = 0 \Leftrightarrow x_1 = x_2$

    2. Symmetrie: Für alle $x_1 , x_2 \in X$ gilt $d(x_1 , x_2) = d(x_2 , x_1)$

    3. Dreiecksungleichung: Für alle $x_1 , x_2 x_3 \in X$ gilt $d(x_1 , x_3) \leq d(x_1 , x_2) + d(x_2 , x_3)$

    \vspace{1\baselineskip}

    Eine Metrik $d$ auf einer Menge $X$ weist je zwei Punkten ihre 
    \fat{Distanz} oder \fat{Abstand} zu.
}

\vspace{1\baselineskip}

\Definition{

    Die \fat{Standardmetrik} $d$ ist definiert durch 
    \begin{align*}
        d(x_1 , x_2) = \abs{x_1 - x_2}
    \end{align*}
    für alle $x_1 , x_2 \in X$.
}

\vspace{1\baselineskip}

\Definition{

    Die \fat{diskrete Metrik} $d$ auf einer Menge $X$ ist definiert durch
    \begin{align*}
        d(x_1 , x_2) = \begin{cases}
            1 \quad \text{ falls } x_1 \neq x_2 \\
            0 \quad \text{ falls } x_1 = x_2
        \end{cases}
    \end{align*}
}

\vspace{1\baselineskip}

\Definition{

    Wir setzen $X = \C$ und definieren die \fat{Manhatten-Metrik} $d_{NY}$ auf $X$ durch
    \begin{align*}
        d_{NY} (z_1 , z_2) = \abs{x_1 - x_2} + \abs{y_1 - y_2}
    \end{align*}
    für komplexe Zahlen $z_1 = x_1 + i y_1$ und $z_2 = x_2 + i y_2$.
}

\vspace{1\baselineskip}

\Definition{

    Eine weitere Metrik auf $\C$ ist die \fat{französische Eisenbahn Metrik}
    $d_{SNFC}$, definiert durch
    \begin{align*}
        d_{SNFC} (z_1, z_2) = \begin{cases}
            \abs{z_1 - z_2} \quad &\text{ falls } z_1 , z_2 \text{ linear abhängig über $\R$ sind}
            \\
            \abs{z_1} + \abs{z_2} \quad &\text{ falls } z_1 , z_2 \text{ linear unabhängig über $\R$ sind}
        \end{cases}
    \end{align*}
    für alle $z_1 , z_2 \in \C$.
}

\vspace{1\baselineskip}

\Definition{

    Sei $(X,d)$ ein metrischer Raum, $x_0 \in X$, $r \geq 0$ reell.
    Wir nennen die Teilmenge
    \begin{align*}
        B(x_0 , r) = \geschwungeneklammer{x \in X \ | \ d(x_0 ,x) < r}
    \end{align*}
    "offener Ball mit Zentrum $x_0$ und Radius $r$".
}

\pagebreak

\Definition{

    Sei $(X,d)$ ein metrischer Raum, $A \subseteq X$ eine Teilmenge.
    Wir sagen $A$ sei beschränkt falls eine reelle Zahl $R \geq 0$ existiert
    mit 
    \begin{align*}
        d(x,y) \leq R \ \forall x,y \in A
    \end{align*}
}

\vspace{1\baselineskip}

\Bemerkung{

    Ist $x_0 \in X$, so ist $A \subseteq X$ beschränkt $\Leftrightarrow$ $\exists R \geq 0$ mit $A \subseteq B(x_0, R)$
}

\vspace{1\baselineskip}

\Definition{

    Sei $(X,d)$ ein metrischer Raum und $A \subseteq X$. Wir sagen $A$ sei
    \fat{offen} in $X$, falls $\forall x_0 \in A \ \exists \delta > 0$ mit
    $B(x_0 , \delta) \subseteq A$. Wir nennen $B \subseteq X$ 
    \fat{abgeschlossen} in $X$, falls $X \backslash B$ offen in $X$ ist.
}

\vspace{1\baselineskip}

\Proposition{

    Sei $(X,d)$ ein metrischer Raum. Jede Vereinigung offener Teilmengen
    von $X$ ist offen. Jeder \underline{endliche} Durchschnitt offener
    Teilmengen von $X$ ist offen.
}

\vspace{1\baselineskip}

\Definition{

    Seien $(X,d_x)$ und $(Y,d_y)$ metrische Räume. Eine Funktion $f: X \rightarrow Y$
    heisst \fat{stetig} im Punkt $x_0 \in X$, falls
    \begin{align*}
        \forall \varepsilon > 0 \ \exists \delta > 0 \text{ mit } d_x(x,x_0) < \delta \Rightarrow d_y ( f(x) , f(x_0)) < \varepsilon \ \forall x \in X
    \end{align*}
    $f$ ist stetig auf $X$ falls $f$ stetig in jedem Punkt $x_0 \in X$ ist.

    \vspace{1\baselineskip}

    Ist $X \subseteq \R$ und $Y \subseteq \R$ mit der Standardmetrik
    $d(x,y) = \abs{x-y}$ auf $X$ und $\R$, so erhalten wir den bekannten
    Stetigkeitsbegriff.
}

\vspace{1\baselineskip}

\Definition{

    Seien $(X,d_x)$ und $(Y,d_y)$ metrische Räume. Eine Funktion $f: X \rightarrow Y$
    heisst \fat{gleichmässig stetig}, falls
    \begin{align*}
        \forall \varepsilon > 0 \ \exists \delta > 0 \text{ mit } d_x (x_1 , x_2) < \delta \Rightarrow d_y (f(x_1) , f(x_2)) < \varepsilon \ \forall x_1 , x_2 \in X
    \end{align*}
}

\vspace{1\baselineskip}

\Definition{

    Seien $(X,d_x)$ und $(Y,d_y)$ metrische Räume. Eine Funktion $f: X \rightarrow Y$
    heisst \fat{Lipschitz stetig}, falls
    \begin{align*}
        \exists L > 0 \text{ (reell) mit } d_y (f(x_1) , f(x_2)) \leq L \cdot d_x (x_1 , x_2)
    \end{align*}
}

\vspace{1\baselineskip}

\Definition{

    Seien $(X,d_x)$ und $(Y,d_y)$ metrische Räume. Eine Funktion $f: X \rightarrow Y$
    heisst \fat{Isometrie}, falls
    \begin{align*}
        d_y (f(x_1) , f(x_2)) = d_x (x_1 , x_2) \ \forall x_1 , x_2 \in X
    \end{align*}
}

\vspace{1\baselineskip}

\Bemerkung{

    $f$ ist eine Isometrie $\Rightarrow$ $f$ ist Lipschitz-stetig
    $\Rightarrow$ $f$ ist gleichmässig stetig $\Rightarrow$ $f$ ist stetig
}

\pagebreak

\Proposition{

    Seien $(X,d)$ und $(Y,d)$ metrische Räume und $f: X \rightarrow Y$
    eine Funktion. Dann sind äquivalent:

    \begin{enumerate}
        \item $f$ ist stetig.
        \item Für jede offene Teilmenge $U \subseteq Y$ ist das Urbild $f^{-1} (U) \subseteq X$
                offen. $f^{-1} = \geschwungeneklammer{x \in X \ | \ f(x) \in U}$.
        \item Ist $(x_n)_{n=0}^{\infty}$ eine konvergente Folge in $X$ mit
                Grenzwert $a \in X$, so ist $(f(x_n))_{n=0}^{\infty}$ konvergent,
                mit Grenzwert $f(a)$.
        \item Für alle \xinX und $\forall \epsilon > 0$ existiert ein $\delta > 0$ so, dass
                $d(x,x_1) < \delta \ \Rightarrow \ d(f(x),f(x_1)) < \epsilon$.
    \end{enumerate}

}

\vspace{2\baselineskip}

\subsection{Konvergenz von Folgen in einem metrischen Raum}

\vspace{1\baselineskip}

\Definition{

    Sei $X$ eine Menge. Eine \fat{Folge} in $X$ ist eine Abbildung $a: \N \rightarrow X$.
    Das Bild $a(n)$ von \ninN schreibt man auch als $a_n$ und bezeichnet es als
    das $n$-te \fat{Folgenglied} von $a$. Anstatt $a:N \rightarrow X$ schreibt
    man oft $(a_n)_{n \in \N}$ oder auch $(a_n)_{n=0}^{\infty}$. Eine Folge
    $(a_n)_{n=0}^{\infty}$ heisst \fat{konstant}, falls $a_n = a_m$ für alle
    $m , n \in \N$, und \fat{schliesslich konstant}, falls ein $N \in \N$
    existiert mit $a_n = a_m$ für alle $m,n \in \N$ mit $m,n \geq N$.
}

\vspace{1\baselineskip}

\Definition{

    Sei $(X,d)$ ein metrischer Raum, und sei $(x_n)_{n=0}^{\infty}$ eine Folge in $X$.
    Ein Element $a \in X$ heisst \fat{Grenzwert} oder \fat{Limes} falls
    \begin{align*}
        \forall \varepsilon > 0 \ \exists N \in \N  \text{ mit }
        n \geq N \Rightarrow d(x_n , a) < \varepsilon
    \end{align*}
}

\vspace{1\baselineskip}

\Proposition{

    Sei $(X,d)$ ein metrischer Raum, $(x_n)_{n=0}^{\infty}$ eine Folge in $X$.
    Sind $a,b \in X$ Grenzwerte dieser Folge, so gilt $a = b$.
}

\vspace{1\baselineskip}

\Definition{

    Eine Folge $(x_n)_{n=0}^{\infty}$ heisst \fat{konvergent}, falls ein
    Grenzwert $a \in X$ für diese Folge existiert. Wir schreiben
    $a = \limes{n \rightarrow \infty} x_n$.
}

\vspace{1\baselineskip}

\Definition{

    Folgen, die nicht konvergieren, nennt man \fat{divergent}.
}

\vspace{1\baselineskip}

\Definition{

    Eine Folge $(x_n)_{n=0}^{\infty}$ in einem metrischen Raum $(X,d)$ heisst
    \fat{beschränkt}, falls eine reelle Zahl $R > 0$ gibt, so dass $d(x_n,x_m) \leq R$
    für alle $n,m \in \N$ gilt.
}

\vspace{1\baselineskip}

\Lemma{

    Jede konvergente Folge ist beschränkt.
}

\vspace{1\baselineskip}

\Definition{

    Sei $(X,d)$ ein metrischer Raum und $(x_n)_{n=0}^{\infty}$ eine Folge
    in $X$. Ein Element $a \in X$ heisst \fat{Häufungspunkt} der Folge,
    falls
    \begin{align*}
        \forall \varepsilon > 0 \ \forall N \in \N \exists n \geq N \text{ mit } d(x_n, a) < \varepsilon
    \end{align*}
}

\vspace{1\baselineskip}

\Bemerkung{

    Häufungspunkte einer Folge sind im Allgemeinen nicht das selbe wie
    Häufungspunkte des Bildes $\geschwungeneklammer{x_n \ | \ n \in \N}$.
}

\vspace{1\baselineskip}

\Bemerkung{

    Eine Folge kann mehrere Häufungspunkte haben, aber nur einen Grenzwert.
}

\vspace{1\baselineskip}

\Proposition{

    Sei $(x_n)_{n=0}^{\infty}$ eine konvergierende Folge in $(X,d)$ mit
    Grenzwert $a \in X$. Dann ist $a$ der einzige Häufungspunkt dieser Folge.
}

\vspace{1\baselineskip}

\Definition{

    Sei $(x_n)_{n=0}^{\infty}$ eine Folge in einer Menge $X$. Eine \fat{Teilfolge}
    von $(x_n)_{n=0}^{\infty}$ ist eine Folge der Form $(x_{f(n)})_{n=0}^{\infty}$,
    wobei $f: \N \rightarrow \N$ eine strend monoton steigende Funktion ist.
}

\vspace{1\baselineskip}

\Lemma{

    Sei $(x_n)_{n=0}^{\infty}$ eine konvergente Folge in einem metrischen Raum.
    Jede Teilfolge von $(x_n)_{n=0}^{\infty}$ konvergiert, und hat den selben
    Grenzwert wie $(x_n)_{n=0}^{\infty}$.
}

\vspace{1\baselineskip}

\Proposition{

    Sei $(x_n)_{n=0}^{\infty}$ eine Folge in einem metrischen Raum $(X,d)$.
    Ein Element $a \in X$ ist genau dann ein Häufungspunkt von $(x_n)_{n=0}^{\infty}$,
    wenn es eine konvergente Teilfolge von $(x_n)_{n=0}^{\infty}$ mit
    Grenzwert $a$ gibt.
}

\vspace{1\baselineskip}

\Korollar{

    Eine konvergierende Folge hat genau einen Häufungspunkt, und zwar ihren
    Grenzwert.
}

\vspace{1\baselineskip}

\Definition{

    Eine Folge $(x_n)_{n=0}^{\infty}$ in einem metrischen Raum ist eine
    \fat{Cauchy-Folge}, falls
    \begin{align*}
        \forall \varepsilon > 0 \ \exists N \in \N \text{ mit } d(x_n,x_m) < \varepsilon \ \forall m,n \geq N
    \end{align*}
}

\vspace{1\baselineskip}

\Proposition{

    Jede konvergente Folge ist eine Cauchy-Folge.
}

\vspace{1\baselineskip}

\Definition{

    Ein metrischer Raum $(X,d)$ heisst \fat{vollständig}, falls jede
    Cauchy-Folge in $(X,d)$ konvergiert.
}

\vspace{1\baselineskip}

\Bemerkung{

    Die Räume $\R$ und $\C$ sind vollständig. $\Q$ jedoch nicht.
}

\vspace{1\baselineskip}

\Definition{

    Als \fat{kanonische Einbettung} $\iota : \Q \rightarrow \R$ bezeichnen
    wir die injektive, lineare Abbildung die $q \in \Q$ die Klasse der
    konstanten Folgen mit Wert $q$ zuordnet.
}

\vspace{1\baselineskip}

\Satz{
    (Banach'scher Fixpunktsatz)

    Sei $(X,d)$ ein nicht leerer, vollständiger metrischer Raum,
    seien $x,y \in X$ und sei
    $T : X \rightarrow X$ eine Abbildung mit der Eigenschaft: Für eine reelle
    Zahl $0 \leq \lambda < 1$ gilt
    \begin{align*}
        d(T(x),T(y)) \leq \lambda \cdot d(x,y)
    \end{align*}
    Dann existiert ein eindeutiges Element $a \in X$ mit $T(a) = a$.
}

\vspace{2\baselineskip}

\subsection{Folgen reeller und komplexer Zahlen}

\vspace{1\baselineskip}

\Satz{

    Sei $D \subseteq \R$ eine Teilmenge, $f: D \rightarrow \C$ eine Funktion und
    $x_0 \in D$. Die Funktion $f$ ist genau dann stetig bei $x_0$, wenn für jede
    konvergente Folge $(y_n)_{n=0}^{\infty}$ in $D$ mit $\limes{n \rightarrow \infty}
    y_n = x_0$ auch die Folge $(f(y_n))_{n=0}^{\infty}$ konvergiert und
    $\limes{n \rightarrow \infty} f(y_n) = f(x_0)$ gilt.
}

\vspace{1\baselineskip}

\Bemerkung{

    Grob gesagt ist der Inhalt dieses Satzes, dass eine Funktion $f: D \rightarrow \R$
    genau dann stetig ist, wenn sie konvergente Folgen auf konvergenten Folgen abbildet,
    mit dem richtigen Grenzwert. Dies bezeichnet man auch als \fat{Folgenstetigkeit}.
}

\vspace{1\baselineskip}

\Proposition{ (Rechenregeln)

    Seien $(x_n)_{n=0}^{\infty}$ und $(y_n)_{n=0}^{\infty}$ konvergente Folgen, dann gilt:
    \begin{align*}
        (x_n)_{n=0}^{\infty} + (y_n)_{n=0}^{\infty} &= (x_n + y_n)_{n=0}^{\infty}
        \\
        \alpha \cdot (x_n)_{n=0}^{\infty} &= (\alpha x_n)_{n=0}^{\infty} \ \forall \alpha \in \C
        \\
        \limesninf (x_n + y_n) &= \limesninf x_n + \limesninf y_n
        \\
        \limesninf (x_n y_n) &= (\limesninf x_n) (\limesninf y_n)
        \\
        \limesninf (\alpha x_n) &= \alpha \limesninf x_n \ \forall \alpha \in \C
        \\
        \limesninf (x_n^{-1}) &= (\limesninf x_n)^{-1}
    \end{align*}
    Bei letzterem ist wichtig: $x_n \neq = 0 \ \forall n \in \N$ und $\limesninf x_n \neq 0$.
    Ausserdem konvergieren alle obigen Folgen. Insbersondere bildet die Menge der konvergenten
    Folgen in $\R^{\N}$ einen Unterraum und Bildung des Grenzwertes ist eine lineare Abbildung
    von diesem Unterraum nach $\R$
}

\vspace{1\baselineskip}

\Proposition{

    Seien $(x_n)_{n=0}^{\infty}$ und $(y_n)_{n=0}^{\infty}$ Folgen reeller Zahlen mit Grenzwerten
    $A = \limesninf x_n$ und $B = \limesninf y_n$.

    1. Falls $A < B$, dann existiert ein $N \in \N$, so dass $x_n < y_n$ für alle $n \geq N$.

    2. Falls $x_n \leq y_n$ für alle $n \in \N$, dann gilt $A \leq B$.
}

\vspace{1\baselineskip}

\Lemma{ (Sandwich)

    Es seien $\xFolge$, $\yFolge$ und $\zFolge$ Folgen reeller Zahlen, so dass für ein
    \NinN \ die Ungleichung $\x_n \leq \y_n \leq z_n$ für alle $n \geq N$ gelten. Angenommen
    $\xFolge$ und $\zFolge$ sind konvergent und haben den selben Grenzwert. Dann ist auch
    die Folge $\yFolge$ konvergent und es gilt:
    \begin{align*}
        \limesninf x_n = \limesninf y_n = \limesninf z_n
    \end{align*}
}

\pagebreak

\Bemerkung{
    
    Beschränkte Folgen reeller Zahlen besitzen immer mindestens einen Häufungspunkt,
    oder äquivalent dazu, konvergierende Teilfolgen. Ausserdem sind monotone und Beschränkte
    Folgen stets konvergent.
}

\vspace{1\baselineskip}

\Satz{

    Eine monotone Folge reeller Zahlen $\xFolge$ konvergiert genau dann, wenn sie
    beschränkt ist. Falls die Folge $\xFolge$ monoton wachsend ist, so gilt
    \begin{align*}
        \limesninf x_n = \supremum \geschwungeneklammer{x_n \ | \ n \in \N}
    \end{align*}
    und falls die Folge $\xFolge$ monoton fallend ist, so gilt entsprechend
    \begin{align*}
        \limesninf x_n = \infimum \geschwungeneklammer{x_n \ | \ n \in \N}
    \end{align*}
}

\vspace{1\baselineskip}

\Definition{

    Eine Folge $\xFolge$ in $\R$ heisst \fat{monoton steigend}, falls $x_n \leq x_m$ für alle
    $n \leq m \ \in \N$. Analoge Definition ür streng monoton steigend.
}

\vspace{1\baselineskip}

\Satz{

    Jede monotone und beschränkte Folge konvergiert.
}

\vspace{1\baselineskip}

\Definition{

    Sei $\xFolge$ eine beschränkte Folge reeller Zahlen. Die reellen Zahlen definiert durch
    \begin{align*}
        \limsupninf x_n = \limesninf \klammer{\supremum \geschwungeneklammer{x_k \ | \ k \geq n}}
        \quad \text{  und  } \quad
        \liminfninf x_n = \limesninf \klammer{\infimum \geschwungeneklammer{x_k \ | \ k \geq n}}
    \end{align*}
    heissen \fat{Limes superior}, respektive \fat{Limes inferior} der Folge $\xFolge$.
}

\vspace{1\baselineskip}

\Satz{

    Der Limes Superior $A = \limsup_{ \rightarrow \infty} x_n$ einer beschränkten Folge reeller
    Zahlen $\xFolge$ erfüllt folgende Eigenschaften: Für alle $\epsilon > 0$ gibt es nur endlich
    viele Folgenglieder $x_n$ mit $x_n > A + \epsilon$, und unendlich viele Folgenglieder
    $x_n$ mit $x_n > A - \epsilon$.
}

\vspace{1\baselineskip}

\Korollar{

    Jede beschränkte Folge reeller Zahlen hat einen Häufungspunkt, und besitzt konvergente
    Teilfolgen.
}

\vspace{1\baselineskip}

\Korollar{

    Jede beschränkte Teilfolge in $\R$ besitzt einen Häufungspunkt.
}

\vspace{1\baselineskip}

\Korollar{

    Eine beschränkte Folge $\xFolge$ in $\R$ konvergiert genau dann, wenn folgendes gilt:
    \begin{align*}
        \limsupninf x_n = \liminfninf x_n
    \end{align*}
}

\vspace{1\baselineskip}

\Satz{

    Jede Cauchy-Folge konvergiert in $\R$.
}

\vspace{1\baselineskip}

\Definition{ (Uneigentliche Grenzwerte)

    Sei $xFolge$ eine Folge reeller Zahlen. Wir sagen $\xFolge$ \fat{divergiert gegen $+\infty$},
    und wir schreiben
    \begin{align*}
        \limesninf x_n = + \infty
    \end{align*}
    falls für jede reelle Zahl $R > 0$ ein \ninN existert, so, dass $x_n > R$ für alle
    $n \geq N$ gilt. Genauso sagen wir, dass $\xFolge$ \fat{gegen $-\infty$ divergiert},
    falls für jede reelle Zahl $R < 0$ ein \NinN existert, so, dass $x_n < R$ für alle
    $n \geq N$ gilt. In beiden Fällen sprechen wir von \fat{uneigentlichen Grenzwerten}.
}

\vspace{1\baselineskip}

\Proposition{

    Sei $\zFolge$ eine Folge in $\C$ mit $z_n = x_n + i y_n$. Also $x_n = \text{Re} (z_n)$,
    $y_n = \text{Im} (z_n)$. Dann gelten folgende Äquivalenzen:
    
    (1) $\zFolge$ konvergiert gegebn $C = A + B i$ $\Leftrightarrow$ $\xFolge$ konvergiert gegen $A$ und $\yFolge$ konvergiert gegen $B$.

    (2) $\zFolge$ ist beschränkt $\Leftrightarrow$ $\xFolge$ und $\yFolge$ sind beschränkt.

    (3) $\zFolge$ ist Cauchy $\Leftrightarrow$ $\xFolge$ und $\yFolge$ sind Cauchy
}

\vspace{1\baselineskip}

\Korollar{

    Jede Cauchy-Folge konvergiert in $\C$.
}

\vspace{1\baselineskip}

\Satz{

    Jede Cauchy-Folge in $\C$ konvergiert.
}

\vspace{1\baselineskip}

\Satz{

    $\R$ und $\C$ sind vollständige metrische Räume.
}



\vspace{2\baselineskip}

\subsection{Die Exponentialfunktion}

\vspace{1\baselineskip}

\Proposition{

    Sei \xinR eine reelle Zahl. Die Folge reeller Zahlen $\aFolge$ gegeben durch
    \begin{align*}
        a_n = \klammer{1 + \frac{x}{n}}^n
    \end{align*}
    ist konvergent, und ihr Grenzwert ist eine positive reelle Zahl.
}

\vspace{1\baselineskip}

\Definition{

    Die \fat{Exponentialfunktion} $\exp: \R \rightarrow \R_{\geq 0}$ ist die durch
    \begin{align*}
        \exp (x) = \limesninf \klammer{1 + \frac{x}{n}}^n
    \end{align*}
    für alle \xinR definierte Funktion. Die \fat{Euersche Zahl} $e \in \R$ ist definiert als
    \begin{align*}
        e = \exp (1) = \limesninf \klammer{1 + \frac{1}{n}}^n
    \end{align*}
}

\vspace{1\baselineskip}

\Satz{

    Die Exponentialfunktion $\exp : \R \rightarrow \R_{> 0}$ ist bijektiv, streng monoton
    steigend, und stetig. Ausserdem gilt:
    \begin{align*}
        \exp (0) &= 1 \\
        \exp (-x) &= \exp (x)^{-1} \ \ \text{für alle \xinX} \\
        \exp (x + y) &= \exp (x) \exp (y) \ \ \text{für alle } x,y \in \R
    \end{align*}
}

\vspace{1\baselineskip}

\Satz{

    Der Logarithmus $\log : \R_{> 0} \rightarrow \R$ ist eine streng monoton wachsende,
    stetige und bijektive Funktion. Des weiteren gilt:
    \begin{align*}
        \log (1) &= 0 \\
        \log (a^{-1}) &= - \log(a) \ \ \text{für alle } a \in \R_{> 0} \\
        \log (a b) &= \log (a) + log (b) \ \ \text{für alle } a,b \in \R_{> 0}
    \end{align*}
    Diese Funktion ist die eindeutige Umkehrfunktion von $\exp: $.
}

\vspace{1\baselineskip}

\Lemma{ (Bernoulli Ungleichung)

    Sei $a \in \R$ mit $a \geq -1$. Dann gilt für alle \ninN:
    \begin{align*}
        \klammer{1 + a}^n \geq 1 + n \cdot a
    \end{align*}
}

\vspace{1\baselineskip}

\Bemerkung{

    Für eine positive Zahl $a > 0$ und beliebigen Exponenten \xinR schreiben wir
    \begin{align*}
        a^x := \exp (x \log (a))
    \end{align*}
}


\vspace{2\baselineskip}

\subsection{Grenzwerte von Funktionen}

\vspace{1\baselineskip}

In diesem ganzen Kapitel gilt folgendes:
Es sei $D \subset \R$ und $x_0 \in \R$. Wir nehmen an, $x_0 \in D$ oder $x_0$ ist
Häufungspunkt von $D$.

\vspace{1\baselineskip}

\Definition{

    Sei $f: D \rightarrow \R$ eine Funktion. Eine reelle Zahl $A$ heisst \fat{Grenzwert von
    $f(x)$ für $x \rightarrow x_0$}, falls für jedes $\epsilon > 0$ ein $\delta > 0$ existiert,
    mit der Eigenschaft
    \begin{align*}
        x \in D \cap (x_0 - \delta , x_0 + \delta) \Rightarrow \abs{f(x) - A} < \epsilon
    \end{align*}
}

\vspace{1\baselineskip}

\Bemerkung{

    Es sei $f: D \rightarrow \R$ eine Funktion. Falls es einen Grenzwert $a$ von $f$ bei
    $x_0$ gibt, dann ist dieser Grenzwert eindeutig bestimmt und wir schreiben
    \begin{align*}
        a = \limes{x \rightarrow x_0} f(x)
    \end{align*}
}

\vspace{1\baselineskip}

\Lemma{

    Sind $f$ und $g$ Funktionen auf $D$, derart, dass die Grenzwerte
    \begin{align*}
        \limes{x \rightarrow x_0} f(x) = A
        \quad \text{   und   } \quad
        \limes{x \rightarrow x_0} g(x) = B
    \end{align*}
    existieren, so existieren auch die Grenzwerte
    \begin{align*}
        \limes{x \rightarrow x_0} f(x) + g(x) = A + B
        \quad \text{   und   } \quad
        \limes{x \rightarrow x_0} f(x) g(x) = A B
    \end{align*}
}

\vspace{1\baselineskip}

\Lemma{

    Sei $f: D \rightarrow \R$ eine Funktion. Ist $x_0$ ein Element von $D$, so ist $f$ genau
    dann stetig bei $x_0$, wenn $\limes{x \rightarrow x_0} f(x) = f(x_0)$ gilt.
}

\vspace{1\baselineskip}

\Bemerkung{

    Es sei $f: D \rightarrow \R$ eine Funktion. Auch wenn $f$ unstetig ist bei $x_0$,
    kann ein Grenzwert $A = \limes{x \rightarrow x_0} f^* (x)$ existieren, wobei
    $f^*$ die einschränkung von $f$ ist auf die Menge $D \backslash \geschwungeneklammer{x_0}$.
    Unter diesen Umständen nennt man den Punkt $x_0 \in D$ eine \fat{hebbare Unstetigkeitsstelle}
    von $f$. Die Funktion definiert durch
    \begin{align*}
        &\overline{f}: D \cup {x_0} \rightarrow \R
        \\
        &\overline{f} (x) = \begin{cases}
            f(x) \quad \text{  falls  } x \in D \\
            a \quad \text{  falls  } x=x_0
        \end{cases}
    \end{align*}
    ist stetig bei $x_0$. Somit wurde die Unstetigkeitsstelle von $\overline{f}$ behoben,
    in dem wir den Wert der Funktion $f$ an der Stelle $x_0$ durch $A$ ersetzt haben.
    Wir nennen $A$ auch einen Grenzwert von $f$ in einer \fat{punktierten Umgebung} von $x_0$.
    Wir nennen $\overline{f}$ die \fat{stetige Fortsetzung} von $f$ auf $D \cup \geschwungeneklammer{x_0}$.  
}

\vspace{1\baselineskip}

\Lemma{

    Sei $f: D \rightarrow \R$ eine Funktion. Dann gilt $A = \limes{x \rightarrow x_0} f(x)$
    genau dann, wenn für jede gegen $x_0$ konvergierende Folge $\yFolge$ in $D$ auch
    $\limes{n \rightarrow \infty} f(y_n) = A$ gilt.
}

\vspace{1\baselineskip}

\Proposition{

    Sei $f: D rightarrow \R$ eine Funktion. Die folgenden Aussagen sind äquivalent
    für $x_0 \in D$:

    (1) Die Funktion $f$ ist stetig bei $x_0$

    (2) $\limes{\stackrel{x \rightarrow x_0}{x \neq x_0}} f(x) = f(x_0)$
    
    (3) $\limes{x \rightarrow x_0}f(x) = f(x_0)$

    (4) Für jede Folge $\yFolge$ in $D$ mit $x_0 = \limes{n \rightarrow \infty} y_n$ gilt
        $\limes{n \rightarrow \infty} = f(x_0)$.
}

\vspace{1\baselineskip}

\Proposition{

    Seien $D,E \in \R$, sei $f: D \rightarrow E$ so, dass $\limes{x \rightarrow x_0} f(x) = a$
    existeirt und $a \in E$ gilt. Sei $g : E \rightarrow \R$ eine Funktion, die bei $a \in E$
    stetig ist. Dann gilt: $\limes{x \rightarrow x_0} g(f(x)) = g(a)$
}

\vspace{1\baselineskip}

\Proposition{

    Wir können konventionen für uneigentliche Grenzwerte einführen. Wir sagen, dass
    $f(x)$ gegen $+ \infty$ für $x \rightarrow x_0$ divergiert und schreiben
    $\limes{x \rightarrow x_0 } f(x) = + \infty$, falls für jede reelle Zahl $R > 0$ ein
    $\delta > 0$ existert, mit der Eigenschaft dass $\abs{x - x_0} < \delta \Rightarrow f(x) > R$
    für alle $x \in D$.
}

\vspace{1\baselineskip}

\Definition{

    Seien $D \subseteq \R$ und $x_0 \in \R$ derart, dass $D \cap [x_0 , x_0 + delta) \neq \emptyset$
    für alle $\delta > 0$ gilt. Sei $f: D \rightarrow \R$ eine Funktion. Eine reelle Zahl
    $A$ heisst \fat{rechtseitiger Grenzwert} von $f(x)$ bei $x_0$, falls jedes $\epsilon > 0$
    ein $\delta > 0$ existiert, mit
    \begin{align*}
        x \in D \cap [x_0 , x_0 + \delta) \Rightarrow \abs{f(x) - A} < \epsilon
    \end{align*}
}

\pagebreak

\Bemerkung{

    Existiert der rechtseitige Grenzwert $A$ von $f$ an der Stelle $x_0$, so benutzen wir
    die Notation
    \begin{align*}
        A = \limes{\stackrel{x \rightarrow x_0}{x_0 \leq x}} f(x)
    \end{align*}
    um dies auszudrücken. Falls $x_0$ Häufungspunkt von $D \cap (x_0 , \infty)$ ist, so
    können wir auch die punktierte Version des rechtseitigen Grenzwertes
    \begin{align*}
        A = \limes{\stackrel{x \rightarrow x_0}{x < x_0}} f(x)
    \end{align*}
    definieren. Analog zum rechtseitigen Grenzwert können wir auch den \fat{linksseitigen Grenzwert}
    definieren.
}

\vspace{1\baselineskip}

\Definition{

    Seien $D \subseteq \R$ und $x_0 \in \R$ derart, dass $D \cap (R, \infty) \neq \emptyset$
    für alle $R > 0$ gilt. Sei $f: D \rightarrow \R$ eine Funktion. Eine reelle Zahl $A$ heisst
    \fat{Grenzwert von $f(x)$ für $x \rightarrow \infty$}, falls für jedes $\epsilon > 0$
    ein $R > 0$ existiert, mit
    \begin{align*}
        x \in D \cap (R , \infty) \Rightarrow \abs{f(x) - A} < \epsilon
    \end{align*} 
}

\vspace{1\baselineskip}

\Bemerkung{
    \begin{center}
        Tabelle 1: Echte Grenzwerte für $x \rightarrow x_{0}$
    
        \vspace{1\baselineskip}
    
    
        \begin{tabular}{c|c|c} 
        Notation & $U_{\delta},$ nichtleer für alle $\delta>0$ & Bedingung: $\forall \varepsilon>0 \exists \delta>0$ \\
        \hline $\lim _{x \rightarrow x_{0}} f(x)=A$ & $\left(x_{0}-\delta, x_{0}+\delta\right) \cap D$ & $x \in U_{\delta} \Longrightarrow|f(x)-A|<\varepsilon$ \\
        $\lim _{x \rightarrow x_{0}} f(x)=A$ & $\left(x_{0}-\delta, x_{0}+\delta\right) \cap\left(D \backslash\left\{x_{0}\right\}\right)$ & $x \in U_{\delta} \Longrightarrow|f(x)-A|<\varepsilon$ \\
        $x \neq x_{0}$ & & \\
        $\lim _{x \rightarrow x_{0}} f(x)=A$ & {$\left[x_{0}, x_{0}+\delta\right) \cap D$} & $x \in U_{\delta} \Longrightarrow|f(x)-A|<\varepsilon$ \\
        $x \geq x_{0}$ & & \\
        $\lim _{x \rightarrow x_{0}} f(x)=A$ & $\left(x_{0}, x_{0}+\delta\right) \cap D$ & $x \in U_{\delta} \Longrightarrow|f(x)-A|<\varepsilon$ \\
        $x>x_{0}$ & & \\
        $\lim _{x \rightarrow x_{0}} f(x)=A$ & $\left(x_{0}-\delta, x_{0}\right] \cap D$ & $x \in U_{\delta} \Longrightarrow|f(x)-A|<\varepsilon$ \\
        $x \leq x_{0}$ & & \\
        $\lim _{x \rightarrow x_{0}} f(x)=A$ & $\left(x_{0}-\delta, x_{0}\right) \cap D$ & $x \in U_{\delta} \Longrightarrow|f(x)-A|<\varepsilon$ \\
        $x<x_{0}$ & &
        \end{tabular}
    
        \vspace{1\baselineskip}
    
        Tabelle 2:
        Uneigentliche Grenzwerte für $x \rightarrow x_{0}$
    
        \vspace{1\baselineskip}
    
    
        \begin{tabular}{c|c|c} 
        Notation & $U_{\delta},$ nichtleer für alle $\delta>0$ & Bedingung: $\forall M>0 \exists \delta>0$ \\
        \hline $\lim _{x \rightarrow x_{0}} f(x)=+\infty$ & Entspr. $\star$ wie in Tabelle 1 & $x \in U_{\delta} \Longrightarrow f(x)>M$ \\
        $\star$ & & \\
        $\lim _{x \rightarrow x_{0}} f(x)=-\infty$ & Entspr. $\star$ wie in Tabelle 1 & $x \in U_{\delta} \Longrightarrow f(x)<-M$ \\
        $\therefore$ &
        \end{tabular}
    
        \vspace{1\baselineskip}
    
        Tabelle 3: Echte Grenzwerte für $x \rightarrow \pm \infty$
    
        \vspace{1\baselineskip}
    
    
        \begin{tabular}{c|c|c} 
        Notation & $U_{R},$ nichtleer für alle $R>0$ & Bedingung: $\forall \varepsilon>0 \exists R>0$ \\
        \hline $\lim _{x \rightarrow+\infty} f(x)=A$ & $(R, \infty) \cap D$ & $x \in U_{R} \Longrightarrow|f(x)-A|<\varepsilon$ \\
        $\lim _{x \rightarrow-\infty} f(x)=A$ & $(-\infty,-R) \cap D$ & $x \in U_{R} \Longrightarrow|f(x)-A|<\varepsilon$
        \end{tabular}
    
        \vspace{1\baselineskip}
    
        Tabelle 4: Uneigentliche Grenzwerte für $x \rightarrow \pm \infty$
        
        \vspace{1\baselineskip}
        
        \begin{tabular}{c|c|l}
        \multicolumn{1}{c|} { Notation } & $U_{R},$ nichtleer für alle $R>0$ & Bedingung: $\forall M>0 \exists R>0$ \\
        \hline $\lim _{x \rightarrow \star} f(x)=+\infty$ & Entspr. $\star$ wie in Tabelle 3 & $x \in U_{R} \Longrightarrow f(x)>M$ \\
        $\lim _{x \rightarrow \star} f(x)=-\infty$ & Entspr. $\star$ wie in Tabelle 3 & $x \in U_{R} \Longrightarrow f(x)<-M$
        \end{tabular}
    \end{center}
}

\pagebreak

\Definition{

    Sei $D \subseteq \R$ eine Teilmenge, sei $f: D \rightarrow \R$ eine Funktion, und
    sei $x_0 \in D$. Falls der rechtseitige Grenzwert
    \begin{align*}
        A = \limes{\stackrel{x \rightarrow x_0}{x \geq x_0}} f(x)
    \end{align*}
    existiert, dann sagen wir, dass $f$ \fat{rechtseitig stetig} ist bei $x_0$. Analog dazu
    definieren wir \fat{linksseitige Stetigkeit}. Wir nennen $x_0 \in D$ eine \fat{Sprungstelle},
    falls die einseitigen Grenzwerte
    \begin{align*}
        \limes{\stackrel{x \rightarrow x_0}{x < x_0}} f(x)
        \quad \text{   und   } \quad
        \limes{\stackrel{x \rightarrow x_0}{x > x_0}} f(x)
    \end{align*}
    beide existieren, aber verschieden sind.
}

\vspace{1\baselineskip}

\Definition{ (Landau Notation)

    Sei $D \subseteq \R$ eine Teilmenge und sei $x_0 \in \R$ ein Element von $D$ oder
    ein Häufungspunkt von $D$. Seien $f,g: D \rightarrow \R$ Funktionen. Wir schreiben
    \begin{align}
        f(x) = O(g(x)) \ \ \text{   für } x \rightarrow x_0
    \end{align}
    falles ein $\delta > 0$ und eine reelle Zahl $M > 0$ existieren, mit
    \begin{align*}
        \abs{x - x_0} < \delta \Rightarrow \abs{f(x)} \leq M \abs{g(x)}
    \end{align*}
    für alle $x \in D$.
    \begin{align}
        f = O(g(x)) \ \ \text{   für } x \rightarrow - \infty
    \end{align}
    falls ein $R \in \R$ und ein $M > 0$ existeren, mit
    \begin{align*}
        x < R \Rightarrow \abs{f(x)} \leq M \abs{g(x)} 
    \end{align*}
    für ein $x \in D$.
    \begin{align}
        f(x) = o(g(x)) \ \ \text{   für } x \rightarrow x_0
    \end{align}
    falls für alle $\epsilon > 0$ ein $delta > 0$ existert, mit
    \begin{align*}
        \abs{x - x_0} \Rightarrow \abs{f(x)} \leq \epsilon \abs{g(x)}
    \end{align*}
    für ein $x \in D$.

    Falls $g(x) \neq 0$ für $x \in D$ mit $\abs{x - x_0} < \delta$ gibt, dann ist die Aussage
    $f(x) = o(g(x))$ für $x \rightarrow x_0$ äquivalent zu:
    \begin{align*}
        \limes{x \rightarrow x_0} \frac{f(x)}{g(x)} = 0
    \end{align*}
}

\vspace{1\baselineskip}



\vspace{2\baselineskip}

\subsection{Normen und Konvergenzen auf Vektorräumen}

\vspace{1\baselineskip}

In diesem Kapitel legen wir einen Körper $\K$ fest, der entweder $\R$ oder $\C$ ist.

\vspace{1\baselineskip}

\Definition{ 

    Sei $V$ ein Vekorraum über $\K$. Eine \fat{Norm} auf $V$ ist eine Abbildung
    $\Norm{\ \cdot \ }: V \rightarrow \R$, die folgende drei Eigenschaften erfüllt.
    
    (1) (Definitheit) Für alle \vinV gilt $\Norm{v} \geq 0$ und $\Norm{v}=0 \Leftrightarrow v = 0$.

    (2) (Homogenität) Für alle \vinV und alle $\alpha \in \R$ (bzw $\alpha \in \C$) gilt
        $\Norm{\alpha v} = \abs{\alpha} \Norm{v}$.

    (3) (Dreiecksungleichung) Für alle $v_1 , v_2 \in V$ gilt $\Norm{v_1 + v_1} \leq \Norm{v_1} + \Norm{v_2}$.

    Man nennt $V$ gemeinsam mit der Norm $\Norm{\ \cdot \ }$ einen \fat{normierten Vektorraum}.
}

\pagebreak

\Beispiel{

    Sei $n \in \N$. Die \fat{Maximumsnorm} oder \fat{Unendlichnorm} $\Norm{ \ \cdot \ }_{\infty}$
    und die \fat{1-Norm} $\Norm{ \ \cdot \ }_1$ auf $\K^n$ sind definiert durch
    \begin{align*}
        \Norm{v}_{\infty} = \maximum \geschwungeneklammer{\abs{v_1},\abs{v_2},\dots,\abs{v_n}}
        \quad \text{   und   } \quad
        \Norm{v}_1 = \sum_{j = 1}^n \abs{v_j}
    \end{align*}
    für $v = (v_1 , \dots , v_n) \in \K^n$. Allgemeiner kann man für eine reelle Zahl $p \geq 1$
    folgende Norm definieren
    \begin{align*}
        \Norm{v}_p = \klammer{\sum_{i=1}^n \abs{v_i}}^{\frac{1}{p}}
    \end{align*}
    Man nennt diese \fat{p-Norm}. Auf dem Vektorraum der stetigen Funktionen $f: [a,b] \rightarrow \R$
    lässt sich folgende Norm definieren für ein $f \in V$.
    \begin{align*}
        \Norm{f}_1 = \intab \abs{f(x)} dx
    \end{align*}
    Diese Norm wird \fat{$L^1$-Norm} genannt. Allgemeiner kann man für reelle Zahlen $p \geq 1$
    folgende Norm definieren
    \begin{align*}
        \Norm{f}_p = \klammer{\intab \abs{f(x)}^p \ dx}^{\frac{1}{p}}
    \end{align*}
    Eine solche Norm nennt man \fat{p-Norm} oder \fat{$L^p$-Norm}. Das $L$ steht für Lebesgue.
    Die Maximumsnorm auf dem Vektorraum der stetigen Funktionen ist wie folgt definiert
    für ein $f \in V$:
    \begin{align*}
        \Norm{f}_{\infty} = \maximum \geschwungeneklammer{\abs{f(x)} \ | \ x \in [a,b]}
    \end{align*}
    Diese nennt man auch \fat{$\infty$-Norm}, \fat{$L^{\infty}$-Norm} oder auch \fat{sup-Norm}.

    \vspace{1\baselineskip}

    Die \fat{Operatornorm} einer Matrix $A \in \text{Mat}_{m,n} (\R)$ ist durch
    \begin{align*}
        \Norm{A}_{\text{op}} = \sup \geschwungeneklammer{\Norm{Ax}_2 \ | \ x \in \R^n \text{ mit } \Norm{x}_2 \leq 1}
    \end{align*}
    definiert, wobei $\standardNorm_2$ für die Euklidische Norm steht. Des Weiteren gilt
    \begin{align*}
        \Norm{Ax}_2 \leq \Norm{A}_{\text{op}} \Norm{x}_2
        \quad \quad \text{   und   } \quad \quad
        \Norm{AB}_{\text{op}} \leq \Norm{A}_{\text{op}} \Norm{B}_{\text{op}}
    \end{align*}
    für alle $x \in \R^n$ und alle $A,B \in$Mat$_{m,n} (\R)$.
}

\vspace{1\baselineskip}

\Definition{

    Sei $V$ ein Vektorraum über $\K$ und seien $\Norm{ \ \cdot \ }_1$ und $\Norm{ \ \cdot \ }_2$
    zwei Normen auch $V$. Wir nennen $\Norm{ \ \cdot \ }_1$ und $\Norm{ \ \cdot \ }_2$
    \fat{äquivalent}, falls Konstanten $A > 0$ und $B>0$ existieren mit
    \begin{align*}
        \Norm{v}_1 \leq A \cdot \Norm{v}_2
        \quad \text{   und   } \quad
        \Norm{v}_2 \leq B \cdot \Norm{v}_1
        \ \ \ \ \ \ \forall v \in V
    \end{align*} 
}

\vspace{1\baselineskip}

\Lemma{

    Sei $V$ ein Vektorraum über $\R$ und $\Norm{ \ \cdot \ }$ eine Norm auf $V$. Dann ist
    \begin{align*}
        d: V \times V \rightarrow \R
        \ \ \ \ \ \ \ \ \ \ \ \
        d(v,w) = \Norm{v - w}
    \end{align*}
    eine Metrik auf $V$.
}

\vspace{1\baselineskip}

\Definition{

    Wir bezeichnen die im obigen Lemma gegebene Metrik die von der Norm $\Norm{ \ \cdot \ }$
    \fat{induzierte Metrik} auf $V$.
}

\pagebreak

\Satz{

    Sei $V$ ein $\K$ Vektorraum, seien $\Norm{ \ \cdot \ }_1$ und $\Norm{ \ \cdot \ }_2$
    Normen auf $V$. Dann sind folgende Aussagen äquivalent:
    
    (1) $\Norm{ \ \cdot \ }_1$ und $\Norm{ \ \cdot \ }_2$ sind äquivalent.

    (2) Die Identitätsabbildungen $V_{\einsNorm} \stackrel{id}{\longrightarrow} V_{\zweiNorm}$
        und $V_{\zweiNorm} \stackrel{id}{\longrightarrow} V_{\einsNorm}$ sind stetig.
    
    (3) Eine Teilmenge $U \subseteq V$ ist offen bezüglich der von $\einsNorm$ induzierten Metrik,
        genau dann, wenn sie offen bezüglich der von $\zweiNorm$ induzierten Metrik ist.

    (4) Eine Folge $(v_n)_{n=0}^{\infty}$ in $V$ konvergiert nach $w \in V$ bezüglich der
        $\einsNorm$ genau dann, wenn $(v_n)_{n=0}^{\infty}$ gegen $w$ bezüglich der
        $\zweiNorm$ konvergiert.
}

\vspace{1\baselineskip}

\Definition{

    Sei $\K$ ein Körper und $V$ ein Vektorraum über $\K$. Ein \fat{inneres Produkt} oder
    \fat{Skalarprodukt} auf $V$ ist eine Abbildung
    \begin{align*}
        \scalprod{-}{-} : V \times V \rightarrow \K
    \end{align*}
    die folgende Eigenschaften erfüllt

    (1) (Sesquilinearität) Für alle $u,v,w \in V$ und $\alpha , \beta \in \K$ gilt
    \begin{align*}
        &\scalprod{\alpha u + \beta v}{w} = \alpha \scalprod{u}{w} + \beta \scalprod{v}{w}
        \quad \text{   und} \\
        &\scalprod{u}{\alpha v + \beta w} = \overline{\alpha} \scalprod{u}{v} + \overline{\beta} \scalprod{u}{w}
    \end{align*}
    (2) (Symmetrie) Für alle $v,w \in V$ gilt $\scalprod{v}{w} = \overline{\scalprod{v}{w}}$.

    (3) ((positiv)Definitheit) Für alle \vinV ist $\scalprod{v}{v}$ reell und nichtnegativ,
        und $\scalprod{v}{v} = 0 \Leftrightarrow v = 0$.

    Das $\overline{.}$ steht für die komplexe Konjugation.
}

\vspace{1\baselineskip}

\Satz{ (Cauchy-Schwarz Ungleichung)

    Sei $V$ ein Vektorraum über $\K$, sei $\scalprod{-}{-}$ ein Skalarprodukt auf $V$
    und sei $\standardNorm: V \rightarrow \R$ gegeben durch $\Norm{v} = \sqrt{\scalprod{v}{v}}$.
    Dann gilt die Ungleichung
    \begin{align*}
        \abs{\scalprod{v}{w}} \leq \Norm{v} \Norm{w}
    \end{align*}
    für alle $v,w \in V$. Des weiteren gilt Gleichheit genau dann, wenn $v$ und $w$ linear
    unabhängig sind.
}

\vspace{1\baselineskip}

\Proposition{

    Sei $V$ ein $\K$ Vekorraum, $\scalprod{-}{-}$ ein Skalarprodukt auf $V$. Dann definiert
    $v \rightmapsto \Norm{v} := \sqrt{\scalprod{v}{v}}$ eine Norm auf $V$. Diese Norm wird
    \fat{2-Norm} oder \fat{Euklidische Norm} genannt.
}

\vspace{1\baselineskip}

\Lemma{

    Sei $V$ ein $\K$-Vektorraum und sei $\standardNorm$ die Norm auf $V$. Sei $(v_n)_{n=0}^{\infty}$
    eine bezüglich der Norm $\standardNorm$ konvergierende Folge in $V$ mit Grenzwert $w \in V$. Dann
    gilt \begin{align*}
        \limesninf \Norm{v_n} = \Norm{w}
    \end{align*}
}

\vspace{1\baselineskip}

\Lemma{

    Sei $V$ ein $\K$-Vektorraum und seien $\einsNorm$ und $\zweiNorm$ zwei äquivalente Normen
    auf $V$. Sei $(v_n)_{n=0}^{\infty}$ eine Folge in $V$. Falls die Folge $(v_n)_{n=0}^{\infty}$
    bezüglich der Norm $\einsNorm$ konvergiert, dann konvergiert sie auch bezüglich der Norm
    $\zweiNorm$. Die Grenzwerte sind in diesem Fall gleich.
}

\vspace{1\baselineskip}

\Satz{

    Sei $V$ ein endlichdimensionaler $\K$-Vekorraum. Alle Normen auf $V$ sind zueinander
    äquivalent.
}

\vspace{1\baselineskip}

\Lemma{

    Sei $d \in \N$. Eine Folge in $\K^d$ konvergiert genau dann bezüglich der euklidischen
    Norm, wenn sie koordinatenweise konvergiert.
}

\vspace{1\baselineskip}

\Satz{ (Heine-Borell)

    Sei $d \in \N$. Jede bezüglich der euklidischen Norm beschränkte Folge in $\K^d$ besitzt
    eine konvergierende Teilfolge, und einen Häufungspunkt.
}

\vspace{1\baselineskip}

\Korollar{

    Sei $n \in \N$. Bezüglich der euklidischen Norm ist jede Cauchy-Folge in $\K^n$ konvergent.
}


\pagebreak

\section{Reihen, Funktionenfolgen und Potenzreihen}

\vspace{1\baselineskip}

\subsection{Reihen komplexer Zahlen}

\vspace{1\baselineskip}

\Definition{

    Sei $\aFolge$ eine Folge komplexer Zahlen, und sei $A$ eine komplexe Zahl. Die Notationen
    \begin{align*}
        A = \sum_{n=0}^{\infty} a_n
        \quad \text{   und   } \quad
        A = \limes{N \rightarrow \infty} \sum_{n=0}^{N} a_n
    \end{align*}
    sind gleichbedeutend. Im Zusammenhang, in dem wir uns für den Grenzwert der \fat{Partialsummen}
    der Folge $(a_k)_k$ interessieren, sprechen wir üblicherweise nicht von einer Folge
    $\aFolge$ sondern von der Reihe
    \begin{align*}
        \sum_{n=0}^{\infty} a_n
    \end{align*}
    und Konvergenz der Reihe. Wir nennen $a_n$ das \fat{$n$-te Glied} oder den
    \fat{$n$-ten Summanden} der Reihe. Wir nenne die Reihe $\sum_{k=1}^{\infty} a_k$
    \fat{konvergent}, falls der Grenzwert existiert, wobei wir diesen dann als
    \fat{Wert der Reihe} bezeichnen. Ansonsten nennen wir die Reihe \fat{divergent}.
}

\vspace{1\baselineskip}

\Proposition{

    Falls die Reihe $\sum_{k=1}^{\infty} a_k$ konvergiert, dann ist die Folge $(a_n)_n$
    eine \fat{Nullfolge}, das heisst, es gilt $\limesninf a_n = 0$.
}

\Bemerkung{

    Die \fat{harmonische Reihe} divergiert. Sie ist wie folgt gegeben:
    \begin{align*}
        \sum_{k=1}^{\infty} \frac{1}{k} 
    \end{align*}
}

\vspace{1\baselineskip}

\Lemma{

    Seien $sum_{k=1}^{\infty} a_k$ und $sum_{k=1}^{\infty} b_k$ konvergente Reihen und
    $\alpha,\beta \in \C$. Dann gilt
    \begin{align*}
        \sum_{k=1}^{\infty} (\alpha a_k + \beta b_k)
        = \alpha \sum_{k=1}^{\infty} a_k + \beta \sum_{k=1}^{\infty} b_k
    \end{align*}
    Insbesondere bilden konvergente Reihen einen Vektorraum über $\C$ und der Wert der Reihe
    stellt eine lineare Abbildung auf diesen Vektorraum nach $\C$ dar.
}

\vspace{1\baselineskip}

\Lemma{

    Sei $\sum_{k=0}^{\infty}$ eine Reihe. Für jedes \NinN ist die Reihe $\sum_{k=N}^{\infty} a_k$
    genau dann konvergent, wenn die Reihe $\sum_{k=0}^{\infty} a_k$ konvergent ist. In diesem
    Fall gilt:
    \begin{align*}
        \sum_{k=0}^{\infty} a_k = \sum_{k=1}^{N-1} a_k + \sum_{k=N}^{\infty} a_k
    \end{align*}
}

\vspace{1\baselineskip}

\Lemma{

    Sei $\sum_{n=1}^{\infty} a_n$ eine konvergente Reihe und $(n_k)_k$ eine streng monoton
    wachsende Folge natürlicher Zahlen. Definiere $A_1 = a_1 + \dots + a_{n_1}$ und
    $A_k = a_{n_{k-1}+1} + \dots + a_{n_k}$ für $k \geq 2$. Dann gilt
    \begin{align*}
        \sum_{k=1}^{\infty} A_k = \sum_{n=1}^{\infty} a_n
    \end{align*}
}

\vspace{1\baselineskip}

\Proposition{

    Für eine Reihe $\sum_{k=1}^{\infty} a_k$ mit nicht-negativen Gliedern $a_k \geq 0$ für
    alle $k \in \N$ bilden die Partialsummen $s_n = \sum_{k=1}^{n} a_k$ eine monoton
    wachsende Folge. Falls diese Folge der Partialsummen beschränkt ist, dann konvergiert
    die Reihe $\sum_{k=1}^{\infty} a_k$. Ansonsten gilt
    \begin{align*}
        \sum_{k=1}^{\infty} a_k = \limesninf s_n = \infty
    \end{align*}
}

\vspace{1\baselineskip}

\Korollar{

    Seien $\sum_{k=1}^{\infty} a_k$ und $\sum_{k=1^{\infty} b_k}$ zwei Reihen mit der
    Eigenschaft $0 \leq a_k \leq b_k$ für alle $k \in \N$. Dann gilt $\sum_{k=1}^{\infty} a_k
    \leq \sum_{k=1}^{\infty} b_k$ und insbesondere gelten die Implikationen
    \begin{align*}
        \sum_{k=1}^{\infty} b_k \ \text{ konvergent } &\Longrightarrow \sum_{k=1}^{\infty} a_k \ \text{ konvergent}
        \\
        \sum_{k=1}^{\infty} a_k \ \text{ divertent } &\Longrightarrow \sum_{K=1}^{\infty} b_k \ \text{ divergent}
    \end{align*}
    Diese beiden Implikationen treffen auch dann zu, wenn $0 \leq a_n \leq b_n$ nur für
    alle hinreichenden grossen $n \in \N$ gilt. Man nennt die Reihe $\sum_{k=1}^{\infty} b_k$
    eine Majorante der Reihe $\sum_{K=1}^{\infty} a_k$, und die letztere eine Minorante
    der Reihe $\sum_{K=1}^{\infty} b_k$. Daher spricht man auch vom \fat{Majoranten-}
    und dem \fat{Minorantenkriterium}.
}

\vspace{1\baselineskip}

\Proposition{ (Verdichtungskriterium)

    Sei $\aFolge$ eine monoton fallende Folge nichtnegativer reeller Zahlen. Dann gilt:
    \begin{align*}
        \sum_{n=0}^{\infty} a_n \ \text{ konvergiert} \Longleftrightarrow \sum_{n=0}^{\infty} 2^n a_{2^n} \ \text{ konvergiert}
    \end{align*}
}

\vspace{1\baselineskip}

\Definition{

    Wir sagen, dass eine Reihe $\sum_{n=1}^{\infty} a_n$ mit komplexen Summanden
    \fat{absolut konvergiert}, falls die Reihe $\sum_{n=1}^{\infty} \abs{a_n}$ konvergiert.
    Die Reihe $\sum_{n=1}^{\infty} a_n$ ist \fat{bedingt konvergent}, falls sie
    konvergiert, aber nicht absolut konvergiert.
}

\vspace{1\baselineskip}

\Proposition{

    Sei $\zFolge$ eine Folge in $\C$. Falls $\sum_{n=0}^{\infty} z_n$ absolut konvergiert,
    dann konvergiert $\sum_{n=0}^{\infty}$ und es gilt die verallgemeinerte Dreiecksungleichung
    \begin{align*}
        \abs{\sum_{n=0}^{\infty} z_n} \leq \sum_{n=0}^{\infty} \abs{z_n}
    \end{align*}
}

\vspace{1\baselineskip}

\Beispiel{

    Die \fat{alternierende harmonische Reihe} konvergiert bedingt.
}

\vspace{1\baselineskip}

\Satz{ (Riemann'scher Umordnungssatz)

    Sei $\sum_{n=1}^{\infty} a_n$ eine bedingte konvergente Reihe mit reellen Gliedern,
    und sei $A \in \R$. Es existiert eine Bijektion $\varphi: \N \rightarrow \N$, so
    dass folgendes gilt:
    \begin{align*}
        A = \sum_{n=0}^{\infty} a_{\varphi (n)}
    \end{align*}
}

\vspace{1\baselineskip}

\Definition{

    Für eine Folge $\aFolge$ nichtnegativer Zahlen bezeichnen wir die Reihe
    $\sum_{n=1}^{\infty} (-1)^{n+1} a_n$ als eine \fat{alternierende} Reihe.
}

\pagebreak

\Proposition{ (Leibnitz-Kriterium)

    Sei $\aFolge$ eine monoton fallende Folge nichtnegativer reeller Zahlen, die gegen
    Null konvergiert. Dann konvergiert die alternierende Reihe $\sum_{k=0}^{\infty} (-1)^k a_k$
    und es gilt für alle \ninN
    \begin{align*}
        \sum_{k=0}^{2n-1} (-1)^k a_k \leq \sum_{k=0}^{\infty} (-1)^k a_k \leq \sum_{k=0}^{2n} (-1)^k a_k
    \end{align*}
}

\vspace{1\baselineskip}

\Satz{ (Cauchy-Kriterium)

    Die Reihe komplexer Zahlen $\sum_{k=1}^{\infty} a_k$ konvergiert genau dann, wenn es
    zu jedem $\epsilon > 0$ ein \NinN gibt, so dass für $n \geq m \geq N$ folgendes
    erfüllt ist:
    \begin{align*}
        \abs{\sum_{k=m}^{n} a_k} < \epsilon
    \end{align*}
}


\vspace{2\baselineskip}

\subsection{Absolute Konvergenz}

\vspace{1\baselineskip}

Die Definition von absoluter Konvergenz, so wie die erste Proposition, sind im vorherigen
Kapitel zu finden.

\vspace{2\baselineskip}

\Korollar{ (Majorantenkriterium von Weierstrass)

    Sei $\aFolge$ eine komplexe und $(b_n)_n$ eine reelle Folge mit $\abs{a_n} \leq b_n$ für
    alle hinreichend grossen $n \in \N$. Falls $\sum_{n=1}^{\infty} b_n$ konvergiert,
    dann ist $\sum_{n=1}^{\infty} a_n$ absolut konvergent, und daher auch konvergent. 
}

\vspace{1\baselineskip}

\Korollar{ (Cauchy-Wurzelkriterium)

    Sei $\aFolge$ eine Folge komplexer Zahlen und
    \begin{align*}
        \alpha = \limsupninf \sqrt[n]{\abs{a_n}} \in \R \cup \geschwungeneklammer{\infty}
    \end{align*}
    ("Grösster Häufungspunkt"). Dann gilt:

    Falls $\alpha < 1$, so konvergiert $\sum_{n=1}^{\infty} a_n$ absolut.

    Falls $\alpha > 1$, so divergiert $\sum_{n=1}^{\infty} a_n$.
    
}

\vspace{1\baselineskip}

\Korollar{ (D'Alemberts Quotientenkriterium)

    Sei $\aFolge$ eine Folge komplexer Zahlen mit $a_n \neq 0$ für alle n $n \in \N$,
    so dass
    \begin{align*}
        \alpha = \limesninf \frac{\abs{a_{n+1}}}{\abs{a_n}}
    \end{align*}
    existiert. Dann gilt:
    
    Falls $\alpha < 1$, dann konvergiert $\sum_{n=1}^{\infty} a_n$ absolut.

    Falls $\alpha > 1$, dann divergiert $\sum_{n=1}^{\infty} a_n$.
}

\vspace{1\baselineskip}

\Satz{ (Umordnungssatz für absolut konvergente Reihen)

    Sei $\sum_{n=0}^{\infty} a_n$ eine absolut konvergente Reihe komplexer Zahlen.
    Sei $\varphi: \N \rightarrow \N$ eine Bijektion. Dann ist die Reihe
    $\sum_{n=0}^{\infty} a_{\varphi (n)}$ absolut konvergent und es gilt
    \begin{align*}
        \sum_{n=0}^{\infty} a_n = \sum_{n=0}^{\infty} a_{\varphi (n)}
    \end{align*}
}

\vspace{1\baselineskip}

\Satz{ (Produktsatz)

    Seien $\sum_{n=0}^{\infty} a_n$ und $\sum_{n=0}^{\infty} b_n$ absolut konvergente Reihen,
    und sei eine Bijektion $\alpha : \N \rightarrow \N \times \N$ durch
    $\alpha (n) = (\varphi (n) , \psi (n))$ gegeben. Dann gilt
    \begin{align*}
        \klammer{\sum_{n=0}^{\infty} a_n} \klammer{\sum_{n=0}^{\infty} b_n}
        = \sum_{n=0}^{\infty} a_{\varphi (n)} b_{\psi (n)}
    \end{align*}
    und die Reihe auf der rechten Seite konvergiert absolut.
}

\vspace{1\baselineskip}

\Korollar{ (Cauchy-Produkt)

    Falls $\sum_{n=0}^{\infty} a_n$ und $\sum_{n=0}^{\infty} b_n$ absolut konvergente
    Reihen mit komplexen Gliedern sind, dann gilt:
    \begin{align*}
        \sum_{n=0}^{\infty} \klammer{\sum_{k=0}^{n} a_{n-k} b_k}
        = \klammer{\sum_{n=0}^{\infty} a_n} \klammer{\sum_{n=0}^{\infty} b_n}
    \end{align*}
    wobei die Reihe $\sum_{n=0}^{\infty} \klammer{\sum_{k=0}^{n} a_{n-k} b_k}$
    absolut konvergent ist.
}


\vspace{2\baselineskip}

\subsection{Konvergenz von Funktonenfolgen}

\vspace{1\baselineskip}

\Definition{

    Sei $D$ eine Menge, sei $\fFolge$ eine Folge von Funktionen $f_n : D \rightarrow \C$,
    und sei $f: D \rightarrow \C$ eine Funktion. Wir sagen die Folge $\fFolge$ konvergiert
    \fat{punktweise} gegen $f$, falls für jedes $x \in D$ die Folge komplexer Zahlen
    $(f_n (x))_{n=0}^{\infty}$ gegen $f(x)$ konvergiert. Wir bezeichnen die Funktion
    $f$ in dem Fall als den \fat{punktweisen Grenzwert} der Folge $\fFolge$.
}

\vspace{1\baselineskip}

\Definition{

    Sei $D$ eine Menge, sei $\fFolge$ eine Folge von Funktionen $f_n : D \rightarrow \C$,
    und sei $f: D \rightarrow \C$ eine Funktion. Wir sagen die Folge $\fFolge$ konvergiert
    \fat{gleichmässig} gegen $f$, falls für jedes $\epsilon > 0$ ein \NinN existiert, so
    dass für alle $n \geq N$ und alle $x \in D$ die folgende Abschätzung gilt.
    \begin{align*}
        \abs{f_n (x) - f(x)} < \epsilon
    \end{align*}
}

\vspace{1\baselineskip}

\Satz{

    Sei $D \subset \C$ und sei $\fFolge$ eine Folge stetiger Funktionen
    $f_n : D \rightarrow \C$ die gleichmässig gegen $f: D \rightarrow \C$ konvergiert.
    Dann ist $f$ stetig.
}

\vspace{1\baselineskip}

\Satz{

    Sei $[a,b]$ ein Intervall und sei $(f_n : [a,b] \rightarrow \C)_{n=0}^\infty$ eine
    Folge integrierbarer Funktionen die gleichmässig gegen eine Funktion
    $f: [a,b] \rightarrow \C$ konvergiert. Dann ist $f$ integrierbar und es gilt:
    \begin{align*}
        \intab f \ dx \ = \ \limesninf \intab f_n \ dx
    \end{align*}
}


\pagebreak

\subsection{Potenzreihen}

\vspace{1\baselineskip}

\Definition{

    Sei $K$ ein Körper. Eine \fat{Potenzreihe} mit Koeffizienten in $\K$ ist eine Folge
    $\aFolge$ in $\K$, suggestiv geschrieben als Reihe
    \begin{align*}
        f(T) = \sum_{n=0}^{\infty} a_n T^n
    \end{align*}
    wobei $T$ als \fat{Variable} bezeichnet wird und $a_n \in \K$ als \fat{Koeffizient}.
    Addition und Multiplikation sind wie folgt definiert.
    \begin{align*}
        \sum_{n=0}^{\infty} a_n T^n + \sum_{n=0}^{\infty} b_n T^n
        &= \sum_{n=0} (a_n + b_n) T^n
        \\
        \klammer{\sum_{n=0}^{\infty} a_n T^n} \klammer{\sum_{n=0}^{\infty} b_n T^n}
        &= \sum_{n=0}^{\infty} \klammer{\sum_{k=0}^{n} a_{n-k} b_k} T^n
    \end{align*}
    Wir schreiben $K$\textlbrackdbl$T$\textrbrackdbl \ für den dadurch entstehenden Ring
    von Potenzreihen mit Koeffizienten in $\K$.
}

\vspace{1\baselineskip}

\Definition{

    Sei $f(T) = \sum_{n=0}^{\infty} a_n T^n \in \C $\textlbrackdbl$ T $\textrbrackdbl \
    eine formale Potenzreihe mit komplexen Koeffizienten. Der \fat{Konverenzradius} von $f$
    ist die Zahl $R \in \R_{\geq 0}$ oder das Symbol $R = \infty$, definiert duch
    \begin{align*}
        \rho = \limsupninf \sqrt[n]{\abs{a_n}}
        \quad \quad \text{    und    } \quad \quad
        R = \begin{cases}
            0 \ \ \text{   falls } \rho = \infty
            \\
            \rho^{-1} \ \ \text{   falls } 0 < \rho < \infty
            \\
            \infty \ \ \text{   falls } \rho = 0
        \end{cases}
    \end{align*}
}

\vspace{1\baselineskip}

\Bemerkung{

    Sei $\sum_{n=0}^{\infty} a_n T^n \in \C$\textlbrackdbl$T$\textrbrackdbl eine Potenzreihe
    mit positivem Konvergenzradius $R$, und sei $r$ eine positive reelle Zahl mit $r < R$.
    Schreibe $D = \overline{B(0,r)}$ und $f_n : D \rightarrow \C$ für die Funktion gegeben
    durch $f_n(z) = \sum_{k=0}^n a_n z^k$.

    (1) Die Reihe $\sum_{n=0}^{\infty} a_n z^n$ konvergiert absolut für alle \zinC
    mit $\abs{z} < R$, und divergiert für alle \zinC mit $\abs{z} > R$.

    (2) Für $z \in B(0,R)$, setze $f(z) = \sum_{n=0}^{\infty} a_n z^n$. Die Folge von
    Funktionen $\fFolge$ konvergiert gleichmässig gegen die Funktion $f|_D$ auf $D$.

    Insbesondere definiert die Potenzreihe die stetige Abbildung $f: B(0,R) \rightarrow \C$.
}

\vspace{1\baselineskip}

\Lemma{

    Seien $D \subseteq \R$, $f_n : D \rightarrow \R$ für alle $n=0,1,2,\dots$
    (Anstatt $\R$ kann man auch $\C$ nehmen) und $f: D \rightarrow \R$.
    Angenommen $f_n$ ist stetig für alle $n \geq 0$ und $\forall \epsilonnull \ \exists N \in \N$
    mit $\Norm{f_n - f}_{\infty} < \epsilonnull \ \forall n \geq N$. (i.e.
    $\limesninf f_n = f$ bezüglich der Norm $\standardNorm_{\infty}$)
    Dann ist $f$ stetig.
}

\vspace{1\baselineskip}

\Lemma{

    Sei $\sum_{n=0}^{\infty} a_n T^n$ eine Potenzreihe mit $a_n \neq 0$ für alle $n \in \N$.
    Der Konvergenzradius $R$ ist gegeben durch
    \begin{align*}
        R = \limesninf \frac{a_n}{a_{n+1}}
    \end{align*}
    falls dieser Grenzwert existiert.
}

\vspace{1\baselineskip}

\Proposition{

    Sei $R \geq 0$, und seien $f(T) = \sum_{n=0}^{\infty}$ und $g(T) = \sum_{n=0}^{\infty} b_n T^n$
    Potenzreihen mit Kovergenzradius mindestens $R$. Dann haben die Summe $f(T) + g(T)$
    und das Produkt $f(T) g(T)$ Konverenzradien mindestens $R$.
}

\vspace{1\baselineskip}

\Lemma{

    Sei $[a,b] \subseteq \R$ ein Intervall mit $a < b$. Sei $\fFolge$ eine Folge
    stetiger Funktionen auf $[a,b]$ und $f: [a,b] \rightarrow \R$ stetig mit
    $\limesninf f_n = f$ bezüglich $\standardNorm_{\infty}$. Dann gilt
    \begin{align*}
        \limesninf \intab f_n \ dx = \intab f \ dx
    \end{align*}
}

\vspace{1\baselineskip}

\Satz{ (Abel'scher Grenzwertsatz)

    Sei $\sum_{n=0}^{\infty} a_n T^n \in \C$\textlbrackdbl$T$\textrbrackdbl eine Potenzreihe
    mit positivem Konvergenzradius $R$, derart, dass die Reihe $\sum_{n=0}^{\infty} a_n R^n$
    konvergiert. Dann gilt
    \begin{align*}
        \limes{\stackrel{t \rightarrow R}{t < R}} \sum_{n=0}^{\infty} a_n t^n = \sum_{n=0}^{\infty} a_n R^n
    \end{align*}
    Anders ausgedrückt: Dann ist die Funktion $f: (-R,R] \rightarrow \C$ gegeben durch
    $f(t) = \sum_{n=0}^{\infty} a_n t^n$ ist stetig bei $R$.
}

\vspace{1\baselineskip}

\Satz{

    Sei $f(T) = \sum_{n=0}^{\infty} a_n T^n \in \C$\textlbrackdbl$T$\textrbrackdbl eine
    Potenzreihe mit Konverenzradius $R>0$. Dann hat die Potenzreihe
    \begin{align*}
        F(T) = \sum_{n=0}^{\infty} \frac{a_n}{n+1} T^{n+1}
    \end{align*}
    denselben Konverenzradius $R$, und es gilt
    \begin{align*}
        \intab f(x) dx = F(b) - F(a) \ \ \ \ \forall a,b \in (-R,R)
    \end{align*}
}

\vspace{1\baselineskip}


\vspace{2\baselineskip}

\subsection{Trigonometrische Funktionen}

\vspace{1\baselineskip}

\Definition{

    Die \fat{Exponentialreihe} ist die Formale Potenzreihe $\sum_{n=0}^{\infty} \frac{T^n}{n!}
    \in \C$\textlbrackdbl$T$\textrbrackdbl. Aus dem Quotientenkriterium folgt, dass der
    Konvergenzradius $R$ unendlich ist. Das heisst, sie konvergiert absolut für alle \zinC
    und die Funktion $f: \C \rightarrow \C$ gegeben durch $f(z) = \sum_{n=0}^{\infty} \frac{z^n}{n!}$
    ist stetig.
}

\vspace{1\baselineskip}

\Proposition{

    Die obige Definition gilt auch für alle $x \in \R$.
}

\vspace{1\baselineskip}

\Korollar{

    Für $a \leq b \in \R$ gilt:
    \begin{align*}
        \intab \exp (x) dx = \exp (b) - \exp (a)
    \end{align*}
}

\vspace{1\baselineskip}

\Definition{

    Die \fat{komplexe Exponentialabbildung} ist die Funktion $f: \C \rightarrow \C$ gegeben durch
    \begin{align*}
        \exp (z) = \sum_{k=0}^{\infty} \frac{1}{n!} z^n
    \end{align*}
    für alle $z \in \C$. Für eine positive reelle Zahl $a \in \R_{>0}$ und \zinC schreiben wir
    $a^z = \exp (z \log (a))$, und insbesondere auch $e^{z} = \exp (z)$ für alle $z \in \C$.
}

\pagebreak

\Satz{

    Die komplexe Exponentialabbildung $\exp : \C \rightarrow \C$ ist stetig. Des Weiteren gilt
    \begin{align*}
        \exp (z+w) = \exp(z) \exp (w)
        \quad \text{     und     } \quad
        \abs{\exp (z)} = \exp (\text{Re} (z))
    \end{align*}
    für alle $z,w \in \C$. Insbesondere gilt $\abs{\exp (i y)} = 1$ für alle $y \in \R$.
}

\vspace{1\baselineskip}

\Definition{

    Wir definieren die \fat{Sinusfunktion} und die \fat{Kosinusfunktion} bei \zinC durch
    \begin{align*}
        \sin (z) = \sum_{n=0}^{\infty} \frac{(-1)^n}{(2n+1)!} z^{2n+1}
        \quad \text{     und     } \quad
        \cos (z) = \sum_{n=0}^{\infty} \frac{(-1)^n}{(2n)!} z^{2n}
    \end{align*}
    Die beiden Potenzreihen konvergieren auf ganz $\C$ und definieren stetige Funktionen.
}

\vspace{1\baselineskip}

\Bemerkung{

    Die Sinusfunktion ist \fat{ungerade}, das heisst es gilt $\sin (-z) = - \sin (z)$,
    und die Kosinusfunktion ist \fat{gerade}, es gilt $\cos (-z) = \cos (z)$ für alle
    $z \in \C$.
}

\vspace{1\baselineskip}

\Satz{

    Für alle \zinC gelten folgende Relationen zwischen Exponential, Sinus- und Kosinusfunktion.
    \begin{align*}
        \exp (iz) &= \cos (z) + i \sin (z)
        \\
        \sin (z) &= \frac{\exp (iz) - \exp (-iz)}{2i}
        \\
        \cos (z) &= \frac{\exp (iz) + \exp (-iz)}{2}
    \end{align*}
    Insbesondere gelten für alle $z,w \in \C$ die folgenden Additionsformeln:
    \begin{align*}
        \sin (z+w) &= \sin(z) \cos(w) + \cos(z) \sin(w)
        \\
        \cos (z+w) &= \cos(z) \cos(w) - \sin(z) \sin(w)
    \end{align*}
}

\vspace{1\baselineskip}

\Bemerkung{

    Im Fall $z=w \in \C$ ergeben sich insbesondere \fat{Winkelverdopplungsformeln}
    \begin{align*}
        \sin(2z) = 2 \sin(z) \cos(z)
        \quad \text{     und     } \quad
        \cos(2z) = \cos(z)^2 - \sin(z)^2
    \end{align*}
    Im Fall $w=-z$ folgt die \fat{Kreisgleichung} für Sinus und Kosinus für alle $z \in \C$.
    \begin{align*}
        1 = \cos(z)^2 + \sin(z)^2
    \end{align*}
}

\vspace{1\baselineskip}

\Satz{

    Es gibt genau eine Zahl $\pi \in (0,4)$ mit $\sin(\pi) = 0$. Für diese Zahl gilt
    \begin{align*}
        \exp (2 \pi i) = 1
    \end{align*}
}

\vspace{1\baselineskip}

\Korollar{

    Für alle \zinC gelten
    \begin{align*}
        \sin (z + \frac{\pi}{2}) &= \cos (z) \hspace{100pt} \cos(z + \frac{\pi}{2}) = - \sin(z)
        \\
        \sin (z+\pi) &= - \sin(z) \hspace{96pt} \cos(z + \pi) = - \cos(z)
        \\
        \sin(z+2\pi) &= \sin(z) \hspace{99pt} \cos(z+2\pi) = \cos(z)
    \end{align*}
}

\vspace{1\baselineskip}

\Bemerkung{

    Der Sinus und der Kosinus sind periodisch mit Periode $2\pi$.
}

\vspace{1\baselineskip}

\Proposition{

    Mit Hilfe der komplexen Exponentialfunkion können wir komplexe Zahlen in
    \fat{Polarkoordinaten} ausdrücken, das heisst, in der Form
    \begin{align*}
        z = r \exp(i\theta) = r \cos (\theta) + i r \sin(\theta)
    \end{align*}
    wobei $r$ der Abstand vom Ursprung $0 \in \C$ zu $z$ ist, also der Betrag $r = \abs{z}$
    von $z$, und $\theta$ der Winkel, der zwischen den Halbgeraden $\R_{\geq 0}$ eingeschlossen
    ist. Falls $z \neq 0$ gilt, so ist der Winkel $\theta$ eindeutig bestimmt, und wird als
    \fat{Argument} von $z$ bezeichnet und als $\theta = \arg (z)$ geschrieben. Die Menge der
    komplexen Zahlem mit Absolutbetrag Eins ist demnach
    \begin{align*}
        \eS^1 = \geschwungeneklammer{z \in \C \ | \ \abs{z} = 1}
        = \geschwungeneklammer{\exp(i \theta) \ | \ \theta \in [0,2 \pi)}
    \end{align*}
    und wird als der \fat{Einheitskreis} in $\C$ bezeichnet.
}

\vspace{1\baselineskip}

\Proposition{ (Existenz von Polarkoordinaten)

    Für alle $z \in \C^{\times}$ existieren eindeutig bestimmte reelle Zahlen $r > 0$ und
    $\theta \in [0,2 \pi)$ mit $z = r \exp (i \theta)$
}

\vspace{1\baselineskip}

\Proposition{

    In Polarkoordinaten lässt sich die Multiplikation auf $\C$ neu interpretieren. Sind
    $z = r \exp (i \varphi)$ und $w = s \exp (i \psi)$ komplexe Zahlen, dann gilt
    \begin{align*}
        z w = r s \exp(i(\varphi + \psi))
    \end{align*}
    Bei Multiplikation von komplexen Zahlen multiplizieren sich die Längen der Vektoren und
    addieren sich die Winkel.
}

\vspace{1\baselineskip}

\Definition{

    Die \fat{Tangensfunktion} und die \fat{Kotangensfunktion} sind durch
    \begin{align*}
        \tan(z) = \frac{\sin(z)}{\cos(z)}
        \quad \text{     und     } \quad
        \cot(z) = \frac{\cos(z)}{\sin(z)}
    \end{align*}
    definiert, für alle \zinC mit $\cos(z) \neq 0$, beziehungsweise $\sin(z) \neq 0$.
}

\vspace{1\baselineskip}

\Definition{

    Der \fat{Sinus Hyperbolicus} und der \fat{Kosinus Hyperbolicus} sind die durch
    die Potenzreihen
    \begin{align*}
        \sinh (z) = \sum_{k=0}^{\infty} \frac{z^{2k+1}}{(2k+1)!}
        \quad \quad \text{     und     } \quad \quad
        \cosh (z) = \sum_{k=0}^{\infty} \frac{z^{2k}}{(2k)!}
    \end{align*}
    definierten Funktionen. Es gilt
    \begin{align*}
        \sinh (z) = - i \sin(iz) = \frac{e^z - e^{-z}}{2}
        \quad \quad \text{     und     } \quad \quad
        \cosh (z) = \cos(iz) = \frac{e^z + e^{-z}}{2}
    \end{align*}
    und also $\exp(z) = \cosh(z) + \sinh(z)$ für alle $z \in \C$. Der \fat{Tangens} und der
    \fat{Kotangens Hyperbolicus} sind durch
    \begin{align*}
        \tanh(z) = \frac{\sinh(z)}{\cosh(z)} = \frac{e^z - e^{-z}}{e^z + e^{-z}}
        \quad \quad \text{     und     } \quad \quad
        \coth(z) = \frac{\cosh(z)}{\sinh(z)} = \frac{e^z - e^{-z}}{e^z - e^{-z}}
    \end{align*}
    für alle \zinC mit $\cosh(z) \neq 0$, beziehungsweise mit $\sinh(z) \neq 0$. Die Funktionen
    $\sinh$ und $\tanh$ sind ungerade und $\cosh$ ist gerade. Es gelten folgende Additionsformeln
    \begin{align*}
        \sinh(z+w) &= \sinh(z) \cosh(w) + \cosh(z) \sinh(w)
        \\
        \cosh(z+w) &= \cosh(z) \cosh(w) + \sinh(z) \sinh(w)
    \end{align*}
    für alle $z,w \in \C$, sowie die Hyperbelgleichung
    \begin{align*}
        \cosh^2(z) - \sinh^2 (z) = 1
    \end{align*}
    für alle $z \in \C$. Des Weiteren gilt
    \begin{align*}
        \intab \sinh(x) dx = \cosh(b) - \cosh(a)
        \quad \quad \text{     und     } \quad \quad
        \intab \cosh(x) dx = \sinh(b) - \sinh(a)
    \end{align*}
    für alle $a<b \in \R$.
}


\pagebreak

\section{Differentialrechnung}

\vspace{1\baselineskip}

\subsection{Die Ableitung}

\vspace{1\baselineskip}

In diesem Kapitel definieren wir eine allgemeine Teilmenge $D \subseteq \R$ ohne isolierte
Punkte. Das heisst, jedes Element $x \in D$ ist ein Häufungspunkt von $D$.
Ein Beispiel für solch eine Menge $D$ sind Intervalle.
Falls nicht explizit anders erwähnt, sind alle Funktionen als reellwertig vorausgesetzt.

\vspace{2\baselineskip}

\Definition{

    Sei $f: D \rightarrow \R$ eine Funktion und $x_0 \in D$. Wir sagen, dass $f$ bei $x_0$
    \fat{differenzierbar} ist, falls der Grenzwert
    \begin{align*}
        f'(x_0) = \limes{\stackrel{x \rightarrow x_0}{x \neq x_0}} \frac{f(x) - f(x_0)}{x - x_0}
        = \limes{\stackrel{h \rightarrow 0}{h \neq 0}} \frac{f(x_0 + h) - f(x_0)}{h}
    \end{align*}
    existiert. In diesem Fall nennen wir $f'(x_0)$ die \fat{Ableitung} von $f$ bei $x_0$.
    Falls $f$ bei jedem Häufungspunkt von $D$ in $D$ differenzierbar ist, dann sagen wir
    auch, dass $f$ auf $D$ \fat{differenzierbar} ist und nennen die daraus entstehende
    Funktion $f': D \rightarrow \R$ die \fat{Ableitung} von $f$.
    Alternative Notation für die Ableitung von $f$ sind $\frac{\partial}{\partial x} f$,
    $\frac{df}{dx}$ oder auch $Df$. Falls $x_0 \in D$ ein rechtseitiger Häufungspunkt
    von $D$ ist, dann ist $f$ bei $x_0$ \fat{rechtseitig differenzierbar}, wenn die
    \fat{rechtseitige Ableitung}
    \begin{align*}
        f_+' (x_0) = \limes{\stackrel{x \rightarrow x_0}{x > x_0}} \frac{f(x) - f(x_0)}{x - x_0}
        = \limes{\stackrel{h \rightarrow 0}{h > 0}} \frac{f(x_0 + h) - f(x_0)}{h}
    \end{align*}
    existiert. \fat{Linksseitige Differenzierbarkeit} und die \fat{Linksseitige Ableitung}
    $f_-' (x_0)$ werden analog über die Bewegung $x \rightarrow x_0$ mit $x < x_0$ definiert.
}

\vspace{1\baselineskip}

\Definition{

    Eine \fat{affine} Funktion ist eine Funktion der Form $x \mapsto sx + r$, für reelle
    Zahlen $s$ und $r$. Der Graph einer affinen Funktion ist eine nichtvertikale
    \fat{Gerade} in $\R^2$. Der Parameter $s$ in der Gleichung $y = sx + r$ wird die
    \fat{Steigung} genannt. Ist $f: D \rightarrow \R$ differenzierbar an der Stelle
    $x_0 \in D$, so wird die Funktion $x \mapsto f'(x_0)(x-x_0)$ \fat{lineare Approximation}
    von $f$ bei $x_0$ oder \fat{Tangente} von $f$ bei $x_0$ genannt.
}

\vspace{1\baselineskip}

\Definition{

    Sei $f: D \rightarrow \R$ eine Funktion. Wir definieren die \fat{höhere Ableitungen}
    von $f$, sofern sei existieren, durch
    \begin{align*}
        f^{(0)} = f \ , \quad f^{(1)} = f' \ , \quad f^{(2)} = f'' \ , \ \dots \ , \quad f^{(n+1)} = (f^{n})'
    \end{align*}
    für alle $n \in \N$. Falls $f^{(n)}$ für ein \ninN existiert, heisst $f$ \fat{n-mal differenzierbar}.
    Falls die $n$-te Ableitung $f^{n}$ zusätzlich stetig ist, heisst $f$ \fat{n-mal stetig
    differenzierbar}. Die Menge der $n$-mal stetig differenzierbaren Funktionen auf $D$
    bezeichnen wir mit $C^n (D)$. Rekursiv definieren wir für $n \geq 1$
    \begin{align*}
        C^n (D) = \geschwungeneklammer{f:D \rightarrow \R \ | \ f \text{ ist differenzierbar und } f' \in C^{n-1} (D)}
    \end{align*}
    und sagen $f \in C^n (D)$ sei von \fat{Klasse $C^n$}. Schliesslich definieren wir
    \begin{align*}
        C^{\infty} (D) = \bigcap_{n=0}^{\infty} C^n (D)
    \end{align*}
    und bezeichnen Funktionen $f \in C^{\infty} (D)$ als \fat{glatt} oder von \fat{Klasse $C^\infty$}
}

\vspace{1\baselineskip}

\Bemerkung{

    Sei $f:D \rightarrow \R$ eine Funktion. Ist $f$ ableitbar bei $x_0 \in D$, dann ist
    $f$ stetig bei $x_0$.
}

\pagebreak

\Proposition{

    Seien $f,g: D \rightarrow \R$ Funktionen und ableitbar an der Stelle $x_0 \in D$.
    Dann sind $f+g$ und $f \cdot g$ ableitbar bei $x_0$ und es gilt:
    \begin{align*}
        (f + g)' (x_0) &= f' (x_0) + g' (x_0)
        (f \cdot g)' (x_0) &= f' (x_0) \cdot g(x_0) + f (x_0) \cdot g(x_0)
    \end{align*}
}

\vspace{1\baselineskip}

\Korollar{

    Seien $f,g : D \rightarrow \R$ $n$-mal differenzierbar. Dann sind $f+g$ und $f \cdot g$
    ebenso $n$-mal differenzierbar und es gilt $f^{(n)} + g^{(n)} = (f+g)^{(n)}$ sowie
    \begin{align*}
        (fg)^{(n)} = \sum_{k=0}^{n} \binom{n}{k} f^{(k)} g^{(n-k)}
    \end{align*}
    Insbesondere ist jedes skalare Vielfache $n$-mal differenzierbar und $(\alpha f)^{(n)}
    = \alpha f^{(n)}$ für alle $\alpha \in \R$.
}

\vspace{1\baselineskip}

\Korollar{

    Polynomfunktionen sind auf ganz $\R$ differenzierbar.
}

\vspace{1\baselineskip}

\Satz{ (Kettenregel)

    Seien $D,E \subseteq \R$ Teilmengen und sei $x_0 \in D$ ein Häufungspunkt. Sei
    $f: D \rightarrow E$ ein bei $x_0$ differenzierbare Funktion, so dass $y_0 = f(x_0)$
    ein Häufungspunkt von $E$ ist, und sei $g: E \rightarrow \R$ eine bei $y_0$
    differenzierbare Funktion. Dann ist $g \circ f : D \rightarrow \R$ in $x_0$
    differenzierbar und
    \begin{align*}
        (g \circ f)' (x_0) = g' (f(x_0)) f'(x_0)
    \end{align*}
}

\vspace{1\baselineskip}

\Korollar{

    Seien $D,E \subseteq \R$ Teilmengen, so dass jeder Punkt in $D$ respektive $E$ ein
    Häufungspunkt von $D$ respektive $E$ ist. Seien $f: D \rightarrow E$ und
    $g: E \rightarrow \R$ beides $n$-mal differenzierbare Funktionen. Dann ist
    $g \circ f : D \rightarrow \R$ auch $n$-mal differenzierbar.
}

\vspace{1\baselineskip}

\Korollar{ (Quotientenregel)

    Sei $d \subseteq \R$ eine Teilmenge, $x_0 \in D$ ein Häufungspunkt und seien
    $f,g : D \rightarrow \R$ bei $x_0$ differenzierbar. Falls $g(x_0) \neq 0$ ist,
    dann ist auch $\frac{f}{g}$ bei $x_0$ differenzierbar und es gilt:
    \begin{align*}
        \klammer{\frac{f}{g}}' (x_0) = - \frac{g(x_0) f' (x_0) - f(x_0) g'(x_0)}{g(x_0)^2}
    \end{align*}
}

\vspace{1\baselineskip}

\Satz{

    Seien $D , E \subseteq \R$ Teilmengen und sei $f: D \rightarrow E$ eine stetige,
    bijektive Abbildung, deren inverse Abbildung $f^{-1} : E \rightarrow D$ ebenfalls
    stetig ist. Falls $f$ in dem Häufungspunkt $x_0 \in D$ differenzierbar ist und
    $f'(x_0) \neq 0$ gilt, dann ist $f^{-1}$ in $y_0 = f(x_0)$ differenzierbar und es gilt
    \begin{align*}
        (f^{-1})'(y_0) = \frac{1}{f' (x_0)}
    \end{align*}
}


\pagebreak

\subsection{Zentrale Sätze der Differentialrechnung}

\vspace{1\baselineskip}

\Definition{
 
    Sei $D \subseteq \R$ eine Teilmenge und $x_0 \in D$. Wir sagen, dass eine Funktion
    $f: D \rightarrow \R$ in $x_0$ ein \fat{lokales Maximum} hat, falls es ein $\delta >0$ gibt,
    so dass für alle $x \in D \cap (x_0 - \delta , x_0 + \delta)$ gilt $f(x) \leq f(x_0)$.
    Falls es sich um eine strikte Ungleichung handelt, also $f(x) < f(x_0)$, dann hat $f$ in
    $x_0$ ein \fat{isoliertes lokales Maximum}. Der Wert $f(x_0)$ wird auch ein \fat{lokaler
    Maximalwert} von $f$ genannt. Ein \fat{lokales Minimum}, ein \fat{isoliertes lokales Minimum}
    und ein \fat{lokaler Minimalwert} von $f$ sind analog definiert. Des Weiteren nennen wir
    $x_0$ ein \fat{lokales Extremum} von $f$ und $f(x_0)$ einen \fat{lokalen Extremwert} von $f$,
    falls $f$ ein lokales Minimum oder ein lokales Maximum in $x_0$ hat.
}

\vspace{1\baselineskip}

\Proposition{

    Sei $D \subseteq \R$ eine Teilmenge und $f$ eine reellwertige Funktion auf $D$.
    Angenommen $f$ ist in einem lokalen Extremum $x_0 \in D$ differenzierbar und $x_0$
    ist sowohl ein rechtsseitiger als auch ein linksseitiger Häufungspunkt von $D$.
    Dann gilt $f'(x_0) = 0$.
}

\vspace{1\baselineskip}

\Korollar{

    Sei $I \subseteq \R$ ein Intervall und $f: I \rightarrow \R$ eine Funktion. Sei $x_0 \in I$
    ein lokales Extremum von $f$. Dann ist mindestens eine der folgenden Aussagen wahr.

    (1) $x_0$ ist ein Randpunkt von $I$.

    (2) $f$ ist nicht ableitbar bei $x_0$.

    (3) $f$ ist bei $x_0$ differenzierbar und $f'(x_0) = 0$. 

    Insbesondere sind alle lokalen Extrema einer differenzierbaren Funktion auf einem offenen
    Intervall Nullstellen der Ableitung.
}

\vspace{1\baselineskip}

\Satz{ (Mittelwertsatz, Rolle)

    Seien $a<b \in \R$ und $f: [a,b] \rightarrow \R$ eine stetige Funktion, die auf dem offenen
    Intervall $(a,b)$ differenzierbar ist. Falls $f(a) = f(b)$ gilt, so existert ein
    $\xi \in (a,b)$ mit $f'(\xi) = 0$.
}

\vspace{1\baselineskip}

\Satz{ (Mittelwertsatz)

    Seien $a<b \in \R$ und $f: [a,b] \rightarrow \R$ eine stetige Funktion, die auf dem offenen
    Intervall $(a,b)$ differenzierbar ist. Dann gibt es ein $\xi \in (a,b)$ mit
    \begin{align*}
        f'(\xi) = \frac{f(b)-f(a)}{b-a}
    \end{align*}
}

\vspace{1\baselineskip}

\Satz{ (Mittelwertsatz nach Cauchy)

    Seien $f$ und $g$ stetige Runktionen auf einem Intervall $[a,b]$ mit $a<b$, so dass $f$
    und $g$ auf $(a,b)$ differenzierbar sind. Dann existert ein $\xi \in (a,b)$ mit
    \begin{align*}
        g'(\xi) \klammer{f(b) - f(a)} = f'(\xi) \klammer{g(b) - g(a)}
        \ \Leftrightarrow \
        \frac{g'(\xi)}{f'(\xi)} = \frac{g(b) - g(a)}{f(b) - f(a)}
    \end{align*}
}

\vspace{1\baselineskip}

\Proposition{

    Sei $I \subseteq \R$ ein Intervall das nicht leer ist und nicht aus einem einzigen Punkt
    besteht. Sei $f: I \rightarrow \R$ eine differenzierbare Funktion. Dann gilt
    \begin{align*}
        f' \geq 0 \ \Leftrightarrow \ f \text{ ist monoton wachsend}
    \end{align*}
    Die Funktion $f$ ist genau dann streng monoton wachsend, wenn es kein nichtleeres, offenes
    Intervall $J \subseteq I$ gibt mit $f'|_J = 0$. Dies ist äquivalent dazu, dass bei der obigen
    Äquivalenz ein striktes ungleich Zeichen steht.
}

\vspace{1\baselineskip}

\Korollar{

    Sei $I \subseteq \R$ ein Intervall mit Endpunkten $a<b$ und $f: I \rightarrow \R$ eine
    Funktion. Dann ist $f$ genau dann konstant, wenn $f$ differenzierbar ist und $f'(x) = 0$
    für alle $x \in I$ gilt.
}

\vspace{1\baselineskip}

\Definition{

    Sei $I \subseteq \R$ ein Interfall und $f: I \rightarrow \R$ eine Funktion. Dann heisst
    $f$ \fat{konvex}, falls für alle $a<b \in I$ und alle $t \in (0,1)$ die Ungleichung
    \begin{align*}
        f \klammer{(1-t) a + t b} \leq (1-t) f(a) + t f(b)
    \end{align*}
    gilt. Wir sagen, dass $f$ \fat{streng konvex} ist, falls in der obigen Ungleichung
    eine strikte Ungleichung gilt. Eine Funktion $g: I \rightarrow \R$ heisst
    \fat{(streng) konkav}, wenn $f = -g$ (streng) konvex ist.

    Merksatz:
    "Hat die Katze Sex, wird der Bauch konvex."
}

\vspace{1\baselineskip}

\Korollar{

    Eine Funktion $f: I \rightarrow \R$ ist genau dann konvex, wenn für alle $a<x<b \in I$
    die folgende Ungleichung gilt:
    \begin{align*}
        \frac{f(x) - f(a)}{x-a} \leq \frac{f(b) - f(x)}{b-x}
    \end{align*}
}

\vspace{1\baselineskip}

\Proposition{

    Sei $I \subseteq \R$ ein Intervall mit Endpunkten $a<b$ und $f: I \rightarrow \R$ eine
    differenzierbare Funktion. Dann ist $f$ genau dann (streng) konvex, wenn $f'$ (streng)
    monoton wachsend ist.
}

\vspace{1\baselineskip}

\Korollar{

    Sei $I \subseteq \R$ ein Intervall mit Endpunkten $a<b$ und $f: I \rightarrow \R$ eine
    zweimal differenzierbare Funktion. Falls $f''(x) \geq 0$ für alle $x \in I$, dann ist $f$
    konvex. Falls $f''(x) > 0$ für alle $x \in I$, dann ist $f$ streng konvex.
}

\vspace{1\baselineskip}

\Satz{ (Regel von de l'Hôpital)

    Seien $a<b$ reelle Zahlen und $f,g : (a,b) \rightarrow \R$ Funktionen. Angenommen
    die folgenden Hypothesen sind erfüllt.

    (1) Die Funktion $f$ und $g$ sind stetig.

    (2) Es gilt $g(x) \neq 0$ und $g'(x) \neq 0$ für alle $x \in (a,b)$

    (3) Es gilt $\limes{x \rightarrow a} g(x) = 0$ und $\limes{x \rightarrow a} = 0$.

    (4) Der Grenzwert $A = \limes{x \rightarrow a} \frac{f'(x)}{g'(x)}$ existert.

    Dann existert auch der Grenzwert $\limes{x \rightarrow a} \frac{f(x)}{g(x)}$ und es gilt
    $\limes{x \rightarrow a} \frac{f(x)}{g(x)} = A$. 
}


\vspace{2\baselineskip}

\subsection{Ableitung trigonometrischer Funktionen}

\vspace{1\baselineskip}

Einige wichtige Ableitungen:
\begin{align*}
    \tan'(x) &= \frac{1}{\cos(x)^2}
    \\
    \arcsin'(x) &= \frac{1}{\sqrt{1-x^2}}
    \\
    \arccos'(x) &= \frac{-1}{\sqrt{1-x^2}}
    \\
    \arctan'(x) &= \cos^2(x) = \frac{1}{1 + x^2}
\end{align*}

\pagebreak

\section{Die Ableitung und das Riemann Integral}

\vspace{1\baselineskip}

\subsection{Der Fundamentalsatz}

\vspace{1\baselineskip}

Wir legen für dieses Kapitel ein kompaktes Intervall $I \subseteq \R$ fest, das nicht leer ist
und nicht aus einem einzelnen Punkt besteht. Integrierbar heisst Riemann-integrierbar.

\vspace{1\baselineskip}

\Definition{

    Eine Funktion $f : I \rightarrow \R$ heisst \fat{lokal integrierbar}, falls für alle
    $a<b \in I$ die eingeschränkte Funktion $f|_{[a,b]}$ integrierbar ist.
}

\vspace{1\baselineskip}

\Definition{

    Sei $a \in I$ und $f: I \rightarrow \R$ lokal integrierbar. Wir nennen $F: I \rightarrow \R$
    gegeben durch
    \begin{align*}
        F(x) = \int_a^x f(f) dt = \begin{cases}
            \ \int_a^x f(t) dt \quad &\text{    falls } a<x \\
            \ 0 \quad &\text{ falls } a=x \\
            \ - \int_a^x f(t) dt \quad &\text{ falls } a>x
        \end{cases}
    \end{align*}
    ein \fat{partikuläres/ spezielles Integral} von $f$.
}

\vspace{1\baselineskip}

\Satz{ (Fundamentalsatz der Interal- und Differentialrechnung)

    Sei $f: I \rightarrow \R$ eine integrierbare Funktion und sei $F$ ein partikuläres
    Integral von $f$ gegeben durch:
    \begin{align*}
        F(x) = \int_a^x f(t) dt = - \int_x^a f(t) dt
    \end{align*}
    Falls $f$ bei $x_0 \in I$ stetig ist, so ist $F$ bei $x_0$ differenzierbar, und es gilt
    $F'(x) = f(x)$. Wir nennen die Funktion $F$ \fat{Stammfunktion} von $f$. Jede stetige
    Funktion $f: I \rightarrow \R$ besitzt so eine Stammfunktion. Zwei Stammfunktionen von $f$
    unterscheiden sich durch eine Konstante.
}

\vspace{1\baselineskip}

\Korollar{

    $D: C^1 (I) \rightarrow C^0 (I)$ ist surjektiv.
}

\vspace{1\baselineskip}

\Korollar{ (Fundamentalsatz der Differential- und Integralrechnung)

    Sei $f: I \rightarrow \R$ stetig und $F: I \rightarrow \R$ eine Stammfunktion von $f$.
    Dann gilt
    \begin{align*}
        \intab f(t) dt = F(b) - F(a)
    \end{align*}
}

\vspace{1\baselineskip}

\Korollar{

    Sei $f(x) = \sum_{n=0}^{\infty} a_n x^n$ eine Potenzreihe mit Konvergenzradius $R>0$.
    Dann ist $f: (-R,R) \rightarrow \R$ differenzierbar und es gilt
    \begin{align*}
        f'(x) = \sum_{n=1}^{\infty} n a_n x^{n-1}
    \end{align*}
    für alle $x \in (-R,R)$, wobei die Potenzreihe rechts ebenfalls Konvergenzradius $R$ hat.
}

\pagebreak

\subsection{Integrationsmethoden}

\vspace{1\baselineskip}

Wir legen für dieses Kapitel ein nichtleeres Intervall $I \subseteq \R$ fest, das nicht nur aus
einem isolierten Punkt besteht. Falls nicht ausdrücklich anders erwähnt, so sind alle
Funktionen in diesem Kapitel reellwertige Funktionen mit Definitionsbereich $I$, die auf jedem
kompakten Intervall $[a,b] \subseteq I$ integrierbar sind. Alle Resultate gelten analog für
komplexwertige Funktionen.

\vspace{1\baselineskip}

\Definition{

    Sei $I \subseteq \R$ ein Intervall und $f: I \rightarrow \R$ eine Funktion.
    Dann nennen wir folgendes Integral das \fat{unbestimmte Integral} mit
    \fat{Integrationskonstante} $C$.
    \begin{align*}
        \int f(x) dx = F(x) + C
    \end{align*}    
}

\vspace{1\baselineskip}

\Bemerkung{

    Seien $f$ und $g$ Funktionen, $a$ und $b$ reelle Zahlen. Aus der Linearität der Ableitung folgt
    \begin{align*}
        \int a f(x) + b g(x) \ dx = a \int f(x) dx \ + \ b \int g(x) dx \ + C
    \end{align*}
}

\vspace{1\baselineskip}

\Definition{ (Partielle Integration)

    Seien $f$ und $g$ Funktionen mit Stammfunktion $F$, bzw $G$. Als \fat{partielle Integration}
    bezeichnet man die folgende Identität
    \begin{align*}
        \int F(x) g(x) dx = F(x) G(x) - \int f(x) G(x) dx \ + C
    \end{align*}
}

\vspace{1\baselineskip}

\Definition{ (Substitutionsmethode)

    Sei $J \subseteq \R$ ein weiteres Intervall und sei $f: I \rightarrow J$ eine differenzierbare
    Funktion. Ist $G: J \rightarrow \R$ differenzierbar mit Ableitung $g = G'$, so gilt nach
    der Kettenregel $G(f(x))' = g(f(x))f'(x)$ für alle $x \in I$. Daraus folgt
    \begin{align*}
        \int (g \circ f) (x) f'(x) dx = G(f(x)) + C
    \end{align*}
    Die Funktion $G$ ist eine Stammfunktion von $g$, es gilt also $G(u) = \int g(u) du$.
    Als \fat{Substitutionsregel} bezeichnen wir die Identität
    \begin{align*}
        \int (g \circ f) (x) f'(x) dx = g(u) du + C
    \end{align*}
    wobei $u = f(x)$. Die Substitutionsregel wird auch \fat{Variablenwechsel} genannt, da man
    sozusagen die Variable $u$ in $\int g(u) du$ durch $u=f(x)$ ersetzt hat.
}

\vspace{1\baselineskip}

\Bemerkung{

    Einige wichtige Integrationen \fat{rationaler Funktionen}. Sei dazu $a \in \R$ und $a \neq I$
    und sei $n \geq 2 \in \N$. Dann gilt:
    \begin{align*}
        \int \frac{1}{x-a} dx &= \log \abs{x-a} + C
        \\
        \int \frac{1}{(x-a)^n} dx &= \frac{(x-a)^{1-n}}{1-n} +C
        \\
        \int \frac{1}{a^2 + x^2} dx &= \frac{\arctan \klammer{\frac{x}{a}}}{a} + C
        \\
        \int \frac{x}{a^2 + x^2} dx &= \frac{\log(a^2 + x^2)}{2} + C
        \\
        \int \frac{x}{(a^2 + x^2)^n} dx &= \frac{(a^2 + x^2)^{1-n}}{2 (1-n)} + C 
    \end{align*}
}

\vspace{1\baselineskip}

\Definition{

    Sei $I \subseteq \R$ ein nichtleeres Intervall, und $f: I \rightarrow \R$ eine lokal integrierbare
    Funktion. Setze $a = \inf (I)$ und $b = \sup (I)$ und wähle $x_0 \in I$. Wir definieren das
    \fat{uneigentliche Integral} als die Summe von Grenzwerten
    \begin{align*}
        \intab f(x) dx = \limes{x \rightarrow a} \int_x^{x_0} f(t) dt + \limes{x \rightarrow b} \int_{x_0}^x f(t) dt
    \end{align*}
    falls beide dieser Grenzwerte existieren. In dem Fall sagen wir auch, dass das uneigentliche
    Integral \fat{konvergiert}. Ansonsten nennen wir das uneigentliche Integral \fat{divergent}.
    Wir benutzen die üblichen Konventionen für die Symbole $- \infty$ und $+ \infty$.
}

\vspace{1\baselineskip}

\Lemma{

    Sei $a \in \R$ und $f: [a,\infty) \rightarrow \R_{\geq 0}$ eine nicht-negative, lokal
    integrierbare Funktion. Dann gilt
    \begin{align*}
        \int_a^{\infty} f(x) dx = \sup \geschwungeneklammer{\intab f(x) dx \ | \ b>a}
    \end{align*} 
}

\vspace{1\baselineskip}

\Satz{ (Integralsatz für Reihen)

    Sei $f: [0, \infty) \rightarrow \R_{\geq 0}$ eine monoton fallende Funktion. Dann gilt
    \begin{align*}
        \sum_{n=1}^{\infty} f(n) \geq \int_0^{\infty} f(x) dx \geq \sum_{n=0}^{\infty} f(n)
    \end{align*}
    Insbesondere gilt die folgende Äquivalenz
    \begin{align*}
        \sum_{n=0}^{\infty} f(n) \text{   konvergiert} \Longleftrightarrow \int_0^{\infty} f(x) dx \text{   konvergiert}
    \end{align*}
}


\vspace{2\baselineskip}

\subsection{Taylorreihen}

\vspace{1\baselineskip}

\Definition{

    Sei $D \subseteq \R$ ein offenes Intervall, und $f: D \rightarrow \R$ eine $n$-mal
    differenzierbare Funktion. Die $n$-te \fat{Taylor-Approximation} von $f$ um einen
    Punkt $x_0 \in D$ ist die Polynomfunktion
    \begin{align*}
        P_n(x) = \sum_{k=0}^{n} \frac{f^{(k)}(x_0)}{k!} (x-x_0)^k
    \end{align*}
    Die Koeffizienten wurden dabei gerade so gewählt, dass $P^{(k)} (x_0) = f^{(k)} (x_0)$
    für $k \in \geschwungeneklammer{0, \dots , n}$ gilt. Falls $f$ glatt ist, dann ist
    die \fat{Taylorreihe} von $f$ um $x_0 \in D$ definiert als die Potenzreihe
    \begin{align*}
        L(T) = \sum_{k=0}^{\infty} \frac{f^{(k)} (x_0)}{k!} T^k
    \end{align*}
    Ist $R$ der Konvergenzradius dieser Reihe, so konvergiert die Reihe der reellen
    Zahlen
    \begin{align*}
        \sum_{k=0}^{\infty} \frac{f^{(k)} (x_0)}{k!} (x-x_0)^k
    \end{align*}
    für alle $x \in \R$ mit Abstand kleiner $R$ von $x_0$ und divergiert für alle
    $x \in \R$ mit $\abs{x-x_0} > R$. Oft nennt man diese Summe auch \fat{Taylorreihe}.
}

\pagebreak

\Satz{ (Taylor-Approximation)

    Sei $D \subseteq \R$ ein offenes Intervall und $f: D \rightarrow \R$ (oder auch
    $D \rightarrow \C$) eine $(n+1)$-mal stetig differenzierbare Funktion. Dann gilt
    für alle $x \in D$
    \begin{align*}
        f(x) = P_n (x) + \int_{x_0}^x f^{(n+1)} (t) \frac{(x-t)^n}{n!} dt
    \end{align*}
    wobei $P_n$ die $n$-te Taylor-Approximation von $f$ ist.
}

\vspace{1\baselineskip}

\Korollar{ (Taylor-Abschätzung)

    Sei $D \subseteq \R$ ein offenes Intervall, $f: D \rightarrow \R$
    (oder auch $D \rightarrow \C$) eine $(n+1)$-mal stetig differenzierbare Funktion,
    und sei $\delta > 0$ und $x_0 \in D$ und setze
    $M = \sup \geschwungeneklammer{\abs{f^{(n+1)} (t) } \ | \ t \in [x_0 - \delta , x_0 + \delta]}$.
    Dann gilt
    \begin{align*}
        \abs{f(x) - P_n (x)} \leq \frac{M \cdot \abs{x - x_0}^{n+1}}{(n+1)!}
    \end{align*}
    Insbesondere ist $f(x) - P_n (x) = O((x-x_0)^{n+1})$ für $x \longrightarrow x_0$.
}

\vspace{1\baselineskip}

\Definition{

    Sei $I \subseteq \R$ ein Intervall, und $x_0 \in I$ ein Punkt im Inneren von $I$.
    Eine glatte Funktion $f: I \rightarrow \R$ heisst \fat{analytisch} bei $x_0$ falls
    ein $\delta > 0$ existiert, derart, dass die Taylorreihe von $f$ um $x_0$ einen
    Konvergenzradius $R > \delta$ hat, und
    \begin{align*}
        f(x) = \sum_{n=0}^{\infty} \frac{f^{(n)} (x_0)}{n!} (x-x_0)^n
    \end{align*}
    für alle $x \in (x_0 - \delta , x_0 + \delta) \cap I$ gilt. Wir sagen $f$ sei
    analysisch auf $I$ falls $f$ analysisch in jedem Punkt von $I$ ist.
}

\vspace{1\baselineskip}

\Bemerkung{

    Analytische Funktionen $f: I \rightarrow \R$ sind also dadurch charakterisiert,
    dass es zu jedem Punkt $x_0$ im Inneren von $I$ eine Potenzreihe gibt, die in einer
    Umgebung von $x_0$ gegen $f$ konvergiert. Eine Abschätzung die garantiert, dass
    die Taylorreihe von $f$ im Punkt $x_0$ gegen $f$ konvergiert, ist, dass für ein
    $\delta > 0$
    \begin{align*}
        \abs{x-x_0} < \delta \Longrightarrow \abs{f^{(n+1)} (x)} \leq c A^n
    \end{align*}
    für alle \ninN und zwei Konstanten $c,A \geq 1$ gilt. Eine Abschätzung dieser Art
    findet man beispielsweise für $\exp , \sin , \cos$ und Kombinationen davon.
}

\vspace{1\baselineskip}

\Satz{

    Seien $a<b$ reelle Zahlen, $f: [a,b] \rightarrow \R$ eine stetige Funktion,
    $n \in \N$ und $x_k = a + k \frac{b-a}{n}$ für $k \in \geschwungeneklammer{0, \dots , n}$.

    \begin{enumerate}[{(1)}]
        \item (Rechteckregel) Falls $f$ stetig differenzierbar ist, dann gilt
                \begin{align*}
                    \intab f(x) dx = \frac{b-a}{n} (f(x_0) + \dots + f(x_{n-1})) + F_1
                \end{align*}
                wobei der Fehler $F_1$ durch
                $\abs{F_1} \leq \frac{(b-a)^2}{2n} \max \geschwungeneklammer{\abs{f'(x)} \ | \ x \in [a,b]}$
                beschränkt ist.
        \item (Sehnentrapezregel) Falls $f$ zweimal stetig differenzierbar ist, dann
                gilt
                \begin{align*}
                    \intab f(x) dx = \frac{b-a}{2n} (f(x_0) + 2 f(x_1) + \dots + 2 f(x_{n-1}) + f(x_n)) + F_2
                \end{align*}
                wobei der Feher $F_2$ durch
                $\abs{F_2} \leq \frac{(b-a)^3}{6n^2} \max \geschwungeneklammer{\abs{f''(x)} \ | \ x \in [a,b]}$
                beschränkt ist.
        \item (Simpson-Regel) Falls $f$ viermal stetig differenzierbar ist und $n$
                gerade ist, dann gilt
                \begin{align*}
                    \intab f(x) dx = \frac{b-a}{3n} \klammer{f(x_0) + 4 f(x_1) + 2 f(x_2) + 4 f(x_3) + f(x_4) + \dots + 2 f(x_{n-2}) + 4 f(x_{n-1}) + f(x_n)} + F_3
                \end{align*}
                wobei der Fehler $F_3$ durch $\abs{F_3} \leq \frac{(b-a)^5}{45n^4} \max \geschwungeneklammer{\abs{f^{(4)}} \ | \ x \in [a,b]}$
                beschränkt ist.
    \end{enumerate}
}


\pagebreak

\section{Topologische Grundbegriffe}

\vspace{1\baselineskip}

\subsection{Topologische Räume}

\vspace{1\baselineskip}

\Definition{

    Ein \fat{topologischer Raum} ist ein geordnetes Paar $(X,\tau)$ bestehend aus einer
    Menge $X$ und einer Familie von Teilmengen $\tau$ von $X$ die den Axiomen $1,2,3$
    genügt. Wir nennen die Familie $\tau$ eine \fat{Topologie} auf $X$ und Teilmengen
    $U \subseteq X$ die zur Familie $\tau$ gehören \fat{offene} Mengen.
    \begin{enumerate}
        \item $\emptyset \subseteq X$ ist offen, und $X \subseteq X$ ist offen.
        \item Beliebige Vereinigungen von offenen Mengen sind offen.
        \item Endliche Durchschnitte von offenen Mengen sind offen.
    \end{enumerate}
    Wir nennen eine Teilmenge $F \subseteq X$ deren Komplement offen ist eine
    \fat{abgeschlossene} Teilmenge. Sei $x \in X$. Eine Teilmenge $V \subseteq X$
    heisst \fat{Umgebung} von $x$, falls eine offene Teilmenge $U \subseteq X$ mit
    $x \in U$ und $U \subseteq V$ existiert. Ist $V$ offen, so nennen wir $V$ eine
    \fat{offene Umgebung} von $x$, und ist $V$ abgeschlossen, so nennen wir $V$ eine
    \fat{abgeschlossene Umgebung} von $x$.
}

\vspace{1\baselineskip}

\Bemerkung{

    Jeder metrischer Raum liefert einen topologischen Raum.
}

\vspace{1\baselineskip}

\Definition{

    Sei $X$ eine Menge. Die Familie aller Teilmengen $\tau = \mathcal{P} (X)$ ist
    eine Topologie. Bezüglich dieser Topologie sind also alle Teilmengen von $X$ offen.
    Sie heisst \fat{disktrete Topologie} auf $X$. Die Familie $\tau =
    \geschwungeneklammer{\emptyset,X}$ ist ebenfalls eine Topologie. Sie wird
    \fat{triviale Topologie} oder auch \fat{indiskrete Topologie} genannt.
    Interessante Topologien sind etwa die bereits bekannten Topologie auf $\R$ oder
    auf $\C$. Wir nennen diese Topologien \fat{Standardtopologien} auf $\R$, bzw $\C$.
    Die Standardtopologie auf $\R^n$ ist analog definiert.
}

\vspace{1\baselineskip}

\Definition{

    Sei $(X,\tau)$ ein topologischer Raum und $Y \subset X$ eine Teilmenge. Die
    \fat{Einschränkung} von $\tau$ auf $Y$ ist die Topologie
    \begin{align*}
        \tau |_{Y} &= \geschwungeneklammer{Y \cap U \ | \ U \in \tau}
        = \geschwungeneklammer{Y \cap U \ | \ U \subseteq X \text{ offen}}
    \end{align*}
    auf $Y$. Wir nennen sie auch \fat{induzierte Topologie} oder \fat{Unterraumtopologie}.
    Eine für die Topologie $\tau |_{Y}$ offene Teilmenge $V \subseteq Y$ nennen wir
    \fat{relativ offen}, und eine für die Topologie $\tau |_{Y}$ abgeschlossene
    Teilmenge nennen wir auch \fat{relativ abgeschlossen} ("offen in der Teilmenge
    wo es lebt").
}

\vspace{1\baselineskip}

\Definition{

    Seien $(X,\tau)$ und $(Y, \sigma)$ topologische Räume. Eine \fat{stetige Abbildung}
    von $(X,\tau)$ nach $(Y, \sigma)$ ist eine Abbildung $f: X \rightarrow Y$ so,
    dass für jede offene Teilmenge $U \subseteq Y$ das Urbild $f^{-1} (U) \subseteq X$
    offen ist.
}

\vspace{1\baselineskip}

\Korollar{

    Seien $X,Y$ und $Z$ topologische Räume. Die Identitätsabbildung id$_{X} : X \rightarrow X$
    ist stetig. Sind $f: X \rightarrow Y$ und $g: Y \rightarrow Z$ stetige Abbildungen,
    so ist die Verknüpfung $g \circ f: X \rightarrow Z$ ebenfalls stetig.
    Ist $f: X \rightarrow Y$ stetig und bijektiv, so ist die Umkehrabbildung
    $f^{-1} : Y \rightarrow X$ im Allgemeinen nicht stetig.
}

\vspace{1\baselineskip}

\Definition{

    Eine bijektive, stetige Abbildung deren inverses ebenfalls stetig ist, heisst
    \fat{Homöomorphismus}.
}

\pagebreak

\Proposition{

    Sei $D \subseteq \R$ eine Teilmenge, und $f:D \rightarrow \R$ eine Funktion.
    Die folgenden Aussagen sind äquivalent.
    \begin{enumerate}
        \item Für jedes $x_0 \in \R$ und jedes $\epsilon > 0$ existiert ein $\delta >0$
                so, dass für alle $x \in D$ folgendes gilt:
                \begin{align*}
                    \abs{x-x_0} < \delta \Longrightarrow \abs{f(x)-f(x_0)} < \epsilon
                \end{align*}
        \item Für jede offene Teilmenge $U \subseteq \R$ ist $f^{-1} (U) \subseteq D$
                offen, für die von der Standardtopologie auf $\R$ induzierte Topologie
                auf $D$.
    \end{enumerate}
}

\vspace{1\baselineskip}

\Definition{

    Sei $(X,\tau)$ ein topologischer Raum und $Y \subseteq X$ eine Teilmenge. Die Menge
    \begin{align*}
        Y^{\circ} = \geschwungeneklammer{x \in Y \ | \ \exists \text{ Umgebung } U \text{ von } x \text{ mit } U \subseteq Y}
    \end{align*}
    wird das \fat{Innere} von $Y$ genannt, die Menge
    \begin{align*}
        \overline{Y} = \geschwungeneklammer{ x \in Y \ | \ U \cap Y \neq \emptyset \text{ für jede Umgebung } U \text{ von } x}
    \end{align*}
    wird \fat{Abschluss} von $Y$ genannt. Der \fat{Rand} von $Y$ wird durch
    $\partial Y = \overline{Y} \backslash Y^{\circ}$ definiert. Die Teilmenge
    $Y \subseteq X$ heisst \fat{dicht}, falls $\overline{Y} = X$ gilt.
}

\vspace{1\baselineskip}

\Definition{

    Sei $\XTopRaum$ ein topologischer Raum, und sei $\xFolge$ eine Folge in $X$.
    Ein Punkt $x \in X$ heisst \fat{Grenzwert} der Folge $\xFolge$, falls für jede
    Umgebung $U$ von $x$ ein \NinN mit
    \begin{align*}
        n \geq N \Rightarrow x_n \in U
    \end{align*}
    existiert. Ein Punkt \xinX heisst \fat{Häufungspunkt} der Folge, falls für jede
    Umgebung $U$ von $x$ und jedes \NinN ein $n \geq N$ mit $x_n \in U$.
}

\vspace{1\baselineskip}

\Definition{

    Sei $\XTopRaum$ ein topologischer Raum. Wir nennen $\XTopRaum$ einen
    \fat{Hausdorff-Raum}, falls alle Punkte $x_1 neq x_2$ von $X$ Umgebungen
    $x_1 \in U_1$ und $x_2 \in U_2$ mit $U_1 \cap U_2 = \emptyset$ existiert.
}

\vspace{1\baselineskip}

\Proposition{

    Sei $\XTopRaum$ ein Hausdorff'scher topologischer Raum und $\xFolge$ eine Folge in
    $X$. Dann hat $\xFolge$ höchstens einen Grenzwert $a \in X$. In dem Fall, schreibe
    $a = \limesninf x_n$.
}

\vspace{1\baselineskip}

\Definition{

    Eine Abbildung $f: X \rightarrow Y$ zwischen Hausdorff'schen topologischen
    Räumen heisst \fat{folgenstetig}, falls für jede konvergente Folge $\xFolge$
    in $X$ mit $\limesninf x_n = x$ folgendes folgt:
    \begin{align*}
        \limesninf f \xFolge = f(x)
    \end{align*}
}

\Bemerkung{

    Stetig impliziert folgenstetig. Die Umkehrung gilt im Allgemeinen aber nicht.
}

\vspace{1\baselineskip}

\Definition{

    Seien $X$ und $Y$ topologische Räume. Wir bezeichnen als \fat{Produkttopologie}
    die Topologie auf $X \times Y$, deren offene Menge gerade diejenigen Teilmengen
    $U \subseteq X \times Y$ sind, die folgende Eigenschaften erfüllen:
    Für jedes $(x,y) \in U$ existiert eine offene Umgebung $V \subseteq X$ von $x$
    und eine offene Umgebung $W \subseteq Y$ von $y$, mit $V \times W \subseteq U$.
}

\vspace{1\baselineskip}

\Proposition{

    Seien $X,Y$ und $Z$ topologische Räume. Eine Abbildung $f: Z \rightarrow X \times Y$
    ist genau dann stetig für die Produkttopologie auf $X \times Y$, wenn die
    folgenden beiden Verknüpfung stetig sind.
    \begin{align*}
        f_X : Z \stackrel{f}{\longrightarrow} X \times Y \stackrel{\pi_X}{\longrightarrow} X
        \quad \text{   und   } \quad
        f_Y : Z \stackrel{f}{\longrightarrow} X \times Y \stackrel{\pi_Y}{\longrightarrow} Y
    \end{align*}
}


\vspace{2\baselineskip}

\subsection{Topologie auf metrischen Räumen}

\vspace{1\baselineskip}

\Definition{

    Sei $(X,d)$ ein metrischer Raum, \xinX und $r>0$ eine reelle Zahl, so schreiben wir
    \begin{align*}
        B(x,r) = \geschwungeneklammer{y \in X \ | \ d(x,y) < r}
    \end{align*}
    und nennen die Menge $B(x,r)$ \fat{offener Ball} oder \fat{offene Kreisscheibe}
    mit Zentrum $x$ und Radius $r$.
}

\vspace{1\baselineskip}

\Definition{

    Sei $(X,d)$ ein metrischer Raum. Eine Teilmenge $U \subseteq \X$ heisst \fat{offen},
    falls für jedes $x \in U$ ein $\epsilon>0$ existiert, derart, dass
    $B(x,\epsilon) \subseteq U$ gilt. Die damit definierte Topologie $\tau_d$ auf der
    Menge $X$ wird die von der Metrik \fat{induzierte Topologie} genannt.
}

\vspace{1\baselineskip}

\Lemma{

    Sei $(X,d)$ ein metrischer Raum.
    \begin{enumerate}
        \item Eine Teilmenge $U \subset X$ ist genau dann offen, wenn für jede
                konvergente Folge in $X$ mit Grenzwert in $U$ fast alle
                Folgenglieder in $U$ liegen.
        \item Eine Teilmenge $A \subset X$ ist genau dann abgeschlossen, wenn für
                jede konvergente Folge $\xFolge$ in $A$ mit $x_n \in A$ für alle
                \ninN auch der Grenzwert in $A$ liegt.
    \end{enumerate}
}

\vspace{1\baselineskip}

\Proposition{

    Sei $d \in \N$. Jede abgeschlossene Teilmenge von $\R^d$ ist vollständig.
}

\vspace{1\baselineskip}

\Definition{

    Seien $(X,d_X)$ und $(Y,d_Y)$ metrische Räume, und sei $f: X \rightarrow Y$ eine
    Funktion.
    \begin{enumerate}
        \item Wir sagen, $f$ ist \fat{$\epsilon-\delta-$stetig}, falls für alle \xinX
                und alle $\epsilon >0$ ein $\delta>0$ existiert, so dass
                $d_X (x,y) < \delta \Longrightarrow d_Y (f(x),f(y)) < \epsilon$ für alle
                \yinX gilt.
        \item Wir sagen, dass $f$ \fat{folgenstetig} ist, falls für jede konvergente
                Folge $\xFolge$ in $X$ mit Grenzwert $x = \limesninf x_n$ die Folge
                $(f(x_n))_{n}$ konvergiert, mit $\limesninf f(x_n) = f(x)$.
        \item Wir sagen, $f$ ist \fat{topologisch stetig}, falls für jede offene
                Teilmenge $U \subseteq Y$ das Urbild $f^{-1} (U)$ offen in $X$ ist.
    \end{enumerate}
}

\vspace{1\baselineskip}

\Proposition{

    Seien $X$ und $Y$ metrische Räume und sei $f: X \rightarrow Y$ eine Funktion. Dann
    sind folgende Aussagen äquivalent:

    (1) Die Funktion $f$ ist $\epsilon-\delta-$stetig.

    (2) Die Funktion $f$ ist folgenstetig.

    (3) Funktion $f$ ist topologisch stetig.
}

\vspace{1\baselineskip}

\Definition{

    Seien $(X,d_X)$ und $(Y,d_Y)$ metrische Räume und $f:X \rightarrow Y$ eine Funktion.
    Die Funktion $f$ heisst \fat{gleichmässig stetig}, falls es zu jedem $\epsilon>0$
    ein $\delta>0$ gibt, so dass für alle $x_1,x_2 \in X$ mit $d_X (x_1,x_2) < \delta$
    auch $d_Y (f(x_1),f(x_2)) < \epsilon$ gilt.
}

\vspace{1\baselineskip}

\Definition{

    Seien $(X,d_X)$ und $(Y,d_Y)$ metrische Räume und $f:X \rightarrow Y$ eine Funktion.
    Die Funktion $f$ heisst \fat{Lipschitz-stetig}, falls es eine reelle Zahl $L \geq 0$
    gibt, genannt \fat{Lipschitz-Konstante}, mit
    \begin{align*}
        d_Y (f(x_1),f(x_2)) \leq L d_X (x_1,x_2)
    \end{align*}
    für alle $x_1 , x_2 \in X$. Es gelten die Implikationen
    \begin{align*}
        f \text{ ist Lipschitz-stetig } \Longrightarrow
        f \text{ ist gleichmässig stetig } \Longrightarrow
        f \text{ ist stetig} 
    \end{align*}
    und im Allgemeinen sind die umgekehrten Implikationen falsch.
}

\vspace{1\baselineskip}

\Satz{ (Banach'sche Fixpunktsatz)

    Sei $(X,d)$ ein nicht-leerer, vollständiger metrischer Raum. Sei $T: X \rightarrow X$
    eine Abbildung und $0 \leq \lambda < 1$ eine reelle Zahl mit der Eigenschaft, dass
    \begin{align*}
        d(T(x_1),T(x_2)) \leq \lambda d(x_1 , x_2)
    \end{align*}
    für alle $x_1 , x_2 \in X$ gilt. Dann existiert genau ein Element $a \in X$ mit
    $T(a) = a$. Die Zahl $\lambda$ nennt man \fat{Lipschitz-Konstante} und die
    Funktion $T$ eine \fat{Lipschitz-Kontraktion}. Ein Punkt \xinX mit $T(x) = x$
    heisst \fat{Fixpunkt} der Abbildung $T$, und der Satz besagt also, dass eine
    Lipschitz-Kontraktion genau einen Fixpunkt besitzt.
}


\vspace{2\baselineskip}

\subsection{Kompaktheit}

\vspace{1\baselineskip}

\Definition{

    Sei $X$ ein topologischer Raum, und sei $A \subseteq X$ eine Teilmenge. Eine
    \fat{offene Überdeckung} von $A$ ist eine Familie offener Mengen $\mathcal{U}$
    von $X$, so, dass
    \begin{align*}
        A \subseteq \bigcup_{U \in \ \mathcal{U}} U
    \end{align*}
    gilt. Eine \fat{Teilüberdeckung} der Überdeckung $\mathcal{U}$ von $A$ ist eine
    offene Überdeckung $\mathcal{V}$ von $A$ mit $\mathcal{V} \subseteq \mathcal{U}$.
    Eine \fat{endliche offene Überdeckung} ist eine offene Überdeckung durch eine
    endliche Familie offener Mengen.
}

\vspace{1\baselineskip}

\Definition{

    Sei $X$ ein topologischer Raum und $A \subseteq X$ eine Teilmenge. Wir sagen,
    $A$ sei \fat{kompakt}, falls jedde offene Überdeckung von $A$ eine endliche
    Teilüberdeckung besitzt. Wie nennen $X$ einen kompakten Topologischen Raum
    wenn $X$ als Teilmenge von $X$ kompakt ist.
}

\vspace{1\baselineskip}

\Proposition{

    Ein topologischer Raum $X$ ist genau dann kompakt, wenn $X$ das folgende
    \fat{Schachtelungsprinzip} erfüllt: Für jede Kollektion $\mathcal{A}$
    abgeschlossener Teilmengen von $X$ mit der Eigenschaft, dass
    $A_1 \cap \dots \cap A_n \neq \emptyset$ für alle \ninN und
    $A_1 , \dots , A_n \in \mathcal{A}$ gilt, gilt auch
    \begin{align*}
        \bigcap_{A \in  \mathcal{A}} A \neq \emptyset
    \end{align*}
}

\vspace{1\baselineskip}

\Korollar{

    Ist $X$ kompakt und $A \subseteq X$ abgeschlossen, so ist $A$ kompakt.
}

\vspace{1\baselineskip}

\Proposition{

    Seien $X$ und $Y$ topologische Räume und sei $f:X \rightarrow Y$ eine stetige
    Abbildung und sei $A \subseteq X$ eine kompakte Teilmenge. Dann ist $f(A)$
    eine kompakte Teilmenge von $Y$.
}

\vspace{1\baselineskip}

\Definition{

    Sei $X$ ein topologischer Raum. Wir sagen, $X$ sei \fat{folgenkompakt}, falls jede
    Folge in $X$ einen Häufungspunkt in $X$ besitzt. (Alternativ: ... falls jede
    Folge in $X$ eine konvergente Teilfolge besitzt.) Ein Teilraum $Y \subseteq X$
    heisst folgenkompakt, falls $Y$ als eigenständiger topologischer Raum folgenkompakt
    ist.
}

\vspace{1\baselineskip}

\Definition{

    Wir nennen einen metrischen Raum $(X,d)$ \fat{kompakt}, wenn $X$ als topologischer
    Raum, also $X$ bezüglich der von der Metrik $d$ induzierten Topologie, kompakt
    ist.
}

\vspace{1\baselineskip}

\Definition{

    Ein metrischer Raum $(X,d)$ heisst \fat{beschränkt}, falls es ein $R>0$ gibt
    mit der Eigenschaft, dass $d(x,y) \leq R$ für alle $x,y \in X$ gilt.
}

\vspace{1\baselineskip}

\Definition{

    Sei $X$ ein metrischer Raum. Wir sagen, dass $X$ \fat{total beschränkt} ist,
    falls es für jedes $r>0$ endlich viele $x_1 , \dots , x_n \in X$ gibt mit
    \begin{align*}
        X = \bigcup_{j=0}^{n} B(x_j , r)
    \end{align*}
}

\vspace{1\baselineskip}

\Definition{

    Sei $(X,d)$ ein metrischer Raum und $\mathcal{U} = (U_i)_{i \in I}$ eine offene
    Überdeckung von $X$. Eine reelle Zahl $\lambda>0$ heisst \fat{Lebesgue Zahl} zu
    dieser Überdeckung, falls für alle \xinX ein $i \in I$ existiert mit
    $B(x,\lambda) \subseteq U_i$.
}

\vspace{1\baselineskip}

\Satz{

    Sei $X$ ein metrischer Raum. Dann sind folgende Aussagen äquivalent.
    \begin{enumerate}[{(1)}]
        \item Mit der von der Metrik induzierten Topologie ist $X$ ein kompakter topologischer Raum.
        \item Jede Folge in $X$ hat einen Häufungspunkt, das heisst, $X$ ist folgenkompakt.
        \item Jede unenendliche Teilmenge von $X$ besitzt einen Häufungspunkt.
        \item Jede stetige, reellwertige Funktion auf $X$ ist beschränkt.
        \item Jede stetige, reellwertige Funktion auf $X$ nimmt ein Maximum und ein Minimum an.
        \item Jede offene Überdeckung von $X$ besitzt eine Lebesgue-Zahl und $X$ ist total beschränkt.
        \item Der metrische Raum $X$ ist total beschränkt und vollständig.
    \end{enumerate}
}

\vspace{1\baselineskip}

\Proposition{

    Seien $X$ und $Y$ metrische Räume und $f: X \rightarrow Y$ eine stetige Funktion.
    Falls $X$ kompakt ist, so ist $f$ gleichmässig stetig.
}

\vspace{1\baselineskip}

\Proposition{

    Sei $X$ ein metrischer Raum, $A \subseteq X$. Ist A kompakt, so ist $A$ beschränkt
    und abgeschlossen. Die Umkehrung gilt im Allgemeinen nicht.
}

\vspace{1\baselineskip}

\Definition{

    Sei $f: X \rightarrow \R$ eine beschränkte, reellwertige Funktion auf einem
    metrischen Raum $X$. Für \xinX und $\delta>0$ ist die \fat{Oszillation} oder
    \fat{Schwankung} von $f$ bei $x$ wie folgt definiert.
    \begin{align*}
        \omega(f,x,\delta) = \limes{\delta \rightarrow 0} \klammer{\sup f(B(x,\delta)) - \inf f(B(x,\delta))}
    \end{align*}
}

\vspace{1\baselineskip}

\Proposition{

    Sei $X$ ein kompakter metrischer Raum und sei $f: X \rightarrow \R$ eine beschränkte
    Funktion. Angenommen es gibt $\eta \geq 0$, so dass $\omega(f,x) \leq \eta$ für
    alle $x \in X$. Dann existiert für jedes $\epsilon > 0$ ein $\delta > 0$, so
    dass für alle \xinX folgendes gilt:
    \begin{align*}
        \omega(f,x,\delta) < \eta + \epsilon
    \end{align*}
}

\vspace{1\baselineskip}

\Lemma{

    Eine folgenkompakte Teilmenge eines metrischen Raumes ist abgeschlossen und
    beschränkt.
}

\vspace{1\baselineskip}

\Satz{ (Heine-Borel)

    Eine Teilmenge $A \subseteq \R^d$ für $d \geq 0$ ist genau dann kompakt, wenn
    sie abgeschlossen und beschränkt ist.
}

\vspace{1\baselineskip}

\Satz{ (Fundamentalsatz der Algebra)

    Jedes nicht-konstante Polynom $f \in \C [T]$ hat eine Nullstelle in $\C$.
}


\vspace{2\baselineskip}

\subsection{Topologische Vektorräume}

\vspace{1\baselineskip}

In diesem Kapitel bezeichnen wir mit $\K$ den Körper der reellen Zahlen, oder den
Körper der komplexen Zahlen.

\vspace{1\baselineskip}

\Definition{

    Ein \fat{topologischer Vektorraum} über $\K$ ist ein Vektorraum $V$ über $\K$
    zusammen mit der Hausdorff'schen Topologie $\tau$ auf $V$, derart, dass die
    Abbildungen
    \begin{align*}
        m: \K \times V \rightarrow V
        \quad \quad \text{   und   } \quad \quad
        s: V \times V \rightarrow V
    \end{align*}
    gegeben durch $m(a,v) = av$ und $s(v,w) = v+w$ stetig sind, bzgl. der
    Produkttopologie auf $\K \times V$ bzw. auf $V \times V$.
}

\vspace{1\baselineskip}

\Proposition{

    Sei $V$ ein normierter $\K$-Vektorraum, versehen mit der Topologie die durch die
    Metrik $d(v,w) = \Norm{v-w}$ induziert wird. Dann sind die obigen Abbildungen
    stetig, und $V$ wird mit dieser Topologie zu einem topologischen Vektorraum.
}

\pagebreak

\Proposition{

    Sei $K$ ein kompakter metrischer Raum, und es bezeichne $\zweiNorm$ die
    euklidische Norm auf $\R^d$. Dann definiert
    \begin{align*}
        \Norm{f}_{\infty} = \sup \geschwungeneklammer{\Norm{f(x)}_{2} \ | \ x \in K}
    \end{align*}
    eine Norm auf $C(K,\R^d)$ und $C(K,\R^d)$ ist bezüglich dieser Norm vollständig.
    Eine Folge $\fFolge$ in $C(K,\R^d)$ konvergiert bezüglich der Norm $\standardNorm_{\infty}$
    gegen $f \in C(K,\R^d)$ genau dann, wenn $\fFolge$ gleichmässig gegen $f$ konvergiert,
    das heisst, wenn es zu jedem $\epsilon>0$ ein \NinN gibt, so dass für alle \ninN
    mit $n \geq N$ und alle $x \in K$ die Abschätzung $\Norm{f_n (x) - f(x)}_2 < \epsilon$
    gilt. Die Norm $\standardNorm_{\infty}$ wird \fat{Supremumsnorm} genannt. Die
    davon induzierte Topologie auf $C(K)$ heisst \fat{Topologie der gleichmässigen
    Konvergenz}.
}

\vspace{1\baselineskip}

\Definition{

    Sei $D$ eine Menge, und sei $V$ ein Unterraum des Vektorraums aller $\R^d$-wertigen
    Funktionen auf $D$. Die \fat{Topologie der punktweisen Konvergenz} ist die
    Topologie auf $V$, deren offene Mengen $U \subseteq V$ wie folgt charakterisiert
    sind: Für jedes $f \in U$ existiert eine endliche Teilmenge $T \subseteq D$ und
    $\epsilon > 0$, so, dass die sogenannte \fat{Standardumgebung}
    \begin{align*}
        B(f,T,\epsilon) = \geschwungeneklammer{f' \in V \ | \ \Norm{f(x) - f'(x)}_2 < \epsilon \text{ für alle } x \in T}
    \end{align*}
    in U enthalten ist.
}

\vspace{1\baselineskip}

\Proposition{

    Sei $D$ eine Menge, und sei $V$ der Vektorraum aller reellwertigen Funktionen
    auf $D$. Die Topologie der punktweisen Konvergenz ist hausdorff'sch und
    kompatibel mit der Skalarmultiplikation und Vektorsumme. Eine Folge $\fFolge$
    konvergiert gegen $f \in V$ bezüglich dieser Topologie genau dann wenn die Folge
    von Funktionen $\fFolge$ punktweise gegen $f$ konvergiert.
}

\vspace{1\baselineskip}

\Definition{

    Sei $X$ ein topologischer Raum und sei $V = C(X,\R^d)$ der Vektorraum aller
    $\R^d$-wertigen, stetigen Funktionen auf $X$. Die \fat{Topologie der kompakten
    Konvergenz} ist die Topologie auf $V$, deren offene Mengen $U \subseteq V$ wie
    folgt charakterisiert sind: Für jedes $f \in U$ existiert eine kompakte Teilmenge
    $T \subseteq X$ und $\epsilon>0$, so, dass die sogenannte \fat{Standardumgebung}
    \begin{align*}
        B(f,T,\epsilon) = \geschwungeneklammer{f' \in V \ | \ \Norm{f(x) - f'(x)}_2 < \epsilon \text{ für alle } x \in T}
    \end{align*}
    in $U$ enthalten ist.
}

\vspace{1\baselineskip}

\Proposition{

    Sei $X$ ein topologischer Raum, und sei $V$ der Vektorraum aller stetigen 
    reellwertigen Funktionen auf $X$. Die Topologie der kompakten Konvergenz ist
    hausdorff'sch und kompatibel mit der Skalarmultiplikation und Vektorsumme. Eine
    Folge $\fFolge$ konvergiert gegen $f \in V$ bezüglich dieser Topologie genau dann
    wenn für jede kompakte Teilmenge $T \subseteq X$ die Folge von Funktionen
    $(f_n |_T)_{n=0}^{\infty}$ auf $T$ gleichmässig gegen $f |_T$ konvergiert.
}


\vspace{2\baselineskip}

\subsection{Zusammenhang}

\vspace{1\baselineskip}

\Definition{

    Sei $X$ ein topologischer Raum. Ein \fat{Weg} oder \fat{Pfad} in $X$ ist eine stetige
    Abbildung $\gamma: [0,1] \rightarrow X$. Wir nennen $\gamma(0)$ den \fat{Startpunkt}
    und $\gamma(1)$ den \fat{Endpunkt}. Dabei sagen wir auch, dass $\gamma$ ein Weg
    von $\gamma(a)$ nach $\gamma(b)$ ist. Einen Pfad $\gamma$ mit $\gamma(0) = \gamma(1)$
    nennen wir \fat{geschlossen}, oder auch eine \fat{Schlaufe}.
}

\pagebreak

\Definition{ (Geometrisch)

    Ein topologischer Raum $X$ heisst \fat{wegzusammenhängend}, falls für alle
    $x_1,x_2 \in X$ eine Pfad $\gamma:[0,1] \rightarrow X$ mit
    $\gamma(0)=x_1$ und $\gamma(1)=x_2$ existiert.
}

\vspace{1\baselineskip}

\Definition{ (Topologisch)

    Ein topologischer Raum $X$ heisst \fat{zusammenhängend}, falls $\emptyset$ und
    $X$ die einzigen Teilmengen von $X$ sind, die offen und abgeschlossen sind.
}

\vspace{1\baselineskip}

\Definition{ (Alternativ)

    Ein topologischer Raum $X$ heisst \fat{zusammenhängend}, falls für alle
    $U_1 \subseteq X$ und $U_2 \subseteq X$ offen, mit $U_1 \cap U_2 = \emptyset$
    und $U_1 \cup U_2 = 0$ direkt folgt, dass $U_1 = \emptyset$ und $U_2 = X$,
    oder $U_2 = \emptyset$ und $U_1 = X$.
}

\vspace{1\baselineskip}

\Definition{

    Eine Teilmenge $U \subseteq X$ eines topologischen Raumes $X$ heisst
    \fat{Zusammenhangskomponente} von $X$, falls $U$ nichtleer, offen, abgeschlossen
    und zusammenhängend ist.
}

\vspace{1\baselineskip}

\Definition{

    Eine Teilmenge $Y \subseteq X$ eines topologischen Raumes heisst \fat{zusammenhängend},
    falls $Y$ bezüglich der Unterraumtopologie als eigenständiger topologischer Raum
    zusammenhängend ist. Im gleichen Sinn sprechen wir von Zusammenhangskomponenten
    von $Y$.
}

\vspace{1\baselineskip}

\Proposition{

    Sei $X$ ein topologischer Raum und seien $Y_1$ und $Y_2$ zusammenhängende Teilräume.
    Falls der Durchschnitt $Y_1 \cap Y_2$ nicht-leer ist, dann ist die Vereinigung
    $Y_1 \cup Y_2$ zusammenhängend.
}

\vspace{1\baselineskip}

\Proposition{

    Sei $X$ ein topologischer Raum und seien $Y_1$ und $Y_2$ wegzusammenhängende Teilräume.
    Falls der Durchschnitt $Y_1 \cap Y_2$ nicht-leer ist, dann ist die Vereinigung
    $Y_1 \cup Y_2$ wegzusammenhängend.
}

\vspace{1\baselineskip}

\Proposition{

    Eine nichtleere Teilmenge $X \subseteq \R$ ist genau dann zusammenhängend, wenn
    $X$ ein Intervall ist.
}

\vspace{1\baselineskip}

\Proposition{

    Seien $X$ und $Y$ topologische Räume und $f: X \rightarrow Y$ eine stetige
    Funktion. Ist $A \subseteq X$ zusammenhängend, dann ist $f(A) \subseteq Y$
    zusammenhängend.
}

\vspace{1\baselineskip}

\Korollar{ (Zwischenwertsatz)

    Sei $I \subseteq \R$ ein Intervall, $f: I \rightarrow \R$ eine stetige Funktion
    und $a,b \in I$. Für jedes $c \in \R$ zwischen $f(a)$ und $f(b)$ gibt es ein
    $x \in I$ zwischen $a$ und $b$, so dass $f(x) = c$ gilt. 
}

\vspace{1\baselineskip}

\Proposition{

    Jeder zusammenhängender topologische Raum ist zusammenhängend.
}

\vspace{1\baselineskip}

\Proposition{

    Sei $U \subseteq \R^d$ eine offene Teilmenge. Dann ist $U$ genau dann wegzusammenhängend,
    wenn $U$ zusammenhängend ist.
}

\vspace{1\baselineskip}

\Korollar{

    Für alle $n \geq 1$ und $r>0$ sind der topologische Raum $\R^n$, sowie die
    Teilräume $B(x,r)$ und $\overline{B(x,r)}$ von $\R^n$, zusammenhängend.
}

\vspace{1\baselineskip}

\Korollar{

    Für alle $n \geq 2$ ist der topologische Raum $\R^n \backslash \geschwungeneklammer{0}$
    zusammenhängend.
}

\vspace{1\baselineskip}

\Definition{

    Sei $X$ ein topologischer Raum und seien $\gamma_0$ und $\gamma_1$ Pfade in $X$
    mit demselben Anfangspunkt $x_0 = \gamma_0 (0) = \gamma_1 (0)$ und demselben
    Endpunkt $x_1 = \gamma_0 (1) = \gamma_1 (1)$. Eine \fat{Homotopie} von $\gamma_0$
    nach $\gamma_1$ ist eine stetige Abbildung 
    
    $H: [0,1] \times [0,1] \rightarrow X$ mit folgenden Eigenschaften:
    \begin{align*}
        H(0,t) = \gamma_0(t), \quad H(1,t) = \gamma_1 (t)
        \quad \quad \text{   und   } \quad \quad
        H(s,0) = x_0 , \quad H(s,1) = x_1
    \end{align*}
    für alle $t \in [0,1]$ und alle $s \in [0,1]$. Wir sagen $\gamma_1$ sei \fat{homotop}
    zu $\gamma_0$, falls es eine Homotopie von $\gamma_0$ nach $\gamma_1$ gibt.
}

\vspace{1\baselineskip}

\Definition{

    Ein topologischer Raum $X$ heisst \fat{einfach zusammenhängend}, falls er
    wegzusammenhängend ist und falls für alle $x_0 , x_1 \in X$ alle Pfade von
    $x_0$ nach $x_1$ homotop zueinander sind.
}

\vspace{1\baselineskip}

\Definition{

    Sei $U \subset \R^n$ eine nichtleere Teilmenge. Wir sagen, dass $U$ \dots
    \begin{enumerate}[{(1)}]
        \item \fat{zusammenhängend} ist, wenn sich die Menge $U$ nicht als disjuknkte
                Vereinigung zweier offener, nichtleerer Teilmengen von $U$ schreiben
                lässt.
        \item \fat{wegzusammenhängend} ist, wenn zu je zwei Punkten $x_0 , x_1 \in U$
                ein Weg in $U$ von $x_0$ nach $x_1$ existiert.
        \item \fat{einfach zusammenhängend} ist, wenn zu je zwei Punkten $x_0 , x_1 \in U$
                ein Weg in $U$ von $x_0$ nach $x_1$ existert, und zwischen zwei solchen
                Wegen eine Homotopie existiert.
        \item \fat{sternförmig} ist, wenn ein $x_0 \in U$ existiert, so dass für alle
                $x_1 \in U$ und $t \in [0,1]$ auch $(1-t)x_0 + t x_1 \in U$ gilt.
        \item \fat{konvex} ist, wenn für alle $x_0 , x_1 \in U$ und alle $t \in [0,1]$
                auch $(1-t)x_0 + tx_1 \in U$ gilt.
    \end{enumerate}
}


\pagebreak

\section{Mehrdimensionale Differentialrechnung}

\vspace{1\baselineskip}

\subsection{Die Ableitung}

\vspace{1\baselineskip}

In diesem Kapitel definieren wir eine allgemeine Teilmenge $D \subseteq \R$ ohne isolierte
Punkte. Das heisst, jedes Element $x \in D$ ist ein Häufungspunkt von $D$.
Ein Beispiel für solch eine Menge $D$ sind Intervalle.
Falls nicht explizit anders erwähnt, sind alle Funktionen als reellwertig vorausgesetzt.

\vspace{2\baselineskip}

\Definition{

    Sei $f: D \rightarrow \R$ eine Funktion und $x_0 \in D$. Wir sagen, dass $f$ bei $x_0$
    \fat{differenzierbar} ist, falls der Grenzwert
    \begin{align*}
        f'(x_0) = \limes{\stackrel{x \rightarrow x_0}{x \neq x_0}} \frac{f(x) - f(x_0)}{x - x_0}
        = \limes{\stackrel{h \rightarrow 0}{h \neq 0}} \frac{f(x_0 + h) - f(x_0)}{h}
    \end{align*}
    existiert. In diesem Fall nennen wir $f'(x_0)$ die \fat{Ableitung} von $f$ bei $x_0$.
    Falls $f$ bei jedem Häufungspunkt von $D$ in $D$ differenzierbar ist, dann sagen wir
    auch, dass $f$ auf $D$ \fat{differenzierbar} ist und nennen die daraus entstehende
    Funktion $f': D \rightarrow \R$ die \fat{Ableitung} von $f$.
    Alternative Notation für die Ableitung von $f$ sind $\frac{\partial}{\partial x} f$,
    $\frac{df}{dx}$ oder auch $Df$. Falls $x_0 \in D$ ein rechtseitiger Häufungspunkt
    von $D$ ist, dann ist $f$ bei $x_0$ \fat{rechtseitig differenzierbar}, wenn die
    \fat{rechtseitige Ableitung}
    \begin{align*}
        f_+' (x_0) = \limes{\stackrel{x \rightarrow x_0}{x > x_0}} \frac{f(x) - f(x_0)}{x - x_0}
        = \limes{\stackrel{h \rightarrow 0}{h > 0}} \frac{f(x_0 + h) - f(x_0)}{h}
    \end{align*}
    existiert. \fat{Linksseitige Differenzierbarkeit} und die \fat{Linksseitige Ableitung}
    $f_-' (x_0)$ werden analog über die Bewegung $x \rightarrow x_0$ mit $x < x_0$ definiert.
}

\vspace{1\baselineskip}

\Definition{

    Eine \fat{affine} Funktion ist eine Funktion der Form $x \mapsto sx + r$, für reelle
    Zahlen $s$ und $r$. Der Graph einer affinen Funktion ist eine nichtvertikale
    \fat{Gerade} in $\R^2$. Der Parameter $s$ in der Gleichung $y = sx + r$ wird die
    \fat{Steigung} genannt. Ist $f: D \rightarrow \R$ differenzierbar an der Stelle
    $x_0 \in D$, so wird die Funktion $x \mapsto f'(x_0)(x-x_0)$ \fat{lineare Approximation}
    von $f$ bei $x_0$ oder \fat{Tangente} von $f$ bei $x_0$ genannt.
}

\vspace{1\baselineskip}

\Definition{

    Sei $f: D \rightarrow \R$ eine Funktion. Wir definieren die \fat{höhere Ableitungen}
    von $f$, sofern sei existieren, durch
    \begin{align*}
        f^{(0)} = f \ , \quad f^{(1)} = f' \ , \quad f^{(2)} = f'' \ , \ \dots \ , \quad f^{(n+1)} = (f^{n})'
    \end{align*}
    für alle $n \in \N$. Falls $f^{(n)}$ für ein \ninN existiert, heisst $f$ \fat{n-mal differenzierbar}.
    Falls die $n$-te Ableitung $f^{n}$ zusätzlich stetig ist, heisst $f$ \fat{n-mal stetig
    differenzierbar}. Die Menge der $n$-mal stetig differenzierbaren Funktionen auf $D$
    bezeichnen wir mit $C^n (D)$. Rekursiv definieren wir für $n \geq 1$
    \begin{align*}
        C^n (D) = \geschwungeneklammer{f:D \rightarrow \R \ | \ f \text{ ist differenzierbar und } f' \in C^{n-1} (D)}
    \end{align*}
    und sagen $f \in C^n (D)$ sei von \fat{Klasse $C^n$}. Schliesslich definieren wir
    \begin{align*}
        C^{\infty} (D) = \bigcap_{n=0}^{\infty} C^n (D)
    \end{align*}
    und bezeichnen Funktionen $f \in C^{\infty} (D)$ als \fat{glatt} oder von \fat{Klasse $C^\infty$}
}

\vspace{1\baselineskip}

\Bemerkung{

    Sei $f:D \rightarrow \R$ eine Funktion. Ist $f$ ableitbar bei $x_0 \in D$, dann ist
    $f$ stetig bei $x_0$.
}

\pagebreak

\Proposition{

    Seien $f,g: D \rightarrow \R$ Funktionen und ableitbar an der Stelle $x_0 \in D$.
    Dann sind $f+g$ und $f \cdot g$ ableitbar bei $x_0$ und es gilt:
    \begin{align*}
        (f + g)' (x_0) &= f' (x_0) + g' (x_0)
        (f \cdot g)' (x_0) &= f' (x_0) \cdot g(x_0) + f (x_0) \cdot g(x_0)
    \end{align*}
}

\vspace{1\baselineskip}

\Korollar{

    Seien $f,g : D \rightarrow \R$ $n$-mal differenzierbar. Dann sind $f+g$ und $f \cdot g$
    ebenso $n$-mal differenzierbar und es gilt $f^{(n)} + g^{(n)} = (f+g)^{(n)}$ sowie
    \begin{align*}
        (fg)^{(n)} = \sum_{k=0}^{n} \binom{n}{k} f^{(k)} g^{(n-k)}
    \end{align*}
    Insbesondere ist jedes skalare Vielfache $n$-mal differenzierbar und $(\alpha f)^{(n)}
    = \alpha f^{(n)}$ für alle $\alpha \in \R$.
}

\vspace{1\baselineskip}

\Korollar{

    Polynomfunktionen sind auf ganz $\R$ differenzierbar.
}

\vspace{1\baselineskip}

\Satz{ (Kettenregel)

    Seien $D,E \subseteq \R$ Teilmengen und sei $x_0 \in D$ ein Häufungspunkt. Sei
    $f: D \rightarrow E$ ein bei $x_0$ differenzierbare Funktion, so dass $y_0 = f(x_0)$
    ein Häufungspunkt von $E$ ist, und sei $g: E \rightarrow \R$ eine bei $y_0$
    differenzierbare Funktion. Dann ist $g \circ f : D \rightarrow \R$ in $x_0$
    differenzierbar und
    \begin{align*}
        (g \circ f)' (x_0) = g' (f(x_0)) f'(x_0)
    \end{align*}
}

\vspace{1\baselineskip}

\Korollar{

    Seien $D,E \subseteq \R$ Teilmengen, so dass jeder Punkt in $D$ respektive $E$ ein
    Häufungspunkt von $D$ respektive $E$ ist. Seien $f: D \rightarrow E$ und
    $g: E \rightarrow \R$ beides $n$-mal differenzierbare Funktionen. Dann ist
    $g \circ f : D \rightarrow \R$ auch $n$-mal differenzierbar.
}

\vspace{1\baselineskip}

\Korollar{ (Quotientenregel)

    Sei $d \subseteq \R$ eine Teilmenge, $x_0 \in D$ ein Häufungspunkt und seien
    $f,g : D \rightarrow \R$ bei $x_0$ differenzierbar. Falls $g(x_0) \neq 0$ ist,
    dann ist auch $\frac{f}{g}$ bei $x_0$ differenzierbar und es gilt:
    \begin{align*}
        \klammer{\frac{f}{g}}' (x_0) = - \frac{g(x_0) f' (x_0) - f(x_0) g'(x_0)}{g(x_0)^2}
    \end{align*}
}

\vspace{1\baselineskip}

\Satz{

    Seien $D , E \subseteq \R$ Teilmengen und sei $f: D \rightarrow E$ eine stetige,
    bijektive Abbildung, deren inverse Abbildung $f^{-1} : E \rightarrow D$ ebenfalls
    stetig ist. Falls $f$ in dem Häufungspunkt $x_0 \in D$ differenzierbar ist und
    $f'(x_0) \neq 0$ gilt, dann ist $f^{-1}$ in $y_0 = f(x_0)$ differenzierbar und es gilt
    \begin{align*}
        (f^{-1})'(y_0) = \frac{1}{f' (x_0)}
    \end{align*}
}


\vspace{2\baselineskip}

\subsection{Höhere Ableitungen und Taylor-Approximationen}

\vspace{1\baselineskip}

\Definition{

    Sei $U \subseteq \R^n$ offen und sei $f:U \rightarrow \R^m$ eine stetige differenzierbare
    Funktion. Dies bedeutet, dass die partielle Ableitung von $f$ auf ganz $U$ existieren
    und stetige Funktionen auf $U$ sind. Die totale Ableitung von $f$ können wir als
    eine stetige Funktion
    \begin{align*}
        Df : U \rightarrow \Hom (\R^n , \R^m)
    \end{align*}
    betrachten. Falls die Vektorwertige Funktion auf $U$ selbst wieder stetig
    differenzierbar ist, so sagen wir, $f$ sei \fat{zweimal stetig differenzierbar}.
    Die Ableitung von $Df$ ist dann eine stetige Funktion
    \begin{align*}
        D^2 f : U \rightarrow \Hom \klammer{\R^n , \Hom (\R^n , \R^m)}
    \end{align*}
    die wir \fat{zweite totale Ableitung} von $f$ nennen.
}

\vspace{1\baselineskip}

\Definition{

    Sei $U \subseteq \R^n$ offen, $f:U \rightarrow \R^m$ eine Funktion und $d \geq 1$.
    Wir sagen, dass $f$ \fat{$d$-mal stetig differenzierbar} ist, falls für alle
    $j_1 , \dots , j_d \in \geschwungeneklammer{1,\dots,n}$ die partielle Ableitung
    \begin{align*}
        \partial_{j_1} \partial_{j_2} \dots \partial_{j_d} f(x)
    \end{align*}
    an jedem Punkt $x \in U$ existiert, und in Abhängigkeit von $x \in U$ eine
    stetige Funktion auf $U$ definiert. Wir schreiben
    \begin{align*}
        C^d (U, \R^m) = \geschwungeneklammer{f: U \rightarrow \R^m \ | \ f \text{ ist $d$-mal stetig differenzierbar}}
    \end{align*}
    für den Vektorraum der $d$-mal stetig differenzierbaren $\R^m$-wertigen Funktionen
    auf $U$. Wir nennen die Funktion $f$ \fat{glatt}, falls $f$ beliebig oft stetig
    differenzierbar ist. Wir schreiben
    \begin{align*}
        C^{\infty} (U,\R^m) = \geschwungeneklammer{f: U \rightarrow \R^m \ | \ f \text{ ist $d$-mal stetig differenzierbar für alle } d\geq 1}
    \end{align*}
    für den Vektorraum der glatten $\R^m$-wertigen Funktionen auf $U$.
}

\vspace{1\baselineskip}

\Satz{ (Satz von Schwarz)

    Sei $U \subseteq \R^n$ offen und $f: U \rightarrow \R$ eine zweimal stetig
    differenzierbare Funktion. Dann gilt $\partial_j \partial_k f = \partial_k \partial_j f$
    für alle $j,k \in \geschwungeneklammer{1,\dots,n}$.
}

\vspace{1\baselineskip}

\Bemerkung{

    Es gilt:
    \begin{align*}
        &\partial_k \partial_j = D^2 f(x) (e_j , e_k) \\
        &\partial_j \partial_k = D^2 f(x) (e_k , e_j)
    \end{align*}
    Von vorher wissen wir, dass die linken Seiten der Gleichungen auch gleich sind,
    und somit folgt, da die bilineare Abbildung symmetrisch ist:
    \begin{align*}
        D^2 f(x) (v,w) = D^2 f(x) (w,v)
    \end{align*}
}

\pagebreak

\Definition{

    Die \fat{Hesse-Matrix} $H(x) = (H_{ij} (x))_{ij} \in \Mat_{n,n} (\R)$ bei
    $x \in U$ einer zweimal stetig differenzierbaren Funktion $f: U \rightarrow \R$
    ist gegeben durch
    \begin{align*}
        H_{ij} (x) = \partial_i \partial_j f(x)
        = \klammer{\frac{\partial^2 f}{\partial x_i \partial x_j} (x)}_{i,j = 1,\dots,n}
        = \begin{pmatrix}
            \frac{\partial^2 f}{\partial x_1 \partial x_1} (x) & \dots & \frac{\partial^2 f}{\partial x_1 \partial x_n} (x) \\
            \vdots & \ddots & \vdots \\
            \frac{\partial^2 f}{\partial x_n \partial x_1} (x) & \dots & \frac{\partial^2 f}{\partial x_n \partial x_n} (x)
        \end{pmatrix}
    \end{align*}
    für $i,j \in \geschwungeneklammer{1,\dots,n}$. Der Satz von Schwarz besagt, dass
    $H(x)$ eine symmetrische Matrix ist.
}

\vspace{1\baselineskip}

\Satz{ (Taylor)

    Sei $U \subseteq \R^n$ offen und $f:U \rightarrow \R$ eine $(d+1)$-mal stetig
    differenzierbare Funktion. Sei $x \in U$ und $h \in \R^n$, so dass $x+t h \in U$
    für alle $t \in [0,1]$. Dann gilt
    \begin{align*}
        f(x+h) = f(x) + \sum_{k=1}^d \frac{1}{k!} D^k f(x) (h,\dots,h) + \int_0^1 \frac{(1-t)^d}{d!} D^{d+1} f(x+th) (h,\dots,h) dt
    \end{align*}
    Man nennt dies die \fat{Taylor Entwicklung mit Restglied} von $f$ an der Stelle $x$.
    Der Hauptterm
    \begin{align*}
        P(h) = f(x) + \sum_{k=1}^d \frac{1}{k!} D^k f(x) (h,\dots,h)
    \end{align*}
    ist gerade wie in der eindimensiomalen Taylor-Approximation eine Polynomiale
    Funktion, diesmal allerdings in $d$ Variablen. Dabei ist $D^k f(x) (h,\dots,h)$
    gerade der homogene Teil vom Grad $k$. Das Integral
    \begin{align*}
        R(h) = \int_0^1 \varphi^{d+1}(t) \frac{(1-t)^d}{d!} dt
    \end{align*}
    heisst \fat{Restglied}. Die Abschätzung $R(h) = O(\Norm{h}^{d+1})$ folgt aus dem
    eindimensiomalen Fall.
}

\vspace{1\baselineskip}

\Proposition{

    Sei $U \subseteq \R^n$ offen, $f:U \rightarrow \R^m$ eine Funktion und sei
    $x_0 \in U$ ein Punkt an dem $f$ differenzierbar ist und ein lokales
    Extremum annimmt. Dann ist $Df(x_0) = 0$.
}

\vspace{1\baselineskip}

\Definition{

    Sei $A \in \Mat_{n,n} (\R)$ symmetrisch. Wir sagen, $A$ sei \fat{positiv definit},
    falls alle Eigenwerte von $A$ positiv (und reell) sind. A ist \fat{negativ definit},
    falls alle Eigenwerte kleiner als $0$ sind. Sonst ist $A$ \fat{indefinit}
}

\vspace{1\baselineskip}

\Korollar{

    Sei $U \subseteq \R^n$ offen, $f: U \rightarrow \R^m$ zweimal stetig differenzierbar,
    und sei $x_0 \in U$ mit $Df(x_0) = 0$. Sei $H(x_0)$ die Hesse Matrix von $f$ im
    Punkt $x_0$.
    \begin{enumerate}
        \item Ist $H(x_0)$ positiv definit, so nimmt $f$ bei $x_0$ ein striktes lokales Minimum an.
        \item Ist $H(x_0)$ negativ definit, so nimmt $f$ bei $x_0$ ein striktes lokales Maximum an.
        \item Ist $H(x_0)$ indefinit und nicht ausgeartet/ nicht singulär (=kein Eigenwert ist $0$), so hat $f$ bei $x_0$ kein lokales Extremum.
    \end{enumerate}
}


\pagebreak

\subsection{Parameterintegrale}

\vspace{1\baselineskip}

\Definition{

    Seien $a<b$ reelle Zahlen, sei $U \subseteq \R^n$ offen, und
    $f: U \times [a,b] \rightarrow \R$ eine Funktion. Ein Integral der Form
    \begin{align*}
        F(x) = \intab f(x,t) dt
    \end{align*}
    wird als \fat{Parameterintegral} bezeichnet. Dabei ist $x$ der \fat{Parameter}
    und $t$ die \fat{Integrationsvariable}. Wir setzen voraus, dass die Funktion $f$
    in $n+1$ Variablen zumindest stetig ist, so, dass insbesondere für jedes fixe
    $x \in U$ die Abbildung $t \mapsto f(x,t)$ stetig, und somit Riemann-integrierbar
    ist.
}

\vspace{1\baselineskip}

\Satz{

    Sei $U \subseteq \R^n$ eine offene Teilmenge, $a<b$ reelle Zahlen und
    $f: U \times [a,b] \rightarrow \R$ stetig. Dann definiert das Parameterintegral
    \begin{align*}
        F(x) = \intab f(x,t) dt
    \end{align*}
    eine stetige Funktion $F: U \rightarrow \R$. Falls die partielle Ableitung
    $\partial_k f$ für $k = 1,\dots,n$ existieren und auf $U \times [a,b]$
    stetig sind, dann ist $F$ stetig differenzierbar, und es gilt
    für alle $x \in U$ und $k \in \geschwungeneklammer{1,\dots,n}$:
    \begin{align*}
        \partial_k F(x) = \intab \partial_k f(x,t) dt
    \end{align*}    
}

\vspace{1\baselineskip}

\Korollar{

   Sei $U \subset \R^n$ offen, seien $a<b$ reelle Zahlen und sei $f: U \times (a,b) \rightarrow \R$
   stetig mit stetigen partiellen Ableitungen $\partial_k f$ für $k \in \geschwungeneklammer{1,\dots,n}$.
   Seien $\alpha,\beta:U \rightarrow (a,b)$ stetig differenzierbar. Dann ist das
   Parameterintegral mit veränderlichen Grenzen
   \begin{align*}
       F(x) = \int_{\alpha(x)}^{\beta(x)} f(t,x) dt
   \end{align*}
   stetig differenzierbar auf $U$, und es gilt für alle $x \in U$
   \begin{align*}
       \partial_k F(x) = f(x,\beta(x)) \partial_k \beta(x) - f(x,\alpha(x)) \partial_k \alpha(x) + \int_{\alpha(x)}^{\beta(x)} \partial_k f(x,t) dt
   \end{align*}
}

\vspace{1\baselineskip}

\Definition{

   Die durch das Parameterintegral
   \begin{align*}
       J_n (x) = \frac{1}{\pi} \int_0^\pi \cos(x \sin(t) - nt) dt
   \end{align*}
   definierte Funktion $J_n : (0, \infty) \rightarrow \R$ wird \fat{Bessel-Funktion
   erster Gattung} genannt, und löst die Differentialgleichung
   \begin{align*}
       x^2 u''(x) + x u'(x) + (x^2 - n^2) u(x) = 0
   \end{align*}
}

\vspace{1\baselineskip}

\Definition{

   Die \fat{Bessel-Funktion zweiter Gattung} ist durch das uneigentliche Integral
   \begin{align*}
       Y_n (x) = \frac{1}{\pi} \int_0^{\pi} \sin(x \sin(t) - nt) x \sin(t) dt
                    - \frac{1}{\pi} \int_0^{\infty} \klammer{\exp (t) + (-1)^n \exp (-n t)} \exp (-x \sinh(t)) dt
   \end{align*}
   für $x \in (0,\infty)$ definiert. Es kann gezeigt werden, dass $Y_n$ die Differentialgleichung
   in der obigen Definition auch erfüllt. Des Weiteren sind $J_n$ und $Y_n$ linear
   unabhängig.
}

\pagebreak

\Definition{

   Sei $f: \R \rightarrow \R$ stetig. Der \fat{Träger} oder \fat{Support} von $f$
   ist definiert als
   \begin{align*}
       \text{supp} (f) = \overline{\geschwungeneklammer{x \in \R \ | \ f(x) \neq 0}}
   \end{align*}
   Die Funktion $f$ hat \fat{kompakten Träger} genau dann, wenn $f(x)=0 \ \forall x \in \R$
   mit $\abs{x}>R$ für ein $R >> 0$. 
}

\vspace{1\baselineskip}

\Definition{

   Sei $f: \R \rightarrow \R$ stetig und sei $\psi: \R \rightarrow \R$ stetig mit
   kompaktem Träger. Die \fat{Faltung} von $f$ mit $\psi$ ist die Funktion
   $\psi \circ f: \R \rightarrow \R$ gegeben durch
   \begin{align*}
       (\psi \circ f)(x) = \intii \psi(x-y) f(y) dy
   \end{align*}
   Falls supp$(\psi) \subseteq [-R,R]$, dann ist folgendes ein eigentliches Integral
   \begin{align*}
       \int_{-R-\abs{x}}^{+R + \abs{x}} \psi (x-y) f(y) dy
   \end{align*}
}

\vspace{1\baselineskip}

\Bemerkung{

   \begin{enumerate}
       \item Es gilt $\psi \circ f = f \circ \psi$.
       \item Die Faltung ist kommutativ, assoziativ und bilinear (distributiv). Das heisst, es bildet einen Ring ohne Einselement.
       \item Ist $f$ stetig und $\psi$ glatt, dann ist $\psi \circ f$ auch glatt und es gilt
                \begin{align*}
                    \frac{\partial}{\partial x} (\psi \circ f)(x) = (\psi' \circ f)(x)
                \end{align*}
   \end{enumerate}
}

\vspace{1\baselineskip}

\Definition{

   Wir nennen \fat{Glättungskern} eine Funktion $\psi: \R \rightarrow \R$ mit
   \begin{enumerate}[{1)}]
       \item $\psi$ ist glatt.
       \item supp$(\psi) \subseteq [- \delta , +\delta]$ für kleines $\delta$.
       \item $\psi(x) \geq 0 \ \forall x \in \R$
       \item $\intii \psi (x) dx = 1$ 
   \end{enumerate}
}

\vspace{1\baselineskip}

\Korollar{

   Ist $\psi$ ein Glättungskern mit supp$(\psi) \subseteq [-\delta,+\delta]$, dann ist
   $2 \psi (2x)$ ein Glättungskern mit Support in $[-\frac{\delta}{2},\frac{\delta}{2}]$.
   Im Allgemeinen gilt:
   \begin{align*}
       \psi_n = 2^n \psi(2^n x) \subseteq [-\delta \cdot s^{-n} , \delta \cdot 2^{-n}]
   \end{align*}
   Setzen wir $f_n = \psi_n \circ f: \psi_n (x) = 2^n \psi (2^n x)$, so konvergiert
   $\fFolge$ gleichmässig gegen $f$ auf jedem kompakten Intervall.
}


\pagebreak

\subsection{Wegintegrale}

\vspace{1\baselineskip}

\Definition{

    Sei $U \subseteq \R^n$ eine offene Teilmenge und $\gamma: [a,b] \rightarrow U$
    ein stetig differenzierbarer Weg. Wir definieren die \fat{Länge} von $\gamma$ als
    \begin{align*}
        L(\gamma) = \intab \Norm{\gamma'(t)} dt
    \end{align*}
    Hier soll $\Norm{\gamma'(t)}$ als \fat{Geschwindigkeit} des Weges zur Zeit $t$
    gelesen werden.
}

\vspace{1\baselineskip}

\Definition{

    Eine stetige Funktion $\gamma:[a,b] \rightarrow \R^n$ heisst \fat{stückweise
    stetig differenzierbar}, falls eine Zerlegung von $[a,b]$: $a=s_0 < s_1 < \dots < s_N = b$
    existiert, so, dass $\gamma |_{[s_{k-1},s_k]}$ für alle
    $k \in \geschwungeneklammer{1,\dots,N}$ stetig differenzierbar ist. Wir sagen
    in dem Fall $\gamma$ sei \fat{stückweise stetig differenzierbar bezüglich}
    dieser Zerlegung, und definieren die \fat{Länge} von $\gamma$ durch
    \begin{align*}
        L(\gamma) = \sum_{k=1}^{\N} \int_{S_{k-1}}^{s_k} \Norm{(\gamma |_{[s_{k-1},s_k]})'(t)} dt
    \end{align*}
}

\vspace{1\baselineskip}

\Definition{

    Sei $f: U \rightarrow \R$ eine stetige reellwertige Funktion. Entlang eines
    stetigen differenzierbaren Weges $\gamma:[a,b] \rightarrow U$ definieren wir
    das \fat{skalare Wegintegral} als
    \begin{align*}
        \intab f(\gamma(t)) \Norm{\gamma'(t)} dt
    \end{align*}
}

\vspace{1\baselineskip}

\Definition{

    Wie benutzen den Term \fat{Gebiet} für offene zusammenhängende Teilmengen
    von $\R^n$.
}

\vspace{1\baselineskip}

\Definition{

    Sei $U \subseteq \R^n$ offen. Ein \fat{Vektorfeld} auf $U$ ist eine Funktion
    $F: U \rightarrow \R^n$. Stetige, stetig differenzierbare oder glatte
    Vektorfelder sind dementsprechend Funktionen $F: U \rightarrow \R^n$.
}

\vspace{1\baselineskip}

\Definition{

    Sei $U \subseteq \R^n$ ein offene Teilmenge und sei $F: U \rightarrow \R^n$ ein
    stetiges Vektorfeld. Wir definieren das \fat{Arbeitsintegral/Wegintegral des
    Vektorfeldes} $F$ entlang eines stückweise stetig differenzierbaren Weges
    $\gamma: [a,b] \rightarrow U$ durch
    \begin{align*}
        \int_{\gamma} F dt = \intab \scalprod{F(\gamma(t))}{\gamma'(t)} dt
    \end{align*}
    Ist $\gamma$ stückweise stetig differenzierbar bezüglich einer Zerlegung
    $a = t_0 < t_1 < \dots < t_N = b$, so ist das Integral als Summe von
    Integralen über Intervalle $[t_{k-1} , t_k]$ zu lesen.
}

\vspace{1\baselineskip}

\Definition{

    Eine stetige, orientierungserhaltende Reparametrisierung von $\gamma:[a,b]
    \rightarrow U$ ist eine Verknüpfung $\gamma \circ \psi$ mit
    \begin{align*}
        [c,d] \stackrel{\psi}{\longrightarrow} [a,b] \stackrel{\gamma}{\longrightarrow} U
    \end{align*}
    wobei $\psi$ stetig ist und $\psi(c) = a$ und $\psi(d) = b$.
}

\vspace{1\baselineskip}

\Lemma{

    Sei $U \subseteq \R^n$ eine offene Teilmenge, $f: U \rightarrow \R^n$ ein
    stetiges Vektorfeld und sei $\gamma:[a,b] \rightarrow \R^d$ ein stetig
    differenzierbarer Weg. Dann ändert sich der Wert des Wegintegrals
    $\int_{\gamma} F dt$ nicht unter orientierungserhaltenden Reparametrisierungen
    von $\gamma$.
}

\vspace{1\baselineskip}

\Lemma{

    Sei $U \subset \R^n$ offen und sei $\gamma:[0,1] \rightarrow U$ ein stückweise
    stetig differenzierbarer Weg. Dann gibt es eine stetig differenzierbare
    Reparametrisierung $\varphi = \gamma \circ \psi$. Darüberhinaus kann man
    $\varphi'(0) = \varphi'(1)=0$ einrichten.

}

\vspace{1\baselineskip}

\Definition{

    Sei $U \subseteq \R^n$ offen und $F: U \rightarrow \R^n$ ein stetiges
    Vektorfeld. Eine stetige differenzierbare Funktion $f: U \rightarrow \R$
    heisst \fat{Potential} für $F$, falls $F = \grad (f)$ gilt.
    Ist $f$ ein Potential, so ist auch $f+c$ für $c \in \R$ ein Potential. Ist $U$
    zusammenhängend, so ist jedes weitere Potential von dieser Form.
}

\vspace{1\baselineskip}

\Proposition{

    Sei $U \subseteq \R^n$ offen und sei $F:U \rightarrow \R^n$ ein stetiges
    Vektorfeld. Angenommen es existiert ein Potential $f:U \rightarrow \R$ für $F$.
    Dann gilt:
    \begin{align*}
        \int_{\gamma} F dt = f(\gamma(1)) - f(\gamma(0))
    \end{align*}
    für jeden stückweise stetig differenzierbaren Pfad $\gamma:[0,1] \rightarrow U$.
}

\vspace{1\baselineskip}

\Definition{

    Sei $U \subseteq \R^n$ offen und sei $F:U \rightarrow \R^n$ ein stetiges
    Vektorfeld. Dann heisst $F$ \fat{konservativ}, falls für alle stückweise
    stetig differenzierbaren Wege $\gamma:[0,1] \rightarrow U$ und
    $\eta:[0,1] \rightarrow U$ die Implikation
    \begin{align*}
        \gamma(0) = \eta(0)
        \quad \text{   und   } \quad
        \gamma(1) = \eta(1)
        \quad \Longrightarrow \quad
        \int_{\gamma} F dt = \int_{\eta} F dt
    \end{align*}
    gilt.
}

\vspace{1\baselineskip}

\Satz{

    Sei $U \subseteq \R^n$ ein Gebiet und $F: U \rightarrow \R^n$ ein stetiges
    Vektorfeld. Dann ist $F$ genau dann konservativ, wenn $F$ ein Potential besitzt.
}

\vspace{1\baselineskip}

\Korollar{

    Sei $U \subseteq \R^n$ offen, und sei $F$ ein stetig differenzierbares
    konservatives Vektorfeld auf $U$, mit Komponenten $F_1 , \dots , F_n$.
    Dann gilt
    \begin{align*}
        \partial_j F_k = \partial_k F_j
    \end{align*}
    für alle $j,k \in \geschwungeneklammer{1,\dots,n}$. Man nennt diese Gleichung
    auch \fat{Integrabilitätsbedingung}. Ausgeschrieben gilt dann
    \begin{align*}
        \partial_j F_k = \partial_j \partial_k f
        = \partial_k \partial_j f = \partial_k F_j
    \end{align*}
}

\vspace{1\baselineskip}

\Satz{

    Sei $U \subseteq \R^n$ offe, und sei $F: U \rightarrow \R^n$ ein stetig
    differenzierbares Vektorfeld, das den Integrabilitätsbedingung
    \begin{align*}
        \partial_k F_j = \partial_k F_j
    \end{align*}
    für alle $j,k \in \geschwungeneklammer{1,\dots,n}$ genügt. Seien
    $\gamma_0 :[0,1] \rightarrow U$ und $\gamma_1 : [0,1] \rightarrow U$ stückweise
    stetig differenzierbare Pfade mti dem selben Anfangspunkt $x_0$ und dem selben
    Endpunkt $x_1$. Sind $\gamma_0$ und $\gamma_1$ homotop, so gilt
    \begin{align*}
        \int_{\gamma_0} F dt = \int_{\gamma_1} F dt
    \end{align*}
}

\pagebreak

\Korollar{

    Sei $U \subseteq \R^n$ offen und einfach zusammenhängend. Ein stetig differenzierbares
    Vektorfeld auf $U$ ist genau dann konservativ, falls es den Integrabilitätsbedingungen
    genügt.
}

\vspace{1\baselineskip}

\Lemma{

    Sei $U \subseteq \R^n$ offen und konvex, und sei $F: U \rightarrow \R^n$ ein
    stetig differenzierbar Vektorfeld, das den Integrabilitätsbedingungen genügt.
    Dann ist $F$ konservativ.
}

\vspace{1\baselineskip}

\Definition{

    Sei $U \subseteq \R^n$ offen und $F:U \rightarrow \R^n$ ein stetiges Vektorfeld.
    Sei $V \subseteq \R^m$ offen und $\varphi:V \rightarrow U$ eine stetig
    differenzierbare Funktion. Das Vektorfeld $\varphi^{*} F$ auf $V$, gegeben durch
    \begin{align*}
        \varphi^{*}F : x \mapsto \sum_{k=1}^{m} \scalprod{\partial_k \varphi (x)}{F(\varphi(x))} e_k
    \end{align*}
    heisst \fat{Pullback} von $F$, oder das entlang $\varphi$ \fat{zurückgezogene}
    Vektorfeld. Die dadurch entstehende Abbildung
    \begin{align*}
        \varphi^{*}: \geschwungeneklammer{\text{Stetige Vektorfelder auf } U}
        \rightarrow \geschwungeneklammer{ \text{Stetige Vektorfelder auf } V}
    \end{align*}
    nennen wir \fat{Zurückziehen} von Vektorfeldern.
}

\vspace{1\baselineskip}

\Proposition{

    Seien $U \subseteq \R^n$ und $V \subseteq \R^m$ und $\varphi: V \rightarrow U$
    offen und sei $\varphi:V \rightarrow U$ eine stetig differenzierbare Funktion.
    \begin{enumerate}[{(1)}]
        \item Die vorher betrachtete Abbildung $\varphi^{*}$ ist linear.
        \item Sei $W \subseteq \R^p$ offen und $\psi:W \rightarrow V$ stetig differenzierbar.
                Dann gilt $\psi^{*} \varphi^{*} F = (\varphi \circ \psi)^{*} F$ für jedes
                stetige Vektorfeld $F$ auf $U$. Ausserdem gilt $\text{id}_U^{*} F = F$.
        \item Sei $\gamma:[0,1] \rightarrow V$ ein stückweise stetig differenzierbarer Weg.
                Dann gilt für jedes stetige Vektorfeld $F$ auf $U$
                \begin{align*}
                    \int_{\gamma} \varphi^{*} F dt = \int_{\varphi \circ \gamma} F dt
                \end{align*}
        \item Ist $F$ ein Vektorfeld von Klasse $C^k$ auf $U$ und $\varphi:V \rightarrow U$
                von Klasse $C^{k+1}$, dann ist $\varphi F$ von Klasse $C^k$.
        \item Sei $F$ ein Vektorfeld von Klasse $C^1$ auf $U$ und $\varphi:V \rightarrow U$
                von Klasse $C^2$. Erfüllt $F$ die Integrabilitätsbedingungen, dann erfüllt
                $\varphi^{*}F$ ebenfalls die Integrabilitätsbedingungen.
    \end{enumerate}
}

\vspace{1\baselineskip}

\Lemma{

    Sei $f:\R^n \rightarrow \R^m$ stetig mit kompaktem Träger $K =
    \geschwungeneklammer{x \in \R \ | \ f(x) \neq 0}$, sei $\epsilon>0$ und $\delta>0$.
    Es existiert eine glatte Funktion $\tilde{f}:\R^n \rightarrow \R^m$ derart, dass
    für alle $x \in \R^n$ folgendes gilt
    \begin{align*}
        \Norm{f(x)-\tilde{f}(x)} < \epsilon
        \quad \quad \text{     und     } \quad \quad
        B(x,\delta) \cap K = \emptyset \Longrightarrow \tilde{f}(x) = 0
    \end{align*}
}

\vspace{1\baselineskip}

\Lemma{

    Seien $\gamma_0 , \gamma_1 \in \Omega$. Sind die Pfade $\gamma_0$ und $\gamma_1$
    homotop, so existiert ein Pfad $\varpi: [0,1] \rightarrow \Omega$ mit
    $\varphi(0) = \gamma_0$ und $\varphi(1) = \gamma_1$.
}


\pagebreak

\section{Anfänge der Differentialgeometrie}

\vspace{1\baselineskip}

\subsection{Sätze zur impliziten Funktion und zur Inversen Abbildung}

\vspace{1\baselineskip}

\Satz{ (Satz der impliziten Funktionen)

    Sei $U \subseteq \R^n \times \R^m$ offen, sei $(x_0 , y_0) \in U$ und sei
    $F: U \rightarrow \R^m$ eine stetige Funktion, die die folgende Bedingungen erfüllt.
    \begin{enumerate}
        \item $F(x_0 , y_0) = 0$
        \item Für alle $k = 1 ,\dots,m$ existiert die partielle Ableitung
                $\partial_{y_k} F: U \rightarrow \R^m$ und ist stetig.
        \item Die Matrix $A = \klammer{\partial_{y_k} F_j (x_0 , y_0)}_{j,k}
                \in \Mat_{m,m} (\R)$ ist invertierbar.
    \end{enumerate}
    Dann existiert $r>0$ und $s>0$ und eine stetige Funktion $f:B(x_0,r) \rightarrow
    B(y_0,s)$, so dass für alle $(x,y) \in B(x_0,r) \times B(y_0,s)$ die Gleichung
    $F(x,y) = 0$ genau dann gilt, wenn $y = f(x)$ gilt.
}

\vspace{1\baselineskip}

\Satz{ (Ableitung der impliziten Funktionen)

    Sei $U \subseteq \R^n \times \R^m$ offen, sei $(x_0 , y_0) \in U$ und sei
    $F: U \rightarrow \R^m$ eine stetige Funktion mit den Eigenschaften aus dem
    obigen Satz und sei $f: B(x_0,r) \rightarrow B(y_0,s)$ die stetige lokale
    Lösungsfunktion aus dem obigen Satz. Angenommen $F$ ist $d$-mal stetig
    differenzierbar für $d \geq 1$. Dann ist die Funktion $f$ ebenso $d$-mal
    stetig differenzierbar und die Ableitung von $f$ bei $x \in B(x_0,r)$ ist durch
    \begin{align*}
        Df(x) = - \klammer{(D_y F)(x,f(x))}^{-1} \circ (D_x F)(x,f(x))
    \end{align*}
    gegeben. Hier bedeutet $D_x F(x_1 , y_1)$ die totale Ableitung der Funktion
    $x \mapsto F(x,y_1)$ am Punkt $x_1$ und $D_y F(x_1,y_1)$ die totale Ableitung
    der Funktion $y \mapsto F(x_1,y)$ am Punkt $y_1$. Es gilt:
    $Df(x):\R^n \rightarrow \R^m$, $D_y F(x,f(x)): \R^m \rightarrow \R^m$ und
    $D_x F(x,f(x)): \R^n \rightarrow \R^m$.
}

\vspace{1\baselineskip}

\Satz{ (Satz zur inversen Abbildung)

    Sei $U \subseteq \R^n$ offen und $f: U \rightarrow \R^n$ eine $d$-mal stetig
    differenzierbare Funktion mit $d \geq 1$. Sei $x_0 \in U$ so, dass $Df(x_0)$
    invertierbar ist. Dann gibt es eine offene Umgebung $U_0 \subseteq U$ von $x_0$
    und eine offene Umgebung $V_0 \subseteq \R^n$ von $y_0 = f(x_0)$, so dass
    $f |_{U_0} : U_0 \rightarrow V_0$ bijektiv ist, und die Umkehrabbildung ebenso
    $d$-mal stetig differenzierbar ist. Des weiteren gilt für alle $x \in U_0$ und
    $y = f(x) \in V_0$
    \begin{align*}
        (Df^{-1})(y) = (Df(x))^{-1}
    \end{align*}
    ("Ableitung der Inverse = Inverse der Ableitung")
}

\vspace{1\baselineskip}

\Definition{

    Seien $U,V \subseteq \R^n$ offen. Eine bijektive, glatte Funktion $f:U \rightarrow V$
    mit glatter Inversen $f^{-1}: V \rightarrow U$ wird \fat{Diffeomorphismus} genannt.
    Sind $f$ und $f^{-1}$ jeweils nur $d$-mal stetig differenzierbar für $d \geq 1$,
    so nennen wir $f$ einen \fat{$C^d$-Diffeomorphismus}.
}

\vspace{1\baselineskip}

\Korollar{

    Sei $U \subseteq \R^n$ offen und sei $f: U \rightarrow \R^n$ eine $d$-mal stetig
    differenzierbare, injektive Funktion mit $d \geq 1$. Angenommen $Df(x)$ sei für
    jeden Puntk $x \in U$ invertierbar. Dann ist $V = f(U) \subseteq \R^n$ offen
    und $f: U \rightarrow V$ ist ein $C^d$-Diffeomorphismus mit
    \begin{align*}
        (Df^{-1})(y) = (Df(x))^{-1}
    \end{align*}
    für alle $x \in U$ und $y = f(x) \in V$.
}


\pagebreak

\subsection{Teilmannigfaltigkeiten des Euklidischen Raumes}

\vspace{1\baselineskip}

\Definition{

    Sei $0 \leq k \leq n$ für $n \geq 1$. Eine Teilmenge $M \subseteq \R^m$ ist eine
    $k$-dimensionale \fat{Teilmannigfaltigkeit}, falls für jeden Punkt $p \in M$
    eine offene Umgebung $U_p$ in $\R^n$ von $p$ und ein Diffeomorphismus
    $\varphi_p : U_p \rightarrow V_p = \varphi_p (U_p)$ auf eine weitere offene
    Teilmenge $V_p \subseteq \R^n$ existiert, so dass
    \begin{align*}
        \varphi_p (U_p \cap M)
        = \geschwungeneklammer{y \in V_p \ | \ y_i = 0 \text{ für alle } i>k}
        = V_p \cap \R^k \times \geschwungeneklammer{0}^{n-k}
    \end{align*}
    gilt. Wir nennen den Diffeomorphismus $\varphi$ eine \fat{Karte} von $M$ um $p$,
    und den zu $\varphi$ inversen Diffeomorphismus $\varphi^{-1}:V \rightarrow U_p$
    eine \fat{Parametrisierung} von $M$ um $p$. Eine Familie von Karten
    $(U_i , V_i , \varphi_i)_{i \in I}$ nennen wir Atlas, falls jeder Punkt von $M$
    im Definitionsbereich eine Karte ist.
}

\vspace{1\baselineskip}

\Proposition{

    Eine Teilmenge $M \subseteq \R^n$ ist genau dann eine $k$-dimensionale
    Teilmannigfaltigkeit, wenn es zu jedem Punkt $p \in M$ eine offene Umgebung
    $U_p$ von $p$ in $\R^n$, eine glatte Funktion $f_p : \tilde{U}_p \rightarrow \R^{n-k}$
    auf einer offenen Teilmenge $\tilde{U}_p \subseteq \R^k$ und eine Permutation
    $\sigma \in \mathcal{S}_n$ gibt, so dass
    \begin{align*}
        M \cap U_p = P_{\sigma} (\text{Graph}(f_p))
    \end{align*}
    gilt. Hierbei bezeichnet $P_{\sigma} : \R^n \rightarrow \R^n$ die von $\sigma$
    induzierte Permutation der Koordinaten.
}

\vspace{1\baselineskip}

\Satz{ (Satz vom konstanten Rang)

    Sei $U \subseteq \R^n$ offen, und $F: U \rightarrow \R^m$ eine glatte Funktion.
    Die Nullstellenmenge $M = \geschwungeneklammer{p \in U \ | \ F(p) = 0}$ ist eine
    $(n-m)$-dimensionale Teilmannigfaltigkeite von $\R^n$, falls für alle $p \in M$
    die lineare Abbildung $DF(p): \R^n \rightarrow \R^m$ surjektiv ist.
    Das heisst, die Jacobi-Matrix muss vollen Rang haben, also Rang $m$, für alle
    $p \in M$.
}

\vspace{1\baselineskip}

\Definition{

    Sei $U \subset \R^n$ offen und sei $f: U \rightarrow \R^m$ eine differenzierbare
    Funktion. Ein Punkt $x \in U$ heisst \fat{kritischer Punkt} von $f$, falls
    $Df(x)$ Rang kleiner als $\min (m,n)$ hat, andernfalls nennt man $x \in U$ einen
    \fat{regulären Punkt} der Abbildung $f: U \rightarrow \R^m$. Das Bild eines
    kritischen Punktes unter $f$ nennt man einen \fat{kritischen Wert}; Punkte in
    $\R^m$ im Komplement der kritischen Werte von $f$ heissen \fat{reguläre Werte}.
}

\vspace{1\baselineskip}

\Definition{

    Sei $M \subseteq \R^n$ eine $k$-dimensionale Teilmannigfaltigkeit. Der
    \fat{Tangentialraum} von $M$ bei $p \in M$ ist durch
    \begin{align*}
        T_p M = \geschwungeneklammer{\gamma'(0) \ | \ \gamma: (-\delta , \delta) \rightarrow M \ \text{ differenzierbar mit } \gamma(0) = p \ , \ \delta>0}
    \end{align*}
    definiert. Das \fat{Tengentialbündel} von $M$ ist durch
    \begin{align*}
        TM = \geschwungeneklammer{(p,v) \in \R^n \times \R^m \ | \ p \in M \ , \ v \in T_p M}
    \end{align*}
    Die Abbildung $\pi: T M \rightarrow M$ gegeben durch $\pi (p,v) = p$
    heisst \fat{kanonische Projektion}, und die Abbildung $0_M : M \rightarrow T M$
    gegeben durch $0_M (p) = (p,0)$ heisst \fat{Nullschnitt}. Eine Abbildung
    $s: M \rightarrow TM$ heisst \fat{Schnitt} von $TM$ oder auch \fat{Vektorfeld}
    auf $M$, falls $\pi \circ s = $ id$_M$ gilt.
}

\vspace{1\baselineskip}

\Satz{

    Sei $M \subseteq \R^n$ eine $k$-dimensionale Teilmannigfaltigkeit. Seien
    $U,V \subseteq \R^n$ offen und sei $\psi: V \rightarrow U$ ein Diffeomorphismus
    mit $U \cap M = \psi (V \cap \R^k)$. Dann ist die Abbildung
    $T \psi : (V \cap \R^k) \times \R^k \rightarrow T(M \cap U)$ gegeben durch
    \begin{align*}
        T \psi (y,h) = \klammer{\psi(y),D \psi (y)(h)}
    \end{align*}
    eine Bijektion. Insbesondere ist für $p = \psi (y_0)$ der Tangentialraum
    $T_p M = $Im$D \psi (y_0)$ ein $k$-dimensionaler Untervektorraum von $\R^n$.
}

\vspace{1\baselineskip}

\Proposition{

    Sei $U \subseteq \R^n$ offen, $F: U \rightarrow \R^m$ eine glatte Funktion und
    $M = F^{-1} (0)$. Falls für alle $p \in M$ die lineare Abbildung
    $DF(p): \R^n \rightarrow \R^m$ surjektiv ist, so ist das Tengentialbündel
    der Mannigfaltigkeit $M$ gegeben durch
    \begin{align*}
        TM = \geschwungeneklammer{(p,v) \in U \times \R^n \ | \ F(p) = 0 \text{  und  } DF(p)(v) = 0}
    \end{align*}
}


\vspace{1\baselineskip}

\subsection{Extremwertprobleme}

\vspace{1\baselineskip}

\Definition{

    Sei $M \subseteq \R^n$ eine Teilmannigfaltigkeit der Dimension $k$, und $p \in M$.
    Der Raum der \fat{Normalvektoren} an $M$ bei $p$ ist definiert als das
    orthogonale Komplement
    \begin{align*}
        N_p M = (T_p M)^{\perp} = \geschwungeneklammer{w \in \R^n \ | \ \scalprod{w}{v} = 0 \ \text{ für alle } v \in T_p M}
    \end{align*}
    zum Tangentialraum $T_p M \subseteq \R^n$. Dies ist ein Vektorraum der Dimension
    $n-k$. Als \fat{Normalenbündel} bezeichnen wir
    \begin{align*}
        N M = (T M)^{\perp} = \geschwungeneklammer{(p,v) \in \R^n \times \R^m \ | \ p \in M \ , \ w \in (T_p M)^{\perp}}
    \end{align*}
    und definieren wie auch für das Tangentialbündel die kanonische Projektion und
    den Nullschnitt.
}

\vspace{1\baselineskip}

\Bemerkung{

    Sei $M$ als Niveaumenge gegeben und $M = F^{-1} (0)$, mit $F: U \rightarrow \R^m$,
    so, dass $0$ ein regulärer Wert ist. Sei $p \in M$. Dann ist also $DF(p): \R^n
    \rightarrow \R^m$ surjektiv (hat vollen Rang) und $T_p M = $Ker$DF(p)$.
    Konkret:
    \begin{align*}
        DF(p) = \begin{pmatrix}
            \grad F_1 (p) \\ \vdots \\ \grad F_m (p)
        \end{pmatrix}
    \end{align*}
    Also folgt:
    \begin{align*}
        v \in T_p M \Leftrightarrow DF(p) v = 0 \Leftrightarrow \scalprod{\grad F_i (p)}{v} = 0 \ \forall i = 1,\dots,m
    \end{align*}
    Also folgt $\grad F_i (p) \in N_p M \ \forall i = 1,\dots,m$. Da $DF(p)$ Rang$=m$
    hat, also vollen Rang, bilden die Vektoren $\geschwungeneklammer{\grad F_i (p) \ | \ i = 1,\dots,m}$
    eine Basis von $N_p M$.
}

\vspace{1\baselineskip}

\Proposition{

    Sei $U \subseteq \R^n$ offen und $M \subseteq U$ eine Teilmannigfaltigkeit von
    $\R^n$. Sei $f: U \rightarrow \R$ eine differenzierbare Funktion. Angenommen
    $f |_{M}$ nimmt in $p \in M$ ein lokales Extremum an. Dann ist $\grad f(p)$
    ein Normalvektoren an $M$ bei $p$.
}

\vspace{1\baselineskip}

\Bemerkung{ (Kochrezept für Extremabestimmung)

    Strategie um lokale Extrema von $f: U \rightarrow \R$ auf einer Teilmenge $M$
    zu finden (Extrema unter Nebenbedingungen):
    \begin{enumerate}[{1)}]
        \item Berechne $N_p M$ für (fast) alle $p \in M$.
        \item Finde alle $p \in M$ mit $\grad f(p) \in N_p M$. Diese Punkte $p$ sind
                dann Kandidaten für lokale Extrema.
        \item Alle Punkte $p \in M$ an denen $N_p M$ nicht definiert ist, oder $f$
                nicht differenzierbar ist, sind auch Kandidaten.
        \item Entscheide ad-hoc ob und welche Kandidaten Extrema sind.
    \end{enumerate}
}

\pagebreak

\Definition{ (Lagrange Multiplikatoren)

    Sei $U \subseteq \R^n$ offen, $F: U \rightarrow \R^m$ mit $0$ als regulären
    Wert und definiere $M = F^{-1} (0)$ als eine Mannigfaltigkeit der Dimension
    $k = n-m$. Sei $f: U \rightarrow \R$ von Klasse $C^1$. Die zu $f$ und $F$
    assoziierte \fat{Lagrange Funktion} ist
    \begin{align*}
        &L: U \times \R^m \rightarrow \R \\
        &L(x,\lambda) = f(x) - \scalprod{\lambda}{F(x)} = f(x) - \sum_{j=0}^{n-k} \lambda_j F_j (x)
    \end{align*}
    Die Komponenten $\lambda \in \R^m$ werden \fat{Lagrange-Multiplikatoren} genannt.
}

\vspace{1\baselineskip}

\Korollar{

    Sei $U \subseteq \R^n$ offen und $M = \geschwungeneklammer{x \in U \ | \ F(x) = 0}$
    eine $k$-dimensionale Teilmannigfaltigkeit gegeben als Nullstellenmenge einer
    glatten Funktion $F: U \rightarrow \R^m$ mit regulärem Wert $0$. Sei
    $f: U \rightarrow \R$ eine differenzierbare Funktion, für die $f |_M$ in $p \in M$
    ein lokales Extremum annimmt, und sei $L$ die zu $F$ und $f$ gehörige
    Lagrange-Funktion. Dann existieren Lagrange-Multiplikatoren $\lambda \in \R^m$,
    so, dass die Gleichungen
    \begin{align*}
        \partial_{x_i} L (p, \lambda) = 0
        \quad \quad \text{   und   } \quad \quad
        \partial_{\lambda_j} L(p,\lambda) = 0
    \end{align*}
    für alle $i \in \geschwungeneklammer{1,\dots,n}$ und $j \in \geschwungeneklammer{1,\dotsm}$
    erfüllt sind.
}


\pagebreak

\section{Mehrdimensionale Integralrechnung}

\vspace{1\baselineskip}

\subsection{Das Riemann-Integral für Quader}

\vspace{1\baselineskip}

\Definition{

    Unter einem \fat{Quader} verstehen wir eine Teilmenge $Q$ von $\R^n$ die ein
    Produkt von Intervallen ist, also
    \begin{align*}
        Q = I_1 \times \dots \times I_n
    \end{align*}
    für Intervalle $I_1 , \dots , I_n \subseteq \R$. Falls die Länge der Intervalle
    $I_1 , \dots , I_n$ alle übereinstimmen, so nennen wir $Q$ euch einen \fat{Würfel}.
    Für $n=2$ sprechen wir auch von Rechtecken.
}

\vspace{1\baselineskip}

\Definition{

    Für beschränkte nichtleere Intervalle $I_1 , \dots , I_n$ ist das \fat{Volumen}
    des Quaders $Q = I_1 \times \dots \times I_n$ in $\R^n$ durch
    \begin{align*}
        \vol (Q) = \prod_{k=1}^{n} (b_k - a_k)
    \end{align*}
    definiert, mit $a_k = \inf I_k$ und $b_k = \sup I_k$.
}

\vspace{1\baselineskip}

\Definition{

    Sei $Q = I_1 \times \dots \times I_n$ ein beschränkter, abgeschlossener
    Quader mit $I_k = [a_k , b_k]$. Unter einer \fat{Zerlegung} von $Q$ verstehen
    wir die Vorgabe einer Zerlegung für jedes Intervall $I_k$. Ist so eine Zerlegung
    gegeben, also
    \begin{align*}
        a_k = x_{k,0} < \dots < x_{k,l(k)} = b_k
    \end{align*}
    für jedes $k$, dann nennen wir für ein $\alpha =
    \klammer{\alpha_1 , \dots , \alpha_n} \in \N^n$ mit $1 \leq \alpha_k \leq l(k)$
    eine \fat{Adresse} zu dieser Zerlegung. Für jede solche Adresse schreiben wir
    \begin{align*}
        Q_{\alpha} = \prod_{k=1}^{n} [x_{k,\alpha_{k}-1} , x_{k,\alpha_k}]
    \end{align*}
    für den entsprechenden abgeschlossenen Teilquader von $Q$. Mit anderen Worten ist
    $Q_{\alpha} \subseteq Q \subseteq \R^n$ die Teilmenge aller jener $(t_1 , \dots , t_n)
    \in \R^n$ für die $x_{k,\alpha_k -1} \leq t_k \leq x_{k,\alpha_k}$ für alle
    $k=1,2,\dots,n$ gilt. Mittels vollständiger Induktion zeigt man die \fat{Additionsformel}
    \begin{align*}
        \vol (Q) = \sum_{\alpha} \vol (Q_{\alpha})
    \end{align*}
    wobei die Summe sich über alle Adressen zur gegebenen Zerlegung erstreckt. Eine
    \fat{Verfeinerung} der Zerlegung ist eine Zerlegung
    \begin{align*}
        a_k = y_{k,0} \leq \dots \leq y_{k,m(k)} = b_k
        \quad \quad \quad k=1,\dots,n
    \end{align*}
    so, dass für jedes fixe $k$ die Zerlegung $a_k = y_{k,0} \leq \dots \leq
    y_{k,m(k)} = b_k$ von $I_k$ eine Verfeinerung von
    $a_k = x_{k,0} \leq \dots \leq x_{k,l(k)} = b_k$ ist. Zu zwei beliebigen
    Zerlegungen von $Q$ existiert stets eine gemeinsame Verfeinerung.
}

\vspace{1\baselineskip}

\Definition{

    Sie $Q \subseteq \R^n$ ein Quader. Eine \fat{Treppenfunktion} auf $Q$ ist eine
    beschränkte Funktion $f: Q \rightarrow \R$ derart, dass die Zerlegung von $Q$
    existiert, so, dass für jede Adresse $\alpha$ die Funktion $f$ konstant auf dem
    offenen Teilquader $Q_{\alpha}^{\circ}$ ist. Wir sagen in dem Fall auch, $f$ sei
    eine Treppenfunktion bezüglich dieser Zerlegung von $Q$. Ist $c_{\alpha}$ der
    konstante Wert von $f$ auf dem offenen Teilquader $Q_{\alpha}^{\circ}$, so
    schreiben wir
    \begin{align*}
        \int_{Q} f(x) dx = \sum_{\alpha} c_{\alpha} \vol (Q_{\alpha})
    \end{align*}
    wobei die Summe sich über alle Adressen zur gegebenen Zerlegung erstreckt.
    Wir nennen diese Zahl \fat{Integral} von $f$ über $Q$.
}

\pagebreak

\Definition{

    Sei $Q \subseteq \R^n$ ein Quader, $\mathcal{T F}$ bezeichne den Vektorraum
    der Treppenfunktionen auf $Q$, und $f: Q \rightarrow \R$ sei eine Funktion.
    Dann definieren wir die Menge der \fat{Untersummen} $\mathcal{U}(f)$ und
    \fat{Obersummen} $\mathcal{O}(f)$ von $f$ durch
    \begin{align*}
        \mathcal{U} (f) = \geschwungeneklammer{\int_Q u \ dx \ | \ u \in \mathcal{TF} \text{ und } u \leq f}
        \quad \quad \quad \quad
        \mathcal{O} (f) = \geschwungeneklammer{\int_Q o \ dx \ | \ o \in \mathcal{TF} \text{ und } f \leq o}
    \end{align*}
    Falls $f$ beschränkt ist, so sind die Mengen nicht leer. Aufgrund der Monotonie
    des Integrals für Treppenfunktionen gilt, falls $f$ beschränkt ist, die Ungleichung
    \begin{align*}
        \sup \mathcal{U} (f) \leq \inf \mathcal{O} (f)
    \end{align*}
}

\vspace{1\baselineskip}

\Definition{

    Sei $Q \subseteq \R^n$ ein Quader und $f: Q \rightarrow \R$ sei eine beschränkte
    Funktion. Wir nennen $\sup \mathcal{U} (f)$ das \fat{untere}, und
    $\inf \mathcal{O} (f)$ das \fat{obere Integral} von $f$. Die Funktion $f$ heisst
    \fat{Riemann-integrierbar}, falls $\sup \mathcal{U} (f) = \inf \mathcal{O} (f)$
    gilt. Der gemeinsame Wert wird in diesem Fall als das \fat{Riemann-Integral}
    von $f$ bezeichnet, und wird wie folgt geschrieben
    \begin{align*}
        \int_Q f(x) dx = \sup \mathcal{U} (f) = \inf \mathcal{O} (f)
    \end{align*}
}

\vspace{1\baselineskip}

\Proposition{

    Sei $f: Q \rightarrow \R$ beschränkt. Die Funktion $f$ ist genau dann
    Riemann-integrierbar, wenn es zu jedem $\epsilon>0$ Treppenfunktionen $u$
    und $o$ auf $Q$ gibt, die folgendes erfüllen
    \begin{align*}
        u \leq f \leq o
        \quad \quad \quad \text{     und     } \quad \quad \quad
        \int_Q (o-u) dx \ < \epsilon
    \end{align*}
}

\Proposition{

    Sie $Q \subseteq \R^n$ ein Quader, und es beizeichne $\mathcal{R}(Q)$ die
    Menge aller Riemann-integrierbaren Funktionen auf $Q$. Dann ist $\mathcal{R}(Q)$
    ein $\R$-Vektorraum bezüglich der Punktweisen Addition und Multiplikation, und
    Integration
    \begin{align*}
        \int_Q : \mathcal{R}(Q) \rightarrow \R
    \end{align*}
    ist eine $\R$-lineare Abbildung. Die Integration ist ausserdem monoton und
    erfüllt die Dreiecksungleichung: Es gilt
    \begin{align*}
        f \leq g \ \Longrightarrow \int_Q f(x) dx \leq \int_Q g(x) dx
        \quad \quad \quad \text{     und     } \quad \quad \quad
        \abs{\int_Q f(x) dx} \leq \int_Q \abs{f(x)} dx
    \end{align*}
    für alle $f,g \in \mathcal{R} (Q)$. Insbesondere ist $\abs{f}$ Riemann-integrierbar.
    Des Weiteren gilt $\mathcal{TF}(Q) \subseteq \mathcal{R}(Q)$.
}

\vspace{1\baselineskip}

\Definition{

    Eine Teilmenge $N \subseteq \R^n$ wird \fat{Lebesgue-Nullmenge} oder einfach
    \fat{Nullmenge} genannt, falls es zu jedem $\epsilon>0$ eine abzählbare
    Familie von offenen Quadern $(Q_k)_{k \in \N}$ in $\R^n$ gibt, so dass
    folgendes gilt
    \begin{align*}
        N \subset \bigcup_{k=0}^{\infty}
        \quad \quad \quad \text{     und     } \quad \quad \quad
        \sum_{k=0}^{\infty} \vol (Q_k) \ < \epsilon
    \end{align*}
}

\Bemerkung{

    Wir sagen, dass eine Aussage $A$ über Elemente $x \in \R^n$ für \fat{fast alle}
    $x \in \R^n$ wahr ist, falls
    \begin{align*}
        \geschwungeneklammer{x \in \R^n \ | \ \neg A (x)}
    \end{align*}
    eine Nullmenge ist.
}

\vspace{1\baselineskip}

\Lemma{

    Eine Teilmenge einer Nullmenge ist eine Nullmenge. Eine abzählbare Vereinigung
    von Nullmengen ist wiederum eine Nullmenge.
}

\pagebreak

\Proposition{

    Eine Menge $X \subseteq \R^n$ mit nichtleerem Inneren ist keine Nullmenge.
}

\vspace{1\baselineskip}

\Proposition{

    Sei $Q \subseteq \R^{n-1}$ ein abgeschlossener Quader und $f: Q \rightarrow \R$
    eine Riemann-integrierbare Funktion. Dann ist der Graph
    $\geschwungeneklammer{(x,f(x)) \ | \ x \in Q} \subseteq \R^n$ von $f$ eine
    Nullmenge. 
}

\vspace{1\baselineskip}

\Satz{ (Lebesgue-Kriterium)

    Eine reellwertige, beschränkte Funktion auf einem abgeschlossenen Quader $Q$ ist
    genau dann Riemann-integrierbar, wenn sie fast überall stetig ist, das heisst,
    falls die folgende Menge eine Nullmenge ist
    \begin{align*}
        N = \geschwungeneklammer{x \in Q \ | \ f \text{ ist unstetig in } x}
    \end{align*}
}

\vspace{1\baselineskip}

\Korollar{

    Sei $Q$ ein abgeschlossener Quader mit nicht-leerem Inneren. Dann ist jede
    stetige Funktion $f: Q \rightarrow \R$ Riemann-integrierbar.
}

\vspace{1\baselineskip}

\Definition{ (Oszillationsmass)

    Sei $f: Q \rightarrow \R$ eine beschränkte Funktion, $x \in \Q$ und $\delta>0$.
    Wir definieren
    \begin{align*}
        \omega(f,x,\delta) := \sup \geschwungeneklammer{f(y) \ | \ y \in B_{\infty} (x,\delta)}
                                - \inf \geschwungeneklammer{f(y) \ | \ y \in B_{\infty} (x,\delta)}
    \end{align*}
    wobei $B_{\infty} (y,\delta)$ für den Ball bezüglich der Supremumsnorm steht.
    So ein Ball ist ein offener achsenparalleler Würfel mit Zentrum $x$ und
    Kantenlänge $2 \delta$. Es gilt $\delta' < \delta \Rightarrow \omega(f,x,\delta')
    \leq \omega(f,x,\delta)$. Wir definieren den Grenzwert
    \begin{align*}
        \omega(f,x) := \limes{\delta \rightarrow 0} \omega(f,x,\delta)
    \end{align*}
    als das \fat{Oszillationsmass}. Es gilt ausserdem $\omega(f,x) = 0 \Leftrightarrow$
    $f$ ist stetig bei $x$.
}

\vspace{1\baselineskip}

\Lemma{

    Sei $f: X \rightarrow \R$ eine beschränkte Funktion auf einem metrischen Raum $X$.
    Für jedes $\eta \geq 0$ ist die Teilmenge $N_{\eta} = \geschwungeneklammer{
    x \in X \ | \ \omega(f,x) \geq \eta} \subseteq \X$ abgeschlossen. 
}

\vspace{1\baselineskip}

\Lemma{

    Sei $K \subseteq Q$ kompakt und sei $\eta >0$ mit $\omega(f,x) \leq \eta \
    \forall x \in K$. Dann existiert für alle $\epsilon>0$ ein $\delta>0$ mit
    $\omega(f,x,\delta) \leq \eta + \epsilon$ für alle $x \in K$.
}

\vspace{1\baselineskip}

\Proposition{

    Sei $Q \subseteq \R^n$ ein abgeschlossener Quader. Eine beschränkte Funktion
    $f:Q \rightarrow \R$ ist genau dann Riemann-integrierbar, wenn es für jedes
    $\epsilon>0$ stetige Funktionen $f_- , f_+ : Q \rightarrow \R$ gibt, die
    folgendes erfüllen
    \begin{align*}
        f_- \leq f \leq f_+
        \quad \quad \quad \text{     und     } \quad \quad \quad
        \int_Q (f_+ - f_-) dx < \epsilon
    \end{align*}
}


\pagebreak

\subsection{Das Riemann-Integral über Jordan-messbare Mengen}

\vspace{1\baselineskip}

\Definition{

    Eine Teilmenge $B$ von $\R^n$ heisst \fat{Jordan-messbar}, falls es einen
    abgeschlossenen Quader $Q$ in $\R^n$ mit $B \subseteq Q$ gibt, so dass die
    charakteristische Funktion $\mathds{1}_B$ auf $Q$ Riemann-integrierbar ist. Das
    \fat{Volumen} oder \fat{Jordan-Mass} von $B$ ist in diesem Fall wie folgt definiert
    \begin{align*}
        \vol (B) = \int_Q \mathds{1}_B dx
    \end{align*}
    Dieses ist unabhängig von der Wahl von $Q$, solange $B \subseteq Q$ gilt.
}

\vspace{1\baselineskip}

\Bemerkung{

    Jordan-Messbarkeit $\neq$ Lebesque-Messbarkeit.
}

\vspace{1\baselineskip}

\Korollar{ (zum Lebesgue-Kriterium)

    Eine Teilmenge $B \subset \R^n$ ist genau dann Jordan-messbar, wenn $B$ beschränkt
    ist und der Rand $\partial B$ eine Nullmenge ist. Sind $B_1 , B_2 \subset \R^n$
    Jordan-messbar, so sind auch $B_1 \cup B_2 , B_1 \cap B_2$ und $B_1 \backslash B_2$
    Jordan-messbar.
}

\vspace{1\baselineskip}

\Proposition{

    Sei $Q \subset R^{n-1}$ ein abgeschlossener Quader und seien $f_- , f_+ :
    Q \rightarrow \R$ Riemann integrierbar, und sei $D \subseteq \Q$ Jordan-messbar.
    Dann ist die folgende Menge Jordan-messbar
    \begin{align*}
        B = \geschwungeneklammer{(x,y) \in \R^n \ | \ x \in D , f_- (x) \leq y \leq f_+ (x)}
    \end{align*}
}

\vspace{1\baselineskip}

\Definition{

    Sei $B \subseteq \R^n$ eine Jordan-messbare Teilmenge und sei $f$ eine reellwertige
    Funktion auf $B$. Dann heisst $f$ \fat{Riemann-integrierbar}, falls es einen
    abgeschlossenen Quader $Q \subseteq \R^n$ mit $B \subseteq Q$ gibt, so dass die
    durch
    \begin{align*}
        f_{!}(x) = \begin{cases}
            f(x) \quad &\text{ falls } x \in B \\
            0 \quad &\text{ falls } x \in Q \backslash B
        \end{cases}
    \end{align*}
    gegebene Funktion $f_{!}:Q \rightarrow \R$ Riemann-integrierbar ist. Wir
    schreiben in diesem Fall
    \begin{align*}
        \int_B f \ dx \ = \ \int_Q f_{!} \ dx
    \end{align*}
    und nennen diese Zahl das \fat{Riemann-Integral} von $f$ über $B$.
    Dieses ist unabhängig von der Wahl von $Q$ und es gilt Linearität, Monotonie
    und Dreiecksungleichung für $\int_B (-) \ dx$.
}

\vspace{1\baselineskip}

\Korollar{ (Lebesgue-Kriterium)

    Sei $B \subseteq \R^n$ Jordan-messbar und sei $f:B \rightarrow \R$ beschränkt.
    Dann ist $f$ genau dann Riemann-integrierbar, wenn $f$ auf $B$ fast überall
    stetig ist, das heisst, wenn die Menge der Unstetigkeitsstellen von $f$ eine
    Nullmenge ist. Insbesondere ist jede beschränkte Funktion auf einer
    Jordan-messbaren Menge Riemann-integrierbar.
}

\vspace{1\baselineskip}

\Proposition{

    Sind $B_1 , B_2 \subseteq \R^n$ Jordan-messbar und $f: B_1 \cup B_2 \rightarrow \R$
    eine Riemann-integrierbare Funktion. Dann sind $f |_{B_1}$ und $f |_{B_2}$
    Riemann-integrierbar und es gilt
    \begin{align*}
        \int_{B_1 \cup B_2} f \ dx = \int_{B_1} f \ dx + \int_{B_2} f\ dx - \int_{B_1 \cap B_2} f \ dx
    \end{align*}
}

\vspace{1\baselineskip}

\Satz{ (Fubini)

    Seien $P \subseteq \R^n$ und $Q \subseteq \R^m$ abgeschlossene Quader, und
    $f: P \times Q \rightarrow \R$ eine Riemann-integrierbare Funktion. Für
    $x \in P$, schreibe $f_x : Q \rightarrow \R$ für die Funktion
    $f_x (y) = f(x,y)$, und definiere
    \begin{align*}
        F_- (x) = \sup \mathcal{U} (f_x)
        \quad \quad \text{   und   } \quad \quad
        F_+ (x) = \inf \mathcal{O} (f_x)
    \end{align*}
    Es existiert eine Nullmenge $N \subseteq P$ so, dass für alle $x \notin N$ die
    Funktion $f_x$ Riemann-integrierbar ist, also
    \begin{align*}
        F_- (x) = F_+ (x) = \int_Q f_x (y) dy = \int_Q f(x,y) dy
    \end{align*}
    gilt. Die Funktion $F_-$ und $F_+$ auf $P$ sind beide Riemann-integrierbar, und
    es gilt
    \begin{align*}
        \int_{P \times Q} f(x,y) d(x,y) = \int_P F_- (x) dx = \int_Q F_+ (x) dx
        = \int_P \int_Q f(x,y) dy dx
    \end{align*}
}

\vspace{1\baselineskip}

\Korollar{

    Sei $Q = [a_1,b_1] \times \dots \times [a_n , b_n]$ ein Quader und sei
    $f:Q \rightarrow \R$ Riemann-integrierbar. Dann gilt
    \begin{align*}
        \int_Q f dx = \int_{a_1}^{b_1} \dots \int_{a_n}^{b_n} f(x_1 , \dots , x_n) dx_n \dots dx_1
    \end{align*}
    falls alle Parameterintegrale existieren. Andernfalls können die Parameterintegrale
    durch Suprema von Untersummen oder auch durch Infima von Obersummen ersetzt werden.
}

\vspace{1\baselineskip}

\Korollar{

    Sei $D \subseteq \R^{n-1}$ eine Jordan-messbare Menge, seien
    $\varphi_- , \varphi_+ : D \rightarrow \R$ Riemann-integrierbar und sei
    $B \subseteq \R^n$ die Jordan-messbare Teilmenge
    \begin{align*}
        B = \geschwungeneklammer{(x,y) \in D \times \R \ | \ \varphi_- (x) \leq y \leq \varphi_+ (x)}
    \end{align*}
    Für jede Riemann-integrierbare Funktion $f$ auf $B$ gilt
    \begin{align*}
        \int_B f(x,y) d(x,y) = \int_D \klammer{\int_{\varphi_- (x)}^{\varphi_+ (x)} f(x,y) dy} dx
    \end{align*}
}

\vspace{1\baselineskip}

\Korollar{ (Prinzip von Cavalieri)

    Sei $B \subseteq [a,b] \times \R^{n-1}$ eine beschränkte und Jordan-messbare Menge.
    Dann gilt
    \begin{align*}
        \vol (B) = \intab \vol (B_t) dt
    \end{align*}
    wobei für $t \in [a,b]$ die Teilmenge $B_t \subseteq \R^{n-1}$ durch
    $B_t = \geschwungeneklammer{y \in \R^{n-1} \ | \ (t,y) \in B}$
    gegeben ist, und für fast alle $t \in [a,b]$ Jordan-messbar ist.
}

\vspace{1\baselineskip}

\Bemerkung{

    Ist $C \subseteq \R^n$ eine Jordan-messbare Teilmenge, $\lambda \in \R$, dann ist
    $\lambda C = \geschwungeneklammer{\lambda x \ | \ x \in C}$ auch messbar und es
    gilt $\vol (\lambda C) = \abs{\lambda}^n \cdot \vol (C)$.
}


\pagebreak

\subsection{Mehrdimensionale Substitutionsregel}

\vspace{1\baselineskip}

\Definition{

    Sei $n \geq 1$ und $U \subseteq \R^n$ offen. Der \fat{Träger} oder \fat{Support}
    einer Funktion $f:U \rightarrow \R$ ist die Menge
    \begin{align*}
        \supp (f) = \overline{\geschwungeneklammer{x \in U \ | \ f(x) \neq 0}} \subseteq U
    \end{align*}
    Achtung! Der Abschluss liegt in $U$. Im Allgemeinen ist dies nicht das selbe wie
    der Abschluss davon in $\R^n$. Wir sagen, dass $f$ \fat{kompakten Träger} hat,
    falls $\supp (f)$ eine kompakte Teilmenge von $U$ ist. Hat $f$ kompakten Träger,
    so ist $\supp (f)$ beschränkt und damit ist $f(x) = 0$ für alle $x \in U \backslash Q$
    für ein grosses Quader $Q$. Wir definieren weiter das Integral
    \begin{align*}
        \int_U f dx = \int_{Q \cap U} f |_{Q \cap U} dx
    \end{align*}
    als Integral über $f$ über einen Quader, der $\supp (f)$ enthält, falls es existiert,
    und nennen in dem Fall $f$ Riemann-integrierbar. Dabei braucht $U$ nicht unbedingt
    beschränkt oder Jordan-messbar zu sein.
}

\vspace{1\baselineskip}

\Satz{

    Seien $X,Y \subseteq \R^n$ offene Teilmengen, sei $\Phi: X \rightarrow Y$ ein
    $C^1$-Diffeomorphismus und sie $f: Y \rightarrow \R$ eine Riemann-integrierbare
    Funktion mit kompaktem Träger. Dann ist die Funktion $\Phi^{*} f: X \rightarrow \R$
    gegeben durch $(\Phi^{*} f)(x) = (f \circ \Phi(x)) \cdot \abs{\det (D \Phi (x))}$
    Riemann-integrierbar, hat kompakten Träger, und es gilt
    \begin{align*}
        \int_Y f(y) dy = \int_X (f \circ \Phi (x)) \cdot \abs{\det (D \Phi (x))} dx
    \end{align*}
}

\vspace{1\baselineskip}

\Proposition{

    Seien $X$ und $Y$ offene Teilmengen von $\R^n$ und sei $\Phi: X \rightarrow Y$
    ein Homeomorphismus. Sei $f:Y \rightarrow \R$ eine Funktion und schreibe
    $g = f \circ \Phi$.
    \begin{enumerate}
        \item Ist $f$ Riemann-integrierbar, so ist auch $g$ Riemann-integrierbar.
        \item Es gilt $\supp(g) = \Phi^{-1} (\supp (f))$.
    \end{enumerate}
}

\vspace{1\baselineskip}

\Lemma{

    Sei $T: \R^n \rightarrow \R^n$ eine invertierbare lineare Abbildung die durch eine
    obere Dreiecksmatrix gegeben ist, und sei $f: \R^n \rightarrow \R$ eine stetige
    Funktion mit kompaktem Träger. Dann gilt
    \begin{align*}
        \int_{\R^n} f(x) dx = \abs{\det (T)} \int_{\R^n} f(T(x)) dx
    \end{align*}
}

\vspace{1\baselineskip}

\Lemma{

    Sei $L: \R^n \rightarrow \R^n$ eine invertierbare lineare Abbildung und
    $f: \R^n \rightarrow \R$ eine Riemann-integrierbare Funktion. Dann ist
    $f \circ L$ Riemann-integrierbar, und es gilt
    \begin{align*}
        \int_{\R^n} f(x) dx = \abs{\det (L)} \int_{\R^n} f(L(x)) dx
    \end{align*}
}

\Korollar{

    Sei $L: \R^n \rightarrow \R^n$ eine lineare Abbildung und $B \subseteq \R^n$ eine
    Jordan-messbare Teilmenge. Dann ist $L(B)$ Jordan-messbar, und es gilt
    $\vol(L(B)) = \abs{\det L} \vol (B)$.
}

\pagebreak

\Korollar{

    Sei $L \in \Mat_n (\R)$ mit Spalten $v_1 , \dots , v_n \in \R^n$. Dann ist das
    \fat{Parallelotop}
    \begin{align*}
        P = L([0,1]^n) = \geschwungeneklammer{\sum_{i=1}^{n} s_i v_i \ | \ 0 \leq s_i \leq 1}
    \end{align*}
    Jordan-messbar und es gilt $\vol (P) = \abs{\det (L)} = \sqrt{\text{gram}(v_1 ,\dots, v_n)}$.
    Hierbei ist gram die \fat{Gramsche Determinante}.
}

\vspace{1\baselineskip}

\Lemma{

    Seien $X \subseteq \R^n$ und $Y \subseteq \R^n$ offen, und sei $\Phi:X \rightarrow Y$
    ein $C^1$-Diffeomorphismus. Sei $Q_0 \subseteq X$ ein achsenparalleler abgeschlossener
    Würfel mit Kantenlänge $2r>0$ und Mittelpunkt $x_0 \in X$. Wir setzen
    $y_0 = \Phi(x_0)$, $L = D \Phi (x_0)$ und
    \begin{align*}
        \sigma = \max \geschwungeneklammer{\Norm{D \Phi (x) - L}_{\text{op}} \ | \ x \in Q_0}
    \end{align*}
    Für jede reelle Zahl $s$ mit $\sigma \Norm{L^{-1}}_{\text{op}} \sqrt{n} \leq s < 1$
    gilt
    \begin{align*}
        y_0 + (1-s)L(Q_0 - x_0) \subseteq \Phi (Q_0) \subseteq y_0 + (1+s)L(Q_0-x_0)
    \end{align*}
}

\vspace{1\baselineskip}

\Lemma{

    Seien $X,Y \subseteq \R^n$ offen, sei $\Phi:X \rightarrow Y$ ein $C^1$-Diffeomorphismus
    und sei $K_0 \subseteq X$ eine kompakte Teilmenge. Dann existiert für jedes
    $\epsilon \in (0,1)$ ein $\delta > 0$ mit folgender Eigenschaft. Für jeden Würfel
    $Q_0$ mit Mittelpunkt $x_0$, Kantenlänge kleiner als $\delta$ und $Q_0 \cap K_0
    \neq \emptyset$ gilt
    \begin{align*}
        \frac{\vol (\Phi (Q_0))}{1 + \epsilon} \leq \abs{\det D \Phi (x_0)} \vol (Q_0)
        \leq \frac{\vol (\Phi (Q_0))}{1- \epsilon}
    \end{align*}
}


\vspace{2\baselineskip}

\subsection{Uneigentliche Mehrfachintegrale}

\vspace{1\baselineskip}

\Definition{

    Sei $B \subseteq \R^n$. Eine \fat{Ausschöpfung} von $B$ ist eine Folge Jordan-messbarer
    Teilmengen $(B_m)_{m=0}^{\infty}$ mit
    \begin{align*}
        B_0 \subseteq B_1 \subseteq B_2 \subseteq B_3 \subseteq \dots
        \quad \quad \quad \text{     und     } \quad \quad \quad
        B = \bigcup_{m=0}^{\infty} B_m
    \end{align*}
    und wir nennen $B$ \fat{ausschöpfbar}, falls solch eine Ausschöpfung existiert.
}

\vspace{1\baselineskip}

\Korollar{

    Jede offene Teilmenge $U \subseteq \R^n$ ist ausschöpfbar. Insbesondere gibt
    es eine Ausschöpfung $(K_m)_{m=0}^{\infty}$ von $U$ durch kompakte
    Jordan-messbare Teilmengen.
}

\vspace{1\baselineskip}

\Definition{

    Sei $B \subseteq \R^n$ ausschöpfbar, und sei $f: B \rightarrow \R$ eine Funktion.
    Wir sagen, dass $f$ auf $B$ \fat{uneigentlich Riemann-integrierbar} ist, falls für
    jede Ausschöpfung $(B_{m})_{m=0}^{\infty}$ von $B$ mit der Eigenschaft, dass für
    alle $m \in \N$ die Einschränkung $f |_{B_m}$ Riemann-integrierbar ist, der
    Grenzwert
    \begin{align*}
        \int_B f dx := \limes{m \rightarrow \infty} \int_{B_m} f dx
    \end{align*}
    existeirt und von der Wahl solch einer Ausschöpfung unabhängig ist. Diesen
    Grenzwert nennen wir in dem Fall das \fat{uneigentliche Integral} von $f$
    über $B$.
}

\vspace{1\baselineskip}

\Proposition{

    Sei $B \subseteq \R^n$ eine Jordan-messbare Teilmenge, sei $(B_m)_{m=0^{\infty}}$
    eine Ausschöpfung von $B$ und sei $f: B \rightarrow \R$ eine Riemann-integrierbare
    Funktion. Dann gelten
    \begin{enumerate}[{1)}]
        \item $\vol (B) = \limes{m \rightarrow \infty} vol(B_m)$
        \item $\int_B f dx = \limes{m \rightarrow \infty} \int_{B_m} f dx$
    \end{enumerate}
    Insbesondere ist $f$ über $B$ uneigentlich Riemann-integrierbar, und das
    uneigentliche Riemann-Integral ist das gewöhnliche Riemann-Integral.
}

\vspace{1\baselineskip}

\Satz{

    Sei $B \subseteq \R^n$ eine Teilmenge, sei $f:B \rightarrow \R_{\geq 0}$ eine
    Funktion und sei $(B_m)_{m=0}^{\infty}$ eine Ausschöpfung von $B$ so dass $f |_{B_m}$
    für jedes $m \in \N$ Riemann-integrierbar ist. Falls der Grenzwert
    \begin{align*}
        I = \limes{m \rightarrow \infty} \int_{B_m} f dx
    \end{align*}
    existiert, so ist $f$ uneigentlich Riemann-integrierbar und das uneigentliche
    Riemann-Integral ist gleich $I$.
}

\vspace{1\baselineskip}

\Bemerkung{

    Ist $B$ ausschöpfbar und $f: B \rightarrow \R$ eine Funktion, so ist $f$ uneigentlich
    Riemann-integrierbar, falls
    \begin{align*}
        f_+ = \max (f,0)
        \quad \quad \text{   und   } \quad \quad
        f_- = \max (-f,0)
    \end{align*}
    uneigentlich Riemann-integrierbar sind. Des weiteren gilt $f = f_+ - f_-$.
}

\vspace{1\baselineskip}

\Bemerkung{

    Das uneigentliche Integral in mehreren Variablen ist nicht kompatibel mit dem
    uneigentlichen Integral in einer Variablen.
}

\vspace{1\baselineskip}

\Satz{

    Seien $X,Y \subseteq \R^n$ offene Teilmengen und sei $\Phi: X \rightarrow \Y$ ein
    Diffeomorphismus. Sei $f: Y \rightarrow \R$ eine uneigentlich Riemann-integrierbare
    Funktion. Dann ist die Funktion $(f \circ \Phi) \abs{\det (D \Phi)}$ uneigentlich
    Riemann-integrierbar und es gilt
    \begin{align*}
        \int_Y f \ dx = \int_X (f \circ \Phi) \abs{\det(D \Phi)} \ dx
    \end{align*}
}

\vspace{1\baselineskip}

\Beispiel{ (Gauss'sche Glockenkurve)

    Sei $f: \R \rightarrow \R$ eine Funktion gegeben durch $f(x) = \exp (-x^2)$.
    Man nennt diese Funktion \fat{Dichtefunktion}.
    Dann ist die Stammfunktion gegeben als
    \begin{align*}
        F(x) = \frac{1}{\sqrt{\pi}} \int_{-\infty}^{x} \exp (-t^2) dt
    \end{align*}
    Diese Funktion heisst \fat{Verteilungsfunktion} der Normalverteilung.
    Wenn wir den Grenzwert des Integrals mit $x \rightarrow \infty$ berechnen
    wollen, erhalten wir
    \begin{align*}
        I = \intii \exp (-t^2) dt = \sqrt{\pi}
    \end{align*}
    Am einfachsten berechnet man dies, indem man $I^2$ berechnet.
}


\pagebreak

\section{Globale Integralsätze}

\vspace{1\baselineskip}

\subsection{Der Divergenzsatz und der Satz von Green in der Ebene}

\vspace{1\baselineskip}

\Definition{

    Sei $U \subseteq \R^n$ offen und sei $F: U \rightarrow \R^n$ ein differenzierbares
    Vektorfeld auf $U$. Die \fat{Divergenz} (Quellenstärke) von $F$ ist die Funktion
    div$(F) : U \rightarrow \R$ gegeben durch
    \begin{align*}
        \text{div} (F) = \text{tr} (Df) = \partial_1 F_1 + \partial_2 F_2 + \dots + \partial_n F_n
    \end{align*}
}

\vspace{1\baselineskip}

\Definition{

    Sei $\gamma = (\gamma_1 , \gamma_2) : [a,b] \rightarrow \R^2$ ein stückweise stetig
    differenzierbarer Pfad. Sei $t \in [a,b]$ so, dass $\gamma$ bei $t$ differenzierbar
    ist. Wir schreiben
    \begin{align*}
        n_{\gamma} (t) = \klammer{\gamma_2' (t) , - \gamma_1' (t)}
    \end{align*}
    und bezeichnen diesen Vektor als \fat{Aussennormale} oder \fat{Normalenvektor}
    an $\gamma$ im Punkt $\gamma (t)$. Dieser Vektor zeigt bezüglich der Laufrichtung
    nach rechts. Wichtig ist, dass die Laufrichtung immer im gegenuhrzeigersinn
    ist, damit die Aussennormale tatsächlich nach aussen zeigt.
    Für $v = (v_1 , v_2) \in \R^2$ ist gilt:
    \begin{align*}
        \scalprod{v}{n_{\gamma'(t)}} = v_1 \gamma_2' (t) - v_2 \gamma_1' (t)
        = \det \begin{pmatrix}
            v_1 & \gamma_1' (t) \\
            v_2 & \gamma_2' (t)
        \end{pmatrix}
        = v \times \gamma' (t)
        = v \wedge \gamma' (t)
    \end{align*}
}

\vspace{1\baselineskip}

\Definition{

    Sei $U \subseteq \R^n$ offen, sei $F: U \rightarrow \R^n$ ein stetiges Vektorfeld
    auf $U$ und $\gamma: [a,b] \rightarrow \R^2$ ein stückweise stetig differenzierbarer
    Pfad. Das \fat{Flussintegral} von $F$ entlang $\gamma$ ist das Integral
    \begin{align*}
        \int_{\gamma} F \ dn_{\gamma} = \intab \scalprod{F (\gamma (t))}{n_{\gamma} (t)} dt
    \end{align*}
}

\vspace{1\baselineskip}

\Definition{

    Sei $U \subseteq \R^2$ offen und $B \subseteq U$ beschränkt und zusammenhängend
    mit glattem Rand. Sei $F: U \rightarrow \R^2$ von Klasse $C^1$. Dann gilt
    \begin{align*}
        \int_B \text{div} F(x) \ dx = \int_{\partial B} F \ dn = \int_{\gamma} F \ dn_{\gamma}
    \end{align*}
    für eine Kurve $\gamma: [a,b] \rightarrow U$ die $\partial B$ parametrisiert.
}

\vspace{1\baselineskip}

\Bemerkung{

    Das Flussintegral bleibt unter Reparametrisierung eines Pfades unverändert.
}

\vspace{1\baselineskip}

\Proposition{

    Sei $U \subseteq \R^2$ offen und sei $F: U \rightarrow \R^2$ ein stetig differenzierbares
    Vektorfeld und $B = [a,b] \times [c,d] \subseteq U$ ein Rechteck. Dann gilt
    \begin{align*}
        \int_B \div (F) \ dx = \int_{\partial B} F \ dn
    \end{align*}
}

\pagebreak

\Korollar{

    Sei $U \subseteq \R^2$ offen und sei $F: U \rightarrow \R^2$ ein stetig
    differenzierbares Vektorfeld. Dann gilt für alle $x_0 \in U$
    \begin{align*}
        \div (F)(x_0) = \limes{h \rightarrow 0} \frac{1}{4 h^2} \int_{\partial \klammer{x_0 + [-h,h]^2}} F \ dn
    \end{align*}
}

\vspace{1\baselineskip}

\Proposition{

    Sei $U \subseteq \R^2$ offen und $F$ ein stetig differenzierbares Vektorfeld
    auf $U$. Seien $a<b$ und $c<d$ reelle Zahlen, so, dass das Rechteck
    $[a,b] \times [c,d]$ in $U$ enthalten ist, und sei $\varphi: [a,b] \rightarrow [c,d]$
    stetig und stückweise stetig differenzierbar. Dann gilt
    für den Bereich $B = \geschwungeneklammer{(x,y) \in U \ | \ x \in [a,b] , \ c \leq y \leq \varphi (x)}$
    \begin{align*}
        \int_B \div (F) \ dx = \int_{\partial B} F \ dn
    \end{align*}
}

\vspace{1\baselineskip}

\Definition{

    Eine abgeschlossene Teilmenge $B \subseteq \R^n$ heisst ein \fat{glatt berandeter
    Bereich}, falls es für jeden Punkt $p \in B$ eine offene Umgebung $U_p$ in $\R^n$
    von $p$ gibt und ein glatter Diffeomorphismus $\varphi_p : U_p \rightarrow V_p
    = \varphi_p (U_p)$ auf eine weitere offene Teilmenge $V_p \subseteq \R^n$ existiert,
    so dass folgendes gilt
    \begin{align*}
        \varphi_p (U_p \cap B) = \geschwungeneklammer{y \in V_p \ | \ y_n \leq 0}
    \end{align*}
}

\Korollar{

    Aus der obigen Definition folgt insbesondere, dass der Rand $\partial B$ eines
    glatt berandeten Bereichs $B$ eine $n-1$-dimensionale Teilmannigfaltigkeit von
    $\R^n$ ist.
}

\vspace{1\baselineskip}

\Korollar{

    Wenn ein Bereich $B$ lokal um jeden Punkt aussieht wie das Gebiet unter dem Graphen einer
    glatten reellwertigen Funktion in $(n-1)$ Variablen, geeignet rotiert, dann ist
    der Bereich $B$ glatt berandet.
}

\vspace{1\baselineskip}

\Lemma{

    Sei $F: \R^n \rightarrow \R$ eine glatte Funktion mit Null als regulären Wert.
    Dann ist die abgeschlossene Teilmenge
    \begin{align*}
        B = \geschwungeneklammer{x \in \R^n \ | \ F(x) \geq 0}
    \end{align*}
    glatt berandet und $\partial B = \geschwungeneklammer{u \in \R^n \ | \ F(u) = 0}$
}

\vspace{1\baselineskip}

\Satz{

    Sei $K \subseteq \R^n$ kompakt und sei $U_1 , \dots , U_N$ eine (endliche) offene
    Überdeckung von $K$. Dann existieren glatte Funktionen $\eta_0 , \dots , \eta_N :
    \R^n \rightarrow [0,1]$ mit:
    \begin{enumerate}[{1)}]
        \item $\supp (\eta_0) \subseteq \R^n \backslash K$
        \item $\supp (\eta_i) \subseteq U_i$ für $i = 1, \dots , N$
        \item $\eta_0 + \dots + \eta_N = 1$
    \end{enumerate}
}

\vspace{1\baselineskip}

\Lemma{

    Sei $f: \R^n \rightarrow \R$ eine stetige Funktion. Sei $\psi : \R^n \rightarrow \R$
    eine glatte Funktion mit kompaktem Träger. Dann ist die durch
    \begin{align*}
        (\psi * f)(x) = \int_{\R^n} \psi (x-y) f(y) dy
    \end{align*}
    definierte Funktion $\psi * f : \R^n \rightarrow \R$ glatt, mit Träger in
    $\supp (\psi) + \supp(f)$.
}

\pagebreak

\Lemma{

    Sei $f: \R^n \rightarrow \R$ stetig, $K \subseteq \R^n$ kompakt und $\epsilon>0$.
    Dann existieren $\delta \in (0,\epsilon)$ und eine glatte Funktion $\psi:
    \R^n \rightarrow [0,\infty)$ so dass
    \begin{align*}
        \abs{x-y} < \delta \Longrightarrow \abs{f(x) - f(y)} < \epsilon
        \quad , \quad 
        \supp (\psi) \subseteq B(0,\delta)
        \quad \text{ und } \quad
        \int_{\R^n} \psi \ dx \ = 1
    \end{align*}
    für alle $x \in K$ und $y \in \R^n$ gilt. Für jede solche Funktion $\psi$
    gilt $\abs{(\psi * f)(x) - f(x)} \leq \epsilon$ für alle $x \in K$.
}

\vspace{1\baselineskip}

\Definition{

    Sei $B \subseteq \R^2$ abgeschlossen. Eine \fat{Parametrisierung des Randes} von
    $B$ ist eine endliche Kollektion stetig differenzierbarer Wege
    $\gamma_k : [a_k,b_k] \rightarrow \partial B$ mit folgenden Eigenschaften:
    \begin{enumerate}
        \item (Überdeckend) Es gilt $\partial B = \bigcup_{k=1}^{K} \gamma_k \klammer{[a_k,b_k]}$.
        \item (Keine Überschneidungen ausser an den Enden) Falls $\gamma_j (s) = \gamma_k (t)$
                für $(j,s) \neq (k,t)$ gilt, so gilt $s \in \geschwungeneklammer{a_j , b_j}$
                und $t \in \geschwungeneklammer{a_k , b_k}$.
        \item (Aufeinanderfolgend) Für jedes $j \in \geschwungeneklammer{1,\dots,K}$ existiert
                genau ein $k \in \geschwungeneklammer{1,\dots,K}$ mit $\gamma_j (b_j)
                = \gamma_k (a_k)$.
        \item (Regularität) Es gilt $\gamma_k'(t) \neq 0$ für alle $k$ und $t \in (a_k,b_k)$.
    \end{enumerate}
    Die Parametrisierung $\gamma_1, \dots ,  \gamma_K$ heisst \fat{positiv orientiert},
    wenn für jedes $k$ und jedes $t \in (a_k,b_k)$ ein $\epsilon>0$ existiert, so dass
    $\gamma(t) + \lambda n_{\gamma} (t) \in B^{\circ}$ für alle $\lambda \in (0,\epsilon)$.
    Dabei ist $n_{\gamma} (t) = \klammer{- \gamma_2'(t),\gamma_1'(t)}$
    die Aussennormale an $\gamma$.
}

\vspace{1\baselineskip}

\Definition{

    Sei $U \subseteq \R^2$ offen und $B \subseteq \R^2$ eine kompakte Teilmenge,
    deren Rand eine positiv orientiert Parametrisierung $\gamma_1 : [a_1 , b_1]
    \rightarrow \partial B \ , \dots , \ \gamma_K : [a_k,b_k] \rightarrow \partial B$
    besitzt. Sei $F: U \rightarrow \R^2$ ein stetig differenzierbares Vektorfeld
    definiert auf einer offenen Umgebung $U$ von $B$. Das \fat{Wegintegral} oder
    \fat{Arbeitsintegral} von $F$ entlang $\partial B$ ist durch
    \begin{align*}
        \int_{\partial B} F \ dt \ = \sum_{k=1}^{K} \int_{a_k}^{b_k} \scalprod{F(\gamma_k (t))}{\gamma_k' (t)} dt
    \end{align*}
    definiert. Das \fat{Flussintegral} von $F$ durch den Rand $\partial B$ ist durch
    \begin{align*}
        \int_{\partial B} F \ dn \ = \sum_{k=1}^{K} \int_{a_k}^{b_k} \scalprod{F(\gamma_k (t))}{n_{\gamma} (t)} dt
    \end{align*}
    definiert. Diese Integrale sind unabhängig von der Wahl der positiv orientierten
    Parametrisierung.
}

\vspace{1\baselineskip}

\Satz{ (Divergenzsatz in der Ebene)

    Sei $U \subseteq \R^2$ offen und $B \subseteq U$ glatt berandet und kompakt.
    Sei $F: U \rightarrow \R^2$ ein Vektorfeld der Klasse $C^1$. Dann gilt
    \begin{align*}
        \int_B \div(F) dx = \int_{\partial B} F \ dn
    \end{align*}
}

\vspace{1\baselineskip}

\Bemerkung{

    Der Satz gilt auch für Bereiche $B \subseteq \R^2$ die durch das Innere von
    Rechtecken $R_i$ abgedeckt werden kann, so dass $B \cap R_i$ geeignet rotiert
    das Gebiet unter einem Graphen einer stückweise stetig differenzierbaren Funktion
    ist. Salopp heisst die, das $B$ "endlich viele Ecken" haben kann.
}

\vspace{1\baselineskip}

\Satz{ (Satz von Green)

    Sei $f: U \rightarrow \R^2$ ein stetig differenzierbares Vektorfeld auf einer
    offenen Menge $U \subseteq \R^2$. Dann gilt für jeden glatt berandeten, kompakten
    Bereich $B \subseteq U$
    \begin{align*}
        \int_B \rot (F) dx = \int_{\partial B} F \ dt
    \end{align*}
    Wobei $\rot (F) = \partial_2 F_1 (x) - \partial_1 F_2 (x)$
}

\pagebreak

\Definition{

    Wir sagen, dass ein stetig differenzierbares Vektorfeld $F:U \rightarrow \R^2$
    \fat{rotationsfrei} ist, falls $\rot (F) = 0$.
}

\vspace{1\baselineskip}

\Korollar{

    Dass $F$ rotationsfrei ist, bedeutet in anderen Worten gerade, dass $F$ die
    Integrabilitätsbedingung erfüllt.
}

\vspace{1\baselineskip}

\Satz{ (Jordanscher Kurvensatz)

    Sei $\gamma: [0,1] \rightarrow \R^2$ ein glatter, regulärer, einfacher,
    geschlossener Weg. Dann kann man das Komplement von $\Gamma = \gamma([0,1])$
    eindeutig als disjunkte Vereinigung
    \begin{align*}
        \R^2 \backslash \Gamma = \text{Inn}(\Gamma) \cup \text{Auss}(\Gamma)
        \quad , \quad
        \text{Inn} (\Gamma) \cap \text{Auss}(\Gamma) = \emptyset
    \end{align*}
    schreiben, wobei dasd Innere $\text{Inn}(\Gamma)$ eine offene, beschränkte,
    zusammenhängende Teilmenge und das Äussere $\text{Auss}(\Gamma)$ eine offene,
    unbeschränkte, zusammenhängende Teilmenge ist. Des Weiteren gilt
    $\partial \text{Inn} (\gamma) = \partial \text{Auss} (\gamma) = \Gamma$.
}


\vspace{2\baselineskip}

\subsection{Oberflächenintegrale}

\vspace{1\baselineskip}

\Definition{

    Eine \fat{Fläche} $S \subseteq \R^3$ ist eine zweidimensionale Teilmannigfaltigkeit.
    Wir werden an zwei Arten von Flächen interessiert sein:
    \begin{enumerate}[{(1)}]
        \item Die Fläche $S$ ist der Rand eines kompakten, glatt berandeten Bereiches
                $B \subset \R^3$. Beispielsweise können $S$ also eine Sphäre oder ein
                Torus sein. Wir sprechen informell von einer \fat{geschlossenen
                Fläche}
        \item Die Fläche $S$ ist Teil einer Grösseren Fläche $M$, so, dass der Abschluss
                von $S$ in $M$ kompakt ist, und der Rand von $S$ in $M$ eine glatte Kurve,
                das heisst, eine Teilmannigfaltigkeit der Dimension $1$. Ein Beispiel für
                so eine Fläche ist die obere Hemisphäre von $\eS^2$. Wir sprechen informell
                von einer \fat{Fläche mit Rand}.
    \end{enumerate}
    Um eine Fläche $S \subseteq \R^3$ lokal zu beschreiben, werden wir Karten und
    Parametrisierungen verwenden. Wir identifizieren im Folgenden $\R^2$ mit dem
    Unterraum $\R^2 \times \geschwungeneklammer{0}$ von $\R^2$.
}

\vspace{1\baselineskip}

\Definition{

    Sei $S \subseteq \R^3$ eine Fläche. Also existiert für jeden Punkt $p \in S$ eine
    offene Umgebung $U \subseteq \R^3$ von $p$ und ein Diffeomorphismus
    $\varphi: U \rightarrow V$ auf eine weitere Teilmenge $V \subseteq \R^3$, so dass
    \begin{align*}
        \varphi (U \cap S) = \geschwungeneklammer{x \in V \ | \ x_3 = 0} = V \cap \R^2
    \end{align*}
    gilt. Den Diffeomorphismus $\varphi_p$ bezeichnet man als \fat{Karte} für die
    Teilmannigfaltigkeit $S$ von $\R^3$ um $p$ und den zu $\varphi$ inversen
    Diffeomorphismus $\psi:V \rightarrow U$ als \fat{Parametrisierungen} von $S$ um $p$.
    Die Diffeomorphismen $\varphi$ und $\psi$ schränken sich zu
    \begin{align*}
        \varphi:U \cap S \rightarrow V \cap \R^2
        \quad \quad \text{   und   } \quad \quad
        \psi: V \cap \R^2 \rightarrow U \cap S
    \end{align*}
    ein. Sind $\varphi_1 : U_1 \rightarrow V_1$ und $\varphi_2: U_2 \rightarrow V_2$
    Karten für $S$, so bezeichnet man den Diffeomorphismus
    \begin{align*}
        \alpha_{12} : \varphi_1^{-1} (U_1 \cap U_2) \stackrel{\varphi_1}{\longrightarrow}
        U_1 \cap U_2 \stackrel{\varphi_2^{-1}}{\longrightarrow} \varphi_2^{-1} (U_1 \cap U_2)
    \end{align*}
    als \fat{Kartenwechsel}, \fat{Transitionsabbildungen} oder auch \fat{Übergangsmorphismus}.
    Dieser Diffeomorphismus schränkt sich zu
    \begin{align*}
        \alpha_{12}:\varphi_1^{-1} (U_1 \cap U_2) \cap \R^2 \rightarrow 
    \varphi_2^{-1} (U_1 \cap U_2) \cap \R^2
    \end{align*}    
    ein. Wir werden die Terminologie nun leicht adaptieren, und Kartenbereiche nicht
    als in $\R^3$ offene Mengen definieren, sondern als relativ offene Mengen von $S$,
    mit Wertebereichen offene Teilmengen von $\R^2$.
}

\pagebreak

\Definition{

    Sei $S \subseteq \R^3$ eine Fläche. Wir nennen \fat{Atlas} für $S$ eine
    Überdeckung $\geschwungeneklammer{U_i \ | \ i \in I}$ von $S$ durch relativ offene
    Teilmengen $U_i \subseteq S$, genannt \fat{Kartenbereiche}, zusammen mit
    Homeomorphismen $\varphi_i : U_i \rightarrow V_i$ auf offene Teilmengen
    $V_i \subseteq \R^2$ genannt \fat{Kartenabbildungen}, so dass die
    \fat{Parametrisierung} $\psi_i = \varphi_i^{-1}$
    \begin{align*}
        \psi_i : V_i \rightarrow U_i \subseteq \R^3
    \end{align*}
    glatt sind, und so, dass die \fat{Kartenwechsel} oder \fat{Transitionsabbildungen}
    $\alpha_{ij}: V_i \rightarrow V_j$ gegeben durch
    \begin{align*}
        \alpha_{ij} : \varphi_i^{-1} (U_i \cap U_j) \stackrel{\varphi_i}{\longrightarrow}
        U_i \cap U_j \stackrel{\varphi_j^{-1}}{\longrightarrow} \varphi_j^{-1} (U_i \cap U_j)
    \end{align*}
    glatt sind.
}

\vspace{1\baselineskip}

\Definition{

    Eine \fat{zweidimensionale, reelle Mannigfaltigkeit} ist ein topologischer Raum
    $X$, zusammen mit einer offenen Überdeckung $(U_i)_{i \in I}$ von $X$ und 
    Homöomorphismen $\varphi_i : U_i \rightarrow V_i$ mit $V_i \subseteq \R^2$
    offen, für alle $i \in I$ so, dass alle Transitionsabbildungen
    \begin{align*}
        \alpha_{ij} : \varphi_i (U_i \cap U_j) \rightarrow \varphi_j (U_i \cap U_j)
    \end{align*}
    glatt sind.
}

\vspace{1\baselineskip}

\Definition{

    Sei $X$ eine $2$-dimensionale reelle Mannigfaltigkeit. Eine Funktion
    $f:X\rightarrow \R^n$ heisst \fat{glatt} $(C^{\infty})$, falls für jede Karte
    $\varphi: U \rightarrow V \subseteq \R^2$ von $X$ die Verknüpfung
    $V \stackrel{\varphi^{-1}}{\longrightarrow} U \stackrel{f}{\longrightarrow} \R^n$
    glatt ist.
}

\vspace{1\baselineskip}

\Bemerkung{

    Eine Mannigfaltigkeit $X$ mit Atlas
    $(U_i \stackrel{\varphi}{\longrightarrow} V_i)_{i \in I}$
    können wir uns vorstellen als
    \Large
    \begin{align*}
        \nicefrac{\klammer{\bigcupdot V_i}}{\alpha_{ij}(X)}
    \end{align*}
    \normalsize
    wobei $\bigcupdot V_i$ die disjunkte Vereinigung von allen $V_i$ ist.
    Ist $X$ aus einer Teilmannigfaltigkeit von $\R^3$ gegben, so erhalten wir
    \begin{align*}
        \varphi_i^{-1} : V_i \rightarrow U_i \subseteq \R^3 
    \end{align*}
    welches kompatibel mit der obigen Verklebung ist.
}

\vspace{1\baselineskip}

\Definition{

    Sei $X$ eine $2$-dimensionale reelle Teilmannigfaltigkeit mit Atlas
    $(\varphi_i : U_i \rightarrow V_i \subseteq \R^2)_{i \in I}$. Eine \fat{Immersion}
    von $X$ nach $\R^3$ ist eine injektive, abgeschlossene Abbildung
    $h: X \rightarrow \R^3$ sodass für jedes $i \in I$ und $x \in V_i$
    \begin{align*}
        D(h \circ \varphi_i^{-1})(x)
    \end{align*}
    injektiv ist.
}

\vspace{1\baselineskip}

\Bemerkung{

    Ist $h: X \rightarrow \R^3$ eine Immersion, dann ist $h(x) \subseteq \R^3$ eine
    $2$-dimensionale Teilmannigfaltigkeit.
}

\vspace{1\baselineskip}

\Definition{

    Sei $S \subseteq \R^3$ eine Fläche, bzw eine $2$-dimensionale Mannigfaltigkeit.
    Wir sagen ein Atlas $(\varphi_i : U_i \rightarrow V_i)_{i \in I}$ sei
    \fat{orientiert}, falls für alle Kartenwechsel $\alpha_{ij}$ die Jacobi-Determinante
    \begin{align*}
        \det (D \alpha_{ij}) > 0
    \end{align*}
    positiv ist. Wir sagen $S$ sei \fat{orientierbar}, falls $S$ einen orientierten
    Atlas besitzt.
}

\pagebreak

\Definition{

    Sei $S \subseteq \R^3$ eine Fläche, bzw eine $2$-dimensionale Teilmannigfaltigkeit.
    Ein stetiges \fat{normiertes Normalenfeld} ist ein stetiger Schnitt des Normalenbündels
    von $S$, also eine stetige Abbildung $n: S \rightarrow \R^3$ mit $n(p) \in (T_p S)^{\perp}$,
    so, dass $\Norm{n(p)} = 1$ für alle $p \in S$ gilt.
}

\vspace{1\baselineskip}

\Proposition{

    Sei $S \subseteq \R^3$ eine Fläche, bzw eine $2$-dimensionale Teilmannigfaltigkeit.
    $S$ ist genau dann orientierbar, wenn ein stetiges normiertes Normalenfeld existiert.
}

\vspace{1\baselineskip}

\Lemma{

    Sei $B \subseteq \R^3$ ein kompakter, glatt berandeter Bereich. Dann ist der Rand
    $S = \partial B$ orientierbar, und es gibt ein eindeutiges stetiges normiertes
    Normalenfeld $n: S \rightarrow \R^3$ so, dass $p+ \epsilon n(p) \notin B$
    für alle $p \in S$ und alle genügent kleinen $\epsilon>0$ gilt.
}

\vspace{1\baselineskip}

\Definition{

    Sei $S \subseteq \R^3$ eine Fläche, und $f: S \rightarrow \R$ eine Funktion.
    Es sei $(\varpi_i : U_i \rightarrow V_i)_{i \in I}$ ein endlicher Atlas von $S$
    mit Parametrisierung $\psi_i = \varphi_i^{-1} : V_i \rightarrow U_i \subseteq S$,
    und seien $(\eta_i : S \rightarrow [0,1])_{i \in I}$ stetige Funktionen mit
    \begin{align*}
        \supp (\eta_i) \subseteq U_i
        \quad \quad \text{     und     } \quad \quad
        \sum_{i \in I} \eta_i = 1
    \end{align*}
    also eine stetige Zerlegung der Eins auf $S$. Schreibe $f_i = \eta_i f$.
    Wir definieren das \fat{skalare Oberflächenintegral} von $f$ auf $S$ als
    \begin{align*}
        \int_S f \ dA =
        \sum_{i \in I} \int_{V_i} (f_i \circ \psi_i) \Norm{\partial_1 \psi_i \wedge \partial_2 \psi_i} dx
    \end{align*}
    falls die zweidimensionale Riemann-Integrale rechterhand existieren. Der
    \fat{Flächeninhalt} von $F$ ist durch
    \begin{align*}
        \vol (S) = \int_S dA = \sum_{i \in I} \int_{V_i} \Norm{\partial_1 \psi_i \wedge \partial_2 \psi_i} dx
    \end{align*}
    definiert.
}

\vspace{1\baselineskip}

\Bemerkung{

    Sei $X \subseteq \R^3$ eine Fläche. Falls $X$ durch Kartenbereiche $U_1 ,\dots,U_N$
    abgedeckt ist, bis auf eine Vereinigung von Punkten und Kurven, mit
    $U_1,\dots,U_N$ disjunkt, dann gilt
    \begin{align*}
        \int_X f \ dA = \sum_{i=1}^{N} \int_{U_i} f |_{U_i} dA
    \end{align*}
}

\vspace{1\baselineskip}

\Definition{

    Sei $S \subseteq \R^3$ eine orientierbare Fläche und sei
    $(\varphi_i : U_i \rightarrow V_i)_{i \in I}$ ein endlicher orientierter Atlas
    von $S$ mit Parametrisierungen $\psi_i = \varphi_i^{-1}$, und sei $(\eta_i)_{i \in I}$
    eine stetige Zerlegung der Eins auf $S$ mit $\supp (\eta_i) \subseteq U_i$.
    Sei $U \subseteq \R^3$ eine offene Menge die $S$ enthält, und sei $F$ ein stetiges
    Vektorfeld auf $U$. Dann definieren wir das \fat{Flussintegral} von $F$ über $S$
    durch
    \begin{align*}
        \int_S F \ dn = \sum_{i \in I} \int_{V_i} \scalprod{\eta_i F \circ \psi_i}{\partial_1 \psi_i \wedge \partial_2 \psi_i} dx
    \end{align*}
}


\pagebreak

\subsection{Der Divergenzsatz im dreidimensionalen Raum}

\vspace{1\baselineskip}

\Proposition{

    Sei $U \subseteq \R^3$ offen und $F:U \rightarrow \R^3$ ein stetig differenzierbares
    Vektorfeld. Sei $Q \subseteq \R^2$ ein abgeschlossenes Rechteck, $z_0 \in \R$ und
    $\varphi: Q \rightarrow [z_0,\infty)$ eine stetig differenzierbare Funktion, so
    dass der abgeschlossene Bereich under dem Graphen von $\varphi$
    \begin{align*}
        B = \geschwungeneklammer{(x,y,z) \in \R^3 \ | \ (x,y) \in \Q \ , \ z_0 \leq z \leq \varphi(x,y)}
    \end{align*}
    in $U$ liegt. Dann gilt
    \begin{align*}
        \int_B \div (F) \ dx \ = \ \int_{\partial B} F \ dn
    \end{align*}
}

\vspace{1\baselineskip}

\Satz{ (Divergenzsatz/ Satz von Gauss)

    Sei $B \subseteq \R^3$ ein kompakter, glatt berandeter Bereich und $F$ ein stetig
    differenzierbares Vektorfeld, definiert auf einer offenen Umgebung von $B$. Dann
    gilt
    \begin{align*}
        \int_B \div (F) \ dx \ = \ \int_{\partial B} F \ dn
    \end{align*}
}


\vspace{2\baselineskip}

\subsection{Der Satz von Stokes im dreidimensionalen Raum}

\vspace{1\baselineskip}

\Bemerkung{

    Flächen mit Rand $S \subseteq \R^3$ sind abgeschlossene Teilmengen einer
    $2$-dimensionalen Teilmannigfaltigkeit $M \subseteq \R^3$, so, dass der Rand von
    $S$ relativ in $M$ eine eindimensionale Teilmannigfaltigkeit ist.
}

\vspace{1\baselineskip}

\Bemerkung{

    Eine wichtige Familie glatt berandeter Flächen sind Graphen glatter Funktionen.
    Ist $U \subseteq \R^2$ offen und $f:U \rightarrow \R$ eine glatte Funktion, so
    ist für jeden glatt berandeten Bereich $B \subseteq U$ der Graph
    \begin{align*}
        S = \geschwungeneklammer{(x,y,z) \in \R^3 \ | \ (x,y) \in B \ \text{ und } z = f(x,y)}
    \end{align*}
    eine glatt berandete Fläche. Die Projektion von $S$ auf $B$ ist eine Karte für
    ganz $S$. Lokal kann jede glatt berandete Fläche als Graph beschrieben werden.
}

\vspace{1\baselineskip}

\Definition{

    Sei $F:U \rightarrow \R^3$ ein stetig differenzierbares Vektorfeld auf einer
    offenen Menge $U \subseteq \R^3$. Die \fat{Wirbelstärke} oder \fat{Rotation}
    von $F$ ist das Vektorfeld auf $U$ definiert durch
    \begin{align*}
        \rot (F(x)) = \begin{pmatrix}
            \partial_2 F_3 - \partial_3 F_2 \\
            \partial_3 F_1 - \partial_1 F_3 \\
            \partial_1 F_2 - \partial_2 F_1
        \end{pmatrix}
    \end{align*}
}

\vspace{1\baselineskip}

\Satz{ (Satz von Stokes)

    Sei $F$ ein stetig differenzierbares Vektorfeld auf einer offenen Teilmengen
    $U \subseteq \R^3$. Sei $S \subseteq U$ eine glatt berandete, kompakte und
    orientierbare Fläche. Dann gilt
    \begin{align*}
        \int_S \rot (F) \ dn \ = \ \int_{\partial S} F \ dt
    \end{align*}
}


\pagebreak

\section{Gewöhnliche Differentialgleichungen}

\vspace{1\baselineskip}

\subsection{Differentialgleichungssysteme}

\vspace{1\baselineskip}

Wir fixieren ein offenes, nichtleeres Intervall $I \subseteq \R$ und $x_0 \in I$.
Wir schreiben $C^{\infty} (I)$ für den $\C$-Vektorraum aller glatten Funktionen
$I \rightarrow \C$.

\vspace{1\baselineskip}

\Definition{

    Wir schreiben $D: C^{\infty} (I) \rightarrow C^{\infty} (I)$ für die Ableitung
    $Df = f'$, aufgefasst als $\C$-linearer Endomorphismus des Vektorraums $C^{\infty} (I)$.
    Ein linearer \fat{Differentialoperator} der Ordnung $d \geq 0$ ist eine lineare
    Abbildung
    \begin{align*}
        L: C^{\infty} (I) \rightarrow C^{\infty} (I)
        \quad , \quad \quad
        \sum_{i=0}^{d} L = a_i D^{i}
    \end{align*}
    die sich als Linearkombination von $D^0 =$ id, $D^1 = D$, $D^2 = D \circ D$, \dots \
    mit Koeffizienten $a_i \in C^{\infty} (I)$ schreibt. Als lineare, gewöhnliche
    \fat{Differentialgleichung} der Ordnung $d$ bezeichnen wir die Gleichung $Lu=g$
    für einen vorgegebenen Differentialoperator $L$ und eine vorgegebene
    \fat{Störfunktion} $g \in C^{\infty} (I)$. Als \fat{Anfangswertproblem} bezeichnet
    man das lineare Gleichungssystem bestehend aus dieser Differentialgleichung und
    $d$ Gleichungen
    \begin{align*}
        &Lu = g \\
        &u(x_0) = w_0 \quad , \quad Du(x_0) = w_1 \quad , \dots , \quad D^{d-1} u(x_0) = w_{d-1}
    \end{align*}
    für vorgegebene \fat{Anfangswerte} $w_0 , w_1 , \dots , w_{d-1} \in \R$.
}

\vspace{1\baselineskip}

\Korollar{

    Um die Lösung der homogenen Differentialgleichung $Lu = 0$ zu finden, setzen wir
    $u(x) = \exp (\alpha x)$ für ein $\alpha \in \C$. Damit gilt $D^{i} u = \alpha^{i} u$
    und
    \begin{align*}
        Lu \ = \ \sum_{i=0}^{d} a_i D^{i} u \ = \ \sum_{i=1}^{d} a_i \alpha^{i} u
    \end{align*}
    Daher ist $u(x) = \exp (\alpha x)$ genau dann eine Lösung der Differentialgleichung
    $Lu=0$, wenn $\alpha \in \C$ eine Nullstelle des sogenannten \fat{charakteristischen
    Polynoms} des Operators $L$
    \begin{align*}
        p(T) = T^d + a_{d-1} T^{d-1} + \dots + a_0 \in \C [T]
    \end{align*}
    ist. Ist $\alpha$ eine reelle Nullstelle von $p(T)$, dann ist durch $u(x) = \exp (\alpha x)$
    eine reellwertige Lösung gegeben. Bei reellen Koeffizienten und einer komplexen
    Nullstelle $\alpha = \beta + \gamma i \ \in \C$ mit $\beta , \gamma \in \R$ ist
    $\overline{\alpha} = \beta - \gamma i$ ebenso eine Nullstelle von $p(T)$. Dann
    sind alle Linearkombinationen $s \exp (\alpha) + t \exp(\overline{\alpha} x)$
    und insbesondere
    \begin{align*}
        \frac{\exp (\alpha x) + \exp (\overline{\alpha} x)}{2} = \exp (\beta x) \cos (\gamma x)
        \quad , \quad
        \frac{\exp (\alpha x) + \exp (\overline{\alpha} x)}{2i} = \exp (\beta x) \sin (\gamma x)
    \end{align*}
    Lösungen von $Lu=0$. Man beachte, dass $\exp (\alpha x)$ und $\exp (\overline{\alpha} x)$
    gleichen $\C$-Unterraum von Lösungen zu $Lu=0$ aufspannen wie $\exp (\beta x) \cos(\gamma x)$
    und $\exp(\beta x) \sin (\gamma x)$.
}

\vspace{1\baselineskip}

\Korollar{

    Wir betrachten nun den Fall, in dem das charakteristische Polynom des
    Differentialoperators $L$ eine mehrfache Nullstelle hat. Dazu untersuchen wir
    Linearkombinationen von Funktionen des Types $f(x) = x^n \exp (\alpha x)$.

    Hierbei ist $n$ die Vielfachheit einer Nullstelle $\alpha$. Als Lösungsansatz
    summieren wir über all die Nullstellen und alle Vielfachheiten. Angenommen
    wir haben zwei Nullstellen $\lambda_1$ und $\lambda_2$ mit Vielfachheiten
    $\mu_1$ bzw $\mu_2$. Dann ist der Ansatz von $u$ gegeben als
    \begin{align*}
        u = \sum_{i=0}^{\mu_1 -1} c_i x^{i} e^{\lambda_1 x} \ + \ \sum_{j=0}^{\mu_2 -1} c_j x^{j} e^{\lambda_2 x}
    \end{align*}
    wobei $c_i$ und $c_j$ Koeffizienten sind, welche mit den Anfangsbedingungen
    herausgefungen werden müssen. Wichtig ist, dass die Koeffizienten in den beiden
    Summen verschieden sind. Die Notation ist etwas ungünstig, da mehrere Koeffizienten
    den gleichen Intex haben könnten.
}

\pagebreak

\Definition{

    Wir bezeichnen den $\C$-Vektorraum der von diesen Funktionen aufgespannt wird,
    also
    \begin{align*}
        \text{PE}_{\C} (I) = \langle x^n \exp (\alpha x) \ | \ n \in \N \ , \ \alpha \in \C \rangle
    \end{align*}
    als Raum der \fat{Exponentialpolynome}. Zusammen mit der üblichen Multiplikation
    bilden Exponentialpolynome eine $\C$-Algebra. Die Ableitung definiert eine
    $\C$-lineare Abbildung
    \begin{align*}
        D: \text{PE}_{\C} (I) \rightarrow \text{PE}_{\C} (I)
        \quad , \quad \quad
        D(x^n e^{\alpha x}) = n x^{n-1} e^{\alpha x} + \alpha x^n e^{\alpha x}
    \end{align*}
    die die Leibnitz-Regel $D(fg) = D(f) g + f D(g)$ erfüllt.
}

\vspace{1\baselineskip}

\Proposition{

    Die Funktion $\geschwungeneklammer{x \mapsto x^n \exp (\alpha x) \ | \ n \in N , \alpha \in \C}$
    bilden eine Basis des $\C$-Vektorraums $\text{PE}_{\C} (I)$.
}

\vspace{1\baselineskip}

\Proposition{

    Sei $p \in \C [T]$ ein Polynom von Grad $d \geq 1$, das mittels
    \begin{align*}
        p(T) = a \prod_{j=1}^{k} (T-\alpha_j)^{d_j}
    \end{align*}
    in $d$ Linearfaktoren zerfällt, wobei wie annehmen, dass $\alpha_i \neq \alpha_j$
    für $i \neq j$ gilt. Sei $L=p(D)$ der Differentialoperator mit charakteristischem
    Polynom $p$. Dann hat die zugehörige homogene Differentialgleichung $Lu =0$
    die $d$ linear unabhängigen Lösungen $u(x) = x^n \exp(\alpha_j x)$ für
    $0 \leq n \leq n_j$ und $1 \leq j \leq k$.
}

\vspace{1\baselineskip}

\Bemerkung{

    Wir wenden uns nun den inhomogenen Problem zu. Für ein Polynom $p(T) \in \C [T]$
    und eine Störfunktion $g \in \text{PE}_{\C} (I)$ gibt es ein einfaches Verfahren,
    mit dem man eine Lösung $u_0$ der inhomogenen Differentialgleichung $p(D)u = g$
    finden kann.
    \begin{enumerate}
        \item Falls $g(t) = q(t) e^{\alpha t}$ für ein Polynom $q$ vom Grad $n$ und
                $\alpha \in \C$ mit $p(\alpha) \neq 0$, dann definiert man $u_0 (t)
                =Q(t)e^{\alpha t}$, wobei $Q(T)$ ein Polynom vom Grad $n$ mit noch zu
                bestimmenden Koeffizienten ist. Nun berechnet man $p(D)u_0$ und
                bestimmt Koeffizienten so dass $p(D)u_0 = g$ gilt.
        \item Falls $g(x) = q(x) e^{\alpha x}$ für ein Polynom $q(T)$ vom Grad $n$ und
                $\alpha \in \C$ mit $p(\alpha) = 0$, dann wiederholt man obiges Verfahren,
                allerdings mit dem Ansatz $u_0 (t) = Q(t) t^m e^{\alpha t}$, wobei $m$
                die Vielfachheit der Nullstelle $\alpha$ von $p(T)$ angibt.
        \item Ein allgemeines $g \in \text{PE}_{\C} (I)$ lässt sich als Linearkombination
                von Ausdrücken wie oben darstellen. Auf Grund der Linearität von $p(D)$
                kann man also obiges Verfahren für Summanden der Form $q(x) e^{\alpha x}$
                in $g$ anwenden und dann die resultierenden Lösungsfunktionen addieren.
    \end{enumerate}
}

\vspace{1\baselineskip}

\Definition{

    Sei $I \subseteq \R$ ein Intervall, $U \subseteq I \times \R^d$ offen und
    $F: U \rightarrow \R^d$ eine stetige Funktion. Eine Differentialgleichung
    der Form
    \begin{align*}
        u'(t) = F(t,u(t))
    \end{align*}
    für eine unbekannte Funktion $u$ bezeichnen wir als $d$-\fat{dimensionale
    Differentialgleichungssysteme erster Ordnung}. Dabei ist der Definitionsbereich
    einer Lösung $u$ möglicherweise nur ein Teilintervall von $I$. Damit die obere
    Gleichung überhaupt Sinn ergibt, muss $(t,u(t)) \in U$ für alle $t$ im
    Definitionsbereich von $u$ gelten. Zu $(t_0 , x_0) \in U$ nennen wir die
    Gleichung
    \begin{align*}
        u'(t) = F(t,u(t))
        \quad , \quad \quad
        u(t_0) = x_0
    \end{align*}
    ein \fat{Anfangswertproblem} zum \fat{Anfangswert} $x_0$ bei $t_0 \in I$.
    Die Differentialgleichung heisst \fat{autonom}, falls $F$ konstant bezüglich
    der Variablen $t$ ist.
}

\vspace{1\baselineskip}

\Proposition{

    Sei $A \in \Mat_d (\R)$, $x_0 \in \R^d$ und $t_0 \in \R$. Das Anfangswertproblem
    für differenzierbare Funktionen $u: \R \rightarrow \C^d$
    \begin{align*}
        u'(t) = Au
        \quad , \quad \quad
        u(t_0) = x_0
    \end{align*}
    hat die eindeutig bestimmte Lösung $u(t) = \exp(A(t-t_0))x_0$.
}

\vspace{1\baselineskip}

\Definition{ (\fat{Trennung der Variablen / Variablenseparation})

    Gegeben sei eine Differentialgleichung erster Ordnung in der Form
    \begin{align*}
        u'(t) = f(t) g(u(t))
    \end{align*}
    für Funktionen $f$ und $g$. Solch eine Differentialgleichung bezeichnet man als
    \fat{separierbar}, da auf der rechten Seite ein Produkt von zwei Funktionen steht,
    von denen eine nur von der Variablen $t$, und die andere nur von der "Variablen"
    $u=u(t)$ abhängt. Wir teilen durch $g(u(t))$ und integrieren:
    \begin{align*}
        \int \frac{u'(t)}{g(u(t))} \ dt \ = \ \int f(t) \ dt \ + C
    \end{align*}
    Im Integral linkerhand können wir die Substitution $s=u(t)$ vornehmen. Ist also
    $G$ eine Stammfunktion von $\frac{1}{g}$ und $F$ eine Stammfunktion von $f$, so
    erhalten wir $G(u(t)) = F(t) + C$ für eine Integrationskonstante $C$. Falls
    $G$ invertierbar ist, ergibt sich damit die Lösung
    \begin{align*}
        u(t) = G^{-1}(F(t) + C)
    \end{align*}
    der Differentialgleichung.
}


\vspace{2\baselineskip}

\subsection{Der Satz von Picard-Lindelöf}

\vspace{1\baselineskip}

\Satz{

    Sei $d \geq 1$, $U \subseteq \R \times \R^d$ offen, $(t_0,x_0) \in U$ und
    $F:U \rightarrow \R^d$ stetig. Angenommen $F$ ist "lokal Lipschitz-stetig im Ort",
    das heisst, für alle $(t_1,x_1) \in U$ existieren $\epsilon>0$ und $L \geq 0$,
    so dass für alle $(t,x_2),(t,x_3) \in B((t_1,x_1),\epsilon) \cap U$ die
    Abschätzung
    \begin{align*}
        \Norm{F(t,x_2) - F(t,x_3)} \leq L \Norm{x_2 - x_3}
    \end{align*}
    gilt. Dann existert ein (nicht unbedingt beschränktes) Intervall $I = I_{max}
    = (a,b) \subseteq \R$ mit $t_0 \in I$ und eine differenzierbare Funktion
    $u: I \rightarrow \R^d$ mit folgenden Eigenschaften:
    \begin{enumerate}
        \item Es gilt $(t,u(t)) \in U$ für alle $t \in I$, und $u$ ist eine Lösung
                des Anfangswertproblems
                \begin{align*}
                    \begin{cases}
                        u'(t) = F(t,u(t)) \quad \text{   für alle } t \in I \\
                        u(t_0) = x_0
                    \end{cases}
                \end{align*}
        \item Für jede weitere Lösung $v:J \rightarrow \R^d$ desselben Anfangswertproblems
                definiert auf einm offen Intervall $J$ mit $t_0 \in J$ gilt
                $J \subseteq I$ und $u |_J = v$.
        \item Die Grenzwerte $\limes{t \rightarrow a} (t,u(t))$ und
                $\limes{t \rightarrow b} (t,u(t))$ existieren nicht in $U$.
    \end{enumerate}
}

\Bemerkung{

    Die Hypothesen sind erfüllt, wenn $F$ von Klasse $C^1$ ist.
}

\vspace{1\baselineskip}

\Bemerkung{

    Informell besagt der Satz von Picard-Lindelöf, dass jedes Anfangswertproblem
    für ein im Ort lokal Lipschitz-stetiges Vektorfeld eine eindeutige Lösung hat.
}

\pagebreak

\Korollar{

    Sie $d \geq 1$ und $a_0 , \dots , a_{d-1} \in C^{\circ} (D,\R)$ für $D \in \R$ offen,
    $t_0 \in D$, $x_0 \in \R^d$. Dann existert ein maximales offenes Intervall
    $I \subseteq \R$ mit $t_0 \in I$ zusammen mit $u:I \rightarrow \R^d$ eindeutig
    bestimmt durch
    \begin{align*}
        &u^{(d)} + a_{d-1} u^{(d-1)} + \dots + a_1 u' + a_0 u = 0 \\
        &\begin{pmatrix}
            u(t_0) \\ u'(t_0) \\ \vdots \\ d^{(d-1)} (t_0)
        \end{pmatrix}
        = x_0
    \end{align*}
    Insbesondere gilt $\dim \ker(L) = d$.
}

\vspace{1\baselineskip}

\Proposition{

    Sei $r>0$, $t_0 \in \R$, $x_0 \in \R^d$ und sei
    \begin{align*}
        F: (t_0 - r , t_0 + r) \times B(x_0 , r) \rightarrow \R^d
    \end{align*}
    eine stetige Funktion. Angenommen es existieren Konstanten $C \geq 1$ und $L>0$
    so dass
    \begin{align*}
        \Norm{F(t,x)} \leq C
        \quad \quad \text{     und     } \quad \quad
        \Norm{F(t,x_1) - F(t,x_2)} \leq L \Norm{x_1 - x_2}
    \end{align*}
    für alle $t \in (t_0 - r , t_0 + r)$ und $x,x_1,x_2 \in B(x_0 ,r)$ gilt.
    Dann existiert für jedes $\delta>0$ mit $\delta< \min
    \geschwungeneklammer{\frac{r}{2C} , \frac{r}{2L}}$ eine eindeutige differenzierbare
    Funktion $u: [t_0 - \delta , t_0 + \delta] \rightarrow B(x_0 ,r)$ die folgendes
    erfüllt:
    \begin{align*}
        \begin{cases}
            &u(t_0) = x_0 \\
            &u'(t) = F(t,u(t)) \quad \text{ für alle } t \in (t_0 - \delta , t_0 + \delta)
        \end{cases}
    \end{align*}
}

\vspace{1\baselineskip}

\Lemma{

    Sei $[a,b] \subseteq \R$ ein Intervall, $f,g: [a,b] \rightarrow \R$ stetige
    Funktionen mit $f$ differenzierbar auf $(a,b)$ und $f'(t) = g(t) \ \forall t \in (a,b)$.
    Dann ist $f$ bei $b$ linksseitig ableitbar und $f'(b) = g(b)$.
}

\vspace{1\baselineskip}

\Definition{ (Die Quasilineare Partielle Differentialgleichung in zwei Variablen)

    Sei $u(x,y)$ eine Funktion in zwei Variablen mit
    \begin{align*}
        F \klammer{x,y,u,\frac{\partial u}{\partial x} , \frac{\partial u}{\partial y}}
    \end{align*}
    eine Funktion in $5$ Variablen und linear in den "letzten beiden. Seien
    $a,b,c$ Funktionen in $3$ Variablen sodass die folgende Gleichung erfüllt ist
    \begin{align*}
        a(x,y,u) \cdot \frac{\partial u}{\partial x} + b(x,y,u) \cdot \frac{\partial u}{\partial y} = c(x,y,u)
        \quad \Longleftrightarrow \quad
        a(x,y,u) \cdot \frac{\partial u}{\partial x} + b(x,y,u) \cdot \frac{\partial u}{\partial y} - c(x,y,u) =0 
    \end{align*}
    Wir definieren eine Funktion $F$ in $3$ Variablen und einen Normalenvektor $n$
    an dem Graphen von $u$ im Punkt $(x,y,z)$.
    \begin{align*}
        &F(x,y,z) = \begin{pmatrix}
            a(x,y,z) \\ b(x,y,z) \\ c(x,y,z)
        \end{pmatrix}
        \\
        &n(x,y,z) = \begin{pmatrix}
            \frac{\partial u}{\partial x} \\ \frac{\partial u}{\partial y} \\ -1
        \end{pmatrix}
    \end{align*}
    Es gilt $n(x,y,z) = \grad (u(x,y) - z)$. Damit vereinfacht sich die Differentialgleichung
    zu
    \begin{align*}
        \scalprod{F(x,y,u)}{n(x,y,u)} = 0
    \end{align*}
    Der Graph von $u$ "fliesst" entlang dem Vektorfeld $F$, i.e.: Für alle $(x,y,z)$
    mit $z = u(x,y)$ im Graphen von $u$ ist $F(x,y,z)$ tangential an den Graphen.
}

\pagebreak

\Satz{ (Cauchy-Kovalevskaya)

    Sei $F = \begin{pmatrix} a \\ b \\ c \end{pmatrix}$ ein Vektorfeld auf (einer
    offenen Menge in) $\R^3$ mit $a(x,y,z) \neq 0$. Dann hat die Differentialgleichung
    \begin{align*}
        \begin{cases}
            a \frac{\partial u}{\partial x} + b \frac{\partial u}{\partial y} = c \\
            u(0,y) = f(y)
        \end{cases}
    \end{align*}
    (mit $f$ gegeben von Klasse $C^1$) eine eindeutige Lösung für kleine $x$,
    i.e. $\abs{x} < T(y)$.
}


\end{document}