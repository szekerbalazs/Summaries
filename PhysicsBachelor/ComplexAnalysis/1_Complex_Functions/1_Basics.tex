\subsection{Basics}

\begin{theorem}
    $\klammer{\cos(\phi) + i \sin(\phi)}^n = \cos(n \phi) + i \sin(n \phi)$
\end{theorem}

\begin{theorem}[Riemann Sphere]
    Represent $\C$ on the unit Sphere $S^2 \subseteq \R^3$ and vice versa.
    Representation $S^2 \rightarrow \C$:
    \begin{align*}
        z = \frac{x_1 + i x_2}{1 - x_3}
    \end{align*}
    Representation $\C \rightarrow S^2$:
    \begin{align*}
        x_1 &= \frac{z - \overline{z}}{i \klammer{1+\abs{z}^2}}
        \\
        x_2 &= \frac{z + \overline{x}}{1+\abs{z}^2}
        \\
        x_3 &= \frac{\abs{z}^2 - 1}{\abs{z}^2 + 1}
    \end{align*}
\end{theorem}

\begin{definition}[Limit]
    The function $f(x)$ is said to have a \fat{limit} $A$ as $x$ tends to $a$
    \begin{align*}
        \limes{x \rightarrow a} f(x) = A
    \end{align*}
    if and only if the following is true:
    For every $\epsilon >0$ there exists a number $\delta >0$ with the property that
    for $x \neq a$: $\abs{x - a} < \delta \ \Rightarrow \ \abs{f(x) - A} < \epsilon$.
\end{definition}

\begin{theorem}[Limes Rules]
    The following are true:
    \begin{align*}
        \limes{x \rightarrow a} \overline{f}(x) &= \overline{A}
        \\
        \limes{x \rightarrow a} \text{Re}(f(x)) &= \text{Re} (A)
        \\
        \limes{x \rightarrow a} \text{Im}(f(x)) &= \text{Im} (A)
    \end{align*} 
    We say $f$ is continuous at $a$ if
    \begin{align*}
        \limes{x \rightarrow a} f(x) = f(a)
    \end{align*}
\end{theorem}

