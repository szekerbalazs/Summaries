\documentclass[a4paper]{article}
% ==== Inputs and Usepackages ====

\usepackage{tablefootnote}
\usepackage{enumerate}
\usepackage{float}
\usepackage{url}
\usepackage{hyperref}
\usepackage{dsfont}
\usepackage{mathrsfs}
\usepackage{amsmath}
\usepackage{amssymb}
\usepackage{amsthm}
\usepackage{amsfonts}
\usepackage{mathtools}
%\usepackage{mathabx}
\usepackage{MnSymbol}
\usepackage{xfrac}
\usepackage{nicefrac}
\usepackage{geometry}
\usepackage{graphicx}
\usepackage{graphics}
\usepackage{latexsym}
\usepackage{setspace}
\usepackage{tikz-cd}
\usepackage{tikz}
 \usetikzlibrary{matrix}
 \usetikzlibrary{calc}
 \usetikzlibrary{circuits.ee.IEC}
\usepackage{circuitikz}

\usepackage{a4wide}
\usepackage{fancybox}
\usepackage{fancyhdr}
\usepackage[utf8]{inputenc}




% ==== Page Settings ====

\hoffset = -1.2 in
\voffset = -0.3 in
\textwidth = 590pt
\textheight = 770pt
\setlength{\headheight}{20pt}
\setlength{\headwidth}{590pt}
\marginparwidth = 0 pt
\topmargin = -0.75 in
\setlength{\parindent}{0cm}


% ==== Presettings for files ====

\pagestyle{fancy}


\cfoot{\thepage}
\lfoot{\href{mailto:szekerb@student.ethz.ch}{szekerb@student.ethz.ch}}
\rfoot{Balázs Szekér, \today}
\lhead{Physics \uproman{3} Summary}


\title{Complex Analysis}
\author{Summary \\ \\ Balázs Szekér \\ \href{mailto:szekerb@student.ethz.ch}{szekerb@student.ethz.ch} \\ \\
Summary of the Lecture Complex Analysis \\ in the Autumn Semester 2020 given by
Afonso Bandeira \\ \\ Swiss Federal Institute of Technology, ETH Zürich} 

\renewcommand{\abstractname}{Preface}

\begin{document}
\begin{titlepage}
    \maketitle
    \thispagestyle{empty}
\end{titlepage}
\pagebreak
\pagenumbering{Roman} 
\thispagestyle{myplain}
\begin{abstract}
    This is a summary of the topics dealt with in the lecture \textit{Complex
    Analysis} in the autumn semester 2020 given by Afonso Bandeira at ETH Zürich.
    This script is based on the lecture notes and the book \textit{Complex Analysis}
    by Lars Ahlfors.\footnote{[Ahlfors] - L. Ahlfors: "Complex analysis. An
    introduction to the theory of analytic functions of one complex variable."
    International Series in Pure and Applied Mathematics. McGraw-Hill Book Co.
    Link to an online PDF: \url{http://people.math.gatech.edu/~mccuan/courses/6321/lars-ahlfors-complex-analysis-third-edition-mcgraw-hill-science_engineering_math-1979.pdf}}
    The list of topics is not exhaustive. Many things were left out and only the
    topics which the author regarded as important are mentioned.
    This summary should neither be considered as a replacement of the lecture nor as
    a sufficient preparation for the exam. This summary should only be a reminder to
    which you can resort in case you quickly want to look something up.
    No liability is accepted in the event of failure to pass the examination.

    \vspace{1\baselineskip}

    If you stumble over mistakes, be it linguistic or thematic, or if you have suggestions
    what to add or how to improve this script, do not hesitate to contact me at
    \href{mailto:szekerb@student.ethz.ch}{szekerb@student.ethz.ch}.

    \vspace{1\baselineskip}

    Many Thanks
    
    \vspace{1\baselineskip}

    Balázs Szekér
\end{abstract}
\newpage
\thispagestyle{myplain}
\tableofcontents
\pagebreak
\pagenumbering{arabic}
%\setcounter{page}{1}

\section{Complex Functions}

\vspace{1\baselineskip}

\subsection{Basics}

\begin{theorem}
    $\klammer{\cos(\phi) + i \sin(\phi)}^n = \cos(n \phi) + i \sin(n \phi)$
\end{theorem}

\begin{theorem}[Riemann Sphere]
    Represent $\C$ on the unit Sphere $S^2 \subseteq \R^3$ and vice versa.
    Representation $S^2 \rightarrow \C$:
    \begin{align*}
        z = \frac{x_1 + i x_2}{1 - x_3}
    \end{align*}
    Representation $\C \rightarrow S^2$:
    \begin{align*}
        x_1 &= \frac{z - \overline{z}}{i \klammer{1+\abs{z}^2}}
        \\
        x_2 &= \frac{z + \overline{x}}{1+\abs{z}^2}
        \\
        x_3 &= \frac{\abs{z}^2 - 1}{\abs{z}^2 + 1}
    \end{align*}
\end{theorem}

\begin{definition}[Limit]
    The function $f(x)$ is said to have a \fat{limit} $A$ as $x$ tends to $a$
    \begin{align*}
        \limes{x \rightarrow a} f(x) = A
    \end{align*}
    if and only if the following is true:
    For every $\epsilon >0$ there exists a number $\delta >0$ with the property that
    for $x \neq a$: $\abs{x - a} < \delta \ \Rightarrow \ \abs{f(x) - A} < \epsilon$.
\end{definition}

\begin{theorem}[Limes Rules]
    The following are true:
    \begin{align*}
        \limes{x \rightarrow a} \overline{f}(x) &= \overline{A}
        \\
        \limes{x \rightarrow a} \text{Re}(f(x)) &= \text{Re} (A)
        \\
        \limes{x \rightarrow a} \text{Im}(f(x)) &= \text{Im} (A)
    \end{align*} 
    We say $f$ is continuous at $a$ if
    \begin{align*}
        \limes{x \rightarrow a} f(x) = f(a)
    \end{align*}
\end{theorem}



\vspace{1\baselineskip}

\subsection{Analytic Functions}

\begin{definition}[Derivative]
    The \fat{derivative} of a function $f$ is defined as
    \begin{align*}
        f'(a) = \limes{x \rightarrow a} \frac{f(x) - f(a)}{x-a}
        = \limes{h \rightarrow 0} \frac{f(a+h) - f(a)}{h}
    \end{align*}
\end{definition}

\begin{theorem}
    Let $f: \C \rightarrow \R$ be a function. If we now want to calculate the derivative,
    of $f$ at a point $z_0$, then we can approach $z_0$ from different sides, in particular
    purely imaginary and purely real. That means:
    \begin{align*}
        f'(z_0) &= \limes{\stackrel{h \rightarrow 0}{h \in \R}} \frac{f(z_0 + h) - f(z_0)}{h}
        \in \R
        \\
        f'(z_0) &= \limes{\stackrel{h \rightarrow 0}{h \in \R}} \frac{f(z_0 + h) - f(z_0)}{i h}
        \in \C
    \end{align*}
    In order for this to be true, $f'(z_0)$ has to be zero. Thus, a real function of a
    complex variable either has derivative zero, or else the the derivative does
    not exist.
\end{theorem}

\begin{theorem}
    If we write $z(t) = x(t) + i y(t)$ we find $z'(t) = x'(t) + i y'(t)$.
\end{theorem}

\begin{definition}[Analytic/Holomorphic]
    A complex function $f$ is called \fat{analytic} or \fat{Holomorphic} if $f$
    possesses a derivative wherever the function is defined.
\end{definition}

\begin{definition}[Cauchy-Riemann Equations]
    Let $f$ be an analytic function.
    If we write $f(z) = f(x,y) = u(x) + i v(y)$ we obtain the \fat{Cauchy-Riemann Equations}.
    \begin{align*}
        \frac{\partial u}{\partial x} = \frac{\partial v}{\partial y}
        \hspace{20pt}
        \frac{\partial u}{\partial y} = - \frac{\partial v}{\partial x}
    \end{align*}
    Derivation:
    \begin{align*}
        f'(z) &= \limes{h \rightarrow 0} \frac{f(z+h) - f(z)}{h} = \frac{\partial f}{\partial x} = \frac{\partial u}{\partial x} + i \frac{\partial v}{\partial x}
        \\
        f'(z) &= \limes{k \rightarrow 0} \frac{f(z+ik) - f(z)}{i k} = -i \frac{\partial f}{\partial y} = -i \frac{\partial u}{\partial y} + \frac{\partial v}{\partial y}
    \end{align*}
\end{definition}

\begin{lemma}
    All analytic functions fulfil the Cauchy-Riemann Equations.
\end{lemma}

\begin{theorem}
    Derivatives of analytic functions are again analytic.
\end{theorem}

\begin{definition}[Harmonic]
    A function $f$ that fulfils the Laplace equation $\Delta f = 0$ is called
    \fat{harmonic}.
\end{definition}

\begin{theorem}
    All analytic functions are harmonic. In particular
    for $f = u + i v$, both $u$ and $v$ are harmonic.
\end{theorem}

\begin{definition}[Harmonic Conjugate]
    If two harmonic functions $u$ and $v$ satisfy the Cauchy-Riemann equations,
    then $v$ is the \fat{harmonic conjugate} of $u$.
\end{definition}

\begin{theorem}
    If $u(x,y)$ and $v(x,y)$ have continuous first-order partial derivatives which
    satisfy the Cauchy-Riemann differential equations, then $f(z) = u(z) + i v(z)$
    is analytic with continuous derivative $f'(z)$, and conversely.
\end{theorem}

\begin{theorem}[Luca's Theorem]
    If all zeros of a polynomial $P(z)$ lie in a half plane, then all
    zeros of the derivative $P'(z)$ lie in the same half plane.
\end{theorem}

\begin{theorem}
    Let $P(z)$ be a polynomial of degree $n$ and let $Q(z)$ be a polynomial of degree
    $m$. We define $R(z) = \frac{P(z)}{Q(z)}$. The following statements are true.
    \begin{enumerate}[(i)]
        \item If $m>n$, then $R(z)$ has a zero of order $m-n$ at $\infty$.
        \item If $m<n$, then $R(z)$ has a pole of order $n-m$ at $\infty$.
        \item If $n=m$, then $R(\infty) = \frac{a_n}{b_n}$.
    \end{enumerate}
\end{theorem}


\vspace{1\baselineskip}

\subsection{Sequences and Series}

\begin{definition}
    A sequence $\geschwungeneklammer{a_n}_1^\infty$ has the \fat{limit} $A$ if to
    every $\epsilon > 0$ there exists an $N$ such that for $n \geq N$: $\abs{a_n - A}
    < \epsilon$. A sequence with a finite limit is called
    \fat{convergent} and if it is not convergent it is \fat{divergent}.
\end{definition}

\begin{definition}[Cauchy Sequence]
    A sequence $\geschwungeneklammer{a_n}_1^\infty$ is called a \fat{Cauchy sequence}
    if it satisfies the following:
    \begin{align*}
        \forall \epsilon > 0 \ \exists N \ : \ \forall n,m > N \ : \ \abs{a_n - a_m} < \epsilon
    \end{align*}
\end{definition}

\begin{definition}[Convergence]
    We can define two different types of \fat{convergence}.
    \begin{enumerate}[(i)]
        \item \fat{pointwise:} $\forall \epsilon > 0 \ \forall x \ \exists N : \forall n > N : \abs{f_n(x) - f(x)} < \epsilon$
        \item \fat{uniformal:} $\forall \epsilon > 0 \ \exists N : \forall x \ \forall n > N : \abs{f_n(x) - f(x)} < \epsilon$
    \end{enumerate}
\end{definition}

\begin{proposition}
    If $f_n$ are continuous and $f_n \rightarrow f$ converges uniformally than $f$
    if continuous.
\end{proposition}

\begin{theorem}[Weierstrass M-Test]
    Suppose that $(f_n)_n$ is a sequence of functions defined on a set $A$ and that
    there is a sequence of non negative numbers $(M_n)_n$ satisfying
    \begin{align*}
        \forall n \geq 1 , \forall x \in A : \ \abs{f_n(x)} \leq M_n
        \ \text{ and } \ \sum_{n=1}^\infty M_n < \infty
    \end{align*}
    Then the series $\sum_{n=1}^\infty f_n(x)$ converges absolutely and uniformaly
    on $A$.
\end{theorem}

\begin{definition}[Radius of Convergence]
    The \fat{radius of convergence} of a series $\geschwungeneklammer{a_n}_1^\infty$ is
    defined as
    \begin{align*}
        R = \frac{1}{\limsup_{n \rightarrow \infty} \sqrt[n]{\abs{a_n}}}
        = \limes{n \rightarrow \infty} \abs{\frac{a_n}{a_{n+1}}}
    \end{align*}
\end{definition}

\begin{theorem}[Abel]
    For every power series there exists a number $R$, $0 \leq R \leq \infty$, called
    the radius of convergence, with the following properties:
    \begin{enumerate}[(i)]
        \item The series converges absolutely for every $z$ with $\abs{z} < R$. If
            $0 \leq \rho < R$ the convergence is uniform for $\abs{z} \leq \rho$.
        \item If $\abs{z} > R$ the terms of the series are unbounded, and the series
            is consequently divergent.
        \item In $\abs{z} < R$ the sum of the series is an analytic function. The
            derivative can be obtained by termwise differentiation, and the derived
            series has the same radius of convergence.
    \end{enumerate}
\end{theorem}



\vspace{1\baselineskip}

\subsection{Exponential and Trigonometric Functions}

\begin{definition}[Complex Logarithm]
    For a complex number $w$, we define the \fat{complex logarithm} as
    \begin{align*}
        \log (w) = \log \klammer{\abs{w}} + i \arg (w)
    \end{align*}
    As we can see, $\log$ is \fat{multivalued}, since $e^{ix} = e^{ix + i 2 \pi  k} \ \forall k \in \Z$.
    Thus
    \begin{align*}
        \log (e^{ix}) = \log \klammer{\abs{e^{ix}}} + i (x+2\pi k)
        = \log(1) + i (x + 2 \pi k)
        = i \klammer{x + 2 \pi k} \ \forall k \in \Z
    \end{align*}
\end{definition}

\begin{definition}[Inverse Sine and Cosine]
    The \fat{inverses} of $\sin$ and $\cos$ are given by
    \begin{align*}
        \arccos (w) &= -i \log \klammer{w \pm \sqrt{w^2 - 1}}
            = \pm i \log \klammer{w + \sqrt{w^2 -1}}
        \\
        \arcsin (w) &= \frac{\pi}{2} - \arccos (w)
    \end{align*}
\end{definition}

\begin{definition}[Branch Cut]
    A \fat{branch cut} is a part of a set where a multivalues function is not defined.
    At this point, two branches are separated.
\end{definition}

\begin{example}[Square Root]
    Let $f$ be $f(z) = \sqrt{z}$. Say $z=r e^{i \theta}$. Since $e^{i \theta} =
    e^{i (\theta + 2 \pi)}$, we have two legitimate solutions.
    \begin{align*}
        \sqrt{z}= \begin{cases}
            \sqrt{r e^{i \theta}} = \sqrt{r} e^{i \theta/2}
            \\
            \sqrt{r e^{i (\theta + 2 \pi)}} = \sqrt{r} e^{i \theta /2 + i \pi}
            = - \sqrt{r} e^{i \theta/2}
        \end{cases}
    \end{align*}
    To make $g(w) = w^2$ injective, we have to restrict $\theta \in
    \klammer{-\frac{\pi}{2} , \frac{\pi}{2}}$. With this restriction we now can
    compute $f(z)$ uniquely. We call such a restriction of the domain a \fat{branch}.
    Note that there are only two branches of the square root, namely $(0,2\pi)$ and
    $(2\pi,4\pi)$. All other branches deliver the same result as one of these two.
    $(0,2\pi)$ maps to $(0,\pi)$ and $(2\pi,4\pi)$ maps to $(\pi,2\pi)$. All other
    branches are equivalent to these branches because they are the same modulo $2\pi$.
\end{example}

\begin{example}[Logarithm]
    The logarithm is also a multivalued function. Analogously to the squareroot
    we can define branch cuts anywhere between $0$ and $2\pi$. The specialty about
    the logarithm is, that we have infinitely many branches.
\end{example}



\pagebreak

\section{Analytic Functions as Mappings}

\vspace{1\baselineskip}

\subsection{Point Set Topology}

\begin{definition}[Metric Space]
    A set $S$ is a \fat{metric space} if there is defined, for every pair $x,y \in S$,
    a nonnegative real number $d(x,y)$ in such a way that the following conditions
    are fulfilled:
    \begin{enumerate}
        \item $d(x,y) = 0$ if and only if $x=y$
        \item $d(y,x) = d(x,y)$
        \item $d(x,z) \leq d(x,y) + d(y,z)$
    \end{enumerate}
\end{definition}

\begin{definition}[Neighborhood]
    A set $Y \subseteq X$ is called a \fat{neighborhood} of $y \in X$ if it contains a ball
    $B(y,\delta)$.
\end{definition}

\begin{definition}[Open Set]
    A set is \fat{open} if it is a neighborhood of each of its elements.
\end{definition}

\begin{theorem}
    The following stetements are true.
    \begin{enumerate}[(i)]
        \item The intersection of a finite number of open sets is open.
        \item The union of any collection of open sets is open.
        \item The union of a finite number of closed sets is closed.
        \item The intersection of any collection of closed sets is closed.
    \end{enumerate}
\end{theorem}

\begin{definition}[Interior]
    The \fat{interior} $\text{Int} (X)$ of a set $X$ is the largest open set contained in $X$.
\end{definition}

\begin{definition}[Closure]
    The \fat{closure} $X^-$ or $\overline{X}$ of a set $X$ is the smalles closed set which
    contains $X$. A point belongs to the closure of $X$ if and only if all its
    neighborhoods intersect $X$.
\end{definition}

\begin{definition}[Boundry]
    The \fat{boundry} $\partial X$ of $X$ is the closure minus the interior. A point belongs to the
    boundry of $X$ if and only if all its neighborhoods intersect both $X$ and $X^C$.
\end{definition}

\begin{definition}[Connectedness]
    A subset of a metric space is \fat{connected} if it cannot be represented as the
    union of two disjoint relatively open sets none of which is empty.
    (A set $S$ if connected if and only if $\not\exists A,B$ open and non-empty such
    that $A \cup B = S$ and $A \cap B = \emptyset$.)
\end{definition}

\begin{theorem}
    A open set is connected if it cannot be decomposed into two open sets, and a
    closed set is connected if it cannot be decomposed into two closed sets.
\end{theorem}

\begin{theorem}
    The nonempty connected subsets of $\R$ are the intervals.
\end{theorem}

\begin{theorem}
    Any closed and bounded nonempty set of real numbers has a minimum and a maximum.
\end{theorem}

\begin{theorem}
    A nonempty open set in the plane is connected if and only if any two of its
    points can be joined by a polygon which lies in the set.
\end{theorem}

\begin{definition}[Region]
    A nonempty connected open set is called \fat{region}.
\end{definition}

\begin{definition}[Completeness]
    A metric space is said to be \fat{complete} if every Cauchy sequence is convergent.
\end{definition}

\begin{definition}[Compactness]
    A set $X$ is \fat{compact} if and only if every open covering of $X$ contains
    a finite subcovering. Equivalent: $S$ is compact if and only if for any open
    covering of $S$ there exists a finite subcovering $O_x$ open, such that
    $S \subseteq \bigcup_{x \in X} O_x$.
\end{definition}

\begin{definition}[Totally Bounded]
    A set $X$ is \fat{totally bounded} if, $\forall \epsilon >0$, $X$ can be covered by
    finitely many balls of radius $\epsilon$.
\end{definition}

\begin{theorem}
    A set is compact if and only if it is complete and totally bounded.
\end{theorem}

\begin{theorem}
    A subset of $\R$ or $\C$ is compact if and only if it is closed and bounded.
\end{theorem}

\begin{theorem}
    A metric space is compact if and only if every infinite sequence has a limit point.
\end{theorem}

\begin{definition}[Continuity]
    A function is \fat{continuous} if and only if the inverse image of every open set is open.
    Equivalent: A function is continuous if and only if the inverse image of every
    closed set is closed.
\end{definition}

\begin{theorem}
    Under a continuous mapping the image of every compact set is compact, and
    consequently closed.
\end{theorem}

\begin{theorem}
    A continuous real-valued function on a compact set has a maximum and a minimum.
\end{theorem}

\begin{theorem}
    Under a continuous mapping the image of any connected set is connected.
\end{theorem}

\begin{definition}[Uniform Continuity]
    A function is called \fat{uniformally continuous} if
    \begin{align*}
        \forall \epsilon > 0 \ \exists \delta > 0 : \forall x_1,x_2 : \ d(x_1,x_2) <
        \delta \Rightarrow d(f(x_1),f(x_2)) < \epsilon
    \end{align*}
\end{definition}

\begin{theorem}
    On a compact set every continuous function is uniformally continuous.
\end{theorem}


\vspace{1\baselineskip}

\subsection{Conformality}

\begin{definition}[Arc]
    An \fat{arc} is a continuous function $\gamma$ that maps $t \in [\alpha,\beta]
    \subseteq \R$ to a plane (or generally on a subset of $\R^n$). For an arc on $\C$
    we can define $\gamma$ as $z(t) = x(t) + i y(t)$ with $x(t),y(t)$ continuous
    functions.
\end{definition}

\begin{definition}[Differentiable Arc]
    An arc is said to be \fat{differentiable} if $z'(t)$ exists and is continuous.
    If $z'(t) \neq 0$, the arc is said to be \fat{regular}. An arc is \fat{piecewise
    differentiable} or {piecewise regular} if the same conditions hold exept for a
    finite number of values $t$. At these $t$ points $z(t)$ shall still be continuous
    with left and right derivatives which are equal to the left and right limits of
    $z'(t)$.
\end{definition}

\begin{definition}[Jordan Arc]
    An arc is \fat{simple}, or a \fat{Jordan arc}, if $z(t_1) = z(t_2)$ only for
    $t_1 = t_2$.
\end{definition}

\begin{definition}[Closed Curve]
    An arc is a \fat{closed arc} if the endpoints coincide: $z(\alpha) = z(\beta)$.
\end{definition}

\begin{definition}[Opposite Arc]
    The \fat{opposite arc} of $z(t)$ for $\alpha \leq t \leq \beta$ is the arc
    $z = z(-t)$ for $-\beta \leq t \leq \alpha$. Opposite arcs are sometimes donated by
    $\gamma$ and $-\gamma$, sometimes $\gamma$ and $\gamma^{-1}$, depending on the
    connection.
\end{definition}

\begin{definition}[Analytic Function]
    A complex-valued function $f(z)$, defined on an open set $\Omega$, is said to be
    \fat{analytic} or \fat{holomorphic} in $\Omega$ if it has a derivative at each
    point of $\Omega$.
\end{definition}

\begin{theorem}
    An analytic function $f: \Omega \rightarrow \C$ in a region $\Omega$ with
    $f'(z) = 0 \ \forall z \in \Omega$, then $f$ is constant in $\Omega$. The same
    is true if either the real part, the imaginary part, the modulus, or the argument
    is constant.
\end{theorem}

\begin{definition}[Image of Arcs]
    Suppose that an arc $\gamma$ with the equation $z=z(t)$ for $\alpha \leq t \leq \beta$
    is contained in a region $\Omega$, and let $f(z)$ be defined and continuous in
    $\Omega$. Then the equation $w=w(t)=f(z(t))$ defines an arc $\gamma'$ in the
    $w$-plane which may be called the \fat{image of $\gamma$}.
\end{definition}

\begin{theorem}
    If $z'(t)$ exists, we find that $w'(t)$ also exists and is determined by
    $w'(t) = f'(z(t)) z'(t)$.
\end{theorem}

\begin{theorem}
    Let $f: \Omega \rightarrow \C$ be analytic and $z_0 \in \Omega$ such that $f'(z_0)
    \neq 0$ and $z'(t_0) \neq 0$. Then $f$ takes tangent arcs at $z_0$ into tangent
    arcs at $f(z_0)$. The function $f$ takes two arcs at a relativ angle $\theta$ at
    $t_0$ to arcs with the same relative error.

    \begin{proof}
        Let $z_1$ and $z_2$ be two arcs that cross $z_0$ at $t_0$. Then the following
        is true.
        \begin{align*}
            z_1(t_0) &= z_2(t_0)
            \\
            f(z_1(t_0)) &= w_1(t_0)
            \\
            f(z_2(t_0)) &= w_2(t_0)
            \\
            \arg \klammer{w_2'(t_0)} - \arg(w_1'(t_0)) &=
            \arg(f'(z_2(t_0)) z_2'(t_0)) - \arg(f'(z_1(t_0)) z_1'(t_0))
            \\
            &= \arg(f'(z_2(t_0))) + \arg(z_2'(t_0)) - \arg(f'(z_1(t_0)))
                - \arg(z_1'(t_0))
            \\
            &= \arg(z_2'(t_0)) - \arg(z_1'(t_0))
        \end{align*}
    \end{proof}
\end{theorem}

\begin{theorem}
    If $f$ is analytic and $f'(z_0) \neq 0$, then $f$ is conformal at $z_0$. 
\end{theorem}


\pagebreak

\section{Complex Integration}

\vspace{1\baselineskip}

\subsection{Fundamental Theorems}

\begin{theorem}
    If $f(z) = u(t) + i v(t)$ is a continuous function, defined on an interval $(a,b)$,
    then we set
    \begin{align*}
        \int_a^b f(t) \ dt = \int_a^b u(t) \ dt + i \int_a^b v(t) \ dt
    \end{align*}
\end{theorem}

\begin{theorem}
    Let $\gamma$ be a piecewise differentiable arc with the equation $z = z(t)$
    with $a \leq t \leq b$. If the function $f(z)$ is defined and continuous
    on $\gamma$, then $f(z(t))$ is also continuous and we can set
    \begin{align*}
        \int_\gamma f(z) \ dz = \int_a^b f(z(t)) z'(t) \ dt
    \end{align*}
\end{theorem}

\begin{theorem}
    Let $-\gamma$ be the opposite arc of $\gamma$. That means, $-\gamma(z) = \gamma(-z)$.
    However, for $-\gamma$, $t$ is in the interval $-b \leq t \leq -a$. Thus:
    \begin{align*}
        \int_{-\gamma} f(z) \ dz = - \int_\gamma f(z) \ dz
    \end{align*}
\end{theorem}

\begin{theorem}
    If an arc $\gamma$ consists of many subarcs $\gamma_1,\dots,\gamma_n$, then
    \begin{align*}
        \int_\gamma f \ dz = \int_{\gamma_1 + \dots + \gamma_n} f \ dz
        = \int_{\gamma_1} f \ dz + \dots + \int_{\gamma_n} f \ dz
    \end{align*}
\end{theorem}

\begin{theorem}
    \begin{align*}
        \int_\gamma f \ \overline{dz} = \overline{\int_\gamma \overline{f} \ dz}
    \end{align*}
\end{theorem}

\begin{theorem}
    If $f = u + i v$, then we can write
    \begin{align*}
        \int_\gamma f \ dz = \int_\gamma (u \ dx - v \ dy) + i \int_\gamma (u \ dy + v \ dx)
    \end{align*}
\end{theorem}

\begin{theorem}
    If two arcs $\gamma_1$ and $\gamma_2$ have the same initial points and the same
    end point, we require
    \begin{align*}
        \int_{\gamma_1} p \ dx + q \ dy = \int_{\gamma_2} p \ dx + q \ dy
    \end{align*}
\end{theorem}

\begin{theorem}
    The integral over any closed curve vanishes.
\end{theorem}

\begin{theorem}
    The line integral $\int_\gamma p \ dx + q \ dy$, defined in $\Omega$, depends
    only on the end points of $\gamma$ if and only if there exists a function
    $U(x,y)$ in $\Omega$ with the partial derivatives $\frac{\partial U}{\partial x} = p$,
    $\frac{\partial U}{\partial y} = q$.
\end{theorem}

\begin{definition}[Exact Differential]
    An expression $p \ dx + q \ dy$ which can be written as
    $d U = \frac{\partial U}{\partial x} \ dx + \frac{\partial U}{\partial y} \ dy$
    is called an \fat{exact differential}.
\end{definition}

\begin{theorem}
    An integral depends only on the end points if and only if the integrand is an exact
    differential.
\end{theorem}

\begin{theorem}
    A function $f$ is an exact differential if and only if there exists a function
    $F$ in $\Omega$ such that
    \begin{align*}
        \frac{\partial F(z)}{\partial x} = f(z)
        \ \ \ \
        \frac{\partial F(z)}{\partial y} = i f(z)
    \end{align*}
    If this is true, $F$ fulfills the Cauchy-Riemann equations.
\end{theorem}

\begin{theorem}
    The integral $\int_\gamma f \ dz$, with continuous $f$, depends only on the
    end points of $\gamma$ if and only if $f$ is the derivative of an analytic
    function in $\Omega$.
\end{theorem}

\begin{theorem}
    If the function $f(z)$ is analytic on $R$, then
    \begin{align*}
        \int_{\partial R} f(z) \ dz = 0
    \end{align*}
\end{theorem}

\begin{theorem}
    Let $f(z)$ be analytic on the set $R'$ obtained from a rectangle $R$ by
    omitting a finite number of interior points $\xi_j$. If it is true that
    $\limes{z \rightarrow \xi_j} (z-\xi_j) f(z) = 0 \ \forall j$, then
    \begin{align*}
        \int_{\partial R} f(z) \ dz = 0
    \end{align*}
\end{theorem}

\begin{theorem}[Cauchy's Theorem on a disk]
    If $f(z)$ is analytic in an open disk $\Delta = \geschwungeneklammer{z \in \C :
    \abs{z - z_0} < r}$, then
    \begin{align*}
        \int_\gamma f(z) \ dz = 0
    \end{align*}
    for every closed curve $\gamma$ in $\Delta$.
\end{theorem}


\vspace{1\baselineskip}

\subsection{Cauchy's Integral Formula}

\begin{theorem}
    If a piecewise differentiable closed curve $\gamma$ does not pass through the point
    $a$, then the following is true.
    \begin{align*}
        \int_\gamma \frac{dz}{z-a} = 2 \pi i
    \end{align*}
    Equivalent: $a \in \C$ and $C = \geschwungeneklammer{z \in \C : \abs{z-a} < r}$
    \begin{align*}
        \int_C \frac{dz}{z-a} = 2 \pi i
    \end{align*}
\end{theorem}

\begin{definition}[Winding Number]
    Let $\gamma$ be a closed piecewise differentiable curve and $a \notin \gamma$.
    Then we define the so called \fat{winding number} $\eta(\gamma,a)$ as
    \begin{align*}
        \eta(\gamma,a) := \frac{1}{2 \pi i} \int_\gamma \frac{1}{z-a} \ dz
    \end{align*}
    This number is always an integer.
\end{definition}

\begin{proposition}
    If $\gamma$ lies inside of a circle, then $\eta(\gamma,a) = 0$ for all points
    $a$ outside of the same circle.
\end{proposition}

\begin{theorem}
    Let $\gamma$ be a closed curve in $\C$. If $a,b \in \C$ are in the same region
    determined by $\gamma$, then $\eta(\gamma,a) = \eta(\gamma,b)$.
\end{theorem}

\begin{theorem}[Cauchy's Integral Formula]
    Let $f: \Delta \rightarrow \C$ be analytic in an open disk $\Delta \subset \C$
    and let $\gamma$ be a closed curve in $\Delta$. For any point $a \notin \gamma$
    the following is true.
    \begin{align*}
        \eta(\gamma,a) f(a) = \frac{1}{2 \pi i} \int_\gamma \frac{f(z)}{z-a} \ dz
    \end{align*}
\end{theorem}

\begin{theorem}[Cauchy's Integral Formula]
    Let $f$ be analytic on an open disk $\Delta$ and let $\gamma$ be a closed
    curve in $\Delta$. If $z \in \gamma$ and $\eta(\gamma,z) = 1$ the following
    is true.
    \begin{align*}
        f(z) = \frac{1}{2 \pi i} \int_\gamma \frac{f(\xi)}{\xi - z} \ d \xi
    \end{align*}
\end{theorem}

\begin{corollary}
    Let $f, \Delta$ and $\gamma$ be as above. Assume $\eta(\gamma,a) = 1$. If $a \neq b$,
    then we can write:
    \begin{align*}
        \frac{1}{2 \pi i}
        \int_\gamma \frac{f(z)}{(z-a)(z-b)} \ dz
        = \frac{1}{2 \pi i (a-b)} \klammer{\int_\gamma \frac{f(z)}{z-a} \ dz -
            \int_\gamma \frac{f(z)}{z-b} \ dz}
        = \frac{1}{a-b} \klammer{\eta(\gamma,a) f(a) - \eta(\gamma,b) f(b)}
    \end{align*}
    Since $\frac{1}{(z-a)(z-b)} = \frac{1}{a-b} \klammer{\frac{1}{z-a} - \frac{1}{z-b}}$.
    If however $a=b$, then:
    \begin{align*}
        \frac{1}{2 \pi i} \int_\gamma \frac{f(z)}{(z-a)^2} \ dz
        = \eta(\gamma,a) f'(a)
    \end{align*}
\end{corollary}

\begin{theorem}
    With the above conditions the following is true.
    \begin{align*}
        f^{(n)} (z) = \frac{n!}{2 \pi i} \int_\gamma \frac{f(\xi)}{(\xi -z)^{n+1}} \ d \xi
    \end{align*}
\end{theorem}

\begin{lemma}
    Suppose that $\phi(\xi)$ is continuous on the arc $\gamma$. Then the function
    \begin{align*}
        F_n (z) = \int_\gamma \frac{\varphi(\xi)}{(\xi-z)^n} \ d \xi
    \end{align*}
    is analytic in each of the regions determined by $\gamma$, and its derivative
    is $F_n'(z) = n F_{n+1} (z)$.
\end{lemma}

\begin{theorem}
    If $f: \Omega \rightarrow \C$ is analytic and $C$ is a circle in $\Omega$ for all
    $z$ in the inside of $C$ it is true that
    \begin{align*}
        f(z) = \frac{1}{2 \pi i} \int_C \frac{f(\xi)}{\xi - z} \ d \xi
        \ \ \text{ and } \ \
        f^{(n)} (z) = \frac{n!}{2 \pi i} \int_C \frac{f(\xi)}{(\xi - z)^{n+1}} \ d \xi
    \end{align*}
\end{theorem}

\begin{theorem}
    If $f: \Omega \rightarrow \C$ is analytic in $\Omega \subseteq \C$ a region,
    then $f$ has $n$ derivatives.
\end{theorem}

\begin{theorem}[Morera's Theorem]
    Let $f: \Omega \rightarrow \C$ be continuous in a region $\Omega$. If for all
    closed paths $\gamma$ it is true that $\int_\gamma f(z) \ dz = 0$, then $f$
    is analytic in $\Omega$.
\end{theorem}

\begin{theorem}
    Let $f$ be analytic in a region containing $C = \geschwungeneklammer{z :
    \abs{z-a} < r}$ and the inside of $C$. Further, let $\max_{z \in C} \abs{f(z)}
    \leq M$. Then
    \begin{align*}
        f^{(n)} (a) = \frac{n!}{2 \pi i} \int_C \frac{f(\xi)}{(\xi - z)^n} \ d \xi
        \leq \frac{n!}{2 \pi} M \frac{1}{r^n} \int_C \abs{d \xi}
        \leq \frac{n!}{2 \pi} M \frac{1}{r^n} 2 \pi
        \leq \frac{n! M}{r^n}
    \end{align*}
    If $f$ is analytic in $B(a,r')$ and $\abs{f(z)} < M \ \forall z \in C$ where
    $C = \geschwungeneklammer{z : \abs{z-a} < r}$ with $r<r'$, then
    $\abs{f^{(n)} (z)} \leq M \frac{n!}{r^n}$.
\end{theorem}

\begin{theorem}[Liouville's Theorem]
    If $f: \C \rightarrow \C$ is analytic in all $\C$ and bounded, then $f$ must
    be constant.
\end{theorem}


\vspace{1\baselineskip}

\subsection{Local Properties of Analytic Functions}

\begin{theorem}
    Cauchy's integral formula still holds if $f$ has singularities
    $\geschwungeneklammer{a_j}_{j \in J}$. The function $f$ is analytic
    in $\Delta \backslash \geschwungeneklammer{a_j}_{j \in J}$ provided that
    $\limes{z \rightarrow a_j} (z-a_j)f(z) = 0 \ \forall j \in J$ as long as
    $\geschwungeneklammer{a_j}_{j \in J} \cap \gamma = \emptyset$ and
    $\geschwungeneklammer{a_j}_{j \in J} \cap \geschwungeneklammer{z} = \emptyset$.
\end{theorem}

\begin{definition}[Removable Singularity]
    Suppose that $f(z)$ is analytic in the region $\Omega'$ obtained by omitting
    a point $a$ from the region $\Omega$. A necessary and sufficient condition that
    there exists an analytic function in $\Omega$ which coincided with $f(z)$ in
    $\Omega'$ is that $\limes{z \rightarrow a} (z-a)f(z-a) = 0$. The extended function
    is uniquely determined by
    \begin{align*}
        f(a) = \frac{1}{2 \pi i} \int_C \frac{f(\xi)}{\xi - a} \ d \xi
    \end{align*}
    where $C$ is a circle with $a$ in the inside of $C$ and $C$ and the inside of $C$
    being in $\Omega$. We call this a \fat{removable singularity}.
\end{definition}

\begin{theorem}
    If $f: \Omega \rightarrow \C$ is analytic in a region $\Omega$ containing $a$,
    then for any $n>0$ integer
    \begin{align*}
        f(z) = f(a) + \frac{(z-a)}{1!} f'(a) + \dotsb + \frac{(z-a)^{n-1}}{(n-1)!} f^{(n-1)} (a)
            + (z-a)^n f_n (a)
    \end{align*}
    where $f_n(z)$ is analytic in $\Omega$ and is given by
    \begin{align*}
        f_n (z) = \frac{1}{2 \pi i} \int_C \frac{f(\xi)}{(\xi - a)^n (\xi - z)} \ d \xi
    \end{align*}
\end{theorem}

\begin{theorem}
    If $f: \Omega \rightarrow \C$ is analytic and $\exists a \in \Omega$ such that
    $f(a) = 0$ and $f^{(n)} (a) = 0$, then $f = 0$ in all of $\Omega$.
\end{theorem}

\begin{definition}[Order of a Zero]
    Take $f: \Omega \rightarrow \C$ analytic and not identically zero. Let $a$ be
    a zero of $f$. Then, $\exists h$ such that $f^{(h)} (a) \neq 0$. Then, $a$ is a
    so called \fat{zero of order $h$}.
\end{definition}

\begin{theorem}
    If $f(a)$ and all derivatives $f^{(n)} (a)$ vanish, we can write
    \begin{align*}
        f(z) = f_n(z) (z-a)^n
    \end{align*}
    with $f_n(a) \neq 0$. Then $\exists \delta > 0$ such that $f_n(z) \neq 0 \
    \forall z : \abs{a-z} < \delta$. Thus, $f(z)$ is non zero in a punctured
    neighborhood of $a$.
\end{theorem}

\begin{definition}[Isolated Zeros]
    The zeros of an analytic function which does not vanish identically are \fat{isolated}.
\end{definition}

\begin{definition}[Isolated Singularity]
    We consider a function $f(z)$ which is analytic in a neighborhood of $a$, exept
    perhaps at $a$ itself. In other words, $f(z)$ shall be analytic in a region
    $0 < \abs{z-a} < \delta$. The point $a$ is called an \fat{isolated singularity}
    of $f(z)$.
\end{definition}

\begin{definition}[Pole]
    If $\limes{z \rightarrow a} f(z) = \infty$, the point $a$ is said to be a \fat{pole}
    of $f(z)$, and we set $f(a) = \infty$.
\end{definition}

\begin{definition}[Order of a Pole]
    There exists a $\delta' \leq \delta$ such that $f(z) \neq 0$ for
    $0<\abs{z-a}<\delta'$. In this region the function $g(z) = \frac{1}{f(z)}$ is
    defined and analytic. But the singularity of $g(z)$ at $a$ is removable, and
    $g(z)$ has an analytic extension with $g(a) = 0$. Since $g(z)$ does not vanish
    identically, the zero at $a$ has a finite order, and we can write $g(z) =
    (z-a)^h g_h (z)$ with $g_h (a) \neq 0$. The number $h$ is the \fat{order of the
    pole}, and $f(z)$ has the representation $f(z) = (z-a)^{-h} f_h(z)$ where
    $f_h(z) = \frac{1}{g_h(z)}$ is analytic and different from zero in a neighborhood
    of $a$.
\end{definition}

\begin{definition}[Meromorphic]
    A function $f(z)$ which is analytic in a region $\Omega$, exept for poles,
    is said to be \fat{meromorphic} in $\Omega$.
    More precisely, to every $a \in \Omega$ there shall exist a neighborhood
    $\abs{z-a} < \delta$, contained in $\Omega$, such that either $f(z)$ is
    analytic in the whole neighborhood, of else $f(z)$ is analytic for
    $0<\abs{z-a}<\delta$, and the isolated singularity is a pole.
\end{definition}

\begin{theorem}
    If $f(z)$ is identically zero in a subregion of $\Omega$, then it is identically
    zero in $\Omega$, and the same is true if $f(z)$ vanishes on an arc which does
    not reduce to a point.
\end{theorem}

\begin{theorem}
    Define the following two conditions. Let $\alpha \in \R$.
    \begin{align*}
        (1) \hspace{20pt} &\limes{z \rightarrow a} \abs{z-a}^\alpha \abs{f(z)} = 0
        \\
        (2) \hspace{20pt} &\limes{z \rightarrow a} \abs{z-a}^\alpha \abs{f(z)} = \infty
    \end{align*}
    If $(1)$ holds for a certain $\alpha$, then it holds for all larger $\alpha$, and
    hence for some integer $m$. Then $(z-a)^m f(z)$ has a removable singularity and
    vanishes for $z=a$. Either $f(z)$ is identically zero, in which case $(1)$ holds
    for all $\alpha$, or $(z-a)^m f(z)$ has a zero of finite order $k$. In the latter
    case it follows at once that $(1)$ holds for all $\alpha > h = m - k$, while $(2)$
    holds for all $\alpha < h$.
    
    Assume that $(2)$ holds for some $\alpha$; then it
    holds for all smaller $\alpha$, and hence for some integer $n$. The function
    $(z-a)^n f(z)$ has a pole of finite order $l$, and setting $h=n+l$ we find again
    that $(1)$ holds for $\alpha > h$ and $(2)$ for $\alpha < h$.

    There are three possibilities here.
    \begin{enumerate}[(i)]
        \item Condition $(1)$ holds for all $\alpha$ and $f(z)$ vanishes identically.
            This case is not very interesting.
        \item There exists an integer $h$ such that $(1)$ holds for $\alpha > h$ and
            $(2)$ for $\alpha < h$. In this case we call $h$ the \fat{algebraic order}
            of $f(z)$ at $a$. It is positive in the case of a pole, negative in case of a
            zero and zero if $f(z)$ is analytic but not $0$ at $a$.
        \item Neither $(1)$ nor $(2)$ holds for any $\alpha$. In this case the point $a$
            is called an \fat{essential isolated singularity}. In the neighborhood of
            an essential singularity $f(z)$ is at the same time unbounded and comes
            arbitrarily close to zero.
    \end{enumerate}
\end{theorem}

\begin{theorem}[Weierstrass]
    An analytic function comes arbitrarily close to any complex value in every neighborhood
    of an essential singularity. Equivalent is: If $a$ is an essential singularity of $f$,
    $\forall \epsilon > 0$ (sufficiently small) $\geschwungeneklammer{f(z) : z \in B(a,\epsilon)}$
    is dense in $\C$.
\end{theorem}

\paragraph{Overview Singularities}
\begin{align*}
    \limes{z \rightarrow a} f(z) \ \text{ exists? } \begin{cases}
        \text{Yes: } \ &\limes{z \rightarrow a} \frac{1}{f(z)} \ \text{ exists? }
        \begin{cases}
            \text{Yes: Removable}
            \\
            \text{No: Zero}
        \end{cases}
        \\
        \text{No: } \ &\limes{z \rightarrow a} \frac{1}{f(z)} \ \text{ exists? }
        \begin{cases}
            \text{Yes: Pole}
            \\
            \text{No: Essential}
        \end{cases}
    \end{cases}
\end{align*}

\subparagraph{Removable Singularity}
Show $\limes{z \rightarrow a} (z-a) f(z) = 0$ and find analytic extension of $f$ in $a$.

\subparagraph{Zero}
Find $n \geq 1$, $n \in \N$ such that $f(z) = (z-a)^n f_n (z)$ with $f_n(a) \neq 0$ and
$f_n$ analytic in $a$. We want $n$ to satisfy
\begin{align*}
    \limes{z \rightarrow a} (z-a)^{-k} f(z) &= 0 \ \forall k < n
    \\
    \limes{z \rightarrow a} (z-a)^{-k} f(z) &= \infty \ \forall k > n
    \\
    \limes{z \rightarrow a} (z-a)^{-k} f(z) &= f_n (a) \neq 0 \ \text{ if } n=k
\end{align*}
$n$ can be found inductively by deriving $f$. Meaning
$f(a) = f'(a) = \dots = f^{(n-1)} (a) = 0$ but $f^{(n)} (a) \neq 0$.

\subparagraph{Poles}
Show $\limes{z \rightarrow a} \abs{f(z)} = \infty$. $f$ has a pole of order $n$
$\Leftrightarrow$ $\frac{1}{f}$ has a zero of order $n$. We do the same as for
the zeros for $\frac{1}{f}$. We want to find $n \geq 1$, $n \in \N$ such that
$f(z) = (z-a)^{-n} f_n(z)$, $f_n(a) < \infty$ and $f_n$ analytic in $a$ and
\begin{align*}
    \limes{z \rightarrow a} f(z) (z-a)^k &= 0 \ \forall k > n
    \\
    \limes{z \rightarrow a} f(z) (z-a)^k &= \infty \ \forall k < n
\end{align*}

\subparagraph{Essential Singularity}
Show that $a$ is not removable and not a pole. Weierstrass Theorem: Show that
$f \klammer{D(a,r) \backslash \geschwungeneklammer{a}} = \C$.

\subparagraph{Classification}
For $n \in \Z$ let $\limes{z \rightarrow a} (z-a)^n f(z) \in \C \backslash
\geschwungeneklammer{0}$.
\begin{align*}
    n \geq 1 &\Rightarrow a \text{ is a pole of order } n
    \\
    n \leq -1 &\Rightarrow a \text{ is a zero of order } -n
    \\
    n = 0 &\Rightarrow a \text{ is a removable singularity}
    \\
    n \text{ does not exist } &\Rightarrow a \text{ is an essential singularity}
\end{align*} 

Find the type of singularity by its expansion. First note, if $f$ has a pole at $a$
of order $n$, $f(z) = (z-a)^{-n} f_n(z)$, where $f_n$ is analytic in $a$
\begin{align*}
    f_n (z) &= \sum_{k=0}^\infty \frac{f_n^{(k)} (a)}{k!} (z-a)^k
    \\
    f(z) &= \frac{f_n (a)}{(z-a)^n} + \frac{f_n'(a)}{(z-a)^{n-1}} + \dots + \frac{f_n^{(n-1)} (a)}{(n-1)! (z-a)}
        + \sum_{k=1}^\infty \frac{f_n^{(k)} (a)}{k!} (z-a)^{k-1}
\end{align*}
We call $S(z) = \sum_{k<0} a_k (z-a)^k$ the \fat{essential / principle part} of $f$.
If $S(z)$ is a finite sum $S(z) = \sum_{k=-m}^{-1} a_k (z-a)^k$, then $a$ is a pole
of order $m$.
If $\limes{z \rightarrow a} (z-a) f(z) = 0 \ \Rightarrow \ \limes{z \rightarrow a} f(z)
\in \C \backslash \geschwungeneklammer{0} \ \Rightarrow \ f$ admits a Taylor expantion.
If $S \equiv 0$, $a$ is removable. If $S$ is an infinite sum, $a$ is essential.


\begin{theorem}
    Let $f(z)$ be an analytic functin which is not identcaly zero in an open disk
    $\Delta$. Let $\gamma$ be a closed curve in $\Delta$ such that $f(z) \neq 0$ on
    $\gamma$. Suppose $f$ has only finitely many zeros $z_1,z_2,\dots,z_n$ in $\Delta$.
    In this case, we can write $f(z) = (z-z_1)(z-z_2) \dots (z-z_n) g(z)$ where $g$
    is analytic and $\neq 0$ in $\Delta$. The \fat{logarithmic derivative} is given
    as
    \begin{align*}
        \frac{f'(z)}{f(z)} &= \frac{1}{z-z_1} + \frac{1}{z-z_2} + \dots + \frac{1}{z-z_n}
            + \frac{g'(z)}{g(z)}
    \end{align*}
    Since $g(z) \neq 0$ in $\Delta$, Cauchy's theorem yields
    \begin{align*}
        \int_\gamma \frac{g'(z)}{g(z)} \ dz = 0
    \end{align*}
    So we find:
    \begin{align*}
        \eta(\gamma,z_1) + \eta(\gamma,z_2) + \dots + \eta(\gamma,z_n) =
        \frac{1}{2 \pi i} \int_\gamma \frac{f'(z)}{f(z)} \ dz
    \end{align*}
\end{theorem}

\begin{theorem}
    Let $z_j$ be the zeros of a function $f(z)$ which is analytic in a disk $\Delta$
    and does not vanish identically, each zero being counted as many times as its
    order indicates. For every closed curve $\gamma$ in $\Delta$ which does not pass
    through a zero
    \begin{align*}
        \sum_j \eta(\gamma,z_j) = \frac{1}{2 \pi i} \int_\gamma \frac{f'(z)}{f(z)} \ dz
    \end{align*}
    where the sum has only a finite number of terms $\neq 0$. Let $w = f(z)$ be
    the function that maps $\gamma$ onto a closed curve $\Gamma$ in the $\omega$-plane,
    and we find
    \begin{align*}
        \int_\Gamma \frac{d w}{w} = \int_\gamma \frac{f'(z)}{f(z)} \ dz
    \end{align*}
    Hence, we arrive at $\eta(\Gamma,0) = \sum_j \eta(\gamma,z_j)$.
\end{theorem}

\begin{theorem}
    Take the same conditions as above. Let each $\eta(\gamma,z_j)$ be either $0$ or $1$.
    Then, we can derive a formula for the total number of zeros enclosed by $\gamma$.
    This is the case, if $\gamma$ is a circle. Donate the names $z_j(a)$ to the roots
    of the equation $f(z) = a$ for some arbitrary complex value. We then obtain
    \begin{align*}
        \sum_j \eta(\gamma,z_j(a)) = \frac{1}{2 \pi i} \int_\gamma \frac{f'(z)}{f(z) - a} \ dz
    \end{align*}
    Hence,
    \begin{align*}
        \eta(\Gamma,a) = \sum_j \eta(\gamma , z_j(a))
    \end{align*}
\end{theorem}

\begin{theorem}
    If $a$ and $b$ are in the same region determined by $\Gamma$, we know that
    $\eta(\Gamma,a) = \eta(\Gamma,b)$, and hence we have also $\sum_j \eta(\gamma,z_j(a))
    = \sum_j \eta(\gamma,z_j(b))$. If $\gamma$ is a circle, it follows that $f(z)$ takes
    the value $a$ and $b$ equally many times inside of $\gamma$.
\end{theorem}

\begin{theorem}
    Suppose that $f(z)$ is analytic at $z_0$, $f(z_0) = w_0$, and that $f(z) -w_0$
    has a zero of order $n$ at $z_0$. $\forall \epsilon > 0$ small enough, $\exists
    \delta > 0$ such that for all $a$ with $\abs{a-w_0} < \delta$ the equation
    $f(z) - a$ has exactly $n$ roots in the dist $\abs{z-z_0} < \epsilon$. 
\end{theorem}

\begin{corollary}
    A nonconstant analytic function maps open sets onto open sets.
\end{corollary}

\begin{theorem}[Maximum Principle]
    If $f(z)$ is analytic and nonconstant in a region $\Omega$, then its absolute
    value $\abs{f(z)}$ has no maximum in $\Omega$.
\end{theorem}

\begin{theorem}
    If $f(z)$ is defined and continuous on a closed bounded set $E$ and analytic
    on the interior of $E$, then the maximum of $\abs{f(z)}$ on $E$ is attained
    on the boundary of $E$.
\end{theorem}


\vspace{1\baselineskip}

\subsection{The General Form of Cauchy's Theorem}

\begin{definition}[Simply Connected]
    A region $\Omega$ is \fat{simply connected} if and only if its complements with respect to the
    extended plane is connected. (Equivalent: If $\Omega$ has no holes.)
\end{definition}

\begin{theorem}
    A region $\Omega$ is simply connected if and only if $\eta(\gamma,a) = 0$ for
    all cycles $\gamma$ in $\Omega$ and all points $a \notin \Omega$.
\end{theorem}

\begin{definition}[Homology]
    A cycle $\gamma$ in an open set $\Omega$ is said to be \fat{homologous to zero}
    with respect to $\Omega$ if $\eta(\gamma,a) = 0 \ \forall a \in \Omega^C$.
    In symbols we write $\gamma \sim 0$ (mod $\Omega$). We say that $\gamma_1 \sim \gamma_2$
    if and only if $\gamma_1 - \gamma_2 \sim 0$ mod $\Omega$.
\end{definition}

\begin{theorem}[General Cauchy's Theorem]
    If $f(z)$ is analytic in $\Omega$, then
    \begin{align*}
        \int_\gamma f(z) \ dz = 0
    \end{align*}
    for every cycle $\gamma$ which is homologous to zero in $\Omega$.
\end{theorem}

\begin{corollary}
    If $f(z)$ is analytic in a simply connected region $\Omega$, then
    \begin{align*}
        \int_\gamma f(z) \ dz = 0
    \end{align*}
    holds for all cycles in $\Omega$.
\end{corollary}

\begin{corollary}
    If $f(z)$ is analytic and $\neq 0$ in a simply connected region $\Omega$,
    then it is possible to define single-valued analytic branches of $\log(f(z))$
    and $\sqrt[n]{f(z)}$ in $\Omega$.
\end{corollary}

\begin{definition}[Multiply Connected]
    A region which is not simply connected is called \fat{multiply connected}.
    More precisely, $\Omega$ is said to have the \fat{finite connectivity} $n$ if
    the complement of $\Omega$ has exactly $n$ components and \fat{infinite connectivity}
    if the complement has infinitely many complements. (Colloquially speaking:
    A region of connectivity $n$ arises by punching $n$ holes in the Riemann sphere.)
\end{definition}

\begin{definition}[Modules of Periodicity]
    Let $f: \Omega \rightarrow \C$ be analytic and $\Omega$ be multiply connected.
    Then we can write
    \begin{align*}
        \int_\gamma f \ dz
        = \int_{c_1 \gamma_1 + \dots + c_{n-1} \gamma_{n-1}} f \ dz
        = c_1 \int_{\gamma_1} f \ dz + \dots + c_{n-1} \int_{\gamma_{n-1}} f \ dz
    \end{align*}
    The numbers
    \begin{align*}
        P_i = \int_{\gamma_i} f \ dz
    \end{align*}
    depend only on the function and not on $\gamma$. They are called \fat{modules
    of periodicity} or \fat{period of $f$}.
\end{definition}


\vspace{1\baselineskip}

\subsection{The Calculus of Residues}

\begin{theorem}[Cauchy's Integral Formula]
    If $f(z)$ is analytic in a region $\Omega$, then
    \begin{align*}
        \eta(\gamma,a) f(a) = \frac{1}{2 \pi i} \int_\gamma \frac{f(z)}{z - a} \ dz
    \end{align*}
    for every cycle $\gamma$ which is homologous to zero in $\Omega$.
\end{theorem}

\begin{definition}[Residue]
    The \fat{residue} of $f(z)$ at an isolated singularity $a$ is the unique complex
    number $R$ which makes $f(z) - \frac{R}{z-a}$ the derivative of a single valued
    analytic function in an annulus $0 < \abs{z-a} < \delta$. Often $R$ is written as
    $R = \text{Res}_{z=a} f(z)$ or $R_j = \text{Res}_{z=a_j} f(z)$.
\end{definition}

\begin{definition}[Residue]
    Equivalent to the above definition is:
    Let $f: D(z_0,r) \backslash \geschwungeneklammer{z_0} \rightarrow \C$ be
    holomorphic with the Laurent expansion $f(z) = \sum_{n=-\infty}^\infty a_n
    (z-z_0)^n$. Then the residue of $f$ at $z_0$ is Res$_{z_0} (f) := a_{-1}$.    
\end{definition}

\begin{theorem}
    If $f(z) - \frac{R}{z-a}$ is the derivative of a single valued analytic function,
    it itself is analytic and thus due to Cauchy's Theorem its integral evaluates to
    zero for every cyclic $\gamma$ which is homologous to zero. Thus we obtain
    \begin{align*}
        \int_\gamma f(z) - \frac{R}{z-a} \ dz &= 0
        \\
        \Leftrightarrow
        \int_\gamma f(z) \ dz = \int_\gamma \frac{R}{z-a} \ dz &= 2 \pi i R
    \end{align*}
    Therefore:
    \begin{align*}
        \text{Res}_{z=a} f(z) = \frac{1}{2 \pi i} \int_\gamma f(z) \ dz
    \end{align*}
\end{theorem}

\begin{theorem}
    Let $f(z)$ be analytic exept for isolated singularities $a_j$ in a region
    $\Omega$. Then
    \begin{align*}
        \frac{1}{2 \pi i} \int_\gamma f(z) \ dz = \sum_j \eta(\gamma,a_j) \text{Res}_{z=a_j} f(z)
    \end{align*}
    for any cycle $\gamma$ which is homologous to zero in $\Omega$ and does not pass
    through any of the points $a_j$. Often only the case $\eta(\gamma,a_j) = 1$ or
    $\eta(\gamma,a_j) = 0$ is considered. Then the theorem simplyfies to
    \begin{align*}
        \frac{1}{2 \pi i} \int_\gamma f(z) \ dz = \sum_j \text{Res}_{z=a_j} f(z)
    \end{align*}
    where the sum is extended over all singularities enclosed by $\gamma$.
\end{theorem}

\begin{theorem}
    Let's say that $f$ has a pole at $z=a$ of order $h$. Then: $(z-a)^{h} f(z)$ is
    analytic in a neighborhood of $a$ and so
    \begin{align*}
        f(z) = B_h (z-a)^{-h} + B_{h-1} (z-a)^{-(h-1)} + \dots + B_1 (z-a)^{-1} + \varphi(z)
    \end{align*}
    with $\varphi (z)$ analytic in a neighborhood of $a$. Then $B_1$ is the residue
    of $f$ at $a$ because $f(z) - \frac{B_1}{z-a}$ is the derivative of a function
    in an annulus $0 < \abs{z-a} < \delta$.
\end{theorem}

\begin{definition}[Bound]
    A cycle $\gamma$ is said to \fat{bound} the region $\Omega$ if and only if
    $\eta(\gamma,a)$ is defined and equal to $1$ for all points $a \in \Omega$
    and either undefined or equal to zero for all points $a \notin \Omega$. 
\end{definition}

\begin{theorem}
    If $\gamma$ bounds $\Omega$ and $f(z)$ is analytic on the set $\Omega + \gamma$,
    then
    \begin{align*}
        \int_\gamma f(z) \ dz = 0
    \end{align*}
    and
    \begin{align*}
        f(z) = \frac{1}{2 \pi i} \int_\gamma \frac{f(\xi)}{\xi - z} \ d \xi
    \end{align*}
    for all $z \in \Omega$. If $f(z)$ is analytic in $\Omega + \gamma$ exept for isolated
    singularities in $\Omega$, then
    \begin{align*}
        \frac{1}{2 \pi i} \int_\gamma f(z) \ dz = \sum_j \text{Res}_{z=a_j} f(z)
    \end{align*}
    where the sum ranges over the singularities $a_j \in \Omega$.
\end{theorem}

\begin{theorem}[Argument Principle]
    If $f(z)$ is meromorphic in $\Omega$ with the zeros $a_j$ and the poles $b_k$,
    then
    \begin{align*}
        \frac{1}{2 \pi i} \int_\gamma \frac{f'(z)}{f(z)} \ dz =
        \sum_j \eta(\gamma,a_j) - \sum_k \eta(\gamma,b_k)
    \end{align*}
    for every cycle $\gamma$ which is homologous to zero in $\Omega$ and does not
    pass through any of the zeros or poles.
\end{theorem}

\begin{theorem}[Rouché]
    Let $\gamma$ be homologous to zero in $\Omega$ and such that $\eta(\gamma,z)$
    is either $0$ or $1$ for any point $z$ not on $\gamma$. Suppose that $f(z)$ and
    $g(z)$ are analytic in $\Omega$ and satisfy the inequality
    $\abs{f(z) - g(z)} < \abs{f(z)}$ on $\gamma$. Then $f(z)$ and $g(z)$ have the same
    number of zeros enclosed by $\gamma$.
\end{theorem}

\begin{theorem}[Rouché]
    Let $\gamma$ be a null-homologous cycle and Jordan bounding the region $D$. Let
    $f,g : U \rightarrow \C$ be holomorphic in $U$ such that in $D \cup \gamma =
    \overline{D} \subseteq U$:
    \begin{align*}
        \forall z \in \gamma: \ \abs{f(z) - g(z)} < \abs{f(z)} + \abs{g(z)}
    \end{align*}
    Then, $f$ and $g$ have the same numbers of zeros in $D$ (with multiplicity).
\end{theorem}

\begin{theorem}
    If $g: \Omega \rightarrow \C$ analytic, then $g(z) \frac{f'(z)}{f(z)}$ has
    residue $g(a) h$ if $a$ is a zero of order $h$ of $f$ and $-g(a) h$ if $a$
    is a pole of order $h$ of $f$.
\end{theorem}

\begin{theorem}
    Let $f(z_0) = w_0$, $f'(z_0) \neq 0$. Then there exists $\delta > 0$ such that
    for $\abs{w - w_0} < \delta$, there exists only one solution to
    $f(z) = w$ in $\abs{z-z_0} < \epsilon$ for an $\epsilon > 0$.
\end{theorem}

\paragraph{Tricks}
\begin{enumerate}[a)]
    \item $f$ has a removable singularity at $z_0$ $\Rightarrow R = 0$
    \item $f$ has a simple pole at $z_0$ $\Rightarrow R = \limes{z \rightarrow z_0} (z-z_0) f(z)$
    \item $f$ has a pole of order $k>1$ at $z_0$ $\Rightarrow R = \frac{1}{(k-1)!} \frac{d^{k-1}}{dz^{k-1}} \klammer{(z-z_0)^k f(z)}_{|_{z=z_0}}$
    \item $f$ has a simple pole at $z_0$ and $g$ analytic in a neighborhood of $z_0$ $\Rightarrow R(f \cdot g) =$ Res$_{z_0} (f \cdot g) = g(z_0) \cdot$ Res$_{z_0} (f)$
\end{enumerate}

\paragraph{Remarks}
\begin{itemize}
    \item For essential singularities, compute the whole Laurent Series.
    \item Res$(f,z_0) = 0 \nRightarrow z_0$ is a removable singularity.
\end{itemize}

\begin{theorem}[Existence of $\log(f)$]
    Let $D$ be a simply connected domain and let $f$ be holomorphic on $D$ such that
    $f(z) \neq 0 \ \forall z \in D$. Then, there exists a holomorphic function $g$
    on $D$ such that $f(z) = e^{g(z)}$. (I.e. $\log$ is holomorphic.) Moreover,
    if $z_0 \in D$ and $f(z_0) = e^{w_0}$, one can pick a function $g$ such that
    $g(z_0) = w_0$.
\end{theorem}

\begin{theorem}
    Let $f: D \rightarrow \C$ be holomorphic in the region $D$ and $f(z) \neq 0 \
    \forall z \in D$. Then, there exists a branch of $\log(f)$ in $D$ such that
    \begin{align*}
        \int_\gamma \frac{f'(z)}{f(z)} \ dz = 0
    \end{align*}
    for all piecewise regular closed paths $\gamma \in \C$.
    This theorem holds in particular for simply connected regions.
\end{theorem}

\begin{theorem}
    Let $P$ and $Q$ be polynomials such that $\frac{P(x)}{Q(x)}$ has no poles in $\R$
    and $\deg \klammer{\frac{P(x)}{Q(x)}} \leq -2$. Then:
    \begin{align*}
        \intii \frac{P(x)}{Q(x)} \ dx = 2 \pi i \sum_{\stackrel{a}{\text{Im}(a) > 0}} \text{res}_a \klammer{\frac{P(z)}{Q(z)}}
    \end{align*}
\end{theorem}

\begin{theorem}
    Let $P$ and $Q$ be polynomials and say $\frac{P(x)}{Q(x)}$ has no poles on
    $(0,\infty)$ and at most a simple pole at $0$. Say $\deg \frac{P}{Q} \leq -2$
    and let $0<\alpha<1$. Then:
    \begin{align*}
        \int_0^\infty x^\alpha \frac{P(x)}{Q(x)} \ dx = \frac{2 \pi i}{1 - e^{2 \pi i \alpha}} \sum_{b \neq 0} \text{Res} \klammer{z^\alpha \frac{P(z)}{Q(z)} , b}
    \end{align*}
\end{theorem}



\pagebreak

\section{Series and Product Developments}

\vspace{1\baselineskip}

\subsection{Power Series Expansion}

\begin{theorem}[Weierstrass]
    Suppose that $f_n(z)$ is analytic in the region $\Omega_n$, and that the
    sequence $\geschwungeneklammer{f_n(z)}$ converges to a limit function $f(z)$ in a
    region $\Omega$, uniformaly on every compact subset of $\Omega$. Then $f(z)$ is
    analytic in $\Omega$. Moreover, $f_n' (z)$ converges uniformaly to $f'(z)$ on
    every compact subset of $\Omega$.
\end{theorem}

\begin{corollary}
    Let $\sum_{n=1}^\infty f_n$ be a series of analytic functions in $U \subseteq \C$
    a region. Assume $\sum_{n=1}^\infty f_n \rightarrow f: U \rightarrow \C$ uniformaly
    on every compact subset of $U$. (Which means that $S_m (z) = f_1(z) + \dots + f_m(z)$
    converges uniformaly locally to $f$.) Then, $f$ is analytic in $U$ and $\forall k
    \in \N : \ \sum_{n=0}^\infty f_n^{(k)} (z) = f^{k} (z)$.
\end{corollary}

\begin{theorem}
    If the functions $f_n(z)$ are analytic and $\neq 0$ in a region $\Omega$, and
    if $f_n(z)$ converges to $f(z)$, uniformaly on every compact subset of $\Omega$,
    then $f(z)$ is either identically zero or never equal to zero in $\Omega$. 
\end{theorem}

\begin{theorem}[Taylor]
    If $f(z)$ is analytic in the region $\Omega$, containing $z_0$, then the
    representation
    \begin{align*}
        f(z) = f(z_0) + \frac{f'(z_0)}{1!} (z-z_0) + \dots + \frac{f^{(n)}(z_0)}{n!} (z-z_0)^n + \dots
    \end{align*}
    is valid in the largest open disk of center $z_0$ contained in $\Omega$.
\end{theorem}

\begin{definition}[Laurent Series]
    A \fat{Laurent Series} is given by
    \begin{align*}
        \sum_{n=-\infty}^\infty a_n \klammer{z-z_0}^n
    \end{align*}
    for a $z_0 \in \C$.
\end{definition}

\begin{definition}[Radius of Convergence]
    The \fat{radius of convergence} is given by
    \begin{align*}
        \rho_+ &:= \frac{1}{\limsup_{n \rightarrow \infty} \sqrt[n]{\abs{a_n}}}
        \\
        \rho_- &:= \limsup_{n \rightarrow \infty} \sqrt[n]{\abs{a_n}}
    \end{align*}
    Assume $\rho_+ > \rho_-$. Then the Laurent series $\sum_{n \in \Z} a_n (z-z_0)^n$
    converges (locally) uniformaly and absolutely in
    $A = \geschwungeneklammer{z \in \C : \rho_- \abs{z-z_0} < \rho_+}$
\end{definition}

\begin{theorem}
    Let $A$ be as before. Let $f: A \rightarrow \C$ be analytic. Then $f$ can be
    expanded to a Laurent series in $A$ and the series is (locally) uniformaly
    and absolutely convergent. The expansion is unique.
    \begin{align*}
        f(z) = \sum_{n \in \Z} a_n (z-z_0)^n
    \end{align*}
\end{theorem}

\begin{theorem}
    If $f: \Omega \backslash \geschwungeneklammer{z_0} \rightarrow \C$ is
    analytic and has an isolated singularity at $z_0$, then
    \begin{enumerate}[(i)]
        \item $z_0$ is removable $\Leftrightarrow \limes{z \rightarrow z_0} (z-z_0)f(z) = 0$
            $\Leftrightarrow \limes{z \rightarrow z_0} \sum_{k=-\infty}^\infty a_k (z-z_0)^{k+1} = 0$
            $\Leftrightarrow \forall k \leq -1: a_k =0$
            $\Leftrightarrow f(z) = \sum_{k=0}^\infty a_k (z-z_0)^k$ and
            $S(z) = \sum_{k=-\infty}^0 a_k (z-z_0)^k \equiv 0$ is the so called \fat{essential
            part}.
        \item $z_0$ is a pole of order $n$ $\Leftrightarrow \limes{z \rightarrow z_0}
            (z-z_0)^n f(z) \in \C \backslash \geschwungeneklammer{0}$
            $\Leftrightarrow \limes{z \rightarrow z_0} \sum_{k=-\infty}^\infty a_k (z-z_0)^{n+k}
                \in \C \backslash \geschwungeneklammer{0}$
            $\Leftrightarrow \forall k < -n : a_k = 0$
            $\Leftrightarrow f(z) = \sum_{k=-n}^\infty a_k (z-z_0)^k$.
            The essential part consists of only finitely many terms.
        \item $z_0$ is an essential singularity $\Leftrightarrow z_0$ is not
            removable and not a pole $\Leftrightarrow f(z) = \sum_{k=-\infty}^\infty
            a_k (z-z_0)^k$ where $a_k \neq 0 \ \forall k < 0$.
    \end{enumerate}
\end{theorem}

\begin{theorem}
    Let $f$ be analytic in $\geschwungeneklammer{z : R_1 < \abs{z} < R_2}$ can be
    written as $f(z) = \sum_{n=0}^\infty A_n (z-z_0)^n$
\end{theorem}



\pagebreak

\section{Useful Tools}

\vspace{1\baselineskip}

\subsection{Infinite Product}

\begin{definition}[Infinite Product]
    The infinite product is given as
    \begin{align*}
        \prod_{n=0}^\infty P_n = P_1 \cdot \dots \cdot P_n \cdot \dots
    \end{align*}
\end{definition}

\begin{theorem}
    We say that $\prod_{n=0}^\infty P_n$ converges if
    \begin{enumerate}[(i)]
        \item $P_n = \prod_{k=1}^n P_k$, $P_n \rightarrow P \neq 0$
        \item There are finitely many zero factors.
    \end{enumerate}
\end{theorem}

\begin{theorem}
    With the necessary condition $a_n \rightarrow 0$
    we can write $\prod_{n=0}^\infty P_n$ as
    \begin{align*}
        \prod_{n=1}^\infty (1+a_n)
    \end{align*}
\end{theorem}

\begin{theorem}
    If no factor is zero, we can understand $\prod_{n=0}^\infty P_n$ as follows.
    \begin{align*}
        S_M &= \sum_{n=1}^M \log \klammer{1+a_n}
        \\
        P_M &= \prod_{n=1}^M (1+a_n)
        \\
        P_M &= \exp(S_M)
    \end{align*}
    If $S_M$ converges, so does $P_M$. In particular:
\end{theorem}

\begin{theorem}
    The finite product $\prod_{n=1}^\infty (1+a_n)$ with $1+a_n \neq 0$ converges
    if and only if $\sum_n \log (1+a_n)$ converges.
\end{theorem}

\begin{theorem}
    $\sum_n \log(1+a_n)$ converges absolutely if and only if $\sum_n a_n$ converges
    absolutely.
\end{theorem}



\end{document}