\section{Electromagnetics in a medium}

We identify two types of motion of charged particles: fast currents within small
distances, due to the motion of charges in atoms and molecules, and slow currents,
which extend at distances much larger than the size of atoms due to free electrons
or ions. In macroscopic measurements we are only interested in slow variations at
the atomic level. We average over the currents within the atoms. We can write:
\begin{align*}
    j^\mu  \approx j_{slow}^\mu + \langle j_{atomic}^\mu \rangle
\end{align*}
We introduce an averaging of a function over some distances via
\begin{align*}
    \langle F(\vec{x},t) \rangle = \int d^3 \vec{y} f(\vec{y}) F(\vec{x}-\vec{y},t)
\end{align*}
where $f(\vec{y})$ is the weighting factor / probability density function. It
is well behaved, smooth, positive, peaks at $0$ and has Norm $1$.

\subsection{Average of the atomic charge density}

The charge density $j_{atomic}^0 = \rho_{atomic}$ corresponds to the charge
density of the charges inside molecules/atoms. We write
\begin{align*}
    \rho_{atomic} = \sum_{n \in \text{ atoms}} \rho_{(n)}
    \hspace{10pt} , \hspace{10pt}
    \rho_{(n)} = \sum_{j \in (n)} q_j \delta(\vec{x} - \vec{x}_n - \vec{x}_j)
\end{align*}
where $\rho_{(n)}$ is the charge density of the $n$-th atom/molecule,
$\vec{x}_n$ is the position of the atom/molecule and $\vec{x}_j$ is the
position of the charge with respect to the "centre" of the molecule.
\begin{align*}
    \langle \rho_{(n)} \rangle &= \sum_{j \in (n)} q_j f(\vec{x} - \vec{x}_n - \vec{x}_j)
    \\
    &= \sum_{j \in (n)} \eckigeklammer{q_j f(\vec{x} - \vec{x}_n) - (q_j \vec{x}_j) \cdot \vec{\nabla}f(\vec{x}-\vec{x}_n) + \dots}
    \\
    &= q_n f(\vec{x} - \vec{x}_n) - \vec{p}_n \cdot \vec{\nabla} f(\vec{x}-\vec{x}_n) + \dots
    \\
    &= q_n f(\vec{x}-\vec{x}_n) - \vec{\nabla} \klammer{\vec{p}_n \cdot f(\vec{x} - \vec{x}_n)} + \dots
\end{align*}
Where we taylor expanded in $\frac{\abs{\vec{x}_j}}{\abs{\vec{x}-\vec{x}_n}}$ and
used that $\vec{\nabla}$ only differentiates with respect to $\vec{x}$. Summing
over all atoms/molecules we can write:
\begin{align*}
    \langle \rho_{atomic} \rangle &\equiv \sum_n \langle \rho_{(n)} \rangle
    = \langle \rho_{eff/atom} \rangle - \vec{\nabla} \cdot \vec{P} + \dots
    \hspace{10pt} \text{with}
    \\
    \langle \rho_{eff/atom} \rangle
    &= \sum_{n \in \text{ atoms}} q_n f(\vec{x} - \vec{x}_n)
    \hspace{10pt} , \hspace{10pt}
    \vec{P} \equiv \sum_{n \in \text{ atoms}} \vec{p}_n \cdot f(\vec{x}-\vec{x}_n)
\end{align*}

\subsection{Average of atomic current density}

We restrict ourselves to non-relativistic velocities. The current density in an
atom/molecule can be written as
\begin{align*}
    \vec{j}_{(n)} = \sum_{k \in n} q_k (\vec{v}_n + \vec{v}_k) \delta(\vec{x} - \vec{x}_n - \vec{x}_k)
\end{align*}
where $\vec{v}_n$ is the velocity of the atom/molecule and $\vec{v}_k$ is the
relative velocity of the charge to the center of the atom.
\begin{align*}
    \langle \vec{j}_n \rangle = \sum_k q_k \klammer{\vec{v}_n + \vec{v}_k}
    f(\vec{x} - \vec{x}_n - \vec{x}_k)
\end{align*}
For $\abs{\vec{x}_k} \ll \abs{\vec{x}_n}$ and $\abs{\vec{v}_n} \ll \abs{\vec{v}_k}$
we can expand:
\begin{align*}
    f(\vec{x} - \vec{x}_n - \vec{x}_k) \approx
    f(\vec{x} - \vec{x}_n) - \vec{x}_k \cdot \vec{\nabla} f(\vec{x} - \vec{x}_n) + \dots
\end{align*}
This yields:
\begin{align*}
    \langle \vec{j}_n \rangle = &\sum_k q_k \vec{v}_k f(\vec{x} - \vec{x}_n)
    + \sum_k q_k \vec{v}_n f(\vec{x} - \vec{x}_n) - \sum_k q_k \vec{v}_k \vec{x}_k \cdot \vec{\nabla} f(\vec{x} - \vec{x}_n)
    \\
    &+ \mathcal{O} \klammer{x_k^2 , x_k v_n , v_n^2}
\end{align*}
We can rewrite:
\begin{align*}
    \sum_k q_k \vec{v}_k f(\vec{x} - \vec{x}_n)
    &\approx \frac{d}{dt} \klammer{\vec{p}_n f(\vec{x} - \vec{x}_n)}
    \\
    \sum_k q_k \vec{v}_k \vec{x}_k \cdot \vec{\nabla} f(\vec{x} - \vec{x}_n)
    &\approx - \vec{\nabla} \times \klammer{\vec{m}_n f(\vec{x} - \vec{x}_n)}
\end{align*}
With $\vec{m}_n$ the magnetic moment of the atom and $\vec{M}$ the magnetization
of the material
\begin{align*}
    \vec{m}_n = \sum_j \frac{q_j}{2} \klammer{\vec{x}_j \times \vec{v}_j}
    \hspace{10pt} , \hspace{10pt}
    \vec{M} = \sum_n \klammer{\vec{m}_n f(\vec{x} - \vec{x}_n)}
\end{align*}
we can write the sum of the average contribution from all atoms as
\begin{align*}
    \langle \vec{j}_{atomic} \rangle &= \sum \langle \vec{j}_{(n)} \rangle
    = \vec{j}_{eff/atomic} + \frac{d \vec{P}}{dt} + \vec{\nabla} \times \vec{M}
    \\
    \vec{j}_{eff/aotmic} &= \sum_n q_n \vec{v}_n f(\vec{x} - \vec{x}_n)
\end{align*}

\subsection{Maxwell equations in a medium}

We approximate the charge and current density as
\begin{align*}
    j^\mu &= j_{free}^\mu + j_{atomic}^\mu \approx j_{free}^\mu + \langle j_{atomic}^\mu \rangle
    \\
    j^0 &\approx \rho_{eff} - \vec{\nabla} \cdot \vec{P} + \dotsb
    \\
    \vec{j} &\approx \vec{j}_{eff} + \frac{\partial \vec{P}}{\partial t} + \vec{\nabla} \times \vec{M} + \dotsb
\end{align*}
With $\vec{M}$ the magnetisation vector. If we substitute into the Maxwell
equations we obtain
\begin{align*}
    \vec{\nabla} \cdot \vec{B} = 0
    \hspace{10pt} &, \hspace{10pt}
    \vec{\nabla} \times \klammer{\vec{B} - \frac{\vec{M}}{c^2 \epsilonnull}}
    = \frac{\vec{j}_{eff}}{c^2 \epsilonnull} + \frac{1}{c^2} \frac{\partial}{\partial t} \klammer{\vec{E} + \frac{\vec{P}}{\epsilonnull}}
    \\
    \vec{\nabla} \times \vec{E} = - \frac{\partial \vec{B}}{\partial t}
    \hspace{10pt} &, \hspace{10pt}
    \vec{\nabla} \cdot \klammer{\vec{E} + \frac{\vec{P}}{\epsilonnull}} = \frac{\rho_{eff}}{\epsilonnull}
\end{align*}
with $\rho_{eff} = \rho_{free} + \rho_{eff/atomic}$ and
$\vec{j}_{eff} = \vec{j}_{free} + \vec{j}_{eff/atomic}$.

\paragraph{The $\vec{D}$ and $\vec{H}$ field}
We define:
\begin{align*}
    \vec{D} \equiv \epsilonnull \vec{E} + \vec{P}
    \hspace{10pt} , \hspace{10pt}
    \vec{H} \equiv \vec{B} - \frac{\vec{M}}{c^2 \epsilonnull}
\end{align*}
Now the Maxwell equations reduce to:
\begin{align*}
    \vec{\nabla} \cdot \vec{D} = \rho
    \hspace{10pt} &, \hspace{10pt}
    \vec{\nabla} \times \vec{E} = - \frac{\partial \vec{B}}{\partial t}
    \hspace{10pt} , \hspace{10pt}
    \vec{\nabla} \cdot \vec{B} = 0
    \\
    \vec{\nabla} \times \vec{H} &= \frac{\vec{j}_{eff}}{\epsilonnull c^2} + \frac{1}{\epsilonnull c^2} \frac{\partial \vec{D}}{\partial t}
\end{align*}

\subsection{Maxwell equations inside a dielectric material}

Assume $\vec{M} = 0$ and $\vec{P} \neq 0$. We find that $\vec{P}$ and $\vec{E}$
are correlated. If we ignore non-linearities we obtain:
\begin{align*}
    \vec{P} = \chi \epsilonnull \vec{E}
\end{align*}
where $\chi$ is the "electric susceptibility". We define:
\begin{align*}
    c_m = \frac{c}{\sqrt{1 + \chi}}
    \hspace{10pt} , \hspace{10pt}
    \epsilon = (1+\chi) \epsilonnull
\end{align*}
Now we can write the Maxwell equations in a dielectric medium as:
\begin{align*}
    \vec{\nabla} \cdot \vec{E}
    \hspace{7pt} , \hspace{7pt}
    \vec{\nabla} \times \vec{E} = - \frac{\partial \vec{B}}{\partial t}
    \hspace{7pt} , \hspace{7pt}
    \vec{\nabla} \cdot \vec{B} = 0
    \hspace{7pt} , \hspace{7pt}
    \vec{\nabla} \times \vec{B} = \frac{\vec{j}_{eff}}{\epsilon c_m^2} + \frac{1}{c_m^2} \frac{\partial \vec{E}}{\partial t}
\end{align*}

\subsection{A model for the dielectric susceptibility $\chi$}

For a dipole we have the differential equation
\begin{align*}
    q \vec{E} = m \klammer{\ddot{x} + \gamma \dot{x} + \omega_0^2 x}
\end{align*}
A solution on the microscopic level is
\begin{align*}
    \vec{x} = \frac{q \vec{E} / m}{\omega_0^2 - \omega^2 + i \omega \gamma}
\end{align*}
So for the dipole moment we obtain:
\begin{align*}
    \vec{P} &= q \vec{x} = \frac{q^2 / m}{\omega_0^2 - \omega^2 + i \omega \gamma} \vec{E}
    \\
    \langle \vec{P} \rangle &= N \vec{x} = \frac{N q^2 / m}{\omega_0^2 - \omega^2 + i \omega \gamma} \vec{E}
\end{align*}
Where $N$ is the density of charges.
When comparing to $\vec{P} = \chi \epsilonnull \vec{E}$ we obtain:
\begin{align*}
    \chi = \chi(\omega) = \frac{\frac{N q^2}{m \epsilonnull}}{\omega_0^2 - \omega^2 + i \omega \gamma}
    = n^2 - 1
\end{align*}
We see that $\chi$ is complex, which has physical consequences on $\epsilon$ and
$c_m$.

\subsection{Waves in a dielectric medium}

If $\vec{j}_{eff} = \rho_{eff} = 0$ ina medium, then we obtain the wave equation
from the Maxwell equations.
\begin{align*}
    \eckigeklammer{\frac{1}{c_m^2} \frac{\partial^2}{\partial t^2} - \vec{\nabla}^2} \vec{E} = 0
\end{align*}
A solution is
\begin{align*}
    \vec{E} = \vec{E}_0 e^{i\klammer{\omega t - \vec{k} \cdot \vec{x}}}
\end{align*}
with $k := \abs{\vec{k}}$:
\begin{align*}
    k^2 = \frac{\omega^2}{c_m^2} = \frac{\omega^2}{c^2} (1+\chi)
    \hspace{10pt} , \hspace{10pt}
    v_{phase} = \frac{\omega}{k} = \frac{c}{n}
    \hspace{10pt} , \hspace{10pt}
    n = \sqrt{1+\chi}
\end{align*}
We see that $n$, the refraction index, is complex.

\subsection{The complex index of refraction}
We can separate the complex and the real part or $n$ into a real and an
imaginary part: $n = n_R - i n_I$. We find that the plane wave propagating
in the dielectric is:
\begin{align*}
    \vec{E} = \vec{E}_0 e^{i \omega \eckigeklammer{t - n \hat{k} \cdot \vec{x}}}
    = \vec{E}_0 e^{i \omega \eckigeklammer{t - \frac{n_R}{c} \hat{k} \cdot \vec{x}}}
    e^{-\omega \frac{n_I}{c} \hat{k} \cdot \vec{x}}
    \hspace{5pt} \Rightarrow \hspace{5pt}
    \abs{\vec{E}} = \abs{\vec{E}_0} e^{- \omega \frac{n_I}{c} \hat{k} \cdot \vec{x}}
\end{align*}

\subsection{Waves in metals}

In metals, electrons move freely at large distances. So we set $\omega_0 = 0$.
\begin{align*}
    \chi(\omega) = \frac{N q^2 / \epsilonnull m}{- \omega^2 + i \omega \gamma}
\end{align*}
The density $N$ can be obtained from macroscopic properties of the metal and
the constant $\gamma$ is an intrinsic parameter of our model. It is related
to the resistanve or its inverse, conductivity. If $\vec{E}$ is constant,
we have a differential equation for $\vec{v}$.
\begin{align*}
    q \vec{E} = m (\dot{\vec{v}} + \gamma \vec{v})
    \hspace{10pt} \Rightarrow \hspace{10pt}
    \vec{v} = \vec{v}_{drift} + \vec{v_0} e^{-\gamma t}
    \hspace{10pt} , \hspace{10pt}
    \vec{v}_{drift} = \frac{q \vec{E}}{m \gamma}
\end{align*}
We further have the relation
\begin{align*}
    \vec{J} = \sigma \vec{E} = N q \vec{v}_{drift} = \frac{N q^2}{m \gamma} \vec{E}
    \hspace{10pt} \Rightarrow \hspace{10pt}
    \gamma = \frac{N q^2}{m \sigma} 
\end{align*}

\subsubsection{Low frequency approximation}

$\omega \rightarrow 0$, so $\omega \gamma \gg \omega^2$. Thus
$n^2 = -i \frac{\sigma}{\epsilonnull \omega} \ \Rightarrow \
\sqrt{\frac{\sigma}{2 \epsilonnull \omega}} (1-i)$ and
$\abs{\vec{E}} = \abs{\vec{E}_0} e^{-\frac{x}{\delta}} \ $ where
$\ \delta = \sqrt{\frac{2 \epsilonnull}{\sigma \omega}} c$.

\subsubsection{High frequency approximation}

$\omega^2 \gg \omega \gamma \ \Rightarrow \ n^2 \approx 1 - \frac{\omega_P^2}{\omega^2} \ $
where $\ \omega_P^2 = \frac{N q^2}{m \epsilonnull}$ is the so called "plasma
frequency". If $\omega < \omega_P$, then $n \in i \R$ and therefore the waves
die off after some length. If $\omega > \omega_P$, then $n \in \R$ and thus the
metal becomes transparent to the electromagnetic wave.

\subsection{Reflection and refraction}

Consider two materials with refraction indices $n_1$ and $n_2$ separated by a
boundary surface on the $yz$ plane. For the electromagnetic fields we have:
% We want to analyse Maxwells equations on the
% boundary to gain some insight on the fields.
% \begin{align*}
%     \partial_x E_x + \partial_y F_y + \partial_z E_z
%     = \vec{\nabla} \cdot \vec{E}
%     = - \frac{1}{\epsilonnull} \vec{\nabla} \cdot \vec{P}
%     = - \frac{1}{\epsilonnull} \klammer{\partial_x P_x + \partial_y P_y + \partial_z P_z}
% \end{align*}
% The derivatives in $x$-direction are much bigger than the other ones, thus
% we obtain:
% \begin{align*}
%     \partial_x E_x = - \frac{1}{\epsilonnull} \partial_x P_x
%     \ &\Rightarrow \
%     \frac{E_{x,2} - E_{x,1}}{\Delta x} = - \frac{1}{\epsilonnull} \frac{P_{x,2} - P_{x,1}}{\Delta x}
%     \\ &\Rightarrow \
%     \epsilonnull E_{x,2} + P_{x,2} = \epsilonnull E_{x,1} + P_{x,1}
% \end{align*}
% Further, with again using that the derivatives with respect to $x$ is much
% larger than the other ones we obtain:
% \begin{align*}
%     \vec{\nabla} \times \vec{E} = - \frac{\partial \vec{B}}{\partial t}
%     \ \Rightarrow \
%     \frac{\partial E_x}{\partial z} - \frac{\partial E_z}{\partial x} = - \frac{\partial B_y}{\partial t}
%     \ \rightsquigarrow \
%     \frac{\partial E_z}{\partial x} = 0
%     \ \rightarrow \
%     E_{z,1} = E_{z,2}
% \end{align*}
% Similarly we obtain $E_{y,1} = E_{y,2}$ and from the other Maxwell eq we obtain
% $\vec{B}_2 = \vec{B_1}$ so in conclusion:
\begin{align*}
    \vec{B}_1 = \vec{B}_2
    \hspace{10pt} , \hspace{10pt}
    \vec{E}_{1,\parallel} = \vec{E}_{2,\parallel}
    \hspace{10pt} , \hspace{10pt}
    \klammer{\epsilonnull \vec{E}_1 + \vec{P}_1}_\perp = \klammer{\epsilonnull \vec{E}_2 + \vec{P}_2}_{\perp}
\end{align*}

\subsubsection{Snell's law}

Consider an incident electromagnetic plane-wave $(\vec{E}_I,\vec{B}_I)$ from
a medium $n_1$ to a medium $n_2$, the reflected $(\vec{E}_R,\vec{B}_R)$ and transmitted
electromagnetic field $(\vec{E}_T,\vec{B}_T)$. The following are true:
\begin{align*}
    \vec{E}_1 &= \vec{E}_I + \vec{E}_R
    \hspace{10pt} , \hspace{10pt}
    \vec{E}_2 = \vec{E}_T
    \hspace{10pt} , \hspace{10pt}
    \vec{B}_1 = \vec{B}_I + \vec{B}_R
    \hspace{10pt} , \hspace{10pt}
    \vec{B}_2 = \vec{B}_T
    \\
    \vec{E}_I &= \hat{e}_I E_I e^{i(\omega_I t - \vec{k}_I \cdot \vec{x})}
    \hspace{5pt} , \hspace{5pt}
    \vec{E}_R = \hat{e}_R E_R e^{i(\omega_R t - \vec{k}_R \cdot \vec{x})}
    \\
    \vec{E}_T &= \hat{e}_T E_T e^{i(\omega_T t - \vec{k}_T \cdot \vec{x})}
    \\
    \vec{B}_I &= \frac{\vec{k}_I \times \vec{E}_I}{\omega_I}
    \hspace{10pt} , \hspace{10pt}
    \vec{B}_R = \frac{\vec{k}_R \times \vec{E}_R}{\omega_R}
    \hspace{10pt} , \hspace{10pt}
    \vec{B}_T = \frac{\vec{k}_T \times \vec{E}_T}{\omega_T}
    \\
    \vec{k}_I \cdot \hat{e}_I &= \vec{k}_R \cdot \hat{e}_R
    = \vec{k}_T \cdot \hat{e}_T = 0
    \\
    \frac{k_I}{\omega_I} &= \frac{k_R}{\omega_R} = \frac{n_1}{c}
    \hspace{10pt} , \hspace{10pt}
    \frac{k_T}{\omega_T} = \frac{n_2}{c}
\end{align*}
With some boundary conditions we obtain:
\begin{align*}
    \omega_I = \omega_R = \omega_T = \omega
    \hspace{10pt} \text{and} \hspace{10pt}
    \vec{k}_{I,\parallel} = \vec{k}_{R,\parallel} = \vec{k}_{T,\parallel}
    = \vec{k}_{\parallel}
\end{align*}
We also obtain Snell's law
\begin{align*}
    \theta_I = \theta_R
    \hspace{10pt} , \hspace{10pt}
    \sin(\theta_T) = \frac{n_1}{n_2} \sin(\theta_I)
\end{align*}


