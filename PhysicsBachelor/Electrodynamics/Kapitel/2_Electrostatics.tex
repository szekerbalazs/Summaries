\section{Electrostatics}

Fixed charge distribution and static electric field.
\begin{align*}
    \vec{\nabla} \cdot \vec{E} = \frac{\rho}{\epsilonnull}
    \hspace{10pt} , \hspace{10pt}
    \vec{\nabla} \times \vec{E} = 0
\end{align*}

\subsection{Coulomb's Law}

\begin{align*}
    \vec{F} = \frac{q}{4 \pi \epsilonnull} \frac{q_1}{\abs{\vec{x} - \vec{y}_1}^3} (\vec{x} - \vec{y}_1)
    \hspace{10pt} &, \hspace{10pt}
    \vec{E} = \frac{1}{4 \pi \epsilonnull} \frac{q_1}{\abs{\vec{x} - \vec{y}_1}^3} (\vec{x} - \vec{y}_1)
    \\
    \vec{E} = \frac{1}{4 \pi \epsilonnull} \sum_{i=1}^N \frac{q_i}{\abs{\vec{x} - \vec{y}_i}^3} (\vec{x} - \vec{y}_i)
    \hspace{10pt} &, \hspace{10pt}
    \vec{E} = \frac{1}{4 \pi \epsilonnull} \int \frac{\rho(\vec{y})}{\abs{\vec{x} - \vec{y}}^3} (\vec{x} - \vec{y}) \ d^3 \vec{y}
\end{align*}

\subsubsection{Dirac Delta}

\begin{align*}
    \intii f(x) \delta(x-y) \ dx = f(y)
    \hspace{5pt} , \hspace{5pt}
    \delta (\vec{x} - \vec{y}) =
    \delta^{(D)} (\vec{x} - \vec{y}) = \prod_{i=1}^D \delta(x_i - y_i)
\end{align*}

\paragraph{Charge density} of a single charge $q$ at a position $\vec{y}$:
$\rho(\vec{x}) = q \delta(\vec{x} - \vec{y})$.

\paragraph{Charge distribution} of many point-like charges $q_i$ at positions $\vec{y}_i$:
$\rho(\vec{x}) = \sum_i q_i \delta(\vec{x} - \vec{y}_i)$

\subsection{Gauss' law from Coulomb's law}
\begin{align*}
    \vec{\nabla}_{\vec{x}} \frac{1}{\abs{\vec{x} - \vec{y}}} &= - \frac{(\vec{x} - \vec{y})}{\abs{\vec{x} - \vec{y}}^3}
    \ \Rightarrow \ \vec{E} = - \vec{\nabla} \Phi(\vec{x})
    \\
    &\Rightarrow \
    \Phi(\vec{x}) = \frac{1}{4 \pi \epsilonnull} \int \frac{\rho(\vec{y})}{\abs{\vec{x} - \vec{y}}} \ d^3 \vec{y}
    \\
    &\Rightarrow \ \vec{\nabla} \cdot \vec{E} = - \vec{\nabla}^2 \Phi = - \Delta \Phi
    = \frac{\rho}{\epsilonnull}
\end{align*}
Further: $\mathbf{\nabla}^2_{\vec{x}} \frac{1}{\abs{\vec{x} - \vec{y}}}
= - 4 \pi \delta(\vec{x} - \vec{y})$

\subsubsection{Integral form of Gauss' law}

\begin{align*}
    \int_{S(V)} \vec{E} \cdot d \vec{S}
    = \int_{V(S)} \vec{\nabla} \cdot \vec{E} \ d^3 \vec{x}
    = \frac{1}{\epsilonnull} \int_{V(S)} \rho(\vec{x}) \ d^3 \vec{x}
\end{align*}

\subsection{Scalar potential}

\begin{align*}
    \vec{\nabla} \times \vec{E} &= - \vec{\nabla} \times \vec{\nabla} \Phi = 0
    \\
    \int_S \vec{\nabla} \times \vec{E} \cdot d \vec{S}
    &= \oint_{\partial S} \vec{E} \cdot d \vec{l} = 0
\end{align*}

Field exerts a force $\vec{F} = q \vec{E} = - q \vec{\nabla} \Phi$.
Work needed to transport a test charge from a position $\vec{x}_A$ to
a position $\vec{x}_B$:
\begin{align*}
    W_{A \rightarrow B} = q \int_{\vec{x}_A}^{\vec{x}_B} \vec{\nabla} \Phi \cdot d \vec{l}
    = q \klammer{\Phi(\vec{x}_B) - \Phi(\vec{x}_A)}
\end{align*}
Work done is independent of the chosen path.

\subsection{Potential energy of a charge distribution}

Potential energy of a system is the energy required to bring the charges to
their positions from infinity. Discrete case:
\begin{align*}
    W = \frac{1}{4 \pi \epsilon_0} \sum_{i=2}^{N} \sum_{j<i} \frac{q_i q_j}{\abs{\vec{x}_i - \vec{x}_j}}
    = \frac{1}{8 \pi \epsilon_0} \sum_{i \neq j} \frac{q_i q_j}{\abs{\vec{x}_i - \vec{x}_j}}
\end{align*}
Continous case:
\begin{align*}
    W &= \frac{1}{8 \pi \epsilon_0} \int \frac{\rho(\vec{x}) \rho(\vec{y})}{\abs{\vec{x} - \vec{y}}} \ d^3 \vec{x} d^2 \vec{y}
    = \frac{1}{2} \int \rho(\vec{x}) \Phi(\vec{x}) \ d^3 \vec{x}
    \\
    &= - \frac{\epsilon_0}{2} \int \Phi(\vec{x}) \vec{\nabla}^2 \Phi(\vec{x}) \ d^3 \vec{x}
    = \frac{\epsilon_0}{2} \int \abs{\vec{E}}^2 \ d^3 \vec{x}
\end{align*}

\paragraph{Energy density of the electric field}
$w = \frac{\epsilon_0}{2} \abs{\vec{E}}^2$


\subsubsection{Self-energy}
Infinities in the total energy occur if we consider discrete distributions with
a density $\rho(\vec{x}) = \sum_i q_i \delta(\vec{x} - \vec{x}_i)$

\subsection{Charged conductors}

Electrons move freely within their mass. $\vec{E}_{\text{inside}} = 0
\Rightarrow \Phi$ const. Flux of electric field through Gauss surface:
\begin{align*}
    \text{Flux} = \int \vec{E} \cdot d \vec{S} = E \Delta S
\end{align*}
Charge enclosed:
\begin{align*}
    \text{Charge} = \sigma(\vec{x}) \Delta S
\end{align*}
with $\sigma(\vec{x})$ the charge surface density. Electric field on the
surface:
\begin{align*}
    E = \frac{\sigma(\vec{x})}{\epsilon_0}
\end{align*}
Energy density on the surface of the conductor:
\begin{align*}
    w = \frac{\epsilon_0}{2} \abs{\vec{E}}^2
    = \frac{\sigma^2(\vec{x})}{2 \epsilon_0}
\end{align*}
Small deformation $\Delta x \Delta S$ leads to a difference in electrostatic
energy:
\begin{align*}
    \Delta W = \int_{V_{\text{out}}} w \ d^3 \vec{x} - \int_{V_{\text{out}} - \Delta x \Delta S} w \ d^3 \vec{x}
    = - \Delta S \Delta x \frac{\sigma^2}{2 \epsilon_0}
\end{align*}
Force needed to undo deformation: $\Delta W = F \Delta x$. Pressure on the
surface:
\begin{align*}
    \frac{\abs{F}}{\Delta S} = \frac{\abs{\Delta W}}{\Delta x \Delta S}
    = \frac{\sigma^2}{2 \epsilonnull}
\end{align*}
