\section{Magnetic Field}

\subsection{Currents}

Define current density $\vec{j}$ as the charge transversing a surface $d \vec{S}$
per unit time and surface:
\begin{align*}
    q N_{\text{escaping}} = \vec{J} \cdot d \vec{S} dt
    \hspace{10pt} \Rightarrow \hspace{10pt}
    \vec{J} = \rho \vec{v}
\end{align*}
\fat{Continuity equation}:
\begin{align*}
    \frac{\partial \rho}{\partial t} + \vec{\nabla} \cdot \vec{J} = 0
\end{align*}

\subsection{Magnetic field of steady currents}

Assume charge density to be constant: $\frac{\partial \rho}{\partial t} = 0$.
Therefore: $\vec{\nabla} \cdot \vec{J} = 0$. Further, the electric and
magnetic field are constant in time:
\begin{align*}
    \vec{\nabla} \cdot \vec{B} = 0
    \hspace{10pt} &, \hspace{10pt}
    c^2 \vec{\nabla} \times \vec{B} = \frac{\vec{J}}{\epsilonnull}
    \\
    \Rightarrow
    \int_{S(V)} \vec{B} \cdot d \vec{S} = 0
    \hspace{10pt} &, \hspace{10pt}
    c^2 \oint_{\Gamma} \vec{B} \cdot d \vec{l} = \frac{\int_{S(\Gamma)} \vec{J} \cdot d \vec{S}}{\epsilonnull}
\end{align*}
We define $I_{\text{through } \Gamma} \equiv \int_{S(\Gamma)} \vec{J} \cdot d \vec{S}$
the total charge passing through the closed loop $\Gamma$ per unit time.
We arrive at \fat{Ampere's Law}:
\begin{align*}
    \oint_{\Gamma} \vec{B} \cdot d \vec{l} = \frac{I_{\text{through } \Gamma}}{\epsilonnull c^2}
\end{align*}

\subsection{Vector potential}

$\vec{\nabla} \cdot \vec{B} = 0$ follows automatically if
$\vec{B} = \vec{\nabla} \times \vec{A}$. From $\vec{\nabla} \times \vec{B}
= \frac{\vec{J}}{\epsilonnull c^2}$ we derive:
\begin{align*}
    \vec{\nabla}^2 \vec{A} - \vec{\nabla} \klammer{\vec{\nabla} \cdot \vec{A}}
    = - \frac{\vec{J}}{c^2 \epsilonnull}
\end{align*}

\subsubsection{Gauge Invariance}

If $\vec{\nabla} \cdot \klammer{\vec{\nabla} \cdot \vec{A}} = 0$ we would
have a poisson equation. We can choose to eliminate this term by gauge
invariance. Consider $\vec{A}$ and $\vec{A'}$ and a scalar functioin $f$.
\begin{align*}
    \vec{A'} = \vec{A} + \vec{\nabla} f
\end{align*}
Easy to see: $\vec{\nabla} \times \vec{A'} = \vec{\nabla} \times \vec{A} = \vec{B}$.
Assume $\vec{\nabla} \cdot \vec{A'} \neq 0$, then:
\begin{align*}
    \vec{\nabla}^2 f = - \vec{\nabla} \cdot \vec{A'}
    \hspace{10pt} \Rightarrow \hspace{10pt}
    f(\vec{r}) = \frac{1}{4 \pi} \int d^3 \vec{x} \frac{\klammer{\vec{\nabla} \cdot \vec{A'}}(\vec{x})}{\abs{\vec{r} - \vec{x}}}
\end{align*}
Then: $\vec{\nabla} \cdot \vec{A} = \vec{\nabla} \cdot \klammer{\vec{A'} + \vec{\nabla} f}
= 0$. This is called \fat{Coulomb gauge}. In that gauge:
\begin{align*}
    \vec{\nabla}^2 \vec{A} = - \frac{\vec{J}}{c^2 \epsilonnull}
    \hspace{10pt} \Rightarrow \hspace{10pt}
    \vec{A}(\vec{x}) = \frac{1}{4 \pi \epsilonnull c^2} \int d^3 \vec{y} \frac{\vec{J}(\vec{y})}{\abs{\vec{x} - \vec{y}}}
\end{align*}

\subsection{Magnetic Dipole}

Consider a steady current $\vec{J}$. We want to compute the vector potential
and the magnetic field in a position $\vec{r}$ which is far from the
current. First we approximate:
\begin{align*}
    \frac{1}{\abs{\vec{r} - \vec{x}}} = \frac{1}{r} + \frac{\vec{x} \cdot \vec{r}}{r^3} + \mathcal{O} \klammer{\frac{x^2}{r^3}}
\end{align*}
First term gives a zero contribution to the vector potential because
$\int d^3 \vec{x} \vec{J} = 0$. Contribution of second term:
\begin{align*}
    \vec{A} &\approx \frac{1}{8 \pi \epsilonnull c^2 r^3} \int d^3 \vec{x}
        \eckigeklammer{\vec{J} \klammer{\vec{x} \cdot \vec{r}} - \klammer{\vec{J} \cdot \vec{r}} \vec{x}}
    \\
    \vec{A} &= \frac{1}{4 \pi \epsilonnull c^2} \frac{\vec{m} \times \vec{r}}{r^3}
\end{align*}
with $\vec{m} = \vec{\mu}$ the magnetic moment
\begin{align*}
    \vec{\mu} = \frac{1}{2} \int d^3 \vec{x} \klammer{\vec{x} \times \vec{J}}
\end{align*}
Special case: steady current circulating anti-clockwise in a wire which lays
on a plane:
\begin{align*}
    \vec{\mu} = \oint \frac{I}{2} \vec{x} \times d \vec{l}
\end{align*}
Notice: $d \vec{S} = \frac{1}{2} \vec{x} \times d \vec{l} \ \Rightarrow \
\vec{m} = I \vec{S}$ with $\vec{S} = \int d \vec{S}$ the total area of the loop.

\vspace{1\baselineskip}

Special case: current created by a single charge $q$ and mass $M$ moving
along a closed loop. We have:
\begin{align*}
    \vec{J} = \rho \vec{v} = q \delta \klammer{\vec{x} - \vec{r}(t)} \vec{v}
    \hspace{10pt} , \hspace{10pt}
    \vec{m} = \frac{1}{2} \int d^2 \vec{x} \ \vec{x} \times \vec{J}
        = \frac{q}{2} \vec{r} \times \vec{v}
        = \frac{2}{2 M} \vec{L}
\end{align*}
with $\vec{L} = \vec{r} \times \klammer{M \vec{v}}$ the angular momentum
of the charge. Magnetic field of a magnetic dipole:
\begin{align*}
    \vec{B}
    &= \vec{\nabla} \times \vec{A}
    = \vec{\nabla} \times \frac{\vec{m} \times \vec{r}}{r^3}
    = - \vec{\nabla} \times \eckigeklammer{\vec{m} \times \vec{\nabla} \frac{1}{r}}
    = \klammer{\vec{m} \cdot \vec{\nabla}} \vec{\nabla} \frac{1}{r}
    \\
    &= - \frac{\vec{m} - 3 \hat{r} \klammer{\hat{r} \cdot \vec{m}}}{r^3}
\end{align*}

\subsection{Relativity of the electric and magnetic field}

Consider a wire on the $x$-axis producing a magnetic field
$B = \frac{1}{2 \pi \epsilonnull c^2} \frac{I}{r}$ at a distance $r$.
Assume induced electrons have inside have a velocity $\vec{v}$.
Imagine another electron outside the wire to move parallel to the wire
at a distance $r$ from it with the same velocity $\vec{v}$. Force
acting on electron outside:
\begin{align*}
    F = q v B = \frac{1}{2 \pi \epsilonnull} \frac{q v I}{c^2 r}
    = \frac{q S}{2 \pi \epsilonnull} \frac{\rho_-}{r} \frac{v^2}{c^2}
\end{align*}
where $I = S \rho_- v$ with $S$ the cross-section area of the wire
and $\rho_-$ the electron charge density inside.

\vspace{1\baselineskip}

Now choose reference frame that moves with a relative velocity
$\vec{v}$ along the wire. Here, electron outside is static ($v' = 0$).
Here, magnetic component of the force must vanish. In original frame:
$\rho = \rho_+ + \rho_- = 0$ and in the moving frame $\rho' = \rho_+'
+ \rho_-' \neq 0$.
\begin{align*}
    \rho_{\text{moving}} = \frac{\rho_{\text{rest}}}{\sqrt{1 - \frac{v^2}{c^2}}}
    \hspace{10pt} \Rightarrow \hspace{10pt}
    \rho_+' = \frac{\rho_+}{\sqrt{1 - \frac{v^2}{c^2}}}
    \hspace{5pt} , \hspace{5pt}
    \rho_-' = \rho_- \sqrt{1 - \frac{v^2}{c^2}}
\end{align*}
Given that $\rho_+ = - \rho_-$ we have:
\begin{align*}
    \rho' = - \rho_- \frac{\frac{v^2}{c^2}}{\sqrt{1 - \frac{v^2}{c^2}}}
\end{align*}
The new electric field and force are:
\begin{align*}
    E' = \frac{1}{2 \pi \epsilonnull} S \frac{\rho'}{r}
    \hspace{10pt} , \hspace{10pt}
    F' = \frac{F}{\sqrt{1 - \frac{v^2}{c^2}}}
\end{align*}