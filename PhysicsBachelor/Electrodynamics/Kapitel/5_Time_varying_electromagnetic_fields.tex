\section{Time varying electromagnetic fields}

\subsection{Charge conservation}

When integrating the continuity equation over all of space, we obtain that
the total charge in the universe has to be constant:
$\frac{\partial Q_{\text{universe}}}{\partial t} = 0 \ \Rightarrow \
Q_{\text{universe}} = \text{const}$.

\subsection{Vector and scalar potential}

\begin{align*}
    \vec{\nabla} \times \vec{E} = - \frac{\partial \vec{B}}{\partial t}
    \rightsquigarrow \vec{\nabla} \times \klammer{\vec{E} + \frac{\partial \vec{A}}{\partial t}} = 0
\end{align*}
This is satisfied if we introduce a scalar potential $\phi$ such that
\begin{align*}
    \vec{E} = - \frac{\partial \vec{A}}{\partial t} - \vec{\nabla} \phi
\end{align*}
We define $\Box \equiv \frac{1}{c^2} \frac{\partial^2}{\partial t^2} - \vec{\nabla}^2$.
Then we have:
\begin{align*}
    \Box \vec{A} + \vec{\nabla} \eckigeklammer{\vec{\nabla} \cdot \vec{A} +
        \frac{1}{c^2} \frac{\partial \phi}{\partial t}} &= \frac{\vec{J}}{c^2 \epsilonnull}
    \\
    \vec{\nabla}^2 \phi - \frac{\partial}{\partial t} \vec{\nabla} \cdot \vec{A} &= \frac{\rho}{\epsilonnull}
    \\
    \Box \phi - \frac{\partial}{\partial t} \eckigeklammer{\vec{\nabla} \cdot \vec{A} +
    \frac{1}{c^2} \frac{\partial \phi}{\partial t}} &= \frac{\rho}{\epsilonnull}
\end{align*}

\subsection{Gauge Invariance}

Electric and magnetic field remain invariant under the following transformation
of the scalar and vector potential:
\begin{align*}
    \phi \rightarrow \phi' = \phi - \frac{\partial f}{\partial t}
    \hspace{20pt} , \hspace{20pt}
    \vec{A} \rightarrow \vec{A}' = \vec{A}' + \vec{\nabla} f
\end{align*}
with $f = f(\vec{x},t)$. We can not choose $f$ such that the potentials
$\vec{A}_L$, $\phi_L$ satisfy
\begin{align*}
    \vec{\nabla} \cdot \vec{A}_L + \frac{1}{c^2} \frac{\partial \phi_L}{\partial t} = 0
\end{align*}
This is called the "Lorentz gauge". Now we have:
\begin{align*}
    \Box \phi_L = \frac{\rho}{\epsilonnull}
    \hspace{20pt} , \hspace{20pt}
    \Box \vec{A}_L = \frac{\vec{J}}{c^2 \epsilonnull}
\end{align*}

\subsection{Electromagnetic waves in empty space}

In empty space $\vec{J} = 0$ and $\rho = 0$. In empty space, all fields
satisfy the same equation:
\begin{align*}
    \Box f = 0
    \hspace{10pt} , \hspace{10pt}
    f \in \geschwungeneklammer{\phi,\vec{A},\vec{E},\vec{B}}
\end{align*}
A solution to this equation is
\begin{align*}
    f(\vec{x},t) = f(\hat{\eta} \cdot \vec{x} - c t)
\end{align*}
with $\hat{\eta}^2 = 1$. In this solution, $f$ depends on a single
combination $u = \hat{\eta} \cdot \vec{x} - c t$. The solution $f$ is
a travelling wave with a speed equal to the speed of light $c$, moving
along the direction of the unit vector $\hat{\eta}$. We can now write:
\begin{align*}
    \vec{E} = \hat{e} E (\hat{\eta} \cdot \vec{x} - c t)
    \hspace{10pt} , \hspace{10pt}
    \vec{B} = \hat{b} B (\hat{\eta} \cdot \vec{x} - c t)
\end{align*}
We find: $\vec{E},\vec{B} \perp \hat{\eta}$ and $\vec{E} \perp \vec{B}$.
Further: $\abs{\vec{E}} = c \cdot \abs{\vec{B}}$.

\subsubsection{Spherical waves}

If we are able to change the charge and current density at one point in
the entire space only, we will generate an electromagnetic wave with
spherical symmetry, i.e. no preferred direction. We still have:
\begin{align*}
    \Box f = 0
    \hspace{10pt} , \hspace{10pt}
    \forall f \in \geschwungeneklammer{\vec{A},\vec{B},\vec{E}}
\end{align*}
Now with spherical symmetry we have: $f(\vec{r},t) = f(r,t)$. So we have:
\begin{align*}
    &\Box f(r,t) =
    \frac{1}{c^2} \frac{\partial^2 f}{\partial t^2} - \frac{1}{r} \frac{\partial^2 (r f)}{\partial r^2}
    = 0
    \\
    &\Rightarrow
    \eckigeklammer{\frac{1}{c^2} \frac{\partial^2}{\partial t^2} - \frac{\partial^2}{\partial r^2}} (r f) = 0
    \\
    &\Rightarrow
    r f = A(r - c t) + B(r + c t)
    \\
    &\Rightarrow
    f(r,t) = \frac{A(r - c t)}{r} + \frac{B(r + c t)}{r}
\end{align*}
The first term corresponds to a spherical wave propagating outwards and
the second term is a spherical wave that propagates inwards. Typically
only the outwards propagaion is realistic.

\subsection{Moving charges in a homogeneous magnetic field}

Consider a $\vec{B}$-field with $\vec{j}_{source}$ and far away at distance
$\vec{r}_0$ a cloud of charges $\Omega$ with $\vec{j} (x) \neq 0$. The
force of the $\vec{B}$-field acting on $\Omega$ is
\begin{align*}
    \vec{F} = (\vec{m} \cdot \vec{\nabla}) \vec{B} \big|_{x=0}
    \hspace{10pt} , \hspace{10pt}
    \vec{m} = \frac{1}{2} \int_\Omega d^3 \vec{x} \ \vec{x} \times \vec{j}
\end{align*}
with $\vec{m}$ the magnetic dipole moment. The potential energy is then:
\begin{align*}
    U = - \vec{m} \cdot \vec{B}
\end{align*}
