\section{General solution of Maxwell equations with sources}

We want to find the solution to Maxwell equations in the presence of sources
($\vec{J},\rho \neq 0$). In the Lorentz gauge we have:
\begin{align*}
    \Box \phi = \frac{\rho}{\epsilonnull}
    \hspace{10pt} , \hspace{10pt}
    \Box \vec{A} = \frac{\vec{J}}{c^2 \epsilonnull}
\end{align*}

Suppose we know a solution $\phi_{sol}$ such that $\Box \phi_{sol} =
\frac{\rho}{\epsilonnull}$. Then $\phi' = \phi_{free} + \phi_{sol}$ with
$\Box \phi_{free} = 0$ is also a solution. 

\subsection{Green's functions}

To solve above differential equations we seek functions $G(\vec{x},\vec{x}',t,t')$
which satisfy
\begin{align*}
    \Box_{\vec{x},t} G(\vec{x},\vec{x}',t,t') = \delta^{(3)}(\vec{x}-\vec{x}') \delta(t-t')
\end{align*}
Then a solution for the scalar potential is:
\begin{align*}
    \phi(\vec{x},t) = \phi_{free}(\vec{r},t) +
        \intii d^3 \vec{x}' d t' \ G(\vec{x},\vec{x}',t,t') \frac{\rho(\vec{x}',t')}{\epsilonnull}
\end{align*}
This violates causality because we need to know future states at times $t' > t$
to determine the one at time $t$. Therefore, we say
\begin{align*}
    \Box = \limes{\delta \rightarrow 0} \Box_\delta
    = \limes{\delta \rightarrow 0} \klammer{\frac{1}{c} \frac{\partial}{\partial t} + \delta}^2 - \vec{\nabla}^2
\end{align*}

\subsection{Fourier transformation}

\begin{align*}
    \tilde{f}(k) = \intii \frac{dt'}{\sqrt{2 \pi}} e^{i k t'} f(t')
    \hspace{10pt} &, \hspace{10pt}
    f(t) = \intii \frac{dk}{\sqrt{2 \pi}} e^{-i k t} \tilde{f}(k)
    \\
    \frac{1}{2 \pi} \intii d \omega \ e^{-i \omega t} = \delta(t)
    = \partial_t \Theta(t) &= \partial_t \klammer{- \frac{1}{2 \pi i}
    \limes{\epsilon \rightarrow 0^+} \intii d \omega \frac{e^{-i \omega t}}{\omega + i \epsilon}}
\end{align*}

\subsection{Fourier transformation and Green's functions}

We want Green's functions which are only dependent on space-time differences
\begin{align*}
    G(\vec{x},\vec{x}',t,t') &= G(\vec{x} - \vec{x}' , t - t')
    = G(\Delta \vec{x} , \delta t)
    \\
    \Box G(\Delta \vec{x},\Delta t) &= \delta(\Delta \vec{x}) \delta(\Delta t)
\end{align*}
If we calculate $\Box_\delta G(\Delta \vec{x},\Delta t)$ and compare with
the expected result, we obtain:
\begin{align*}
    \tilde{G}(E,\vec{k}) = \frac{-c}{(E + i \delta)^2 - \vec{k}^2}
\end{align*}
From this we calculate:
\begin{align*}
    G(\Delta \vec{x} , \Delta t) &= \limes{\delta \rightarrow 0^+}
        - c \int \frac{d^3 \vec{k} dE}{(2 \pi)^4} \frac{e^{-i (E c \Delta t - \vec{k} \cdot \Delta \vec{x})}}{(E + i \delta)^2 - \vec{k}^2}
    \\
    &= \frac{1}{4 \pi \abs{\vec{x} - \vec{x}'}} \delta \klammer{t - t' - \frac{\abs{\vec{x} - \vec{x}'}}{c}} \Theta(t>t')
\end{align*}
So we obtain:
\begin{align*}
    \phi(\vec{x},t) = \frac{1}{4 \pi \epsilonnull} \int d^3 \vec{x}'
        \frac{\rho \klammer{\vec{x}' , t - \frac{\abs{\vec{x} - \vec{x}'}}{c}}}{\abs{\vec{x} - \vec{x}'}}
\end{align*}
If $c \rightarrow \infty$ we obtain Coulomb's law.
For the vector potential we obtain:
\begin{align*}
    \vec{A} (\vec{x},t) = \frac{1}{4 \pi \epsilon_0 c^2} \int d^3 \vec{x}'
        \frac{\vec{J}\klammer{\vec{x}' , t - \frac{\abs{\vec{x} - \vec{x}'}}{c}}}{\abs{\vec{x} - \vec{x}'}}
\end{align*}
The retarded Green's function is given as:
\begin{align*}
    G(\vec{x} - \vec{x}' , t - t') = \frac{1}{2 \pi} \delta
    \klammer{(t - t')^2 - \frac{\abs{\vec{x} - \vec{x}'}^2}{c^2}} \Theta(t>t')
\end{align*}

\subsection{Potential of a moving charge with a constant velocity}

We calculate the scalar and vector potential of a point-like charge $q$,
moving with a velocity $\vec{v}$. The charge density is
\begin{align*}
    \rho(\vec{x}',t') = q \delta (\vec{x}' - \vec{v} t')
\end{align*}
So the scalar potential is:
\begin{align*}
    \Phi(\vec{x},t) = \frac{q}{2 \pi \epsilonnull} \int dt' \delta \klammer{(t - t')^2 - \frac{\abs{\vec{x} - \vec{v} t}^2}{c^2}} \Theta(t>t')
\end{align*}
We want to find the zeros of the argument of the delta funciton. Therefore,
we decompose $\vec{x}$ into $\vec{x} = \vec{x}_{\parallel} + \vec{x}_{\perp}$.
\begin{align*}
    0 = (t-t')^2 - \frac{\abs{\vec{x} - \vec{v} t'}^2 }{c^2}
    = t'^2 - 2 t' t + t^2 - \frac{(x_\parallel - v t')^2 + x_{\perp}^2}{c^2}
\end{align*}
We can now define "boosted variables"
\begin{align*}
    x_b = \gamma \klammer{x_\parallel - v t}
    \hspace{10pt} , \hspace{10pt}
    t_b = \gamma \klammer{t - \frac{x_\parallel v}{c^2}}
    \hspace{10pt} , \hspace{10pt}
    \gamma = \frac{1}{\sqrt{1 - \frac{v^2}{c^2}}}
\end{align*}
We introduce:
\begin{align*}
    \tau^2 = c^2 t^2 - \klammer{x_\parallel^2 + x_\perp^2}
    = c^2 t_b^2 - (x_b^2 + x_\perp^2)
    \hspace{10pt} , \hspace{10pt}
    r_b^2 = x_b^2 + x_\perp^2
\end{align*}
Then:
\begin{align*}
    \frac{t'^2}{\gamma^2} - 2 \frac{t'}{\gamma} + \frac{\tau^2}{c^2} = 0
    \hspace{10pt} , \hspace{10pt}
    \Delta = \frac{4 (t_b^2 - \tau^2 / c^2)}{\gamma^2} = \frac{4 r_b^2}{\gamma^2 c^2}
\end{align*}
Now the solutions are:
\begin{align*}
    t_{\pm} = \gamma \klammer{t_b \pm \frac{r_b}{c}}
\end{align*}
The delta function becomes:
\begin{align*}
    \delta \klammer{(t - t')^2 - \frac{\abs{\vec{x} - \vec{v} t}^2}{c^2}} \Theta(t>t')
    = \frac{c \gamma}{2 r_b} \Theta(t>t') \delta(t' - t_-)
\end{align*}
In conclusion we obtain:
\begin{align*}
    \Phi(\vec{x},t) &= \frac{q}{4 \pi \epsilonnull} \frac{\gamma}{r_b}
    = \frac{q}{4 \pi \epsilonnull} \frac{1}{\sqrt{1 - \frac{v^2}{c^2}}}
        \frac{1}{\eckigeklammer{\klammer{\frac{x_\parallel - v t}{\sqrt{1 - \frac{v^2}{c^2}}}}^2 + x_\perp^2}^{\frac{1}{2}}}
    \\
    \vec{A}(\vec{x},t) &= \frac{\vec{v}}{c^2} \Phi(\vec{x},t)
\end{align*}
In a reference frame where the charge is at rest, the scalar and vector
potentials are:
\begin{align*}
    \Phi(\vec{x},t)|_{rest} = \frac{q}{4 \pi \epsilonnull} \frac{1}{\eckigeklammer{x_\parallel^2 + x_\perp^2}}
    \hspace{10pt} , \hspace{10pt}
    \vec{A}(\vec{x},t)|_{rest} = 0
\end{align*}
We have the following transformations:
\begin{align*}
    x_\parallel \rightarrow \gamma \klammer{x_\parallel - v t}
    \hspace{10pt} , \hspace{10pt}
    x_\perp \rightarrow x_\perp
\end{align*}
And we define the following "four-vector":
\begin{align*}
    \begin{pmatrix}
        \Phi \\ \vec{A}
    \end{pmatrix} \equiv A^\mu
\end{align*}