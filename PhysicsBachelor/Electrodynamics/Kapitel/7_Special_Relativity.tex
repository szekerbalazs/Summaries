\section{Special Relativity}

In special relativity different coordinate systems are related via
Lorentz transformations:
\begin{align*}
    x^\mu \rightarrow x^{\mu'} = \Lambda_{\ \nu}^\mu x^\nu + \rho^\mu
    \hspace{10pt} \text{with} \hspace{10pt}
    x^\mu = \begin{pmatrix}
        x^0 \\ x^1 \\ x^2 \\ x^3
    \end{pmatrix}
    = \begin{pmatrix}
        c t \\ x^1 \\ x^2 \\ x^3
    \end{pmatrix}
    \hspace{10pt} , \hspace{10pt}
    \Lambda_{\ \nu}^\mu \equiv \frac{\partial x'^\mu}{\partial x^\nu}
\end{align*}
$\Lambda_{\ \nu}^\mu$ satisfies: $\Lambda^\mu_{\ \rho} \Lambda_{ \ \sigma}^\nu g_{\mu \nu}
= g_{\rho \sigma}$ with $g_{\mu \nu}$ the metric:
\begin{align*}
    g_{\mu \nu} = \begin{pmatrix}
        1 & 0 & 0 & 0 \\
        0 & -1 & 0 & 0 \\
        0 & 0 & -1 & 0 \\
        0 & 0 & 0 & -1
    \end{pmatrix}
\end{align*}

\subsection{Proper Time}

Lorentz transformations leave invariant "proper-time" intervals which are
defined as
\begin{align*}
    d \tau^2 \equiv c^2 dt^2 - d \vec{x}^2 = g_{\mu \nu} dx^{\mu} dx^\nu
\end{align*}
Further relations:
\begin{align*}
    d \tau^2 = d \tau'^2
    \hspace{10pt} &, \hspace{10pt}
    g_{\rho \sigma} = g_{\mu \nu} \frac{\partial x^\mu}{\partial x_\rho} \frac{\partial x^\nu}{\partial x_\sigma}
    \\
    \frac{\partial x'^{\mu}}{\partial x^\sigma} \frac{\partial x^\sigma}{\partial x'^{\nu}} = \delta_{\mu \nu}
    \hspace{10pt} &, \hspace{10pt}
    \frac{\partial^2 x'^{\mu}}{\partial x^\epsilon \partial x^\rho} = 0
\end{align*}

\subsection{Subgroups of Lorentz transformations}

Group properties: $x^\mu \stackrel{(\Lambda_1,\rho_1)}{\longrightarrow}
x'^{\mu} \stackrel{(\Lambda_2 , \rho_2)}{\longrightarrow} x''^{\mu}$. Then
for $x^\mu \stackrel{(\Lambda_3 , \rho_3)}{\longrightarrow} x''^\mu$ we have:
\begin{align*}
    \Lambda_{3 \rho}^\mu \equiv \Lambda_{2 \nu}^\mu \Lambda_{1 \rho}^\nu
    \hspace{10pt} , \hspace{10pt}
    \rho_3^\mu = \Lambda_{2 \nu}^\mu \rho_1^\nu + \rho_2^\mu
    \hspace{10pt} , \hspace{10pt}
    x''^\mu = \Lambda_{3 \nu}^\mu x^\nu + \rho_3^\mu
\end{align*}
The set of all Lorentz transformations is called the inhomogeneous Lorentz
group or the Poincare' group. The subset of transformations with $\rho^\mu = 0$
is known as the homogeneous Lorentz group. Here:
\begin{align*}
    (\Lambda_{\ 0}^0)^2 - \sum_{i=1}^3 (\Lambda_{\ 0}^i)^2 = 1
    \ \Rightarrow \ (\Lambda_{\ 0}^0)^2 \geq 1
    \\
    g = \Lambda^T g \Lambda \ \Rightarrow \ \det (\Lambda)^2 = 1
    \ \Rightarrow \ \det(\Lambda) = \pm 1
\end{align*}
The subgroup of transformations with $\det(\Lambda) = 1$ and $\Lambda_{\ 0}^0 \geq 1$
is known as the proper group of Lorentz transformations. Transformations with
$\Lambda_0^0 \geq 1$ are known as orthochronous Lorentz transformations
(preserve right flow of time).

Assume a reference frame $O$ in which a particle appears at rest and $O'$
a reference frame in where the particle appears to move with a velocity
$\vec{v}$. Then:
\begin{align*}
    dx'^\mu = \Lambda_{\ \nu}^\mu dx^\nu = \Lambda_{\ 0}^\mu = ct
    \hspace{10pt} , \hspace{10pt}
    dt' = \Lambda_{\ 0}^0 dt
    \hspace{10pt} , \hspace{10pt}
    dx'^i = \Lambda_{\ 0}^i c dt
    \\
    v^i \equiv \frac{dx'^i}{dt'} = c \frac{\Lambda_{\ 0}^i}{\Lambda_{\ 0}^0}
    \ \Rightarrow \ \Lambda_{\ 0}^i = \frac{v^i}{c} \Lambda_{\ 0}^0
    \hspace{10pt} , \hspace{10pt}
    \Lambda_{\ 0}^0 = \gamma
    \ \Rightarrow \ \Lambda_{\ 0}^i = \gamma \frac{v^i}{c}
\end{align*}
For coordinate systems $O$ and $O'$ with parallel axes we find that
\begin{align*}
    \Lambda_{\ j}^i = \delta_j^i + \frac{v^i v^j}{v^2} (\gamma - 1)
    \hspace{10pt} , \hspace{10pt}
    \Lambda_{\ j}^0 = \gamma \frac{v^j}{c}
\end{align*}
The group of Rotations:
\begin{align*}
    \Lambda_{\ 0}^0 = 1
    \hspace{5pt} , \hspace{5pt}
    \Lambda_{\ 0}^i = \Lambda_{\ i}^0 = 0
    \hspace{5pt} , \hspace{5pt}
    \Lambda_{\ j}^i = R_{ij}
    \hspace{10pt}
    \text{with } \det(R) = 1
    \hspace{5pt} , \hspace{5pt}
    R^T R = 1
\end{align*}

\subsection{Time dilation}

Consider an inertial observer $O$ which looks as a clock at rest. Here:
\begin{align*}
    d t = \Delta t
    \hspace{10pt} , \hspace{10pt}
    d \vec{x} = 0
    \hspace{10pt} , \hspace{10pt}
    d \tau = c \Delta t
\end{align*}
A second observer sees the clock with velocity $\vec{v}$. Here:
\begin{align*}
    dt' = \Delta t'
    \hspace{10pt} , \hspace{10pt}
    d \vec{x}' = \vec{v} dt'
    \hspace{10pt} , \hspace{10pt}
    d \tau' = c \Delta t' \sqrt{1 - \frac{\vec{v}^2}{c^2}}
\end{align*}
So we find:
\begin{align*}
    \Delta t' = \frac{\Delta t}{\sqrt{1 - \frac{\vec{v}^2}{c^2}}}
    = \gamma \Delta t
\end{align*}

\subsection{Doppler Effect}

Take moving clock to be a lightsource with frequency $\omega = \frac{2 \pi}{\Delta t}$.
For observer where light source is moving with velocity $\vec{v}$:
$dt' = \gamma \Delta t$ and the distance of the observer from the light
source increased by $v_r dr'$ where $v_r$ is the component of the velocity
of the light source along the direction of sight of the observer. Time
elapsing between the reception of two seccessive light wave fronts from the
observer:
\begin{align*}
    c dt_0 = c dt' + v_r dt'
\end{align*}
Frequency measured by the observer: ($\star$ case if $v_r = v$)
\begin{align*}
    \omega' = \frac{2 \pi}{dt_0} = \frac{\sqrt{1 - \frac{v^2}{c^2}}}{1 + \frac{v_r}{c}} \omega
    \hspace{10pt} \stackrel{\star}{=} \sqrt{\frac{1 - \frac{v}{c}}{1 + \frac{v}{c}}} \omega
\end{align*}

\subsection{Particle dynamics}

Question: How to compute relativistic force. Newtonian expressions for the
force should be valid if a particle is at rest. An elegant solution is to
define a relativistic force acting on a particle as
\begin{align*}
    f^\mu = m c^2 \frac{d^2 x^\mu}{d \tau^2}
\end{align*}
with $m$ the mass of the particle. If the particle is at rest: $d \tau = c dt$.
Therefore, in the restframe of the particle:
\begin{align*}
    f_{\text{rest}}^0 = m c \frac{d^2 t}{d t^2} = 0
    \hspace{10pt} , \hspace{10pt}
    f_{\text{rest}}^i = m \frac{d^2 x^i}{d t^2} = F_{\text{Newton}}^i
    \hspace{10pt} , \hspace{10pt} i = 1,2,3
\end{align*}
where $\vec{F}_{\text{Newton}}$ is the force-vector known from Newtonian
mechanics. The time component vanishes. Under Lorentz transformation
\begin{align*}
    f'^\mu = \Lambda_{\ \nu}^\mu f^\nu
\end{align*}
Transformation from the rest frame of a particle to a frame where the particle
moves with a velocity $\vec{v}$, we have
\begin{align*}
    f^\mu = \Lambda_{\ \nu}^\mu (\vec{v}) f_{\text{rest}}^\nu
\end{align*}
With
\begin{align*}
    \Lambda_{\ 0}^0 (\vec{v}) = \gamma
    \hspace{7pt} , \hspace{7pt}
    \Lambda_{\ 0}^i (\vec{v}) = \Lambda_{\ i}^0 (\vec{v}) = \gamma \frac{v^i}{c}
    \hspace{7pt} , \hspace{7pt}
    \Lambda_{\ j}^i (\vec{v}) = \delta_j^i + \frac{v^i v^j}{v^2} (\gamma - 1)
\end{align*}
Therefore:
\begin{align*}
    \vec{f} = \vec{F}_{\text{Newton}} + (\gamma - 1) \frac{\vec{v} \klammer{\vec{F}_{\text{Newton} \cdot \vec{v}}}}{v^2}
    \hspace{10pt} , \hspace{10pt}
    f^0 = \gamma \frac{\vec{v} \cdot \vec{F}_{\text{Newton}}}{c}
    = \frac{\vec{v}}{c} \cdot \vec{f}
\end{align*}
In Newtonian mechanics we can calculate the trajectory $\vec{x}(t)$ of a
particle by solving a differential equation. Analogously we could solve for
$x^\mu = x^\mu (\tau)$. Therefore, we need to find $\tau = \tau(x^0)$. A second
constraint is:
\begin{align*}
    \Omega \equiv g_{\mu \nu} \frac{d^2 x^\mu}{d \tau^2} \frac{d x^\nu}{d \tau}
    = \frac{2}{m c^2} g_{\mu \nu} f^\mu \frac{d x^\mu}{d \tau}
\end{align*}
The rhs is a Lorentz invariant quantity. We can derive this in the rest frame.
Here $x^\mu = (c t,\vec{0})$ and $f^\mu = (0,\vec{F}_{\text{Newton}})$.
\begin{align*}
    \frac{d \Omega}{d \tau} = \frac{2}{m c^2} \klammer{f^0 \frac{d x^0}{d \tau} - \vec{f} \cdot \vec{x}} = 0
\end{align*}
Therefore, $\Omega$ is always a constant. $\Omega(\tau) = \Omega(\tau) =$
constant $= 1$.

\subsection{Energy and momentum}

The relativistic four-momentum is given as:
\begin{align*}
    p^\mu = m c \frac{d x^\mu}{d \tau}
    \hspace{10pt} , \hspace{10pt}
    p^0 = m \gamma c
    \hspace{10pt} , \hspace{10pt}
    p^i = m \gamma v^i
\end{align*}
with $d \tau = \frac{c \mathds{d} t}{\gamma}$. We identify the relativistic
energy of a particle with
\begin{align*}
    E = c p^0 = m \gamma c^2
\end{align*}
We obtain the relation:
\begin{align*}
    E = \sqrt{\vec{p}^2 c^2 + m^2 c^4}
\end{align*}

\subsection{The inverse of a Lorentz transformation}

We define the inverse of the metric matrix $g_{\mu \nu}$ as $g^{\mu \nu}$.
\begin{align*}
    g^{\mu \nu} g_{\nu \rho} = \delta^\mu_{\ \nu}
    \hspace{10pt} \Rightarrow \hspace{10pt}
    g^{\mu \nu} = g_{\mu \nu} = \text{diag}(1,-1,-1,-1)
\end{align*}
The inverse of a Lorentz transformation $\Lambda_{\ \nu}^\mu$ is
$\Lambda_\mu^{\ \nu}$.
\begin{align*}
    \Lambda_\mu^{\ \nu} \equiv g_{\mu \rho} g^{\nu \sigma} \Lambda_{\ \sigma}^\rho
\end{align*}
If $\Lambda_{\ \nu}^\mu$ is a velocity $\vec{v}$ boost transformation, then
\begin{align*}
    \Lambda_0^{\ 0} (\vec{v}) = \gamma
    \hspace{5pt} , \hspace{5pt}
    \Lambda_i^{\ 0} (\vec{v}) = \Lambda_0^{\ i} (\vec{v}) = - \gamma \frac{v^i}{c}
    \hspace{5pt} , \hspace{5pt}
    \Lambda_i^{\ j} (\vec{v}) = \delta_i^j + \frac{v^i v^j}{v^2} (\gamma - 1)
\end{align*}
Therefore: $\Lambda_\mu^{\ \nu} (\vec{v}) = \Lambda_{\ \nu}^\mu (- \vec{v})$.

\subsection{Vectors and Tensors}

\paragraph{Contravariant vectors} transform according to the rule:
\begin{align*}
    V^\mu \rightarrow V'^\mu = \Lambda_{\ \nu}^\mu V^\nu
\end{align*}
These are vectors that transform the same way as space-time coordinates $x^\mu$.

\paragraph{Covariant vectors} transform according to the rule
\begin{align*}
    U_\mu = \Lambda_\mu^{\ \nu} U_\nu
\end{align*}
Derivatives $\frac{\partial}{\partial x^\mu}$ transform in this way.
These transformations transform according to the inverse Lorentz
transformation.

\paragraph{Dual Vectors} For every contravariant vector $U^\mu$ there
is a dual vector which is covariant.
\begin{align*}
    U_\mu = g_{\mu \nu} U^\nu
    \hspace{10pt} , \hspace{10pt}
    U^\rho = g^{\rho \mu} U_\mu
\end{align*}
The scalar product of a contravariant and a covariant vector is invariant
under Lorentz transformation.
\begin{align*}
    A \cdot B \equiv A^\mu B_\mu = A_\mu B^\mu
\end{align*}
The D'Alembert operator $\Box \equiv \partial^2 \equiv \partial_\mu \partial^\mu
= \frac{1}{c^2} \frac{\partial^2}{\partial t^2} - \vec{\nabla}^2$ is invariant
under Lorentz transformation.

\paragraph{Tensor} We define a tensor with multiple "up" and/or "down"
indices to be an object $T^{\mu_1 \mu_2 \dots}_{\nu_1 \nu_2 \dots}$
which transforms as
\begin{align*}
    T_{\nu_1 \nu_2 \dots}^{\mu_1 \mu_2 \dots} \rightarrow
    \Lambda_{\ \rho_1}^{\mu_1} \Lambda_{\ \rho_2}^{\mu_2} \dots
    \Lambda_{\nu_1}^{\ \sigma_1} \Lambda_{\nu_2}^{\ \sigma_2} \dots
    T_{\sigma_1 \sigma_2 \dots}^{\rho_1 \rho_2 \dots}
\end{align*}

\subsection{Currents and densities}

For $n$ particles with charges $q_n$ and positions $\vec{r}_n (t)$ the
charge and current density are:
\begin{align*}
    \rho(\vec{x},t) &= \sum_n q_n \delta(\vec{x} - \vec{r}_n (t))
    \\
    \vec{j} &= \sum_n q_n \frac{d \vec{r}_n (t)}{dt} \delta(\vec{x} - \vec{r}_n (t))
    = \sum_n q_n \frac{d \vec{x}}{dt} \delta(\vec{x} - \vec{r}_n (t))
\end{align*}
We can combine the charge and current densities into one object:
\begin{align*}
    j^\mu \equiv \klammer{c \rho , \vec{j}}
    \hspace{5pt} \text{ with } \hspace{5pt}
    j^\mu (\vec{x},t) = \sum_n q_n \frac{d x^\mu}{dt} \delta(\vec{x} - \vec{r}_n (t))
\end{align*}
This is a contravariant four-vector. $\delta$-function in four-dimensions:
\begin{align*}
    \delta(x^\mu - y^\mu) &= \delta(x^0 - y^0) \delta(\vec{x} - \vec{y})
    = \frac{1}{c} \delta(t_x - t_y) \delta(\vec{x} - \vec{y})
    \\
    \delta(U'^\mu) &= \delta(\Lambda^\mu_{\ \nu} U^\nu) = \frac{\delta(U^\nu)}{\abs{\det(\Lambda)}}
    = \delta(U^\nu)
\end{align*}
Current-density in integralform:
\begin{align*}
    j^\mu (\vec{x},t) &= \sum_n q_n \int dt' \frac{dx^\mu}{dt'} \delta(\vec{x} - \vec{r}_n(t)) \delta(t' - t)
    \\
    &= c \sum_n q_n \int dt' \frac{dx^\mu}{dt'} \delta(x^\mu - r^\mu_n (t))
    \\
    &= c \sum_n q_n \int d \tau \frac{dx^\mu}{d \tau} \delta(x^\mu - r_n^\mu (\tau))
\end{align*}
Continuity equation:
\begin{align*}
    \frac{\partial \rho}{\partial t} + \vec{\nabla} \cdot \vec{j} = 0
    \hspace{10pt} \rightarrow \hspace{10pt}
    \partial_\mu j^\mu = 0
\end{align*}

\subsection{Energy-Momentum tensor}

For $n$ particles at positions $\vec{r}_n (t)$ the energy density and
"energy-current density" is:
\begin{align*}
    \text{energy density} &= \sum_n E_n(t) \delta(\vec{x} - \vec{r}_n (t))
    \\
    \text{energy current density} &= \sum_n E_n (t) \frac{d \vec{r}_n}{dt} \delta(\vec{x} - \vec{r}_n (t))
\end{align*}
Combine the two:
\begin{align*}
    \sum_n E_n(t) \frac{d r^\nu_n}{dt} \delta(\vec{x} - \vec{r}_n (t))
\end{align*}
We define the "Energy-momentum Tensor" as
\begin{align*}
    T^{\mu \nu} \equiv \sum_n p_n^\mu \frac{d r_n^\mu}{dt} \delta(\vec{x} - \vec{r}_n (t))
    = \sum_n \int d \tau p_n^\mu \frac{d x_n^\nu}{d \tau} \delta(x^\rho - r_n^\rho (\tau))
\end{align*}
It transforms as the product of two-four vectors under Lorentz transformation
\begin{align*}
    T'^{\mu \nu} = \Lambda^\mu_{\ \rho} \Lambda^\nu_{\ \sigma} T^{\rho \sigma}
\end{align*}
It is symmetric: $T^{\mu \nu} = T^{\nu \mu}$ and takes the form:
\begin{align*}
    T^{\mu \nu} \equiv \sum_n \frac{p_n^\mu p_n^\nu}{E_n} \delta(\vec{x} - \vec{r}_n(t))
\end{align*}
We arrive at the equation:
\begin{align*}
    \partial_\nu T^{\mu \nu} = G^\mu =
    \sum_n \frac{\partial p_n^\mu}{\partial t} \delta(\vec{x} - \vec{r}_n)
    = \sum_n \frac{\partial \tau}{\partial t} f_n^\mu (t) \delta(\vec{x} - \vec{r}_n)
\end{align*}
with $G^\mu$ the "density of force". For free particles, $p_n^\mu =$constant.
Therefore, $\partial_\nu T^{\mu \nu} = 0$. Energy momentum Tensor is also conserved
if the particles interact only at the points where they collide with each other.
If the continuity equation is satisfied, then the following vector is conserved.
\begin{align*}
    P^\mu \equiv \int d^3 \vec{x} T^{\mu 0} = \text{const}
    =
    P^\mu = \sum_n \int d^3 \vec{x} p_n^\mu \delta(\vec{x} - \vec{r}_n (t))
    = \sum_n p_n^\mu
\end{align*}
