\subsection{Risks}

\begin{quote}
    "Life is risk. Risk is life."
\end{quote}

\begin{definition}[Risk]
    \begin{itemize}
        \item A risk is a potential event with negative consequences that
            has not happened yet. However, a risk could also be defined as
            the event with unforeseen positive consequences.
        \item A possibility of Loss - Not the loss itself!
            \begin{itemize}
                \item A source of problem during a project
                \item Avoid labelling the cost of a risk as a risk. Find the source!
                \item Stike at the root of the problem, not the leaves!
            \end{itemize}
        \item Something that makes the project special.
            \begin{itemize}
                \item In the widest sense, everything is a risk.
            \end{itemize}
    \end{itemize}
\end{definition}

\paragraph{Types of Risk}
\begin{itemize}
    \item Industrial Risks
        \begin{itemize}
            \item Change in technology, productivity, prices
            \item false estimation of the rated capacity
            \item time needed for the construction and running-in periods,
                political, social and business environment
        \end{itemize}
    \item Operational Risk
        \begin{itemize}
            \item Lack of entrepreneurship skills.
            \item Poor unterstanding of market dynamics.
            \item Poorly available consultancy servises and information systems.
            \item Poof understanding of how to prepare a business plan.
            \item Natural risks.
        \end{itemize}
    \item Market Risks
        \begin{itemize}
            \item Unforeseeable inflation and exchange rates change.
            \item Customer behaviour to buy foreign gooods.
            \item Inadequate infrastructure.
            \item Shrinking market because of foreign competitors.
            \item Defaulting or insolvency, Credit risk.
        \end{itemize}
    \item Cultural risks
    \item Natural risks
    \item Economic and political risks
\end{itemize}

\begin{definition}[Hazard]
    A hazard is an act or phenomenon posing potential harm to some person(s)
    or thing(s), i.e. a source of harm, and its potential consequences.
    Hazards need to be identified and considered in projects' lifecycle analyses
    since they could pose threat and could lead to project failures.
\end{definition}

\begin{definition}[Unknown Unknown]
    A type of uncertainty in which you do not know that this uncertainty
    even exists.
\end{definition}

\begin{definition}[Risk assessment]
    Risk assessment consists of: Hazard identification, Event probability
    assessment and Consequence assessment.
\end{definition}

\begin{definition}[Risk Control]
    Risk control require the definition of acceptable and comparative evaluation
    through monitoring and decision analysis. Risk control also includes
    failure prevention and consequence mitigation. 
\end{definition}

\begin{definition}[Risk Communication]
    Risk communication involves perceptions of risk and depends on the
    audience targeted. Hence, it is classified into Risk communication
    to the media, to the public and to the engineering community.
\end{definition}

\begin{definition}[Human Errors]
    Human errors are unwanted circtumstances caused by humans that result
    in deviations from expected norms that place systems at risk.
    It is important to identify the relevant errors to make accurate risk
    asssessment. Human error identification techniques should provide a
    comprehensive structure for determining significant human errors within
    the system.
    \begin{itemize}
        \item \underline{Human error Modelling}: Currently, there is no
            consensus on how to model humans reliably. The human error
            rate estimates are often based on simulation tests, models,
            and expert estimation.
        \item \underline{Human Error Quantification}: Still a developing
            science requiring understanding of human performance, cognitive
            processing, and human perceptions.
    \end{itemize}
\end{definition}
