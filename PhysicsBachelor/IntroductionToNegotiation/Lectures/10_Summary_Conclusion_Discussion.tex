\section{Summary, Conclusion, Discussion}

\subsection{NECON recommendations}

NECON = Negotiation and Conflict Management

\begin{enumerate}
    \item Before asking for, accepting, or refusing a negotiation, undertake a thorough
        analysis. Refusing one can be an unfriendly act and/or signal that there is
        no room for concessions.

        There are players who perceive the start of negotiation already as a concession
        (although this is not true in rational/friendly environments).
    
    \item Negotiate in good faith, create confidence, while remaining critical. Be
        polite, but be insistent. Do nod play games, do not use dirty tricks. If
        the other does it show that you do not agree. Use the "Tit for Tat" -
        Strategy in a reasonable way.

    \item Stick to a coherent line of argumentation and stick to your engagement
        in the negotiation. Do not change argumentation unless new, objective
        elements appear.

    \item Proceed carefully in the negotiation:
        \begin{enumerate}[(i)]
            \item Do not disclose your red lines ("restrictance points", "reservation
                prices"). In case you think the indication of a red line (true or
                pretended) is useful for tactical reasons, take into account that
                in the event of not respecting it at a later stage you could lose
                credibility.
            \item Do not burn bridges - maybe you will have to change your position.
        \end{enumerate}
    \item Demands / Counter Demands:
        \begin{enumerate}[(i)]
            \item have to be well founded, and
            \item should be based on your own realistic assessment of the possibilies
                of the other side (evidently, this assessment does not need to
                coincide with the offers made by the other side).
        \end{enumerate}
    \item Negotiation Engineering: The method based on a decomposition and
        formalization of the negotiation problem, where the heuristic application
        of mathematical methods facilitates the process of reaching an agreement.
        Identify the difficult elements of the negotiation.
    \item When there are several issues on the table, link them together, if the
        added value for the negotiation of the more controversial issues is larger
        than the potential negative / delaying effect for the negotiation of the
        less controversial issue.
    \item If you have a bigger interest in a negotiated solution than the other
        side, make proposals that solve the problem of the other side (while
        evidently fulfilling your own requirements). If your proposals are rejected
        for comprehensible reasons, make new proposals.
    \item Proceed in phases. Fix (orally or written) the intermedaite result. Make
        clear that you do not intend to backpedal (and that you expect the other
        side not to do it either), but do not give up your pledge before you are
        sure to get the counterpart. Accordingly, make it also clear that
        "nothing is agreed until everything is agreed".
\end{enumerate}


\subsection{Benefits and limits of models}

\subsubsection{Game Theory}

\begin{itemize}
    \item Game theory is a powerful tool for the generation of insights into
        problems
    \item Game theory needs a considerably high level of abstraction - in order
        to allow sensible propositions. In difficult problems the complexity
        may often not be reduced to the level which would allow the use of the
        game theoretical approach.
    \item Game theory is based on utilitarian reasoning, which is not always
        applicable in real life.
\end{itemize}

\subsubsection{Harvard model}

\begin{itemize}
    \item Good remarks regarding mutually beneficial agreements
    \item Recommendations are on a relatively high level of abstaction and
        therefore globally applicable
    \item Per se correct, but in a specific situation often not enough
\end{itemize}

\subsubsection{Negotiation Engineering}

\begin{itemize}
    \item Uses different engineering approaches and puzzling to find a solution
        for complex problems; sort of a mixture of "Harvard" and Game theory.
    \item Approaching political problems with technical means is not always accepted
    \item Not every problem can be devided in subproblems to solve them with technical
        means.
\end{itemize}

\subsubsection{In General}

\begin{itemize}
    \item There is no cookbock or a receipe for good negotiation: It lies in the
        nature of negotiation that there is no "one size fits all" model
    \item Thinking in models is very helpful. It allows to structure a problem
        and to reduce it to its most formal natre: This is already an important
        part of the solution
    \item For every problem, one has to choose the best approach in relation to
        the concrete situation: This lies in the nature of a heuristic method.
\end{itemize}

