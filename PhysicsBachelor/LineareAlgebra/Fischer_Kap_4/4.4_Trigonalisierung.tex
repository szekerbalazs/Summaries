\section{Trigonalisierung}

\Definition{
    
    Sei $F: V \rightarrow V$ ein Endomorphismus und $W \subset V$ ein UVR. $W$ heisst
    \fat{$F$-invariant}, wenn $F(W) \subset W$ ($\Leftrightarrow \forall w \in W: \
    F(w) \in W$)
}

\vspace{1\baselineskip}

\Bemerkung{

    Ist $W \subset V$ ein $F$-invarianter UVR, so ist $P_{F |_W}$ ein Teiler von $P_F$.
    Also $P_F (t) = P_{F |_W} (t) \cdot Q(t)$.
    (Erinnerung: $F |_W: W \rightarrow W$ mit $w \mapsto F(w)$.)
}

\vspace{1\baselineskip}

\Definition{

    Unter einer \fat{Fahne} $(V_r)$ in einem $n$-dimensionalen Vektorraum $V$ versteht
    man eine Kette
    \begin{align*}
        \geschwungeneklammer{0} = V_0 \subset V_1 \subset \dots \subset V_n = V
    \end{align*}
    von UVR mit $\dim V_r = r$. Ist $F \in$ End$(V)$, so heisst die Fahne
    $F$-invariant, wenn
    \begin{align*}
        F(V_r) \subset V_r \quad \text{   für alle } r \in \geschwungeneklammer{0,\dots,n}
    \end{align*}
}

\Bemerkung{

    Für $F \in$ End$(V)$ sind folgende Bedingungen gleichwertig:
    \begin{enumerate}[i)]
        \item Es gibt eine $F$-invariante Fahne in $V$.
        \item Es gibt eine Basis $\mathcal{B}$ von $V$, so dass $M_{\mathcal{B}} (F)$
                eine obere Dreiecksmatrix ist.
    \end{enumerate}
    Ist das der Fall, so heisst $F$ \fat{trigonalisierbar}.
}

\vspace{1\baselineskip}

\Definition{

    Eine Matrix $A \in M(n \times n ; K)$ heisst \fat{trigonalisierbar}, wenn
    $A: K^n \rightarrow K^n$ mit $v \mapsto A v$ trigonalisierbar ist.
    Das heisst, es existiert ein $S \in \text{GL}(n,K)$ sodass $S A S^{-1}$ eine obere
    Dreiecksmatrix ist.
}

\vspace{1\baselineskip}

\Satz{ (\fat{Trigonalisierungssatz})

    Für einen Endomorphismus $F$ eines $n$-dimensionalen $K$-VR sind folgende
    Bedingungen äquivalent:
    \begin{enumerate}[i)]
        \item $F$ ist trigonalisierbar.
        \item Das charakteristische Polynom $P_F$ zerfällt in Linearfaktoren, d.h.
            \begin{align*}
                P_F = \pm (t - \lambda_1) \cdot \dots \cdot (t - \lambda_n)
                \quad \text{   mit } \lambda_1 , \dots , \lambda_n \in K
            \end{align*}
    \end{enumerate}
}

\Korollar{

    Jeder Endomorphismus eines endlich-dimensionalen komplexen VR ist trigonalisierbar.
}

\vspace{1\baselineskip}

\fat{\large Algorithmus zur Bestimmung der Trigonalform} \normalsize

\vspace{1\baselineskip}

\fat{1. Schritt}

Wir betrachten $W_1 = K^n$ mit der Basis $\mathcal{B}_1 = \mathcal{K}$ und den
Endomorphismus $A_1 = A$. Zu $\lambda_1$ berechnet man einen Eigenvektor $V_1 \in K^n$.
Nach dem Austauschlemma bestimmt man ein $j_1 \in \geschwungeneklammer{1,\dots,n}$, so
dass
\begin{align*}
    \mathcal{B}_2 := \klammer{v_1 , e_1 , \dots , \hat{e}_{j_1} , \dots , e_n}
\end{align*}
wieder eine Basis von $K^n$ ist. Das Zeichen $\hat{}$ bedeutet dabei, dass $e_{j_1}$
ausgelassen wird. Wir betrachten die Transformationsmatrix $S_1^{-1} := T_{\mathcal{B}_1}^{\mathcal{B}_2}$
mit der Basis $\mathcal{B}_2$ als Spalten. Dann ist
\begin{align*}
    A_2 := S_1 \cdot A \cdot S_1^{-1} = \begin{pmatrix}
        \lambda_1 & * & \dots & * \\
        0 & & & \\
        \vdots & & A_2' & \\
        0 & & & 
    \end{pmatrix}
\end{align*}

\fat{2. Schritt}

Wir betrachten $W_2$ mit der Basis
\begin{align*}
    \mathcal{B}_2' := \klammer{e_1 , \dots , \hat{e}_{j_1} , \dots , e_n}
\end{align*}
und den Endomorphismus $A_2'$. Es ist
\begin{align*}
    P_{A'_2} = \pm (t-\lambda_2) \cdot \dots \cdot (t-\lambda_n)
\end{align*}
Zu $\lambda_2$ berechnet man einen Eigenvektor $v_2 \in W_2$, und man wählt ein
$j_2 \neq j_1$, so dass 
\begin{align*}
    \mathcal{B}_3' := \klammer{v_2 , e_1 , \dots , \hat{e}_{j_1} , \dots , \hat{e}_{j_2} , \dots , e_n}
\end{align*}
eine Basis von $W_2$ ist, also
\begin{align*}
    \mathcal{B}_3 := \klammer{v_1, v_2 , e_1 , \dots , \hat{e}_{j_1} , \dots , \hat{e}_{j_2} , \dots , e_n}
\end{align*}
eine Basis von $K^n$ ist. Mit der Transformationsmatrix $S_2^{-1} = T_{\mathcal{B}_1}^{\mathcal{B}_3}$
erhält man
\begin{align*}
    A_3 = S_2 \cdot A \cdot S_2^{-1} = \begin{pmatrix}
        \lambda_1 & * & \dots & \dots & * \\
        0 & \lambda_2 & * & \dots & * \\
        \vdots & 0 & & & \\
        \vdots & \vdots & & A_3' & \\
        0 & 0 & & &
    \end{pmatrix}
\end{align*}
Bei der Berechnung von $S_2$ kann man benutzen, dass
\begin{align*}
    T_{\mathcal{B}_1}^{\mathcal{B}_3} = T_{\mathcal{B}_1}^{\mathcal{B}_2} \cdot T_{\mathcal{B}_2}^{\mathcal{B}_3}
    \quad \text{   und   } \quad
    T_{\mathcal{B}_2}^{\mathcal{B}_3} = \begin{pmatrix}
        1 & 0 & \dots & 0 \\
        0 & & & \\
        \vdots & & T_{\mathcal{B}'_2}^{\mathcal{B}'_3} & \\
        0 & & &
    \end{pmatrix}
\end{align*}
Spätestens im $(n-1)$-ten Schritt erhält man eine obere Dreiecksmatrix $A_n$, denn
$A_n'$ ist eine $(1 \times 1)$-Matrix. Also ist
\begin{align*}
    D := A_n = S_{n-1} \cdot A \cdot S_{n-1}^{-1}
\end{align*}
eine obere Dreiecksmatrix.








