\section{Die Jordansche Normalform}

\vspace{1\baselineskip}

\Definition{

    Sei $\lambda$ ein Eigenwert von $F \in$ End$(V)$ und $\dim V = n < \infty$. Dann
    definieren wir den \fat{Hauptraum} (verallgemeinerter Eigenraum) von $F$ bzgl $\lambda$
    als
    \begin{align*}
        \text{Hau} (F;\lambda) = \ker \klammer{F- \lambda id}^r
    \end{align*}
    wobei $r$ die algebraische Vielfachheit von $\lambda$ ist.
    (d.h. $P_F (t) = (t-\lambda)^r \cdot (\dots)$)
}

\vspace{1\baselineskip}

\Bemerkung{

    Es gilt Eig$(F,\lambda) = \ker (F - \lambda \text{id}) \subset$ Hau$(F; \lambda)$
}

\vspace{1\baselineskip}

\Satz{ (\fat{Satz über die Hauptraumzerlegung})

    Sei $F \in$ End$_K (V)$ und
    \begin{align*}
        P_F = \pm (t-\lambda_1)^{r_1} \cdot \dots \cdot (t-\lambda_k)^{r_k}
    \end{align*}
    mit paarweise verschiedenen $\lambda_1 , \dots , \lambda_k \in K$.
    Es sei $V_i :=$ Hau$(F;\lambda_i) \subset V$ für jedes $\lambda_i$
    der Hauptraum. Dann gilt:
    \begin{enumerate}[1)]
        \item $F(V_i) \subset V_i$ und $\dim V_i = r_i$ für $i = 1,\dots,k$
        \item $V = V_1 \oplus \dots \oplus V_k$
        \item $F$ hat eine Zerlegung $F = F_D + F_N$ mit
                \begin{enumerate}[a)]
                    \item $F_D$ diagonalisierbar
                    \item $F_N$ nilpotent
                    \item $F_D \circ F_N = F_N \circ F_D$
                \end{enumerate}
        \item $F_D$ und $F_N$ lassen sich als Polynome in $F$ schreiben.
                Insbesondere kommutieren sie mit $F$.
        \item Die Zerlegung $F=F_D + F_N$ ist eindeutig wenn man a),b) und c)
                verlangt.
    \end{enumerate}
}

\vspace{1\baselineskip}

\Korollar{

    Sei $A \in M(n \times n; K)$, so dass
    $P_A = \pm (t-\lambda_1)^{r_1} \cdot \dots \cdot (t-\lambda_k)^{r_k}$.
    Dann gibt es eine invertierbare Matrix $S \in$ GL$(n;K)$ derart, dass
    \begin{align*}
        S A S^{-1} = \begin{pmatrix}
            \lambda_1 E_{r_1} + N_1 & & 0 \\
            & \ddots & \\
            0 & & \lambda_k E_{r_k} + N_k
        \end{pmatrix}
        := \tilde{A}
    \end{align*}
    mit $\lambda_i E_{r_i} + N_i$ Blöcken für alle $i = 1,\dots,k$
    welche wie folgt gegeben sind:
    \begin{align*}
        \lambda_i E_{r_i} + N_i = \begin{pmatrix}
            \lambda_i & & * \\
            & \ddots & \\
            0 & & \lambda_i
        \end{pmatrix}
        \in M(r_i \times r_i \ ; K)
    \end{align*}
    mit nilpotenten $N_i$. Insbesondere ist $\tilde{A} = D + N$, wobei
    $D$ eine Diagonalmatrix und $N$ nilpotent (strikte obere Dreiecksmatrix)
    ist. Und es gilt $D \cdot N = N \cdot D$.
}

\vspace{1\baselineskip}

\Lemma{ (Lemma von Fitting)

    Zu einem $G \in$ End$_K (V)$ betrachten wir die beiden Zahlen
    \begin{align*}
        &d:= \min \geschwungeneklammer{l \in \N : \ \ker(G^l) = \ker (G^{l+1})}
        \\
        &r:= \mu(P_G \ ; 0)
    \end{align*}
    wobei $G^0 :=$ id$_V$. Dann gilt:
    \begin{enumerate}[1)]
        \item $d = \min \geschwungeneklammer{l: \ \text{Im} \ G^l = \text{Im} \ G^{l+1}}$
        \item $\ker \ G^{d+1} = \ker \ G^{d}$, $\Im \ G^{d+1} = \Im \ G^d$ für alle $i \in \N$.
        \item Die Räume $U := \ker \ G^d$ und $W:= \Im \ G^d$ sind $G$-invariant.
        \item $(G | U)^d = 0$ und $G |_W : W \rightarrow W$ ist ein Isomorphismus.
        \item Für das Minimalpolynom von $G |_U$ gilt $M_{G |_U} = t^d$.
        \item $V = U \oplus W$, $\dim U = \mu(P_G,0) = r\geq d$, $\dim W = n-r$.
    \end{enumerate}
    Insbesondere gibt es eine Basis $\mathcal{B}$ von $V$, so dass
    \begin{align*}
        M_{\mathcal{B}} (G) = \begin{pmatrix}
            N & 0 \\
            0 & C
        \end{pmatrix}
        \quad \text{   mit   } \
        N^d = 0
        \ \text{ und } \
        C \in \GL (n-r \ ; K)
    \end{align*}
}

\vspace{1\baselineskip}

\Definition{

    Wir definieren die \fat{Jordanmatrix} als
    \begin{align*}
        J_k := \begin{pmatrix}
            0 & 1 & & 0 \\
            & \ddots & \ddots & \\
            & & \ddots & 1 \\
            0 & & & 0
        \end{pmatrix}
        \in M(k \times k \ ; K)
    \end{align*}
    $J_k$ ist nilpotent und $(J_k)^k = 0$. $k$ ist die minimale Potenz mit dieser Eigenschaft.
}

\vspace{1\baselineskip}

\Theorem{

    Sei $G$ ein nilpotenter Endomorphismus eines $K$-VR $V$ und $d:= \min
    \geschwungeneklammer{l: G^l = 0}$. Dann gibt es eindeutig bestimmte Zahlen
    $s_1,\dots,s_d \in \N$ mit
    \begin{align*}
        s_1 + 2 \cdot s_2 + 3 \cdot s_3 + \dots + d \cdot s_d = r = \dim V
    \end{align*}
    und eine Basis $\mathcal{B}$ von $V$, so dass
    \begin{align*}
        M_{\mathcal{B}} (G) = \begin{pmatrix}
            J_d & & & & & & & & & \\
            & \ddots & & & & & & & & \\
            & & J_d & & & & & & & \\
            & & & J_{d-1} & & & & 0 & & \\
            & & & & \ddots & & & & & \\
            & & & & & J_{d-1} & & & & \\
            & & & & & & \ddots & & & \\
            & & & 0 & & & & J_1 & & \\
            & & & & & & & & \ddots & \\
            & & & & & & & & & J_1 \\
        \end{pmatrix}
    \end{align*}
    mit $s_d$-mal $J_d$, $s_{d-1}$-mal $J_{d-1}$, \dots , $s_1$-mal $J_1$.
    Beachte: $J_1 = 0$.
}

\vspace{1\baselineskip}

\Definition{ (\fat{Jordansche Normalform})

    Sei $F \in \End_K (V)$ derart, dass das charakteristische Polynom in Linearfaktoren
    zerfällt, also
    \begin{align*}
        P_F = \pm (t-\lambda_1)^{r_1} \cdot \dots \cdot (t-\lambda_k)^{r_k}
    \end{align*} 
    mit paarweise verschiedenen $\lambda_1,\dots,\lambda_k \in K$. Dann gibt es eine Basis
    $\mathcal{B}$ von $V$, so, dass
    \begin{align*}
        M_{\mathcal{B}} (F) = \begin{pmatrix}
            \lambda_1 E_{r_1} + N_1 & & 0 \\
            & \ddots & \\
            0 & & \lambda_k E_{r_k} + N_k
        \end{pmatrix}
    \end{align*}
    Mit $\lambda_i E_{r_i} + N_i$ als Blöcken, wobei $N_i$ für $1,\dots,k$ in der
    Normalform ist. Ausgeschrieben bedeutet das
    \begin{align*}
        \lambda_i E_{r_i} + N_i =
        \begin{pmatrix}
            J_k (\lambda_i) & 0 & \\
            & J_k (\lambda_i) & 0 \\
            & & J_k (\lambda_i)
        \end{pmatrix}
    \end{align*}
    mit
    \begin{align*}
        J_k (\lambda_i) = \begin{pmatrix}
            \lambda_i & 1 & & \\
            & \ddots & \ddots & \\
            & & \ddots & 1 \\
            & & & \lambda_i
        \end{pmatrix}
    \end{align*}
    und Nullen auf der Nebendiagonalen, wo keine $1$-en sind.
    Solch $J_k$ nennt man \fat{Jordanblöcke} der Länge $d$ zu $\lambda_i$.
    der oberste und grösste Jordanblock in $\lambda_i R_{r_i} + N_i$ hat die
    Grösse $d_i$ mit
    \begin{align*}
        1 \leq d_i = \min \geschwungeneklammer{l: N_i^l = 0} \leq r_i
    \end{align*}
    das ist die Vielfachheit der Nullstelle $\lambda_i$ im Minimalpolynom von
    $F$. Für $1 \leq j \leq d_i$ seien $s_j^{(i)} \geq 0$ die Anzahl der Jordanblöcke
    der Grösse $j$ zu $\lambda_i$ in $\lambda_i E_{r_i} + N_i$. Es ist $s_{d_i}^{(i)} \leq 1$,
    und durch Abzählung der Längen folgt:
    \begin{align*}
        d_i s_{d_i}^{(i)} + (d-1) s_{d_i -1}^{(i)} + \dots + s_1^{(i)} = r_i
    \end{align*}
    Des Weiteren gilt: $r_1 + \dots + r_k = n$.

    Die Reihenfolge der Jordanblöcke ist unwesentlich, da sie durch eine Permutation der
    Basisvektoren beliebig verändert werden kann.
}

\vspace{1\baselineskip}

\Korollar{

    Für ein $F \in \End_k (V)$ sind folgende Bedingungen äquivalent:
    \begin{enumerate} [i)]
        \item $F$ ist diagonalisierbar
        \item $M_F = (t-\lambda_1) \cdot \dots \cdot (t-\lambda_k)$, wobei $\lambda_1,\dots,
                \lambda_k$ die verschiedenen Eigenwerte von $F$ bezeichnen.
        \item Es gibt paarweise verschiedene $\lambda_1,\dots\lambda_m \in K$, so, dass
                \begin{align*}
                    (F-\lambda_1 \id_V) \circ \dots \circ (F-\lambda_m \id_V) = 0 \in \End_K (V)
                \end{align*}
    \end{enumerate}
}

\vspace{1\baselineskip}

\Korollar{

    Allgemein gilt für ein $F \in \End_K (V)$:
    \begin{align*}
        M_F (t) = (t-\lambda_1)^{d_1} \cdot \dots \cdot (t-\lambda_k)^{d_k}
    \end{align*}
    wobei $d_i$ die Länge des grössten Jordanblockes zu $\lambda_i$ ist.
}
