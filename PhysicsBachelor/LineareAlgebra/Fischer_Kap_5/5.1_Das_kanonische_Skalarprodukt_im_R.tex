\section{Das kanonische Skalarprodukt im $\R^n$}

\vspace{1\baselineskip}

\Definition{

    Das \fat{kanonische Skalarprodukt} ist eine Abbildung
    \begin{align*}
        \scalprod{ \ }{ \ } : \R^n \times \R^n &\longrightarrow \R
        \\
        (x,y) &\mapsto \scalprod{x}{y}
    \end{align*}
    mit $x = (x_1,\dots,x_n)$ und $y=(y_1,\dots,y_n)$ $\in \R^n$. Schreibt man $x$ und $y$ als
    Spaltenvektoren, so ist:
    \begin{align*}
        \scalprod{x}{y} = x^T \cdot y = (x_1,\dots,x_n) \cdot \begin{pmatrix}
            y_1 \\ \vdots \\ y_n
        \end{pmatrix}
        = \sum_{i=1}^n x_i y_i
    \end{align*}
    Formal muss folgendes erfüllt sein für $x,x',y,y' \in \R^n$ und $\lambda \in \R$:

    (1) (Bilinearität) $\scalprod{x+x'}{y} = \scalprod{x}{y} + \scalprod{x'}{y}$

        \hspace{17pt} $\scalprod{x}{y+y'} = \scalprod{x}{y} + \scalprod{x}{y'}$

        \hspace{17pt} $\scalprod{\lambda x}{y} = \lambda \scalprod{x}{y}$

        \hspace{17pt} $\scalprod{x}{\lambda y} = \lambda \scalprod{x}{y}$

    (2) (Symmetrie) $\scalprod{x}{y} = \scalprod{y}{x}$

    (3) (Positive Definitheit) $\scalprod{x}{x} \geq 0 \quad \text{   und   }  \quad 
                \scalprod{x}{x} = 0 \Leftrightarrow$ 
                
                \hspace{17pt} $x=0$
}

\vspace{1\baselineskip}

\Definition{

    Die \fat{Norm} ist definiert als eine Abbildung:
    \begin{align*}
        &\Norm{ \ } : \R^n \rightarrow \R_{\geq 0} \\
        &x \mapsto \Norm{x} := \sqrt{\scalprod{x}{x}}
    \end{align*}
    mit folgenden Eigenschaften:
    \begin{enumerate}[{\fat{N1}}]
        \item $\Norm{x} = 0 \Leftrightarrow x=0$
        \item $\lambda x = \abs{\lambda} \Norm{x}$
        \item $\Norm{x+y} \leq \Norm{x} + \Norm{y}$
    \end{enumerate}
}

\vspace{1\baselineskip}

\Definition{

    Analog zur Norm, kann man den \fat{Abstand} definieren
    \begin{align*}
        &d: \R^n \times \R^n \rightarrow \R_{\geq 0} \\
        &d(x,y):= \Norm{y-x} = \sqrt{(y_1 - x_1)^2 + \dots + (y_n - x_n)^2}
    \end{align*}
    mit folgenden Eigenschaften
    \begin{enumerate}[{\fat{D1}}]
        \item $d(x,y) = 0 \Leftrightarrow x=y$
        \item $d(x,y) = d(y-x)$
        \item $d(x,z) \leq d(x,y) + d(y,z)$
    \end{enumerate}
}

\vspace{1\baselineskip}

\Satz{ (Ungleichung von Cauchy-Schwarz)

    $\abs{\scalprod{x}{y}} \leq \Norm{x} \cdot \Norm{y}$
    \ und \
    $\abs{\scalprod{x}{y}} = \Norm{x} \cdot \Norm{y}$
    \ genau dann, wenn $x$ und $y$ linear unabhängig sind.
}

\vspace{1\baselineskip}

\Definition{

    Wir definieren den Winkel $\vartheta$ zwischen zwei Vektoren $x,y \in \R^n$ als
    \begin{align*}
        \vartheta = \angle (x,y) =
        \arccos \klammer{\frac{\scalprod{x}{y}}{\Norm{x} \cdot \Norm{y}}}
    \end{align*}
    mit
    \begin{align*}
        &\angle (x,y) = \angle (y,x) \\
        &\angle(x,y) = \angle(\alpha x, \beta y) \\
        &\scalprod{x}{y} = \cos (\angle(x,y)) \cdot \Norm{x} \cdot \Norm{y}
    \end{align*}
}

\vspace{1\baselineskip}

\Definition{

    Wir sagen $x,y \in \R^n$ sind \fat{orthogonal} zueinander, schreibe $x \perp y$, wenn
    $\scalprod{x}{y} = 0$.
}
