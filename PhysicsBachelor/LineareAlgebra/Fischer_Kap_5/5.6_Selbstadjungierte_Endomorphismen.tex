\section{Selbstadjungierte Endomorphismen}

\vspace{1\baselineskip}

\Definition{

    Sei $V$ ein endlichdimensionaler euklidischer bzw. unitärer VR bezeichnet. Zu einem
    $F \in \End (V)$ definieren wir den \fat{adjungierten Endomorphismus} $F^{\text{ad}}$,
    charakterisert durch
    \begin{align*}
        \scalprod{F(v)}{w} = \scalprod{v}{F^{\text{ad}} (w)}
    \end{align*}
}

\vspace{1\baselineskip}

\Bemerkung{

    Es gilt $M_{\mathcal{B}} (F^{\text{ad}}) = \klammer{M_{\mathcal{B}} (F)}^{\dagger}$
}

\vspace{1\baselineskip}

\Definition{

    Ein Endomorphismus $F$ eines euklidischen bzw. unitären VR $V$ heisst \fat{selbstadjungieret},
    wenn
    \begin{align*}
        \scalprod{F(v)}{w} = \scalprod{v}{F(w)} \quad \forall v,w \in V
    \end{align*}
}

\vspace{1\baselineskip}

\Satz{

    Sei $F$ ein Endomorphismus von $V$ und $\mathcal{B}$ eine Orthonormalbasis. Dann gilt:
    $F$ selbstadjungieret $\Leftrightarrow$ $M_{\mathcal{B}} (F)$ symmetrisch bzw. hermitisch.
}

\vspace{1\baselineskip}

\Lemma{

    Ist $F$ selbstadjungieret, so sind (auch im komplexen Fall) alle Eigenwerte reell.
    Insbesondere hat eine hermitische Matrix nur reelle Eigenwerte.
}

\vspace{1\baselineskip}

\Theorem{

    Ist $F$ ein selbstadjungiereter Endomorphismus eines euklidischen bzw. unitären VR, so
    gibt es eine Orthonormalbasis von $V$, die aus Eigenvektoren von $F$ besteht.
    Insbesondere ist $F$ diagonalisierbar.
}

\vspace{1\baselineskip}

\Korollar{

    Ist $A \in M(n \times n ; \K)$ eine symmetrische bzw. hermitische Matrix, so
    gibt es eine orthogonale bzw. unitäre Matrix $S$, so dass
    \begin{align*}
        S^{\dagger} \cdot A \cdot S = \begin{pmatrix}
            \lambda_1 & & 0 \\
            & \ddots & \\
            0 & & \lambda_n
        \end{pmatrix}
    \end{align*}
    mit $\lambda_1,\dots,\lambda_n \in \R$.
}

\vspace{1\baselineskip}

\Korollar{

    Sind $\lambda_1,\dots,\lambda_k$ die verschiedenen Eigenwerte eines selbstadjungierten
    oder unitären Endomorphismus $F$ von $V$, so ist
    \begin{align*}
        V = \text{Eig} (F;\lambda_1) \obot \dots \obot \text{Eig} (F;\lambda_k)
    \end{align*}
}

\vspace{1\baselineskip}

\large \fat{Algorithmus zur Berechnung einer Orthonormalbasis aus Eigenvektoren} \normalsize

\begin{enumerate}[{1)}]
    \item Man bestimme die Linearfaktorzerlegung des charakteristischen Polynoms
            $P_F = \pm (t-\lambda_1)^{r_1} \cdot \dots \cdot (t-\lambda_k)^{r_k}$
            wobei $\lambda_1,...,\lambda_k$ paarweise verschieden sind.
    \item Für jeden Eigenwert $\lambda$ der Vielfachheit $r$ bestimme man eine Basis von
            Eig$(F;\lambda)$ durch Lösen eines linearen Gleichungssystems.
    \item Man orthonormalisiere die in $2)$ erhaltene Basen mit dem Gram-Schmidt Verfahren
            und zwar unabhängig voneinander in den verschiedenen Eigenräumen.
    \item Die $k$ Basen der Eigenräume aus $3)$ bilden zusammen die gesuchte Basis aus
            Eigenvektoren von $V$.
\end{enumerate}

\vspace{1\baselineskip}

\Bemerkung{

    Sei $A \in M(n \times n \ ; K)$. Wenn wie eine Orthonormalbasis $\mathcal{B} = (v_1,\dots,v_n)$
    haben, so gilt
    \begin{align*}
        T:=
        \begin{pmatrix}
            \vdots & & \vdots \\
            v_1 & \dots & v_n \\
            \vdots & & \vdots  
        \end{pmatrix}
        = T_{\mathcal{K}}^{\mathcal{B}}
    \end{align*}
    und
    \begin{align*}
        T^T A T = \begin{pmatrix}
            \lambda_1 & & 0 \\
            & \ddots & \\
            0 & & \lambda_n
        \end{pmatrix}
        := D
    \end{align*}
    oder analog $T \cdot D \cdot T^T = A$.    
}

\vspace{1\baselineskip}

\Lemma{

    Jede symmetrische Matrix $A \in M(n \times n \ ; \R)$ hat einen reellen Eigenwert.
}

\vspace{1\baselineskip}

\Theorem{ (Sazt von Courant-Fischer)

    Sei $V$ ein endlich dimensionaler euklidischer oder unitärer VR mit $n = \dim V$ und
    $F \in \End (V)$ ein selbstadjungiereter Endomorphismus. (Also gilt $\scalprod{x}{Fx}
    = \scalprod{x}{Fx}$.) Seien die Eigenwerte von
    $F$ der Grösse nach geordnet $\lambda_1 \leq \dots \leq \lambda_n$. Dann gilt
    \begin{align*}
        \lambda_k = 
        \min_{\stackrel{U \subset V}{\dim U = k}} \max_{\stackrel{x \in U}{x \neq 0}} \frac{\scalprod{x}{Fx}}{\scalprod{x}{x}}
        \ = \
        \max_{\stackrel{U \subset V}{\dim U = n-k+1}} \min_{\stackrel{x \in U}{x \neq 0}} \frac{\scalprod{x}{Fx}}{\scalprod{x}{x}}
    \end{align*}
    Die erste Extremisierung wird dabei über UVR $U \subset V$ der angegebenen Dimension
    durchgeführt, die zweite über nicht-verschwindende Vektoren in $U$. Man beachte, dass
    für den grössten und kleinsten Eigenwert insbesondere gilt
    \begin{align*}
        \lambda_n = \max_{\stackrel{x \in V}{x \neq 0}} \frac{\scalprod{x}{Fx}}{\scalprod{x}{x}}
        \\
        \lambda_1 = \min_{\stackrel{x \in V}{x \neq 0}} \frac{\scalprod{x}{Fx}}{\scalprod{x}{x}}
    \end{align*}
}
