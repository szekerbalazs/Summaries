\section{Hauptachsentransformation}

\vspace{1\baselineskip}

\fat{Hauptachsentransformation symmetrischer Matrizen}

\vspace{1\baselineskip}

Sei $A \in M(n \times n \ ; \R)$ symmetrisch und $s$ die durch $A$ beschriebene Bilinearform
auf $\R^n$. Dann gilt:
\begin{enumerate}[{1)}]
    \item Ist $\mathcal{B} = (w_1,\dots,w_n)$ eine Orthonormalbasis des $\R^n$ bezüglich des
            Standardskalarpdoduktes bestehend aus Eigenvektoren des Endomorphimsus $A$, so ist
            \begin{align*}
                M_{\mathcal{B}} (s) = \begin{pmatrix}
                    \lambda_1 & & 0 \\
                    & \ddots & \\
                    0 & & \lambda_n
                \end{pmatrix}
                \quad \text{, dh. } s(w_i , w_j) = \lambda_i \cdot \delta_{ij}
            \end{align*}
            wobei $\lambda_1,\dots,\lambda_n$ die Eigenwerte von $A$ sind.
    \item Es gibt eine Basis $\mathcal{B}'$ des $\R^n$, so dass
            \begin{align*}
                M_{\mathcal{B}'} (s) = \begin{pmatrix}
                    E_k & & 0 \\
                    & - E_k & \\
                    0 & & 0 
                \end{pmatrix}
                = D'
            \end{align*}
            d.h. es gibt ein $T' \in \GL (n;\R)$ mit $D' = (T')^T \cdot A \cdot T^T$
\end{enumerate}

\vspace{1\baselineskip}

\fat{Vorsicht!}
Die Eigenwerte von $A$ sind nur Invarianten des Endomorphimsus, für eine Bilinearform
ist der Begriff Eigenwert sinnlos.

\vspace{1\baselineskip}

\Bemerkung{

    Betrachte die quadratische Gleichung $q(x) = s(x,x) = x^T A x = 1$ mit $x \in \R^n$.
    Die Lösungsmenge nennt man \fat{Quadrik}. Falls $A$ positiv definit ist, so ist dies ein
    Elipsoid. Die Hauptachsen entsprechen gerade den Eigenvektoren von $A$.
}

\vspace{1\baselineskip}

\Korollar{

    Eine symmetrische Matrix $A \in M(n \times n \ ; \R)$ ist genau dann positiv definit,
    wenn ihre Eigenwerte $\lambda_1,\dots,\lambda_n$ positiv sind.
}

\vspace{1\baselineskip}

\Korollar{

    Sei $A \in M(n \times n \ ; \R)$ eine symmetrische Matrix und
    $P_A = (-1)^n t^n + \alpha_{n-1} t^{n-1} + \dots + a_1 t + \alpha_0$ ihr charakteristisches
    Polynom. Dann gilt: $A$ positiv definit $\Leftrightarrow$ $(-1)^j \alpha_j > 0$ für
    $j=0,\dots,(n-1)$.
}

\vspace{1\baselineskip}

\Lemma{

    Das reelle Polynom $f(t) = t^n + \alpha_{n-1} t^{n-1} + \dots + \alpha_1 t + \alpha_0$
    hat reelle NST $\lambda_1 , \dots , \lambda_n$, also
    $f(t) = (t-\lambda_1) \cdot \dots \cdot (t-\lambda_n)$, also
    \begin{enumerate}[{a)}]
        \item $(\lambda_i < 0 \ \forall i) \Leftrightarrow (\alpha_i > 0 \ \forall i)$
        \item $(\lambda_i > 0 \ \forall i) \Leftrightarrow ((-1)^{n-i} \alpha_i > 0 \ \forall i)$
    \end{enumerate}
}

\vspace{1\baselineskip}

\Definition{

    Wir definieren den \fat{Ausartungsraum} von $s$ als:
    \begin{align*}
        V_0 := \geschwungeneklammer{v \in V \ : \ s(v,w) = 0 \ \forall w \in V} \subset V
    \end{align*}
    Wir definieren den \fat{Rang} von $s$ als    
    $\rang (s) := \dim V - \dim V_0$.
}

\vspace{1\baselineskip}

\Bemerkung{

    Ist $V = \R^n$ und $A = M (s)$, so ist
    \begin{align*}
        V_0 &= \ker A = \geschwungeneklammer{v \in V \ | \ A v = 0}
        \\ &= \geschwungeneklammer{v \in V \ | \ w^T A v = 0 \ \forall w \in V}        
    \end{align*}
    und $\rang (A) = \rang (s)$. Ist für allgemeine $V$ die Matrix $A = M (s)$ bzgl irgendeiner
    Basis, so ist $\rang (s) = \rang (A)$.
}

\vspace{1\baselineskip}

\fat{Diagonalisierung einer quadratischen Form}

\vspace{1\baselineskip}

Sei $V$ ein $\R$-VR mit $n:= \dim V$ und $s$ eine symmetrische Bilinearform mit der zugehörigen
quadratischen Form $q$ auf $V$. Dann gibt es eine Basis

$\B = (v_1,\dots,v_k,v_{k+1},\dots,v_r,v_{r+1},\dots,v_n)$

von $V$ mit folgenden Eigenschaften: Es ist $r = \rang (s)$ und für
\begin{align*}
    v=\sum_{i=1}^n \alpha_i v_i \in V
    \quad \text{  gilt  } \quad
    q(v) = \sum_{i=1}^k \alpha_i^2 - \sum_{i=k+1}^r \alpha_i^2
\end{align*}
Insbesondere hat man eine Zerlegung $V = V_+ \oplus V_- \oplus V_0$ mit $q(v) > 0$ für
$0 \neq v \in V_+$, $q(v)<0$ für $0 \neq v \in V_-$ und $q(v)=0$ für $v \in V_0$.

\vspace{1\baselineskip}

\Satz{ (Trägheitssatz von Sylvester)

    Hat man für eine quadratische Form $q$ auf einem $\R$-VR $V$ zwei Zerlegungen
    $V = V_+ \oplus V_- \oplus V_0 = V_+' \oplus V_-' \oplus V_0'$ mit
    $q(v)>0$ für $0 \neq v \in V_+$ und $0 \neq v \in V_+'$, sowie
    $q(v)<0$ für $0 \neq v \in V_-$ und $0 \neq v \in V_-'$ so folgt:
    $\dim V_+' = \dim V_+$ und $\dim V_-' = \dim V_-$.

    Die Zahlen $r_+ (q) := \dim V_+$ und $r_- (q) := \dim V_-$ sind also neben dem Rang weitere
    Invarianten der quadratischen Form. Das Paar $(r_+ (q) , r_- (q))$ wird auch
    \fat{Signatur} von $q$ genannt.
}

\vspace{1\baselineskip}

\Korollar{

    Sei $A \in M(n \times n \ ;\R)$ symmetrisch und $S \in \GL (n:\R)$. Dann haben
    $A$  und $S^T A S$ mit Vielfachheit gezählt die gleichen Anzahl positiver und negativer
    Eigenwerte. Insbesondere ist $S^T S = S^T E_n S$ positiv definit.
}

\vspace{1\baselineskip}

\fat{Vorsicht!}
Beachte, dass die Eigenwerte von $S^{-1} A S$ gleich denen von $A$ sind. Die von $S^T A S$
sind im Allgemeinen unterschiedlich, aber das Korollar sagt, dass zumindest die Vorzeichen
erhalten bleiben.

\vspace{1\baselineskip}

\Satz{ \fat{(Orthogonalisierungssatz)}

    Sei $V$ ein endlichdimensionaler VR über einem Körper $K$ mit char$(K) \neq 2$ und
    $s: V \times V \rightarrow K$ eine symmetrische Bilinearform. Dann gibt es eine
    Basis $\B = (v_1,\dots,v_n)$ von $V$ mit $s(v_i , v_j) = 0$ für $i \neq j$.
}

\vspace{1\baselineskip}

\Bemerkung{

    Ist $q: V \rightarrow K$ die zur Bilinearform $s: V \times V \rightarrow K$ gehörige
    quadratische Form, und $\B = (v_1,\dots,v_n)$ wie im obigen Satz, so folgt, mit
    $\alpha_j = q(v_j)$, dass für $v=x_1 v_1 + \dots + x_n v_n \in V$ gilt:
    $q(v) = \alpha_1 x_1^2 + \dots + \alpha_n x_n^2$. Die $\alpha_j$ sind dabei nicht
    eindeutig durch $s$ festgelegt. Ersetzt man zB. $v_j$ durch $\beta v_j$, so muss man
    $\alpha_j = q(v_j)$ ersezen durch $\beta^2 \alpha_j$. Insbesondere, falls $K = \C$,
    kann man so alle $\alpha_j$ zu $1$ oder $0$ werden lassen.
}

\vspace{1\baselineskip}

\Korollar{

    Zu einer symmetrischen Matrix $A \in M(n \times n \ ; K)$ gibt es ein $S \in \GL(n;K)$,
    so dass
    \begin{align*}
        S^T A S = \begin{pmatrix}
            \alpha_1 & & 0 \\
            & \ddots & \\
            0 & & \alpha_n
        \end{pmatrix}
    \end{align*}
}

\vspace{1\baselineskip}

\large \fat{Algorithmus zur Bestimmung von $S$ und $D$ im reellen Fall} \normalsize

\begin{enumerate}[{1)}]
    \item Wir schreiben die Matrix $A$ und die Einheitsmatrix $E_n$ übereinander. Also $\frac{A}{E_n}$
    \item Wir führen in beiden Matrizen jeweils elementare Spaltenumformung durch, und dann,
            \underline{nur in A} noch die zugehörige Zeilentransformation. Dies ergibt mit
            $C_1 , \dots , C_r$ die entsprechenden Elementarpatrizen.
            $\frac{C_r^T \cdot \dots C_1^T \cdot A \cdot C_1 \cdot \dots \cdot C_r}{C_1 \cdot \dots \cdot C_r}
            = \frac{D}{S}$
    \item Hat man die Diagonalmatrix $D$ erzeugt, so gilt
    
            $S^T \cdot A \cdot S = D$
\end{enumerate}

\vspace{1\baselineskip}

\Bemerkung{

    Sei $A \in M(n \times n \ ; \R)$ und $S \in \GL(n;\R)$, so dass
    \begin{align*}
        S^T A S = \begin{pmatrix}
            \alpha & & 0 \\
            & \ddots & \\
            0 & & \alpha_n
        \end{pmatrix}
    \end{align*}
    so folt: $A$ positiv definit $\Leftrightarrow$ $\alpha_i >0 \ \forall i = 1,\dots,n$.
}

\vspace{1\baselineskip}

\Definition{

    Sei $A_k$ die linke obere $k$-reihige und $k$-spaltige Teilmatrix von $A$. Ihre
    Determinante $\det A_k$ heisst \fat{Hauptminor} von $A$.
}

\vspace{1\baselineskip}

\Satz{ \fat{(Hauptminoren-Kriterium für Definitheit)}

    Für eine symmetrische Matrix $A \in M(n \times n \ ; \R)$ gilt:

    $A$ positiv definit $\Leftrightarrow$ $\det A_k > 0$ für $k = 1,\dots,n$
}

\vspace{1\baselineskip}

\Theorem{

    Sei $A \in M(n \times n \ ; \C)$ eine hermitische Matrix, also $A^{\dagger} =
    \overline{(A^T)} = A$. Sei $s$ die entsprechende hermitische Sesquilinearform auf dem
    $\C^n$, also $s(x,y) = x^T A \overline{y}$. Dann gilt:
    \begin{enumerate}[{(1)}]
        \item Es gibt eine Orthonormalbasis $\B$ des $\C^n$ (bzgl. des kanonischen Skalarproduktes),
                so dass
                \begin{align*}
                    M_{\B} (s) = \begin{pmatrix}
                        \lambda_1 & 0 & \dots & 0 \\
                        0 & \lambda_2 & \ddots & \vdots \\
                        \vdots & \ddots & \ddots & 0 \\
                        0 & \dots & 0 & \lambda_n
                    \end{pmatrix}
                    =: D
                \end{align*}
                diagonal ist, mit $\lambda_1,\dots,\lambda_n \in \R$ den Eigenwerten von $A$.
                Äquivalent in Matrixform: Es gibt eine unitäre Matrix $U \in U(n)$ so dass
                $U^T A \overline{U}$.
        \item Es gibt eine Basis $\B'$ des $\C^n$ und Zahlen $k,l$, so dass
                    \begin{align*}
                        M_{\B'} (s) = \begin{pmatrix}
                            E_k & 0 & 0 \\
                            0 & - E_l & 0 \\
                            0 & 0 & 0
                        \end{pmatrix}
                    \end{align*}
                    Äquivalent, in Matrixform: Es gibt eine invertierbare Matrix $S \in \GL(n,\C)$
                    so dass $S^T A \overline{S} = D'$.
        \item Die Zahlen $k,l$ aus $(2)$ sind die Anzahl der positiven bzw negativen Eigenwerte von
                $A$, mit Vielfachheit gezählt, und unabhängig von der Wahl von $\B'$. Alternativ,
                in Matrixform: Für jede invertierbare Matrix $S \in \GL(n,\C)$ sodass $S^T A \overline{S}$
                die diagonal ist, befinden sich genau $k$ positive und genau $l$ negative Enträge auf der
                Diagonalen von $S^T A \overline{S}$.
    \end{enumerate}
}

\vspace{1\baselineskip}

\Theorem{

    Sei $A \in M(n \times n , \C)$ eine hermitische Matrix. Dann sind äquivalent:
    \begin{enumerate}[{(1)}]
        \item $A$ ist positiv definit, also $x^T A \overline{x} > 0$ für alle $0 \neq x \in \C^n$
        \item Alle Eigenwerte von $A$ sind positiv.
        \item Es gibt eine Matrix $S \in \GL(n,\C)$, so dass $S^T A \overline{S}$ diagonal ist mit
            positiven Diagonaleinträgen.
        \item Es gibt eine Matrix $T \in \GL(n,\C)$ so dass $A = T^{\dagger} T$
        \item Sei $P_A (t) = (-t)^n + \alpha_{n-1} t^{n-1} + \dots + \alpha_1 t + \alpha_0 \in \R [t]$
            das charakteristische Polynom von $A$. Dann gilt $(-1)^j \alpha_j > 0$ für $j=0,\dots,n-1$
        \item Sei $A_k \in M(k \times k , \C)$ die $k \times k$ Untermatrix von $A$ "oben links".
            Dann gilt $\det A_k > 0$ für $k=1,\dots,n$
    \end{enumerate}
}

\large \fat{Algorithmus zur Bestimmung von $S$ und $D$ im komplexen Fall} \normalsize

\vspace{1\baselineskip}

Analog zum reellen Fall, nur, dass man bei der Zeilentransformation von $A$ jeweils
die Koeffizienten komplex konjugieren muss.

\vspace{1\baselineskip}

\Definition{

    Man nennt eine Bilinearform bzw. Sesquilinearform $s$ auf dem $\K$-VR $V$ \fat{positiv
    semidefinit}, falls $s(v,v) \geq 0 \ \forall v \in V$. Man nennt $s$ \fat{negativ definit},
    falls $s(v,v) < 0 \ \forall v \neq 0 \in V$ und negativ semidefinit, falls $s(v,v) \leq 0
    \forall v \in V$. Eine symmetrische bzw. hermitische Matrix $A$ ist positiv oder negativ
    (semi)definit, falls entsprechendes für die zugehörige Bilinearform $s(x,y) =
    x^T A \overline{y}$ auf $\K^n$ gilt. Es gilt:
    \begin{enumerate}[{(1)}]
        \item $A$ negativ definit $\Leftrightarrow$ $-A$ positiv definit
        \item $A$ negativ semidefinit $\Leftrightarrow$ $-A$ positiv semidefinit
    \end{enumerate}
}
