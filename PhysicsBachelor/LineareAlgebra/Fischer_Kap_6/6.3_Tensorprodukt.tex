\section{Tensorprodukt}

\vspace{1\baselineskip}

\Bemerkung{

    Seien $V$ bzw. $W$ Vektorräume über $K$ mit Basen $(v_i)_{i \in I}$ bzw. $(w_j)_{j \in J}$.
    Ist $U$ ein weiterer $K$-VR, so gibt es zu einer beliebigen vorgegebenen Familie
    $(u_{ij})_{(i,j) \in I \times J}$ in $U$ genau eine bilineare Abbildung
    $\xi: V \times W \rightarrow U$ \ mit \ $\xi (v_i , w_j) = u_{ij}$ \  $\forall (i,j) \in I \times J$
}

\vspace{1\baselineskip}

\Theorem{}{

    Seien $V$ und $W$ Vektorräume über $K$. Dann gibt es einen $K$-VR \
    $V \otimes_K W$ zusammen mit einer bilinearen Abbildung \
    $\eta : V \times W \rightarrow V \times_K W$ \ die folgende universelle Eigenschaft
    haben: Zu jedem $K$-VR $U$ zusammen mit einer bilinearen Abbildung
    $\xi : V \times W \rightarrow U$ gibt es genau eine lineare Abbildung
    $\xi_{\otimes} : V \otimes_K W \rightarrow U$ mit $\xi = \xi_{\otimes} \circ \eta$.
    Das kann man durch ein kommutatives Diagramm illustrieren:
    \begin{center}
        \begin{tikzcd}
            V \times W \arrow{d}[swap]{\eta} \arrow{rd}{\xi} & \\
            V \otimes_K W \arrow{r}[swap]{\xi_{\otimes}} & U
        \end{tikzcd}
    \end{center}
    Weiter gilt: Falls $\dim V , \dim W < \infty$, so ist
    
    $\dim (V \otimes_K W) = (\dim V) \cdot (\dim W)$.

    Falls klar ist, welches $K$ Grundkörper ist, schreibt man nur $\otimes$ statt $\otimes_K$.
    Man nennt $V \otimes_K W$ das \fat{Tensorprodukt} von $V$ und $W$ über $K$. Die Elemente
    von $V \otimes_K W$ heissen \fat{Tensoren}.
}

\vspace{1\baselineskip}

\Korollar{ \fat{(Rechenregeln für Tensoren)}

    Ist $\eta: V \times W \rightarrow V \otimes W$ und $v \otimes w := \eta (v,w)$, so gilt
    für $v,v' \in V$, $w,w' \in W$ und $\lambda \in K$:
    \begin{enumerate}[{a)}]
        \item $v \otimes w + v' \otimes w = (v+v') \otimes w$ und $v \otimes w + v \otimes w' = v \otimes (w + w')$
        \item $(\lambda \cdot v) \otimes w = v \otimes (\lambda \cdot w) = \lambda \cdot (v \otimes w)$
    \end{enumerate}
}

\vspace{1\baselineskip}

\Bemerkung{}{

    Der VR $V \otimes W$ ist durch die universelle Eigenschaft eindeutig bis auf einen
    eindeutigen Isomorphismus bestimmt. Sei $V \tilde{\otimes} W$ und $\tilde{\eta}:
    V \times W \rightarrow V \tilde{\otimes} W$ ein anderes Paar aus $VR$ und bilinearen
    Abbildungen, das die universelle Eigenschaft erfüllt. Dann gibt es Abbildungen:
    \begin{center}
        \begin{tikzcd}
            & V \times W \arrow{ld}[swap]{\tilde{\eta}} \arrow{rd}{\eta} & \\
            V \tilde{\otimes} W \arrow[rr, shift right, swap, "\exists ! \tilde{\eta}_{\otimes}"] & &
            V \otimes W \arrow[ll, shift right, swap,"\exists ! \eta_{\otimes}"]
        \end{tikzcd}
    \end{center}
    Betrachte:
    \begin{center}
        \begin{tikzcd}
            & V \times W \arrow{ld}[swap]{\eta} \arrow{rd}{\eta} & \\
            V \otimes W \arrow[rr, shift left, "\text{id}"] \arrow[rr, shift right, swap, "\eta_{\otimes} \circ \tilde{\eta}_{\otimes}"] & &
            V \otimes W 
        \end{tikzcd}
    \end{center}
    Es ist $\eta_{\otimes} \circ \tilde{\eta}_{\otimes} \circ \eta = \eta_{\otimes} \circ \tilde{\eta} = \eta$.
    Wegen der Eindeutigkeit der universellen Eigenschaft gilt also
    $\eta_{\otimes} \circ \tilde{\eta}_{\otimes} = \id_{V \tilde{\otimes} W}$.
    Man spricht deswegen normalerweise von \underline{dem} Tensorprodukt $V \otimes W$
    von $V$ und $W$.
}

\vspace{1\baselineskip}

\Definition{

    Sei $s: V \times W \rightarrow K$ eine Bilinearform. Schreibe $\Bil (V,W;K)$ für den
    VR solcher bilinearer Abbildungen, also $s \in \Bil (V,W;K)$. Wir erhalten die
    lineare Abbildung $s_{\otimes} : V \otimes W \rightarrow K$ mit $s(v,w) = s(v \otimes w)$.
    Also $s_{\otimes} \in (V \otimes W)^{*}$. Die Abbildung $\Bil (V,W;K) \rightarrow (V \otimes W)^{*}$
    mit $s \mapsto s_{\otimes}$ ist wieder linear. Sei umgekehrt $\phi \in (V \otimes W)^{*}$,
    dann ist $\phi \circ \eta \in \Bil (V,W;K)$ mit $\eta: V \times W \rightarrow V \otimes W$
    bilinear. Die Abbildungen $\Bil (V,W;K) \rightarrow (V \otimes W)^{*}$ mit $s \mapsto s_{\otimes}$
    und $(V \otimes W)^{*} \rightarrow \Bil (V,W;K)$ mit $\phi \mapsto \phi \circ \eta$ sind
    zueinander invers.
}

\vspace{1\baselineskip}

\Bemerkung{

    Für zwei Linearformen $\phi \in V^{*}$ und $\psi \in W^{*}$ ist das Produkt
    \begin{align*}
        \phi \cdot \psi : V \times W \rightarrow K \quad \text{mit} \quad (v,w) \mapsto \phi (v) \cdot \psi (v)
    \end{align*}
    eine Bilinearform, die zugehörige Abbildung
    \begin{align*}
        (\phi \cdot \psi)_{\otimes} : V \otimes W \rightarrow K \quad \text{ mit } \quad
        v \otimes w \mapsto \phi(v) \cdot \psi (w)
    \end{align*}
    ist linear, also ist $(\phi \cdot \psi)_{\otimes} \in (V \otimes W)^{*}$.
    Die so entstandene Abbildung
    \begin{align*}
        V^{*} \times W^{*} \rightarrow (V \otimes W)^{*} \quad \text{ mit } \quad
        (\phi,\psi) \mapsto (\phi \cdot \psi)_{\otimes}
    \end{align*}
    ist wiederum bilinear, dh. sie wird zu einer linearen
    Abbildung
    \begin{align*}
        V^{*} \otimes W^{*} \rightarrow (V \otimes W)^{*} \quad \text{ mit } \quad
        \phi \otimes \psi \mapsto (\phi \cdot \psi)_{\otimes}
    \end{align*}
    Das $\phi \otimes \psi$ eine Linearform auf $V \otimes W$ erklärt, kann man auch sehen:
    \begin{align*}
        (\phi \otimes \psi)(v \otimes w) := \phi (v) \cdot \psi (w)
    \end{align*}
}

\vspace{1\baselineskip}

\Bemerkung{

    Die durch Multiplikation mit Skalaren in $W$ erhaltene Abbildung
    \begin{align*}
        V^{*} \times W \rightarrow \Hom (V,W) \quad &\text{mit} \quad
        (\phi , w) \mapsto \phi ( \ ) \cdot w
        \\
        \text{wobei} \quad \phi ( \ ) \cdot w : V \rightarrow W \quad &\text{mit} \quad
        v \mapsto \phi (v) \cdot w
    \end{align*}
    ist bilinear, dazu gehört die lineare Abbildung
    \begin{align*}
        V^{*} \otimes W \rightarrow \Hom (V,W) \quad \text{ mit } \quad
        \phi \otimes w \mapsto \phi ( \ ) \cdot w
    \end{align*}
    Das kann man wieder kürzer ausdrücken durch
    \begin{align*}
        (\phi \otimes w)(v) := \phi (v) \cdot w
    \end{align*}
}

\vspace{1\baselineskip}

\Satz{

    Sind $V$ und $W$ endlich-dimensionale VR, so sind die beiden kanonischen Abbildungen
    \begin{align*}
        &\alpha: V^{*} \otimes W^{*} \rightarrow (V \otimes W)^{*}
        \quad \text{und} \quad \\
        &\beta: V^{*} \otimes W^{*} \rightarrow \Hom (V,W)
    \end{align*}
    Isomorphismen.
}

\vspace{1\baselineskip}

\Satz{

    $V^{*} \otimes W^{*} \cong (V \otimes W)^{*}$ und $(V \otimes W)^{*} \cong \Bil (V,W;K)$
}

\vspace{1\baselineskip}

\Korollar{

    Sei $u \in K^m \otimes K^n \cong J^{m \cdot n}$, also
    $u = \sum_{i=1}^{m} \sum_{j=1}^{n} u_{ij} e_i \otimes e_j$ mit $e_i \in K^m$ und
    $e_j \in K^n$ Basisvektoren. Die Koordinaten $u_{ij}$ kann man auf drei Weisen
    aufschreiben:
    \begin{enumerate}[{(i)}]
        \item Als Matrix $(u_{ij})_{ij} \in \Mat (m \times n ; K)$
        \item Als $m \cdot n$ Vektor wobei man die Basen von $K^m \otimes K^n$ wie folgt
                ordnet: $e_1 \otimes e_1 , e_2 \otimes e_1 ,\dots, e_m \otimes e_1,
                e_1 \otimes e_2 ,\dots, e_m \otimes e_2,\dots,e_1 \otimes e_n ,\dots,
                e_m \otimes e_n$.
        \item Als $m \cdot n$ Vektor wobei man die Basen von $K^m \otimes K^n$ wie folgt
                ordnet: $e_1 \otimes e_1 , e_1 \otimes e_2 ,\dots, e_1 \otimes e_n,
                e_2 \otimes e_1,\dots, e_2 \otimes e_1,\dots,e_m \otimes e_1 ,\dots,
                e_m \otimes e_n$.
    \end{enumerate}
}

\vspace{1\baselineskip}

\Definition{

    Sind $V$ und $W$ $K$-VR, so heisst eine bilineare Abbildung
    \begin{align*}
        \xi: V \times V \rightarrow W
    \end{align*}
    \fat{symmetrisch}, wenn $\xi (v,v') = \xi (v' ,v)$ für alle $v,v' \in V$
    \fat{alternierend}, wenn $\xi (v,v) = 0$ für alle $v \in V$
}

\vspace{1\baselineskip}

\Bemerkung{

    Ist $\xi$ alternierend, so gilt $\xi (v',v) = - \xi (v,v') \ \forall v,v' \in V$.
    Im Fall char$(K) \neq 2$ gilt auch die Umkehrung (also
    alternierend $\Leftrightarrow$ schiefsymmetrisch).
}

\vspace{1\baselineskip}

\Definition{

    $S(V) := \text{span} (v \otimes v' - v' \otimes v)_{v,v' \in V} \subset V \otimes V$
    \ \ und

    $A(V) := \text{span} (v \otimes v)_{v \in V} \subset V \otimes V$
}

\vspace{1\baselineskip}

\Lemma{

    Für jedes bilineare $\xi : V \times V \rightarrow W$ gilt:

    $\xi$ symmetrisch $\Leftrightarrow \ S(V) \subset \ker \xi_{\otimes}$

    $\xi$ alternierend $\Leftrightarrow \ A(V) \subset \ker \xi_{\otimes}$
}

\vspace{1\baselineskip}

\Theorem{}{

    Für jeden $K$-VR $V$ mit $\dim V \geq 2$ gibt es einen $K$-VR $V \wedge V$ zusammen mit
    einer alternierenden Abbildung $\wedge : V \times V \rightarrow V \wedge V$,
    die folgende universelle Eigenschaft haben: zu jedem $K$-VR $W$ zusammen mit einer
    alternierenden Abbildung $\xi : V \times V \rightarrow W$ gibt es genau eine lineare
    Abbildung $\xi_{\wedge}$ derart, dass das Diagramm
    \begin{center}
        \begin{tikzcd}
            V \times V \arrow{d}[swap]{\wedge} \arrow{rd}{\xi} & \\
            V \wedge V \arrow{r}[swap]{\exists ! \ \xi_{\wedge}} & W
        \end{tikzcd}
    \end{center}
    kommutiert. Ist $(\BasisV)$ eine Basis von $V$, so ist durch $v_i \wedge v_j :=
    \wedge(v_i,v_j)$ mit $1 \leq i < j \leq n$ eine Basis von $V \wedge V$ gegeben.
    Insbesondere ist
    \begin{align*}
        \dim (V \wedge V) = \binom{n}{2} = \frac{n (n-1)}{2}
    \end{align*}
}

\vspace{1\baselineskip}

\Korollar{ (Rechenregeln für das äussere Produkt)

    Für $v,v',w,w' \in V$ und $\lambda \in K$ gilt:
    \begin{enumerate}[{a)}]
        \item $(v+v') \wedge w = v \wedge w + v' \wedge w$ und $v \wedge (w+w') = v \wedge w + v \wedge w'$
        \item $(\lambda v) \wedge w = v \wedge (\lambda w) = \lambda (v \wedge w)$
        \item $v \wedge v = 0$ und $v' \wedge v = - v \wedge v'$ 
    \end{enumerate}
}

\vspace{1\baselineskip}

\Theorem{}{

    Zu jedem $K$-VR $V$ gibt es einen $K$-VR $V \vee V$ und eine symmetrische bilineare
    Abbildung $\vee : V \times V \rightarrow V \vee V$ mit folgender universellen
    Eigenschaft: Sei $\xi: V \times V \rightarrow W$ symmetrisch (und bilineare), dann
    gibt es genau eine lineare Abbildung $\xi_{\vee} \rightarrow W$, sodass
    \begin{center}
        \begin{tikzcd}
            V \times V \arrow{d}[swap]{\vee} \arrow{rd}{\xi} & \\
            V \vee V \arrow{r}[swap]{\exists ! \ \xi_{\vee}} & W
        \end{tikzcd}
    \end{center}
    kommutiert. Ist $(\BasisV)$ eine Basis von $V$, so bilden die Vektoren
    $v_i \vee v_j = \vee (v_i , v_j)$ mit $1 \leq i \leq j \leq n$ eine Basis von
    $V \vee V$. Insbesondere ist
    \begin{align*}
        \dim (V \vee V) = \binom{n+1}{2} = \frac{n(n+1)}{2}
    \end{align*}
    $V \vee V$ heisst 2. symmetrische Potenz von $V$ (Symmetrischer Produktraum).
}

\vspace{1\baselineskip}

\Proposition{}{

    Sei nun $\Bil (V;K) = \Bil (V,V;K)$ der Raum der bilinearen Abbildungen
    $V \times V \rightarrow K$ und $\text{Alt}^2 (V,K) \subset \Bil (V;K)$ der UVR der
    alternierenden Bilinearformen und $\text{Sym}^2 (V;K) \subset \Bil (V;K)$ der UVR der
    symmetrischen Bilinearformen. Dann haben wir folgende Abbildungen:
    \begin{center}
        \begin{tikzcd}
            V^{*} \wedge V^{*} \arrow{r}{(1a)} & \text{Alt}^2 (V;K) \arrow{r}{(2a)} \arrow{d} & (V \wedge V)^{*} \arrow{d} \\
            V^{*} \otimes V^{*} \arrow{u} \arrow{d} \arrow{r}{(1)} & \Bil (V;K) \arrow{r}{\stackrel{(2)}{\cong}} & (V \otimes V)^{*} \\
            V^{*} \vee V^{*} \arrow{r}{(1s)} & \text{Sym}^2 (V;K) \arrow{u} \arrow{r}{\stackrel{(2s)}{\cong}} & (V \vee V)^{*} \arrow{u}
        \end{tikzcd}
    \end{center}
    Die Abbildungen $(2a)$, $(2s)$ und $(2)$ erhalten wir aus der universellen Eigenschaften.
    Die Abbildung $(1a)$ ist definiert durch
    \begin{align*}
        V^{*} \wedge V^{*} &\rightarrow \text{Alt}^2 (V;K) \\
        (\phi , \psi) &\mapsto \klammer{(v,w) \mapsto \phi (v) \psi (w) - \phi (w) \psi (v)}
    \end{align*}
    Die Abbildung $(1s)$ ist
    \begin{align*}
        V^{*} \vee V^{*} &\rightarrow \text{Sym}^2 (V;K) \\
        (\phi , \psi) &\mapsto \klammer{(v,w) \mapsto \phi(v) \psi(w) + \phi(w) \psi(v)}
    \end{align*}
}

\Satz{

    Ist $\dim V < \infty$, so ist $(1a)$ ein Isomorphismus, und falls zusätlich
    $\text{char} (K) \neq 2$, so ist $(1s)$ auch ein Isomorphismus.
}

\vspace{1\baselineskip}

\Definition{

    Seien $V,V',W,W'$ $K$-VR und $F: V \rightarrow V'$, $G: W \rightarrow W'$ lineare
    Abbildungen. Dann ist das Tensorprodukt der Abbildungen $F$ und $G$ die eindeutig
    definierte lineare Abbildung
    \begin{align*}
        F \otimes G : V \otimes W \rightarrow V' \otimes W'
    \end{align*}
    so dass $(F \otimes G)(v \otimes w) = F(v) \otimes G(w) \ \forall v \in V, \ w \in W$.
    Etwas ausführlicher sind die linearen Abbildungen $V \otimes W \rightarrow V' \otimes W'$
    nach der universellen Eigenschaft für $V \otimes W$ bijektiv zu den linearen Abbildungen
    $V \times W \rightarrow V' \otimes W'$. Die bilineare Abbildung $(v,w) \mapsto F(v) \otimes
    G(w)$ entspricht dabei der linearen Abbildung $F \otimes G$.
}

\vspace{1\baselineskip}

\Satz{

    Für $V,V',W,W'$ endlich dimensionale $K$-VR ist das Tensorprodukt von Abbildungen
    \begin{align*}
            &\otimes : \Hom_K (V,V') \otimes \Hom_K (W,W') \\
            &\hspace{60pt} \rightarrow \Hom_K (V \otimes W , V' \otimes W')
    \end{align*}
    ein Vektorraumisomorphismus.
}

\vspace{1\baselineskip}

\Bemerkung{

    Wir wollen uns nun anschauen, wie die Matrix von $F \otimes G$ bezüglich einer Basis
    aussieht. Sei $\operatorname{dazu} \mathcal{A}=\left(v_{1}, \ldots, v_{m}\right)$ eine
    Basis von $V, \mathcal{A}^{\prime}=\left(v_{1}^{\prime}, \ldots, v_{m^{\prime}}^{\prime}\right)$
    eine Basis von $V^{\prime}, \mathcal{B}=\left(w_{1}, \ldots, w_{n}\right)$ eine Basis
    von $W$ und $\mathcal{B}^{\prime}=\left(w_{1}^{\prime}, \ldots, w_{n^{\prime}}^{\prime}\right)$
    eine Basis von $W^{\prime} .$ Seien $A=\left(a_{i j}\right)=M_{\mathcal{A}^{\prime}}^{\mathcal{A}}(F)$
    und $B=\left(b_{i j}\right)=M_{\mathcal{B}^{\prime}}^{\mathcal{B}}(G)$ die entsprechenden
    Matrizen von $F$ und $G .$ Eine Basis von $V \otimes W$ ist gegeben durch die Familie
    $\mathcal{A} \times \mathcal{B}:=\left(v_{i} \otimes w_{j}\right)_{i=1, \ldots, n}$. Um
    die Koordinaten von Vektoren bezüglich dieser Basis als Spaltenvektoren schreiben zu können,
    und entsprechend auch Matrizen aufschreiben zu können, müssen wir die Elemente dieser Basis
    noch ordnen. Genauer: Wir hatten die Matrix einer Abbildung definiert bezüglich Basen,
    deren Indexmenge $1,2 \ldots$ war. Wir müssen also die Indexmenge in der Familie
    $\mathcal{A} \times \mathcal{B}$ noch von $\{1, \ldots, m\} \times$ $\{1, \ldots, n\}$
    auf die Indexmenge $\{1, \ldots, m n\}$ abändern. Es gibt hierfür zwei natürliche Möglichkeiten:
    \begin{itemize}
        \item Wir können die Ordnung
                \[
                \mathcal{C}_{1}:=\left(v_{1} \otimes w_{1}, v_{2} \otimes w_{1}, \ldots, v_{m} \otimes w_{1}, v_{1} \otimes w_{2}, \ldots, v_{m} \otimes w_{n}\right)
                \]
                verwenden, und entsprechend die Ordnung
                \[
                \mathcal{C}_{1}^{\prime}:=\left(v_{1}^{\prime} \otimes w_{1}^{\prime}, v_{2}^{\prime} \otimes w_{1}^{\prime}, \ldots, v_{m^{\prime}}^{\prime} \otimes w_{1}^{\prime}, v_{1}^{\prime} \otimes w_{2}^{\prime}, \ldots, v_{m^{\prime}}^{\prime} \otimes w_{n^{\prime}}^{\prime}\right)
                \]
                für die entsprechende Basis von $V^{\prime} \otimes W^{\prime}$
        \item Wir können die Ordnung
                \[
                \mathcal{C}_{2}:=\left(v_{1} \otimes w_{1}, v_{1} \otimes w_{2}, \ldots, v_{1} \otimes w_{n}, v_{2} \otimes w_{1}, \ldots, v_{m} \otimes w_{n}\right)
                \]
                verwenden, und entsprechend die Ordnung
                \[
                \mathcal{C}_{2}^{\prime}:=\left(v_{1}^{\prime} \otimes w_{1}^{\prime}, v_{1}^{\prime} \otimes w_{2}^{\prime}, \ldots, v_{1}^{\prime} \otimes w_{n^{\prime}}^{\prime}, v_{2}^{\prime} \otimes w_{1}^{\prime}, \ldots, v_{m^{\prime}}^{\prime} \otimes w_{n^{\prime}}^{\prime}\right)
                \]
                für die entsprechende Basis von $V^{\prime} \otimes W^{\prime}$
    \end{itemize}

    \vspace{1\baselineskip}

    In jedem Falle ist
    \[
    (F \otimes G)\left(v_{i} \otimes w_{j}\right) = F(v_i) \otimes G(w_j) =\sum_{i^{\prime}=1}^{m^{\prime}} \sum_{j^{\prime}=1}^{n^{\prime}} a_{i^{\prime} i} b_{j^{\prime} j} v_{i^{\prime}}^{\prime} \otimes w_{j^{\prime}}^{\prime}
    \]
    Übersetzt in Matrizenform heisst dies: \\
    \[
    \begin{array}{l}
    M_{C_{1}^{\prime}}^{C_{1}}(F \otimes G)=\left(\begin{array}{cccc}
    A b_{11} & A b_{12} & \cdots & A b_{1 n} \\
    A b_{21} & A b_{22} & \cdots & A b_{2 n} \\
    \vdots & \vdots & \ddots & \vdots \\
    A b_{n^{\prime} 1} & A b_{n^{\prime} 2} & \cdots & A b_{n^{\prime} n}
    \end{array}\right) \\  \in M\left(m^{\prime} n^{\prime} \times m n, K\right) \\ \\
    M_{C_{2}^{\prime}}^{C_{2}}(F \otimes G)=\left(\begin{array}{cccc}
    a_{11} B & a_{12} B & \cdots & a_{1 m} B \\
    a_{21} B & a_{22} B & \cdots & a_{2 m} B \\
    \vdots & \vdots & \ddots & \vdots \\
    a_{m^{\prime} 1} B & a_{m^{\prime 2} 2} B & \cdots & a_{m^{\prime} m} B
    \end{array}\right) \\ \in M\left(m^{\prime} n^{\prime} \times m n, K\right)
    \end{array}
    \]
}

\vspace{1\baselineskip}

\Definition{

    Sei $G$ eine Gruppe und $V$ ein $K$-VR. Dann ist eine Darstellung von $G$ auf $V$ ein
    Gruppenhomomorphismus $\rho : G \rightarrow \GL (V)$ wobei $\GL (V)$ die Gruppe der
    Vektorraumisomorphismen von $V$ ist. Man sagt auch \textit{$G$ wirkt auf $V$}.
}

\vspace{1\baselineskip}

\Satz{

    Sei $G$ eine endliche Gruppe und $\rho$ eine Darstellung von $G$ auf dem $K$-VR $V$.
    Nehme an, dass char$(K)$ die Gruppenordnung $\abs{G}$ nicht teilt, also dass
    $\abs{G} \neq 0$ in $K$ ist. Dann ist die kanonische Abbildung $V^G \rightarrow V_G$
    ein Isomorphismus,, und die inverse Abbildung ist gegeben durch die
    Symmetrisierung
    \begin{align*}
        \Sigma_G : V_G &\rightarrow V^G \\
        v + U &\mapsto \frac{1}{\abs{G}} \sum_{g \in G} \rho (g) (v)
    \end{align*}
}

\vspace{1\baselineskip}

\Bemerkung{

    Wir wenden dies nun an auf das symmetrische und äussere Produkt. Sei dazu immer
    vereinfachend $K$ ein Körper der Charakteristik $0,$ so dass die entsprechende
    Voraussetzung im Satz immer erfült ist, und so dass wir alternierende Bilinearformen
    mit schiefsymmetrischen identifizieren können. Das symmetrische Produkt haben wir als die
    Koinvarianten definiert
        \[
        \vee^{n} V=\left(\otimes^{n} V\right)_{S_{n}} = \nicefrac{\otimes^n V}{S^n (v)}
        \]
    wobei hier die Wirkung der symmetrischen Gruppe $S_{n}$, ohne Vorzeichen,
    zu Grunde gelegt ist. Das äussere Produkt kann man analog definieren als die Koinvarianten
        \[
        \wedge^{n} V=\left(\otimes^{n} V\right)_{S_{n}}
        \]
    wobei nun aber die Darstellung mit Vorzeichen verwendet wird. (Man beachte, dass die
    Darstellung in der Notation unterschlagen wird, und dass hier char $(K)=0$ angenommen ist.
    Je nach Kontext und Anwendung wird das symmetrische bzw. äussere Produkt manchmal auch
    als Invarianten definiert. Dies ist gerechtfertigt durch den obigen Satz, der insbesondere
    besagt, dass (Achtung: in Charakteristik 0) Invarianten und Koinvarianten natürlich
    identifiziert werden können,
        \[
        \left(\otimes^{n} V\right)_{S_{n}} \cong\left(\otimes^{n} V\right)^{S_{n}}
        \]
    Der Raum der Bilinearformen auf dem endlich dimensionalen
    Vektorraum $V$ sind z.B. identifiziert mit
    $(V \otimes V)^{*} \cong V^{*} \otimes V^{*} .$ Die symmetrischen Bilinearformen sind
    gerade die, die invariant sind unter der $S_{2}$ -Wirkung (ohne Vorzeichen)
        \[
        \mathrm{Sym}^{2}(V ; K) \cong\left(V^{*} \otimes V^{*}\right)^{S_{2}} \subset \mathrm{Bil}(V ; K)
        \]
    Wir haben gesehen, dass man dies identifizieren kann mit
    $V^{*} \vee V^{*}=\left(V^{*} \otimes V^{*}\right)_{S_{2}}$ falls char $(K) \neq 2$
    Analoges gilt für das äussere Produkt, wobei man dabei in diesem Fall noch die
    Beschränkung an die Charakteristik fallen lassen konnte.
}
