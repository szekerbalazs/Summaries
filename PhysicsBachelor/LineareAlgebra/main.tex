\documentclass[a4paper,leqno,twocolumn]{article}
% ==== Inputs and Usepackages ====

\usepackage{tablefootnote}
\usepackage{enumerate}
\usepackage{float}
\usepackage{url}
\usepackage{hyperref}
\usepackage{dsfont}
\usepackage{mathrsfs}
\usepackage{amsmath}
\usepackage{amssymb}
\usepackage{amsthm}
\usepackage{amsfonts}
\usepackage{mathtools}
%\usepackage{mathabx}
\usepackage{MnSymbol}
\usepackage{xfrac}
\usepackage{nicefrac}
\usepackage{geometry}
\usepackage{graphicx}
\usepackage{graphics}
\usepackage{latexsym}
\usepackage{setspace}
\usepackage{tikz-cd}
\usepackage{tikz}
 \usetikzlibrary{matrix}
 \usetikzlibrary{calc}
 \usetikzlibrary{circuits.ee.IEC}
\usepackage{circuitikz}

\usepackage{a4wide}
\usepackage{fancybox}
\usepackage{fancyhdr}
\usepackage[utf8]{inputenc}




% ==== Page Settings ====

\hoffset = -1.2 in
\voffset = -0.3 in
\textwidth = 590pt
\textheight = 770pt
\setlength{\headheight}{20pt}
\setlength{\headwidth}{590pt}
\marginparwidth = 0 pt
\topmargin = -0.75 in
\setlength{\parindent}{0cm}


% ==== Presettings for files ====

\pagestyle{fancy}


\cfoot{\thepage}
\lfoot{\href{mailto:szekerb@student.ethz.ch}{szekerb@student.ethz.ch}}
\rfoot{Balázs Szekér, \today}
\lhead{Physics \uproman{3} Summary}

% ==== General Commands ====

\newcommand{\sframebox}[1]{
        \framebox[500pt][l]{\parbox{490pt}{
            #1
        }
    }

}

\newcommand{\cframebox}[2]{
        \fbox{\parbox{#1pt}{
            #2
        }
    }
}









% ====== Maths ======


% ==== Formats ====
\newcommand{\boldline}[1]{\textbf{\underline{#1}}}
\newcommand{\uproman}[1]{\uppercase\expandafter{\romannumeral#1}}
\newcommand{\lowroman}[1]{\romannumeral#1\relax}
\newcommand{\fat}[1]{\textbf{#1}}
\newcommand{\Loesung}{\begin{center}\textbf{Lösung}\end{center}}

\newcommand{\Korollar}[1]{\textbf{Korollar} \vspace{1\baselineskip} #1}
\newcommand{\Beispiel}[1]{\textbf{Beispiel} \vspace{1\baselineskip} #1}
\newcommand{\Beweis}[1]{\textbf{Beweis} \vspace{1\baselineskip} #1}
\newcommand{\Proposition}[1]{\textbf{Proposition} \vspace{1\baselineskip} #1}
\newcommand{\Satz}[1]{\textbf{Satz} \vspace{1\baselineskip} #1}
\newcommand{\Definition}[1]{\textbf{Definition} \vspace{1\baselineskip} #1}
\newcommand{\Lemma}[1]{\textbf{Lemma}\vspace{1\baselineskip} #1}
\newcommand{\Bemerkung}[1]{\textbf{Bemerkung} \vspace{1\baselineskip} #1}
\newcommand{\Theorem}[1]{\textbf{Theorem} \vspace{1\baselineskip} #1}




% ==== mathsymbols ====
\newcommand{\Q}{\mathbb{Q}}
\newcommand{\R}{\mathbb{R}}
\newcommand{\N}{\mathbb{N}}
\newcommand{\Z}{\mathbb{Z}}
\newcommand{\C}{\mathbb{C}}
\newcommand{\K}{\mathbb{K}}
\newcommand{\eS}{\mathbb{S}}
\newcommand{\X}{$X$ }
\newcommand{\Y}{$Y$ }
\newcommand{\x}{$x$ }
\newcommand{\y}{$y$ }
\newcommand{\B}{\mathcal{B}}
\newcommand{\A}{\mathcal{A}}
\renewcommand{\S}{\mathcal{S}}
\renewcommand{\P}{\mathcal{P}}



% ==== math operators ====
\newcommand{\klammer}[1]{\left( #1 \right)} 
\newcommand{\eckigeklammer}[1]{\left[ #1 \right]}
\newcommand{\geschwungeneklammer}[1]{\left\{ #1 \right\}}
\newcommand{\floor}[1]{\left\lfloor #1 \right\rfloor}
\newcommand{\ceil}[1]{\left\lceil #1 \right\rceil}
\newcommand{\scalprod}[2]{\left\langle #1 , #2 \right\rangle}
\newcommand{\abs}[1]{\left\vert #1 \right\vert} 
\newcommand{\Norm}[1]{\left\vert\left\vert #1 \right\vert\right\vert}
\newcommand{\intab}{\int_a^b}
\newcommand{\intii}{\int_{-\infty}^\infty} 
\newcommand{\cint}[2]{\int_{#1}^{#2}}
\newcommand{\csum}[2]{\sum_{#1}^{#2}}
\newcommand{\limes}[1]{\lim\limits_{#1}}
\newcommand{\limessup}[1]{\limsup\limits_{#1}}
\newcommand{\limesinf}[1]{\liminf\limits_{#1}}
\newcommand{\limesninf}{\limes{n \rightarrow \infty}}
\newcommand{\limsupninf}{\limessup{n \rightarrow \infty}}
\newcommand{\liminfninf}{\limesinf{n \rightarrow \infty}}
\newcommand{\standardNorm}{\Norm{ \ \cdot \ }}
\newcommand{\einsNorm}{\Norm{ \ \cdot \ }_1}
\newcommand{\zweiNorm}{\Norm{ \ \cdot \ }_2}
\newcommand{\Hom}{\text{Hom}}
\newcommand{\Mat}{\text{Mat}}
\newcommand{\grad}{\text{grad}}
\newcommand{\vol}{\text{vol}}
\newcommand{\supp}{\text{supp}}
\newcommand{\rot}{\text{rot}}
\renewcommand{\div}{\text{div}}



% ==== Analysis ====
\newcommand{\supremum}{\text{sup}}
\newcommand{\infimum}{\text{inf}}
\newcommand{\maximum}{\text{max}}
\newcommand{\minimum}{\text{min}}

\newcommand{\xinX}{$x \in X$ }
\newcommand{\yinY}{$y \in Y$ }
\newcommand{\xyinX}{$x,y \in X$ }
\newcommand{\yinX}{$y \in X$ }
\newcommand{\xinR}{$x \in \R$ }
\newcommand{\xyinR}{$x,y \in \R$ }
\newcommand{\zinC}{$z \in \C$ }
\newcommand{\ninN}{$n \in \N$ }
\newcommand{\NinN}{$N \in \N$}
\newcommand{\angeordneterK}{$(K,\leq)$ }
\newcommand{\xFolge}{(x_n)_{n=0}^{\infty}}
\newcommand{\yFolge}{(y_n)_{n=0}^{\infty}}
\newcommand{\zFolge}{(z_n)_{n=0}^{\infty}}
\newcommand{\aFolge}{(a_n)_{n=0}^{\infty}}
\newcommand{\fFolge}{(f_n)_{n=0}^{\infty}}

\newcommand{\XsubR}{$X \subset \R$ }
\newcommand{\XsubeqR}{$X \subseteq \R$ }

\newcommand{\XFam}{\mathcal{X}}
\newcommand{\PFam}{\mathcal{P}}

\newcommand{\offenesintervall}[2]{$\left( #1 , #2 \right)$ }
\newcommand{\abgeschlossenesintervall}[2]{$\left( #1 , #2 \right)$ }

\newcommand{\XTopRaum}{(X,\tau)}


% ==== Lineare Algebra ====
\newcommand{\vinV}{$v \in V$ }
\newcommand{\uinU}{$u \in U$ }
\newcommand{\winW}{$w \in W$ }
\newcommand{\vwinV}{$v,w \in V$ }

\newcommand{\BasisV}{v_1 , \dots , v_n}
\newcommand{\BasisU}{u_1 , \dots , u_n}
\newcommand{\BasisW}{w_1 , \dots , w_n}

\newcommand{\transpose}[1]{#1^t}
\newcommand{\inverse}[1]{#1^{-1}}
\newcommand{\ddvec}[3]{\left( #1,#2,#3 \right)}
\newcommand{\tdvec}[2]{\left( #1 , #2 \right)}

\newcommand{\Edrei}{\begin{pmatrix}
    1 & 0 & 0 \\
    0 & 1 & 0 \\
    0 & 0 & 1
\end{pmatrix}}

\newcommand{\id}{\text{id}}
\newcommand{\GL}{\text{GL}}
\newcommand{\End}{\text{End}}
\renewcommand{\Im}{\text{Im}}
\renewcommand{\Re}{\text{Re}}
\renewcommand{\ker}{\text{Ker}}
\newcommand{\rang}{\text{rang}}
\newcommand{\ad}{\text{ad}}
\newcommand{\Eig}{\text{Eig}}
\newcommand{\Bil}{\text{Bil}}
\newcommand{\sign}{\text{sign}}
\newcommand{\tr}{\text{tr}}



% ====== Physics ======

% ==== Physicssymbols ====
\newcommand{\epsilonnull}{\epsilon_0}
\newcommand{\munull}{\mu_0}
\newcommand{\rn}{r_0}
\newcommand{\Rn}{R_0}
\newcommand{\rhonull}{\rho_0}
\newcommand{\Rhonull}{\varrho_0}


% ==== Notation ====
\newcommand{\Etot}{E_{\text{Tot}}}
\newcommand{\Wtot}{W_{\text{Tot}}}
\newcommand{\Ftot}{F_{\text{Tot}}}
\newcommand{\vtot}{v_{\text{Tot}}}
\newcommand{\atot}{a_{\text{Tot}}}
\newcommand{\mtot}{m_{\text{Tot}}}
\newcommand{\Mtot}{M_{\text{Tot}}}

\newcommand{\Ekin}{E_{\text{Kin}}}
\newcommand{\Epot}{E_{\text{Pot}}}
\newcommand{\Edef}{E_{\text{Def}}}

\newcommand{\Fg}{F_g}
\newcommand{\FN}{F_N}
\newcommand{\Fz}{F_z}
\newcommand{\FC}{F_C}

\newcommand{\xn}{x_0}
\newcommand{\xN}{x_n}
\newcommand{\vn}{v_0}
\newcommand{\vN}{v_n}
\newcommand{\an}{a_0}
\newcommand{\aN}{a_n}

\newcommand{\dt}{\Delta t}
\newcommand{\dx}{\Delta x}
\newcommand{\dv}{\Delta v}
\newcommand{\da}{\Delta a}
\newcommand{\dE}{\Delta E}
\newcommand{\dW}{\Delta W}
\newcommand{\dF}{\Delta F}


% ==== Relativity ====
\newcommand{\relsqrt}{\sqrt{1-\frac{v^2}{c^2}}}
\newcommand{\relgamma}{\frac{1}{\relsqrt}}


% ==== Constants ====
\newcommand{\g}{9.81}





\renewcommand{\epsilon}{\varepsilon}
\clubpenalty = 10000
\widowpenalty = 10000

\begin{document}  

\section{Recap Lineare Algebra \uproman{1}}

\Definition{ (\fat{Gruppen})

    Ein Tupel $(G,*,e)$ ist eine Gruppe falls folgendes erfüllt ist:
    \begin{enumerate}[{\fat{G1)}}]
        \item Assiziativität: $(a*b)*c = a*(b*c)$
        \item Neutrales Element: $\exists e \in G$ mit
                \begin{enumerate}[{a)}]
                    \item $e*a = a \ \forall a \in G$
                    \item $\forall a \in G \ \exists a' \in G$ mit $a*a' = e$
                \end{enumerate}
    \end{enumerate}
    Eine Gruppe heisst \fat{abelsch}, wenn $a*b = b*a \ \forall a,b \in G$.
}

\vspace{1\baselineskip}

\Definition{ (\fat{Gruppenhomomorphismus})

    Seien $(G,\circ_G,e_G)$ und $(H,\circ_H,e_H)$ zwei Gruppen. Eine Abbildung $\varphi:
    G \rightarrow H$ heisst \fat{Gruppenhomomorphismus} wenn $\forall a,b \in G$
    folgendes gilt:
    \begin{align*}
        \varphi(a \circ_G b) = \varphi(a) \circ_H \varphi(b)
    \end{align*}
}

\vspace{1\baselineskip}

\Definition{ (\fat{Ring})

    Ein \fat{Ring} ist ein Tupel $(R,+,\cdot,0)$ bestehend aus einer Menge $R$, zwei
    Verknüpfungen $+$ und $\cdot$ und ein neutrales Element $0 \in R$ sodass:
    \begin{enumerate}[{\fat{R1)}}]
        \item $(R,+,0)$ ist eine abelsche Gruppe
        \item die Multiplikation ist assoziativ
        \item Distributivgesetz: $a \cdot (b+c) = ab + ac$ und $(a+b) \cdot c = ac + bc$.
    \end{enumerate}
    Ein Ring heisst \fat{kommutativ}, wenn $a \cdot b = b \cdot a \ \forall a,b \in R$.
    Ein Element $1 \in R$ heisst \fat{Einselement}, wenn $1 \cdot a = a \cdot 1 = a
    \forall a \in R$. 
    Ein Element $0 \in R$ heisst \fat{Nullelement}, wenn $0 \cdot a = a \cdot 0 = 0
    \forall a \in R$.
}

\vspace{1\baselineskip}

\Definition{

    Ein Ring heisst \fat{Nullteilerfrei}, wenn $\forall a,b \in R$ gilt:
    $a \cdot b \Rightarrow a = 0$ oder $b=0$.
}

\vspace{1\baselineskip}

\Definition{ (\fat{Körper})

    Ein \fat{Körper} ist ein Tupel $(K,+,\cdot,0,1)$ mit
    \begin{enumerate}[\fat{{K1)}}]
        \item $(K,+,0)$ ist eine abelsche Gruppe
        \item $(K \backslash \geschwungeneklammer{0} , \cdot ,1)$ ist eine abelsche Gruppe
        \item Distributivgesetz: $a \cdot (b+c) = ab + ac$ und $(a+b) \cdot c = ac + bc$.
    \end{enumerate}
}

\vspace{1\baselineskip}

\Definition{ (\fat{Vektorraum})

    Sei $K$ ein Körper. Eine Menge $V$ zusammen mit einer inneren Verknüpfung
    $+$ und einer äusseren Verknüpfung $\cdot$ heisst \fat{$K$-Vektorraum} wenn:
    \begin{enumerate}[{\fat{V1)}}]
        \item $(V,+,0_V)$ eine abelsche Gruppe bildet mit $0$ als neutralem Element
        \item Skalare Multiplikation erfüllt $\forall \lambda \in K \ \forall v,w \in V$:
                \begin{align*}
                    &(\lambda+\mu) v = \lambda v + \mu v
                    \quad \quad 
                    \lambda(\mu v) = (\lambda \mu) v
                    \\
                    &\lambda(v+w) = \lambda v + \lambda w
                    \quad \quad
                    1 \cdot v = v
                \end{align*}
    \end{enumerate}
}

\vspace{1\baselineskip}

\Definition{ (\fat{Lineare Abbildungen})

    Eine Abbildung $F: V \rightarrow W$ zwischen $K$-Vektorräumen $V,W$ heisst
    \fat{$K$-linear}, wenn $\forall v,w \in V$ und $\lambda \in K$ gilt:
    \begin{enumerate}[{\fat{L1)}}]
        \item $F(v+w) = F(v) + F(w)$
        \item $F(\lambda v) = \lambda F(v)$ 
    \end{enumerate}
}

\Bemerkung{

    Wir nennen $F$ einen
    \begin{itemize}
        \item \fat{Isomorphismus}, wenn $F$ bijektiv ist
        \item \fat{Endomorphismus}, wenn $V = W$
        \item \fat{Automorphismus}, wenn $V=W$ und $F$ bijektiv
    \end{itemize}
}

\vspace{1\baselineskip}

\Bemerkung{
    \begin{enumerate}[{a)}]
        \item Sind $(v_i)_{i \in I}$ linear abhängig, so sind auch $(F(v_i))_{i \in I}$ linear abhängig.
        \item Ist $F$ ein Isomorphismus, so ist auch $F^{-1}:W \rightarrow V$ linear.
    \end{enumerate}
}

\Definition{

    $\Hom_K (V,W) := \geschwungeneklammer{F:V \rightarrow W \ | \ F \text{ ist $K$-linear}}$
    
    \vspace{1\baselineskip}

    $\text{End}(V) := \Hom(V,V)$
}

\vspace{1\baselineskip}

\Bemerkung{

    Ist $F: V \rightarrow W$ linear, so gilt:
    \begin{enumerate}[{a)}]
        \item $F$ surjektiv $\Leftrightarrow$ Im$(F) = W$
        \item $F$ injektiv $\Leftrightarrow$ Ker$(F) = \geschwungeneklammer{0}$
        \item $F$ injektiv und $v_1 , \dots , v_n \in V$ linear unabhängig, so sind auch
                die Bilder $F(v_1) , \dots , F(v_n)$ linear unabhängig.
    \end{enumerate}
}

\Definition{

    Eine Teilmenge $X$ eines $K$-Vektorraumes $V$ heisst \fat{affiner Unterraum},
    falls es ein $v \in V$ und einen Untervektorraum $W \subset V$ gibt, sodass
    $X=v+W := \geschwungeneklammer{v+w \ | \ w \in W} = \geschwungeneklammer{
    u \in V \ | \ \exists w \in W \text{ mit } u=v+w}$
}

\vspace{1\baselineskip}

\Korollar{

    Zwischen zwei endlich dimensionalen Vektorräumen gibt es genau dann einen
    Isomorphismus, wenn $\dim (V) = \dim (W)$.
}

\vspace{1\baselineskip}

\Satz{ (\fat{Fakrorisierungssatz})

    Sei $F: V \rightarrow W$ linear und $A = (u_1,\dots,u_r,v_1,\dots,v_k)$ eine Basis
    von $V$ mit $\ker (F) = \text{span} (v_1,\dots,v_k)$. Definieren wir
    $U=\text{span} (u_1,\dots,u_r)$ so gilt:
    \begin{enumerate}[{1)}]
        \item $V=U \oplus \ker F$
        \item Die Einschränkung $F |_U : U \rightarrow \text{Im} F$ ist ein Isomorphismus
        \item Bezeichnet $P: V = U \oplus \ker F \rightarrow U$, $v = u + v' \mapsto u$,
                die Projektion auf den ersten Summanden, so ist $F=(F |_U) \circ P$.
                In Form eines Diagrammes hat man                    
    \end{enumerate} 
}
\begin{center}
    \begin{tikzcd}
        V \arrow{d}[swap]{P} \arrow{rd}{F} & \\
        U \arrow{r}[swap]{F |_U} & Im F \subset W
    \end{tikzcd}
\end{center}
Insbesondere hat jede nichtleere Faser $F^{-1} (w)$ mit $U$ genau einen
Schnittpunkt, und es ist $P(v) = F^{-1} (F(v)) \cap U$.
Man kann also $F: V \rightarrow W$ zerlegen (fakrorisieren) in drei
Anteile: Parallelprojektion, Isomorphismus und Inklusion des Bildes.

\vspace{1\baselineskip}

\Satz{

    Sei $V$ ein $K$-VR und $U \subset V$ ein UVR. Daann kann man die Menge
    $V/U$ auf genau eine Weise so zu einem $K$-Vektorraum machen, dass die
    kanonische Abbildung $\varrho: V \rightarrow V/U$ gegeben durch
    $v \mapsto v + U$ linear wird. Weiter gilt:
    \begin{enumerate}[{1)}]
        \item $\varrho$ ist surjektiv
        \item $\ker \varrho = U$
        \item $\dim V/U = \dim V - \dim U$ falls $\dim V < \infty$
        \item Der Quotientenvekrorraum $V/U$ hat folgende universelle
                Eigenschaft: Ist $F:V \rightarrow W$ eine lineare Abbildung
                mit $U \subseteq \ker F$, so gibt es genau eine lineare
                Abbildung $\overline{F}: V/U \rightarrow W$ mit
                $F = \overline{F} \circ \varrho$. Das kann man in Form eines kommutativen
                Diagrammes schreiben
    \end{enumerate}
}
\begin{center}
    \begin{tikzcd}
        V \arrow{r}{F} \arrow{d}[swap]{\varrho} & W \\
        V/U \arrow{ru}[swap]{\overline{F}} & 
    \end{tikzcd}
\end{center}
Man nennt $V/U$ den \fat{Quotientenvekrorraum} von $V$ nach $U$.
Diese Bezeichnung entspricht der Vorstellung, dass man $U$ aus $V$
"herausdividiert", weil $U$ in $V/U$ zur Null wird.


\section{Trigonalisierung}

\Definition{
    
    Sei $F: V \rightarrow V$ ein Endomorphismus und $W \subset V$ ein UVR. $W$ heisst
    \fat{$F$-invariant}, wenn $F(W) \subset W$ ($\Leftrightarrow \forall w \in W: \
    F(w) \in W$)
}

\vspace{1\baselineskip}

\Bemerkung{

    Ist $W \subset V$ ein $F$-invarianter UVR, so ist $P_{F |_W}$ ein Teiler von $P_F$.
    Also $P_F (t) = P_{F |_W} (t) \cdot Q(t)$.
    (Erinnerung: $F |_W: W \rightarrow W$ mit $w \mapsto F(w)$.)
}

\vspace{1\baselineskip}

\Definition{

    Unter einer \fat{Fahne} $(V_r)$ in einem $n$-dimensionalen Vektorraum $V$ versteht
    man eine Kette
    \begin{align*}
        \geschwungeneklammer{0} = V_0 \subset V_1 \subset \dots \subset V_n = V
    \end{align*}
    von UVR mit $\dim V_r = r$. Ist $F \in$ End$(V)$, so heisst die Fahne
    $F$-invariant, wenn
    \begin{align*}
        F(V_r) \subset V_r \quad \text{   für alle } r \in \geschwungeneklammer{0,\dots,n}
    \end{align*}
}

\Bemerkung{

    Für $F \in$ End$(V)$ sind folgende Bedingungen gleichwertig:
    \begin{enumerate}[i)]
        \item Es gibt eine $F$-invariante Fahne in $V$.
        \item Es gibt eine Basis $\mathcal{B}$ von $V$, so dass $M_{\mathcal{B}} (F)$
                eine obere Dreiecksmatrix ist.
    \end{enumerate}
    Ist das der Fall, so heisst $F$ \fat{trigonalisierbar}.
}

\vspace{1\baselineskip}

\Definition{

    Eine Matrix $A \in M(n \times n ; K)$ heisst \fat{trigonalisierbar}, wenn
    $A: K^n \rightarrow K^n$ mit $v \mapsto A v$ trigonalisierbar ist.
    Das heisst, es existiert ein $S \in \text{GL}(n,K)$ sodass $S A S^{-1}$ eine obere
    Dreiecksmatrix ist.
}

\vspace{1\baselineskip}

\Satz{ (\fat{Trigonalisierungssatz})

    Für einen Endomorphismus $F$ eines $n$-dimensionalen $K$-VR sind folgende
    Bedingungen äquivalent:
    \begin{enumerate}[i)]
        \item $F$ ist trigonalisierbar.
        \item Das charakteristische Polynom $P_F$ zerfällt in Linearfaktoren, d.h.
            \begin{align*}
                P_F = \pm (t - \lambda_1) \cdot \dots \cdot (t - \lambda_n)
                \quad \text{   mit } \lambda_1 , \dots , \lambda_n \in K
            \end{align*}
    \end{enumerate}
}

\Korollar{

    Jeder Endomorphismus eines endlich-dimensionalen komplexen VR ist trigonalisierbar.
}

\vspace{1\baselineskip}

\fat{\large Algorithmus zur Bestimmung der Trigonalform} \normalsize

\vspace{1\baselineskip}

\fat{1. Schritt}

Wir betrachten $W_1 = K^n$ mit der Basis $\mathcal{B}_1 = \mathcal{K}$ und den
Endomorphismus $A_1 = A$. Zu $\lambda_1$ berechnet man einen Eigenvektor $V_1 \in K^n$.
Nach dem Austauschlemma bestimmt man ein $j_1 \in \geschwungeneklammer{1,\dots,n}$, so
dass
\begin{align*}
    \mathcal{B}_2 := \klammer{v_1 , e_1 , \dots , \hat{e}_{j_1} , \dots , e_n}
\end{align*}
wieder eine Basis von $K^n$ ist. Das Zeichen $\hat{}$ bedeutet dabei, dass $e_{j_1}$
ausgelassen wird. Wir betrachten die Transformationsmatrix $S_1^{-1} := T_{\mathcal{B}_1}^{\mathcal{B}_2}$
mit der Basis $\mathcal{B}_2$ als Spalten. Dann ist
\begin{align*}
    A_2 := S_1 \cdot A \cdot S_1^{-1} = \begin{pmatrix}
        \lambda_1 & * & \dots & * \\
        0 & & & \\
        \vdots & & A_2' & \\
        0 & & & 
    \end{pmatrix}
\end{align*}

\fat{2. Schritt}

Wir betrachten $W_2$ mit der Basis
\begin{align*}
    \mathcal{B}_2' := \klammer{e_1 , \dots , \hat{e}_{j_1} , \dots , e_n}
\end{align*}
und den Endomorphismus $A_2'$. Es ist
\begin{align*}
    P_{A'_2} = \pm (t-\lambda_2) \cdot \dots \cdot (t-\lambda_n)
\end{align*}
Zu $\lambda_2$ berechnet man einen Eigenvektor $v_2 \in W_2$, und man wählt ein
$j_2 \neq j_1$, so dass 
\begin{align*}
    \mathcal{B}_3' := \klammer{v_2 , e_1 , \dots , \hat{e}_{j_1} , \dots , \hat{e}_{j_2} , \dots , e_n}
\end{align*}
eine Basis von $W_2$ ist, also
\begin{align*}
    \mathcal{B}_3 := \klammer{v_1, v_2 , e_1 , \dots , \hat{e}_{j_1} , \dots , \hat{e}_{j_2} , \dots , e_n}
\end{align*}
eine Basis von $K^n$ ist. Mit der Transformationsmatrix $S_2^{-1} = T_{\mathcal{B}_1}^{\mathcal{B}_3}$
erhält man
\begin{align*}
    A_3 = S_2 \cdot A \cdot S_2^{-1} = \begin{pmatrix}
        \lambda_1 & * & \dots & \dots & * \\
        0 & \lambda_2 & * & \dots & * \\
        \vdots & 0 & & & \\
        \vdots & \vdots & & A_3' & \\
        0 & 0 & & &
    \end{pmatrix}
\end{align*}
Bei der Berechnung von $S_2$ kann man benutzen, dass
\begin{align*}
    T_{\mathcal{B}_1}^{\mathcal{B}_3} = T_{\mathcal{B}_1}^{\mathcal{B}_2} \cdot T_{\mathcal{B}_2}^{\mathcal{B}_3}
    \quad \text{   und   } \quad
    T_{\mathcal{B}_2}^{\mathcal{B}_3} = \begin{pmatrix}
        1 & 0 & \dots & 0 \\
        0 & & & \\
        \vdots & & T_{\mathcal{B}'_2}^{\mathcal{B}'_3} & \\
        0 & & &
    \end{pmatrix}
\end{align*}
Spätestens im $(n-1)$-ten Schritt erhält man eine obere Dreiecksmatrix $A_n$, denn
$A_n'$ ist eine $(1 \times 1)$-Matrix. Also ist
\begin{align*}
    D := A_n = S_{n-1} \cdot A \cdot S_{n-1}^{-1}
\end{align*}
eine obere Dreiecksmatrix.










\section{Potenzen eines Endomorphismus}

\vspace{1\baselineskip}

\Bemerkung{

    Das Einsetzen eines Endomorphismus in Polynome ist beschrieben durch die
    Abbildung
    \begin{align*}
        \Phi_F : K [t] &\rightarrow \text{End} (V) \\
        P(t) &\mapsto P(F)
    \end{align*}
    Dies ist ein Homomorphismus von Ringen und auch von $K$-VR. Das Bild
    \begin{align*}
        K[F] = \geschwungeneklammer{P(F): P(t) \in K[t]} \subset \text{End} (V)
    \end{align*}
    ist ein kommutativer Unterring des (nicht kommutativen) Ringes End$(V)$, und der
    Kern
    \begin{align*}
        \mathcal{I}_F := \geschwungeneklammer{P(t) \in K[t]: \ P(F) = 0} \subseteq K[t]
    \end{align*}
    heisst \fat{Ideal} von $F$.
}

\vspace{1\baselineskip}

\Satz{ (Satz von Cayley-Hamilton)

    Sei $V$ ein endlichdimensionaler $K$ Vektorraum, $F \in \text{End} (V)$ und
    $P_F \in K[t]$ sein charakteristisches Polynom. Dann ist
    \begin{align*}
        P_F (F) = 0 \in \text{End} (V)
    \end{align*}
    Insbesondere gilt für jede Matrix $A \in M(n \times n ; K)$
    \begin{align*}
        P_A (A) = 0 \in M(n \times n ; K)
    \end{align*}
}

\vspace{1\baselineskip}

\Definition{

    Eine Teilmenge $\mathcal{I}$ eines kommutativen Ringes $R$ heisst \fat{Ideal},
    wenn gilt:
    \begin{enumerate}[{\fat{I1)}}]
        \item $P,Q \in \mathcal{I} \Rightarrow P - Q \in \mathcal{I}$
        \item $P \in \mathcal{I}$, $Q \in R \Rightarrow Q \cdot P \in \mathcal{I}$
    \end{enumerate}
}

\vspace{1\baselineskip}

\Satz{

    Zu jedem Ideal $\mathcal{I} \subseteq K[t]$ mit $\mathcal{I} \neq \geschwungeneklammer{0}$
    gibt es ein eindeutiges Polynom $M$ mit folgenden Eigenschaften:
    \begin{enumerate}[{1)}]
        \item $M$ ist normiert, d.h. $M = t^d + \dots$, wobei $d = \deg M$.
        \item Für jedes $P \in \mathcal{I}$ gibt es ein $Q \in K[t]$ mit $P = Q \cdot M$.
    \end{enumerate}
    $M$ heisst \fat{Minimalpolynom} von $\mathcal{I}$, im Fall $\mathcal{I} = \mathcal{I}_F$
    Minimalpolynom von $F$.
}

\vspace{1\baselineskip}

\Satz{

    Sei $V$ ein $n$-dimensionaler $K$-Vektorraum, $F \in$ End$(V)$. Dann gilt:
    \begin{enumerate}[{1)}]
        \item $M_F$ teilt $P_F$
        \item $P_F$ teilt $M_F^n$
    \end{enumerate}
}

\vspace{1\baselineskip}

\Definition{

    Man nennt $F \in$ End$_K (V)$ \fat{nilpotent}, wenn $F^k = 0$ für ein $k \geq 1$.
}

\vspace{1\baselineskip}

\Satz{

    Ist $F \in$ End$_K (V)$ und $n = \dim V$, so sind folgende Aussagen äquivalent:
    \begin{enumerate}[i)]
        \item $F$ ist nilpotent
        \item $F^d = 0$ für ein $d$ mit $1 \leq d \leq n$
        \item $P_F = \pm t^n$
        \item Es gibt eine Basis $\mathcal{B}$ von $V$, so dass
            \begin{align*}
                M_{\mathcal{B}} (F) = \begin{pmatrix}
                    0 & & * \\
                    & \ddots & \\
                    0 & & 0
                \end{pmatrix}
            \end{align*}
    \end{enumerate}
}


\section{Die Jordansche Normalform}

\vspace{1\baselineskip}

\Definition{

    Sei $\lambda$ ein Eigenwert von $F \in$ End$(V)$ und $\dim V = n < \infty$. Dann
    definieren wir den \fat{Hauptraum} (verallgemeinerter Eigenraum) von $F$ bzgl $\lambda$
    als
    \begin{align*}
        \text{Hau} (F;\lambda) = \ker \klammer{F- \lambda id}^r
    \end{align*}
    wobei $r$ die algebraische Vielfachheit von $\lambda$ ist.
    (d.h. $P_F (t) = (t-\lambda)^r \cdot (\dots)$)
}

\vspace{1\baselineskip}

\Bemerkung{

    Es gilt Eig$(F,\lambda) = \ker (F - \lambda \text{id}) \subset$ Hau$(F; \lambda)$
}

\vspace{1\baselineskip}

\Satz{ (\fat{Satz über die Hauptraumzerlegung})

    Sei $F \in$ End$_K (V)$ und
    \begin{align*}
        P_F = \pm (t-\lambda_1)^{r_1} \cdot \dots \cdot (t-\lambda_k)^{r_k}
    \end{align*}
    mit paarweise verschiedenen $\lambda_1 , \dots , \lambda_k \in K$.
    Es sei $V_i :=$ Hau$(F;\lambda_i) \subset V$ für jedes $\lambda_i$
    der Hauptraum. Dann gilt:
    \begin{enumerate}[1)]
        \item $F(V_i) \subset V_i$ und $\dim V_i = r_i$ für $i = 1,\dots,k$
        \item $V = V_1 \oplus \dots \oplus V_k$
        \item $F$ hat eine Zerlegung $F = F_D + F_N$ mit
                \begin{enumerate}[a)]
                    \item $F_D$ diagonalisierbar
                    \item $F_N$ nilpotent
                    \item $F_D \circ F_N = F_N \circ F_D$
                \end{enumerate}
        \item $F_D$ und $F_N$ lassen sich als Polynome in $F$ schreiben.
                Insbesondere kommutieren sie mit $F$.
        \item Die Zerlegung $F=F_D + F_N$ ist eindeutig wenn man a),b) und c)
                verlangt.
    \end{enumerate}
}

\vspace{1\baselineskip}

\Korollar{

    Sei $A \in M(n \times n; K)$, so dass
    $P_A = \pm (t-\lambda_1)^{r_1} \cdot \dots \cdot (t-\lambda_k)^{r_k}$.
    Dann gibt es eine invertierbare Matrix $S \in$ GL$(n;K)$ derart, dass
    \begin{align*}
        S A S^{-1} = \begin{pmatrix}
            \lambda_1 E_{r_1} + N_1 & & 0 \\
            & \ddots & \\
            0 & & \lambda_k E_{r_k} + N_k
        \end{pmatrix}
        := \tilde{A}
    \end{align*}
    mit $\lambda_i E_{r_i} + N_i$ Blöcken für alle $i = 1,\dots,k$
    welche wie folgt gegeben sind:
    \begin{align*}
        \lambda_i E_{r_i} + N_i = \begin{pmatrix}
            \lambda_i & & * \\
            & \ddots & \\
            0 & & \lambda_i
        \end{pmatrix}
        \in M(r_i \times r_i \ ; K)
    \end{align*}
    mit nilpotenten $N_i$. Insbesondere ist $\tilde{A} = D + N$, wobei
    $D$ eine Diagonalmatrix und $N$ nilpotent (strikte obere Dreiecksmatrix)
    ist. Und es gilt $D \cdot N = N \cdot D$.
}

\vspace{1\baselineskip}

\Lemma{ (Lemma von Fitting)

    Zu einem $G \in$ End$_K (V)$ betrachten wir die beiden Zahlen
    \begin{align*}
        &d:= \min \geschwungeneklammer{l \in \N : \ \ker(G^l) = \ker (G^{l+1})}
        \\
        &r:= \mu(P_G \ ; 0)
    \end{align*}
    wobei $G^0 :=$ id$_V$. Dann gilt:
    \begin{enumerate}[1)]
        \item $d = \min \geschwungeneklammer{l: \ \text{Im} \ G^l = \text{Im} \ G^{l+1}}$
        \item $\ker \ G^{d+1} = \ker \ G^{d}$, $\Im \ G^{d+1} = \Im \ G^d$ für alle $i \in \N$.
        \item Die Räume $U := \ker \ G^d$ und $W:= \Im \ G^d$ sind $G$-invariant.
        \item $(G | U)^d = 0$ und $G |_W : W \rightarrow W$ ist ein Isomorphismus.
        \item Für das Minimalpolynom von $G |_U$ gilt $M_{G |_U} = t^d$.
        \item $V = U \oplus W$, $\dim U = \mu(P_G,0) = r\geq d$, $\dim W = n-r$.
    \end{enumerate}
    Insbesondere gibt es eine Basis $\mathcal{B}$ von $V$, so dass
    \begin{align*}
        M_{\mathcal{B}} (G) = \begin{pmatrix}
            N & 0 \\
            0 & C
        \end{pmatrix}
        \quad \text{   mit   } \
        N^d = 0
        \ \text{ und } \
        C \in \GL (n-r \ ; K)
    \end{align*}
}

\vspace{1\baselineskip}

\Definition{

    Wir definieren die \fat{Jordanmatrix} als
    \begin{align*}
        J_k := \begin{pmatrix}
            0 & 1 & & 0 \\
            & \ddots & \ddots & \\
            & & \ddots & 1 \\
            0 & & & 0
        \end{pmatrix}
        \in M(k \times k \ ; K)
    \end{align*}
    $J_k$ ist nilpotent und $(J_k)^k = 0$. $k$ ist die minimale Potenz mit dieser Eigenschaft.
}

\vspace{1\baselineskip}

\Theorem{

    Sei $G$ ein nilpotenter Endomorphismus eines $K$-VR $V$ und $d:= \min
    \geschwungeneklammer{l: G^l = 0}$. Dann gibt es eindeutig bestimmte Zahlen
    $s_1,\dots,s_d \in \N$ mit
    \begin{align*}
        s_1 + 2 \cdot s_2 + 3 \cdot s_3 + \dots + d \cdot s_d = r = \dim V
    \end{align*}
    und eine Basis $\mathcal{B}$ von $V$, so dass
    \begin{align*}
        M_{\mathcal{B}} (G) = \begin{pmatrix}
            J_d & & & & & & & & & \\
            & \ddots & & & & & & & & \\
            & & J_d & & & & & & & \\
            & & & J_{d-1} & & & & 0 & & \\
            & & & & \ddots & & & & & \\
            & & & & & J_{d-1} & & & & \\
            & & & & & & \ddots & & & \\
            & & & 0 & & & & J_1 & & \\
            & & & & & & & & \ddots & \\
            & & & & & & & & & J_1 \\
        \end{pmatrix}
    \end{align*}
    mit $s_d$-mal $J_d$, $s_{d-1}$-mal $J_{d-1}$, \dots , $s_1$-mal $J_1$.
    Beachte: $J_1 = 0$.
}

\vspace{1\baselineskip}

\Definition{ (\fat{Jordansche Normalform})

    Sei $F \in \End_K (V)$ derart, dass das charakteristische Polynom in Linearfaktoren
    zerfällt, also
    \begin{align*}
        P_F = \pm (t-\lambda_1)^{r_1} \cdot \dots \cdot (t-\lambda_k)^{r_k}
    \end{align*} 
    mit paarweise verschiedenen $\lambda_1,\dots,\lambda_k \in K$. Dann gibt es eine Basis
    $\mathcal{B}$ von $V$, so, dass
    \begin{align*}
        M_{\mathcal{B}} (F) = \begin{pmatrix}
            \lambda_1 E_{r_1} + N_1 & & 0 \\
            & \ddots & \\
            0 & & \lambda_k E_{r_k} + N_k
        \end{pmatrix}
    \end{align*}
    Mit $\lambda_i E_{r_i} + N_i$ als Blöcken, wobei $N_i$ für $1,\dots,k$ in der
    Normalform ist. Ausgeschrieben bedeutet das
    \begin{align*}
        \lambda_i E_{r_i} + N_i =
        \begin{pmatrix}
            J_k (\lambda_i) & 0 & \\
            & J_k (\lambda_i) & 0 \\
            & & J_k (\lambda_i)
        \end{pmatrix}
    \end{align*}
    mit
    \begin{align*}
        J_k (\lambda_i) = \begin{pmatrix}
            \lambda_i & 1 & & \\
            & \ddots & \ddots & \\
            & & \ddots & 1 \\
            & & & \lambda_i
        \end{pmatrix}
    \end{align*}
    und Nullen auf der Nebendiagonalen, wo keine $1$-en sind.
    Solch $J_k$ nennt man \fat{Jordanblöcke} der Länge $d$ zu $\lambda_i$.
    der oberste und grösste Jordanblock in $\lambda_i R_{r_i} + N_i$ hat die
    Grösse $d_i$ mit
    \begin{align*}
        1 \leq d_i = \min \geschwungeneklammer{l: N_i^l = 0} \leq r_i
    \end{align*}
    das ist die Vielfachheit der Nullstelle $\lambda_i$ im Minimalpolynom von
    $F$. Für $1 \leq j \leq d_i$ seien $s_j^{(i)} \geq 0$ die Anzahl der Jordanblöcke
    der Grösse $j$ zu $\lambda_i$ in $\lambda_i E_{r_i} + N_i$. Es ist $s_{d_i}^{(i)} \leq 1$,
    und durch Abzählung der Längen folgt:
    \begin{align*}
        d_i s_{d_i}^{(i)} + (d-1) s_{d_i -1}^{(i)} + \dots + s_1^{(i)} = r_i
    \end{align*}
    Des Weiteren gilt: $r_1 + \dots + r_k = n$.

    Die Reihenfolge der Jordanblöcke ist unwesentlich, da sie durch eine Permutation der
    Basisvektoren beliebig verändert werden kann.
}

\vspace{1\baselineskip}

\Korollar{

    Für ein $F \in \End_k (V)$ sind folgende Bedingungen äquivalent:
    \begin{enumerate} [i)]
        \item $F$ ist diagonalisierbar
        \item $M_F = (t-\lambda_1) \cdot \dots \cdot (t-\lambda_k)$, wobei $\lambda_1,\dots,
                \lambda_k$ die verschiedenen Eigenwerte von $F$ bezeichnen.
        \item Es gibt paarweise verschiedene $\lambda_1,\dots\lambda_m \in K$, so, dass
                \begin{align*}
                    (F-\lambda_1 \id_V) \circ \dots \circ (F-\lambda_m \id_V) = 0 \in \End_K (V)
                \end{align*}
    \end{enumerate}
}

\vspace{1\baselineskip}

\Korollar{

    Allgemein gilt für ein $F \in \End_K (V)$:
    \begin{align*}
        M_F (t) = (t-\lambda_1)^{d_1} \cdot \dots \cdot (t-\lambda_k)^{d_k}
    \end{align*}
    wobei $d_i$ die Länge des grössten Jordanblockes zu $\lambda_i$ ist.
}


\section{Das kanonische Skalarprodukt im $\R^n$}

\vspace{1\baselineskip}

\Definition{

    Das \fat{kanonische Skalarprodukt} ist eine Abbildung
    \begin{align*}
        \scalprod{ \ }{ \ } : \R^n \times \R^n &\longrightarrow \R
        \\
        (x,y) &\mapsto \scalprod{x}{y}
    \end{align*}
    mit $x = (x_1,\dots,x_n)$ und $y=(y_1,\dots,y_n)$ $\in \R^n$. Schreibt man $x$ und $y$ als
    Spaltenvektoren, so ist:
    \begin{align*}
        \scalprod{x}{y} = x^T \cdot y = (x_1,\dots,x_n) \cdot \begin{pmatrix}
            y_1 \\ \vdots \\ y_n
        \end{pmatrix}
        = \sum_{i=1}^n x_i y_i
    \end{align*}
    Formal muss folgendes erfüllt sein für $x,x',y,y' \in \R^n$ und $\lambda \in \R$:

    (1) (Bilinearität) $\scalprod{x+x'}{y} = \scalprod{x}{y} + \scalprod{x'}{y}$

        \hspace{17pt} $\scalprod{x}{y+y'} = \scalprod{x}{y} + \scalprod{x}{y'}$

        \hspace{17pt} $\scalprod{\lambda x}{y} = \lambda \scalprod{x}{y}$

        \hspace{17pt} $\scalprod{x}{\lambda y} = \lambda \scalprod{x}{y}$

    (2) (Symmetrie) $\scalprod{x}{y} = \scalprod{y}{x}$

    (3) (Positive Definitheit) $\scalprod{x}{x} \geq 0 \quad \text{   und   }  \quad 
                \scalprod{x}{x} = 0 \Leftrightarrow$ 
                
                \hspace{17pt} $x=0$
}

\vspace{1\baselineskip}

\Definition{

    Die \fat{Norm} ist definiert als eine Abbildung:
    \begin{align*}
        &\Norm{ \ } : \R^n \rightarrow \R_{\geq 0} \\
        &x \mapsto \Norm{x} := \sqrt{\scalprod{x}{x}}
    \end{align*}
    mit folgenden Eigenschaften:
    \begin{enumerate}[{\fat{N1}}]
        \item $\Norm{x} = 0 \Leftrightarrow x=0$
        \item $\lambda x = \abs{\lambda} \Norm{x}$
        \item $\Norm{x+y} \leq \Norm{x} + \Norm{y}$
    \end{enumerate}
}

\vspace{1\baselineskip}

\Definition{

    Analog zur Norm, kann man den \fat{Abstand} definieren
    \begin{align*}
        &d: \R^n \times \R^n \rightarrow \R_{\geq 0} \\
        &d(x,y):= \Norm{y-x} = \sqrt{(y_1 - x_1)^2 + \dots + (y_n - x_n)^2}
    \end{align*}
    mit folgenden Eigenschaften
    \begin{enumerate}[{\fat{D1}}]
        \item $d(x,y) = 0 \Leftrightarrow x=y$
        \item $d(x,y) = d(y-x)$
        \item $d(x,z) \leq d(x,y) + d(y,z)$
    \end{enumerate}
}

\vspace{1\baselineskip}

\Satz{ (Ungleichung von Cauchy-Schwarz)

    $\abs{\scalprod{x}{y}} \leq \Norm{x} \cdot \Norm{y}$
    \ und \
    $\abs{\scalprod{x}{y}} = \Norm{x} \cdot \Norm{y}$
    \ genau dann, wenn $x$ und $y$ linear unabhängig sind.
}

\vspace{1\baselineskip}

\Definition{

    Wir definieren den Winkel $\vartheta$ zwischen zwei Vektoren $x,y \in \R^n$ als
    \begin{align*}
        \vartheta = \angle (x,y) =
        \arccos \klammer{\frac{\scalprod{x}{y}}{\Norm{x} \cdot \Norm{y}}}
    \end{align*}
    mit
    \begin{align*}
        &\angle (x,y) = \angle (y,x) \\
        &\angle(x,y) = \angle(\alpha x, \beta y) \\
        &\scalprod{x}{y} = \cos (\angle(x,y)) \cdot \Norm{x} \cdot \Norm{y}
    \end{align*}
}

\vspace{1\baselineskip}

\Definition{

    Wir sagen $x,y \in \R^n$ sind \fat{orthogonal} zueinander, schreibe $x \perp y$, wenn
    $\scalprod{x}{y} = 0$.
}


\section{Das Vektorprodukt im $\R^3$}

\Definition{

    Wir definieren das \fat{Vektorprodukt} als eine Abbildung
    \begin{align*}
        &\R^3 \times \R^3 \rightarrow \R^3 \\
        &(x,y) \mapsto x \times y := (x_2 y_x - x_3 y_2 , x_3 y_1 - x_1 y_3 , x_1 y_2 - x_2 y_1)
    \end{align*}
    mit

    (1) $(x+x') \times y = x \times y + x' \times y$

        \hspace{17pt} $x \times (y+y') = x \times y + x \times y'$

        \hspace{17pt} $\lambda x \times y = \lambda (x \times y)$

        \hspace{17pt} $x \times \lambda y = \lambda (x \times y)$
        
    (2) $y \times y = - y \times x$ also $x \times x = 0$

    (3) $x \times y = 0 \Leftrightarrow x,y$ sind linear abhängig
}

\vspace{1\baselineskip}

\Bemerkung{

    Für $x,y,z \in \R^3$ gilt:
    \begin{align*}
        &\scalprod{x \times y}{z} = \det \begin{pmatrix}
            x_1 & x_2 & x_3 \\
            y_1 & y_2 & y_3 \\
            z_1 & z_2 & z_3
        \end{pmatrix}
        \\
        &\scalprod{x \times y}{x} = \scalprod{x \times y}{y} = 0
        \\
        &\Norm{x \times y}^2 = \Norm{x}^2 \Norm{y}^2 - \scalprod{x}{y}^2
        \\
        &\Norm{x \times y} = \Norm{x} \cdot \Norm{y} \cdot \sin \angle (x,y)
    \end{align*}
}

\section{Das kanonische Skalarprodukt im $\C^n$}

\vspace{1\baselineskip}

\Definition{

    Wir definieren das \fat{kanonische Skalarprodukt} als eine Abbildung
    $\C^n \times \C^n \rightarrow \C$ gegeben durch
    \begin{align*}
        (z,w) \mapsto \scalprod{z}{w}_{\C} := z^T \overline{w} = \sum_{j=1}^n z_j \overline{w}_j
    \end{align*}
    mit folgenden Eigenschaften (Sei $\scalprod{ \ }{ \ }$ immer $\scalprod{ \ }{ \ }_{\C}$):

    (1) $\scalprod{z+z'}{w} = \scalprod{z}{w} + \scalprod{z'}{w}$

    \hspace{13pt} $\scalprod{z}{w+w'} = \scalprod{z}{w} + \scalprod{z}{w'}$

    \hspace{13pt} $\scalprod{\lambda z}{w} = \lambda \scalprod{z}{w}$

    \hspace{13pt} $\scalprod{z}{\lambda w} = \overline{\lambda} \scalprod{z}{w}$

    (2) $\scalprod{w}{z} = \overline{\scalprod{z}{w}}$

    (3) $\scalprod{z}{z} \in \R_{\geq 0}$ und $\scalprod{z}{z} = 0 \Leftrightarrow z=0$

    für $z,z',w,w' \in \C^n$ und $\lambda \in \C$.
}

\vspace{1\baselineskip}

\Definition{

    Wir können analog die \fat{Norm} in $\C^n$ definieren:
    \begin{align*}
        \C^n &\rightarrow \R_{\geq 0} \\
        z &\mapsto \Norm{z} := \sqrt{\scalprod{z}{z}_{\C}}
    \end{align*}
}

\section{Bilinearformen und Sesquilinearformen}

\Definition{

    Sei $K$ ein Körper und seien $V_1,V_2,W$ $K$-VR. eine Abbildung
    $s: V_1 \times V_2\rightarrow W$ gegeben durch $(v,w) \mapsto s(v,w)$
    heisst \fat{Bilinearform}, wenn gilt:
    \begin{enumerate}[{\fat{B1}}]
        \item $s(\alpha_1 v_1 + \alpha_2 v_2 , w) = \alpha_1 s(v_1,w) + \alpha_2 s(v_2,w)$
        \item $s(v,\beta_1 w_1 + \beta_2 w_2) = \beta_1 s(v,w_1) + \beta_2 s(v,w_2)$
    \end{enumerate}
    Die Abbildung $s$ heisst \fat{symmetrisch}, falls $s(v,w) = s(w,v)$
    und \fat{alternierend} oder \fat{schiefsymmetrisch}, wenn $s(w,v) = - s(v,w)$.    
}

\vspace{1\baselineskip}

\Definition{

    Sei $V$ ein endlich dimensionaler $K$-VR, $s: V \times V \rightarrow K$ eine Bilinearform
    und sei $\mathcal{B} = (v_1,\dots,v_n)$ eine geordnete Basis von $V$. Die darstellende
    Matrix von $s$ bezüglich $\mathcal{B}$ ist $M_{\mathcal{B}} (s) =: A = (a_{ij})_{ij}
    \in M(n \times n \ ;K)$ definiert durch
    \begin{align*}
        a_{ij} = s(v_i,v_j)
    \end{align*}
}

\vspace{1\baselineskip}

\Bemerkung{

    Sei $s$ eine Bilinearform auf $V$ mit Basis $\mathcal{B}$ und $\Phi_{\mathcal{B}}:
    K^n \rightarrow V$ das zugehörige Koordinatensystem. Wir betrachten die Matrix
    $A = M_{\mathcal{B}} (s)$ und für $v,w \in V$ die Koordinaten $x = \Phi_{\mathcal{B}}^{-1} (v)$,
    $y = \Phi_{\mathcal{B}}^{-1} (w)$. Dann gilt:
    \begin{align*}
        s(v,w) = x^T A y
    \end{align*}
}

\vspace{1\baselineskip}

\Satz{

    Sei $V$ ein endlichdimensionaler $K$-VR und $\mathcal{B}$ eine Basis. Dann ist die
    Abbildung
    $ s \mapsto M_{\mathcal{B}} (s) $
    von den Bilinearformen auf $V$ in $M(n \times n \ ;K)$ bijektiv, und $s$ ist genau dann
    symmetrisch, wenn $M_{\mathcal{B}} (s)$ symmetrisch ist.
}

\vspace{1\baselineskip}

\Lemma{

    Gegeben seien $A,B \in M(n \times n \ ; K)$ derart, dass
    \begin{align*}
        x^T A y = x^T B y
    \end{align*}
    für alle Spaltenvektoren $x,y \in K^n$. Dann ist $A=B$.
}

\vspace{1\baselineskip}

\Bemerkung{ (\fat{Transformationsformel})

    Sei $V$ ein endlichdimensionaler VR mit Basen $\mathcal{A}, \mathcal{B}$ und sei
    $T_{\mathcal{A}}^{\mathcal{B}}$ die entsprechende Transformationsmatrix. Für jede
    Bilinearform $s$ auf $V$ gilt dann:
    \begin{align*}
        M_{\mathcal{B}} (s) = \klammer{T_{\mathcal{A}}^{\mathcal{B}}}^T \cdot M_{\mathcal{A}} (s) \cdot T_{\mathcal{A}}^{\mathcal{B}}
    \end{align*}
    Man beachte, dass für einen \underline{Endomorphismus} $F$ von $V$ gilt:
    \begin{align*}
        M_{\mathcal{B}} (F) = \klammer{T_{\mathcal{A}}^{\mathcal{B}}}^{-1} \cdot M_{\mathcal{A}} (F) \cdot T_{\mathcal{A}}^{\mathcal{B}}
    \end{align*}
}

\vspace{1\baselineskip}

\Definition{

    Ist $s:V \times V \rightarrow K$ eine symmetrische Bilinearform, so erhält man
    daraus eine Abbildung $q: V \rightarrow K$ gegeben durch $v \mapsto q(v) := s(v,v)$.
    Sie heisst die zu $s$ gehörige \fat{quadratische Form}. Da $s$ bilinear ist, folgt:
    $q(\lambda v) = \lambda^2 q(v)$. Ist insbesondere $V = K^n$ und $s$ durch eine
    symmetrische Matrix $A=(a_{ij})$ gegeben, so ist
    \begin{align*}
        q(x_1,\dots,x_n) = \sum_{i,j = 1}^n a_{ij} x_i x_j = \sum_{i=1}^n a_{ii} x_i^2
        + \sum_{1 \leq i < j \leq n} 2 a_{ij} x_i x_j
    \end{align*}
    das ist ein homogenes quadratisches Polynom.
}

\vspace{1\baselineskip}

\Bemerkung{ (\fat{Polarisierung})

    Ist char$(K) \neq 2$, so gilt für jede symmetrische Bilinearform $s$ und die
    zugehörige quadratische Form $q$ auf $V$
    \begin{align*}
        s(v,w) = \frac{1}{2} (q(v+w) - q(v) - q(w))
    \end{align*}
    Insbesondere ist $s$ also aus $q$ rekonstruierbar.
}

\vspace{1\baselineskip}

\Definition{

    Sei $V$ ein komplexer VR. Eine Abbildung $F: V \rightarrow V$ heisst \fat{semilinear},
    wenn für $v,w \in V$ und $\lambda \in \C$ gilt:
    \begin{align*}
        F(v+\lambda w) = F(v) + \overline{\lambda} F(w)
    \end{align*}
}

\vspace{1\baselineskip}

\Definition{

    Sei $V$ ein komplexer VR. Eine Abbildung $s: V \times V \rightarrow \C$ heisst
    \fat{sesquilinear}, wenn sie linear im ersten Argument ist, und semilinear
    im zweiten, dh.
    \begin{enumerate}[{\fat{S1}}]
        \item $s(v+\lambda v' ,w) = s(v,w) + \lambda s(v',w)$
        \item $s(v,w+\lambda w') = s(v,w) +  \overline{\lambda} s(v,w')$
    \end{enumerate}
    Die Abbildung heisst \fat{hermitisch}, wenn zusätlich
    \begin{enumerate} [{\fat{H}}]
        \item $s(w,v) = \overline{s(v,w)}$
    \end{enumerate}
}

\vspace{1\baselineskip}

\Bemerkung{ (Matrix Schreibweise)

    Ist $\mathcal{A}$ Basis von $V$ und $A = M_{\mathcal{A}} (s) = \klammer{s(v_i,v_j)}_{ij}$,
    $v=\Phi_{\mathcal{A}} (x)$ und $w= \Phi_{\mathcal{A}} (y)$, so ist
    \begin{align*}
        s(v,w) = x^T A \overline{y}
    \end{align*}
    Ist $\mathcal{B}$ eine weitere Basis, $B = M_{\mathcal{B}} (s)$ und
    $T = T_{\mathcal{A}}^{\mathcal{B}}$, so hat man die \fat{Transformationsformel}
    \begin{align*}
        B = T^T A \overline{T}
    \end{align*}
    Weiter ist die Sesquilinearformen $s$ genau dann hermitisch, wenn die Matrix
    $A=M_{\mathcal{A}} (s)$ hermitisch ist, dh.
    \begin{align*}
        A^T = \overline{A} \Longleftrightarrow A = \overline{(A^T)} = A^{\dagger}
    \end{align*}
    Schliesslich hat man auch noch im Komplexen für Sesquilinearformen eine
    Polarisierung
    \begin{align*}
        s(v,w) = \frac{1}{4} \klammer{q(v+w) - q(v-w) + i q(v+iw) - i q(v-iw)}
    \end{align*}
}

\vspace{1\baselineskip}

\Definition{

    Sei $\K = \R$ oder $\K = \C$. Sei $V$ ein $\K$-VR und $s: V \times V \rightarrow \K$
    eine symmetrische Bilinearform (bzw. hermitische Form). Dann heisst $s$
    \fat{positiv definit}, wenn $s(v,v) > 0$ für jedes $v \in V$ mit $v \neq 0$.
    Man beachte, dass auch im hermitischen Fall $s(v,v) \in \R$ ist.
    Eine symmetrische (bzw. hermitische) Matrix $A$ heisst \fat{positiv definiert},
    wenn $x^T A \overline{x} >0$ für jeden Spaltenvektore $x \neq 0$ aus $\K^n$.
}

\vspace{1\baselineskip}

\Definition{

    Zur Abkürzung nennt man eine positiv definite symmetrische Bilinearform
    bzw. hermitische Form ein \fat{Skalarprodukt}, und einen reellen bzw Komplexen
    Vektorraum zusammen mit einem Skalarprodukt einen \fat{euklidischen} ($\K=\R$) bzw
    \fat{unitären} ($\K = \C$) Vektorraum.
}

\vspace{1\baselineskip}

\Definition{

    Sei $V$ ein euklidischer bzw. unitärer VR mit Skalarprodukt $\scalprod{ \ }{ \ }$.
    Wir definieren die \fat{Norm} als $\Norm{v} = \sqrt{\scalprod{v}{v}}$
    und die \fat{Metrik} als $d(v,w) = \Norm{w-v}$
}

\vspace{1\baselineskip}

\Satz{ (Ungleichung von Cauchy-Schwarz)

    Ist $V$ ein euklidischer bzw unitärer VR, so gilt $\forall v,w \in V$
    \begin{align*}
         \abs{\scalprod{v}{w}} \leq \Norm{v} \cdot \Norm{w}
    \end{align*}
    und die Gleichheit gilt genau dann, wenn $v$ und $w$ linear abhängig sind.
}

\vspace{1\baselineskip}

\Definition{

    Sei $V$ ein euklidischer bzw unitärer VR.
    \begin{enumerate}[{a)}]
        \item Zwei Vektoren $v,w \in V$ heissen \fat{orthogonal} $v \perp w \Leftrightarrow \scalprod{v}{w} = 0$
        \item Zwei UVR $U,W \subset V$ heissen \fat{orthogonal} $U \perp W \Leftrightarrow u \perp w \ \forall u \in U \ , \ \forall w \in W$
        \item Ist $U \subset V$ ein UVR, so definiert man sein \fat{orthogonales Komplement}
                $U^{\perp} := \geschwungeneklammer{v \in V \ : \ v \perp u \ \forall u \in U}$.
                Es gilt $U^{\perp} \subset V$.
        \item Eine Familie $(\BasisV)$ in $V$ heisst \fat{orthogonal}, wenn $v_i \perp v_j \ \forall i \neq j$.
                Sie heisst \fat{orthonormal}, falls zusätlich $\Norm{v_i} = 1 \ \forall i$. Und
                \fat{Orthonormalbasis}, falls sie auch eine Basis ist, dh. eine Basis mit $\scalprod{v_i}{v_j} = \delta_{ij}$
        \item Ist $V = V_1 \oplus \dots \oplus V_k$, so heisst die direkte Summe \fat{orthogonal}: $V = V_1 \obot \dots \obot V_k$ falls $V_i \perp V_j \ \forall i \neq j$
    \end{enumerate}
}

\vspace{1\baselineskip}

\Bemerkung{

    Ist $(\BasisV)$ eine orthogonale Familie in $V$ und $v_i \neq 0 \ \forall i$, so gilt:
    \begin{enumerate}[{a)}]
        \item Die Familie $(\alpha_1 v_1 , \dots , \alpha_n v_n$ mit $\alpha_i := \Norm{v_i}^{-1}$ ist orthonormal.
        \item $(\BasisV)$ ist linear unabhängig.
    \end{enumerate}
}

\vspace{1\baselineskip}

\Bemerkung{

    Sei $(\BasisV)$ eine Orthonormalbasis von $V$ und \vinV beliebig. Setze man
    $\lambda_i := \scalprod{v}{v_i}$ so ist $v = \lambda_1 v_1 + \dots + \lambda_n v_n$
}

\vspace{1\baselineskip}

\Satz{ (\fat{Orthonormalisierungssatz})

    Sei $V$ ein endlichdimensionaler euklidischer bzw. unitärer VR und $W \subset V$
    ein UVR mit Orthonormalbasis $(w_1,\dots,w_m)$. Dann gibt es eine Ergänzung zu einer Orthonormalbasis
    $(w_1,\dots,w_m,w_{m+1},\dots,w_n)$ von $V$.
}

\vspace{1\baselineskip}

\Korollar{

    Jeder endlichdimensionaler euklidische bzw. unitäre VR besitzt eine Orthonormalbasis.
}

\vspace{1\baselineskip}

\Korollar{

    Ist $W$ UVR eines euklidischen bzw. unitären VR $V$, so gilt: $V = W \obot W^{\perp}$
    und $\dim V = \dim W + \dim W^{\perp}$.
}

\vspace{1\baselineskip}

\large \fat{Gram-Schmidt Algorithmus} \normalsize

\vspace{1\baselineskip}

Sei eine Basis $(\BasisV)$ von $V$ gegeben. Ziel ist es, eine ONB zu bauen.
Gehe dafür wie folgt vor:
\begin{enumerate}[{1)}]
    \item Setze $w_1 = \frac{1}{\Norm{v_1}} \cdot v_1$
    \item Setze $w_2 = \frac{1}{\Norm{w_2'}} \cdot w_2'$ \ mit  $w_2' = v_2 - \scalprod{v_2}{w_1} \cdot w_1$
    \item ...
    \item Setze $w_n = \frac{1}{\Norm{w_n'}} \cdot w_n'$ \ mit $w_n' = v_n - \scalprod{v_n}{w_1} \cdot w_1 - \dots - \scalprod{v_n}{w_{n-1}} \cdot w_{n-1}$
\end{enumerate}
So erhält man eine ONB $(\BasisW)$ von $V$.

\vspace{1\baselineskip}

\Definition{

    Sei $V$ ein endlichdimensionaler VR und seien $v_1,\dots,v_m \in V$ gegeben, wobei
    $m \leq n$. Dann nennt man
    \begin{align*}
        G(v_1,\dots,v_m) := \det \begin{pmatrix}
            \scalprod{v_1}{v_1} & \dots & \scalprod{v_1}{v_m} \\
            \vdots & \ddots & \vdots \\
            \scalprod{v_m}{v_1} & \dots & \scalprod{v_m}{v_m}
        \end{pmatrix}
    \end{align*}
    die \fat{Gramsche Determinante} von $v_1,\dots,v_m$.
}

\vspace{1\baselineskip}

\Bemerkung{

    Es gilt stets $G(v_1,\dots,v_m) \geq 0$ und $G(v_1,\dots,v_m) > 0 \Leftrightarrow
    (v_1,\dots,v_m)$ linear unabhängig.
}

\vspace{1\baselineskip}

\Definition{ (\fat{Ungleichung von Hadamard})

    Seien $v_1,\dots,v_m$ beliebige Vektoren in einem $n$-dimensionalen Vektorraum $V$ mit
    $m \leq n$. Dann ist
    \begin{align*}
        \vol (v_1,\dots,v_m) \leq \Norm{v_1} \cdot \dots \cdot \Norm{v_m}
    \end{align*}
    und Gleichheit besteht genau dann, wenn die $v_i$ paarweise orthogonal sind, dh.
    wenn der aufgespannte Spat ein Quadrat ist.
}




\section{Orthogonale und unitäre Endomorphismen}

\vspace{1\baselineskip}

\Definition{

    Sei $V$ ein euklidischer bzw. unitärer VR und $F$ ein Endomorphismus von $F$. Dann heisst
    $F$ \fat{Orthogonal} bzw. \fat{unitär}, wenn
    \begin{align*}
        \scalprod{F(v)}{F(w)} = \scalprod{v}{w} \ \forall v,w \in V
    \end{align*}
}

\Bemerkung{

    Ein orthogonales bzw. unitäres $F \in \End (V)$ hat folgende weitere Eigenschaften
    \begin{enumerate} [{a)}]
        \item $\Norm{F(v)} = \Norm{v} \ \forall v \in V$
        \item $v \perp w \Rightarrow F(v) \perp F(w)$
        \item $F$ ist Isomorphismus und $F^{-1}$ ist orthogonal bzw. unitär.
        \item Ist $\lambda \in \K$ Eigenwert von $F$, so ist $\abs{\lambda} = 1$
        \item Sei $G \in \End (V)$ auch orthogonal/unitär, dann ist $F \circ G$ wieder orthogonal/unitär.
    \end{enumerate}
}

\Lemma{

    Ist $F \in \End (V)$ mit $\Norm{F(v)} = \Norm{v} \ \forall v \in V$, so ist $F$
    orthogonal bzw. unitär. 
}

\vspace{1\baselineskip}

\Definition{

    Eine Matrix $A \in \GL (n:\R)$ heisst \fat{orthogonal}, falls $A^{-1} = A^T$ und
    entsprechend heisst $A \in \GL (n;\C)$ \fat{unitär}, wenn $A^{-1} = \overline{(A^T)} = A^{\dagger}$.
    Es gilt, $\abs{\det A}^2 = 1 \Rightarrow \det A = \pm 1$. Man nennt $A$ \fat{eigentlich
    orthogonal}, wenn $\det A = +1$. Das heisst, $A$ ist orientierungstreu. Es gibt drei Gruppen
    von orthogonalen Matrizen
    \begin{enumerate}
        \item \fat{(orthogonale Gruppe)}
        
                $O(n) := \geschwungeneklammer{A \in \GL (n;\R) : \ A^{-1} = A^T}$
        \item \fat{(spezielle orthogonale Gruppe)}
        
                $SO(n) := \geschwungeneklammer{A \in O(n): \ \det A = 1}$
        \item \fat{(unitäre Gruppe)} 
        
            $U(n) := \geschwungeneklammer{A \in \GL (n; \C) : \ A^{-1} = \overline{(A^T)}}$
    \end{enumerate}
}

\vspace{1\baselineskip}

\Bemerkung{

    Für $A \in M(n \times n ; \K)$ sind folgende Bedingungen äquivalent:
    \begin{enumerate}[{i)}]
        \item $A$ ist orthogonal bzw. unitär
        \item Die Spalten von $A$ sind eine Orthogonalbasis von $\K^n$.
        \item Die Zeilen von $A$ sind eine Orthogonalbasis von $\K^n$.
    \end{enumerate}
}

\vspace{1\baselineskip}

\Satz{

    Sei $V$ ein euklidischer, bzw unitärer VR mit einer Orthogonalbasis $\mathcal{B}$ und $F$
    ein Endomorphismus von $V$. Dann gilt:
    $F$ orthogonal (bzw. unitär) $\Leftrightarrow$ $M_{\mathcal{B}} (F)$ orthogonal (bzw. unitär)
}

\vspace{1\baselineskip}

\Lemma{

    Ist $A \in O(2)$, so gibt es ein $\alpha \in [0,2\pi[$, so dass
    \begin{align*}
        A = \begin{pmatrix}
            \cos (\alpha) & - \sin (\alpha) \\
            \sin (\alpha) & \cos (\alpha)
        \end{pmatrix}
        \quad \text{  oder  } \quad
        A = \begin{pmatrix}
            \cos (\alpha) & \sin (\alpha) \\
            \sin (\alpha) & - \cos (\alpha)
        \end{pmatrix}
    \end{align*}
    Im ersten Fall ist $A \in SO(2)$, die Abbildung ist eine \fat{Drehung}. Im zweiten
    Fall ist $\det A = -1$, die Abbildung ist eine \fat{Spiegelung}.
}

\vspace{1\baselineskip}

\Bemerkung{

    Ist $\det A = +1$, so gibt es nur im Fall $\alpha = 0$ oder $\alpha = \pi$ Eigenwerte,
    nämlich $+1$ bzw. $-1$ mit jeweils Vielfachheit $2$. Ist $\det A = -1$, so gibt es
    Eigenwerte $+1$ und $-1$ und die zugehörigen Eigenvektore stehen senkrecht aufeinander.
}

\Bemerkung{

    Sie $F: \R^3 \rightarrow \R^3$ orthogonal. Dan hat $P_F$ Grad $3$ und insbesondere eine
    reelle NST, also hat $F$ einen reellen Eigenwert $\lambda = \pm 1$. Sei $w_1$ ein
    Eigenvektor dazu (o.B.d.A $\Norm{w_1} = 1$). Nun können wir ihn zu einer Orthogonalbasis
    $(w_1 , w_2 , w_3)$ ergänzen. Bezeichnet $W \subset \R^3$ die von $w_2$ und $w_3$
    aufgespannte Ebene, so folgt, dass $F(W) = W$ Also ist
    \begin{align*}
        M_{\mathcal{B}} (F) = \begin{pmatrix}
            \lambda_1 & 0 & 0 \\
            0 & A' & \\
            0 & & 
        \end{pmatrix} =: A
    \end{align*}
    mit $A' \in SO(2)$. Weiter ist $\det A = \lambda_1 \cdot \det A'$. Nun gibt es eine
    Fallunterscheidung: Sei $\det F = \det A = +1$.
    \begin{enumerate}
        \item Ist $\lambda_1 = -1$, so muss $\det A' = -1$ sein. Daher kann man $w_2$ und $w_3$
                als Eigenvektoren zu den Eigenwerten $\lambda_2 = +1$ und $\lambda_3 = -1$
                wählen, dh.
                \begin{align*}
                    A = \begin{pmatrix}
                        -1 & 0 & 0 \\
                        0 & 1 & 0 \\
                        0 & 0 & -1
                    \end{pmatrix}
                \end{align*}
        \item Ist $\lambda_1 = +1$, so muss auch $\det A' = +1$ sein, also gibt es ein
                $\alpha \in [0,2 \pi[$, so dass
                \begin{align*}
                    A = \begin{pmatrix}
                        1 & 0 & 0 \\
                        0 & \cos (\alpha) & - \sin (\alpha) \\
                        0 & \sin (\alpha) & \cos (\alpha)
                    \end{pmatrix}
                \end{align*}
    \end{enumerate}
    Ist $\det F = -1$, so gibt es bei geeigneter Wahl von $w_2$ und $w_3$ für $A$ die
    Möglichkeiten
    \begin{align*}
        \begin{pmatrix}
            1 & 0 & 0 \\
            0 & 1 & 0 \\
            0 & 0 & -1
        \end{pmatrix}
        \quad \text{  und  } \quad
        \begin{pmatrix}
            -1 & 0 & 0 \\
            0 & \cos (\alpha) & - \sin (\alpha) \\
            0 & \sin (\alpha) & \cos (\alpha)
        \end{pmatrix}
    \end{align*}
}

\vspace{1\baselineskip}

\Theorem{

    Jeder unitäre Endomorphismen $F$ eines unitären VR besitzt eine Orthogonalbasis aus
    Eigenvektoren von $F$. Insbesondere ist er diagonalisierbar.
}

\vspace{1\baselineskip}

\Korollar{

    Zu $A \in U(n)$ gibt es ein $S \in U(n)$ mit
    \begin{align*}
        \overline{(S^T)} \cdot A \cdot S = \begin{pmatrix}
            \lambda_1 & & 0 \\
            & \ddots & \\
            0 & & \lambda_n
        \end{pmatrix}
    \end{align*}
    wobei $\lambda_i \in \C$ mit $\abs{\lambda_i} = 1 \ \forall i = 1,\dots,n$.
}

\vspace{1\baselineskip}

\Theorem{

    Ist $F$ ein orthogonaler Endomorphismus eines euklidischen VR $V$, so gibt es in $V$
    eine Orthogonalbasis $\mathcal{B}$ derart, dass für $j=1,\dots,k$:
    \begin{align*}
        M_{\mathcal{B}} (F) = \begin{pmatrix}
            +1 & & & & & & & & \\
            & \ddots & & & & & & & \\
            & & +1 & & & & 0 & & \\
            & & & -1 & & & & & \\
            & & & & \ddots & & & & \\
            & & & & & -1 & & & \\
            & & 0 & & & & A_1 & & \\
            & & & & & & & \ddots & \\
            & & & & & & & & A_k 
        \end{pmatrix}
    \end{align*}
    mit
    \begin{align*}
        A_j = \begin{pmatrix}
            \cos (\vartheta_j) & - \sin (\vartheta_j) \\
            \sin (\vartheta_j) & \cos (\vartheta_j)
        \end{pmatrix}
        \in SO(2)       
    \end{align*}
    mit $\vartheta_j \in [0,2 \pi[$ aber $ \vartheta_j \neq 0 , \pi$
}

\vspace{1\baselineskip}

\Lemma{

    Zu einem orthogonalen Endomorphismus $F$ eines euklidischen VR $V$ mit $\dim V \geq 1$
    gibt es stets einen UVR $W \subset V$ mit
    \begin{align*}
        F(W) \subset W
        \quad \text{  und  } \quad
        1 \leq \dim W \leq 2
    \end{align*}
}


\section{Selbstadjungierte Endomorphismen}

\vspace{1\baselineskip}

\Definition{

    Sei $V$ ein endlichdimensionaler euklidischer bzw. unitärer VR bezeichnet. Zu einem
    $F \in \End (V)$ definieren wir den \fat{adjungierten Endomorphismus} $F^{\text{ad}}$,
    charakterisert durch
    \begin{align*}
        \scalprod{F(v)}{w} = \scalprod{v}{F^{\text{ad}} (w)}
    \end{align*}
}

\vspace{1\baselineskip}

\Bemerkung{

    Es gilt $M_{\mathcal{B}} (F^{\text{ad}}) = \klammer{M_{\mathcal{B}} (F)}^{\dagger}$
}

\vspace{1\baselineskip}

\Definition{

    Ein Endomorphismus $F$ eines euklidischen bzw. unitären VR $V$ heisst \fat{selbstadjungieret},
    wenn
    \begin{align*}
        \scalprod{F(v)}{w} = \scalprod{v}{F(w)} \quad \forall v,w \in V
    \end{align*}
}

\vspace{1\baselineskip}

\Satz{

    Sei $F$ ein Endomorphismus von $V$ und $\mathcal{B}$ eine Orthonormalbasis. Dann gilt:
    $F$ selbstadjungieret $\Leftrightarrow$ $M_{\mathcal{B}} (F)$ symmetrisch bzw. hermitisch.
}

\vspace{1\baselineskip}

\Lemma{

    Ist $F$ selbstadjungieret, so sind (auch im komplexen Fall) alle Eigenwerte reell.
    Insbesondere hat eine hermitische Matrix nur reelle Eigenwerte.
}

\vspace{1\baselineskip}

\Theorem{

    Ist $F$ ein selbstadjungiereter Endomorphismus eines euklidischen bzw. unitären VR, so
    gibt es eine Orthonormalbasis von $V$, die aus Eigenvektoren von $F$ besteht.
    Insbesondere ist $F$ diagonalisierbar.
}

\vspace{1\baselineskip}

\Korollar{

    Ist $A \in M(n \times n ; \K)$ eine symmetrische bzw. hermitische Matrix, so
    gibt es eine orthogonale bzw. unitäre Matrix $S$, so dass
    \begin{align*}
        S^{\dagger} \cdot A \cdot S = \begin{pmatrix}
            \lambda_1 & & 0 \\
            & \ddots & \\
            0 & & \lambda_n
        \end{pmatrix}
    \end{align*}
    mit $\lambda_1,\dots,\lambda_n \in \R$.
}

\vspace{1\baselineskip}

\Korollar{

    Sind $\lambda_1,\dots,\lambda_k$ die verschiedenen Eigenwerte eines selbstadjungierten
    oder unitären Endomorphismus $F$ von $V$, so ist
    \begin{align*}
        V = \text{Eig} (F;\lambda_1) \obot \dots \obot \text{Eig} (F;\lambda_k)
    \end{align*}
}

\vspace{1\baselineskip}

\large \fat{Algorithmus zur Berechnung einer Orthonormalbasis aus Eigenvektoren} \normalsize

\begin{enumerate}[{1)}]
    \item Man bestimme die Linearfaktorzerlegung des charakteristischen Polynoms
            $P_F = \pm (t-\lambda_1)^{r_1} \cdot \dots \cdot (t-\lambda_k)^{r_k}$
            wobei $\lambda_1,...,\lambda_k$ paarweise verschieden sind.
    \item Für jeden Eigenwert $\lambda$ der Vielfachheit $r$ bestimme man eine Basis von
            Eig$(F;\lambda)$ durch Lösen eines linearen Gleichungssystems.
    \item Man orthonormalisiere die in $2)$ erhaltene Basen mit dem Gram-Schmidt Verfahren
            und zwar unabhängig voneinander in den verschiedenen Eigenräumen.
    \item Die $k$ Basen der Eigenräume aus $3)$ bilden zusammen die gesuchte Basis aus
            Eigenvektoren von $V$.
\end{enumerate}

\vspace{1\baselineskip}

\Bemerkung{

    Sei $A \in M(n \times n \ ; K)$. Wenn wie eine Orthonormalbasis $\mathcal{B} = (v_1,\dots,v_n)$
    haben, so gilt
    \begin{align*}
        T:=
        \begin{pmatrix}
            \vdots & & \vdots \\
            v_1 & \dots & v_n \\
            \vdots & & \vdots  
        \end{pmatrix}
        = T_{\mathcal{K}}^{\mathcal{B}}
    \end{align*}
    und
    \begin{align*}
        T^T A T = \begin{pmatrix}
            \lambda_1 & & 0 \\
            & \ddots & \\
            0 & & \lambda_n
        \end{pmatrix}
        := D
    \end{align*}
    oder analog $T \cdot D \cdot T^T = A$.    
}

\vspace{1\baselineskip}

\Lemma{

    Jede symmetrische Matrix $A \in M(n \times n \ ; \R)$ hat einen reellen Eigenwert.
}

\vspace{1\baselineskip}

\Theorem{ (Sazt von Courant-Fischer)

    Sei $V$ ein endlich dimensionaler euklidischer oder unitärer VR mit $n = \dim V$ und
    $F \in \End (V)$ ein selbstadjungiereter Endomorphismus. (Also gilt $\scalprod{x}{Fx}
    = \scalprod{x}{Fx}$.) Seien die Eigenwerte von
    $F$ der Grösse nach geordnet $\lambda_1 \leq \dots \leq \lambda_n$. Dann gilt
    \begin{align*}
        \lambda_k = 
        \min_{\stackrel{U \subset V}{\dim U = k}} \max_{\stackrel{x \in U}{x \neq 0}} \frac{\scalprod{x}{Fx}}{\scalprod{x}{x}}
        \ = \
        \max_{\stackrel{U \subset V}{\dim U = n-k+1}} \min_{\stackrel{x \in U}{x \neq 0}} \frac{\scalprod{x}{Fx}}{\scalprod{x}{x}}
    \end{align*}
    Die erste Extremisierung wird dabei über UVR $U \subset V$ der angegebenen Dimension
    durchgeführt, die zweite über nicht-verschwindende Vektoren in $U$. Man beachte, dass
    für den grössten und kleinsten Eigenwert insbesondere gilt
    \begin{align*}
        \lambda_n = \max_{\stackrel{x \in V}{x \neq 0}} \frac{\scalprod{x}{Fx}}{\scalprod{x}{x}}
        \\
        \lambda_1 = \min_{\stackrel{x \in V}{x \neq 0}} \frac{\scalprod{x}{Fx}}{\scalprod{x}{x}}
    \end{align*}
}


\section{Hauptachsentransformation}

\vspace{1\baselineskip}

\fat{Hauptachsentransformation symmetrischer Matrizen}

\vspace{1\baselineskip}

Sei $A \in M(n \times n \ ; \R)$ symmetrisch und $s$ die durch $A$ beschriebene Bilinearform
auf $\R^n$. Dann gilt:
\begin{enumerate}[{1)}]
    \item Ist $\mathcal{B} = (w_1,\dots,w_n)$ eine Orthonormalbasis des $\R^n$ bezüglich des
            Standardskalarpdoduktes bestehend aus Eigenvektoren des Endomorphimsus $A$, so ist
            \begin{align*}
                M_{\mathcal{B}} (s) = \begin{pmatrix}
                    \lambda_1 & & 0 \\
                    & \ddots & \\
                    0 & & \lambda_n
                \end{pmatrix}
                \quad \text{, dh. } s(w_i , w_j) = \lambda_i \cdot \delta_{ij}
            \end{align*}
            wobei $\lambda_1,\dots,\lambda_n$ die Eigenwerte von $A$ sind.
    \item Es gibt eine Basis $\mathcal{B}'$ des $\R^n$, so dass
            \begin{align*}
                M_{\mathcal{B}'} (s) = \begin{pmatrix}
                    E_k & & 0 \\
                    & - E_k & \\
                    0 & & 0 
                \end{pmatrix}
                = D'
            \end{align*}
            d.h. es gibt ein $T' \in \GL (n;\R)$ mit $D' = (T')^T \cdot A \cdot T^T$
\end{enumerate}

\vspace{1\baselineskip}

\fat{Vorsicht!}
Die Eigenwerte von $A$ sind nur Invarianten des Endomorphimsus, für eine Bilinearform
ist der Begriff Eigenwert sinnlos.

\vspace{1\baselineskip}

\Bemerkung{

    Betrachte die quadratische Gleichung $q(x) = s(x,x) = x^T A x = 1$ mit $x \in \R^n$.
    Die Lösungsmenge nennt man \fat{Quadrik}. Falls $A$ positiv definit ist, so ist dies ein
    Elipsoid. Die Hauptachsen entsprechen gerade den Eigenvektoren von $A$.
}

\vspace{1\baselineskip}

\Korollar{

    Eine symmetrische Matrix $A \in M(n \times n \ ; \R)$ ist genau dann positiv definit,
    wenn ihre Eigenwerte $\lambda_1,\dots,\lambda_n$ positiv sind.
}

\vspace{1\baselineskip}

\Korollar{

    Sei $A \in M(n \times n \ ; \R)$ eine symmetrische Matrix und
    $P_A = (-1)^n t^n + \alpha_{n-1} t^{n-1} + \dots + a_1 t + \alpha_0$ ihr charakteristisches
    Polynom. Dann gilt: $A$ positiv definit $\Leftrightarrow$ $(-1)^j \alpha_j > 0$ für
    $j=0,\dots,(n-1)$.
}

\vspace{1\baselineskip}

\Lemma{

    Das reelle Polynom $f(t) = t^n + \alpha_{n-1} t^{n-1} + \dots + \alpha_1 t + \alpha_0$
    hat reelle NST $\lambda_1 , \dots , \lambda_n$, also
    $f(t) = (t-\lambda_1) \cdot \dots \cdot (t-\lambda_n)$, also
    \begin{enumerate}[{a)}]
        \item $(\lambda_i < 0 \ \forall i) \Leftrightarrow (\alpha_i > 0 \ \forall i)$
        \item $(\lambda_i > 0 \ \forall i) \Leftrightarrow ((-1)^{n-i} \alpha_i > 0 \ \forall i)$
    \end{enumerate}
}

\vspace{1\baselineskip}

\Definition{

    Wir definieren den \fat{Ausartungsraum} von $s$ als:
    \begin{align*}
        V_0 := \geschwungeneklammer{v \in V \ : \ s(v,w) = 0 \ \forall w \in V} \subset V
    \end{align*}
    Wir definieren den \fat{Rang} von $s$ als    
    $\rang (s) := \dim V - \dim V_0$.
}

\vspace{1\baselineskip}

\Bemerkung{

    Ist $V = \R^n$ und $A = M (s)$, so ist
    \begin{align*}
        V_0 &= \ker A = \geschwungeneklammer{v \in V \ | \ A v = 0}
        \\ &= \geschwungeneklammer{v \in V \ | \ w^T A v = 0 \ \forall w \in V}        
    \end{align*}
    und $\rang (A) = \rang (s)$. Ist für allgemeine $V$ die Matrix $A = M (s)$ bzgl irgendeiner
    Basis, so ist $\rang (s) = \rang (A)$.
}

\vspace{1\baselineskip}

\fat{Diagonalisierung einer quadratischen Form}

\vspace{1\baselineskip}

Sei $V$ ein $\R$-VR mit $n:= \dim V$ und $s$ eine symmetrische Bilinearform mit der zugehörigen
quadratischen Form $q$ auf $V$. Dann gibt es eine Basis

$\B = (v_1,\dots,v_k,v_{k+1},\dots,v_r,v_{r+1},\dots,v_n)$

von $V$ mit folgenden Eigenschaften: Es ist $r = \rang (s)$ und für
\begin{align*}
    v=\sum_{i=1}^n \alpha_i v_i \in V
    \quad \text{  gilt  } \quad
    q(v) = \sum_{i=1}^k \alpha_i^2 - \sum_{i=k+1}^r \alpha_i^2
\end{align*}
Insbesondere hat man eine Zerlegung $V = V_+ \oplus V_- \oplus V_0$ mit $q(v) > 0$ für
$0 \neq v \in V_+$, $q(v)<0$ für $0 \neq v \in V_-$ und $q(v)=0$ für $v \in V_0$.

\vspace{1\baselineskip}

\Satz{ (Trägheitssatz von Sylvester)

    Hat man für eine quadratische Form $q$ auf einem $\R$-VR $V$ zwei Zerlegungen
    $V = V_+ \oplus V_- \oplus V_0 = V_+' \oplus V_-' \oplus V_0'$ mit
    $q(v)>0$ für $0 \neq v \in V_+$ und $0 \neq v \in V_+'$, sowie
    $q(v)<0$ für $0 \neq v \in V_-$ und $0 \neq v \in V_-'$ so folgt:
    $\dim V_+' = \dim V_+$ und $\dim V_-' = \dim V_-$.

    Die Zahlen $r_+ (q) := \dim V_+$ und $r_- (q) := \dim V_-$ sind also neben dem Rang weitere
    Invarianten der quadratischen Form. Das Paar $(r_+ (q) , r_- (q))$ wird auch
    \fat{Signatur} von $q$ genannt.
}

\vspace{1\baselineskip}

\Korollar{

    Sei $A \in M(n \times n \ ;\R)$ symmetrisch und $S \in \GL (n:\R)$. Dann haben
    $A$  und $S^T A S$ mit Vielfachheit gezählt die gleichen Anzahl positiver und negativer
    Eigenwerte. Insbesondere ist $S^T S = S^T E_n S$ positiv definit.
}

\vspace{1\baselineskip}

\fat{Vorsicht!}
Beachte, dass die Eigenwerte von $S^{-1} A S$ gleich denen von $A$ sind. Die von $S^T A S$
sind im Allgemeinen unterschiedlich, aber das Korollar sagt, dass zumindest die Vorzeichen
erhalten bleiben.

\vspace{1\baselineskip}

\Satz{ \fat{(Orthogonalisierungssatz)}

    Sei $V$ ein endlichdimensionaler VR über einem Körper $K$ mit char$(K) \neq 2$ und
    $s: V \times V \rightarrow K$ eine symmetrische Bilinearform. Dann gibt es eine
    Basis $\B = (v_1,\dots,v_n)$ von $V$ mit $s(v_i , v_j) = 0$ für $i \neq j$.
}

\vspace{1\baselineskip}

\Bemerkung{

    Ist $q: V \rightarrow K$ die zur Bilinearform $s: V \times V \rightarrow K$ gehörige
    quadratische Form, und $\B = (v_1,\dots,v_n)$ wie im obigen Satz, so folgt, mit
    $\alpha_j = q(v_j)$, dass für $v=x_1 v_1 + \dots + x_n v_n \in V$ gilt:
    $q(v) = \alpha_1 x_1^2 + \dots + \alpha_n x_n^2$. Die $\alpha_j$ sind dabei nicht
    eindeutig durch $s$ festgelegt. Ersetzt man zB. $v_j$ durch $\beta v_j$, so muss man
    $\alpha_j = q(v_j)$ ersezen durch $\beta^2 \alpha_j$. Insbesondere, falls $K = \C$,
    kann man so alle $\alpha_j$ zu $1$ oder $0$ werden lassen.
}

\vspace{1\baselineskip}

\Korollar{

    Zu einer symmetrischen Matrix $A \in M(n \times n \ ; K)$ gibt es ein $S \in \GL(n;K)$,
    so dass
    \begin{align*}
        S^T A S = \begin{pmatrix}
            \alpha_1 & & 0 \\
            & \ddots & \\
            0 & & \alpha_n
        \end{pmatrix}
    \end{align*}
}

\vspace{1\baselineskip}

\large \fat{Algorithmus zur Bestimmung von $S$ und $D$ im reellen Fall} \normalsize

\begin{enumerate}[{1)}]
    \item Wir schreiben die Matrix $A$ und die Einheitsmatrix $E_n$ übereinander. Also $\frac{A}{E_n}$
    \item Wir führen in beiden Matrizen jeweils elementare Spaltenumformung durch, und dann,
            \underline{nur in A} noch die zugehörige Zeilentransformation. Dies ergibt mit
            $C_1 , \dots , C_r$ die entsprechenden Elementarpatrizen.
            $\frac{C_r^T \cdot \dots C_1^T \cdot A \cdot C_1 \cdot \dots \cdot C_r}{C_1 \cdot \dots \cdot C_r}
            = \frac{D}{S}$
    \item Hat man die Diagonalmatrix $D$ erzeugt, so gilt
    
            $S^T \cdot A \cdot S = D$
\end{enumerate}

\vspace{1\baselineskip}

\Bemerkung{

    Sei $A \in M(n \times n \ ; \R)$ und $S \in \GL(n;\R)$, so dass
    \begin{align*}
        S^T A S = \begin{pmatrix}
            \alpha & & 0 \\
            & \ddots & \\
            0 & & \alpha_n
        \end{pmatrix}
    \end{align*}
    so folt: $A$ positiv definit $\Leftrightarrow$ $\alpha_i >0 \ \forall i = 1,\dots,n$.
}

\vspace{1\baselineskip}

\Definition{

    Sei $A_k$ die linke obere $k$-reihige und $k$-spaltige Teilmatrix von $A$. Ihre
    Determinante $\det A_k$ heisst \fat{Hauptminor} von $A$.
}

\vspace{1\baselineskip}

\Satz{ \fat{(Hauptminoren-Kriterium für Definitheit)}

    Für eine symmetrische Matrix $A \in M(n \times n \ ; \R)$ gilt:

    $A$ positiv definit $\Leftrightarrow$ $\det A_k > 0$ für $k = 1,\dots,n$
}

\vspace{1\baselineskip}

\Theorem{

    Sei $A \in M(n \times n \ ; \C)$ eine hermitische Matrix, also $A^{\dagger} =
    \overline{(A^T)} = A$. Sei $s$ die entsprechende hermitische Sesquilinearform auf dem
    $\C^n$, also $s(x,y) = x^T A \overline{y}$. Dann gilt:
    \begin{enumerate}[{(1)}]
        \item Es gibt eine Orthonormalbasis $\B$ des $\C^n$ (bzgl. des kanonischen Skalarproduktes),
                so dass
                \begin{align*}
                    M_{\B} (s) = \begin{pmatrix}
                        \lambda_1 & 0 & \dots & 0 \\
                        0 & \lambda_2 & \ddots & \vdots \\
                        \vdots & \ddots & \ddots & 0 \\
                        0 & \dots & 0 & \lambda_n
                    \end{pmatrix}
                    =: D
                \end{align*}
                diagonal ist, mit $\lambda_1,\dots,\lambda_n \in \R$ den Eigenwerten von $A$.
                Äquivalent in Matrixform: Es gibt eine unitäre Matrix $U \in U(n)$ so dass
                $U^T A \overline{U}$.
        \item Es gibt eine Basis $\B'$ des $\C^n$ und Zahlen $k,l$, so dass
                    \begin{align*}
                        M_{\B'} (s) = \begin{pmatrix}
                            E_k & 0 & 0 \\
                            0 & - E_l & 0 \\
                            0 & 0 & 0
                        \end{pmatrix}
                    \end{align*}
                    Äquivalent, in Matrixform: Es gibt eine invertierbare Matrix $S \in \GL(n,\C)$
                    so dass $S^T A \overline{S} = D'$.
        \item Die Zahlen $k,l$ aus $(2)$ sind die Anzahl der positiven bzw negativen Eigenwerte von
                $A$, mit Vielfachheit gezählt, und unabhängig von der Wahl von $\B'$. Alternativ,
                in Matrixform: Für jede invertierbare Matrix $S \in \GL(n,\C)$ sodass $S^T A \overline{S}$
                die diagonal ist, befinden sich genau $k$ positive und genau $l$ negative Enträge auf der
                Diagonalen von $S^T A \overline{S}$.
    \end{enumerate}
}

\vspace{1\baselineskip}

\Theorem{

    Sei $A \in M(n \times n , \C)$ eine hermitische Matrix. Dann sind äquivalent:
    \begin{enumerate}[{(1)}]
        \item $A$ ist positiv definit, also $x^T A \overline{x} > 0$ für alle $0 \neq x \in \C^n$
        \item Alle Eigenwerte von $A$ sind positiv.
        \item Es gibt eine Matrix $S \in \GL(n,\C)$, so dass $S^T A \overline{S}$ diagonal ist mit
            positiven Diagonaleinträgen.
        \item Es gibt eine Matrix $T \in \GL(n,\C)$ so dass $A = T^{\dagger} T$
        \item Sei $P_A (t) = (-t)^n + \alpha_{n-1} t^{n-1} + \dots + \alpha_1 t + \alpha_0 \in \R [t]$
            das charakteristische Polynom von $A$. Dann gilt $(-1)^j \alpha_j > 0$ für $j=0,\dots,n-1$
        \item Sei $A_k \in M(k \times k , \C)$ die $k \times k$ Untermatrix von $A$ "oben links".
            Dann gilt $\det A_k > 0$ für $k=1,\dots,n$
    \end{enumerate}
}

\large \fat{Algorithmus zur Bestimmung von $S$ und $D$ im komplexen Fall} \normalsize

\vspace{1\baselineskip}

Analog zum reellen Fall, nur, dass man bei der Zeilentransformation von $A$ jeweils
die Koeffizienten komplex konjugieren muss.

\vspace{1\baselineskip}

\Definition{

    Man nennt eine Bilinearform bzw. Sesquilinearform $s$ auf dem $\K$-VR $V$ \fat{positiv
    semidefinit}, falls $s(v,v) \geq 0 \ \forall v \in V$. Man nennt $s$ \fat{negativ definit},
    falls $s(v,v) < 0 \ \forall v \neq 0 \in V$ und negativ semidefinit, falls $s(v,v) \leq 0
    \forall v \in V$. Eine symmetrische bzw. hermitische Matrix $A$ ist positiv oder negativ
    (semi)definit, falls entsprechendes für die zugehörige Bilinearform $s(x,y) =
    x^T A \overline{y}$ auf $\K^n$ gilt. Es gilt:
    \begin{enumerate}[{(1)}]
        \item $A$ negativ definit $\Leftrightarrow$ $-A$ positiv definit
        \item $A$ negativ semidefinit $\Leftrightarrow$ $-A$ positiv semidefinit
    \end{enumerate}
}


\section{Dualräume}

\Definition{

    Ist $V$ ein $K$-VR, so heisst
    \begin{align*}
        V^{*} := \Hom_K (V,K) = \geschwungeneklammer{\varphi:V \rightarrow K: \ \varphi \text{ linear}}
    \end{align*}
    der \fat{Dualraum} von $V$. Die Elemente von $V^{*}$ heissen Linearformen auf $V$.
    Sei $\B = (v_1,\dots,v_n)$ eine Basis von $V$. So gibt es zu jedem $i \in \geschwungeneklammer{1,\dots,n}$
    genau eine lineare Abbildung $v_i^{*} : V \rightarrow K$ mit $v_i^{*} (v_j) = \delta_{ij}$.
}

\vspace{1\baselineskip}

\Bemerkung{

    Für jede Basis $\B = (v_1,\dots,v_n)$ von $V$ ist $\B^{*} = (v_1^{*} ,\dots,v_n^{*})$
    eine Basis von $V^{*}$. Man nennt $\B^{*}$ die zu $\B$ \fat{duale Basis}.
}

\vspace{1\baselineskip}

\Korollar{

    Zu jedem \vinV mit $v \neq 0$ gibt es ein $\varphi \in V^{*}$ mit $\varphi(v) \neq 0$.
}

\vspace{1\baselineskip}

\Korollar{

    Zu jeder Basis $\B = (v_1,\dots,v_n)$ von $V$ gibt es einen Isomorphismus
    $\Psi_{\B}: V \rightarrow V^{*}$ mit $\Psi_{\B}(v_i) = v_i^{*}$.
}

\fat{Vorsicht!}
Dieser Isomorphismus hängt von der Auswahl der Basis ab, ebenso ist $\varphi$ aus dem ersten
Korollar abhängig von der Ergänzung.

\vspace{1\baselineskip}

\Definition{

    Ist $V$ ein $K$-VR und $U \subset V$ ein UVR, so heisst
    \begin{align*}
        U^0 := \geschwungeneklammer{\varphi \in V^{*} : \varphi (u) = 0 \ \forall u \in U} \subset V^{*}
    \end{align*}
    der zu $U$ \fat{orthogonale Raum} (oder der \fat{Annullator} von $U$).
}

\vspace{1\baselineskip}

\Satz{

    Für jeden UVR $U \subset V$ gilt $\dim U^0 = \dim V - \dim U$. Genauer gilt:
    Ist $(u_1,\dots,u_k)$ Basis von $U$ und $B=(u_1,\dots,u_k,v_1,\dots,v_r)$ Basis von $V$,
    so bilden die Linearformen $v_1^{*},\dots,v_r^{*}$ aus $\B^{*}$ eine Basis von $U^0$.
}

\vspace{1\baselineskip}

\Definition{}{

    Seien $V$ und $W$ $K$-VR und seien $F$ und $\psi$ lineare Abbildungen. Es gelte
    \begin{center}
        \begin{tikzcd}
            V \arrow{r}{F} \arrow{dr}[swap]{\psi \circ F} & W \arrow{d}{\psi} \\
            & K
        \end{tikzcd}
    \end{center}
    Dann ist $\psi \in W^{*}$ und es folgt $\psi \circ F \in V^{*}$. Also können eine
    \fat{duale Abbildung} $F^{*}: W^{*} \rightarrow V^{*}$ mit $\psi \mapsto F^{*}(\psi)
    := \psi \circ F$ erklären. Aus
    \begin{align*}
        F^{*}(\lambda_1 \psi_1 + \lambda_2 + \psi_2) &= (\lambda_1 \psi_1 + \lambda_2 \psi_2) \circ F
        \\
        &= \lambda(\psi_1 \circ F) + \lambda_2 (\psi_2 \circ F)
        \\
        &= \lambda F^{*} (\psi_1) + \lambda_2 F^{*} (\psi_2)
    \end{align*}
    folgt die Linearität von $F^{*}$. Also hat man noch abstrakter eine Abbildung
    $\Hom_K (V,W) \rightarrow \Hom_K (W^{*},V{*})$ mit $F \mapsto F^{*}$, die ein
    Vektorraumisomorphismus ist.
}

\vspace{1\baselineskip}

\Satz{

    Gegeben seien $K$-VR $V$ und $W$ mit Basen $\mathcal{A}$ und $\mathcal{B}$, sowie eine
    lineare Abbildung $F: V \rightarrow W$. Dann gilt 
    \begin{align*}
        M_{\mathcal{A}^{*}}^{\B^{*}} (F^{*}) = \klammer{M_{\mathcal{A}}^{\B} (F)}^T 
    \end{align*}     
    Kurz ausgedrückt: Die duale Abbildung wird
    bezüglich der dualen Basen durch die transponierte Matrix beschriebe.
}

\vspace{1\baselineskip}

\Satz{

    Ist $F: V \rightarrow W$ eine lineare Abbildung zwischen endlichdimensionalen VR, so
    gilt
    \begin{align*}
        \Im F^{*} = \klammer{\ker F}^0
        \quad \text{  und  } \quad
        \ker F^{*} = \klammer{\Im F}^{0}
    \end{align*}
}

\vspace{1\baselineskip}

\Korollar{

    Unter den obigen Voraussetzungen gilt $\rang F^{*} = \rang F$.
}

\vspace{1\baselineskip}

\Korollar{

    Für jede Matrix $A \in M(n \times n ; K)$ gilt: Zeilenrang $A = $ Spaltenrang $A$.
}

\vspace{1\baselineskip}

\Definition{

    Den Dualraum kann man zu jedem VR bilden, also auch zu $V^{*}$. Auf diese Weise erhält man
    zu $V$ den \fat{Bidualraum} $V^{**} := (V^{*})^{*} = \Hom (V^{*},K)$.
}

\vspace{1\baselineskip}

\Satz{

    Für jeden endlichdimensionalen $K$-VR $V$ ist die kanonische Abbildung
    $\iota: V \rightarrow V^{**}$ ein Isomorphismus. Man kann also $V$ mit $V^{**}$
    identifizieren und in suggestiver Form $V(\varphi) = \varphi(v)$.
}

\vspace{1\baselineskip}

\Korollar{

    Für jede lineare Abbildung $F: V \rightarrow W$ gilt $F^{**} = F$.
    (Sofern man $V$ und $V^{**}$ und $W$ und $W^{**}$ jeweils mit $\iota$ identifizieren kann.)
}

\vspace{1\baselineskip}

\Bemerkung{

    Für jeden UVR $W \subset V$ gilt $(W^0)^0 = W \subset V \cong V^{**}$.
}


\section{Dualität und Skalarprodukte}
\renewcommand{\phi}{\varphi}

\vspace{1\baselineskip}

\Definition{

    Eine Abbildung $\phi: V_1 \times V_k \rightarrow W$ heisst \fat{multilinear}
    (oder \fat{$k$-linear}), wenn sie linear in jedem Argument ist, also

    $\phi(v_1,\dots,v_i' + \lambda v_i'',\dots,v_n) = \phi(v_1,\dots,v_i,\dots,v_n) +
    \lambda \phi(v_1,\dots,v_i'',\dots,v_n)$

    Im Fall $k=2$ heisst $\phi$ bilinear. Für ein bilineare $b: V \rightarrow W \rightarrow K$
    haben wir lineare Abbildungen $b_v : W \rightarrow K$ mit $w \mapsto b(v,w)$ für ein fixes
    $v \in V$ und $b_w : V \rightarrow K$ mit $v \mapsto b(v,w)$ für ein fixes $w \in W$.

    Die Abbildungen $v \mapsto b_v$ und $w \mapsto b_w$ sind ihrerseits linear, sodass man
    lineare Abbildungen erhält:

    $b': V \rightarrow W^{*}$ mit $v \mapsto b_v \in W^{*}$

    $b'': W \rightarrow V^{*}$ mit $w \mapsto b_w \in V^{*}$
}

\vspace{1\baselineskip}

\Satz{

    Sind $V$ und $W$ endlichdimensional und ist $b: V \times W \rightarrow K$ eine
    nicht ausgeartete Bilinearform, so sind die folgenden Abbildungen
    Isomorphismen.

    $b': V \rightarrow W^{*}$ und $b'': W \rightarrow V^{*}$
}

\vspace{1\baselineskip}

\Korollar{

    In einem euklidischen VR $V$ ist die Abbildung $\Psi: V \rightarrow V^{*}$ mit
    $v \mapsto \scalprod{v}{ \ }$ ein kanonischer Isomorphismus.
}

\vspace{1\baselineskip}

\Satz{

    Sei $V$ ein euklidischer VR und $\Psi: V \rightarrow V^{*}$ der kanonische
    Isomorphismus. Dann gilt:
    \begin{enumerate}[{1)}]
        \item Für jeden UVR $U \subset V$ ist $\Psi (U^{\perp}) = U^0$ mit $U^{\perp} \subset V$
        \item Ist $\B=(v_1,\dots,v_n)$ eine Orthonormalbasis von $V$ und
                $\B^{*}=(v_1^{*},\dots,v_n^{*})$ die duale Basis, so ist $\Psi(v_i) = v_i^{*}$
    \end{enumerate}
}

\vspace{1\baselineskip}

\Definition{}{

    Seien $V$ und $W$ euklidische VR, und sei $F:V \rightarrow W$ eine lineare Abbildung.
    Dazu konstruiren wir eine lineare Abbildung $F^{\text{ad}}:W \rightarrow V$ mit
    $\scalprod{F(v)}{w} = \scalprod{v}{F(w)} \ \forall v \in V$ und $\forall w \in W$.
    Sind $\Phi$ und $\Psi$ die kanonischen Isomorphismen, so ist dies äquivalent dazu, dass
    folgendes Diagramm kommutiert.
    \begin{center}
        \begin{tikzcd}
            V \arrow{d}[swap]{\Phi} & W \arrow{l}[swap]{F^{\text{ad}}} \arrow{d}{\Psi} \\
            V^{*} & W^{*} \arrow{l}{F^{*}}
        \end{tikzcd}
    \end{center}
    dh. $F^{\text{ad}} = \Phi^{-1} \circ F^{*} \circ \Psi$. Es gilt nämlich:

    $\Psi (w) = \scalprod{ \ }{w}$ \ also \ $F^{*} (\Psi (w)) = \scalprod{F( \ )}{w}$ \ und

    $F^{*} (\Psi (w)) = \Psi (F^{\text{ad}} (w)) = \scalprod{ \ }{F^{\text{ad}} (w)}$
}

\vspace{1\baselineskip}

\Bemerkung{

    Sind $V$ und $W$ euklidische VR mit Orthonormalbasen $\mathcal{A}$ und $\B$, so gilt
    für jede lineare Abbildung $F: V \rightarrow W$: $M_{\mathcal{A}}^{\B} (F^{\text{ad}})=
    \klammer{M_{\mathcal{A}}^{\B} (F)}^T $
}

\vspace{1\baselineskip}

\Satz{

    Ist $F: V \rightarrow W$ eine lineare Abbildung zwischen euklidischen Vektorräumen,
    so gilt $\Im F^{\text{ad}} = (\ker F)^{\perp}$ und $\ker F^{\text{ad}} = (\Im F)^{\perp}$.
    Insbesondere hat man im Fall $V = W$ orthogonale Zerlegungen $V = \ker F \obot \Im F^{\text{ad}}
    = \ker F^{\ad} \obot \Im F$. Ist überdies $F$ selbstadjungiert, dh. $F = F^{\ad}$, so gilt:
    $V = \ker F \obot \Im F$
}

\vspace{1\baselineskip}

\Definition{

    Sei $V$ ein $\C$-VR. Sei eine sesquilineare Abbildung $s: V \times V \rightarrow \C$
    gegeben. So erhält man wegen der Linearität im ersten Argument eine Abbildung
    $s'': V \rightarrow V^{*}$ mit $v \mapsto \scalprod{ \ }{v}$. Ist $V$ ein unitärer VR und
    $F \in \End (V)$, so ist die adjungierte Abbildung $F^{\ad} := \Psi^{-1} \circ F^{*}
    \circ \Psi$ wieder $\C$-linear.
}

\vspace{1\baselineskip}

\Satz{

    Sei $F$ ein Endomorphismus eines unitären VR $V$. Der dazu adjungierte Endomorphismus
    $F^{\ad}$ hat folgende Eigenschaften:
    \begin{enumerate}[{1)}]
        \item $\scalprod{F(v)}{w} = \scalprod{v}{F^{ad} (w)} \ \forall v,w \in V$
        \item $\Im F^{\ad} = (\ker F)^{\perp}$ und $\ker F^{\ad} = (\Im F)^{\perp}$
        \item Ist $\B$ eine Orthonormalbasis von $V$, so gilt $M_{\B} (F^{\ad}) =
                \klammer{M_{\B} (F)}^{\dagger} = \overline{((M_{\B} (F))^T)}$
    \end{enumerate}
}

\vspace{1\baselineskip}

\Bemerkung{

    Sowohl im reellen als auch im komplexen Fall gilt $(F^{\ad})^{\ad} = F$.
    Insbesondere gilt auch $\scalprod{v}{F(w)} = \scalprod{F^{\ad} (v)}{w} \ \forall v,w$
}

\vspace{1\baselineskip}

\Definition{

    Ein Endomorphismus $F$ eines unitären VR $V$ heisst \fat{normal}, wenn
    $F \circ F^{\ad} = F^{\ad} \circ F$. Entsprechend heisst eine Matrix $A \in M(n \times n; \C)$
    \fat{normal}, wenn $A \cdot A^{\dagger} = A^{\dagger} \cdot A$
}

\vspace{1\baselineskip}

\Bemerkung{

    Jedes unitäre $F$ ist normal und jedes selbstadjungierte $F$ ist normal.
}

\vspace{1\baselineskip}

\Satz{

    Ist $V$ unitär und $F \in \End (V)$ normal, so gilt
    $\ker F^{\ad} = \ker F$ und $\Im F^{\ad} = \Im F$.
    Insbesondere hat man eine orthogonale Zerlegung $V = \ker F \obot \Im F$.
}

\vspace{1\baselineskip}

\Korollar{

    Ist $F$ normal, so ist
    $\Eig (F;\lambda) = \Eig (F^{\ad} ; \overline{\lambda}) \ \forall \lambda \in \C$.
}

\vspace{1\baselineskip}

\Theorem{ \fat{(Spektralsatz)}

    Für einen Endomorphismus $F$ eines unitären $\C$-VR $V$ sind folgende Bedingungen
    äquivalent:
    \begin{enumerate}[{i)}]
        \item Es gibt eine Orthonormalbasis von $V$ bestehend aus Eigenvektoren von $F$.
        \item $F$ ist normal.
    \end{enumerate}
}

\vspace{1\baselineskip}

\Korollar{

    Ein $A \in M(n \times n ; \C)$ ist genau dann normal, wenn es ein $S \in U(n)$ gibt, so
    dass $S A S^{-1}$ eine Diagonalmatrix ist.
}


\section{Tensorprodukt}

\vspace{1\baselineskip}

\Bemerkung{

    Seien $V$ bzw. $W$ Vektorräume über $K$ mit Basen $(v_i)_{i \in I}$ bzw. $(w_j)_{j \in J}$.
    Ist $U$ ein weiterer $K$-VR, so gibt es zu einer beliebigen vorgegebenen Familie
    $(u_{ij})_{(i,j) \in I \times J}$ in $U$ genau eine bilineare Abbildung
    $\xi: V \times W \rightarrow U$ \ mit \ $\xi (v_i , w_j) = u_{ij}$ \  $\forall (i,j) \in I \times J$
}

\vspace{1\baselineskip}

\Theorem{}{

    Seien $V$ und $W$ Vektorräume über $K$. Dann gibt es einen $K$-VR \
    $V \otimes_K W$ zusammen mit einer bilinearen Abbildung \
    $\eta : V \times W \rightarrow V \times_K W$ \ die folgende universelle Eigenschaft
    haben: Zu jedem $K$-VR $U$ zusammen mit einer bilinearen Abbildung
    $\xi : V \times W \rightarrow U$ gibt es genau eine lineare Abbildung
    $\xi_{\otimes} : V \otimes_K W \rightarrow U$ mit $\xi = \xi_{\otimes} \circ \eta$.
    Das kann man durch ein kommutatives Diagramm illustrieren:
    \begin{center}
        \begin{tikzcd}
            V \times W \arrow{d}[swap]{\eta} \arrow{rd}{\xi} & \\
            V \otimes_K W \arrow{r}[swap]{\xi_{\otimes}} & U
        \end{tikzcd}
    \end{center}
    Weiter gilt: Falls $\dim V , \dim W < \infty$, so ist
    
    $\dim (V \otimes_K W) = (\dim V) \cdot (\dim W)$.

    Falls klar ist, welches $K$ Grundkörper ist, schreibt man nur $\otimes$ statt $\otimes_K$.
    Man nennt $V \otimes_K W$ das \fat{Tensorprodukt} von $V$ und $W$ über $K$. Die Elemente
    von $V \otimes_K W$ heissen \fat{Tensoren}.
}

\vspace{1\baselineskip}

\Korollar{ \fat{(Rechenregeln für Tensoren)}

    Ist $\eta: V \times W \rightarrow V \otimes W$ und $v \otimes w := \eta (v,w)$, so gilt
    für $v,v' \in V$, $w,w' \in W$ und $\lambda \in K$:
    \begin{enumerate}[{a)}]
        \item $v \otimes w + v' \otimes w = (v+v') \otimes w$ und $v \otimes w + v \otimes w' = v \otimes (w + w')$
        \item $(\lambda \cdot v) \otimes w = v \otimes (\lambda \cdot w) = \lambda \cdot (v \otimes w)$
    \end{enumerate}
}

\vspace{1\baselineskip}

\Bemerkung{}{

    Der VR $V \otimes W$ ist durch die universelle Eigenschaft eindeutig bis auf einen
    eindeutigen Isomorphismus bestimmt. Sei $V \tilde{\otimes} W$ und $\tilde{\eta}:
    V \times W \rightarrow V \tilde{\otimes} W$ ein anderes Paar aus $VR$ und bilinearen
    Abbildungen, das die universelle Eigenschaft erfüllt. Dann gibt es Abbildungen:
    \begin{center}
        \begin{tikzcd}
            & V \times W \arrow{ld}[swap]{\tilde{\eta}} \arrow{rd}{\eta} & \\
            V \tilde{\otimes} W \arrow[rr, shift right, swap, "\exists ! \tilde{\eta}_{\otimes}"] & &
            V \otimes W \arrow[ll, shift right, swap,"\exists ! \eta_{\otimes}"]
        \end{tikzcd}
    \end{center}
    Betrachte:
    \begin{center}
        \begin{tikzcd}
            & V \times W \arrow{ld}[swap]{\eta} \arrow{rd}{\eta} & \\
            V \otimes W \arrow[rr, shift left, "\text{id}"] \arrow[rr, shift right, swap, "\eta_{\otimes} \circ \tilde{\eta}_{\otimes}"] & &
            V \otimes W 
        \end{tikzcd}
    \end{center}
    Es ist $\eta_{\otimes} \circ \tilde{\eta}_{\otimes} \circ \eta = \eta_{\otimes} \circ \tilde{\eta} = \eta$.
    Wegen der Eindeutigkeit der universellen Eigenschaft gilt also
    $\eta_{\otimes} \circ \tilde{\eta}_{\otimes} = \id_{V \tilde{\otimes} W}$.
    Man spricht deswegen normalerweise von \underline{dem} Tensorprodukt $V \otimes W$
    von $V$ und $W$.
}

\vspace{1\baselineskip}

\Definition{

    Sei $s: V \times W \rightarrow K$ eine Bilinearform. Schreibe $\Bil (V,W;K)$ für den
    VR solcher bilinearer Abbildungen, also $s \in \Bil (V,W;K)$. Wir erhalten die
    lineare Abbildung $s_{\otimes} : V \otimes W \rightarrow K$ mit $s(v,w) = s(v \otimes w)$.
    Also $s_{\otimes} \in (V \otimes W)^{*}$. Die Abbildung $\Bil (V,W;K) \rightarrow (V \otimes W)^{*}$
    mit $s \mapsto s_{\otimes}$ ist wieder linear. Sei umgekehrt $\phi \in (V \otimes W)^{*}$,
    dann ist $\phi \circ \eta \in \Bil (V,W;K)$ mit $\eta: V \times W \rightarrow V \otimes W$
    bilinear. Die Abbildungen $\Bil (V,W;K) \rightarrow (V \otimes W)^{*}$ mit $s \mapsto s_{\otimes}$
    und $(V \otimes W)^{*} \rightarrow \Bil (V,W;K)$ mit $\phi \mapsto \phi \circ \eta$ sind
    zueinander invers.
}

\vspace{1\baselineskip}

\Bemerkung{

    Für zwei Linearformen $\phi \in V^{*}$ und $\psi \in W^{*}$ ist das Produkt
    \begin{align*}
        \phi \cdot \psi : V \times W \rightarrow K \quad \text{mit} \quad (v,w) \mapsto \phi (v) \cdot \psi (v)
    \end{align*}
    eine Bilinearform, die zugehörige Abbildung
    \begin{align*}
        (\phi \cdot \psi)_{\otimes} : V \otimes W \rightarrow K \quad \text{ mit } \quad
        v \otimes w \mapsto \phi(v) \cdot \psi (w)
    \end{align*}
    ist linear, also ist $(\phi \cdot \psi)_{\otimes} \in (V \otimes W)^{*}$.
    Die so entstandene Abbildung
    \begin{align*}
        V^{*} \times W^{*} \rightarrow (V \otimes W)^{*} \quad \text{ mit } \quad
        (\phi,\psi) \mapsto (\phi \cdot \psi)_{\otimes}
    \end{align*}
    ist wiederum bilinear, dh. sie wird zu einer linearen
    Abbildung
    \begin{align*}
        V^{*} \otimes W^{*} \rightarrow (V \otimes W)^{*} \quad \text{ mit } \quad
        \phi \otimes \psi \mapsto (\phi \cdot \psi)_{\otimes}
    \end{align*}
    Das $\phi \otimes \psi$ eine Linearform auf $V \otimes W$ erklärt, kann man auch sehen:
    \begin{align*}
        (\phi \otimes \psi)(v \otimes w) := \phi (v) \cdot \psi (w)
    \end{align*}
}

\vspace{1\baselineskip}

\Bemerkung{

    Die durch Multiplikation mit Skalaren in $W$ erhaltene Abbildung
    \begin{align*}
        V^{*} \times W \rightarrow \Hom (V,W) \quad &\text{mit} \quad
        (\phi , w) \mapsto \phi ( \ ) \cdot w
        \\
        \text{wobei} \quad \phi ( \ ) \cdot w : V \rightarrow W \quad &\text{mit} \quad
        v \mapsto \phi (v) \cdot w
    \end{align*}
    ist bilinear, dazu gehört die lineare Abbildung
    \begin{align*}
        V^{*} \otimes W \rightarrow \Hom (V,W) \quad \text{ mit } \quad
        \phi \otimes w \mapsto \phi ( \ ) \cdot w
    \end{align*}
    Das kann man wieder kürzer ausdrücken durch
    \begin{align*}
        (\phi \otimes w)(v) := \phi (v) \cdot w
    \end{align*}
}

\vspace{1\baselineskip}

\Satz{

    Sind $V$ und $W$ endlich-dimensionale VR, so sind die beiden kanonischen Abbildungen
    \begin{align*}
        &\alpha: V^{*} \otimes W^{*} \rightarrow (V \otimes W)^{*}
        \quad \text{und} \quad \\
        &\beta: V^{*} \otimes W^{*} \rightarrow \Hom (V,W)
    \end{align*}
    Isomorphismen.
}

\vspace{1\baselineskip}

\Satz{

    $V^{*} \otimes W^{*} \cong (V \otimes W)^{*}$ und $(V \otimes W)^{*} \cong \Bil (V,W;K)$
}

\vspace{1\baselineskip}

\Korollar{

    Sei $u \in K^m \otimes K^n \cong J^{m \cdot n}$, also
    $u = \sum_{i=1}^{m} \sum_{j=1}^{n} u_{ij} e_i \otimes e_j$ mit $e_i \in K^m$ und
    $e_j \in K^n$ Basisvektoren. Die Koordinaten $u_{ij}$ kann man auf drei Weisen
    aufschreiben:
    \begin{enumerate}[{(i)}]
        \item Als Matrix $(u_{ij})_{ij} \in \Mat (m \times n ; K)$
        \item Als $m \cdot n$ Vektor wobei man die Basen von $K^m \otimes K^n$ wie folgt
                ordnet: $e_1 \otimes e_1 , e_2 \otimes e_1 ,\dots, e_m \otimes e_1,
                e_1 \otimes e_2 ,\dots, e_m \otimes e_2,\dots,e_1 \otimes e_n ,\dots,
                e_m \otimes e_n$.
        \item Als $m \cdot n$ Vektor wobei man die Basen von $K^m \otimes K^n$ wie folgt
                ordnet: $e_1 \otimes e_1 , e_1 \otimes e_2 ,\dots, e_1 \otimes e_n,
                e_2 \otimes e_1,\dots, e_2 \otimes e_1,\dots,e_m \otimes e_1 ,\dots,
                e_m \otimes e_n$.
    \end{enumerate}
}

\vspace{1\baselineskip}

\Definition{

    Sind $V$ und $W$ $K$-VR, so heisst eine bilineare Abbildung
    \begin{align*}
        \xi: V \times V \rightarrow W
    \end{align*}
    \fat{symmetrisch}, wenn $\xi (v,v') = \xi (v' ,v)$ für alle $v,v' \in V$
    \fat{alternierend}, wenn $\xi (v,v) = 0$ für alle $v \in V$
}

\vspace{1\baselineskip}

\Bemerkung{

    Ist $\xi$ alternierend, so gilt $\xi (v',v) = - \xi (v,v') \ \forall v,v' \in V$.
    Im Fall char$(K) \neq 2$ gilt auch die Umkehrung (also
    alternierend $\Leftrightarrow$ schiefsymmetrisch).
}

\vspace{1\baselineskip}

\Definition{

    $S(V) := \text{span} (v \otimes v' - v' \otimes v)_{v,v' \in V} \subset V \otimes V$
    \ \ und

    $A(V) := \text{span} (v \otimes v)_{v \in V} \subset V \otimes V$
}

\vspace{1\baselineskip}

\Lemma{

    Für jedes bilineare $\xi : V \times V \rightarrow W$ gilt:

    $\xi$ symmetrisch $\Leftrightarrow \ S(V) \subset \ker \xi_{\otimes}$

    $\xi$ alternierend $\Leftrightarrow \ A(V) \subset \ker \xi_{\otimes}$
}

\vspace{1\baselineskip}

\Theorem{}{

    Für jeden $K$-VR $V$ mit $\dim V \geq 2$ gibt es einen $K$-VR $V \wedge V$ zusammen mit
    einer alternierenden Abbildung $\wedge : V \times V \rightarrow V \wedge V$,
    die folgende universelle Eigenschaft haben: zu jedem $K$-VR $W$ zusammen mit einer
    alternierenden Abbildung $\xi : V \times V \rightarrow W$ gibt es genau eine lineare
    Abbildung $\xi_{\wedge}$ derart, dass das Diagramm
    \begin{center}
        \begin{tikzcd}
            V \times V \arrow{d}[swap]{\wedge} \arrow{rd}{\xi} & \\
            V \wedge V \arrow{r}[swap]{\exists ! \ \xi_{\wedge}} & W
        \end{tikzcd}
    \end{center}
    kommutiert. Ist $(\BasisV)$ eine Basis von $V$, so ist durch $v_i \wedge v_j :=
    \wedge(v_i,v_j)$ mit $1 \leq i < j \leq n$ eine Basis von $V \wedge V$ gegeben.
    Insbesondere ist
    \begin{align*}
        \dim (V \wedge V) = \binom{n}{2} = \frac{n (n-1)}{2}
    \end{align*}
}

\vspace{1\baselineskip}

\Korollar{ (Rechenregeln für das äussere Produkt)

    Für $v,v',w,w' \in V$ und $\lambda \in K$ gilt:
    \begin{enumerate}[{a)}]
        \item $(v+v') \wedge w = v \wedge w + v' \wedge w$ und $v \wedge (w+w') = v \wedge w + v \wedge w'$
        \item $(\lambda v) \wedge w = v \wedge (\lambda w) = \lambda (v \wedge w)$
        \item $v \wedge v = 0$ und $v' \wedge v = - v \wedge v'$ 
    \end{enumerate}
}

\vspace{1\baselineskip}

\Theorem{}{

    Zu jedem $K$-VR $V$ gibt es einen $K$-VR $V \vee V$ und eine symmetrische bilineare
    Abbildung $\vee : V \times V \rightarrow V \vee V$ mit folgender universellen
    Eigenschaft: Sei $\xi: V \times V \rightarrow W$ symmetrisch (und bilineare), dann
    gibt es genau eine lineare Abbildung $\xi_{\vee} \rightarrow W$, sodass
    \begin{center}
        \begin{tikzcd}
            V \times V \arrow{d}[swap]{\vee} \arrow{rd}{\xi} & \\
            V \vee V \arrow{r}[swap]{\exists ! \ \xi_{\vee}} & W
        \end{tikzcd}
    \end{center}
    kommutiert. Ist $(\BasisV)$ eine Basis von $V$, so bilden die Vektoren
    $v_i \vee v_j = \vee (v_i , v_j)$ mit $1 \leq i \leq j \leq n$ eine Basis von
    $V \vee V$. Insbesondere ist
    \begin{align*}
        \dim (V \vee V) = \binom{n+1}{2} = \frac{n(n+1)}{2}
    \end{align*}
    $V \vee V$ heisst 2. symmetrische Potenz von $V$ (Symmetrischer Produktraum).
}

\vspace{1\baselineskip}

\Proposition{}{

    Sei nun $\Bil (V;K) = \Bil (V,V;K)$ der Raum der bilinearen Abbildungen
    $V \times V \rightarrow K$ und $\text{Alt}^2 (V,K) \subset \Bil (V;K)$ der UVR der
    alternierenden Bilinearformen und $\text{Sym}^2 (V;K) \subset \Bil (V;K)$ der UVR der
    symmetrischen Bilinearformen. Dann haben wir folgende Abbildungen:
    \begin{center}
        \begin{tikzcd}
            V^{*} \wedge V^{*} \arrow{r}{(1a)} & \text{Alt}^2 (V;K) \arrow{r}{(2a)} \arrow{d} & (V \wedge V)^{*} \arrow{d} \\
            V^{*} \otimes V^{*} \arrow{u} \arrow{d} \arrow{r}{(1)} & \Bil (V;K) \arrow{r}{\stackrel{(2)}{\cong}} & (V \otimes V)^{*} \\
            V^{*} \vee V^{*} \arrow{r}{(1s)} & \text{Sym}^2 (V;K) \arrow{u} \arrow{r}{\stackrel{(2s)}{\cong}} & (V \vee V)^{*} \arrow{u}
        \end{tikzcd}
    \end{center}
    Die Abbildungen $(2a)$, $(2s)$ und $(2)$ erhalten wir aus der universellen Eigenschaften.
    Die Abbildung $(1a)$ ist definiert durch
    \begin{align*}
        V^{*} \wedge V^{*} &\rightarrow \text{Alt}^2 (V;K) \\
        (\phi , \psi) &\mapsto \klammer{(v,w) \mapsto \phi (v) \psi (w) - \phi (w) \psi (v)}
    \end{align*}
    Die Abbildung $(1s)$ ist
    \begin{align*}
        V^{*} \vee V^{*} &\rightarrow \text{Sym}^2 (V;K) \\
        (\phi , \psi) &\mapsto \klammer{(v,w) \mapsto \phi(v) \psi(w) + \phi(w) \psi(v)}
    \end{align*}
}

\Satz{

    Ist $\dim V < \infty$, so ist $(1a)$ ein Isomorphismus, und falls zusätlich
    $\text{char} (K) \neq 2$, so ist $(1s)$ auch ein Isomorphismus.
}

\vspace{1\baselineskip}

\Definition{

    Seien $V,V',W,W'$ $K$-VR und $F: V \rightarrow V'$, $G: W \rightarrow W'$ lineare
    Abbildungen. Dann ist das Tensorprodukt der Abbildungen $F$ und $G$ die eindeutig
    definierte lineare Abbildung
    \begin{align*}
        F \otimes G : V \otimes W \rightarrow V' \otimes W'
    \end{align*}
    so dass $(F \otimes G)(v \otimes w) = F(v) \otimes G(w) \ \forall v \in V, \ w \in W$.
    Etwas ausführlicher sind die linearen Abbildungen $V \otimes W \rightarrow V' \otimes W'$
    nach der universellen Eigenschaft für $V \otimes W$ bijektiv zu den linearen Abbildungen
    $V \times W \rightarrow V' \otimes W'$. Die bilineare Abbildung $(v,w) \mapsto F(v) \otimes
    G(w)$ entspricht dabei der linearen Abbildung $F \otimes G$.
}

\vspace{1\baselineskip}

\Satz{

    Für $V,V',W,W'$ endlich dimensionale $K$-VR ist das Tensorprodukt von Abbildungen
    \begin{align*}
            &\otimes : \Hom_K (V,V') \otimes \Hom_K (W,W') \\
            &\hspace{60pt} \rightarrow \Hom_K (V \otimes W , V' \otimes W')
    \end{align*}
    ein Vektorraumisomorphismus.
}

\vspace{1\baselineskip}

\Bemerkung{

    Wir wollen uns nun anschauen, wie die Matrix von $F \otimes G$ bezüglich einer Basis
    aussieht. Sei $\operatorname{dazu} \mathcal{A}=\left(v_{1}, \ldots, v_{m}\right)$ eine
    Basis von $V, \mathcal{A}^{\prime}=\left(v_{1}^{\prime}, \ldots, v_{m^{\prime}}^{\prime}\right)$
    eine Basis von $V^{\prime}, \mathcal{B}=\left(w_{1}, \ldots, w_{n}\right)$ eine Basis
    von $W$ und $\mathcal{B}^{\prime}=\left(w_{1}^{\prime}, \ldots, w_{n^{\prime}}^{\prime}\right)$
    eine Basis von $W^{\prime} .$ Seien $A=\left(a_{i j}\right)=M_{\mathcal{A}^{\prime}}^{\mathcal{A}}(F)$
    und $B=\left(b_{i j}\right)=M_{\mathcal{B}^{\prime}}^{\mathcal{B}}(G)$ die entsprechenden
    Matrizen von $F$ und $G .$ Eine Basis von $V \otimes W$ ist gegeben durch die Familie
    $\mathcal{A} \times \mathcal{B}:=\left(v_{i} \otimes w_{j}\right)_{i=1, \ldots, n}$. Um
    die Koordinaten von Vektoren bezüglich dieser Basis als Spaltenvektoren schreiben zu können,
    und entsprechend auch Matrizen aufschreiben zu können, müssen wir die Elemente dieser Basis
    noch ordnen. Genauer: Wir hatten die Matrix einer Abbildung definiert bezüglich Basen,
    deren Indexmenge $1,2 \ldots$ war. Wir müssen also die Indexmenge in der Familie
    $\mathcal{A} \times \mathcal{B}$ noch von $\{1, \ldots, m\} \times$ $\{1, \ldots, n\}$
    auf die Indexmenge $\{1, \ldots, m n\}$ abändern. Es gibt hierfür zwei natürliche Möglichkeiten:
    \begin{itemize}
        \item Wir können die Ordnung
                \[
                \mathcal{C}_{1}:=\left(v_{1} \otimes w_{1}, v_{2} \otimes w_{1}, \ldots, v_{m} \otimes w_{1}, v_{1} \otimes w_{2}, \ldots, v_{m} \otimes w_{n}\right)
                \]
                verwenden, und entsprechend die Ordnung
                \[
                \mathcal{C}_{1}^{\prime}:=\left(v_{1}^{\prime} \otimes w_{1}^{\prime}, v_{2}^{\prime} \otimes w_{1}^{\prime}, \ldots, v_{m^{\prime}}^{\prime} \otimes w_{1}^{\prime}, v_{1}^{\prime} \otimes w_{2}^{\prime}, \ldots, v_{m^{\prime}}^{\prime} \otimes w_{n^{\prime}}^{\prime}\right)
                \]
                für die entsprechende Basis von $V^{\prime} \otimes W^{\prime}$
        \item Wir können die Ordnung
                \[
                \mathcal{C}_{2}:=\left(v_{1} \otimes w_{1}, v_{1} \otimes w_{2}, \ldots, v_{1} \otimes w_{n}, v_{2} \otimes w_{1}, \ldots, v_{m} \otimes w_{n}\right)
                \]
                verwenden, und entsprechend die Ordnung
                \[
                \mathcal{C}_{2}^{\prime}:=\left(v_{1}^{\prime} \otimes w_{1}^{\prime}, v_{1}^{\prime} \otimes w_{2}^{\prime}, \ldots, v_{1}^{\prime} \otimes w_{n^{\prime}}^{\prime}, v_{2}^{\prime} \otimes w_{1}^{\prime}, \ldots, v_{m^{\prime}}^{\prime} \otimes w_{n^{\prime}}^{\prime}\right)
                \]
                für die entsprechende Basis von $V^{\prime} \otimes W^{\prime}$
    \end{itemize}

    \vspace{1\baselineskip}

    In jedem Falle ist
    \[
    (F \otimes G)\left(v_{i} \otimes w_{j}\right) = F(v_i) \otimes G(w_j) =\sum_{i^{\prime}=1}^{m^{\prime}} \sum_{j^{\prime}=1}^{n^{\prime}} a_{i^{\prime} i} b_{j^{\prime} j} v_{i^{\prime}}^{\prime} \otimes w_{j^{\prime}}^{\prime}
    \]
    Übersetzt in Matrizenform heisst dies: \\
    \[
    \begin{array}{l}
    M_{C_{1}^{\prime}}^{C_{1}}(F \otimes G)=\left(\begin{array}{cccc}
    A b_{11} & A b_{12} & \cdots & A b_{1 n} \\
    A b_{21} & A b_{22} & \cdots & A b_{2 n} \\
    \vdots & \vdots & \ddots & \vdots \\
    A b_{n^{\prime} 1} & A b_{n^{\prime} 2} & \cdots & A b_{n^{\prime} n}
    \end{array}\right) \\  \in M\left(m^{\prime} n^{\prime} \times m n, K\right) \\ \\
    M_{C_{2}^{\prime}}^{C_{2}}(F \otimes G)=\left(\begin{array}{cccc}
    a_{11} B & a_{12} B & \cdots & a_{1 m} B \\
    a_{21} B & a_{22} B & \cdots & a_{2 m} B \\
    \vdots & \vdots & \ddots & \vdots \\
    a_{m^{\prime} 1} B & a_{m^{\prime 2} 2} B & \cdots & a_{m^{\prime} m} B
    \end{array}\right) \\ \in M\left(m^{\prime} n^{\prime} \times m n, K\right)
    \end{array}
    \]
}

\vspace{1\baselineskip}

\Definition{

    Sei $G$ eine Gruppe und $V$ ein $K$-VR. Dann ist eine Darstellung von $G$ auf $V$ ein
    Gruppenhomomorphismus $\rho : G \rightarrow \GL (V)$ wobei $\GL (V)$ die Gruppe der
    Vektorraumisomorphismen von $V$ ist. Man sagt auch \textit{$G$ wirkt auf $V$}.
}

\vspace{1\baselineskip}

\Satz{

    Sei $G$ eine endliche Gruppe und $\rho$ eine Darstellung von $G$ auf dem $K$-VR $V$.
    Nehme an, dass char$(K)$ die Gruppenordnung $\abs{G}$ nicht teilt, also dass
    $\abs{G} \neq 0$ in $K$ ist. Dann ist die kanonische Abbildung $V^G \rightarrow V_G$
    ein Isomorphismus,, und die inverse Abbildung ist gegeben durch die
    Symmetrisierung
    \begin{align*}
        \Sigma_G : V_G &\rightarrow V^G \\
        v + U &\mapsto \frac{1}{\abs{G}} \sum_{g \in G} \rho (g) (v)
    \end{align*}
}

\vspace{1\baselineskip}

\Bemerkung{

    Wir wenden dies nun an auf das symmetrische und äussere Produkt. Sei dazu immer
    vereinfachend $K$ ein Körper der Charakteristik $0,$ so dass die entsprechende
    Voraussetzung im Satz immer erfült ist, und so dass wir alternierende Bilinearformen
    mit schiefsymmetrischen identifizieren können. Das symmetrische Produkt haben wir als die
    Koinvarianten definiert
        \[
        \vee^{n} V=\left(\otimes^{n} V\right)_{S_{n}} = \nicefrac{\otimes^n V}{S^n (v)}
        \]
    wobei hier die Wirkung der symmetrischen Gruppe $S_{n}$, ohne Vorzeichen,
    zu Grunde gelegt ist. Das äussere Produkt kann man analog definieren als die Koinvarianten
        \[
        \wedge^{n} V=\left(\otimes^{n} V\right)_{S_{n}}
        \]
    wobei nun aber die Darstellung mit Vorzeichen verwendet wird. (Man beachte, dass die
    Darstellung in der Notation unterschlagen wird, und dass hier char $(K)=0$ angenommen ist.
    Je nach Kontext und Anwendung wird das symmetrische bzw. äussere Produkt manchmal auch
    als Invarianten definiert. Dies ist gerechtfertigt durch den obigen Satz, der insbesondere
    besagt, dass (Achtung: in Charakteristik 0) Invarianten und Koinvarianten natürlich
    identifiziert werden können,
        \[
        \left(\otimes^{n} V\right)_{S_{n}} \cong\left(\otimes^{n} V\right)^{S_{n}}
        \]
    Der Raum der Bilinearformen auf dem endlich dimensionalen
    Vektorraum $V$ sind z.B. identifiziert mit
    $(V \otimes V)^{*} \cong V^{*} \otimes V^{*} .$ Die symmetrischen Bilinearformen sind
    gerade die, die invariant sind unter der $S_{2}$ -Wirkung (ohne Vorzeichen)
        \[
        \mathrm{Sym}^{2}(V ; K) \cong\left(V^{*} \otimes V^{*}\right)^{S_{2}} \subset \mathrm{Bil}(V ; K)
        \]
    Wir haben gesehen, dass man dies identifizieren kann mit
    $V^{*} \vee V^{*}=\left(V^{*} \otimes V^{*}\right)_{S_{2}}$ falls char $(K) \neq 2$
    Analoges gilt für das äussere Produkt, wobei man dabei in diesem Fall noch die
    Beschränkung an die Charakteristik fallen lassen konnte.
}


\section{Multilineare Algebra}

\vspace{1\baselineskip}

\Definition{

    Eine Abbildung $\xi: V_1 \times \dots \times V_k \rightarrow W$ heisst \fat{multilinear}
    oder \fat{$k$-fach linear}, wenn für jedes $i \in \geschwungeneklammer{1,\dots,k}$ und fest
    gewählnte $v_j \in V_j$ ($j=1,\dots,i-1,i+1,\dots,k$) die Abbildung
    \begin{align*}
        V_i \rightarrow W
        \quad \text{mit} \quad
        v \mapsto \xi(v_1,\dots,v_{i-1},v_{i+1},\dots,v_k)
    \end{align*}
    $K$-linear ist. Kurz ausgedrückt: $\xi$ heisst multilinear, wenn sie linear ist in jedem
    Argument.
}

\vspace{1\baselineskip}

\Theorem{}{

    Zu $K$-VR $V_1,\dots,V_k$ gibt es einen $K$-VR $V_1 \otimes \dots \otimes V_k$ zusammen
    mit einer universellen multilinearen Abbildung
    \begin{align*}
        \eta: V_1 \times \dots \times V_k &\rightarrow V_1 \otimes \dots \otimes V_k
        \quad \text{mit}  \\
        (v_1,\dots,v_k) &\mapsto v_1 \otimes \dots \otimes v_k
    \end{align*}
    dh. zu jeder multilinearen Abbildung $\xi: V_1 \times \dots \times V_k \rightarrow W$
    gibt es genau eine lineare Abbildung $\xi_{\otimes}$ derart, dass das Diagramm
    \begin{center}
        \begin{tikzcd}
            V_1 \times \dots \times V_k \arrow{d}[swap]{\eta} \arrow{rd}{\xi} & \\
            V_1 \otimes \dots \otimes V_k \arrow{r}[swap]{\xi_{\otimes}} & W
        \end{tikzcd}
    \end{center}
    kommutiert. Sind alle $V_j$ endlichdimensional mit Basen
    $(v_1^{(j)},\dots,v_{r_j}^{(j)})$ für $j=1,\dots,k$, so ist eine Basis von
    $V_1 \otimes \dots \otimes V_k$ gegeben durch die Produkte
    $v_{i_1}^{(1)} \otimes \dots \otimes v_{i_k}^{(k)}$ mit $1 \leq i_j \leq r_j$.
    Insbesondere ist $\dim (V_1 \otimes \dots \otimes V_k) = \dim V_1 \cdot \dots \cdot \dim V_k$.
}

\vspace{1\baselineskip}

\Bemerkung{

    Sei $\B = (\BasisW)$ eine andere Basis von $V$. Sei $A = a_{ij} = T_{\B}^{\A}$ die
    entsprechende Transformationsmatrix, so dass (mit $A^{-1} = a_{ji}$)
    \begin{align*}
        v_i = \sum_{j=1}^{n} a_{ji} w_j
        \quad \text{  und  } \quad 
        w_i = \sum_{j=1}^{n} \tilde{a}_{ji} v_j
    \end{align*}
    Dann gilt für die Dualbasis:
    \begin{align*}
        v_i^{*} = \sum_{j=1}^n \tilde{a}_{ij} w_j^{*}
        \quad \text{  und  } \quad
        w_i^{*} = \sum_{j=1}^n a_{ij} v_j^{*}
    \end{align*}
}

\vspace{1\baselineskip}

\Definition{

    Sind $V$ und $W$ $K$-VR, so heisst eine $k$-fach lineare Abbildung
    $\xi: V^k \rightarrow W$
    \begin{itemize}
        \item \fat{symmetrisch}, wenn für jede Permutation $\sigma \in \fat{S}_k$        
                $\xi (v_1,\dots,v_k) =\xi (v_{\sigma (1)},\dots,v_{\sigma (k)})$ gilt.                
        \item \fat{antisymmetrisch}, wenn $\xi (v_1,\dots,v_k) = 0$, falls $v_i = v_j$ für ein
                Paar $(i,j)$ mit $i \neq j$
    \end{itemize}
}

\vspace{1\baselineskip}

\Bemerkung{

    Ist $\xi$ alternierend und $\sigma \in \fat{S}_k$, so ist
    $\xi(v_{\sigma (1)},\dots,v_{\sigma(k)}) = \sign (\sigma) \cdot \xi(v_1,\dots,v_k)$
}

\vspace{1\baselineskip}

\Definition{

    In $\bigotimes^k V$ betrachten wir die folgenden UVR:
    \begin{itemize}
        \item $S^k (V) := \text{span} (v_1 \otimes \dots \otimes v_k - v_{\sigma (1)} \otimes \dots \otimes v_{\sigma (k)})$
        
                wobei $v_1 \otimes \dots \otimes v_k \in \bigotimes^k V$ und $\sigma \in \fat{S}_k$
        \item $A^k (V) := \text{span} (v_1 \otimes \dots \otimes v_k)$
        
                wobei $v_i = v_j$ für ein $(i,j)$ mit $i \neq j$
    \end{itemize}
}

\vspace{1\baselineskip}

\Lemma{

    Für jedes $k$-fach lineare $\xi: V^k \rightarrow W$ gilt:
    \begin{itemize}
        \item $\xi$ symmetrisch $\Leftrightarrow \ S^k (V) \subset \ker \ \xi_{\otimes}$
        \item $\xi$ alternierend $\Leftrightarrow \ A^k (V) \subset \ker \ \xi_{\otimes}$
    \end{itemize}
}

\vspace{1\baselineskip}

\Theorem{}{

    Zu einem $K$-VR $V$ und einer natürlichen Zahl $k \geq 1$ gibt es einen $K$-VR
    $\bigwedge^k V$ zusammen mit einer universellen alternierenden Abbildung
    $\wedge: V^k \rightarrow \bigwedge^k V$, dh. zu jeder alternierenden Abbildung
    $\xi : V^k \rightarrow W$ gibt es genau eine lineare Abbildung $\xi_{\wedge}$
    derart, dass das Diagramm
    \begin{center}
        \begin{tikzcd}
            V^k \arrow{d}[swap]{\wedge} \arrow{rd}{\xi} & \\
            \bigwedge^k V \arrow{r}[swap]{\xi_{\wedge}} & W 
        \end{tikzcd}
    \end{center}
    kommutiert. Ist $(\BasisV)$ eine Basis von $V$, so ist eine Basis von
    $\bigwedge^k V$ gegeben durch die Produkte
    \begin{align*}
        v_{i_1} \wedge \dots \wedge v_{i_k}
        \quad \text{  mit } 1 \leq i_1 < \dots < i_k \leq n
    \end{align*}
    Insbesondere ist $\dim \bigwedge^k V = \binom{n}{k}$ für $1 \leq k \leq n = \dim V$.
    Für $k>n$ setzt man $\bigwedge^k V = 0$. $\bigwedge^k V$ heisst \fat{$k$-te äussere
    Potenz} oder \fat{äusseres Produkt} der Ordnung $k$ von $V$.
}

\vspace{1\baselineskip}

\Theorem{}{

    Zu jedem $K$-VR $V$ und einer natürlichen Zahl $k \geq 1$ gibt es einen $K$-VR
    $\bigvee^k V$ zusammen mit einer symmetrischen Abbildung
    $\vee: V^k \rightarrow \bigvee^k V$, sodass für jede symmetrische Abbildung
    $\xi: V^k \rightarrow W$ genau ein $\xi_{\vee}$ existiert, sodass
    \begin{center}
        \begin{tikzcd}
            V^k \arrow{d}[swap]{\vee} \arrow{rd}{\xi} & \\
            \bigvee^k V \arrow{r}[swap]{\exists ! \ \xi_{\vee}} & W
        \end{tikzcd}
    \end{center}
    kommutiert. Ist $(\BasisV)$ eine Basis von $V$, so ist eine Basis von $\bigwedge^k V$
    gegeben durch die Produkte
    \begin{align*}
        v_{i_1} \vee \dots \vee v_{i_k}
        \quad \text{   mit } 1 \leq i_i \leq \dots \leq i_k \leq n
    \end{align*}
    Insbesondere ist $\dim \bigvee^k V = \binom{n+k-1}{k}$. $\bigvee^k V$ heisst
    \fat{$k$-te symmetrische Potenz} oder \fat{symmetrische Produkte} der Ordnung $k$
    von $V$.    
}

\section{Die Singulärwertzerlegung}

\vspace{1\baselineskip}

\Definition{

    Sei $\K = \R$ oder $\C$ und $A \in M (m \times n , \K)$ eine reelle oder komplexe Matrix.
    Dann gibt es Matrizen $U \in O(m)$ (für $\K = \R)$ bzw $U \in U(m)$ (für $\K = \C$) und
    $V \in O(n)$ bzw. $V \in U(n)$ und eine "diagonale" Matrix $K \in M(m \times n , \R)$
    der Form
    \begin{align*}
        D = \begin{pmatrix}
            \sigma_1 & 0 & \dots & 0 \\
            0 & \sigma_2 & \ddots & \vdots \\
            \vdots & \ddots & \ddots & \vdots
        \end{pmatrix}
    \end{align*}
    mit Diagonaleinträgen $\sigma_j \in \R$, $\sigma_j \geq 0$ so dass $A = U D V^{\dagger}$.
    Diese Zerlegung heisst \fat{Singulärwertzerlegung} von $A$, und die nichtnegativen reellen
    Zahlen $\sigma_1 , \dots, \sigma_{\min \geschwungeneklammer{m,n}}$ heissen
    \fat{Singulärwerte} von $A$. In der Regel ordnet man die Singulärwerte so, dass
    $\sigma_1 \geq \dots \geq  \sigma_{\min \geschwungeneklammer{m,n}}$.
}

\vspace{1\baselineskip}

\Definition{ (Alternativ)

    Zu $A \in M(m \times n , \K)$ vom Rang $r$ gibt es Matrizen $\tilde{U} \in M(m \times r,\K)$
    und $\tilde{V} \in M(n \times r , \K)$ mit orthonormalen Spalten und Zahlen
    $\sigma_1 \geq \dots \geq \sigma_r > 0$ so dass $A = \tilde{U} \tilde{D} (\tilde{V})^{\dagger}$
    mit
    \begin{align*}
        \tilde{D} = \begin{pmatrix}
            \sigma_1 & 0 & \dots & 0 \\
            0 & \sigma_2 & \ddots & \vdots \\
            \vdots & \ddots & \ddots & 0 \\
            0 & \dots & 0 & \sigma_r
        \end{pmatrix}
    \end{align*}
    Um diese Zerlegung aus obiger Definition zu erhalten, ordnet man zunächst die Singulärwerte
    wie angegeben. Dann definiert man $\tilde{U}$ als die ersten $r$ Spalten von $U$ und
    $\tilde{V}$ als die ersten $r$ Spalten von $V$. Umgekehrt erhält man $U$ aus $\tilde{U}$
    einfach, indem man die durch die Spalten von $\tilde{U}$ gegebene orthonormale Familie zu
    einer Orthonormalbasis des $\K^m$ fortsetzt.
}

\vspace{1\baselineskip}

\Bemerkung{

    \begin{itemize}
        \item Der Rang von $A$ ist gleich dem Rang von $D$.
        \item Der Rang von $A$ ist die Anzahl der nicht verschwindenden Singulärwerte von $A$.
        \item Die ersten $\min \geschwungeneklammer{m,n}$ Spalten von $U$ (bzw $V$) sind linke
                bzw. rechte \fat{Singulärvektoren} von $A$. Genauer nennt man \fat{Einheitsvektoren}
                $u \in \K^m$, $v \in \K^n$ linke bzw. rechte Singulärvektoren zum Singulärwert
                $\sigma \in \R_{\geq 0}$ falls gilt: $A v = \sigma u$ und $A^{\dagger} u = \sigma v$.
        \item Man kann die Singulärwertzerlegung auch nochmals äquivalent "matrixfrei" umformulieren:
            Sei $F \in \Hom (V,W)$ eine lineare Abbildung zwischen den endlichdimensionalen unitären
            bzw. euklidisch VR $V$ und $W$. Dann gibt es Orthonormalbasen $\A$ von $V$ und $\B$ von $W$
            so dass $M_{\B}^{\A} (F) = D \in M(n \times m , \R)$ Diagonalform hat mit nichtnegativen
            reellen Diagonaleinträgen.
    \end{itemize}
}

\vspace{1\baselineskip}

\Lemma{

    Sei $A \in M(m \times n , \K)$ eine beliebige Matrix vom Rang $r$. Dann sind die hermitischen
    (bzw. symmetrischen) Matriizen $A^{\dagger} A$ und $A A^{\dagger}$ positiv semidefinit und
    haben Rang $r$. Genauer gilt, dass $\ker A^{\dagger} A = \ker A$ und dass
    $\Im A A^{\dagger} = \Im A$
}

\vspace{1\baselineskip}

\Bemerkung{

    Die Singulärwerte von $A$ sind durch $A$ eindeutig bestimmt. Sei nämlich
    $A = U D V^{\dagger}$ eine Singulärwertzerlegung von $A$. Dann gilt:
    $AA^{\dagger} = UDD^T U^{\dagger}$. Also sind die Singulärwerte von $A$ gerade die
    Wurzeln der grössten $\min \geschwungeneklammer{m,n}$ Eigenwerte von $AA^{\dagger}$
    (und $A^{\dagger} A$).
}

\vspace{1\baselineskip}

\Korollar{

    Sei $A \in M(m \times n , \K)$ eine Matrix und $\sigma_1 \geq \dots \geq \sigma_p \geq 0$
    die Singulärwerte von $A$ mit $p:= \min \geschwungeneklammer{m,n}$. Dann gilt:
    \begin{align*}
        \sigma_j = \min_{\stackrel{U \subset \K^n}{\dim U = n-j+1}}
            \max_{\stackrel{u \in U}{u \neq 0}} \frac{\Norm{A u}}{\Norm{u}}
        = \min_{\stackrel{U \subset K^m}{\dim U = m-j+1}} \max_{\stackrel{u \in U}{u \neq 0}}
            \frac{\Norm{A^{\dagger} u}}{\Norm{u}}
    \end{align*}
    wobei $A^{\dagger} u \in \K^m$ und $u \in \K^n$. Insbesondere gilt:
    \begin{align*}
        \sigma_1 = \max_{\stackrel{u \in \K^n}{u \neq 0}} \frac{\Norm{A u}}{\Norm{u}}
        = \max_{\stackrel{u \in \K^n}{\Norm{u} = 1}} \Norm{A u}
    \end{align*}
}

\vspace{1\baselineskip}

\Definition{

    Die \fat{Spektralnorm} (oder $2$-Norm) von $A \in M(m \times n , \K)$ ist der maximale
    Singulärwert von $A$:
    \begin{align*}
        \Norm{A}_2 := \sigma_1 (A) = \max_{\stackrel{u \in \K^n}{u \neq 0}} \frac{\Norm{A u}}{\Norm{u}}
        = \max_{\stackrel{u \in \K^n}{\Norm{u} = 1}} \Norm{A u}
    \end{align*}
}

\vspace{1\baselineskip}

\Bemerkung{

    Dies ist einfach die von der üblichen $2$-Norm auf $\K^m$ bzw. $\K^n$ abgeleitete natürliche
    Operatornorm. Allgemeiner kann man für $F \in \Hom (V,W)$ mit $V,W$ (sagen wir) endlichdimensionalen
    normierten (nicht Null-)VR die folgende Norm erklären
    \begin{align*}
        \Norm{F}_{V,W} := \max_{\stackrel{v \in V}{v \neq 0}} \frac{\Norm{F(v)}_W}{\Norm{v}_V}
    \end{align*}
}

\vspace{1\baselineskip}

\Definition{

    Wir suchen das (bzw. ein) $x \in \K^n$ das den Fehler $\Norm{Ax-b}$ minimiert, bzw
    äquivalen den quadratischen Fehler $E(x):= \Norm{Ax-b}^2$ minimiert. Aus der Analysis
    wissen wir, dass ein solches $x$ erfüllt $0 = DE(x)$ (Ableitung von $E$). Das heisst,
    es muss für alle $h \in \K^n$ gelten:
    \begin{align*}
        0 &= DE(x) h = D((Ax-b)^{\dagger} (Ax-b)) h \\
        &= D (x^{\dagger} A^{\dagger} A x - b^{\dagger} A x - x^{\dagger} A^{\dagger} b + b^{\dagger} b) h \\
        &= h^{\dagger} A^{\dagger} A x + x^{\dagger} A^{\dagger} A h - b^{\dagger} A h - h^{\dagger} A^{\dagger} b \\
        &= 2 \Re (h^{\dagger} (A^{\dagger} A x - A^{\dagger} b))
    \end{align*}
    Also muss gelten: $A^{\dagger} A x = A^{\dagger} b$. Falls $A^{\dagger} A$ invertierbar ist,
    so ist $x = (A^{\dagger} A)^{-1} A b$. Sei nun $A = \tilde{U} \tilde{D} \tilde{V}^{\dagger}$
    die Singulärwertzerlegung in der reduzierten Form, insbesondere ist also $\tilde{D} \in
    M(r \times r ; \R)$ invertierbar. Dann folgt: $\tilde{V} \tilde{D}^2 \tilde{V}^{\dagger} x
    = \tilde{V} \tilde{D} \tilde{U}^{\dagger} b$. Diese Gleichung hat die Lösung:
    $x = \tilde{V} \tilde{D}^{-1} \tilde{U}^{\dagger} b$, dann folgt:
    $\tilde{V} \tilde{D}^2 \tilde{V}^{\dagger} \tilde{V} \tilde{D}^{-1} \tilde{U}^{\dagger} b
    = \tilde{V} \tilde{D} \tilde{U}^{\dagger} b$. Man nennt die Matrix
    $\tilde{V} \tilde{D}^{-1} \tilde{U}^{\dagger}$ auch \fat{Pseudoinverse} von $A$.
}

\vspace{1\baselineskip}

\Definition{

    Sei $A = (a_{ij}) \in M(m \times n,\K)$ eine Matrix. Dann definieren wir die
    \fat{Frobenius-Norm} von $A$ als
    \begin{align*}
        \Norm{A}_F := \sqrt{\sum_{i=1}^m \sum_{j=1}^n \abs{a_{ij}}^2} = \sqrt{\tr (A^{\dagger} A)}
    \end{align*}
    wobei hier $\tr$ die Spur ist, das heisst die Summe der Diagonalelemente. Dies ist einfach
    die übliche $2$-Norm auf $\K^{m \cdot n} \cong M(m \times n, \K)$. Dies kommt von folgendem
    Skalarprodukt auf dem Raum $M(m \times n,\K)$.
    \begin{align*}
        \scalprod{A}{B} = \tr (A^T \overline{B}) = \tr (B^{\dagger} A)
    \end{align*}
}

\vspace{1\baselineskip}

\Bemerkung{ (Eigenschaften tr)

    \begin{itemize}
        \item $\tr (B^T) = \tr (B)$
        \item $\tr (B^\dagger) = \tr (\overline{B^T}) = \overline{ \tr (B)}$
        \item $\tr (AB) = \tr (BA)$ für $A \in M(m \times n ; \K)$ und $B \in M(n \times m; \K)$
    \end{itemize}
}

\vspace{1\baselineskip}

\Lemma{

    Sei $A \in M(m \times n,\K)$ eine Matrix und $U \in M(m \times m , \K)$ und
    $V \in M(n \times n, \K)$ unitär bzw. orthogonal. Dann gilt
    \begin{align*}
        \Norm{A}_F = \Norm{UAV}_F = \Norm{UAV^{\dagger}}_F = \Norm{D}_F
        = \sqrt{\sigma_1^2 + \dots + \sigma_p^2}
    \end{align*}
    falls $\sigma_1 , \dots , \sigma_p$ (mit $p= \min \geschwungeneklammer{m,n}$) die
    Singulärwerte von $A$ sind.
}

\vspace{1\baselineskip}

\Theorem{ (Eckart-Young-Mirsky)

    Sei $A = UDV^{\dagger}$ die Singulärwertzerlegung von $A$ und seien die Singulärwerte
    $\sigma_1 \geq \sigma_2 \geq \dots $ wie oben der Grösse nach sortiert. Seien
    $u_1 , u_2 , \dots , u_m \in \K^m$ die Spalten von $U$ und $v_1 , \dots , v_n \in \K^n$
    die Spalten von $V$ und seien $U_k = (u_1 \ \dots \ u_k) \in M(m \times k , \K)$ und
    $V_k ( v_1 \ \dots \ v_k) \in M(n \times k,\K)$ die Matrizen bestehend aus ersten $k$
    Spalten von $U$ bzw. $V$. Sei
    \begin{align*}
        A_k := U_k D_k V_k^{\dagger} = \sum_{j=1}^k \sigma_j u_j v_j^{\dagger}
    \end{align*}
    mit
    \begin{align*}
        D_k = \begin{pmatrix}
            \sigma_1 & & 0 \\
            & \ddots & \\
            0 & & \sigma_k
        \end{pmatrix}
        \in M(k \times k , \R)
    \end{align*}
    Dann gilt für jede Matrix $B \in M(m \times n , \K)$ vom Rang höchstens $k$, dass
    \begin{align*}
        \Norm{A-B}_F \geq \Norm{A-A_k} = \sqrt{\sum_{j=k+1}^r \sigma_j^2}
    \end{align*}
    Also ist $A_k$ eine Lösung des Optimierungsproblemes.
}

\vspace{1\baselineskip}

\Bemerkung{

    Tatsächlich kann man hier auch die Frobeniusnorm durch die Spektralnorm ersetzen,
    man hat also für das gleiche $A_k$ und alle $B$ vom Rang $\leq k$ auch
    \begin{align*}
        \Norm{A-B}_2 \geq \Norm{A-A_k} = \sigma_{k+1}
    \end{align*}
}

\vspace{1\baselineskip}

\Korollar{

    Seien $A,B \in M(m \times n,\K)$ und $i,j = 1,2,\dots$ beliebig. Dann gilt
    \begin{align*}
        \sigma_{i+j-1} (A+B) \leq \sigma_i (A) + \sigma_j (B)
    \end{align*}
    Falls hierbei ein Index grösser ist als $\min \geschwungeneklammer{m,n}$ so definiert
    man den entsprechenden Singulärwert als $0$.
}

\vspace{1\baselineskip}

\Bemerkung{

    $\sigma_1 (A-A_j) = \sigma_{j+1} (A) \ \forall A \in M(m \times n , \K)$
}

\vspace{1\baselineskip}

\Satz{

    Sei $A \in M(n \times n , \R)$ eine quadratische reelle Matrix und $A=UDV^{T}$ die
    Singulärwertzerlegung. Dann minimiert $R_0 = UV^T$ den Abstand $\Norm{A-R}_F$
    unter allen orthogonalen Matrizen $R \in O(n)$.
}


\section{Verallgemeinerung der Jordanschen Normalform}

\vspace{1\baselineskip}

\Definition{

    Setze für ein $A \in M(k \times k,\K)$:
    \begin{align*}
        A^{\otimes s} := \begin{pmatrix}
            A & & 0 \\
            & \ddots & \\
            0 & & A
        \end{pmatrix}
        \in M(ks \times ks ; \K)
    \end{align*}
}

\vspace{1\baselineskip}

\Theorem{}{

    Sei $F \in \End (V)$ und $V$ ein $\K$-VR mit $\dim V = n < \infty$ dessen charakteristisches
    Polynom in Linearfaktoren zerfällt. ($P_F (t) = \pm (t-\lambda_1)^{r_1} \cdot \dots \cdot
    (t-\lambda_k)^{r_k}$) mit $\lambda_1 , \dots , \lambda_k$ den paarweise verschiedenen
    Eigenwerten. Sei $V_j = \ker ((F-\lambda_j \id)^{r_j})$ der zu $\lambda_j$ gehörige
    Verallgemeinerte Eigenraum und $d_{jr} = \dim \ker ((F-\lambda_j \id_V)^r)$ für
    $r=1,2,\dots$ und setze $d_{jr} = 0$ für $r \leq 0$. Dann gilt:
    \begin{enumerate}[{(i)}]
        \item $\dim V_j = r_j$ und $V$ zerfällt in die $F$-invarianten UVR $V_j$: $V = V_1 \otimes \dots \otimes V_k$
        \item Es gibt eine Basis $\B$ von $V$, so dass $M_{\B} (F) = $
        
        
            \begin{align*}
                \begin{pmatrix}
                    J_1 (\lambda_1)^{\oplus s_{11}} \\
                    & J_2 (\lambda_1)^{\oplus s_{12}} \\
                    & & \ddots \\
                    & & & J_{r_1} (\lambda_1)^{\oplus s_{1 r_1}} \\
                \end{pmatrix}
                \\
                \begin{pmatrix}                
                    J_1 (\lambda_2)^{\oplus s_{21}} \\
                    & \ddots \\
                    & & J_{r_k} (\lambda_k)^{\oplus s_{k r_k}}
                \end{pmatrix}
            \end{align*}

            \vspace{1\baselineskip}

            \fat{Achtung! Eine Matrix, hat nicht als ganzes Platz!}

            \vspace{1\baselineskip}

            Dabei gilt $s_{jr} = 2d_{jr} - d_{j (r+1)} - d_{j (r-1)}$
        \item Das Minimalpolynom ist $M_F (t) = (t-\lambda_1)^{l_1} \cdot \dots \cdot (t-\lambda_k)^{l_k}$
            mit $l_j$ dem grössten Intex mit $s_{j l_j} \neq 0$.
    \end{enumerate}
    Man beachte auch, dass (wie wir gezeigt haben) für $r>r_{j}$ gilt, dass
    $d_{j r}=d_{j r_{j}}=\operatorname{dim} V_{j} .$ Insbesondere ist auch $s_{j r}=0$ für
    $r>r_{j} .$ Ausserdem gilt
    \[
    r_{j}=\operatorname{dim} V_{j}=s_{j 1}+2 s_{j 2}+\cdots r_{j} s_{j r_{j}}
    \]
    Wir wollen auch kurz an die Konstruktion der Basis $\mathcal{B}$ erinnern. Wir brauchen
    dazu nur den Fall zu betrachten, dass alle Eigenwerte gleich sind, indem wir separat Basen
    auf jedem $V_{j}$ definieren. Sei also der notationellen Einfachheit halber gleich $k=1$
    oben, und setze $s_{r}:=s_{1 r}$ und $G:=\left(F-\lambda_{1} i d_{V}\right) .$ Dann
    betrachten wir die Untervektorräume
    \[
    \{0\}=U_{0} \subset U_{1} \subset \cdots \subset U_{d}=V
    \]
    $\operatorname{mit} U_{r}=\operatorname{Ker} G^{r}$ und für ein $d \leq n$ so dass
    $U_{d}=V .$ Dann wählen wir $w_{1}^{(d)}, \ldots, w_{s_{d}}^{(d)} \in U_{d}=V$ so, dass
    die
    Bilder in $U_{d} / U_{d-1}$ eine Basis von $U_{d} / U_{d-1}$ bilden. Dann wählen wir
    $w_{1}^{(d-1)}, \ldots, w_{s_{d-1}}^{(d-1)} \in U_{d-1}$ so, dass die Bilder von
    \[
    G w_{1}^{(d)}, \ldots, G w_{s_{d}}^{(d)}, w_{1}^{(d-1)}, \ldots, w_{s_{d-1}}^{(d-1)} \in U_{d-1}
    \]
    eine Basis von $U_{d-1} / U_{d-2}$ bilden, usw. Unsere Basis ist dann
    \begin{align*}
        \B = ( w_1^{(1)} , \dots , w_{s_1}^{(1)} , G w_1^{(2)} , w_1^{(2)} , \dots , 
            G w_{s_2}^{(2)} , w_{s_2}^{(2)} , \dots , \\  G^{d-1} w_{s_d}^{(d)} , \dots ,
            G w_{s_d}^{(d)} , w_{s_d}^{(d)} )
    \end{align*}
    Wir wollen die jeweils "höchsten" Vektoren $w_{i}^{(j)}$ auch zyklische Vektoren nennen,
    um sie von den anderen Basiselementen zu unterscheiden, die durch anwenden von $G$
    erhalten werden.

    Aus der Konstruktion folgt auch die Formel in $(ii)$. Genauer ist $s_{r}$ eindeutig bestimmt
    durch die Rekursionsformel (absteigende Rekursion)
    \[
    s_{r}=\operatorname{dim} U_{r}-\operatorname{dim} U_{r-1}-\left(s_{r+1}+\cdots+s_{d}\right)
    \]
    Diese Rekursion wird gelöst durch $s_{r}=2 \operatorname{dim} U_{r}-\operatorname{dim}
    U_{r-1}-\operatorname{dim} U_{r+1},$ denn mit $d_{r}:=\operatorname{dim} U_{r}=d_{1 r}$
    gilt
    \[
    \begin{aligned}
    s_{d} &=\operatorname{dim} U_{d}-\operatorname{dim} U_{d-1}=2 \operatorname{dim}
    U_{d}-\operatorname{dim} U_{d-1}-\operatorname{dim} U_{d+1} \\
    s_{r} &=d_{r}-d_{r-1}-\left(s_{r+1}+\cdots+s_{d}\right) \\
    &=d_{r}-d_{r-1}-\underbrace{\left(-d_{r}+2 d_{r+1}-d_{r+2}+\cdots-d_{d-1}+2
    d_{d}-d_{d+1}\right)}_{\text {Teleskopsumme }} \\
    &=d_{r}-d_{r-1}-(-d_{r}+d_{r+1}+d_{d}-\underbrace{d_{d+1}}_{=d_{d}}) \\
    &=2 d_{r}-d_{r-1}-d_{r+1}
    \end{aligned}
    \]
}

\Satz{

    Sei $f \in \R [t]$ ein reelles Polynom vom Grad $n \geq 1$. Dann hat $f$ eine Zerlegung
    $f = a(t-\lambda_1) \dots (t-\lambda_k) \cdot g_1 \dots g_m$,
    wobei $a \in \R$ und die $\lambda_j \in \R$ die reellen NST sind und die $g_j \in \R [t]$
    normierte quadratische Polynome ohne reelle NST sind. Insbesondere ist hier $n=k+2m$ und
    falls $n$ ungerade ist, muss $f$ mindestens eine reelle NST haben.
}

\vspace{1\baselineskip}

\Korollar{

    Sei nun $F \in \End (V)$ ein Endomorphismus des endlich dimensionalen $\R$-VR $V$.
    Fasst man die gleichen Faktoren wie im obigen Satz zusammen, hat man eine Faktorisierung
    des charakteristischen Polynoms
    
    $P_F (t) = \pm (t-\lambda_1)^{r_1} \dots (t-\lambda_k)^{r_k}
    \cdot g_1^{q_1} \dots g_m^{q_m}$,
    
    wobei $\lambda_1 , \dots , \lambda_k$ die paarweise
    verschiedenen reellen NST sind und $g_1 , \dots , g_m \in \R [t]$ paarweise
    verschiedene normierte quadratische reelle Polynome ohne reelle NST. Es gilt insbesondere

    $n=\dim V = r_1 + \dots + r_k + 2 \klammer{q_1 + \dots + q_m}$

    Betrachte nun einen quadratischen Faktor $g_j$ mit nicht-reellen NST $\mu_j ,
    \overline{\mu}_j \in \C \backslash \R$. Sei $\mu_j= a_j + i b_j$ mit $a_j , b_j \in \R$.
    Dann ist

    $g_j = t^2 - 2 \Re (\mu_j) t + \abs{\mu_j}^2 = t^2 - 2 a_j t + a_j^2 + b_j^2
    = (t-a_j)^2 + b_j^2 = P_{A_j} (t)$

    das charakteristische Polynom der Matrix
    \begin{align*}
        A_j := \begin{pmatrix}
            a_j & b_j \\
            -b_j & a_j
        \end{pmatrix}
    \end{align*}
    Beachte auch, dass diese Matrix eine Drehstreckung beschreibt, also von er Form $c R$ ist
    mit $c = \sqrt{a_i^2 + b_i^2} \in \R_{>0}$, $R \in SO(2)$. Schliesslich definieren wir noch
    für $A \in M(2 \times 2,\R)$ die Block-Jordanmatrix
    \begin{align*}
        J_r (A):= \begin{pmatrix}
            A & E_2 & & & \\
            & A & E_2 & & \\
            & & \ddots & \ddots & \\
            & & & \ddots & E_2 \\
            & & & & A
        \end{pmatrix}
    \end{align*}
}

\vspace{1\baselineskip}

\Theorem{

    Sei $F \in \End (V)$ für einen $\K$-VR $V$ mit $n = \dim V < \infty$. Sei
    $P_F (t) = \pm (t-\lambda_1)^{r_1} \dots (t-\lambda_k)^{r_k} \cdot g_1^{q_1} \dots g_m^{q_m}$
    das charakteristische Polynom. Sei $V_j = \ker (F-\lambda_j \id_V)^{r_j}$ für
    $j=1,\dots,k$ und $V_j' = \ker (g_j (F))^{q_j}$ für $j=1,\dots,m$. Sei ausserdem
    $d_{jr} = \dim \ker (F-\lambda_j \id_V)^r$ und $d'_{jr} \dim \ker (g_j (F)^r)$ für
    $r=1,2,\dots$ und setze $d_{jr} = d'_{jr} = 0$ für $r \leq 0$. Dann gilt:
    \begin{enumerate}[{(i)}]
        \item Es gilt $\dim V_j = r_j$ und $\dim V'_j = 2 q_j$ und $V$ zerfällt wie folgt in
                $F$-invariante UVR: $V = V_1 \oplus \dots \oplus V_k \oplus V_1' \dots \oplus V_m'$
        \item Es gibt eine reelle Basis $\B$ von $V$, sodass $M_{\mathbb{B}} (F) =$
        
                \hspace{1cm}

                \begin{align*}
                    &\begin{pmatrix}
                        J_1 (\lambda_1)^{\oplus s_{11}} & \\
                        & \ddots & \\
                        & & J_{r_k} (\lambda_k)^{\oplus s_{k r_k}} \\
                        & & & J_1 (A_1)^{\oplus s'_{11}} \\
                        & & & & J_2 (A_1)^{\oplus s'_{12}}
                    \end{pmatrix}
                    \\
                    &\begin{pmatrix}
                        \ddots \\
                        & J_{q_1} (A_1)^{\oplus s'_{1q_1}} \\
                        & & J_1 (A_2)^{\oplus s'_{2q_1}} \\
                        & & & \ddots \\
                        & & & & J_{q_m} (A_m)^{\oplus s'_{mq_m}}
                    \end{pmatrix}
                \end{align*}

                \vspace{1\baselineskip}

                \fat{Eine Matrix, hat nicht als ganzes platz!}

                \vspace{1\baselineskip}

                für $A_j$ wie im obigen Korollar. Dabei gilt: $s_{jr} = 2d_{jr} - d_{j (r+1)}
                d_{j (r-1)}$ für $j=1,\dots,k$ und alle $r$ und
                $2s'_{jr} = 2 d'_{jr} - d'_{j(r+1)} - d'_{j (r-1)}$ für $j=1,\dots,m$ und alle $r$.
        \item Das Minimalpolynom ist $M_F (t) = (t-\lambda_1)^{l_1} \dots (t-\lambda_k)^{l_k} \cdot
                g_1^{l'_1} \dots g_m^{l'_m}$ mit $l_j$ bzw. $l'_j$ den grössten Indizes mit jeweils
                $s_{j l_j} \neq 0$ bzw. $s'_{s l'_j} \neq 0$. 
    \end{enumerate}
}

\vspace{1\baselineskip}

\Korollar{

    Zwei reelle Matrizen sind \fat{ähnlich} (über $\R$) genau wenn ihre charakteristischen
    Polynome übereinstimmen und zusätzlich alle Zahlen $d_{jr},d'_{jr}$ wie im Theorem.
}

\vspace{1\baselineskip}

\Korollar{

    Zwei reelle Matrizen sind ähnlich über $\R$ genau dann, wenn sie ähnlich über $\C$ sind.
}

\vspace{1\baselineskip}

\Korollar{

    Jede reelle Matrix ist ähnlich zu ihrer Transponierten.
}

\vspace{1\baselineskip}

\Definition{

    Sei $f = t^n + a_{n-1} t^{n-1} + \dots + a_1 t + a_0 \in \K[t]$ ein Polynom vom Grad
    $n \geq 0$. Dann ist die \fat{Begleitmatrix} von $f$ die folgende $n \times n$ Matrix:
    \begin{align*}
        B_f := \begin{pmatrix}
            -a_{n-1} & 1 & 0 & \dots & 0 \\
            -a_{n-2} & 0 & 1 & \ddots & \vdots \\
            \vdots & \vdots & \ddots & \ddots & 0 \\
            -a_1 & 0 & \dots & 0 & 1 \\
            -a_0 & 0 & \dots & 0 & 0 
        \end{pmatrix}
    \end{align*}
}

\vspace{1\baselineskip}

\Satz{

    Das charakteristische und das Minimalpolynom von $B_f$ sind gleich $f$, bis auf Vorzeichen
    $(-1)^n P_{B_f} = M_{B_f} = f$
}

\vspace{1\baselineskip}

\Definition{

    \begin{itemize}
        \item Sei $R$ ein kommutativer Ring und $f,g \in R \backslash \geschwungeneklammer{0}$.
                Dann heisst \fat{$f$ teilt $g$}, oder in Symbolen $f \mid g$, falls ein $h \in R$
                existiert mit $fh = g$. Falls $f$ das Element $g$ nicht teilt schreiben wir
                $f \nmid g$.
        \item Ein \fat{Ideal} $I \subset R$ des kommutativen Ringes $R$ ist ein Unterring, für
                den zusätzlich $I R \subset I$ gilt. Äquivalent und expliziter:
                Für $f,g \in I$, $h \in R$ ist $f-g \in I$ und $fh \in I$.
                Für $I_1 , I_2 \subset R$ Ideale sind $I_1 \cap I_2$ und $I_1 + I_2$ auch wieder
                Ideale.
        \item Für jeden Körper $K$ ist der Polynomring $K[t]$ ein Hauptidealring. Konkret gibt es
                zu jedem Ideal $\geschwungeneklammer{0} \neq I \subset K[t]$ ein eindeutiges
                normiertes Polynom kleinsten Grades in $I$ und heisst Minimalpolynom von $I$.
        \item Für jeden Körper $K$ ist $K[t]$ ein faktorieller Ring, dh. er hat keine Nullteiler:
                aus $fg = 0$ folgt $f=0$ oder $g=0$.
    \end{itemize}
    Für Polynome $f,g \in K[t]$ mit ($f,g \neq 0$) definieren wir den \fat{grössten gemeinsamen
    Teiler} ggT$(f,g) \in K[t]$ von $f$ und $g$ als das normierte Polynom vom grössten Grad,
    das sowohl $f$ als auch $g$ teilt. Dies ist auch das Minimalpolynom von $fK[t]+gK[t]$,
    also ggT$(f,g) = M_{fK[t]+gK[t]}$. Insbesondere existieren in diesem Fall also
    $a,b \in K[t]$ so dass ggT$(f,g)=af+bg$.

    \vspace{1\baselineskip}

    Ebenso definieren wir das \fat{kleinste gemeinsame Vielfaches} kgV$(f,g)$ als das
    Polynom vom kleinsten Grad, dass sowohl von $f$ als auch von $g$ geteilt wird.
    Es ist dann kgV$(f,g) = M_{fK[t] \cap gK[t]}$ das Minimalpolynom des Schnittes der von $f$
    und $g$ erzeugten Ideale.
}

\vspace{1\baselineskip}

\Lemma{

    Für ein Polynom $f \in K[t]$ positiven Grades sind äquivalent:
    \begin{itemize}
        \item Aus $f=gh$ folgt, dass ein $0 \neq c \in K$ existiert, so dass $f=cg$ oder
            $f=ch$. (Also hat $f$ keine nichttrivialen Teiler, also nur die Teiler $c$ und $cf$
            für $0 \neq x \in K$. Man sagt auch $f$ ist \fat{irreduzibel}.)
        \item Aus $f \mid gh$ (für $0 \neq g,h \in K[t]$) folgt, dass $f \mid g$  oder $f \mid h$.
            (amsn sagt in diesem Fall auch $f$ ist \fat{prim})
    \end{itemize}
}

\vspace{1\baselineskip}

\Satz{ (Primfaktorzerlegung für Polynome)

    Sei $0 \neq f \in K[t]$ ein Polynom. Dann gibt es irreduzible normierte Polynome
    (positiven Grades) $p_1,\dots,p_n \in K[t]$ (mit $n \geq 0$) und ein $0 \neq c \in K$
    so dass $f=c \cdot p_1 \dots p_n$. Diese Faktorisierung ist eindeutig, bis auf der
    Anordnung  der $p_j$.
}

\vspace{1\baselineskip}

\Definition{

    Ein Ideal $I \subset R$ in einem kommutativen Ring $R$ heisst \fat{maximal}, falls
    $I \neq R$ und es kein Ideal $J$ gibt, so dass $I \subsetneq J \subsetneq R$.
    $I$ heisst \fat{prim}, falls $I \neq R$ und für alle $f,g \in R$ gilt
    $fg \in I \Rightarrow f \in I \ \lor \ g \in I$.

    Im Fall $R=K[t]$ folgt direkt aus der Definition, dass ein Ideal ein Primideal ist,
    genau dann, wenn sein Minimalpolynom prim ist.

    Ebenso ist das Ideal maximal, genau dann, wenn sein Minimalpolynom keine nicht-trivialen
    Teiler hat, also irreduzibel ist.
    
    Also sind für $K[t]$ die Primideale und die maximalen Ideale identisch.
}

\vspace{1\baselineskip}

\Lemma{

    Sei $I \subsetneq R$ ein Ideal im kommutativen Ring mit Eins $R$. Dann gilt:
    \begin{itemize}
        \item I ist maximal genau dann, wenn $R/I$ mit der von $R$ geerbten Addition und
            Multiplikation ein Körper ist.
        \item I ist genau dann prim, wenn $R/I$ nullteilerfrei ist.
    \end{itemize}
}

\vspace{1\baselineskip}

\Definition{ (Jordansche Normalform)

    Wir definieren nun die Matrix
    \begin{align*}
        \tilde{E}_n := \begin{pmatrix}
            0 & 0 & \dots & 0 \\
            \vdots & \ddots & \ddots & \vdots \\
            0 & 0 & \ddots & 0 \\
            1 & 0 & \dots & 0
        \end{pmatrix}
        \in M(n \times n, K)
    \end{align*}
    mit einem Eintrag $1$ unten links und sonst nur Nullen. Ausserdem definieren wir
    für $A \in M (n \times n , K)$
    \begin{align*}
        \tilde{J}_r := \begin{pmatrix}
            A & \tilde{E}_n & & & \\
            & A & \tilde{E}_n & & \\
            & & \ddots & \ddots & \\
            & & & \ddots & \tilde{E}_n \\
            & & & & A
        \end{pmatrix}
    \end{align*}
    Für ein irreduzibles normiertes Polynom $f$ nennen wir $\tilde{J}_r (B_f)$ den zu
    $f$ gehörigen Jordanblock der grösse $nr$.
}

\vspace{1\baselineskip}

\Theorem{ (Jordansche Normalform)

    Sei $V$ ein endlichdimensionaler $K$-VR und $F \in \End (V)$. Sei
    $P_F = \pm p_1^{r_1} \dots p_k^{r_k}$ die Faktorisierung des charakteristischen
    Polynoms in irreduzible normierte Polynome $p_1,\dots,p_k$, die wir als paarweise
    verschieden annehmen können. Dann gibt es eine Basis $\B$ von $V$, so dass
    $M_{\B}$ blockdiagonal ist, wobei die Blöcke auf der Blockdiagonalen alle
    Jordanblöcke von der Form $\tilde{J}_r (B_{p_j})$ sind mit $j=1,\dots,k$ und
    $r=1,\dots,r_j$. Sei dabei $s_{jr}$ die Anzahl der Jordanblöcke von der Grösse
    $\tilde{J}_r (B_{p_j})$. Dann gilt für alle $j$
    \begin{align*}
        \sum_{r=1}^{r_j} s_{jr} = r_j \cdot \deg p_j = \sum_{j=1}^{r_j} s_{jr} \cdot \deg p_j \cdot r
    \end{align*}
    und
    \begin{align*}
        s_{jr} = \frac{1}{\deg p_j} ( 2 \dim \ker (p_j (F)^r) - \dim \ker (p_j (F)^{r+1}) \\ - \dim \ker (p_j (F)^{r-1}))
    \end{align*}
    Insbesondere ist also die Jordansche Normalform wieder eindeutig,, bis auf Permutation der
    Diagonalblöcke.
}

\vspace{1\baselineskip}

\Satz{

    Sei $V$ ein endlichdimensionaler $K$-VR und $F \in \End (V)$. Sei 
    $P_F = \pm p_1^{r_1} \dots p_k^{r_k}$ die Faktorisierung des charakteristischen
    Polynoms von $F$ in irreduzible Polynome. Sei $V_j := \ker \klammer{p_j^{r_j} (F)}$.
    Dann zerfällt $V$ in die direkte Summe der $F$-invarianten UVR $V_j$;
    $V=V_1 \oplus \dots \oplus V_k$. Ferner gilt für das charakteristische Polynom
    der Einschränkung $F \mid V_j$ von $F$ auf den invarianten UVR $V_j$, dass
    $P_{F \mid V_j} = \pm p_j^{r_j}$. Insbesondere ist also $\dim_K (V_j) = r_j \cdot \deg p_j$.
}


\section{Beweise}

\subsection{Tensorprodukt}

Wir wählen Basen $\left(v_{i}\right)_{i \in I}$ von $V$ und
$\left(w_{j}\right)_{j \in J}$ von $W$ und definieren $V \otimes W$ als den
Vektorraum der endlichen Linearkombinationen von formalen Ausdrücken der Form
$v_{i} \otimes w_{j} .$ Präziser ausgedrückt betrachten wir den $K$ -Vektorraum
\[
\operatorname{Abb}(I \times J, K)=\{\tau: I \times J \rightarrow K\}
\]
$V \otimes W:=\{\tau: I \times J \rightarrow K: \tau(i, j) \neq 0 \text { für nur endlich
viele }(i, j) \in I \times J\}$
Dann ist $v_{i} \otimes w_{j} \in V \otimes W$ die Abbildung, deren Wert an der einzigen
Stelle $(i, j)$ gleich 1 und sonst 0 ist. Offenbar ist
$\left(v_{i} \otimes w_{j}\right)_{(i, j) \in I \times J}$ eine Basis von $V \otimes W$,
denn für $\tau \in V \otimes W$ gilt
\[
\tau=\sum_{i, j} \tau(i, j)\left(v_{i} \otimes w_{j}\right)
\]
wobei nur endlich viele Summanden $\neq 0$ sind. Also haben wir ein Erzeugendensystem. Ist
\[
\tau:=\sum_{i, j} \alpha_{i j}\left(v_{i} \otimes w_{j}\right)=0
\]
so gilt $\tau(i, j)=\alpha_{i j}=0,$ weil die Nullabbildung überall den Wert Null hat.
Zur Definition von $\eta$ genügt es
\[
\eta\left(v_{i}, w_{j}\right):=v_{i} \otimes w_{j}
\]
zu setzen. sind beliebige Vektoren
\[
v=\sum_{i}^{\prime} \lambda_{i} v_{i} \in V \quad \text { und } \quad
w=\sum_{j} \mu_{j} w_{j} \in W
\]
gegeben, so ist wegen der Bilinearität von $\eta$
\[
v \otimes w:=\eta(v, w)=\eta\left(\sum_{i}^{\prime} \lambda_{i} v_{i},
\sum_{j}^{\prime} \mu_{j} w_{j}\right)=\sum_{i, j}^{\prime} \lambda_{i} \mu_{j}\left(v_{i}
\otimes w_{j}\right)
\]
Nun zur universellen Eigenschaft: Ist $\xi: V \times W \rightarrow U$ gegeben, so betrachten
wir die Vektoren $u_{i j}:=\xi\left(v_{i}, w_{j}\right) \in U .$ Wegen der Bedingung
$\xi=\xi_{\otimes} \circ \eta$ muss
\[
\xi_{\otimes} \circ \eta (v_i , w_j) = \xi_{\otimes}\left(v_{i} \otimes w_{j}\right)=u_{i j}
\]
sein. Es gibt genau eine lineare Abbildung $\xi_{\otimes}: V \otimes W \rightarrow U$
mit dieser Eigenschaft. Weiter ist 
\[
\xi_{\otimes}\left(\sum_{i, j}^{\prime} \alpha_{i j}\left(v_{i} \otimes w_{j}\right)\right)=
\sum_{i, j}^{\prime} \alpha_{i j} u_{i j}
\]
Also ist $\xi_{\otimes}(v \otimes w)=\xi(v, w)$ für alle $(v, w) \in V \times W,$ und die
universelle Eigenschaft ist bewiesen.

Der Zusatz über die Dimensionen ist klar, denn besteht $I$ aus $m$ und $J$ aus Elementen, so besteht $I \times J$ aus $m \cdot n$ Elementen.



\subsection{Das äussere Produkt}

Wir erklären das äußere Produkt als Quotientenvektorraum des Tensorproduktes:
\[
V \wedge V:=(V \otimes V) / A(V)
\]
Bezeichnet
\[
\varrho: V \otimes V \rightarrow V \wedge V
\]
die kanonische Abbildung, so erklären wir $\wedge:=\varrho \circ \eta .$
Für $v, v^{\prime} \in V$ ist also
\[
v \wedge v^{\prime}:=\wedge\left(v, v^{\prime}\right)=
\varrho\left(\eta\left(v, v^{\prime}\right)\right)=
\varrho\left(v \otimes v^{\prime}\right)
\]
Die Abbildung $\wedge$ ist bilinear und auch alternierend. Zum Beweis der
universellen Eigenschaft betrachten wir das folgende Diagramm:
\begin{center}
    \begin{tikzcd}
        V \times V \arrow{dddd}[swap]{\wedge} \arrow{rdd}[swap]{\eta} \arrow{rrrdd}{\xi} & & & \\
        & & & & \\
        & V \otimes V \arrow{rr}{\xi_{\otimes}} \arrow{ldd}[swap]{\varrho} & & W \\
        & & & & \\
        V \wedge V \arrow{rrruu}[swap]{\xi_{\wedge}} & & &
    \end{tikzcd}
    \hspace{10pt} $(*)$
\end{center}
Zu $\xi$ gibt es ein eindeutiges lineares $\xi_{\otimes}$ und nach der universellen

Eigenschaft des Quotienten ein eindeutiges lineares
$\xi_{\wedge} .$ Aus der Kommutativität des Diagramms $(*)$ folgt
\[
\xi_{\wedge}\left(v \wedge v^{\prime}\right)=\xi\left(v, v^{\prime}\right)
\]
Es bleibt die Behauptung über die Basis von $V \wedge V$ zu zeigen, das ist die
einzige Schwierigkeit. Da $V \otimes V$ von Tensoren der Form
$v_{i} \otimes v_{j}$ erzeugt wird, erzeugen die Produkte $v_{i} \wedge v_{j}$
den Raum $V \wedge V .$ Wegen $v_{i} \wedge v_{i}=0$ und $v_{j} \wedge v_{i}=
-v_{i} \wedge v_{j}$ sind schon die $\binom{n}{2}$ Produkte
\[
v_{i} \wedge v_{j} \quad \text { mit } i<j
\]
ein Erzeugendensystem, und es genügt, die lineare Unabhängigkeit zu beweisen.
Dazu betrachten wir den Vektorraum $W=K^{N}$ mit $N=\binom{n}{2}$ und wir
bezeichnen die kanonische Basis von $K^{N}$ mit
\[
\left(e_{i j}\right)_{1 \leq i<j \leq n}
\]
Eine alternierende Abbildung $\xi: V \times V \rightarrow K^{N}$ konstruieren
wir wie folgt: sind
\[
v=\sum \lambda_{i} v_{i} \quad \text { und } \quad v^{\prime}=\sum \mu_{i} v_{i}
\]
in $V$ gegeben, so betrachten wir die Matrix
\[
A=\left(\begin{array}{lll}
\lambda_{1} & \cdots & \lambda_{n} \\
\mu_{1} & \cdots & \mu_{n}
\end{array}\right)
\]
und bezeichnen mit $a_{i j}:=\lambda_{i} \mu_{j}-\lambda_{j} \mu_{i}$ den 
entsprechenden 2 -Minor von $A$ Dann ist durch
\[
\xi\left(v, v^{\prime}\right):=\sum_{i<j} a_{i j} e_{i j}
\]
eine sehr gute alternierende Abbildung gegeben: wegen der universellen
Eigenschaft muß
\[
\xi_{\wedge}\left(v_{i} \wedge v_{j}\right)=\xi\left(v_{i}, v_{j}\right)=e_{i j}
\]
sein, und aus der linearen Unabhängigkeit der $e_{i j}$ in $K^{N}$ folgt die
Behauptung. Die so erhaltene Abbildung
\[
\xi_{\wedge}: V \wedge V \rightarrow K^{N}
\]
ist ein Isomorphismus. Er liefert eine etwas konkretere Beschreibung des
äußeren Produktes.




\subsection{Tensor mit Homomorphismen}


Seien $U, V, W, Z$ endlich dimensionale $K$ -Vektorräume, $A \in \operatorname{Hom}(U, V), B \in$ $\mathrm{Hom}(W, Z)$ und betrachten Sie die lineare Abbildung $^{1}$
$$
\Psi: \operatorname{Hom}(V, W) \rightarrow \operatorname{Hom}(U, Z), F \mapsto B \circ F \circ A
$$
Zeigen Sie, dass unter der Identifikation Hom $(V, W) \cong V^{*} \otimes W$ aus der Vorlesung $\Psi$ identifiziert werden kann mit dem Homomorphismus
$$
\Phi: V^{*} \otimes W \rightarrow U^{*} \otimes Z, \quad \Phi=A^{*} \otimes B
$$
wobei $A^{*}$ die zu $A$ duale Abbildung ist.
Lösung:
Wir erinnern an die Identifikation
$$
f_{V, W}: V^{*} \otimes W \stackrel{\tilde{\rightarrow}}{\operatorname{Hom}}(V, W), \quad \varphi \otimes w \mapsto \varphi() \cdot w
$$
Wir wollen also argumentieren, dass folgendes Diagramm kommutiert
Seien $u \in U$ und $\varphi \otimes w \in V^{*} \otimes W .$ Da $V^{*} \otimes W$ von Tensoren der Form $\varphi \otimes w$ erzeugt wird, genügt es die Behauptung hierfür zu zeigen. Es gilt
$$
\begin{aligned}
\left(\left(\Psi \circ f_{V, W}\right)(\varphi \otimes w)\right)(u) &=\left(B \circ f_{V, W}(\varphi, w) \circ A\right)(u) \\
&=B((\varphi \circ A)(u) \cdot w) \\
&=\varphi(A(u)) \cdot B(w) \\
&=A^{*}(\varphi)(u) \cdot B(w) \\
&=f_{U, Z}\left(\left(A^{*} \otimes B\right)(\varphi \otimes w)\right)(u) \\
&=\left(\left(f_{U, Z} \circ \Phi\right)(\varphi \otimes w)\right)(u)
\end{aligned}
$$

\subsection{Symmetrisches Produkt}

Sei $U=\operatorname{span}\left\{v \otimes v^{\prime}-v^{\prime} \otimes v \mid v, v^{\prime} \in V\right\} \subset V \otimes V .$ Wir definieren das symmetrische
Produkt $V \vee V$ als den Quotientenvektorraum
$$
V \vee V=(V \otimes V) / U
$$
Die universelle Eigenschaft folgt nun direkt aus der universellen Eigenschaft des Quotienten: Die bilineare Abbildung $\xi$ induziert eine eindeutige Abbildung $\xi^{\prime}: V \otimes$ $V \rightarrow W,$ so dass $\xi$ genau die Komposition $V \times V \rightarrow V \otimes V \rightarrow W$ ist. Symmetrie der Abbildung $\xi$ bedeutet gerade, dass $U \subset \operatorname{Ker}\left(\xi^{\prime}\right) .$
Um die Behauptung bzgl. der Basis zu beweisen, bemerken wir zunächst, dass
$$
U=\operatorname{span}\left\{v_{i} \otimes v_{j}-v_{j} \otimes v_{i} \mid i<j\right\}
$$
$\operatorname{Sei} \tilde{U}=\operatorname{span}\left\{v_{i} \otimes v_{j}\right\}_{i \leqslant j} \subset V \otimes V .$ Es genügt zu zeigen, $\operatorname{dass} U \oplus \tilde{U}=V \otimes V$
Dann bildet $\tilde{U}$ isomorph auf $V \vee V$ ab und die Bilder der $v_{i} \otimes v_{j}$ bilden eine Basis des Quotienten.
Nun ist für $i \leqslant j$ aber $v_{j} \otimes v_{i}=-\left(v_{i} \otimes v_{j}-v_{j} \otimes v_{i}\right)+v_{i} \otimes v_{j} \in U+\tilde{U},$ also ist die Basis
$\left\{v_{i} \otimes v_{j}\right\}_{i, j}$ von $V \otimes V$ komplett in $U+\tilde{U}$ enthalten und folglich $U+U^{\prime}=V \otimes V$ Andererseits ist $\operatorname{dim} U \leqslant\left(\begin{array}{c}n \\ 2\end{array}\right)$ und $\operatorname{dim} \tilde{U} \leqslant\left(\begin{array}{c}n+1 \\ 2\end{array}\right),$ also $\operatorname{dim} U+\tilde{U} \leqslant n^{2}=\operatorname{dim} V \otimes V$
Insgesamt daher $V \otimes V=U \oplus \tilde{U}$.


\subsection{k-faches symmetrisches Produkt}

Es sei $S_{k}$ die symmetrische Gruppe. Wir betrachten den Untervektorraum des $k$ -fachen Tensorprodukts
$$
S^{k}(V)=\operatorname{span}\left\{v_{1} \otimes \cdots \otimes v_{k}-v_{\sigma 1} \otimes \cdots \otimes v_{\sigma k} \mid v_{i} \in V, \sigma \in S_{k}\right\} \subset \bigotimes^{k} V
$$
Wir definieren das symmetrische Produkt als den Quotientenvektorraum
$$
\bigvee^{k} V=\bigotimes^{k} V / S^{k}(V)
$$
und bezeichnen mit $v_{1} \vee \cdots \vee v_{k}$ die Restklasse von $v_{1} \otimes \cdots \otimes v_{k}$ im Quotienten. Hierdurch ist ein $K$ -Vektorraum wohldefiniert. Die Abbildung
$$
\left(v_{1}, \ldots, v_{k}\right) \mapsto v_{1} \vee \cdots \vee v_{k}
$$
definiert die gesuchte symmetrische multilineare Abbildung
$$
\vee: \underbrace{V \times \cdots \times V}_{k-m a l} \rightarrow \bigvee^{k} V
$$
Die universelle Eigenschaft folgt nun ganz parallel zu Serie $9,$ Aufg. 3 unmittelbar aus der Definition einer symmetrischen multilinearen Abbildung, zusammen mit der universellen Eigenschaft des Quotientenvektorraums.

Wir zeigen schließlich die Behauptung bzgl. der Basis. Da $v_{i_{1}} \otimes \cdots \otimes v_{i_{k}}$ eine Basis des Tensorprodukts bilden, erzeugen die Bilder, also $v_{i_{1}} \vee \cdots \vee v_{i_{k}}$ das symmetrische Produkt $\mathrm{V}^{k} V .$ Da aber für alle $\sigma \in S_{k}$
$$
v_{i_{1}} \vee \cdots \vee v_{i_{k}}=v_{\sigma i_{1}} \vee \cdots \vee v_{\sigma i_{k}}
$$
wird $V^{k} V$ erzeugt durch alle $v_{i_{1}} \vee \cdots \vee v_{i_{k}}$ mit $i_{1} \leqslant \cdots \leqslant i_{k} .$ Wir zeigen noch die lineare Unabhängigkeit. Sei hierfür
$$
N=\left(\begin{array}{c}
n+k-1 \\
k
\end{array}\right)
$$
die Anzahl solcher $v_{i_{1}} \vee \cdots \vee v_{i_{k}} .$ Wir definieren eine Surjektion (tatsächlich ein Isomorphismus)
$$
\bigvee^{k} V \rightarrow K^{N}
$$
Sei $e_{i_{1}, \ldots, i_{k}}$ eine Basis von $K^{N} .$ Für $w_{i}=\sum_{j=1}^{n} a_{i j} v_{j} \in V$ definiert die Zuordnung
$$
\left(w_{1}, \ldots, w_{k}\right) \mapsto \sum_{i_{1} \leqslant \cdots \leqslant i_{k}}\left(\sum_{\sigma \in S_{k}} a_{1 \sigma i_{1}} \cdots a_{k \sigma i_{k}} \cdot e_{i_{1}, \ldots, i_{k}}\right)
$$
eine symmetrische multilineare Abbildung
$$
\underbrace{V \times \cdots \times V}_{k-m a l} \rightarrow K^{N}
$$
Aus der universellen Eigenschaft erhalten wir daher eine lineare Abbildung
$$
\bigvee^{k} V \rightarrow K^{N}
$$
$\operatorname{mit} v_{i_{1}} \vee \cdots \vee v_{i_{k}} \mapsto e_{i_{1}, \ldots, i_{k}} .$ Die Abbildung ist insbesondere surjektiv und die
$v_{i_{1}} \vee \cdots \vee v_{i_{k}}$ daher linear unabängig.



\end{document}