\section{Darstellungen von Gruppen}

\subsection{Definitionen}

\begin{definition}[Darstellung]
    Eine Darstellung einer Gruppe $G$ auf einem VR $V \neq 0$ ist ein Homomorphismus
    $\rho: G \rightarrow \GL(V)$. Der VR $V$ heisst dann Darstellungsraum
    der Darstellung $\rho$. Also ordnet eine Darstellung $\rho$ jedem Element
    $g \in G$ eine invertierbare lineare Abbildung $\rho(g): V \rightarrow V$
    zu, so dass $\forall g,h \in G$ die Darstellungseigenschaft
    $\rho(g h) = \rho(g) \rho(h)$ gilt.
\end{definition}

\begin{definition}[reguläre Darstellung]
    Die reguläre Darstellung einer endlichen Gruppe $G$ ist die Darstellung
    auf dem Raum $\C(G)$ aller Funktionen $G \rightarrow \C$,
    \begin{align*}
        (\rho_{reg}(g)f)(h) = f(g^{-1} h)
        \hspace{15pt} , \hspace{15pt}
        f \in \C(G) , g,h \in G
    \end{align*}
    Alternativ: $\C(G)$ hat eine Basis $\geschwungeneklammer{\delta_g}_{g \in G}$ mit
    $\delta_g(g) = 1$ und $\delta_g(h) = 0$ wenn $h \neq g$. Dann ist
    $\rho_{reg}$ die Darstellung, s.d. $\rho_{reg}(g) \delta_h = \delta_{gh}$.
\end{definition}

\paragraph{Notation}
Wir notieren Darstellungen als $(\rho,V)$ oder $\rho$ oder $V$ falls keine
Verwirrung entsteht.

\begin{definition}[Homomorphismus von Darstellungen]
    Ein Homomorphismus von Darstellungen $(\rho_1,V_1) \rightarrow (\rho_2,V_2)$
    ist eine lineare Abbildung $\varphi: V_1 \rightarrow V_2$ s.d.
    $\varphi \rho_1(g) = \rho_2(g)\varphi \ \forall g \in G$.
\end{definition}

\begin{definition}[Äquivalent]
    Zwei Darstellungen $(\rho_1,V_1)$, $(\rho_2,V_2)$ sind äquivalent (oder isomorph)
    falls ein bijektiver Homomorphismus von Darstellungen $\varphi: V_1 \rightarrow
    V_2$ existiert.
\end{definition}

\begin{korollar}
    Der Vektorraum aller Homomorphismen $(\rho_1,V_1) \rightarrow
    (\rho_2,V_2)$ wird mit $\Hom_G(V_1,V_2)$ oder $\Hom_G((\rho_1,V_1),(\rho_2,V_2))$
    bezeichnet.
\end{korollar}

\begin{definition}[invarianter Unterraum]
    Ein invarianter Unterraum einer Darstellung $(\rho,V)$ ist ein UVR
    $W \subset V$ mit $\rho(g)W \subset W \ \forall g \in G$. 
\end{definition}

\begin{definition}[(Ir-)reduzibel]
    Eine Darstellung $(\rho,V)$ heisst irreduzibel, falls sie keine invarianten
    Unterräume ausser $V$ und $\geschwungeneklammer{0}$ besitzt, sonst
    reduzibel.
\end{definition}

\begin{lemma}
    Ist $W \neq \geschwungeneklammer{0}$ ein invarianter
    Unterraum, so ist die Einschränkung $\rho_{|W}: G \rightarrow \GL(W)$,
    $g \mapsto \rho(g)_{|W}$ eine Darstellung: $(\rho_{|W} , W)$ ist eine
    Unterdarstellung von $(\rho,V)$.
\end{lemma}

\begin{definition}[vollständig reduzibel]
    Eine Darstellung $(\rho,V)$ heisst vollständig reduzibel, falls
    invariante UVR $V_1,\dots,V_n$ existieren, s.d.
    $V = V_1 \oplus \dots \oplus V_n$ und die Unterdarstellungen
    $(\rho_{|V_i},V_i)$ irreduzibel sind. Eine solche Zerlegung von $V$
    heisst Zerlegung in irreduzible Darstellungen.
\end{definition}

\begin{bemerkung}
    Nicht jede reduzible Darstellung ist vollständig reduzibel.
\end{bemerkung}

\begin{lemma}
    Sei $(\rho,V)$ eine endlichdimensionale Darstellung s.d. $\forall$
    invarianten UVR $W \subset V$ $\exists$ ein invarianter UVR $W'$ mit
    $V = W \oplus W'$. Dann ist $(\rho,V)$ vollständig reduzibel.
\end{lemma}

\subsection{Unitäre Darstellungen}

\begin{definition}[unitäre Darstellung]
    Eine Darstellung $\rho$ auf einem VR $V$ mit Skalarprodukt heisst
    unitär falls $\rho(g)$ unitär ist $\forall g \in G$.
    Sei $\rho(g)$ unitär, dann: $\rho(g)^\ast = \rho(g)^{-1} \ \forall g \in G$
    bzw. $\rho(g^{-1}) = \rho(g)^\ast$
\end{definition}

\begin{satz}
    Endliche unitäre Darstellungen sind vollständig reduzibel.
\end{satz}

\begin{satz}
    Sei $(\rho,V)$ eine endliche Darstellung einer endlichen Gruppe $G$.
    Dann $\exists$ ein Skalarprodukt $( \ , \ )$ auf $V$, s.d. $(\rho,V)$
    unitär ist.
\end{satz}

\begin{korollar}
    Darstellungen von endlichen Gruppen sind vollständig reduzibel.
\end{korollar}

\subsection{Das Lemma von Schur}

\begin{satz}[Lemma von Schur]
    Seien $(\rho_1,V_1),(\rho_2,V_2)$ irreduzible komplexe endlichdimensionale
    Darstellungen von $G$.
    \begin{enumerate}[(i)]
        \item $\varphi \in \Hom_G(V_1,V_2) \Rightarrow \varphi \equiv 0$ oder
            $\varphi$ ist ein Isomorphismus.
        \item $\varphi \in \Hom_G (V_1,V_1)$. Dann ist $\varphi = \lambda \text{Id}_{V_1}$
            für $\lambda \in \C$.
    \end{enumerate}
\end{satz}

\begin{korollar}
    Jede irreduzible endlichdimensionale komplexe Darstellung einer abelschen
    Gruppe ist eindimensional.
\end{korollar}
