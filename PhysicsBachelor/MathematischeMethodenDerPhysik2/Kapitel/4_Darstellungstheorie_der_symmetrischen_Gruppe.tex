\section{Darstellungstheorie der symmetrischen Gruppe}

\subsection{Partitionen}

\begin{definition}[Partition]
    Sei $n \geq 1$ eine natürliche Zahl. Eine Partition von $n$ ist eine
    Zerlegung von $n$ in eine Summe positiver ganzer Zahlen
    $n = \lambda_1 + \lambda_2 + \dots + \lambda_k$. Reihenfolge der Summanden
    ist nicht wichtig. Jede Partition ist eindeutig bestimmt durch die Anzahlen
    $i_1,i_2,\dots$ der Zahlen $1,2,\dots$ in der Zerlegung, wobei
    \begin{align*}
        n = \sum_{j \geq 1} j i_j
    \end{align*}
    Wir schreiben $\underline{i}$ für $(i_1,i_2,\dots)$
\end{definition}

\begin{definition}[Young-Diagramm]
    Ein graphischer Weg eine Partition $n = \lambda_1 + \dots + \lambda_k$
    darzustellen ist durch ein Young-Diagramm. Zuerst muss man die $\lambda_j$
    so sortieren, dass $\lambda_1 \geq \lambda_2 \geq \dots \geq \lambda_k$.
    Dann ist das zugehörige Young-Diagramm eine Anordnung von $n$ Kästli,
    mit $\lambda_i$ Kästli in der $i$-ten Zeile.
\end{definition}

\begin{definition}
    Seien $\lambda,\lambda'$ Young-Diagramme mit jeweils $n$ Kästli, dann
    sagen wir $\lambda \geq \lambda'$ genau dann wenn $\lambda = \lambda'$,
    oder falls die erste nicht verschwindende Zahl $\lambda_i - \lambda_i'$
    positiv ist. Entsprechend sagen wir $\lambda > \lambda'$, falls
    $\lambda \geq \lambda'$ und $\lambda \neq \lambda'$. Hierbei ist $\lambda_i$
    die Anzahl der Kästli in der $i$-ten Zeile.
\end{definition}

\subsection{Permutationen der Konjugationsklassen}

Wir können ein Element $\sigma \in S_n = \text{Bij}\klammer{\geschwungeneklammer{1,\dots,n}}$
auf verschiedene Arten aufschreiben. Als Wertetabelle oder Zyklenschreibweise:
\begin{align*}
    \sigma = \begin{pmatrix}
        1 & 2 & 3 & 4 & 5 \\ 4 & 1 & 5 & 2 & 3
    \end{pmatrix}
    = (142)(35)
    = (53)(421)
\end{align*}
Die Längen der Zyklen bestimmen eine Partition von $n$. Die einzelnen Zyklen
kann man auch verstehen als die Bahn in $\geschwungeneklammer{1,\dots,n}$
unter der Wirkung der von $\sigma$ erzeugten Untergruppe von $S_n$, wobei
die zyklische Ordnung auf den Zyklen unberücksichtigt bleibt. Wir schreiben
$i_k (\sigma)$ für die Anzahl der Zyklen der Länge $k$ in der Zyklenschreibweise
von $\sigma$, und $\underline{i}(\sigma) = \klammer{i_1 (\sigma),i_2(\sigma),\dots}$.
Offensichtlich gilt $\sum_{k \geq 1} k i_k (\sigma) = n$ und damit bestimmt
$\underline{i}(\sigma)$ eine Partition von $n$.

\begin{lemma}
    Sei $\tau \in S_n$ eine Permutation. In Zyklenschreibweise:
    $\tau = (i_{1,1} \dotsb i_{1,\lambda_1}) (i_{2,1} \dotsb i_{2,\lambda_2}) \dotsb (i_{k,1} \dots i_{k,\lambda_k})$.
    Sei $\sigma \in S_n$ beliebig. Dann gilt in Zyklenschreibweise:
    $\sigma \tau \sigma^{-1} = \klammer{\sigma(i_{1,1}) \dotsb \sigma(i_{1,\lambda_1})}
    \klammer{\sigma(i_{2,1}) \dotsb \sigma(i_{2,\lambda_2})} \dotsb
    \klammer{\sigma(i_{k,1}) \dotsb \sigma(i_{k,\lambda_k})} =: \tau'$
\end{lemma}

\begin{korollar}
    \begin{enumerate}[(1)]
        \item $\forall k =1,2,\dots$ ist die Anzahl $i_k (\tau)$ der Zyklen
            der Länge $k$ in der Zyklendartellung von $\tau \in S_n$ eine
            Klassenfunktion, d.h. $i_k(\sigma \tau \sigma^{-1}) = i_k(\tau) \
            \forall \sigma,\tau \in S_n$.
        \item Zwei Permutationen $\tau,\tau' \in S_n$ sind genau dann in der gleichen
            Konjugationsklasse, wenn $\underline{i}(\tau) = \underline{i}(\tau')$.
        \item Die Konjugationsklassen von $S_n$ sind also in $1-1$-Korrespondenz zu
            den Partitionen von $n$.
    \end{enumerate}
\end{korollar}

\subsection{Die Gruppenalgebra einer endlichen Gruppe}

\begin{definition}[Gruppenalgebra]
    Sei $G$ eine endliche Gruppe. Dann ist die Gruppenalgebra $\C[G]$ der
    VR der formalen Linearkombinationen $\sum_{g \in G} a_g g$ mit
    $a_g \in \C$. Insbesondere ist $G \subset \C[G]$ eine Basis von
    $\C[G]$. Die Gruppenalgebra ist ein Ring mit dem biliniaren, assoziativen
    Produkt $\C[G] \otimes \C[G] \rightarrow \C[G]$:
    \begin{align*}
        \klammer{\sum_{g \in G} a_g g} \klammer{\sum_{g' \in G} a_{g'}' g'}
        = \sum_{g \in G} \klammer{\sum_{\stackrel{h,h' \in G}{hh' = g}} a_h a_{h'}'} g
        = \sum_{g \in G} \sum_{g' \in G} a_g a_{g'}' (g \cdot g')
    \end{align*}
    und dem Einselement dem neutralen Element der Gruppe $1 \in G \subset
    \C[G]$. Der VR $\C[G]$ der komplexwertigen Funktionen auf der Gruppe $G$
    ist der Dualraum von $\C[G]$, also $\C(G) = \C[G]^\ast$. Die Gruppenalgebra
    trägt eine Darstellung der Gruppe $G$ durch Linksmultiplikation,
    $\rho_{GA}(g) p = g p$, wobei $G \subset \C[G]$ verstenden ist. Diese
    Darstellung ist äquivalent zur regulären Darstellung auf $\C(G)$, wobei
    der Isomorphismus $\C[G] \rightarrow \C(G)$ das Basiselement $g \in G$
    abbildet auf $\delta_g$.
\end{definition}

\begin{theorem}
    Sei $\rho: G \rightarrow \GL(V) \subset \End(V)$ eine Darstellung. Dann
    können wir diese linear fortsetzen zu einer linearen Abbildung
    \begin{align*}
        \rho: \C[G] \rightarrow \End(V)
        \hspace{10pt} , \hspace{10pt}
        \rho \klammer{\sum_{g \in G} a_g g} = \sum_{g \in G} a_g \rho(g)
    \end{align*}
    Diese Abbildung erfüllt $\rho(xy) = \rho(x) \rho(y) \ \forall x,y \in \C[G]$,
    ist also auch ein Ringhomomorphismus.
\end{theorem}

\begin{satz}
    Sei $\rho_1,\dots,\rho_k$ eine Liste der inäquivalenten irreduziblen
    komplexen Darstellungen von $G$ mit Darstellungsräumen $V_1,\dots,V_k$.
    Dann ist die direkte Summe $\bigoplus_{j=1}^k \End(V_j)$ wieder ein
    Ring. Wählen wir Basen auf den $V_j$, so können wir $\bigoplus_{j=1}^k \End(V_j)$
    identifizieren mit dem Ring der Blockdiagonalmatrizen
    \begin{align*}
        \begin{pmatrix}
            A_1 & & \\
            & \ddots & \\
            & & A_k
        \end{pmatrix}
    \end{align*}
    mit Diagonalblöcken $A_j$ der Grösse $\dim(V_j) \times \dim(V_j)$
\end{satz}

\begin{satz}
    Die folgende Abbildung von VR ist ein Isomorphismus von Ringen.
    \begin{align*}
        \phi: \C[G] &\rightarrow \bigoplus_{j=1}^{k} \End(V_j)
        \\
        x &\mapsto \klammer{\rho_1(x),\rho_2(x),\dots,\rho_k(x)}
    \end{align*}
\end{satz}

\subsection{Irreduzible Darstellungen}

\begin{definition}[Young-Schema]
    Ein Young-Schema ist ein Young-Diagramm, dessen $n$ Kästli mit den Zahlen
    $1,\dots,n$ gefüllt sind, wobei jede Zahl genau einmal vorkommt. Zu jedem
    Young-Diagramm $\lambda$ definieren wir das Young-Schema $\hat{\lambda}_{norm}$,
    das aus $\lambda$ gewonnen wird durch füllen der Kästli mit den Zahlen
    $1,2,\dots,n$ aufsteigend von links nach rechts und dann von oben nach unten.
\end{definition}

\begin{definition}
    Für $\lambda$ (bzw. $\hat{\lambda}$) ein Young-Diagramm (bzw. Young-Schema)
    sei $\lambda^T$ (bzw. $\hat{\lambda}^T$) das Young-Diagramm (bzw. Young-Schema),
    dass durch Spiegelung von $\lambda$ (bzw. $\hat{\lambda}$) an der zweiten
    Diagonale gewonnnen wird.
\end{definition}

\begin{definition}
    Zu jedem Young-Schema $\hat{\lambda}$ definieren wir nun eine Untergruppe
    $G_{\hat{\lambda}} \subset S_n$, wobei $\sigma \in G_{\hat{\lambda}}$ genau
    dann wenn $\forall j \in \geschwungeneklammer{1,\dots,n}$ die Zahl
    $\sigma(j)$ in der gleichen Zeile in $\hat{\lambda}$ steht wie $j$.
\end{definition}

\begin{definition}
    Zu jedem Young-Schema $\hat{\lambda}$ ordnen wir nun die folgenden
    beiden Elemente der Gruppenalgebra $\C[S_n]$ zu:
    \begin{align*}
        s_{\hat{\lambda}} := \sum_{\sigma \in G_{\hat{\lambda}}} \sigma
        \hspace{10pt} , \hspace{10pt}
        a_{\hat{\lambda}} := \sum_{\sigma \in G_{\hat{\lambda}^T}} \text{sgn}(\sigma) \sigma
    \end{align*}
\end{definition}
Wir erweitern die Definitionen von $G_{\hat{\lambda}},s_{\hat{\lambda}},a_{\hat{\lambda}}$
von Young-Schemata auf Young-Diagramme, indem wir definieren:
\begin{align*}
    G_{\lambda} := G_{\hat{\lambda}_{norm}}
    \hspace{10pt} , \hspace{10pt}
    s_\lambda := s_{\hat{\lambda}_{norm}}
    \hspace{10pt} , \hspace{10pt}
    a_{\lambda} := a_{\hat{\lambda}_{norm}}
\end{align*}

\begin{definition}
    Zu einem Young-Diagramm $\lambda$ mit $n$ Kästli definieren wir den
    invarianten Unterraum
    \begin{align*}
        V_{\lambda} = \C[S_n] s_\lambda a_\lambda
        = \geschwungeneklammer{ x s_\lambda a_\lambda \ | \ x \in \C[S_n]}
        \subset \C[S_n]
    \end{align*}
    Ferner definieren wir die Darstellung $\rho_\lambda$ von $S_n$ auf
    $V_\lambda$ durch Einschränkung der Darstellung auf $\C[S_n]$ durch
    Linksmultiplikation.
\end{definition}

\begin{satz}
    Die Darstellungen $\rho_\lambda$ der vorherigen Definition sind
    irreduzibel, und für $\lambda \neq \lambda'$ sind $\rho_\lambda$
    und $\rho_{\lambda'}$ inäquivalent.
\end{satz}

\begin{definition}
    $c_{\hat{\lambda}} = s_{\hat{\lambda}} a_{\hat{\lambda}}$ ,
    $c_\lambda = s_\lambda a_\lambda$
\end{definition}

\begin{lemma}
    Sei $\hat{\lambda}$ ein Young-Schema mit $n$ Kästli, mit unterliegendem
    Young-Diagramm $\lambda$. Dann gilt:
    \begin{enumerate}[(1)]
        \item Das neutrale Element von $G$ hat Koeffizient $1$ in $c_{\hat{\lambda}}$.
            Insbesondere gilt $c_{\hat{\lambda}} \neq 0$.
        \item $\forall g \in G_{\hat{\lambda}}$ ist $g s_{\hat{\lambda}} = s_{\hat{\lambda}} = s_{\hat{\lambda}}$
        \item $\forall h \in G_{\hat{\lambda}^T}$ ist $h a_{\hat{\lambda}} = a_{\hat{\lambda}} = \text{sgn}(h)a_{\hat{\lambda}}$
        \item Für $\sigma \in S_n$ beliebig gilt
            \begin{align*}
                G_{\sigma \hat{\lambda}} = \geschwungeneklammer{\sigma g \sigma^{-1} \ | \ g \in G_{\hat{\lambda}}}
            \end{align*}
            wobei das Young-Schema $\sigma \hat{\lambda}$ aus $\hat{\lambda}$
            durch Anwendungen von $\sigma$ auf die Einträge gewonnen ist.
            Insbesondere gilt damit auch $\sigma s_{\hat{\lambda}} \sigma^{-1} =
            s_{\sigma \hat{\lambda}}$ und $\sigma a_{\hat{\lambda}} \sigma^{-1}
            = a_{\sigma \hat{\lambda}}$.
    \end{enumerate}
\end{lemma}

\begin{lemma}
    Seien $\hat{\lambda} , \hat{\mu}$ Young-Schemata mit $n$ Kästli, mit
    unterliegenden Young-Diagrammen $\lambda,\mu$.
    \begin{enumerate}[(1)]
        \item Sei $\lambda > \mu$ und $x \in \C[G]$ beliebig. Dann gilt
            $s_{\hat{\lambda}} x a_{\hat{\mu}} = 0$, und damit insbesondere
            $c_{\hat{\lambda}} c_{\hat{\mu}} = 0 = c_{\hat{\lambda}} x c_{\hat{\mu}}$
        \item Sei $\lambda = \mu$. Dann gilt genau eine der beiden Aussagen:
            \begin{enumerate}[(a)]
                \item $\exists i \neq j$ Zahlen, die in $\hat{\lambda}$ in einer
                    Zeile, und in $\hat{\mu}$ in einer Spalte vorkommen.
                \item $\exists h_1 \in G_{\hat{\lambda}}$ und $h_2 \in G_{\hat{\mu}^T}$,
                    so dass $h_1 \hat{\lambda} = h_2 \hat{\mu}$
            \end{enumerate}
        \item $\forall x \in \C[G]$ ist $s_{\hat{\lambda}} x a_{\hat{\lambda}}$
            ein Vielfaches von $s_{\hat{\lambda}} a_{\hat{\lambda}} = c_{\hat{\lambda}}$.
            Insbesondere ist $c_{\hat{\lambda}} x c_{\hat{\lambda}}$ ein Vielfaches
            von $c_{\hat{\lambda}}$
    \end{enumerate}
\end{lemma}

\begin{lemma}
    \begin{enumerate}[(1)]
        \item Sei $A$ eine komplexe $n \times n$-Matrix so dass für alle $n \times n$-Matrizen
            $X$ gilt, dass $A X A$ ein Vielfaches von $A$ ist. Dann gibt es Vektoren
            $u,v \in \C^n$, so dass $A = u v^\dagger$.
        \item Sei $A = \begin{pmatrix}
            A_1 & & \\ & \ddots & \\ & & A_k
        \end{pmatrix}$ eine Blockdiagonalmatrix mit Diagonalblöcken der Grösse
        $d_j \times d_j$ mit $j = 1,\dots,k$. Es gelte für jede Blockdiagonalmatrix
        $X = \begin{pmatrix}
            X_1 & & \\ & \ddots & \\ & & X_k
        \end{pmatrix}$ gleicher Form, dass $A X A$ ein Vielfaches von $A$ ist.
        Dann existiert ein $j \in \geschwungeneklammer{1,\dots,k}$ und $u,v \in \C^{d_j}$,
        so dass $A_i = 0$ für $i \neq j$ und $A_j = u v^\dagger$.
    \end{enumerate}
\end{lemma}

\subsection{Die Charakterformel von Frobenius}

\begin{satz}[Frobeniusformel]
    Sei $\lambda$ ein Young-Diagramm mit $n$ Kästli. Sei $\underline{i} =
    (i_1,i_2,\dots)$ eine Partition von $n$, und $C_{\underline{i}}$ die
    zugehörige Konjugationsklasse von $S_n$. Dann gilt
    \begin{align*}
        \chi_{\rho_{\lambda}} \klammer{C_{\underline{i}}} =
        \klammer{\Delta (x) \prod_k P_k^{i_k} (x)}_{x^{\lambda + \rho}}
    \end{align*}
    mit der Folgenden Notation:
    \begin{itemize}
        \item $x = (x_1,\dots,x_n)$, und für einen Multiindex $\lambda =
            (\lambda_1,\dots,\lambda_n)$, $x^\lambda = x_1^{\lambda_1}
            \dotsb x_n^{\lambda_n}$.
        \item $\Delta(x) = \prod_{1 \leq i < j \leq n} (x_i - x_j)$ ist
            die Vandermonde-Determinante.
        \item $P_k (x) = x_1^k + x_2^k + \dotsb + x_n^k$
        \item Die Notation $\klammer{Q}_{x^a}$ bezeichnet den Koeffizienten
            von $x^a$ im Polynom $Q(x)$.
        \item $\rho = (n-1,n-2,\dots,1,0)$
    \end{itemize}
\end{satz}

\begin{definition}[Haken]
    Der $i,j$-Haken des Young-Diagrammes $\lambda$ als die Menge der Kästli
    die rechts neben, oder unter dem Kästli an der Stelle $i,j$ stehen,
    inklusive des Kästli $i,j$ selbts.
\end{definition}

\begin{definition}[Hakenlänge]
    Die Hakenlänge $h(i,j)$ ist die Anzahl Kästli im $i,j$-Haken.
\end{definition}

\begin{korollar}[Hakenlängenformel]
    Die Dimension der irreduziblen Darstellung $\rho_\lambda$ von $S_n$ ist
    \begin{align*}
        \dim (\rho_\lambda) = \frac{n!}{\prod_{i,j} h(i,j)}
    \end{align*}
    wobei $h(i,j)$ die Länge des $i,j$-Hakens im Young-Diagramm $\lambda$ ist.
    Das Produkt läuft über die Koordinaten $i,j$ von allen Kästli in $\lambda$.
\end{korollar}
