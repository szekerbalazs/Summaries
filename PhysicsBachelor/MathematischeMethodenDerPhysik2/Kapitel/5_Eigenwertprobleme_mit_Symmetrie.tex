\section{Eigenwertprobleme mit Symmetrie}

\subsection{Eigenwerte und Eigenvektoren}

\begin{satz}
    Sei $\rho: G \rightarrow \GL(V)$ eine endlichdimensionale komplexe
    Darstellung einer komplexen endlichen Gruppe $G$, und $A: V \rightarrow V$
    eine diagonalisierbare lineare Selbstabbildung, so dass $\rho(g) A =
    A \rho(g) \ \forall g \in G$. Sei $V = V_1 \oplus \dotsb \oplus V_n$
    eine Zerlegung von $V$ in irreduzible Darstellungen. Dann hat $A$
    höchstens $n$ verschiedene Eigenwerte. Bezeichnet $d_i$ die
    Dimension von $V_i$, so hat $A$ bezüglich einer passenden Basis die
    Diagonalform
    \begin{align*}
        \text{diag} (\underbrace{\lambda_1,\dots,\lambda_1}_{d_1 \text{ mal}}
            ,\dots,\underbrace{\lambda_n,\dots,\lambda_n}_{d_n \text{ mal}})
    \end{align*}
    für gewisse (nicht notwendigerweise verschiedene) komplexe Zahlen
    $\lambda_1,\dots,\lambda_n$.
\end{satz}

\begin{satz}
    Seien $G,V,A$ wie im vorherigen Satz. Seien $\forall \ i \neq j$ die
    Darstellungen $V_i, V_j$ nicht äquivalent. Dann ist, $\forall \ i$,
    $A V_i \subset V_i$ und die Einschränkung von $A$ auf $V_i$ ist
    $A_{| V_i} = \lambda_i 1_{V_i}$ für ein $\lambda_i \in \C$.
    Also ist $A$ bezüglich einer Basis $V$ mti Basisvektoren in $\bigcup_i V_i$
bereits diagonal.
\end{satz}

\begin{bemerkung}
    Im allgemeinen Fall können die Eigenvektoren wie folgt
    bestimmt werden. Sei $V = W_1 \oplus \dotsb \oplus W_k$ die kanonische
    Zerlegung der Darstellung $\rho$. Nach dem Lemma von Schur ist $A W_i
    \subset W_i$. Also können wir $A$ separat in jedem $W_i$ diagonalisieren,
    und wir haben das Problem auf den Fall reduziert, wo $V$ eine direkte
    Summe von zueinander äquivalenten irreduziblen Darstellungen ist.
    Der allgemeine Fall, wo $V$ eine direkte Summe von $n$ zueinander
    äquivalenten Derstellungen $V_\alpha$ wird wie folgt behandelt.
    Die Isomorphismen zwischen den Darstellungen erlauben und für jedes
    $\alpha = 1,\dots,n$ eine Basis $(e_i^\alpha)_{i=1,\dots,d}$ von $V_\alpha$
    zu wählen, so dass die Matrix von $\rho(g)$ bezüglich der Basis
    $e_1^1,\dots,e_d^1,\dots,e_1^n,\dots,e_d^n$ kästchendiagonalform mit
    gleichen diagonalen $d \times d$ Kästchen hat. Nach Schur hat dann die
    Matrix von $A$ bezüglich der umnummerierten Basis $e_1^1,\dots,e_1^n,\dots,
    e_d^1,\dots,e_d^n$ die folgende Form
    \begin{align*}
        \begin{pmatrix}
            a & & \\
            & \ddots & \\
            & & a
        \end{pmatrix}
        \hspace{20pt} , \text{ mit } a = (a_{ij})_{\stackrel{i=1,\dots,n}{j=1,\dots,n}}
        \in \Mat(n \times n)
    \end{align*}
    mit $n \times n$ Kästchen $a=(a_{\alpha \beta})$ gegeben durch
    \begin{align*}
        A e_i^\alpha = \sum_{\beta} a_{\beta \alpha} e_i^\beta
    \end{align*}
\end{bemerkung}


\subsection{Kleine Schwingungen von Molekülen}

Wir betrachten kleine Schwingungen eines Moleküls (bzw eines Systems
von $N$ Teilchen) aus einer Ruhelage. Die $N$ Teilchen haben Massen
$m_i \ (i=1,\dots,N)$ und Koordinaten $y = (\vec{y}_1,\dots,\vec{y}_N) \in
\R^{3 N}$, mit $\vec{y}_i \in \R^3$ der Position der $i$-ten Teilchens.
Die potentielle Energie sei $V(y) = (\vec{y}_1,\dots,\vec{y}_N)$. Die
Bewegungsgleichung ist
\begin{align*}
    m_i \ddot{\vec{y}}_i \stackrel{(\ast)}{=} - \frac{\partial V}{\partial \vec{y}_i} (y(t))
     \ \forall i = 1,\dots,N
\end{align*}
Sei $y^\ast \in \R^{3 N}$, $y^\ast = (\vec{y}_1^\ast,\dots,\vec{y}_N^\ast)$
ein Gleichgewichtspunkt, d.h. $\vec{\nabla} V(y^\ast) = 0$ und betrachte
kleine Auslenkungen $y(t) = y^\ast + x(t)$. Entwickeln in eine Taylorreihe
um $y^\ast$ ergibt aus $(\ast)$:
\begin{align*}
    m_i \ddot{\vec{x}}_i^\alpha &=
    \sum_{j,\beta} \frac{\partial^2 V}{\partial y_i^\alpha \partial y_j^\beta} (y^\ast) x_j^\beta
    + \underbrace{\mathcal{O}(\abs{x}^2)}_{\text{Vernachlässigen für kleine $x$}}
    \\
    \Leftrightarrow \ddot{\vec{x}}(t) &= - A x(t)
\end{align*}
mit $A$ der Matrix mit Matrixelementen
\begin{align*}
    \frac{1}{m_i} \frac{\partial^2 V}{\partial y_i^\alpha \partial y_j^\beta} (y^\ast)
\end{align*}
$A$ ist diagonalisierbar. Zur Lösung der DGL verwenden wir den Ansatz
$x(t) = e^{i \omega t} x_0$ mit $x_0 \in \R^{3 N}$. Dann folgt:
$\omega^2 x_0 = A x_0$. Die positiven Wurzeln der Eigenwerte von $A$
heissen Eigenfrequenzen des Systems. Nun zu den Symmetrien:

\begin{itemize}
    \item Zunächst soll $V$ invariant sein unter orthogonaler Transformaion,
        d.h. $\forall \ R \in O(3): \ V(R \vec{y}_1 , \dots , R \vec{y}_N)
        = V(\vec{y}_1 ,\dots,\vec{y}_N)$ (und Invarianz unter Translation).
    \item Ausserdem soll $V$ invariant sein unter Vertauschung gleichartiger
        Teilchen, d.h. $V(\vec{y}_{\sigma(1)},\dots,\vec{y}_{\sigma(N)}) =
        V(\vec{y}_1,\dots,\vec{y}_N) \ \forall \sigma \in S \subset S_N$ mit
        $S$ einer geeigneten Untergruppe von $S_N$. Ausserdem $m_{\sigma(i)}
        = m_i \ \forall \sigma \in S$.
    \item Vor der Wahl von $y^\ast$ ist die Symmetriegruppe des Systems
        $O(3) \times S$ mit der Darstellung $\rho: O(3) \times S \rightarrow
        \GL(\R^{3N})$ gegeben durch $\rho(R,\sigma)(\vec{y}_1,\dots,\vec{y}_N)
        = \klammer{R y_{\sigma^{-1}(1)},\dots,R \vec{y}_{\sigma^{-1}(N)}}$
    \item Wir betrachten den Unterraum $G = \geschwungeneklammer{g \in 
        O(3) \times S \ | \ \rho(g)(y^\ast) = y^\ast} \subset O(3) \times S$,
        die wieder durch Einschränkung von $\rho$ auf $\R^{3 N}$ wirkt. Es
        folgt, dass $\forall g \in G: \ \rho(g) A = A \rho(g)$, also
        $A \in \Hom_{G} (\R^{3 N},\R^{3 N})$.
\end{itemize}

\subsection{Beispiel: Eigenfrequenzen von $CH_4$}

Seien $\vec{y}_1,\dots,\vec{y}_4$ die Koordinaten der $H$-Atome und
$\vec{y}_C$ die des $C$-Atoms. Sei die Gleichgewichtlage
$\vec{y}^\ast = (\vec{y}_1^\ast , \dots,\vec{y}_4^\ast , \vec{y}_C^\ast)$
so, dass $\vec{y}_C^\ast = 0$ und die $\vec{y}_j^\ast$ die Eckpunkte eines
regulären Tetraeders bilden mit Zentrum $\vec{y}_C^\ast$. In diesem Fall ist
$G \cong T \cong S_4$ die Tetraedergruppe. Wir betrachten die Charaktertafel

\begin{table}[h]
    \centering
    \begin{tabular}{c|c c c c c}
        $24 T$ & $[1]$ & $8 [r_3]$ & $3 [r_2]$ & $6 [s_4]$ & $6 [\tau]$ \\ \hline
        $\chi_1$ & $1$ & $1$ & $1$ & $1$ & $1$ \\
        $\chi_2$ & $2$ & $-1$ & $2$ & $0$ & $0$ \\
        $\chi_3$ & $1$ & $1$ & $1$ & $-1$ & $-1$ \\
        $\chi_4$ & $3$ & $0$ & $-1$ & $1$ & $-1$ \\
        $\chi_5$ & $3$ & $0$ & $-1$ & $-1$ & $1$    
    \end{tabular}
\end{table}

$1$ ist die Identität; $r_3$ ist die Drehung um eine Achse durch eine Ecke
mit Winkel $2 \pi / 3 = 120^\circ$; $r_2$ ist eine Drehung um eine Achse, die
senkrecht durch eine Kante geht, mit winkel $2 \pi / 2 = \pi = 180^\circ$;
$s_4$ ist die Zusammensetzung einer $120^\circ$ Drehung $r_3$ um eine Achse
durch eine Ecke, sagen wir $\vec{v}_4$, und den Mittelpunkt mit einer Spiegelung
um eine Ebene die durch zwei andere Ecken $\vec{v}_1,\vec{v}_2$ und den
Mittelpunkt geht; schliesslich ist $\tau$ die Spiegelung bezüglich einer
durch eine Kante und den Mittelpunkt gehenden Ebene. Die entsprechende
Permutation der Ecken $1,2,3,4$ und des Mittelpunktes $C$ sind
\begin{align*}
    r_3 : \begin{pmatrix}
        1 & 2 & 3 & 4 & C \\ 1 & 3 & 4 & 2 & C
    \end{pmatrix}
    \hspace{10pt} &, \hspace{10pt}
    r_2 : \begin{pmatrix}
        1 & 2 & 3 & 4 & C \\ 2 & 1 & 4 & 3 & C
    \end{pmatrix}
    \\
    s_4 : \begin{pmatrix}
        1 & 2 & 3 & 4 & C \\ 4 & 1 & 2 & 3 & C
    \end{pmatrix}
    \hspace{10pt} &, \hspace{10pt}
    \tau : \begin{pmatrix}
        1 & 2 & 3 & 4 & C \\ 1 & 2 & 4 & 3 & C
    \end{pmatrix}
\end{align*}
Weiter gilt:
\begin{align*}
    \chi_1: \
    \begin{ytableau}
        \ & \ & \ & \
    \end{ytableau}
    \hspace{10pt} &, \hspace{10pt}
    \chi_2: \
    \begin{ytableau}
        \ & \ \\ \ & \
    \end{ytableau}
    \hspace{10pt} , \hspace{10pt}
    \chi_3 : \
    \begin{ytableau}
        \ \\ \ \\ \ \\ \
    \end{ytableau}
    \\
    \chi_4 : \
    \begin{ytableau}
        \ & \ \\ \ \\ \
    \end{ytableau}
    \hspace{10pt} &, \hspace{10pt}
    \chi_5 : \
    \begin{ytableau}
        \ & \ & \ \\ \
    \end{ytableau}
\end{align*}

Für $\rho(\tau)$ gilt:
\begin{align*}
    \rho(\tau) = \begin{pmatrix}
        \tau & 0 & 0 & 0 & 0 \\
        0 & \tau & 0 & 0 & 0 \\
        0 & 0 & 0 & \tau & 0 \\
        0 & 0 & \tau & 0 & 0 \\
        0 & 0 & 0 & 0 & \tau
    \end{pmatrix}
    \ \in \Mat(15 \times 15)
    \hspace{10pt} \text{ mit } \tau \in \Mat(3 \times 3)
\end{align*}
Es gilt $\tr(\rho(g)) = \tr(R) \cdot N_R$ mit $g = (R,\sigma_R) \in G$
und $N_R$ der Anzahl Diagonalblöcke $\neq 0$, d.h. $\geschwungeneklammer{
i \ | \ \sigma_R (i) = i}$. Für $\tau$: $\tr(\rho(\tau,\sigma_\tau)) = 3 \cdot
\tr(\tau)$. Rechnungen:
\begin{enumerate}[]
    \item Eine Drehung $R$ um $\theta$ hat bzgl einer Basis die Form
        \begin{align*}
            R(\theta) =
            \begin{pmatrix}
                \cos(\theta) & - \sin(\theta) & 0 \\
                \sin(\theta) & \cos(\theta) & 0 \\
                0 & 0 & 1
            \end{pmatrix}
            \ \rightarrow \ \tr(R) = 2 \cos(\theta) + 1
        \end{align*}
        \begin{align*}
            \Rightarrow
            \tr(r_3) = 2 \cos \klammer{\frac{2 \pi}{3}} + 1 = 0
            \hspace{10pt} , \hspace{10pt}
            \tr(r_2) = 2 \cos(\pi) + 1 = -1
        \end{align*}
    \item Eine Drehspiegelung um $\theta$:
        \begin{align*}
            \begin{pmatrix}
                \cos(\theta) & - \sin(\theta) & 0 \\
                \sin(\theta) & \cos(\theta) & 0 \\
                0 & 0 & -1
            \end{pmatrix}
            \ \rightarrow \ \tr(R) = 2 \cos(\theta) - 1
        \end{align*}
        \begin{align*}
            \tr(s_4) = 2 \cos \klammer{\frac{\pi}{2}} - 1 = -1
            \hspace{10pt} , \hspace{10pt}
            \tr(\tau) = 2 \cdot 1 - 1 = 1
        \end{align*}
    \item $N_1 = 5$ , $N_{r_3} = 2$ , $N_{r_2} = 1$ , $N_{s_4} = 1$ , $N_\tau = 3$
    \item Es folgt:
        \begin{table}[h]
            \centering
            \begin{tabular}{c|ccccc}
                 & $[1]$ & $[r_3]$ & $[r_2]$ & $[s_4]$ & $[\tau]$ \\ \hline
                $\chi_\rho$ & $15$ & $0$ & $-1$ & $-1$ & $3$ 
            \end{tabular}
        \end{table}
    \item Berechne Vielfachheiten:
        $n_j = \scalprod{\chi_\rho}{\chi_j} \ \Rightarrow \ $
        $n_1 = 1$ , $n_2 = 1$ , $n_3 = 0$ , $n_4 = 1$ , $n_5 = 3$
    \item Somit: $\rho \cong \rho_1 \oplus \rho_2 \oplus \rho_4 \oplus \rho_5 \oplus \rho_5 \oplus \rho_5$.
        D.h. es gibt höchstens $6$ Eigenfrequenzen (d.h. EW von $A$).
\end{enumerate}
Nicht alle EW von $A$ entsprechen Schwingungen. Manche sind $=0$ wegen der
Translationsinvarianz (T) und Drehinvarianz (D).
\begin{enumerate}[]
    \item \underline{(T)}: Entspricht $x = (\vec{a},\dots,\vec{a})$. Dies
        wird durch $\rho(R)$ abgebildet auf $(R\vec{a},\dots,R\vec{a})$.
        Entspricht Darstellung $R \mapsto R$ von (T) mit Charakter $\chi_5$.
    \item \underline{(D)}: Entspricht $x = (\vec{b} \wedge \vec{y}_1^\ast,\dots,
        \vec{b} \wedge \vec{y}_c^\ast)$ mit $b \in \R^3$. Es gilt:
        $\rho(R) x = \klammer{R(\vec{b} \wedge y_{\sigma_R^{-1} (1)}),\dots,
        R (\vec{b} \wedge y_{\sigma_R^{-1} (c)})}$ mit $\wedge$ dem Kreuzprodukt.
        $R(x \wedge y) = \det(R) (R x \wedge R y) \ \forall R \in O(3)$. Somit:
        $\rho(R) x = \det(R) \klammer{R b \wedge \vec{y}_1^\ast ,\dots, R b \wedge
        \vec{y}_c^\ast}$ entspricht der Darstellung $R \mapsto \det(R) \cdot R$
        auf $\R^3$ entspricht $\chi_4$.
\end{enumerate}
Auf dem orthogonalen Komplement von $\rho_4$ und $\rho_5$ zerlegt sich unsere
Darstellung als $\rho \cong \rho_1 \oplus \rho_2 \oplus \rho_5 \oplus \rho_5$.
Somit erhält man höchstens $4$ verschiedene Eigenfrequenzen.

