\documentclass[a4paper,leqno,twocolumn]{article}
% ==== Inputs and Usepackages ====

% ==== General Commands ====

\newcommand{\sframebox}[1]{
        \framebox[500pt][l]{\parbox{490pt}{
            #1
        }
    }

}

\newcommand{\cframebox}[2]{
        \fbox{\parbox{#1pt}{
            #2
        }
    }
}









% ====== Maths ======


% ==== Formats ====
\newcommand{\boldline}[1]{\textbf{\underline{#1}}}
\newcommand{\uproman}[1]{\uppercase\expandafter{\romannumeral#1}}
\newcommand{\lowroman}[1]{\romannumeral#1\relax}
\newcommand{\fat}[1]{\textbf{#1}}
\newcommand{\Loesung}{\begin{center}\textbf{Lösung}\end{center}}

\newcommand{\Korollar}[1]{\textbf{Korollar} \vspace{1\baselineskip} #1}
\newcommand{\Beispiel}[1]{\textbf{Beispiel} \vspace{1\baselineskip} #1}
\newcommand{\Beweis}[1]{\textbf{Beweis} \vspace{1\baselineskip} #1}
\newcommand{\Proposition}[1]{\textbf{Proposition} \vspace{1\baselineskip} #1}
\newcommand{\Satz}[1]{\textbf{Satz} \vspace{1\baselineskip} #1}
\newcommand{\Definition}[1]{\textbf{Definition} \vspace{1\baselineskip} #1}
\newcommand{\Lemma}[1]{\textbf{Lemma}\vspace{1\baselineskip} #1}
\newcommand{\Bemerkung}[1]{\textbf{Bemerkung} \vspace{1\baselineskip} #1}


\newcommand{\Titel}[1]{
    \begin{center}
        \fat{\LARGE #1} \normalsize
    \end{center}
}

\newcommand{\Aufgabe}[1]{
    \begin{center}
        \fat{\Large Aufgabe #1} \normalsize
    \end{center}
}

\newcommand{\Task}[1]{
    \begin{center}
        \fat{\Large Task #1} \normalsize
    \end{center}
}



% ==== mathsymbols ====
\newcommand{\Q}{\mathbb{Q}}
\newcommand{\R}{\mathbb{R}}
\newcommand{\N}{\mathbb{N}}
\newcommand{\Z}{\mathbb{Z}}
\newcommand{\C}{\mathbb{C}}
\newcommand{\K}{\mathbb{K}}
\newcommand{\eS}{\mathbb{S}}
\newcommand{\X}{$X$ }
\newcommand{\Y}{$Y$ }
\newcommand{\x}{$x$ }
\newcommand{\y}{$y$ }



% ==== math operators ====
\newcommand{\klammer}[1]{\left( #1 \right)} 
\newcommand{\eckigeklammer}[1]{\left[ #1 \right]}
\newcommand{\geschwungeneklammer}[1]{\left\{ #1 \right\}}
\newcommand{\floor}[1]{\lfloor #1 \rfloor}
\newcommand{\scalprod}[2]{\left\langle #1 , #2 \right\rangle}
\newcommand{\abs}[1]{\left\vert #1 \right\vert} 
\newcommand{\Norm}[1]{\left\vert\left\vert #1 \right\vert\right\vert}
\newcommand{\intab}{\int_a^b}
\newcommand{\intii}{\int_{-\infty}^\infty} 
\newcommand{\cint}[2]{\int_{#1}^{#2}}
\newcommand{\csum}[2]{\sum_{#1}^{#2}}
\newcommand{\limes}[1]{\lim\limits_{#1}}


% ==== Analysis ====
\newcommand{\supremum}{\text{sup}}
\newcommand{\infimum}{\text{inf}}
\newcommand{\maximum}{\text{max}}
\newcommand{\minimum}{\text{min}}
\newcommand{\oBdA}{\text{o.B.d.A}}

\newcommand{\xinX}{$x \in X$ }
\newcommand{\yinY}{$y \in Y$ }
\newcommand{\xyinX}{$x,y \in X$ }
\newcommand{\yinX}{$y \in X$ }
\newcommand{\xinR}{$x \in \R$ }
\newcommand{\xyinR}{$x,y \in \R$ }
\newcommand{\zinC}{$z \in \C$ }
\newcommand{\ninN}{$n \in \N$ }
\newcommand{\angeordneterK}{$(K,\leq)$ }

\newcommand{\XsubR}{$X \subset \R$ }
\newcommand{\XsubeqR}{$X \subseteq \R$ }

\newcommand{\XFam}{\mathcal{X}}
\newcommand{\PFam}{\mathcal{P}}

\newcommand{\offenesintervall}[2]{$\left( #1 , #2 \right)$ }
\newcommand{\abgeschlossenesintervall}[2]{$\left( #1 , #2 \right)$ }


% ==== Lineare Algebra ====
\newcommand{\vinV}{$v \in V$ }
\newcommand{\uinU}{$u \in U$ }
\newcommand{\winW}{$w \in W$ }
\newcommand{\vwinV}{$v,w \in V$ }

\newcommand{\BasisV}{$v_1 , \dots , v_n$ }
\newcommand{\BasisU}{$u_1 , \dots , u_n$ }
\newcommand{\BasisW}{$w_1 , \dots , w_n$ }

\newcommand{\transpose}[1]{#1^t}
\newcommand{\inverse}[1]{#1^{-1}}
\newcommand{\ddvec}[3]{\left( #1,#2,#3 \right)}
\newcommand{\tdvec}[2]{\left( #1 , #2 \right)}

\newcommand{\Edrei}{\begin{pmatrix}
    1 & 0 & 0 \\
    0 & 1 & 0 \\
    0 & 0 & 1
\end{pmatrix}}






% ====== Physics ======

% ==== Physicssymbols ====
\newcommand{\epsilonnull}{\varepsilon_0}
\newcommand{\rn}{r_0}
\newcommand{\Rn}{R_0}
\newcommand{\rhonull}{\rho_0}
\newcommand{\Rhonull}{\varrho_0}
\newcommand{\munull}{\mu_0}


% ==== Notation ====
\newcommand{\Etot}{E_{Tot}}
\newcommand{\Wtot}{W_{Tot}}
\newcommand{\Ftot}{F_{Tot}}
\newcommand{\vtot}{v_{Tot}}
\newcommand{\atot}{a_{Tot}}
\newcommand{\mtot}{m_{Tot}}
\newcommand{\Mtot}{M_{Tot}}

\newcommand{\Ekin}{E_{Kin}}
\newcommand{\Epot}{E_{Pot}}
\newcommand{\Edef}{E_{Def}}

\newcommand{\Fg}{F_g}
\newcommand{\FN}{F_N}
\newcommand{\Fz}{F_z}
\newcommand{\FC}{F_C}

\newcommand{\xn}{x_0}
\newcommand{\xN}{x_n}
\newcommand{\vn}{v_0}
\newcommand{\vN}{v_n}
\newcommand{\an}{a_0}
\newcommand{\aN}{a_n}

\newcommand{\dt}{\Delta t}
\newcommand{\dx}{\Delta x}
\newcommand{\dv}{\Delta v}
\newcommand{\da}{\Delta a}
\newcommand{\dE}{\Delta E}
\newcommand{\dW}{\Delta W}
\newcommand{\dF}{\Delta F}


% ==== Relativity ====
\newcommand{\relsqrt}{\sqrt{1-\frac{v^2}{c^2}}}
\newcommand{\relgamma}{\frac{1}{\relsqrt}}


% ==== Constants ====
\newcommand{\g}{9.81}





\usepackage{url}
\usepackage[hidelinks]{hyperref}
%\usepackage{enumitem}
\usepackage{enumerate}
\usepackage{nicefrac}
\usepackage{dsfont}
\usepackage{mathrsfs}
\usepackage{amsmath}
\usepackage{amssymb}
\usepackage{amsthm}
\usepackage{amsfonts}
\usepackage{mathtools}
%\usepackage{mathabx}
\usepackage{MnSymbol}
\usepackage{xfrac}
\usepackage{geometry}
\usepackage{graphicx}
\usepackage{graphics}
\usepackage{latexsym}
\usepackage{setspace}
\usepackage{tikz-cd}
\usepackage{tikz}
 \usetikzlibrary{matrix}
 \usetikzlibrary{calc}

\usepackage{a4wide}
\usepackage{fancybox}
\usepackage{fancyhdr}
\usepackage[ngerman]{datetime}
\usepackage[utf8]{inputenc}





% ==== Page Settings ====

\hoffset = -0.45 in
\voffset = 0 in
\textwidth = 500pt
\textheight = 700pt
\setlength{\headheight}{23pt}
\setlength{\headwidth}{500pt}
\marginparwidth = 0 pt
\topmargin = -0.75 in
\setlength{\parindent}{0cm}





% ==== Presettings for files ====

\newdateformat{myformat}{\THEDAY{ten }\monthname[\THEMONTH], \THEYEAR}


\pagestyle{fancy}

\cfoot{\thepage}
\lfoot{\href{mailto:szekerb@student.ethz.ch}{szekerb@student.ethz.ch}}
\rfoot{Balázs Szekér}

% ==== General Commands ====

\newcommand{\sframebox}[1]{
        \framebox[500pt][l]{\parbox{490pt}{
            #1
        }
    }

}

\newcommand{\cframebox}[2]{
        \fbox{\parbox{#1pt}{
            #2
        }
    }
}









% ====== Maths ======


% ==== Formats ====
\newcommand{\boldline}[1]{\textbf{\underline{#1}}}
\newcommand{\uproman}[1]{\uppercase\expandafter{\romannumeral#1}}
\newcommand{\lowroman}[1]{\romannumeral#1\relax}
\newcommand{\fat}[1]{\textbf{#1}}
\newcommand{\Loesung}{\begin{center}\textbf{Lösung}\end{center}}

\newcommand{\Korollar}[1]{\textbf{Korollar} \vspace{1\baselineskip} #1}
\newcommand{\Beispiel}[1]{\textbf{Beispiel} \vspace{1\baselineskip} #1}
\newcommand{\Beweis}[1]{\textbf{Beweis} \vspace{1\baselineskip} #1}
\newcommand{\Proposition}[1]{\textbf{Proposition} \vspace{1\baselineskip} #1}
\newcommand{\Satz}[1]{\textbf{Satz} \vspace{1\baselineskip} #1}
\newcommand{\Definition}[1]{\textbf{Definition} \vspace{1\baselineskip} #1}
\newcommand{\Lemma}[1]{\textbf{Lemma}\vspace{1\baselineskip} #1}
\newcommand{\Bemerkung}[1]{\textbf{Bemerkung} \vspace{1\baselineskip} #1}


\newcommand{\Titel}[1]{
    \begin{center}
        \fat{\LARGE #1} \normalsize
    \end{center}
}

\newcommand{\Aufgabe}[1]{
    \begin{center}
        \fat{\Large Aufgabe #1} \normalsize
    \end{center}
}

\newcommand{\Task}[1]{
    \begin{center}
        \fat{\Large Task #1} \normalsize
    \end{center}
}



% ==== mathsymbols ====
\newcommand{\Q}{\mathbb{Q}}
\newcommand{\R}{\mathbb{R}}
\newcommand{\N}{\mathbb{N}}
\newcommand{\Z}{\mathbb{Z}}
\newcommand{\C}{\mathbb{C}}
\newcommand{\K}{\mathbb{K}}
\newcommand{\eS}{\mathbb{S}}
\newcommand{\X}{$X$ }
\newcommand{\Y}{$Y$ }
\newcommand{\x}{$x$ }
\newcommand{\y}{$y$ }



% ==== math operators ====
\newcommand{\klammer}[1]{\left( #1 \right)} 
\newcommand{\eckigeklammer}[1]{\left[ #1 \right]}
\newcommand{\geschwungeneklammer}[1]{\left\{ #1 \right\}}
\newcommand{\floor}[1]{\lfloor #1 \rfloor}
\newcommand{\scalprod}[2]{\left\langle #1 , #2 \right\rangle}
\newcommand{\abs}[1]{\left\vert #1 \right\vert} 
\newcommand{\Norm}[1]{\left\vert\left\vert #1 \right\vert\right\vert}
\newcommand{\intab}{\int_a^b}
\newcommand{\intii}{\int_{-\infty}^\infty} 
\newcommand{\cint}[2]{\int_{#1}^{#2}}
\newcommand{\csum}[2]{\sum_{#1}^{#2}}
\newcommand{\limes}[1]{\lim\limits_{#1}}


% ==== Analysis ====
\newcommand{\supremum}{\text{sup}}
\newcommand{\infimum}{\text{inf}}
\newcommand{\maximum}{\text{max}}
\newcommand{\minimum}{\text{min}}
\newcommand{\oBdA}{\text{o.B.d.A}}

\newcommand{\xinX}{$x \in X$ }
\newcommand{\yinY}{$y \in Y$ }
\newcommand{\xyinX}{$x,y \in X$ }
\newcommand{\yinX}{$y \in X$ }
\newcommand{\xinR}{$x \in \R$ }
\newcommand{\xyinR}{$x,y \in \R$ }
\newcommand{\zinC}{$z \in \C$ }
\newcommand{\ninN}{$n \in \N$ }
\newcommand{\angeordneterK}{$(K,\leq)$ }

\newcommand{\XsubR}{$X \subset \R$ }
\newcommand{\XsubeqR}{$X \subseteq \R$ }

\newcommand{\XFam}{\mathcal{X}}
\newcommand{\PFam}{\mathcal{P}}

\newcommand{\offenesintervall}[2]{$\left( #1 , #2 \right)$ }
\newcommand{\abgeschlossenesintervall}[2]{$\left( #1 , #2 \right)$ }


% ==== Lineare Algebra ====
\newcommand{\vinV}{$v \in V$ }
\newcommand{\uinU}{$u \in U$ }
\newcommand{\winW}{$w \in W$ }
\newcommand{\vwinV}{$v,w \in V$ }

\newcommand{\BasisV}{$v_1 , \dots , v_n$ }
\newcommand{\BasisU}{$u_1 , \dots , u_n$ }
\newcommand{\BasisW}{$w_1 , \dots , w_n$ }

\newcommand{\transpose}[1]{#1^t}
\newcommand{\inverse}[1]{#1^{-1}}
\newcommand{\ddvec}[3]{\left( #1,#2,#3 \right)}
\newcommand{\tdvec}[2]{\left( #1 , #2 \right)}

\newcommand{\Edrei}{\begin{pmatrix}
    1 & 0 & 0 \\
    0 & 1 & 0 \\
    0 & 0 & 1
\end{pmatrix}}






% ====== Physics ======

% ==== Physicssymbols ====
\newcommand{\epsilonnull}{\varepsilon_0}
\newcommand{\rn}{r_0}
\newcommand{\Rn}{R_0}
\newcommand{\rhonull}{\rho_0}
\newcommand{\Rhonull}{\varrho_0}
\newcommand{\munull}{\mu_0}


% ==== Notation ====
\newcommand{\Etot}{E_{Tot}}
\newcommand{\Wtot}{W_{Tot}}
\newcommand{\Ftot}{F_{Tot}}
\newcommand{\vtot}{v_{Tot}}
\newcommand{\atot}{a_{Tot}}
\newcommand{\mtot}{m_{Tot}}
\newcommand{\Mtot}{M_{Tot}}

\newcommand{\Ekin}{E_{Kin}}
\newcommand{\Epot}{E_{Pot}}
\newcommand{\Edef}{E_{Def}}

\newcommand{\Fg}{F_g}
\newcommand{\FN}{F_N}
\newcommand{\Fz}{F_z}
\newcommand{\FC}{F_C}

\newcommand{\xn}{x_0}
\newcommand{\xN}{x_n}
\newcommand{\vn}{v_0}
\newcommand{\vN}{v_n}
\newcommand{\an}{a_0}
\newcommand{\aN}{a_n}

\newcommand{\dt}{\Delta t}
\newcommand{\dx}{\Delta x}
\newcommand{\dv}{\Delta v}
\newcommand{\da}{\Delta a}
\newcommand{\dE}{\Delta E}
\newcommand{\dW}{\Delta W}
\newcommand{\dF}{\Delta F}


% ==== Relativity ====
\newcommand{\relsqrt}{\sqrt{1-\frac{v^2}{c^2}}}
\newcommand{\relgamma}{\frac{1}{\relsqrt}}


% ==== Constants ====
\newcommand{\g}{9.81}





\renewcommand{\epsilon}{\varepsilon}
\clubpenalty = 10000
\widowpenalty = 10000

\theoremstyle{definition}
\newtheorem*{theorem}{Theorem}
\newtheorem*{definition}{Definition}
\newtheorem*{lemma}{Lemma}
\newtheorem*{proposition}{Proposition}
\newtheorem*{beispiel}{Beispiel}
\newtheorem*{korollar}{Korollar}
\newtheorem*{satz}{Satz}
\newtheorem*{bemerkung}{Bemerkung}

\usepackage{ytableau}


\begin{document}

\section{Gruppen}

\subsection{Grundbegriffe}

\begin{definition}[Gruppe]
    Eine Gruppe $G$ ist eine Menge mit einem Produkt (Multiplikation)
    $G \times G \rightarrow G$, $(g,h) \mapsto gh$, sodass
    \begin{enumerate}[(i)]
        \item $(g h)k = g (h k) \ \forall g,h,k \in G$ (Assoziativgesetz)
        \item $\exists$ neutrales Element $1 \in G$ mit $1 g = g 1 = g \ \forall g \in G$
        \item $\forall g \in G \ \exists$ ein Inverses $g^{-1} \in G$, sodass
            $g g^{-1} = g^{-1} g = 1$
    \end{enumerate}
    Eine Gruppe heisst \fat{abelsch}, falls $gh = hg \ \forall g,h \in G$
\end{definition}

\begin{definition}[Untergruppe]
    Eine Untergruppe $H$ einer Gruppe $G$ ist eine nichtleere Teilmenge
    von $G$, so dass $h_1 , h_2 \in H \Rightarrow h_1 h_2 \in H$ und
    $h \in H \Rightarrow h^{-1} \in H$. Eine Untergruppe einer Grupppe
    ist selbst eine Gruppe.
\end{definition}

\begin{definition}[Direktes Produkt]
    Das direkte Produkt $G_1 \times G_2$ zweier Gruppen ist das kartesische
    Produkt mit Multiplikation $(g_1,g_2)(h_1,h_2) = (g_1 h_1 , g_2 h_2)$.
    Es ist eine Gruppe mit neutralem Element $(1,1)$ und Inversem
    $(g_1,g_2)^{-1} = (g_1^{-1} , g_2^{-1})$
\end{definition}

\begin{definition}[Diedergruppen $D_n$]
    $D_n$ für $n \geq 3$ besteht aus den orthogonalen Transformationen der
    Ebene die ein reguläres im Ursprung zentriertes $n$-Eck invariant lassen.
    Es enthält eine Drehung $R$ mit Winkel $\frac{2 \pi}{n}$ und eine Spiegelung
    $S$ um eine fixe Achse durch den Ursprung. Falls
    $\geschwungeneklammer{v_i}_{i \in \geschwungeneklammer{0,\dots,n-1}}$
    die Eckpunkte sind, dann gilt: $R v_i = v_{i+1}$ und $S v_i = v_{n-i}$.
\end{definition}

\begin{lemma}
    Es gilt: $\abs{D_n} = 2n$ und die Elemente von $D_n$ sind
    $1,R,R^2,\dots R^{n-1},S,RS,R^2S,\dots R^{n-1} S$.
\end{lemma}

\begin{definition}[Skalarprodukt]
    \begin{align*}
        (x,y) = \sum_{i=1}^n \overline{x}_i y_i
        \hspace{20pt} , \hspace{20pt}
        (x,y)_{p,q} = \sum_{i=1}^p x_i y_i - \sum_{i=p+1}^{p+q} x_i y_j
    \end{align*}
\end{definition}

\begin{definition}[Verschiedene Gruppen]
    \begin{align*}
        GL(n,\R) &= \geschwungeneklammer{\text{invertierbare reelle } n \times n \text{ Matrizen}}
        \\
        GL(n,\C) &= \geschwungeneklammer{\text{invertierbare komplexe } n \times n \text{ Matrizen}}
        \\
        GL(V) &= \geschwungeneklammer{\text{invertierbare lineare Abbildung } V \rightarrow V}
        \\
        O(n) &= \geschwungeneklammer{A \in GL(n,\R) \ \big| \ A^T A = 1} \ \text{(Orthogonale Gruppe)}
        \\
        &= \geschwungeneklammer{A \ \big| \ (Ax,Ay) = (x,y) \ \forall x,y \in \R^n}
        \\
        O(p,q) &= \geschwungeneklammer{A \in GL(p+q,\R) \ \big| \ (A x, A y)_{p,q} = (x,y)_{p,q}}
        \\
        U(n) &= \geschwungeneklammer{A \in GL(n,\C) \ \big| \ A^* A = 1}
        \\
        &= \geschwungeneklammer{A \in GL(n,\C) \ \big| \ (Az,Aw) = (z,w) \ \forall z,w \in \C^n}
        \\
    \end{align*}
\end{definition}

\begin{definition}[Sympeplektische Gruppe]
    Sei $\omega$ die folgende antisymmetrischen Bilinearform auf $\R$
    \begin{align*}
        \omega(X,Y) = \sum_i^n \klammer{X_{2i-1} Y_{2i} - X_{2i} Y_{2i-1}}
    \end{align*}
    wobei $X_i$ die $i$-te Komponente von $X \in \R^{2n}$ bezeichnet. Die
    sympeplektische Gruppe ist dann
    \begin{align*}
        Sp(2n) = \geschwungeneklammer{A \in GL(n,\R) \ \big| \
        \omega(Ax,Ax') = \omega(x,x') \ \forall x,x' \in \R^{2n}}
    \end{align*}
\end{definition}

\begin{definition}[Spezielle Gruppen]
    Sei $G$ eine Untergruppe von $GL(n,\R)$ oder $GL(n,\C)$.
    \begin{align*}
        SG &= \geschwungeneklammer{A \in G \ \big| \ \det A = 1} \subseteq G
        \\
        SL &= \klammer{SGL(n,K)} = \geschwungeneklammer{A \in GL(n,K) \ \big| \ \det A = 1}
        \\
        SO(n) &= \geschwungeneklammer{A \in SL(n,\R) \ \big| \ A^T A = 1} = \geschwungeneklammer{A \in O(n) \ \big| \ \det A = 1}
        \\
        SU(n) &= \geschwungeneklammer{A \in SL(n,\C) \ \big| \ A^* A = 1} = \geschwungeneklammer{A \in U(n) \ \big| \ \det A = 1}
    \end{align*}
\end{definition}

\begin{definition}[Gruppenwirkung / Gruppenoperation]
    Eine Gruppenwirkung/ Gruppenoperation von $G$ auf eine Menge $M$ ist eine
    Abbildung $G \times M \rightarrow M$, $(g,x) \mapsto gx$ sodass
    $g_1 (g_2 x) = (g_1 g_2) x \ \forall g_1 , g_2 \in G$, $x \in M$. Man sagt
    $G$ wirkt/ operiert auf $M$.
\end{definition}

\begin{definition}[Gruppenhomomorphismus]
    Ein (Gruppen-) Homomorphismus $\varphi: G \rightarrow H$ ist eine Abbildung
    zwischen Gruppen $G$ und $H$ sodass $\varphi(g,h) = \varphi(g) \varphi(h)
    \ \forall g,h \in G$. Ist $\varphi$ bijektiv, so heisst $\varphi$
    Isomorphismus und $G$ und $H$ isomorph. Verknüpfungen von Homomorphismen
    sind wieder Homomorphismen.
\end{definition}

\begin{definition}[Kern und Bild]
    \begin{align*}
        \ker(\varphi) &= \geschwungeneklammer{g \in G \ | \ \varphi(g) = 1} \subset G
        \\
        \Im(\varphi) &= \geschwungeneklammer{\varphi(g) \ | \ g \in G} \subset H
    \end{align*}
\end{definition}

\begin{satz}
    Sei $\varphi: G \rightarrow H$ ein Homomorphismus.
    \begin{enumerate}[(i)]
        \item $\varphi(1) = 1$, $\varphi(g)^{-1} = \varphi(g^{-1})$
        \item $\varphi$ ist genau dann injektiv wenn $\ker(\varphi) = \geschwungeneklammer{1}$
    \end{enumerate}
\end{satz}

\begin{definition}[Linksnebenklassen]
    Sei $H$ eine Untergruppe einer Gruppe $G$. Die Menge $G/H$ der
    (Links-)Nebenklassen von $H$ in $G$ ist die Menge der Äquivalenzklassen
    bezüglich der Äquivalenzrelation \
    $g_1 \sim g_2 \Leftrightarrow \exists h \in H$ mit $g_2 = g_1 h$
\end{definition}

\begin{definition}[Normalteiler]
    Ein Normalteiler von $G$ ist eine Untergruppe $H$ mit der Eigenschaft,
    dass $g h g^{-1} \in H \ \forall g \in G, h \in H$.
\end{definition}

\begin{satz}
    Sei $H$ ein Normalteiler von $G$ und es bezeichne $[g]$ die Klasse
    von $g$ in $G/H$. Dann ist für alle $g_1,g_2$ das Produkt
    $[g_1][g_2] = [g_1 g_2]$ wohldefiniert, und $G/H$ ist mit diesem
    Produkt eine Gruppe, welche Faktorgruppe von $G$ mod $H$ heisst.
\end{satz}

\begin{satz}
    Für jeden Homomorphismus $\varphi: G \rightarrow H$ ist $\ker(\varphi)$
    ein Normalteiler von $G$, denn $\varphi(1) = 1 \Rightarrow \varphi(g h g^{-1})
    = \varphi(g) \varphi(h) \varphi(g^{-1}) = \varphi(g) \varphi(g)^{-1} = 1$
\end{satz}

\begin{satz}
    Sei $\varphi: G \rightarrow H$ ein Homomorphismus von Gruppen. Dann gilt
    $G/ \ker(\varphi) \cong \Im(\varphi)$. Der Isomorphismus ist
    $[g] \mapsto \varphi(g)$ für beliebige Wahl der Representanten g.
\end{satz}

\begin{definition}[Automorphismus]
    Sei $H$ eine Gruppe. Dann ist
    $\text{Aut}(H) = \geschwungeneklammer{\varphi: H \rightarrow H \ | \ \text{Gruppenisomorphismus}}$
    die Gruppe der Gruppenisomorphismen von $H$.
\end{definition}

\begin{definition}[Semidirektes Produkt]
    Seien $G$ und $H$ Gruppen und $\rho : G \rightarrow \text{Aut}(H)$,
    $g \mapsto \rho_g$, ein Homomorphismus, wobei $\rho_g = \rho(g) \in
    \text{Aut}(H)$. Dann ist $G \times H$ mit Multiplikation
    $(g_1 , h_1)(g_2 , h_2) = (g_1 g_2 , h_1 \rho_{g_1}(h_2))$ eine Gruppe,
    das semidirekte Produkt $G \ltimes_\rho H$.
\end{definition}

\subsection{Lie-Gruppen}

\begin{definition}[Lie-Gruppen]
    Eine Lie-Gruppe ist eine Gruppe, die gleichzeitig eine $C^\infty$-Mannigfaltigkeit
    ist, so dass Multiplikation und Inversion $C^\infty$-Abbildungen sind.
\end{definition}

\begin{definition}[Stetigkeit für Untergruppen von $GL(n)$]
    Wir fassen $G \subset GL(n,\R)$, $GL(n,\C)$ als Teilmenge von
    $\C^{n^2}$ auf, indem wir die Matrixelemente einer Matrix $A \in G$
    als Punkt $(A_{11} , A_{12} , \dots , A_{nn})$ in $\R^{n^2}$ bzw.
    $\C^{n^2}$ schreiben. Diese Identifikation definiert die Struktur
    eines Metrischen Raumes auf $G$. Der Abstand $d(A,B)$ zwischen zwei
    Matrizen aus $G$ ist
    \begin{align*}
        d(A,B)^2 = \sum_{i,j}^n \abs{A_{ij} - B_{ij}}^2 = \tr(A-B)^\ast (A-B)
    \end{align*}
\end{definition}

\begin{satz}
    Sei $G$ eine Untergruppe von $GL(n,\K)$. Dann sind Multiplikation
    $G \times G \rightarrow G$, $(A,B) \mapsto AB$ und die Inversion
    $G \rightarrow G$, $A \mapsto A^{-1}$ stetige Abbildungen.
\end{satz}

\begin{definition}[Weg]
    Ein Weg in einem metrischen Raum $X$ ist eine stetige Abbildung
    $w: [0,1] \rightarrow X$. Er verbindet $w(0)$ mit $w(1)$. $X$ ist
    wegzusammenhängend, falls $\forall x,y \in X \ \exists$ ein Weg, der
    $x$ mit $y$ verbindet.
\end{definition}

\begin{satz}
    Die Wegzusammenhangskomponenten von $X$ sind die Äquivalenzklassen
    bezüglich \ $x \sim y \Leftrightarrow \exists \text{Weg } w:[0,1] \rightarrow X$
    mit $w(0) = x$ und $w(1) = y$.
\end{satz}

\begin{definition}[(Weg-)Zusammenhangskomponente]
    Sei $\K = \C$ oder $\R$. Die (Weg-)Zusammenhangskomponenten von $G \subset \GL(n,\K)$
    sind die Äquivalenzklassen bezüglich $\sim$. Besteht $G$ aus einer einzigen
    Zusammenhangskomponente, so heisst $G$ (weg-)zusammenhängend.
\end{definition}

\begin{satz}
    Sei $G \subset GL(n,\K)$ eine Untergruppe/ Lie-Gruppe und $G_0 \subset G$
    die Wegzusammenhangskomponente der 1. Dann ist $G_0$ ein Normalteiler von
    $G$ und $G/G_0$ ist isomorph zu der Gruppe der Wegzusammenhangskomponenten.
\end{satz}

\begin{satz}
    \begin{enumerate}[(i)]
        \item $SO(n), SU(n), U(n)$ sind zusammenhängend.
        \item $O(n)$ besteht aus zwei Zusammenhangskomponenten:
            $\geschwungeneklammer{A \in O(n) \ | \ \det(A) = 1}$ und
            $\geschwungeneklammer{A \in O(n) \ | \ \det(A) = -1}$.
    \end{enumerate}
\end{satz}

\begin{theorem}[Spektralsatz]
    Für einen Endomorphismus $F$, mit Darstellungsmatrix $A \in M(n \times n ; \C)$,
    eines unitären $\C$-VR $V$ sind folgende Aussagen äquivalent:
    \begin{enumerate}[i)]
        \item Es gibt eine Orthonormalbasis von $V$ bestehend aus Eigenvektoren
        von $F$.
        \item $F$ ist normal.
        \item $\exists S \in U(n)$ s.d. $S A S^{-1} = D$ für $D$ eine Diagonalmatrix.
    \end{enumerate}
\end{theorem}

\subsection{Bahnformel}

\begin{definition}[Bahn]
    Sei $G$ eine endliche Gruppe, die auf der Menge $X$ operiert. Zu $x \in X$
    definieren wir die Bahn von $x$
    \begin{align*}
        G x := \geschwungeneklammer{g x \ | \ g \in G} \subset X
    \end{align*}
\end{definition}

\begin{definition}[Stabilisator]
    Sei $G$ und $X$ wie oben. Dann:
    \begin{align*}
        \text{Stab}_x := \geschwungeneklammer{g \in G \ | \ g x = x} \subset G
    \end{align*}
    Stab$_x$ ist eine Untergruppe.
\end{definition}

\begin{satz}[Bahnensatz und Bahnenformel]
    Wirkt die Gruppe $G$ auf der Menge $X$, dann ist für jedes $x \in X$
    die Abbildung
    \begin{align*}
        G / \text{Stab}_x \rightarrow G x
        \hspace{15pt} , \hspace{15pt}
        [g] \mapsto g x
    \end{align*}
    wohldefiniert und eine Bijektion. Insbesondere gilt für endliches $G$
    die Bahnenformel $\abs{G} = \abs{\text{Stab}_x} \abs{G x}$
\end{satz}


\section{Darstellungen von Gruppen}

\subsection{Definitionen}

\begin{definition}[Darstellung]
    Eine Darstellung einer Gruppe $G$ auf einem VR $V \neq 0$ ist ein Homomorphismus
    $\rho: G \rightarrow \GL(V)$. Der VR $V$ heisst dann Darstellungsraum
    der Darstellung $\rho$. Also ordnet eine Darstellung $\rho$ jedem Element
    $g \in G$ eine invertierbare lineare Abbildung $\rho(g): V \rightarrow V$
    zu, so dass $\forall g,h \in G$ die Darstellungseigenschaft
    $\rho(g h) = \rho(g) \rho(h)$ gilt.
\end{definition}

\begin{definition}[reguläre Darstellung]
    Die reguläre Darstellung einer endlichen Gruppe $G$ ist die Darstellung
    auf dem Raum $\C(G)$ aller Funktionen $G \rightarrow \C$,
    \begin{align*}
        (\rho_{reg}(g)f)(h) = f(g^{-1} h)
        \hspace{15pt} , \hspace{15pt}
        f \in \C(G) , g,h \in G
    \end{align*}
    Alternativ: $\C(G)$ hat eine Basis $\geschwungeneklammer{\delta_g}_{g \in G}$ mit
    $\delta_g(g) = 1$ und $\delta_g(h) = 0$ wenn $h \neq g$. Dann ist
    $\rho_{reg}$ die Darstellung, s.d. $\rho_{reg}(g) \delta_h = \delta_{gh}$.
\end{definition}

\paragraph{Notation}
Wir notieren Darstellungen als $(\rho,V)$ oder $\rho$ oder $V$ falls keine
Verwirrung entsteht.

\begin{definition}[Homomorphismus von Darstellungen]
    Ein Homomorphismus von Darstellungen $(\rho_1,V_1) \rightarrow (\rho_2,V_2)$
    ist eine lineare Abbildung $\varphi: V_1 \rightarrow V_2$ s.d.
    $\varphi \rho_1(g) = \rho_2(g)\varphi \ \forall g \in G$.
\end{definition}

\begin{definition}[Äquivalent]
    Zwei Darstellungen $(\rho_1,V_1)$, $(\rho_2,V_2)$ sind äquivalent (oder isomorph)
    falls ein bijektiver Homomorphismus von Darstellungen $\varphi: V_1 \rightarrow
    V_2$ existiert.
\end{definition}

\begin{korollar}
    Der Vektorraum aller Homomorphismen $(\rho_1,V_1) \rightarrow
    (\rho_2,V_2)$ wird mit $\Hom_G(V_1,V_2)$ oder $\Hom_G((\rho_1,V_1),(\rho_2,V_2))$
    bezeichnet.
\end{korollar}

\begin{definition}[invarianter Unterraum]
    Ein invarianter Unterraum einer Darstellung $(\rho,V)$ ist ein UVR
    $W \subset V$ mit $\rho(g)W \subset W \ \forall g \in G$. 
\end{definition}

\begin{definition}[(Ir-)reduzibel]
    Eine Darstellung $(\rho,V)$ heisst irreduzibel, falls sie keine invarianten
    Unterräume ausser $V$ und $\geschwungeneklammer{0}$ besitzt, sonst
    reduzibel.
\end{definition}

\begin{lemma}
    Ist $W \neq \geschwungeneklammer{0}$ ein invarianter
    Unterraum, so ist die Einschränkung $\rho_{|W}: G \rightarrow \GL(W)$,
    $g \mapsto \rho(g)_{|W}$ eine Darstellung: $(\rho_{|W} , W)$ ist eine
    Unterdarstellung von $(\rho,V)$.
\end{lemma}

\begin{definition}[vollständig reduzibel]
    Eine Darstellung $(\rho,V)$ heisst vollständig reduzibel, falls
    invariante UVR $V_1,\dots,V_n$ existieren, s.d.
    $V = V_1 \oplus \dots \oplus V_n$ und die Unterdarstellungen
    $(\rho_{|V_i},V_i)$ irreduzibel sind. Eine solche Zerlegung von $V$
    heisst Zerlegung in irreduzible Darstellungen.
\end{definition}

\begin{bemerkung}
    Nicht jede reduzible Darstellung ist vollständig reduzibel.
\end{bemerkung}

\begin{lemma}
    Sei $(\rho,V)$ eine endlichdimensionale Darstellung s.d. $\forall$
    invarianten UVR $W \subset V$ $\exists$ ein invarianter UVR $W'$ mit
    $V = W \oplus W'$. Dann ist $(\rho,V)$ vollständig reduzibel.
\end{lemma}

\subsection{Unitäre Darstellungen}

\begin{definition}[unitäre Darstellung]
    Eine Darstellung $\rho$ auf einem VR $V$ mit Skalarprodukt heisst
    unitär falls $\rho(g)$ unitär ist $\forall g \in G$.
    Sei $\rho(g)$ unitär, dann: $\rho(g)^\ast = \rho(g)^{-1} \ \forall g \in G$
    bzw. $\rho(g^{-1}) = \rho(g)^\ast$
\end{definition}

\begin{satz}
    Endliche unitäre Darstellungen sind vollständig reduzibel.
\end{satz}

\begin{satz}
    Sei $(\rho,V)$ eine endliche Darstellung einer endlichen Gruppe $G$.
    Dann $\exists$ ein Skalarprodukt $( \ , \ )$ auf $V$, s.d. $(\rho,V)$
    unitär ist.
\end{satz}

\begin{korollar}
    Darstellungen von endlichen Gruppen sind vollständig reduzibel.
\end{korollar}

\subsection{Das Lemma von Schur}

\begin{satz}[Lemma von Schur]
    Seien $(\rho_1,V_1),(\rho_2,V_2)$ irreduzible komplexe endlichdimensionale
    Darstellungen von $G$.
    \begin{enumerate}[(i)]
        \item $\varphi \in \Hom_G(V_1,V_2) \Rightarrow \varphi \equiv 0$ oder
            $\varphi$ ist ein Isomorphismus.
        \item $\varphi \in \Hom_G (V_1,V_1)$. Dann ist $\varphi = \lambda \text{Id}_{V_1}$
            für $\lambda \in \C$.
    \end{enumerate}
\end{satz}

\begin{korollar}
    Jede irreduzible endlichdimensionale komplexe Darstellung einer abelschen
    Gruppe ist eindimensional.
\end{korollar}


\section{Darstellungstheorie von endlichen Gruppen}

Es bezeichne $G$ stets eine endliche Gruppe. Alle Darstellungen werden
endlichdimensional und komplex angenommen.

\subsection{Orthogonalitätsrelationen der Matrixelemente}

\begin{satz}
    Sei $\rho: G \rightarrow \GL(V)$ eine irreduzible Darstellung der Gruppe $G$
    der Dimension $d$. Aus der Existenz eines Skalarproduktes auf $V$ bezüglich
    wessen $\rho$ unitär ist folgt dass für alle $g \in G$ die Matrix
    $(\rho_{ij}(g))$ von $\rho(g)$ bezüglich einer beliebigen orthonormierten
    Basis unitär ist: $\rho_{ij}(g^{-1}) = \overline{\rho_{ji}(g)}$
\end{satz}

\begin{satz}
    Seien $\rho: G \rightarrow \GL(V)$, $\rho' : G \rightarrow \GL(V)$ irreduzible
    unitäre Darstellungen einer endlichen Gruppe $G$. Es bezeichnen $(\rho_{ij}(g))$,
    $(\rho_{kl}' (g))$ die Matrizen von $\rho(g)$, $\rho'(g)$ bezüglich orthonormierten
    Basen $V$, bzw. $V'$.
    \begin{enumerate}[(i)]
        \item Sind $\rho$, $\rho'$ inäquivalent, so gilt für alle $i,j,k,l$
            \begin{align*}
                \frac{1}{\abs{G}} \sum_{g \in G} \overline{\rho_{ij}(g)} \rho_{kl}' (g) = 0
            \end{align*}
        \item Für alle $i,j,k,l$ gilt
            \begin{align*}
                \frac{1}{\abs{G}} \sum_{g \in G} \overline{\rho_{ij}(g)} \rho_{kl}(g) = \frac{1}{\dim V} \delta_{ik} \delta_{jl}
            \end{align*}
    \end{enumerate}
\end{satz}

\subsection{Charakteren}

\begin{definition}[Charakter]
    Der Charakter einer endlichdimensionalen Darstellung $\rho: G \rightarrow \GL(V)$
    einer Gruppe $G$ ist die komplexwertige Funktion auf $G$:
    \begin{align*}
        \chi_\rho (g) = \tr(\rho(g)) = \sum_{j = 1}^{\dim (V)} \rho_{jj}(g)
    \end{align*}
    Hier sind $\rho_{ij}(g)$ die Matrixelemente bezüglich einer beliebigen
    Basis von $G$.
\end{definition}

\begin{satz}
    \begin{enumerate}[(i)]
        \item $\chi_\rho (g) = \chi_{\rho} (h g h^{-1})$, äquivalent: $\chi_\rho$
            nimmt einen konstanten Wert auf jeder Konjugationsklasse an.
        \item Sind $\rho$, $\rho'$ äquivalente Darstellungen, so gilt $\chi_\rho = \chi_{\rho'}$
    \end{enumerate}
\end{satz}

\begin{definition}[Konjugationsklasse]
    Die Konjugationsklassen von $G$ sind die Mengen der Form
    $\geschwungeneklammer{h g h^{-1} \ | \ h \in G}$, oder äquivalent
    die Bahnen bzgl der Wirkung von $G$ auf sich selbst durch Konjugation
    $h \cdot g = h \cdot g \cdot h^{-1}$, oder äquivalent die Äquivalenzklasse
    bzgl. $g \sim g' \Leftrightarrow \exists h \in G : g' = h g h^{-1}$.
\end{definition}

\begin{lemma}
    \begin{enumerate}[(i)]
        \item $\chi_\rho (1) = \dim(V)$
        \item $\chi_{\rho \oplus \rho'} = \chi_{\rho} + \chi_{\rho'}$
        \item $\chi_{\rho} (g^{-1}) = \overline{\chi_\rho (g)} , \ \forall g \in G$
    \end{enumerate}
\end{lemma}

\subsection{Der Charakter der regulären Darstellung}
\begin{align*}
    \chi_{reg} (g) = \begin{cases}
        \abs{G},& \hspace{10pt} \text{falls } g = 1
        \\
        0,& \hspace{10pt} \text{sonst}
    \end{cases}
\end{align*}

\subsection{Orthogonalitätsrelationen der Charakteren}

\begin{definition}[Skalarprodukt]
    Wir führen das folgende Skalarprodukt auf dem Raum $\C(G)$ aller
    komplexwertigen Funktionen auf $G$ ein.
    \begin{align*}
        (f_1,f_2) = \frac{1}{\abs{G}} \sum_{g \in G} \overline{f_1(g)} f_2(g)
    \end{align*}
\end{definition}

\begin{satz}
    Seien $\rho$, $\rho'$ irreduzible Darstellungen der endlichen Gruppe $G$,
    und seien $\chi_\rho$, $\chi_{\rho'}$ ihre Charakteren. Dann gilt
    \begin{enumerate}[(i)]
        \item Sind $\rho$, $\rho'$ inequivalent, so gilt $(\chi_\rho,\chi_{\rho'}) = 0$
        \item Sind $\rho$, $\rho'$ äquivalent, so gilt $(\chi_\rho,\chi_{\rho'}) = 1$
    \end{enumerate}
\end{satz}

\begin{korollar}
    Ist $\rho = \rho_1 \oplus \dots \oplus \rho_n$ eine Zerlegung einer
    Darstellung $\rho$ in irreduzible Darstellungen, und $\sigma$ eine
    irreduzible Darstellung, so ist die Anzahl $\rho_i$ die äquivalent
    zu $\sigma$ sind gleich $(\chi_\rho,\chi_\sigma)$.
\end{korollar}

\begin{korollar}
    $\rho$ irreduzibel $\Leftrightarrow$ $(\chi_\rho,\chi_\rho) = 1$
\end{korollar}

\subsection{Zerlegung der regulären Darstellung}

\begin{satz}
    Jede irreduzible Darstellung $\sigma$ einer endlichen Gruppe $G$ kommt
    in der regulären Darstellung vor. Hat eine irreduzible Darstellung die
    Dimension $d$, so kommt sie $d$ mal in der regulären Darstellung vor.
    Äquivalent: Für jede irreduzible Darstellung $\sigma$ der Dimension $d$,
    $(\chi_\sigma,\chi_{reg}) = d$. Das ergibt sich aus:
    \begin{align*}
        n_\sigma = (\chi_\sigma,\chi_{reg}) = \frac{1}{\abs{G}} \sum_{g \in G} \overline{\chi_\sigma(g)} \chi_{reg}(g)
        = \chi_\sigma (1) = d
    \end{align*}
\end{satz}

\begin{korollar}
    Eine endliche Gruppe $G$ besitzt endlich viele Äquivalenzklassen irreduzibler
    Darstellungen. Ist $\rho_1,\dots,\rho_k$ eine Liste von irreduziblen
    inäquivalenten Darstellungen, eine in jeder Äquivalenzklasse, so gilt
    für ihre Dimensionen $d_i$:
    \begin{align*}
        d_1^2 + \dots + d_k^2 = \abs{G} = \sum_{j=1}^k (\dim(\rho_j))^2
    \end{align*}
    Es gilt $\chi_{reg}(g) = \sum_i d_i \chi_{\rho_i}(g)$. Für $g=1$ erhalten
    wir das obere Resultat.
\end{korollar}

\begin{korollar}
    Sei $\rho_1,\dots,\rho_k$ eine Liste irreduzibler inäquivalenter unitärer
    Darstellungen wie im vorherigen Korollar. Es bezeichne $\rho_{\alpha,ij}(g)$,
    $\alpha = 1,\dots,k$, $1 \leq i , j \leq d_\alpha$ die Matrixelemente von
    $\rho_\alpha (g)$ bezüglich einer orthonormierten Basis. Dann bilden die
    Funktionen $\rho_{\alpha,ij}$ eine orthogonale Basis von $\C(G)$.
\end{korollar}

\begin{definition}[Klassenfunktion]
    Eine Funktion $f: G \rightarrow \C$ heisst Klassenfunktion falls
    $f(g h g^{-1}) = f(h)$ für alle $g,h \in G$.
\end{definition}

\begin{lemma}
    Die Klassenfunktionen sind ein UVR von $\C(G)$ und damit selbst ein
    Hilbertraum.
\end{lemma}

\begin{korollar}
    Sei $G$ eine endliche Gruppe. Die Charakteren $\chi_1,\dots,\chi_k$ der
    irreduziblen Darstellungen von $G$ bilden eine orthonormierte Basis des
    Hilbertraums der Klassenfunktionen.
\end{korollar}

\begin{korollar}
    Eine endliche Gruppe hat so viele Äquivalenzklassen irreduzibler
    Darstellungen wie Konjugationsklassen.
\end{korollar}

\subsection{Die Charaktertafel einer endlichen Gruppe}

\begin{definition}[Charaktertafel]
    Die Charaktertafel ist eine Tabelle
    \begin{center}
        \begin{tabular}{c | c c c c c}
            $G$ & $[1]$ & $\dots$ & $c$ & $\dots$ & $[\dots]$ \\ \hline
            $\chi_1$ & $\ddots$ & & $\ddots$ & & \\
            $\vdots$ & & $\ddots$ & & $\ddots$ & \\
            $\chi_j$ & & & $\chi_j(c)$ & & \\
            $\vdots$ & $\ddots$ & & & $\ddots$ & \\
            $\chi_k$ & & $\ddots$ & & & $\ddots$
        \end{tabular}
    \end{center}
\end{definition}
mit $[1],\dots,c,\dots,[\dots]$ den Konjugationsklassen,
$\chi_1,\dots,\chi_j,\dots,\chi_k$ den irreduziblen Charakteren
und $\chi_j(c)$ den Werten der Charaktere. Oft schreibt man auch die Ordnung
der jeweiligen Äquivalenzklasse neben die Klasse und die Ordnung der Gruppe
neben $G$. Zeilen sind orthogonal, insbesondere
\begin{align*}
    \klammer{\sqrt{\frac{\abs{c_j}}{\abs{G}}} \chi_i (c_j)}_{ij} \in O(n)
\end{align*}
Das heisst, die Matrix hat auch orthonormale Spalten. Es gelten:
\begin{align*}
    \sum_{\alpha=1}^k \frac{\abs{c_\alpha}}{\abs{G}} \overline{\chi_i(c_\alpha)} \chi_j (c_\alpha) &= \delta_{i,j}
    \\
    \sum_{j=1}^k \overline{\chi_j (c_\alpha)} \chi_j (c_\alpha) &= \frac{\abs{G}}{\abs{c_\alpha}} \delta_{\alpha,\beta}
\end{align*}

\paragraph{Tricks zum finden der Charaktertafel}
\begin{itemize}
    \item $\abs{G} = \sum_{j=1}^k \klammer{\dim(\rho_j)}^2$
    \item Orthogonalität der Zeilen und Spalten
    \item Existenz der trivialen Darstellung (und allenfalls der Vorzeichendarstellung)
    \item 1. Spalte enthält die Dimensionen.
\end{itemize}

\subsection{Die kanonische Zerlegung einer Darstellung}

\begin{satz}
    Sei $G$ eine endliche Gruppe und $\rho_i : G \rightarrow \GL(V)$,
    $i=1,\dots,k$ einer Liste aller inäquivalenter irreduziblen
    Darstellungen von $G$. Sei eine Darstellung $\rho$ auf einem VR $V$
    gegeben. Es sei $V = U_1 \oplus \dots \oplus U_n$ eine Zerlegung in
    irreduzible invarianten Unterräume. $\forall i = 1,\dots,k$ definieren
    wir $W_i$ als die direkte Summe aller derjenigen $U_j$, so dass
    $\rho_{|_{U_j}}$ äquivalent zu $\rho_i$ ist. Dann ist
    $V = W_1 \oplus \dots \oplus W_k$, wobei $W_i = 0$ sein darf.
\end{satz}

\begin{satz}
    Die Zerlegung $V = W_1 \oplus \dots \oplus W_k$ ist unabhängig von der
    Wahl der Zerlegung von $V$ in irreduziblen Darstellungen. Die Projektion
    $p_i : V \rightarrow W_i$, $w_1 \oplus \dots \oplus w_k \mapsto w_i$ ist
    gegeben durch:
    \begin{align*}
        p_i(v) = \frac{\dim (V_i)}{\abs{G}} \sum_{g \in G} \overline{\chi_i(g)} \rho(g) v
    \end{align*}
\end{satz}

\begin{bemerkung}
    Die Zerlegung $V = W_1 \oplus \dots \oplus W_k$ heisst kanonische
    Zerlegung. Die Unterräume $W_i$ heissen isotypische Komponenten.
\end{bemerkung}

\subsection{Beispiel: Die Diedergruppe $D_n$}

Jede Darstellung ist eindeutig bestimmt durch $\rho(R) =: \overline{R}$
und $\rho(S) =: \overline{S} \in \GL(V)$. Es gilt:
\begin{align*}
    \overline{R}^n = \overline{S}^2 = 1
    \hspace{10pt} , \hspace{10pt}
    \overline{S} \overline{R} = \overline{R}^{-1} \overline{S}
    \hspace{10pt} , \hspace{10pt}
    R^a S^b R^{a'} S^{b'} = R^{a + a' - 2 b a'} S^{b + b'}
\end{align*}

\paragraph{Eindimensionale Darstellung $V = \C \backslash \geschwungeneklammer{0}$:}
\begin{enumerate}[]
    \item \underline{$n$ ungerade}: 2 (irreduzible) 1-dim Darstellungen: $\rho_{\pm}$
    \item \underline{$n$ gerade}: 4 1-dim Darstellungen: $\rho_{\pm \pm}$
\end{enumerate}

\paragraph{Irreduzible 2-dim Darstellung $V = \C^2$:}
Sei $v \in V$ ein EV von $\overline{R} \in \GL(2,\C)$ zum EW $\epsilon$,
$\overline{R} v = \epsilon v$. Dann gilt: $\overline{S} \overline{R} v
= \epsilon \overline{S} v = \overline{R}^{-1} \overline{S} v \Leftrightarrow
\overline{R} \overline{S} v = \frac{1}{\epsilon} \overline{S} v$. D.h.
$\overline{S} v$ ist ein EV von $\overline{R}$ zum EW $\frac{1}{\epsilon}$.
$v,\overline{S}v$ sind linear unabhängig also eine Basis von $\C^2$. Bezüglich
dieser Basis gilt dann:
\begin{align*}
    \overline{R} = \begin{pmatrix}
        \epsilon & 0 \\ 0 & \frac{1}{\epsilon}
    \end{pmatrix}
    \hspace{10pt} , \hspace{10pt}
    \overline{S} = \begin{pmatrix}
        0 & 1 \\ 1 & 0
    \end{pmatrix}
\end{align*}
Weiter gilt $\epsilon = e^{\frac{2 \pi i}{n} j} \ , \ j \in \Z$.
Mit diesen $\epsilon$ können wir Darstellungen $\rho_j$ definieren.
Wir beschränken uns auf die Werte $j=1,2,\dots,\floor{\frac{n-1}{2}}$.

Für die Charaktere der Darstellungen gilt:
\begin{align*}
    \chi_j (R^a) = \epsilon_j^a + \epsilon_j^{-a} = 2 \cos \klammer{\frac{2 \pi j}{n} a}
    \hspace{10pt} , \hspace{10pt}
    \chi_j (R^a S) = 0
\end{align*}
Nebenrechnung: $\lambda$ $n$-te Einheitswurzel:
\begin{align*}
    \sum_{a=0}^{n-1} \lambda^a = \begin{cases}
        0 \hspace{10pt} \lambda \neq 1
        \\
        n \hspace{10pt} \lambda = 1
    \end{cases}
\end{align*}

Es gilt: $\klammer{\chi_i , \chi_j} = \delta_{ij}$. Somit sind die
$\rho_i$ irreduzibel und $\rho_i , \rho_j$ sind äquivalent für $i \neq j$
($i,j \in \geschwungeneklammer{1,\dots,\floor{\frac{n-1}{2}}}$).

Die gefundene Liste von irreduziblen Darstellungen ist vollständig.

\subsection{Kompakte Gruppen}
Die "Mittelung" für die Darstellungstheorie endlicher Gruppen
$\frac{1}{\abs{G}} \sum_{g \in G} f(g)$ lässt sich für kompakte Gruppen
verallgemeinern zu:
\begin{align*}
    \int_G f(g) dg
\end{align*}
und wird Haar Mass genannt. Es hat folgende Eigenschaften:
\begin{align*}
    \int_G 1 \ dg = 1
    \hspace{10pt} , \hspace{10pt}
    \int_G f(g h) \ dg = \int_G f(g) \ dg \ \forall h \in G
\end{align*}
Es gilt Orthogonalität für Matrixelemente und Charaktere bzgl.
\begin{align*}
    (f_1,f_2) = \int_G \overline{f_1(g)} f_2(g) \ dg
\end{align*}


\section{Darstellungstheorie der symmetrischen Gruppe}

\subsection{Partitionen}

\begin{definition}[Partition]
    Sei $n \geq 1$ eine natürliche Zahl. Eine Partition von $n$ ist eine
    Zerlegung von $n$ in eine Summe positiver ganzer Zahlen
    $n = \lambda_1 + \lambda_2 + \dots + \lambda_k$. Reihenfolge der Summanden
    ist nicht wichtig. Jede Partition ist eindeutig bestimmt durch die Anzahlen
    $i_1,i_2,\dots$ der Zahlen $1,2,\dots$ in der Zerlegung, wobei
    \begin{align*}
        n = \sum_{j \geq 1} j i_j
    \end{align*}
    Wir schreiben $\underline{i}$ für $(i_1,i_2,\dots)$
\end{definition}

\begin{definition}[Young-Diagramm]
    Ein graphischer Weg eine Partition $n = \lambda_1 + \dots + \lambda_k$
    darzustellen ist durch ein Young-Diagramm. Zuerst muss man die $\lambda_j$
    so sortieren, dass $\lambda_1 \geq \lambda_2 \geq \dots \geq \lambda_k$.
    Dann ist das zugehörige Young-Diagramm eine Anordnung von $n$ Kästli,
    mit $\lambda_i$ Kästli in der $i$-ten Zeile.
\end{definition}

\begin{definition}
    Seien $\lambda,\lambda'$ Young-Diagramme mit jeweils $n$ Kästli, dann
    sagen wir $\lambda \geq \lambda'$ genau dann wenn $\lambda = \lambda'$,
    oder falls die erste nicht verschwindende Zahl $\lambda_i - \lambda_i'$
    positiv ist. Entsprechend sagen wir $\lambda > \lambda'$, falls
    $\lambda \geq \lambda'$ und $\lambda \neq \lambda'$. Hierbei ist $\lambda_i$
    die Anzahl der Kästli in der $i$-ten Zeile.
\end{definition}

\subsection{Permutationen der Konjugationsklassen}

Wir können ein Element $\sigma \in S_n = \text{Bij}\klammer{\geschwungeneklammer{1,\dots,n}}$
auf verschiedene Arten aufschreiben. Als Wertetabelle oder Zyklenschreibweise:
\begin{align*}
    \sigma = \begin{pmatrix}
        1 & 2 & 3 & 4 & 5 \\ 4 & 1 & 5 & 2 & 3
    \end{pmatrix}
    = (142)(35)
    = (53)(421)
\end{align*}
Die Längen der Zyklen bestimmen eine Partition von $n$. Die einzelnen Zyklen
kann man auch verstehen als die Bahn in $\geschwungeneklammer{1,\dots,n}$
unter der Wirkung der von $\sigma$ erzeugten Untergruppe von $S_n$, wobei
die zyklische Ordnung auf den Zyklen unberücksichtigt bleibt. Wir schreiben
$i_k (\sigma)$ für die Anzahl der Zyklen der Länge $k$ in der Zyklenschreibweise
von $\sigma$, und $\underline{i}(\sigma) = \klammer{i_1 (\sigma),i_2(\sigma),\dots}$.
Offensichtlich gilt $\sum_{k \geq 1} k i_k (\sigma) = n$ und damit bestimmt
$\underline{i}(\sigma)$ eine Partition von $n$.

\begin{lemma}
    Sei $\tau \in S_n$ eine Permutation. In Zyklenschreibweise:
    $\tau = (i_{1,1} \dotsb i_{1,\lambda_1}) (i_{2,1} \dotsb i_{2,\lambda_2}) \dotsb (i_{k,1} \dots i_{k,\lambda_k})$.
    Sei $\sigma \in S_n$ beliebig. Dann gilt in Zyklenschreibweise:
    $\sigma \tau \sigma^{-1} = \klammer{\sigma(i_{1,1}) \dotsb \sigma(i_{1,\lambda_1})}
    \klammer{\sigma(i_{2,1}) \dotsb \sigma(i_{2,\lambda_2})} \dotsb
    \klammer{\sigma(i_{k,1}) \dotsb \sigma(i_{k,\lambda_k})} =: \tau'$
\end{lemma}

\begin{korollar}
    \begin{enumerate}[(1)]
        \item $\forall k =1,2,\dots$ ist die Anzahl $i_k (\tau)$ der Zyklen
            der Länge $k$ in der Zyklendartellung von $\tau \in S_n$ eine
            Klassenfunktion, d.h. $i_k(\sigma \tau \sigma^{-1}) = i_k(\tau) \
            \forall \sigma,\tau \in S_n$.
        \item Zwei Permutationen $\tau,\tau' \in S_n$ sind genau dann in der gleichen
            Konjugationsklasse, wenn $\underline{i}(\tau) = \underline{i}(\tau')$.
        \item Die Konjugationsklassen von $S_n$ sind also in $1-1$-Korrespondenz zu
            den Partitionen von $n$.
    \end{enumerate}
\end{korollar}

\subsection{Die Gruppenalgebra einer endlichen Gruppe}

\begin{definition}[Gruppenalgebra]
    Sei $G$ eine endliche Gruppe. Dann ist die Gruppenalgebra $\C[G]$ der
    VR der formalen Linearkombinationen $\sum_{g \in G} a_g g$ mit
    $a_g \in \C$. Insbesondere ist $G \subset \C[G]$ eine Basis von
    $\C[G]$. Die Gruppenalgebra ist ein Ring mit dem biliniaren, assoziativen
    Produkt $\C[G] \otimes \C[G] \rightarrow \C[G]$:
    \begin{align*}
        \klammer{\sum_{g \in G} a_g g} \klammer{\sum_{g' \in G} a_{g'}' g'}
        = \sum_{g \in G} \klammer{\sum_{\stackrel{h,h' \in G}{hh' = g}} a_h a_{h'}'} g
        = \sum_{g \in G} \sum_{g' \in G} a_g a_{g'}' (g \cdot g')
    \end{align*}
    und dem Einselement dem neutralen Element der Gruppe $1 \in G \subset
    \C[G]$. Der VR $\C[G]$ der komplexwertigen Funktionen auf der Gruppe $G$
    ist der Dualraum von $\C[G]$, also $\C(G) = \C[G]^\ast$. Die Gruppenalgebra
    trägt eine Darstellung der Gruppe $G$ durch Linksmultiplikation,
    $\rho_{GA}(g) p = g p$, wobei $G \subset \C[G]$ verstenden ist. Diese
    Darstellung ist äquivalent zur regulären Darstellung auf $\C(G)$, wobei
    der Isomorphismus $\C[G] \rightarrow \C(G)$ das Basiselement $g \in G$
    abbildet auf $\delta_g$.
\end{definition}

\begin{theorem}
    Sei $\rho: G \rightarrow \GL(V) \subset \End(V)$ eine Darstellung. Dann
    können wir diese linear fortsetzen zu einer linearen Abbildung
    \begin{align*}
        \rho: \C[G] \rightarrow \End(V)
        \hspace{10pt} , \hspace{10pt}
        \rho \klammer{\sum_{g \in G} a_g g} = \sum_{g \in G} a_g \rho(g)
    \end{align*}
    Diese Abbildung erfüllt $\rho(xy) = \rho(x) \rho(y) \ \forall x,y \in \C[G]$,
    ist also auch ein Ringhomomorphismus.
\end{theorem}

\begin{satz}
    Sei $\rho_1,\dots,\rho_k$ eine Liste der inäquivalenten irreduziblen
    komplexen Darstellungen von $G$ mit Darstellungsräumen $V_1,\dots,V_k$.
    Dann ist die direkte Summe $\bigoplus_{j=1}^k \End(V_j)$ wieder ein
    Ring. Wählen wir Basen auf den $V_j$, so können wir $\bigoplus_{j=1}^k \End(V_j)$
    identifizieren mit dem Ring der Blockdiagonalmatrizen
    \begin{align*}
        \begin{pmatrix}
            A_1 & & \\
            & \ddots & \\
            & & A_k
        \end{pmatrix}
    \end{align*}
    mit Diagonalblöcken $A_j$ der Grösse $\dim(V_j) \times \dim(V_j)$
\end{satz}

\begin{satz}
    Die folgende Abbildung von VR ist ein Isomorphismus von Ringen.
    \begin{align*}
        \phi: \C[G] &\rightarrow \bigoplus_{j=1}^{k} \End(V_j)
        \\
        x &\mapsto \klammer{\rho_1(x),\rho_2(x),\dots,\rho_k(x)}
    \end{align*}
\end{satz}

\subsection{Irreduzible Darstellungen}

\begin{definition}[Young-Schema]
    Ein Young-Schema ist ein Young-Diagramm, dessen $n$ Kästli mit den Zahlen
    $1,\dots,n$ gefüllt sind, wobei jede Zahl genau einmal vorkommt. Zu jedem
    Young-Diagramm $\lambda$ definieren wir das Young-Schema $\hat{\lambda}_{norm}$,
    das aus $\lambda$ gewonnen wird durch füllen der Kästli mit den Zahlen
    $1,2,\dots,n$ aufsteigend von links nach rechts und dann von oben nach unten.
\end{definition}

\begin{definition}
    Für $\lambda$ (bzw. $\hat{\lambda}$) ein Young-Diagramm (bzw. Young-Schema)
    sei $\lambda^T$ (bzw. $\hat{\lambda}^T$) das Young-Diagramm (bzw. Young-Schema),
    dass durch Spiegelung von $\lambda$ (bzw. $\hat{\lambda}$) an der zweiten
    Diagonale gewonnnen wird.
\end{definition}

\begin{definition}
    Zu jedem Young-Schema $\hat{\lambda}$ definieren wir nun eine Untergruppe
    $G_{\hat{\lambda}} \subset S_n$, wobei $\sigma \in G_{\hat{\lambda}}$ genau
    dann wenn $\forall j \in \geschwungeneklammer{1,\dots,n}$ die Zahl
    $\sigma(j)$ in der gleichen Zeile in $\hat{\lambda}$ steht wie $j$.
\end{definition}

\begin{definition}
    Zu jedem Young-Schema $\hat{\lambda}$ ordnen wir nun die folgenden
    beiden Elemente der Gruppenalgebra $\C[S_n]$ zu:
    \begin{align*}
        s_{\hat{\lambda}} := \sum_{\sigma \in G_{\hat{\lambda}}} \sigma
        \hspace{10pt} , \hspace{10pt}
        a_{\hat{\lambda}} := \sum_{\sigma \in G_{\hat{\lambda}^T}} \text{sgn}(\sigma) \sigma
    \end{align*}
\end{definition}
Wir erweitern die Definitionen von $G_{\hat{\lambda}},s_{\hat{\lambda}},a_{\hat{\lambda}}$
von Young-Schemata auf Young-Diagramme, indem wir definieren:
\begin{align*}
    G_{\lambda} := G_{\hat{\lambda}_{norm}}
    \hspace{10pt} , \hspace{10pt}
    s_\lambda := s_{\hat{\lambda}_{norm}}
    \hspace{10pt} , \hspace{10pt}
    a_{\lambda} := a_{\hat{\lambda}_{norm}}
\end{align*}

\begin{definition}
    Zu einem Young-Diagramm $\lambda$ mit $n$ Kästli definieren wir den
    invarianten Unterraum
    \begin{align*}
        V_{\lambda} = \C[S_n] s_\lambda a_\lambda
        = \geschwungeneklammer{ x s_\lambda a_\lambda \ | \ x \in \C[S_n]}
        \subset \C[S_n]
    \end{align*}
    Ferner definieren wir die Darstellung $\rho_\lambda$ von $S_n$ auf
    $V_\lambda$ durch Einschränkung der Darstellung auf $\C[S_n]$ durch
    Linksmultiplikation.
\end{definition}

\begin{satz}
    Die Darstellungen $\rho_\lambda$ der vorherigen Definition sind
    irreduzibel, und für $\lambda \neq \lambda'$ sind $\rho_\lambda$
    und $\rho_{\lambda'}$ inäquivalent.
\end{satz}

\begin{definition}
    $c_{\hat{\lambda}} = s_{\hat{\lambda}} a_{\hat{\lambda}}$ ,
    $c_\lambda = s_\lambda a_\lambda$
\end{definition}

\begin{lemma}
    Sei $\hat{\lambda}$ ein Young-Schema mit $n$ Kästli, mit unterliegendem
    Young-Diagramm $\lambda$. Dann gilt:
    \begin{enumerate}[(1)]
        \item Das neutrale Element von $G$ hat Koeffizient $1$ in $c_{\hat{\lambda}}$.
            Insbesondere gilt $c_{\hat{\lambda}} \neq 0$.
        \item $\forall g \in G_{\hat{\lambda}}$ ist $g s_{\hat{\lambda}} = s_{\hat{\lambda}} = s_{\hat{\lambda}}$
        \item $\forall h \in G_{\hat{\lambda}^T}$ ist $h a_{\hat{\lambda}} = a_{\hat{\lambda}} = \text{sgn}(h)a_{\hat{\lambda}}$
        \item Für $\sigma \in S_n$ beliebig gilt
            \begin{align*}
                G_{\sigma \hat{\lambda}} = \geschwungeneklammer{\sigma g \sigma^{-1} \ | \ g \in G_{\hat{\lambda}}}
            \end{align*}
            wobei das Young-Schema $\sigma \hat{\lambda}$ aus $\hat{\lambda}$
            durch Anwendungen von $\sigma$ auf die Einträge gewonnen ist.
            Insbesondere gilt damit auch $\sigma s_{\hat{\lambda}} \sigma^{-1} =
            s_{\sigma \hat{\lambda}}$ und $\sigma a_{\hat{\lambda}} \sigma^{-1}
            = a_{\sigma \hat{\lambda}}$.
    \end{enumerate}
\end{lemma}

\begin{lemma}
    Seien $\hat{\lambda} , \hat{\mu}$ Young-Schemata mit $n$ Kästli, mit
    unterliegenden Young-Diagrammen $\lambda,\mu$.
    \begin{enumerate}[(1)]
        \item Sei $\lambda > \mu$ und $x \in \C[G]$ beliebig. Dann gilt
            $s_{\hat{\lambda}} x a_{\hat{\mu}} = 0$, und damit insbesondere
            $c_{\hat{\lambda}} c_{\hat{\mu}} = 0 = c_{\hat{\lambda}} x c_{\hat{\mu}}$
        \item Sei $\lambda = \mu$. Dann gilt genau eine der beiden Aussagen:
            \begin{enumerate}[(a)]
                \item $\exists i \neq j$ Zahlen, die in $\hat{\lambda}$ in einer
                    Zeile, und in $\hat{\mu}$ in einer Spalte vorkommen.
                \item $\exists h_1 \in G_{\hat{\lambda}}$ und $h_2 \in G_{\hat{\mu}^T}$,
                    so dass $h_1 \hat{\lambda} = h_2 \hat{\mu}$
            \end{enumerate}
        \item $\forall x \in \C[G]$ ist $s_{\hat{\lambda}} x a_{\hat{\lambda}}$
            ein Vielfaches von $s_{\hat{\lambda}} a_{\hat{\lambda}} = c_{\hat{\lambda}}$.
            Insbesondere ist $c_{\hat{\lambda}} x c_{\hat{\lambda}}$ ein Vielfaches
            von $c_{\hat{\lambda}}$
    \end{enumerate}
\end{lemma}

\begin{lemma}
    \begin{enumerate}[(1)]
        \item Sei $A$ eine komplexe $n \times n$-Matrix so dass für alle $n \times n$-Matrizen
            $X$ gilt, dass $A X A$ ein Vielfaches von $A$ ist. Dann gibt es Vektoren
            $u,v \in \C^n$, so dass $A = u v^\dagger$.
        \item Sei $A = \begin{pmatrix}
            A_1 & & \\ & \ddots & \\ & & A_k
        \end{pmatrix}$ eine Blockdiagonalmatrix mit Diagonalblöcken der Grösse
        $d_j \times d_j$ mit $j = 1,\dots,k$. Es gelte für jede Blockdiagonalmatrix
        $X = \begin{pmatrix}
            X_1 & & \\ & \ddots & \\ & & X_k
        \end{pmatrix}$ gleicher Form, dass $A X A$ ein Vielfaches von $A$ ist.
        Dann existiert ein $j \in \geschwungeneklammer{1,\dots,k}$ und $u,v \in \C^{d_j}$,
        so dass $A_i = 0$ für $i \neq j$ und $A_j = u v^\dagger$.
    \end{enumerate}
\end{lemma}

\subsection{Die Charakterformel von Frobenius}

\begin{satz}[Frobeniusformel]
    Sei $\lambda$ ein Young-Diagramm mit $n$ Kästli. Sei $\underline{i} =
    (i_1,i_2,\dots)$ eine Partition von $n$, und $C_{\underline{i}}$ die
    zugehörige Konjugationsklasse von $S_n$. Dann gilt
    \begin{align*}
        \chi_{\rho_{\lambda}} \klammer{C_{\underline{i}}} =
        \klammer{\Delta (x) \prod_k P_k^{i_k} (x)}_{x^{\lambda + \rho}}
    \end{align*}
    mit der Folgenden Notation:
    \begin{itemize}
        \item $x = (x_1,\dots,x_n)$, und für einen Multiindex $\lambda =
            (\lambda_1,\dots,\lambda_n)$, $x^\lambda = x_1^{\lambda_1}
            \dotsb x_n^{\lambda_n}$.
        \item $\Delta(x) = \prod_{1 \leq i < j \leq n} (x_i - x_j)$ ist
            die Vandermonde-Determinante.
        \item $P_k (x) = x_1^k + x_2^k + \dotsb + x_n^k$
        \item Die Notation $\klammer{Q}_{x^a}$ bezeichnet den Koeffizienten
            von $x^a$ im Polynom $Q(x)$.
        \item $\rho = (n-1,n-2,\dots,1,0)$
    \end{itemize}
\end{satz}

\begin{definition}[Haken]
    Der $i,j$-Haken des Young-Diagrammes $\lambda$ als die Menge der Kästli
    die rechts neben, oder unter dem Kästli an der Stelle $i,j$ stehen,
    inklusive des Kästli $i,j$ selbts.
\end{definition}

\begin{definition}[Hakenlänge]
    Die Hakenlänge $h(i,j)$ ist die Anzahl Kästli im $i,j$-Haken.
\end{definition}

\begin{korollar}[Hakenlängenformel]
    Die Dimension der irreduziblen Darstellung $\rho_\lambda$ von $S_n$ ist
    \begin{align*}
        \dim (\rho_\lambda) = \frac{n!}{\prod_{i,j} h(i,j)}
    \end{align*}
    wobei $h(i,j)$ die Länge des $i,j$-Hakens im Young-Diagramm $\lambda$ ist.
    Das Produkt läuft über die Koordinaten $i,j$ von allen Kästli in $\lambda$.
\end{korollar}


\section{Eigenwertprobleme mit Symmetrie}

\subsection{Eigenwerte und Eigenvektoren}

\begin{satz}
    Sei $\rho: G \rightarrow \GL(V)$ eine endlichdimensionale komplexe
    Darstellung einer komplexen endlichen Gruppe $G$, und $A: V \rightarrow V$
    eine diagonalisierbare lineare Selbstabbildung, so dass $\rho(g) A =
    A \rho(g) \ \forall g \in G$. Sei $V = V_1 \oplus \dotsb \oplus V_n$
    eine Zerlegung von $V$ in irreduzible Darstellungen. Dann hat $A$
    höchstens $n$ verschiedene Eigenwerte. Bezeichnet $d_i$ die
    Dimension von $V_i$, so hat $A$ bezüglich einer passenden Basis die
    Diagonalform
    \begin{align*}
        \text{diag} (\underbrace{\lambda_1,\dots,\lambda_1}_{d_1 \text{ mal}}
            ,\dots,\underbrace{\lambda_n,\dots,\lambda_n}_{d_n \text{ mal}})
    \end{align*}
    für gewisse (nicht notwendigerweise verschiedene) komplexe Zahlen
    $\lambda_1,\dots,\lambda_n$.
\end{satz}

\begin{satz}
    Seien $G,V,A$ wie im vorherigen Satz. Seien $\forall \ i \neq j$ die
    Darstellungen $V_i, V_j$ nicht äquivalent. Dann ist, $\forall \ i$,
    $A V_i \subset V_i$ und die Einschränkung von $A$ auf $V_i$ ist
    $A_{| V_i} = \lambda_i 1_{V_i}$ für ein $\lambda_i \in \C$.
    Also ist $A$ bezüglich einer Basis $V$ mti Basisvektoren in $\bigcup_i V_i$
bereits diagonal.
\end{satz}

\begin{bemerkung}
    Im allgemeinen Fall können die Eigenvektoren wie folgt
    bestimmt werden. Sei $V = W_1 \oplus \dotsb \oplus W_k$ die kanonische
    Zerlegung der Darstellung $\rho$. Nach dem Lemma von Schur ist $A W_i
    \subset W_i$. Also können wir $A$ separat in jedem $W_i$ diagonalisieren,
    und wir haben das Problem auf den Fall reduziert, wo $V$ eine direkte
    Summe von zueinander äquivalenten irreduziblen Darstellungen ist.
    Der allgemeine Fall, wo $V$ eine direkte Summe von $n$ zueinander
    äquivalenten Derstellungen $V_\alpha$ wird wie folgt behandelt.
    Die Isomorphismen zwischen den Darstellungen erlauben und für jedes
    $\alpha = 1,\dots,n$ eine Basis $(e_i^\alpha)_{i=1,\dots,d}$ von $V_\alpha$
    zu wählen, so dass die Matrix von $\rho(g)$ bezüglich der Basis
    $e_1^1,\dots,e_d^1,\dots,e_1^n,\dots,e_d^n$ kästchendiagonalform mit
    gleichen diagonalen $d \times d$ Kästchen hat. Nach Schur hat dann die
    Matrix von $A$ bezüglich der umnummerierten Basis $e_1^1,\dots,e_1^n,\dots,
    e_d^1,\dots,e_d^n$ die folgende Form
    \begin{align*}
        \begin{pmatrix}
            a & & \\
            & \ddots & \\
            & & a
        \end{pmatrix}
        \hspace{20pt} , \text{ mit } a = (a_{ij})_{\stackrel{i=1,\dots,n}{j=1,\dots,n}}
        \in \Mat(n \times n)
    \end{align*}
    mit $n \times n$ Kästchen $a=(a_{\alpha \beta})$ gegeben durch
    \begin{align*}
        A e_i^\alpha = \sum_{\beta} a_{\beta \alpha} e_i^\beta
    \end{align*}
\end{bemerkung}


\subsection{Kleine Schwingungen von Molekülen}

Wir betrachten kleine Schwingungen eines Moleküls (bzw eines Systems
von $N$ Teilchen) aus einer Ruhelage. Die $N$ Teilchen haben Massen
$m_i \ (i=1,\dots,N)$ und Koordinaten $y = (\vec{y}_1,\dots,\vec{y}_N) \in
\R^{3 N}$, mit $\vec{y}_i \in \R^3$ der Position der $i$-ten Teilchens.
Die potentielle Energie sei $V(y) = (\vec{y}_1,\dots,\vec{y}_N)$. Die
Bewegungsgleichung ist
\begin{align*}
    m_i \ddot{\vec{y}}_i \stackrel{(\ast)}{=} - \frac{\partial V}{\partial \vec{y}_i} (y(t))
     \ \forall i = 1,\dots,N
\end{align*}
Sei $y^\ast \in \R^{3 N}$, $y^\ast = (\vec{y}_1^\ast,\dots,\vec{y}_N^\ast)$
ein Gleichgewichtspunkt, d.h. $\vec{\nabla} V(y^\ast) = 0$ und betrachte
kleine Auslenkungen $y(t) = y^\ast + x(t)$. Entwickeln in eine Taylorreihe
um $y^\ast$ ergibt aus $(\ast)$:
\begin{align*}
    m_i \ddot{\vec{x}}_i^\alpha &=
    \sum_{j,\beta} \frac{\partial^2 V}{\partial y_i^\alpha \partial y_j^\beta} (y^\ast) x_j^\beta
    + \underbrace{\mathcal{O}(\abs{x}^2)}_{\text{Vernachlässigen für kleine $x$}}
    \\
    \Leftrightarrow \ddot{\vec{x}}(t) &= - A x(t)
\end{align*}
mit $A$ der Matrix mit Matrixelementen
\begin{align*}
    \frac{1}{m_i} \frac{\partial^2 V}{\partial y_i^\alpha \partial y_j^\beta} (y^\ast)
\end{align*}
$A$ ist diagonalisierbar. Zur Lösung der DGL verwenden wir den Ansatz
$x(t) = e^{i \omega t} x_0$ mit $x_0 \in \R^{3 N}$. Dann folgt:
$\omega^2 x_0 = A x_0$. Die positiven Wurzeln der Eigenwerte von $A$
heissen Eigenfrequenzen des Systems. Nun zu den Symmetrien:

\begin{itemize}
    \item Zunächst soll $V$ invariant sein unter orthogonaler Transformaion,
        d.h. $\forall \ R \in O(3): \ V(R \vec{y}_1 , \dots , R \vec{y}_N)
        = V(\vec{y}_1 ,\dots,\vec{y}_N)$ (und Invarianz unter Translation).
    \item Ausserdem soll $V$ invariant sein unter Vertauschung gleichartiger
        Teilchen, d.h. $V(\vec{y}_{\sigma(1)},\dots,\vec{y}_{\sigma(N)}) =
        V(\vec{y}_1,\dots,\vec{y}_N) \ \forall \sigma \in S \subset S_N$ mit
        $S$ einer geeigneten Untergruppe von $S_N$. Ausserdem $m_{\sigma(i)}
        = m_i \ \forall \sigma \in S$.
    \item Vor der Wahl von $y^\ast$ ist die Symmetriegruppe des Systems
        $O(3) \times S$ mit der Darstellung $\rho: O(3) \times S \rightarrow
        \GL(\R^{3N})$ gegeben durch $\rho(R,\sigma)(\vec{y}_1,\dots,\vec{y}_N)
        = \klammer{R y_{\sigma^{-1}(1)},\dots,R \vec{y}_{\sigma^{-1}(N)}}$
    \item Wir betrachten den Unterraum $G = \geschwungeneklammer{g \in 
        O(3) \times S \ | \ \rho(g)(y^\ast) = y^\ast} \subset O(3) \times S$,
        die wieder durch Einschränkung von $\rho$ auf $\R^{3 N}$ wirkt. Es
        folgt, dass $\forall g \in G: \ \rho(g) A = A \rho(g)$, also
        $A \in \Hom_{G} (\R^{3 N},\R^{3 N})$.
\end{itemize}

\subsection{Beispiel: Eigenfrequenzen von $CH_4$}

Seien $\vec{y}_1,\dots,\vec{y}_4$ die Koordinaten der $H$-Atome und
$\vec{y}_C$ die des $C$-Atoms. Sei die Gleichgewichtlage
$\vec{y}^\ast = (\vec{y}_1^\ast , \dots,\vec{y}_4^\ast , \vec{y}_C^\ast)$
so, dass $\vec{y}_C^\ast = 0$ und die $\vec{y}_j^\ast$ die Eckpunkte eines
regulären Tetraeders bilden mit Zentrum $\vec{y}_C^\ast$. In diesem Fall ist
$G \cong T \cong S_4$ die Tetraedergruppe. Wir betrachten die Charaktertafel

\begin{table}[h]
    \centering
    \begin{tabular}{c|c c c c c}
        $24 T$ & $[1]$ & $8 [r_3]$ & $3 [r_2]$ & $6 [s_4]$ & $6 [\tau]$ \\ \hline
        $\chi_1$ & $1$ & $1$ & $1$ & $1$ & $1$ \\
        $\chi_2$ & $2$ & $-1$ & $2$ & $0$ & $0$ \\
        $\chi_3$ & $1$ & $1$ & $1$ & $-1$ & $-1$ \\
        $\chi_4$ & $3$ & $0$ & $-1$ & $1$ & $-1$ \\
        $\chi_5$ & $3$ & $0$ & $-1$ & $-1$ & $1$    
    \end{tabular}
\end{table}

$1$ ist die Identität; $r_3$ ist die Drehung um eine Achse durch eine Ecke
mit Winkel $2 \pi / 3 = 120^\circ$; $r_2$ ist eine Drehung um eine Achse, die
senkrecht durch eine Kante geht, mit winkel $2 \pi / 2 = \pi = 180^\circ$;
$s_4$ ist die Zusammensetzung einer $120^\circ$ Drehung $r_3$ um eine Achse
durch eine Ecke, sagen wir $\vec{v}_4$, und den Mittelpunkt mit einer Spiegelung
um eine Ebene die durch zwei andere Ecken $\vec{v}_1,\vec{v}_2$ und den
Mittelpunkt geht; schliesslich ist $\tau$ die Spiegelung bezüglich einer
durch eine Kante und den Mittelpunkt gehenden Ebene. Die entsprechende
Permutation der Ecken $1,2,3,4$ und des Mittelpunktes $C$ sind
\begin{align*}
    r_3 : \begin{pmatrix}
        1 & 2 & 3 & 4 & C \\ 1 & 3 & 4 & 2 & C
    \end{pmatrix}
    \hspace{10pt} &, \hspace{10pt}
    r_2 : \begin{pmatrix}
        1 & 2 & 3 & 4 & C \\ 2 & 1 & 4 & 3 & C
    \end{pmatrix}
    \\
    s_4 : \begin{pmatrix}
        1 & 2 & 3 & 4 & C \\ 4 & 1 & 2 & 3 & C
    \end{pmatrix}
    \hspace{10pt} &, \hspace{10pt}
    \tau : \begin{pmatrix}
        1 & 2 & 3 & 4 & C \\ 1 & 2 & 4 & 3 & C
    \end{pmatrix}
\end{align*}
Weiter gilt:
\begin{align*}
    \chi_1: \
    \begin{ytableau}
        \ & \ & \ & \
    \end{ytableau}
    \hspace{10pt} &, \hspace{10pt}
    \chi_2: \
    \begin{ytableau}
        \ & \ \\ \ & \
    \end{ytableau}
    \hspace{10pt} , \hspace{10pt}
    \chi_3 : \
    \begin{ytableau}
        \ \\ \ \\ \ \\ \
    \end{ytableau}
    \\
    \chi_4 : \
    \begin{ytableau}
        \ & \ \\ \ \\ \
    \end{ytableau}
    \hspace{10pt} &, \hspace{10pt}
    \chi_5 : \
    \begin{ytableau}
        \ & \ & \ \\ \
    \end{ytableau}
\end{align*}

Für $\rho(\tau)$ gilt:
\begin{align*}
    \rho(\tau) = \begin{pmatrix}
        \tau & 0 & 0 & 0 & 0 \\
        0 & \tau & 0 & 0 & 0 \\
        0 & 0 & 0 & \tau & 0 \\
        0 & 0 & \tau & 0 & 0 \\
        0 & 0 & 0 & 0 & \tau
    \end{pmatrix}
    \ \in \Mat(15 \times 15)
    \hspace{10pt} \text{ mit } \tau \in \Mat(3 \times 3)
\end{align*}
Es gilt $\tr(\rho(g)) = \tr(R) \cdot N_R$ mit $g = (R,\sigma_R) \in G$
und $N_R$ der Anzahl Diagonalblöcke $\neq 0$, d.h. $\geschwungeneklammer{
i \ | \ \sigma_R (i) = i}$. Für $\tau$: $\tr(\rho(\tau,\sigma_\tau)) = 3 \cdot
\tr(\tau)$. Rechnungen:
\begin{enumerate}[]
    \item Eine Drehung $R$ um $\theta$ hat bzgl einer Basis die Form
        \begin{align*}
            R(\theta) =
            \begin{pmatrix}
                \cos(\theta) & - \sin(\theta) & 0 \\
                \sin(\theta) & \cos(\theta) & 0 \\
                0 & 0 & 1
            \end{pmatrix}
            \ \rightarrow \ \tr(R) = 2 \cos(\theta) + 1
        \end{align*}
        \begin{align*}
            \Rightarrow
            \tr(r_3) = 2 \cos \klammer{\frac{2 \pi}{3}} + 1 = 0
            \hspace{10pt} , \hspace{10pt}
            \tr(r_2) = 2 \cos(\pi) + 1 = -1
        \end{align*}
    \item Eine Drehspiegelung um $\theta$:
        \begin{align*}
            \begin{pmatrix}
                \cos(\theta) & - \sin(\theta) & 0 \\
                \sin(\theta) & \cos(\theta) & 0 \\
                0 & 0 & -1
            \end{pmatrix}
            \ \rightarrow \ \tr(R) = 2 \cos(\theta) - 1
        \end{align*}
        \begin{align*}
            \tr(s_4) = 2 \cos \klammer{\frac{\pi}{2}} - 1 = -1
            \hspace{10pt} , \hspace{10pt}
            \tr(\tau) = 2 \cdot 1 - 1 = 1
        \end{align*}
    \item $N_1 = 5$ , $N_{r_3} = 2$ , $N_{r_2} = 1$ , $N_{s_4} = 1$ , $N_\tau = 3$
    \item Es folgt:
        \begin{table}[h]
            \centering
            \begin{tabular}{c|ccccc}
                 & $[1]$ & $[r_3]$ & $[r_2]$ & $[s_4]$ & $[\tau]$ \\ \hline
                $\chi_\rho$ & $15$ & $0$ & $-1$ & $-1$ & $3$ 
            \end{tabular}
        \end{table}
    \item Berechne Vielfachheiten:
        $n_j = \scalprod{\chi_\rho}{\chi_j} \ \Rightarrow \ $
        $n_1 = 1$ , $n_2 = 1$ , $n_3 = 0$ , $n_4 = 1$ , $n_5 = 3$
    \item Somit: $\rho \cong \rho_1 \oplus \rho_2 \oplus \rho_4 \oplus \rho_5 \oplus \rho_5 \oplus \rho_5$.
        D.h. es gibt höchstens $6$ Eigenfrequenzen (d.h. EW von $A$).
\end{enumerate}
Nicht alle EW von $A$ entsprechen Schwingungen. Manche sind $=0$ wegen der
Translationsinvarianz (T) und Drehinvarianz (D).
\begin{enumerate}[]
    \item \underline{(T)}: Entspricht $x = (\vec{a},\dots,\vec{a})$. Dies
        wird durch $\rho(R)$ abgebildet auf $(R\vec{a},\dots,R\vec{a})$.
        Entspricht Darstellung $R \mapsto R$ von (T) mit Charakter $\chi_5$.
    \item \underline{(D)}: Entspricht $x = (\vec{b} \wedge \vec{y}_1^\ast,\dots,
        \vec{b} \wedge \vec{y}_c^\ast)$ mit $b \in \R^3$. Es gilt:
        $\rho(R) x = \klammer{R(\vec{b} \wedge y_{\sigma_R^{-1} (1)}),\dots,
        R (\vec{b} \wedge y_{\sigma_R^{-1} (c)})}$ mit $\wedge$ dem Kreuzprodukt.
        $R(x \wedge y) = \det(R) (R x \wedge R y) \ \forall R \in O(3)$. Somit:
        $\rho(R) x = \det(R) \klammer{R b \wedge \vec{y}_1^\ast ,\dots, R b \wedge
        \vec{y}_c^\ast}$ entspricht der Darstellung $R \mapsto \det(R) \cdot R$
        auf $\R^3$ entspricht $\chi_4$.
\end{enumerate}
Auf dem orthogonalen Komplement von $\rho_4$ und $\rho_5$ zerlegt sich unsere
Darstellung als $\rho \cong \rho_1 \oplus \rho_2 \oplus \rho_5 \oplus \rho_5$.
Somit erhält man höchstens $4$ verschiedene Eigenfrequenzen.



\section{Die Drehgruppe und die Lorentzgruppe}

\subsection{Isometrien des Euklidischen Raums}

\begin{definition}[Euklidischer Raum]
    Der Euklidische Raum ist der VR $\R^3$ versehen mit dem Skalarprodukt
    $x \cdot y = x_1 y_1 + x_2 y_2 + x_3 y_3$.
\end{definition}

\begin{definition}[Euklidischer Abstand]
    Der Euklidische Abstand zwischen zwei Punkten $x$ und $y$ ist
    $d(x,y) = \abs{x - y}$ wobei $\abs{x} = \sqrt{x \cdot x}$.
\end{definition}

\begin{definition}[Isometrie]
    Eine Isometrie des Euklidischen Raums ist eine bijektive Abbildung
    $f: \R^3 \rightarrow \R^3$, die Abstände erhält: $d(f(x),f(y)) = d(x,y)
    \ \forall x,y$. Insbesondere sind Isometrien stetige Abbildungen.
\end{definition}

\begin{satz}
    Sei $f$ eine Isometrie des Euklidischen Raums. Dann ist $f$ von der Form
    $f(x) = R x + a$ wobei $R \in O(3)$ und $a \in \R^3$. Dies gilt in
    beliebigen Dimensionen.
\end{satz}

\subsection{Die Drehgruppe $SO(3)$}

Die Spiegelung $P: x \mapsto - x$ ($P = - \mathds{1}$) hat Determinante
$-1$ und jede Matrix in $O(3)$ ist von der Form $R$ oder $P R = - R$ für
$R \in SO(3)$. Da $P$ mit allen $O \in O(3)$ kommutiert, können wir
identifizieren: $O(3) \cong SO(3) \times \Z_2$. Jede Matrix in $SO(3)$
ist von der Form:
\begin{align*}
    O R_3 (\vartheta) O^{-1}
    \hspace{10pt} , \hspace{10pt}
    R_3 (\vartheta) = \begin{pmatrix}
        \cos(\vartheta) & - \sin(\vartheta) & 0 \\
        \sin(\vartheta) & \cos(\vartheta) & 0 \\
        0 & 0 & 1
    \end{pmatrix}
    \hspace{10pt} , \hspace{10pt}
    O \in SO(3)
\end{align*}
Die Matrix $R_3 (\vartheta)$ entspricht einer Drehung um die Achse $e_3'
= n = O e_3$ mit Winkel $\vartheta$. Der Drehwinkel $\vartheta$ wird im
Gegenuhrzeigersinn gemessen. Die Matrix $R(n,\vartheta) = O R_3 (\vartheta)
O^{-1}$ wird Drehung um $n$ mit Winkel $\vartheta$ genannt. Es gilt
$R(-n,\vartheta) = R(n,2 \pi - \vartheta)$.

\begin{lemma}
    Sei $O \in SO(3)$ und $n = O e_3$. Dann ist
    \begin{align*}
        R(n,\vartheta) x &= O R_3 (\vartheta) O^{-1} x
        \\
        &= (x \cdot n) n + \eckigeklammer{x - (x \cdot n) n} \cos(\vartheta)
            + n \wedge x \ \sin(\vartheta)
    \end{align*}
\end{lemma}

\begin{lemma}
    \begin{enumerate}[(i)]
        \item $R(n,\vartheta) = R(-n,-\vartheta) = R(n,-\vartheta)^{-1}$
        \item $R(n_1,\vartheta_1)R(n_2,\vartheta_2) = R(n_2',\vartheta_2) R(n_1,\vartheta_1)$,
            wobei $n_2' = R(n_1,\vartheta_1) n_2$. 
    \end{enumerate}
\end{lemma}

\subsection{Die Eulerwinkel}

\begin{definition}
    $R_j(\alpha) = R(e_j , \alpha)$
\end{definition}

\begin{satz}
    $R_1(\vartheta)$ und $R_3(\vartheta)$ erzeugen die Gruppe $SO(3)$.
\end{satz}

\begin{satz}
    Jedes $A \in SO(3)$ lässt sich schreiben als
    \begin{align*}
        A = R_3 (\varphi) R_1 (\vartheta) R_3 (\psi)
    \end{align*}
    mit $\varphi \in [0,2 \pi[ \ , \ \varphi \in [0,\pi] \ , \ \psi \in
    [0,2 \pi[$. Die Winkel $\varphi,\vartheta,\psi$ heissen Eulerwinkel.
    Es gilt: $\vartheta = \angle (e_3,e_3')$, $\varphi = \angle(e_1,e)$,
    $\psi = \angle (e,e_1')$. Hierbei ist $e$ ein Einheitsvektor längs der
    Geraden, welche durch Schneiden der durch $e_1$,$e_2$ aufgespannte Ebene
    und der durch $e_1'$,$e_2'$ aufgespannten Ebene entsteht. Mit $e_i' =
    A e_i$.
\end{satz}

\subsection{Der Homomorphismus $SU(2) \rightarrow SO(3)$}\label{Hom_SU2_SO3}

Die Gruppe $SU(2)$ kann geometrisch als dreidimensionale Sphäre $S^3$ aufgefasst
werden.

\begin{lemma}
    Jede Matrix $A \in SU(2)$ ist von der Form
    \begin{align*}
        A = \begin{pmatrix}
            \alpha & \beta \\ - \overline{\beta} & \overline{\alpha}
        \end{pmatrix}
        \hspace{10pt} , \hspace{10pt}
        \alpha,\beta \in \C
        \hspace{10pt} , \hspace{10pt}
        \abs{\alpha}^2 + \abs{\beta}^2 = 1
    \end{align*}
\end{lemma}

\begin{definition}[$H_0$]
    $H_0$ ist der reelle Vektorraum aller hermitischen spurfreien $2 \times 2$
    Matrizen. Also:
    \begin{align*}
        H_0 = \geschwungeneklammer{
            \begin{pmatrix}
                z & x - i y \\ x + i y & -z
            \end{pmatrix}
            \ \Big| \ x,y,z \in \R
        }
    \end{align*}
    Es ist $\dim(H_0) = 3$. Für $X,Y \in H_0$ definieren wir das Skalarprodukt
    $(X,Y) = \frac{1}{2} \tr(XY)$. Für $A \in SU(2)$ definiere die lineare
    Abbildung $\phi(A) : H_0 \rightarrow H_0$ durch
    \begin{align*}
        \phi(A) X = A X A^\ast = A X A^{-1}
    \end{align*}
    $\phi(A)X$ ist hermitesch und spurfrei.
\end{definition}

\begin{satz}
    \begin{enumerate}[(i)]
        \item $\phi(AB) = \phi(A) \phi(B)$ , $A,B \in SU(2)$
        \item $\klammer{\phi(A)X,\phi(A)Y} = \klammer{X,Y}$ , $A \in SU(2)$, $X,Y \in H_0$
    \end{enumerate}
\end{satz}

Wir betrachten die ONB von $H_0$ gegeben durch die Pauli-Matrizen. Mittels
dieser Basis identifizieren wir $\klammer{\R^3 , (\cdot,\cdot)_{std}}
\stackrel{\cong}{\longrightarrow} \klammer{H_0,(\cdot,\cdot)}$
\begin{align*}
    x = (x_1,x_2,x_3)^T \mapsto \hat{x} = \sum_{i=1}^3 x_i \sigma_i
\end{align*}
wobei $\sigma_i$ die Pauli-Matrizen sind:
\begin{align*}
    \sigma_1 = \begin{pmatrix}
        0 & 1 \\ 1 & 0
    \end{pmatrix}
    \hspace{10pt} , \hspace{10pt}
    \sigma_2 = \begin{pmatrix}
        0 & -i \\ i & 0
    \end{pmatrix}
    \hspace{10pt} , \hspace{10pt}
    \sigma_3 = \begin{pmatrix}
        1 & 0 \\ 0 & -1
    \end{pmatrix}
\end{align*}
Diese Matrizen sind eine ONB, denn $\tr(\sigma_i \sigma_j) = 2 \delta_{ij}$.

\vspace{1\baselineskip}

Wir bezeichnen ebenfalls mit $\phi(A) \in O(3)$ die Matrix von $\phi(A)$ in
der ONB $\sigma_1,\sigma_2,\sigma_3$. Da $SU(2)$ zusammenhängend ist (Jede
Matrix $A \in SU(2)$ ist von der Form $A_1$ mit $A_t = B \begin{pmatrix}
    e^{i t \theta} & 0 \\ 0 & e^{-i t \theta}
\end{pmatrix} B^{-1}$, $B \in SU(2)$, und der Weg $t \mapsto A_t$ verbindet
$\mathds{1}$ mit $A$) und $\phi$ stetig ist, folgt $\det(\phi(A)) = 1 \
\forall A \in SU(2)$, also definiert $\phi$ eine Homomorphismus
$\phi: SU(2) \rightarrow SO(3)$.

\begin{satz}
    $\phi: SU(2) \rightarrow SO(3)$ ist surjektiv mit Kern
    $\geschwungeneklammer{\pm \mathds{1}}$. Also ist
    \begin{align*}
        SU(2) / \geschwungeneklammer{\pm \mathds{1}} \cong SO(3)
    \end{align*}
\end{satz}

\begin{bemerkung}
    Es gilt:
    \begin{align*}
        R(n,\theta) = \phi(\mathds{1} \cos(\theta/2) - i \hat{n} \sin(\theta/2))
        \hspace{10pt} , \hspace{10pt}
        n \in \R^3
        \hspace{10pt} , \hspace{10pt}
        \abs{n} = 1
    \end{align*}
\end{bemerkung}

\subsection{Der Minkowski-Raum}

Der Minkowski-Raum (auch Raumzeit) ist $\R^4$ versehen mit der symmetrischen
nicht degenerierten Bilinearform
\begin{align*}
    (x,y) = x^0 y^0 - x^1 y^1 - x^2 y^2 - x^3 y^3
    \hspace{10pt} , \hspace{10pt} x,y \in \R^4
\end{align*}

Ein Vektor $x \in \R^4$ heisst zeitartig falls $(x,x) > 0$, raumartig falls
$(x,x) < 0$ und lichtartig falls $(x,x) = 0$. Die Menge der lichtartigen
Vektoren heisst Lichtkegel $K$.


\subsection{Die Lorentzgruppe}

Die Lorentzgruppe $O(1,3)$ ist die Gruppe aller linearen Transformationen von
$\R^4$ die die Minkowskimetrik erhalten:
\begin{align*}
    O(1,3) = \geschwungeneklammer{A \in \GL(4,R) \ | \ (A x , A y) = (x,y)
    \hspace{5pt} , \hspace{5pt} \forall x,y \in \R^4}
\end{align*}

Äquivalent ist
\begin{align*}
    O(1,3) = \geschwungeneklammer{A \in \GL(4,\R) \ | \ A^T g A = g}
    \hspace{5pt} \text{mit} \hspace{5pt}
    g = \begin{pmatrix}
        1 & 0 & 0 & 0 \\
        0 & -1 & 0 & 0 \\
        0 & 0 & -1 & 0 \\
        0 & 0 & 0 & -1
    \end{pmatrix}
\end{align*}

Also insbesondere $\det(A) = \pm 1 \ \forall A \in O(1,3)$.

\begin{definition}
    Eine Basis $b_0,\dots,b_3$ von $\R^4$ heisst orthonormiert (bzgl der
    Minkowskimetrik) falls $(b_i,b_j) = g_{ij}$ für alle $i,j = 0,\dots,3$.
\end{definition}

\begin{satz}
    Sind $(b_i)_{i=0}^3$, $(b_i')_{i=0}^3$ zwei orthonormierte Basen vom
    Minkowskiraum $\R^4$, so existiert genau eine Lorentztransformation 
    $A$, so dass $b_j' = A b_j$.
\end{satz}

\begin{korollar}
    Eine $4 \times 4$ Matrix ist genau dann in $O(1,3)$ wenn ihre Spalten
    bzgl der Minkowskimetrik orthonormiert sind.
\end{korollar}

\subsection{Beispiele von Lorentztransformationen}

\paragraph{(a) Orthogonale Transformationen von $\R^3$}
Ist $R \in O(3)$ eine orthogonale Transformation, so ist die $4 \times 4$
Matrix
\begin{align*}
    R := \begin{pmatrix}
        1 & 0 & 0 & 0 \\
        0 & & & \\
        0 & & R & \\
        0 & & & 
    \end{pmatrix}
\end{align*}
eine Lorentztransformation. Wir können also $O(3)$ als Untergruppe von $O(1,3)$
auffassen.

\paragraph{(b) Lorentzboost}
Der Lorentzboost in der $3$-Richtung mit Rapidität $\chi \in \R$ ist die
Lorentztransformation
\begin{align*}
    L(\chi) = \begin{pmatrix}
        \cosh(\chi) & 0 & 0 & \sinh(\chi) \\
        0 & 1 & 0 & 0 \\
        0 & 0 & 1 & \\
        \sinh(\chi) & 0 & 0 & \cosh(\chi)
    \end{pmatrix}
    \hspace{5pt} \in O(1,3)
\end{align*}
Da $\cosh(\chi)^2 - \sinh(\chi)^2 = 1$, sind die Spalten orthonormiert. Weiter
gilt $L(\chi_1) L(\chi_2) = L(\chi_1 + \chi_2)$. Also bilden diese Matrizen eine
zu $\R$ isomorphe Untergruppe.

\paragraph{(c) Diskrete Lorentztransformationen}
Die Lorentztransformationen $P$ ("Raumspiegelung") und $T$ ("Zeitumkehr")
\begin{align*}
    P = \begin{pmatrix}
        1 & 0 & 0 & 0 \\
        0 & -1 & 0 & 0 \\
        0 & 0 & -1 & 0 \\
        0 & 0 & 0 & -1
    \end{pmatrix} = g
    \hspace{10pt} , \hspace{10pt}
    T = \begin{pmatrix}
        -1 & 0 & 0 & 0 \\
        0 & 1 & 0 & 0 \\
        0 & 0 & 1 & 0 \\
        0 & 0 & 0 & 1
    \end{pmatrix}
    = - g
\end{align*}
bilden mit $1$ und $PT$ eine abelsche Untergruppe der Ordnung $4$.

\begin{lemma}
    Für alle Lorentztransformationen $A$ gilt:
    \begin{align*}
        A^T = P A^{-1} P = T A^{-1} T = g A^{-1} g
    \end{align*}
\end{lemma}

\begin{korollar}
    Eine $4 \times 4$ Matrix ist genau dann in $O(1,3)$ wenn ihre Zeilen
    bezüglich der Minkowskimetrik orthonormiert sind.
\end{korollar}


\subsection{Strukturen der Lorentzgruppe}

\begin{definition}[$O_+ (1,3)$]
    Sei $O_+ (1,3) = \geschwungeneklammer{A \in O(1,3) \ | \ A_{00} > 0}$.
    Solche Transformationen heissten orthochron, d.h. zeitrichtungerhaltend.
\end{definition}

\begin{definition}[$Z_+$]
    Sei $Z_+ \subset \R^4$ die Menge der Zeitartigen Vektoren $x$ mit $x^0 > 0$.
\end{definition}

\begin{satz}
    $O_+ (1,3)$ ist eine Untergruppe von $O(1,3)$. Sie besteht aus den
    Lorentztransformationen die $Z_+$ nach $Z_+$ abbilden.
\end{satz}

\begin{definition}[$SO_+ (1,3)$]
    Die orthocrhone spezielle Lorentzgruppe $SO_+(1,3)$ ist die Gruppe der
    orthochronen Lorentztransformationen mit Determinante $1$.
    \begin{align*}
        SO_+ (1,3) := \geschwungeneklammer{A \in O_+ (1,3) \ | \ \det(A) = 1}
    \end{align*}
    Insbesondere ist $SO(3) \subset SO_+ (1,3)$.
\end{definition}

\begin{satz}
    Jede Lorentztransformation liegt in genau einer der folgenden Klassen:
    $SO_+ (1,3)$ , $\geschwungeneklammer{P X \ | \ X \in SO_+ (1,3)}$ ,
    $\geschwungeneklammer{T X \ | \ X \in SO_+ (1,3)}$ oder
    $\geschwungeneklammer{P T X \ | \ X \in SO_+ (1,3)}$.
\end{satz}

Wenn $X \in SO_+ (1,3)$, dann ist

\begin{table}[h]
    \centering
    \begin{tabular}{c|cc}
         & $\det = 1$ & $\det = -1$ \\ \hline
        $A_{00} > 0$ & $X$ & $PX$ \\
        $A_{00} < 0$ & $PTX$ & $TX$
    \end{tabular}
\end{table}

\begin{lemma}
    Jede orthochrone spezielle Lorentztransformation ist von der Form
    $R_1 L(\chi) R_2$, mit $\chi \in \R$ und $R_1 , R_2 \in SO(3)$.
\end{lemma}

\begin{bemerkung}
    Es folgt, dass $SO_+ (1,3)$ zusammenhängend ist, da die stetige Abbildung
    \begin{align*}
        SO(3) \times \R \times SO(3) &\rightarrow SO_+ (3)
        \\
        (R_1,\chi,R_2) &\mapsto R_1 L(\chi) R_2
    \end{align*}
    surjektiv ist, und die linke Seite zusammenhängend. Das heisst $O(1,3)$
    hat also die $4$ Zusammenhangskomponenten $SO_+ (1,3)$ , $P SO_+ (1,3)$
    , $T SO_+ (1,3)$ , $P T SO_+ (1,3)$.
\end{bemerkung}


\subsection{Intertiale Bezugssysteme}

In der speziellen Relativitätstheorie heisst eine orthonormierte Basis $(b_i)$
ein (inertiales) Bezugssystem. Ein Punkt $x$ im Minkowskiraum heisst Ereignis.
Die Koordinaten von $x$ im Bezugssystem $(b_i)$ sind $x = \sum x^i b_i$
gegeben. Ein Punktteilchen wird in einem Bezugssystem durch eine Bahn (auch
Weltlinie genannt) $\vec{x}(t)$ beschrieben, die die Raumkoordinaten als
Funktion der Zeit angibt.
\begin{align*}
    x^0 = c t \hspace{10pt} , \hspace{10pt} \vec{x} = \vec{x} (t)
    \hspace{10pt} , \hspace{10pt} t \in \R
\end{align*}
Für Teilchen mit $v<c$ ist die Weltlinie eine Kurve im Minkowskiraum, deren
Tangentialvektor $dx/dt$ stets zeitartig ist. Ist $(b_i')$ ein zweites
Bezugssystem und $\Lambda \in O(1,3)$ mit $b_i = \Lambda b_i' = \sum_j
\Lambda_{ji} b_j$, so werden die Koordinaten $x'^{i}$ eines Ereignis im
Bezugssystem $(b_i)$ durch die Lorentztransformation $\Lambda$ gegeben:

\begin{align*}
    x'^{i} = \sum_j = \Lambda_{ij} x^j
\end{align*}

\subsection{Der Isomorphismus $SL(2,\C) / \geschwungeneklammer{\pm 1} \rightarrow SO_+ (1,3)$}

\begin{definition}[$H$]
    Der vierdimensionale Raum $H$ ist der Raum aller hermitischen
    $2 \times 2$ Matrizen. Diese haben die Form
    \begin{align*}
        \hat{x} = \begin{pmatrix}
            x^0 + x^3 & x^1 - i x^2 \\
            x^1 + i x^2 & x^0 - x^3
        \end{pmatrix}
        = x^0 \mathds{1} + \sum_{j=1}^3 x^j \sigma_j
    \end{align*}
    mit $x \in \R^4$ und $\sigma_i$ den Pauli Matrizen.
\end{definition}

\begin{lemma}
    Für alle $x \in \R^4$ gilt $(x,x) = \det(\hat{x})$
\end{lemma}

\begin{satz}
    Für jede Matrix $A \in \text{SL}(2,\C)$ definieren wir die lineare Abbildung
    von $H$ nach $H$: $X \mapsto A X A^\ast$. Also gibt es eine lineare Abbildung
    $\phi(A)$ von $\R^4$ nach $\R^4$, so dass
    \begin{align*}
        A \hat{x} A^\ast &= \widehat{\phi(A) x}
    \end{align*}
    Es gilt: $\det(A X A^\ast) = \det(A) \det(X) \det(A^\ast) = \det(x)
    \abs{\det(A)}^2 = \det(X)$ für $A \in SL(2,\C)$. Es folgt, dass
    $\phi(A) \in O(1,3)$.
\end{satz}

\begin{satz}
    Die Abbildung $\phi$ ist ein surjektiver Homomorphismus von $SL(2,\C)$
    nach $SO_+ (1,3)$ mit Kern $\geschwungeneklammer{\pm 1}$. Also induziert
    $\phi$ einen Isomorphismus $SL(2,\C) / \geschwungeneklammer{\pm 1}
    \rightarrow SO_+ (1,3)$. Die Einschränkung von $\phi$ auf $SU(2) \subset
    SL(2,\C)$ ist der Homomorphismus $SU(2) \rightarrow SO(3)$.
\end{satz}


\section{Lie-Algebren}

\subsection{Die Exponentialabbildung}

Sei $\Mat(n,\K)$, für $\K = \R$ oder $\C$, der Vektorraum aller $n \times n$
Matrizen mit Elementen in $\K$.

\begin{definition}[Frobeniusnorm]
    \begin{align*}
        \Norm{x} = \klammer{\sum_{i,j} \abs{x_{ij}}^2}^{1/2}
        = \klammer{\tr \klammer{X^\ast X}}^{1/2}
    \end{align*}
\end{definition}

\begin{lemma}
    $\Norm{X Y} \leq \Norm{X} \Norm{Y} \ , \ \ \forall X,Y \in \Mat(n,\K)$
\end{lemma}

\begin{lemma}
    Die folgende Reihe konvergiert absolut (normal)
    \begin{align*}
        \exp(X) = \sum_{k=0}^\infty \frac{1}{k!} X^k
    \end{align*}
    $\forall X \in \Mat(n,\K)$.
    Normal heisst hier $\sum_{k=0}^\infty \Norm{\frac{1}{k!} X^k} < \infty$.
\end{lemma}

\begin{bemerkung}
    Es folgt, dass die Matrixelemente von $\exp(X)$ absolut konvergente Reihen
    in den Matrixelementen $X_{ij}$ von $X$ sind, und somit analytisch von
    $X_{ij}$ abhängen.
\end{bemerkung}

\begin{lemma}
    Seien $X,Y \in \Mat(n,\K)$ für $\K = \R$ oder $\C$.
    \begin{enumerate}[(i)]
        \item $\exp(X) \exp(Y) = \exp(X+Y)$ falls $XY = YX$
        \item $\exp(X)$ ist invertierbar mit $\exp(X)^{-1} = \exp(-X)$
        \item $A \exp(X) A^{-1} = \exp(A X A^{-1})$, $A \in \GL(n,\K)$
        \item $\det(\exp(X)) = \exp(\tr(X))$
        \item $\exp(X^\ast) = \klammer{\exp(X)}^\ast$ , $\exp(X^T) = (\exp(X))^T$
    \end{enumerate}
\end{lemma}

\begin{definition}[Exponentialabbildung]
    Die Abbildung $\Mat(n,\K) \rightarrow \GL(n,\K)$, $X \mapsto \exp(X)$
    heisst Exponentialabbildung.
\end{definition}

\begin{bemerkung}
    Für Nilpotente Matrizen $N$ ($N^{k+1} = 0$) gilt:
    \begin{align*}
        \exp(N) = \mathds{1} + N + \frac{N^2}{2!} + \dotsb + \frac{N^k}{k!}
    \end{align*}
\end{bemerkung}

\begin{lemma}
    Die Abbildung $\exp: \Mat(n,\K) \rightarrow \GL(n,\K)$ ist in einer
    Umgebung von $0$ invertierbar, d.h. es existiert eine Umgebung $U$ von
    $0$ so dass die Abbildung $\exp: U \mapsto \exp(U)$ invertierbar ist. Die
    inverse Abbildung ist durch die folgende absolut konvergente Potenzreihe
    gegeben für $\Norm{X - \mathds{1}} < 1$.
    \begin{align*}
        \log(X) = \sum_{n=1}^\infty (-1)^{n+1} \frac{(X - \mathds{1})^n}{n}
    \end{align*}
\end{lemma}

\subsection{Einparametergruppen}

\begin{definition}[Einparametergruppe]
    Eine Abbildung $\R \rightarrow \GL(n,\K)$, $t \mapsto X(t)$, $\K = \R$ oder
    $\C$, heisst Einparametergruppe falls sie stetig differenzierbar ist und
    ein Gruppenhomomorphismus ist, d.h. $X(0) = \mathds{1}$ und für alle
    $t,s \in \R$ gilt: $X(s+t) = X(s) X(t)$.
\end{definition}

\begin{bemerkung}
    Das Bild einer solchen Abbildung ist eine Untergruppe mit $X(t)^{-1} = X(-t)$.
\end{bemerkung}

\begin{satz}
    \begin{enumerate}[(i)]
        \item $\forall X \in \Mat(n,\K)$ ist $t \mapsto \exp(t X)$ eine
            Einparametergruppe.
        \item Alle Einparametergruppen sind von dieser Form.
    \end{enumerate}
\end{satz}

\subsection{Matrix-Lie-Gruppen}

\begin{definition}[Lie-Algebra/Lie-Gruppe]
    Sei $G \subset \GL(n,\K)$ eine abgeschlossene Untergruppe von $\GL(n,\K)$
    (abgeschlossen heisst: Für jede Folge $(g_j)$ in $G$, die in $\GL(n,\K)$
    konvergiert, liegt der Grenzwert $\limes{j \rightarrow \infty} g_j$ auch
    in $G$). Wir definieren
    \begin{align*}
        \Lie(G)= \geschwungeneklammer{X \in \Mat(n,\K) \ | \ \exp(t X) \in G \ \forall t \in \R}
    \end{align*}
    $\Lie(G)$ heisst Lie-Algebra der Lie-Gruppe $G$.
\end{definition}

\begin{definition}[(Matrix-)Lie-Gruppe]
    Eine (Matrix-)Lie-Gruppe ist eine abgeschlossene Untergruppe von $\GL(n,\K)$.
\end{definition}

\begin{satz}
    Sei $G$ eine abgeschlossene Untergruppe von $\GL(n,\K)$, $\K = \R$ oder $\C$.
    Dann ist $\Lie(G)$ ein reeller VR, und $\forall X,Y \in \Lie(G)$,
    $X Y - Y X \in \Lie(G)$.
\end{satz}

\begin{lemma}
    $\Lie(G)$ besteht aus allen Tangentialvektoren $\dot{X}(0) = \frac{d}{dt}
    X(t) |_{t=0}$ von glatten Kurven $]-\epsilon,\epsilon[ \rightarrow G$ mit
    $X(0) = \mathds{1}$ und $\epsilon > 0$. Also ist $\Lie(G) = T_{\mathds{1}} G$
    der Tangentialraum an der Stelle $\mathds{1}$. Also insb. ein VR.
\end{lemma}

\begin{bemerkung}
    Die Gruppen $(S)U(n,m)$, $(S)O(n,m)$, $\GL(n,\K)$, $SL(n,\K)$, $Sp(2n)$
    sind alle Matrix-Lie-Gruppen. Sie werden nämlich als Mengen von gemeinsamen
    Nullstellen von stetigen Funktionen $f: \GL(n,\K) \rightarrow \K$ definiert.
\end{bemerkung}

\begin{definition}[Kommutator]
    Für $X,Y \in \Mat(n,\K)$ definieren wir den Kommutator als
    \begin{align*}
        [X,Y] = XY - YX
    \end{align*}
\end{definition}

\begin{lemma}
    Eigenschaften des Kommutators sind:
    \begin{enumerate}[(i)]
        \item $[\lambda X + \mu Y,Z] = \lambda[X,Z] + \mu[Y,Z]$
        \item $[X,Y] = - [Y,Z]$
        \item $[[X,Y],Z] + [[Z,X],Y] + [[Y,Z],X] = 0$
    \end{enumerate}
\end{lemma}

\begin{definition}[Lie-Algebra]
    Eine Lie-Algebra ist ein $\K$-VR $\g$, versehen mit einer bilinearen
    Abbildung ("Lie-Klammer") $[ \ , \ ]: \g \times \g \rightarrow \g$,
    welche die obigen Eigenschaften (i)-(iii) erfüllt.
\end{definition}

\begin{definition}[Homomorphismus]
    Ein Homomorphismus $\varphi: \g_1 \rightarrow \g_2$ von Lie-Algebren $\g_1,
    \g_2$ ist eine lineare Abbildung, die erfüllt:
    \begin{align*}
        \varphi \klammer{[X,Y]} = [\varphi(X),\varphi(Y)]
    \end{align*}
    Ist $\varphi$ bijektiv, so nennt man $\varphi$ einen Isomorphismus.
\end{definition}

\begin{beispiel}
    $\Lie(GL(n,\K)) = \Mat(n,\K)$ als reeller VR betrachtet. Diese
    Lie-Algebra wird mit $\gl(n,\K)$ bezeichnet. Eine Basis von $\gl(n,\R)$
    ist durch die matrizen $E_{ij}$, $i,j=1,\dots,n$ mit Matrixelementen
    $(E_{ij})_{kl} = \delta_{ik} \delta_{jl}$. Die Lie-Algebra Struktur
    ist in dieser Basis durch die Kommutationsrelationen
    \begin{align*}
        [E_{ij},E_{kl}] = E_{il} \delta_{jk} - E_{jk} \delta_{il}
    \end{align*}
    gegeben. Die Dimension ist $n^2$. In $\gl(n,\C)$ hat man die Basis
    $\klammer{E_{kl},i E_{kl}}_{k,l=1}^n$. $\dim(\gl(n,\C)) = 2 n^2$.
\end{beispiel}

\begin{lemma}
    \begin{align*}
        \u(n) &:= \Lie(U(n)) = \geschwungeneklammer{X \in \Mat(n,\C) \ | \ X^\ast = -X}
        \\
        \sl(n,\C) &:= \Lie(SL(n,\C)) = \geschwungeneklammer{A \in \Mat(n,\C) \ | \ \tr(A) = 0}
        \\
        \su(n) &:= \Lie(SU(n)) = \geschwungeneklammer{X \in \Mat(n,\C) \ | \ X^\ast = -X \ , \ \tr(X) = 0}
        \\
        &= \geschwungeneklammer{A \in \sl(n,\C) \ | \ A^\ast = - A}
    \end{align*}
    Es gilt: $\dim(\u(n)) = n^2$ , $\dim(\su(n)) = n^2 - 1$.
\end{lemma}

\begin{lemma}
    \begin{align*}
        &\o(n) := \Lie(O(n))
        \hspace{10pt} , \hspace{10pt}
        \so(n) := \Lie(SO(n))
        \\
        &\o(n) = \so(n) = \geschwungeneklammer{X \in \Mat(n,\R) \ | \ X^T = - X}
    \end{align*}
\end{lemma}

\begin{beispiel}[$\su(2)$]
    Eine Basis ist durch die Pauli Matrizen gegeben.
    \begin{align*}
        t_1 = i \sigma_1
        \hspace{10pt} , \hspace{10pt}
        t_2 = i \sigma_2
        \hspace{10pt} , \hspace{10pt}
        t_3 = i \sigma_3
    \end{align*}
    Es gilt $[t_j,t_k] = - \sum_{l=1}^3 2 \epsilon_{jkl} t_l$
\end{beispiel}

\subsection{Die Campbell-Baker-Hausdorff Formel}

\begin{satz}[CBH]
    Seien $X,Y \in \Mat(n,\K)$. Für $t$ klein genug gilt
    \begin{align*}
        \exp(tX) \exp(tY) = \exp \klammer{t X + t Y + \frac{t^2}{2} [X,Y] + O(t^3)}
    \end{align*}
\end{satz}

\begin{satz}[CBH vollständig]
    Für kleine $t$ gilt
    \begin{align*}
        \exp(tX) \exp(tY) = \exp \klammer{\sum_{k=1}^\infty t^k Z_k}
    \end{align*}
    wobei $Z_k$ eine Linearkombination von $k$-fachen Kommutatoren ist, d.h.
    von Ausdrücken, die aus $X$ und $Y$ durch $(k-1)$-fache Anwendung der
    Operatoren $[X,\cdot],[Y,\cdot]$ erzeugt werden. Bsp:
    \begin{align*}
        Z_1 = X + Y
        \hspace{10pt} &, \hspace{10pt}
        \frac{1}{2} [X,Y]
        \\
        Z_3 = \frac{1}{12} \klammer{[X,[X,Y]] + [Y,[y,X]]}
        \hspace{10pt} &, \hspace{10pt}
        Z_4 = - \frac{1}{24} [X,[Y,[X,Y]]]
    \end{align*}
\end{satz}

\begin{definition}[Unter-/Teilmannigfaltigkeit]
    Folgende Aussagen sind äquivalent:
    \begin{enumerate}[(i)]
        \item $M \subseteq \R^n$ ist eine $k$-dim Unter-/Teilmannigfaltigkeit.
        \item $\forall p \in M \ \exists U_p \subseteq \R^n$ offene Umgebung von $p$
            und ein Diffeomorphismus $\Phi_p : U_p \rightarrow V_p \subseteq \R^n$ offen
            sodass $\Phi_p (U_p \cap M) = V_p \cap \R^k$
        \item $\forall p \in M \ \exists U_p \in \R^n$ (offene Umgebung von $p$)
            und $f_p : \R^k \supseteq \tilde{U}_p \rightarrow \R^{n-k}$ glatt und
            $\sigma \in S_n$ sodass $M \cap U_p = \sigma \text{Graph}(f_p)$.
        \item $\forall p \in M \ \exists U_p \subseteq \R^n$ (offene Umgebung von $p$)
            und $\varphi_p : \R^k \supseteq \tilde{U}_p \rightarrow \R^n$ glatte
            Einbettung mit Bild $V = U_p \cap G$
    \end{enumerate}
    $\tilde{U}_p$ ist offen.
    Eine Glatte Einbettung ist eine glatte Abbildung mit
    \begin{itemize}
        \item $d \varphi_p$ hat in jedem Punkt maximal Rang $k$.
        \item $\varphi_p$ ist ein Homöomorphismus auf ihr Bild.
    \end{itemize}
\end{definition}

\begin{satz}
    Sei $G \subseteq \GL(n,\K)$ eine Matrix-Lie-Gruppe mit Lie-Algebra
    $\mathfrak{g}$ und Exponentialabbildung $\exp: \mathfrak{g} \rightarrow G$.
    Dann gibt es eine offene Umgebung $U \subseteq \mathfrak{g}$ von $0$ und
    eine offene Umgebung $V \subseteq G$ von $\mathds{1}$ so dass
    $\exp: U \rightarrow \GL(n,\K)$ eine glatte Einbettung ist mit Bild
    $\exp(U) = V \cap G$ 
\end{satz}

\begin{satz}
    $G \subseteq \GL(n,\K) \subseteq \R^{n^2}$ ist eine Untermannigfaltigkeit.
\end{satz}

\begin{satz}
    Sei $G \subseteq \GL(n,\K)$ eine Lie-Gruppe mit Lie-Algebra $\mathfrak{g}
    \subseteq \gl(n,\K)$. Die Gruppe aller Matrizen der Form
    $\exp(X_1) \dotsb \exp(X_k)$ mit $X_1,\dots,X_k \in \mathfrak{g}$ und
    $k \geq 1$ ist die Zusammenhangskomponente von $\mathds{1} \in G$.
\end{satz}

\begin{beispiel}
    Jede unitäre Matrix ist von der Form
    \begin{align*}
        U = A \text{diag} (e^{i \varphi_1},\dots,e^{i \varphi_n}) A^{-1}
        \hspace{10pt} , \hspace{10pt} A \in U(n)
    \end{align*}
    Also ist $U = \exp(X)$, $X = A \text{diag}(i \varphi_1,\dots,i \varphi_n) A^{-1}$
    und $\exp(t X)$ ist eine Einparametergruppe in $U(n)$. Es folgt, dass
    $\exp: \u(n) \rightarrow U(n)$ surjektiv ist.
\end{beispiel}


\section{Darstellungen von Lie-Gruppen}

\subsection{Definitionen}

\begin{definition}[Darstellung einer Lie-Gruppe]
    Eine Darstellung einer Lie-Gruppe $G$ auf einem ($\R$ oder $\C$) endlichdimensionalen
    VR $V \neq 0$ ist ein stetiger Homomorphismus $\rho: G \rightarrow \GL(V)$.
    Stetigkeit bedeutet, dass die Matrixelemente von $\rho(g)$ bezüglich einer
    beliebigen Basis stetig von $g \in G$ abhängen.
\end{definition}

\begin{definition}[komplex/reell]
    Eine Darstellung heisst komplex oder reell wenn $V$ ein komplexer bzw.
    reeller VR ist.
\end{definition}

\begin{definition}[Dimension]
    Die Dimension einer Darstellung ist die Dimension des Darstellungsraums $V$.
\end{definition}

Wenn nichts anderes gesagt, betrachten wir komplexe Darstellungen.


\subsection{Beispiele}

Die Gruppe $U(1) = \geschwungeneklammer{z \in \C \ | \ \abs{z} = 1} \cong SO(2)
\cong S^1$ ist eine kompakte abelsche Lie-Gruppe. Also ist jede Darstellung
vollständig reduzibel. Die irreduziblen Darstellungen sind eindimensional.

\begin{satz}
    Für jedes $n \in \Z$ ist $\rho_n: U(1) \rightarrow \GL(\C) \backslash \geschwungeneklammer{0}$,
    $z \mapsto z^n$ eine Darstellung von $U(1)$. Jede irreduzible Darstellung
    von $U(1)$ ist äquivalent zu $\rho_n$ für geeignetes $n$.
\end{satz}

Die Gruppe $SU(2)$ der unitären $2 \times 2$ Matrizen der Determinante $1$ ist
ebenfalls kompakt aber nicht abelsch. Wir haben also wiederum vollständige
Reduzibilität. $SU(2)$ ist eine kompakte Lie-Gruppe.

\begin{satz}
    Für jedes $n=0,1,2,\dots$ existiert eine irreduzible Darstellung $\rho_n :
    SU(2) \rightarrow \GL(V_n)$ der Dimension $n+1$. Jede irreduzible Darstellung
    von $SU(2)$ ist äquivalent zu $\rho_n$ für geeignetes $n$.
\end{satz}

Wir konstruieren nun diese Darstellungen. Sei $\C[z_1,\dots,z_n]_k$ der VR
der Polynome in $z_1,\dots,z_n$ die homogen sind vom Grad $k$. Das heisst
$p(\lambda z_1,\dots,\lambda z_n) = \lambda^k p(z_1,\dots,z_n)$. Die Basis
ist gegeben durch $z^\alpha$ mit $\alpha = (\alpha_1,\dots,\alpha_2)$ und
$\abs{\alpha} = n$. Wir setzen $V_n = \C[z_1,z_2]_n = \text{span} \geschwungeneklammer{
z_1^n , z_1^{n-1} z_2,\dots,z_2^n}$. Sei ferner $\rho_n : SU(2) \rightarrow
\GL(V_n)$ definiert durch $\klammer{\rho_n (A)(p)}(z) = p\klammer{A^{-1} z}$.
Hierbei ist $A \in SU(2)$, $p \in V_n$ und $z=(z_1,z_2)^T$. Es ist klar, dass
die rechte Seite wieder ein Polynom in $z_1,z_2$ ist homogen vom Grad $n$.
Die Darstellungseigenschaft ist erfüllt:
\begin{align*}
    \klammer{\rho_n(A) \rho_n(B) p}(z) &=
    \klammer{\rho_n(B) p} \klammer{A^{-1} z}
    = p \klammer{B^{-1} A^{-1} z}
    \\
    &= p \klammer{\klammer{A B}^{-1} z}
    = \klammer{\rho_n(A B) p}(z)
\end{align*}
Stetigkeit:
\begin{align*}
    A &= \begin{pmatrix}
        a & b \\ c & d
    \end{pmatrix}
    \hspace{10pt} , \hspace{10pt}
    \det(A) = 1
    \hspace{10pt} , \hspace{10pt}
    A^{-1} = A^\ast = \begin{pmatrix}
        \overline{a} & \overline{c} \\ \overline{b} & \overline{d}
    \end{pmatrix}
    \\
    &\Rightarrow p(A^{-1} z) = p \klammer{A^{-1} \begin{pmatrix}
        z_1 \\ z_2
    \end{pmatrix}} = p \klammer{\begin{pmatrix}
        \overline{a} z_1 + \overline{c} z_2 \\
        \overline{b} z_1 + \overline{d} z_2
    \end{pmatrix}}
\end{align*}
Die Koeffizienten sind Polynome in $\overline{a},\overline{b},\overline{c},
\overline{d}$ und damit stetige Funktionen von $a,b,c,d$.

\begin{bemerkung}
    Die Darstellung $\rho_{2j}$ ($j=0,1/2,1,3/2,\dots$) heisst Spin $j$
    Darstellung in der phsyikalischen Literatur.
\end{bemerkung}

\begin{bemerkung}
    Es gilt: $\rho_n (-A) = (-1)^n \rho_n (A)$. Also definiert für $n$
    gerade, und nur dann, $\rho_n$ eine Darstellung von $SU(2) \backslash
    \geschwungeneklammer{\pm \mathds{1}} \cong SO(3)$.
\end{bemerkung}

\subsection{Darstellungen von Lie-Algebren}

\begin{lemma}
    Sei $\rho: G \rightarrow \GL(V)$ eine Darstellung einer Lie-Gruppe $G$.
    Dann bildet $\rho$ Einparametergruppen nach Einparametergruppen ab.
\end{lemma}

\begin{definition}
    Sei $\rho: G \rightarrow \GL(V)$ eine Darstellung und $X \in \Lie(G)$.
    Definiere:
    \begin{align*}
        \rho_\ast (X) = \frac{d}{dt} \Big|_{t=0} \rho \klammer{\exp(t X)}
        \in \Lie (\GL(V)) = \gl(V)
    \end{align*}
    $\rho_\ast$ ist eine Abbildung $\Lie(G) \rightarrow \gl(V)$.
\end{definition}

\begin{definition}
    Sei $\g$ eine Lie-Algebra über $\R$ oder $\C$. Eine Darstellung
    von $\g$ auf einen VR $V \neq \geschwungeneklammer{0}$ ist ein
    (Lie-Algebra-) Homomorphismus / eine $\R$- (bzw. $\C$-) lineare Abbildung
    $\tau: \g \rightarrow \gl(V)$, so dass
    \begin{align*}
        \eckigeklammer{\tau(X),\tau(Y)}  = \tau(\eckigeklammer{X,Y})
    \end{align*}
\end{definition}

\begin{definition}[Invariant]
    Ein UR $U \subseteq V$ heisst invariant falls $\tau(X) U \subseteq U$
    $\forall X \in \g$.
\end{definition}

\begin{definition}[Irreduzibel]
    Eine Darstellung heisst irreduzibel, wenn die einzigen invarianten
    UR $U = \geschwungeneklammer{0}$ und $U=V$ sind.
\end{definition}

\begin{definition}[Vollständig reduzibel]
    Eine Darstellung heisst vollständig reduzibel, falls
    $V = V_1 \oplus \dotsb \oplus V_k$ mit $V_1,\dots,V_k$ den invarianten
    UR, so dass die Einschränkungen von $\tau$ auf $V_1,\dots,V_k$ irreduzibel
    sind.
\end{definition}

\begin{definition}[komplex/reell]
    Darstellungen heissen komplex bzw reell je nach dem ob $V$ komplex
    oder reell ist.
\end{definition}

\begin{satz}
    Sei $\rho: G \rightarrow \GL(V)$ eine Darstellung der Lie-Gruppe $G$.
    Dann ist $\rho_\ast$ eine Darstellung der reellen Lie-Algebren $\Lie(G)$.
    Die Einschränkung von $\rho$ auf die Einskomponente $G_0$ von $G$ ist
    eindeutig durch $\rho_\ast$ bestimmt.
\end{satz}

\begin{satz}
    Sei $\rho: G \rightarrow \GL(V)$ Darstellung einer zusammenhängenden
    Lie-Gruppe $G$. Dann ist $\rho$ genau dann irreduzibel (bzw. vollständig
    reduzibel) wenn $\rho_\ast$ irreduzibel (bzw. vollständig reduzibel) ist.
\end{satz}

\begin{beispiel}[Triviale Darstellung]
    $V = \C$, $\rho_\ast (X) = 0 \ \forall X \in \g$, $\rho(g) = 1$.
\end{beispiel}

\begin{beispiel}[Adjungierte Darstellung]
    Sei $\text{Ad}: G \rightarrow \GL(\g)$,
    \begin{align*}
        \text{Ad} (g) X = g X g^{-1}
    \end{align*}
    die adjungierte Darstellung von $G$ auf $\g = \Lie(G)$. Die adjungierte
    Darstellung von $\g$ ist $\ad = \text{Ad}_\ast$:
    \begin{align*}
        \ad (X) Y = \frac{d}{dt} \Big|_{t=0} \exp (t X) Y \exp(-tX) = \eckigeklammer{X,Y}
    \end{align*}
\end{beispiel}

\begin{bemerkung}
    \begin{itemize}
        \item $\rho_\ast$ ist die Ableitung von $\rho: G \rightarrow \GL(V)$
            an der Stelle $1$. Also
            \begin{align*}
                \rho_\ast = d \rho(1) : T_1 G = \Lie(G) \rightarrow T_1 \GL(V) = \gl(V)
            \end{align*}
        \item Die Aussagen oben bleiben richtig, wenn man $\GL(V)$ durch allgemeinere
            Lie-Gruppen ersetzt. Für $\rho: G \rightarrow H$ einem Homomorphismus
            von Lie-Gruppen ist
            \begin{align*}
                \rho_\ast = dp(1) : \Lie(G) = T_1 G \rightarrow \Lie(G) = T_1 H
            \end{align*}
            ein Lie-Algebra-Homomorphismus.
    \end{itemize}
\end{bemerkung}

\subsection{Irreduzible Darstellungen von $SU(2)$}

\begin{lemma}
    \begin{enumerate}[(i)]
        \item Jedes $Z \in \sl(n,\C)$ kann eindeutig als $Z=X+iY$
            geschrieben werden mit $X,Y \in \su(n)$.
        \item Sei $\tau$ eine (komplexe) Darstellung von $\su(n)$ auf $V$.
            Dann definiert
            \begin{align*}
                \tau_\C (X+iY) = \tau(X) + i \tau(Y)
            \end{align*}
            eine $\C$-lineare Darstellung der Lie-Algebra $\sl(n,\C)$, deren
            Einschränkung auf $\su(n)$ mit $\tau$ übereinstimmt.
        \item $\tau_\C$ ist genau dann irreduzibel (vollständig reduzibel)
            wenn $\tau$ irreduzibel (vollständig reduzibel) ist.
    \end{enumerate}
    Die Darstellung $\tau_\C$ heisst Komplexifizierung von $\tau$. Oft
    wird die Vereinfachung der Notation $\tau$ statt $\tau_\C$ geschrieben.
\end{lemma}

\begin{theorem}
    Wir klassifizieren die endlichdimensionalen irreduziblen komplexen
    $\C$-linearen Darstellungen von $\sl(2,\C) = \geschwungeneklammer{
    X \in \Mat(2,\C) \ | \ \tr(X) = 0}$. Eine Basis von $\sl(2,\C)$ ist
    \begin{align*}
        h = \begin{pmatrix}
            1 & 0 \\ 0 & -1
        \end{pmatrix}
        \hspace{10pt} , \hspace{10pt}
        e = \begin{pmatrix}
            0 & 1 \\ 0 & 0
        \end{pmatrix}
        \hspace{10pt} , \hspace{10pt}
        f = \begin{pmatrix}
            0 & 0 \\ 1 & 0
        \end{pmatrix}
    \end{align*} 
\end{theorem}

\begin{lemma}
    $[h,e] = 2e$ , $[h,f] = -2f$ , $[e,f] = h$
\end{lemma}

\begin{korollar}
    Ist $\tau: \sl(2,\C) \rightarrow \gl(V)$ eine $\C$-lineare Darstellung,
    so erfüllen
    \begin{align}\label{7.1}
        H = \tau(h)
        \hspace{10pt} , \hspace{10pt}
        E = \tau(e)
        \hspace{10pt} , \hspace{10pt}
        F = \tau(f)
    \end{align}
    die Relationen
    \begin{align}\label{7.2}
        [H,E] = 2 E
        \hspace{10pt} , \hspace{10pt}
        [H,F] = -2F
        \hspace{10pt} , \hspace{10pt}
        [E,F] = H
    \end{align}
    Umgekehrt, sind $H,E,F$ lineare Selbstabbildungen eines komplexen VR $V$,
    die (\ref{7.2}) erfüllen, so existiert eine eindeutige $\C$-lineare
    Darstellung $\tau: \sl(2,\C) \rightarrow \gl(V)$, so dass (\ref{7.1}) gilt.
\end{korollar}

Sei $(\tau,V)$ eine irreduzible Darstellung von $\sl(2,\C)$ und $\lambda \in \C$
der Eigenwert von $H$ mit dem grössten Realteil, $v_0$ ein Eigenvektor zu
$\lambda$: $H v_0 = \lambda v_0$ mit $v_0 \neq 0$.

\begin{lemma}
    \begin{enumerate}[(i)]
        \item $E v_0 = 0$.
        \item Sei $v_k = F^k v_0$. Dann gilt:
            \begin{align*}
                H v_k &= (\lambda - 2k) v_k
                \\
                E v_k &= k (\lambda - k + 1) v_{k-1}
            \end{align*}
    \end{enumerate}
\end{lemma}

Das heisst, $\text{span} (v_0,v_1,\dots)$ ist ein invarianter UR, also wegen
der Irreduzibilität von $V$ gilt $V = \text{span} (v_0,v_1,\dots)$. Die
Vektoren $v_0,v_1,\dots$ sind linear unabhängig, denn sie gehören zu
verschiedenen EW von $H$. Also ist $V$ nur dann endlichdimensionalen, wenn ein
$n \geq 0$ existiert mit $v_{n+1} = 0$. Sei $v_{n+1} =0$, und $v_m \neq 0$
für $m \leq n$. Dann ist $0 = E v_{n+1} = (n+1)(\lambda-n)v_n$ was nur
möglich ist wenn $\lambda = n = 1,2,\dots$.

\begin{satz}
    Sei $n=1,2,\dots$ und $v_0,\dots,v_n$ die Standardbasis von $V_n = \C^{n+1}$.
    Dann definiert
    \begin{align*}
        H v_m &= (n-2m) v_m \\
        E v_m &= m(n+1-m) v_{m-1} \\
        F v_m &= v_{m+1}
    \end{align*}
    eine irreduzible Darstellung $\tau_n$ von $\sl(2,\C)$. Jede komplexe
    $(n+1)$-dimensionale irreduzible Darstellung von $\sl(2,\C)$ ist
    äquivalent zu $\tau_n$.
\end{satz}

\begin{bemerkung}
    Die Operatoren $E,F$ werden oft Auf- und Absteigeoperatoren genannt.
\end{bemerkung}

Wir zeigen nun, dass alle so konstruierten Darstellungen $\tau_n$ aus
Darstellungen $\rho_n$ von $SL(2,\C)$ kommen. Sei $U_n = \C [z_1,z_2]_n$ der
Raum aller homogenen Polynome in zwei Variablen $(z_1,z_2) \in \C^2$ vom
Grad $n$. $U_n$ hat Dimension $n+1$ mit Basis $z_1^n,z_1^{n-1} z_2 ,\dots,
z_1 z_2^{n-1} , z_2^n$.

Wir definieren die Darstellung $\rho_n : SL(2,\C) \rightarrow U_n$
gegeben durch:
\begin{align*}
    \klammer{\rho_n (A) p}(z) = p \klammer{A^{-1} z}
\end{align*}
mit $A \in SL(2,\C) , \ p \in U_n = \C[z_1,z_2] , \ z=(z_1,z_2)$. Dies ist
eine Darstellung und ist insbesondere stetig. Wir berechnen $\rho_{n\ast}:
\sl(2,\C) \rightarrow \gl(U_n)$
\begin{align*}
    \klammer{\rho_{n \ast}(h)p}(z) &= \klammer{- z_1 \frac{\partial}{\partial z_1} + z_2 \frac{\partial}{\partial z_2}} p(z)
    \\
    \klammer{\rho_{n \ast}(e)p}(z) &= - z_2 \frac{\partial}{\partial z_1} p(z)
    \\
    \klammer{\rho_{n \ast}(f)p}(z) &= - z_1 \frac{\partial}{\partial z_2} p(z)
\end{align*}
Wir sehen, dass diese Darstellung äquivalent ist zur Darstellung $\tau_n$
vom obigen Satz. Der Isomorphismus ist
\begin{align*}
    V_n &\rightarrow U_n
    \\
    v_m &\mapsto \frac{(-1)^m}{(n-m)!} z_1^m z_2^{n-1}
\end{align*}
wobei $m=0,1,\dots,n$. Wir wollen noch zeigen, dass die Darstellung
von $SU(2)$ $\rho_n$ unitär ist bzgl. eines geeigneten Skalarproduktes.
Dazu reskalieren wir die Basis $\geschwungeneklammer{v_m}$. Sei
\begin{align*}
    u_m = \lambda_m v_m
    \hspace{10pt} \text{ mit }
    \lambda_m = \sqrt{\frac{(n-m)!}{m!}}
\end{align*}
Dann hat man:
\begin{align*}
    H u_m &= (n-2m) u_m \\
    E u_m &= \sqrt{m(n+1-m)} u_{m-1} \\
    F u_{m-1} &= \sqrt{m(n+1-m)} u_m
\end{align*}
Es gilt $H^\ast = H$ und $E^\ast = F$, wobei $\ast$ bezüglich des Skalarproduktes
definiert ist, in dem $\geschwungeneklammer{u_i}$ eine ONB ist. Allgemeiner gilt
dann $\rho_{n \ast}(X)^\ast = \rho_{n \ast} (X^\ast)$ für $X \in \sl(2,\C)$
und speziell $\rho_{n \ast}(X)^\ast = \rho_{n \ast} (X^\ast) = - \rho_{n \ast} (X)$
für $X \in \su(2)$. Es folgt, dass $\rho_n$ eine unitäre Darstellung von $SU(2)$
ist.

\begin{satz}
    Zu jedem $n=1,2,\dots$ gibt es bis auf Äquivalenz genau eine irreduzible
    Darstellung $(\rho_n,U_n)$ von $SU(2)$ der Dimension $n+1$. Dabei ist
    \begin{align*}
        U_n = \geschwungeneklammer{\sum_{m=0}^n c_m z_1^m z_2^{n-m}
        \ \Big| \ c_m \in \C} = \C[z_1,z_2]_n
    \end{align*}
    der Raum der homogenen Polynome vom Grad $n$ in zwei Unbekannten,
    und für $A \in SU(2)$, $f \in U_n$
    \begin{align*}
        \klammer{\rho_n (A)f}(z) = f(A^{-1} z)
    \end{align*}
    $\rho_n$ ist unitär bezüglich des Skalarproduktes in dem die Basis
    \begin{align*}
        \frac{z_1^m z_2^{n-m}}{\sqrt{m! (n-m)!}}
    \end{align*}
    orthonormiert ist.
\end{satz}

\begin{bemerkung}
    Allgemein nennt man eine Darstellung $\tau: \g \rightarrow \gl(V)$
    ($\g$ reelle Lie-Algebra, $V$ ein $\C$-VR) unitär, falls $\tau(X)^\ast
    = - \tau(X) \ \forall X \in \g$ wobei $(-)^\ast$ bzgl eines Skalarproduktes
    genommen wird.
\end{bemerkung}

\begin{bemerkung}
    Jede Darstellung $\rho$ von $SO(3)$ auf $V$ induziert eine Darstellung
    $\rho \circ \varphi$ von $SU(2)$, wobei $\varphi: SU(2) \rightarrow SO(3)$
    der in \ref{Hom_SU2_SO3} definierte Homomorphismus ist. Die Darstellung
    $\rho \circ \varphi$ hat die Eigenschaft $\rho \circ \varphi(-\mathds{1}) =
    \rho \circ \varphi(\mathds{1}) = \mathds{1}$, da $-\mathds{1} \in \ker(\varphi)$.
    Umgekehrt definiert jede Darstellung von $SU(2)$, die erfüllt $\rho(-\mathds{1})
    = \mathds{1}$, eine Darstellung von $SO(3) \cong SU(2) /
    \geschwungeneklammer{\pm \mathds{1}}$. Also haben wir eine 1:1-Korrespondenz
    zwischen den Darstellungen von $SO(3)$ und den Darstellungen von $SU(2)$,
    die $\rho(-\mathds{1}) = \mathds{1}$ erfüllen. Man prüft leicht, dass
    dabei die irreduziblen Darstellungen wieder auf irreduzible abgebildet werden.
    Die irreduzible Darstellung von $SO(3)$ entsprechen also den irreduziblen
    Darstellungen von $SU(2)$, die $\rho(-\mathds{1}) = \mathds{1}$ erfüllen.
    Wegen $\rho_n(-\mathds{1}) = (-1)^n$ sind dies gerade die $\rho_n$ mit
    $n$ gerade, bzw die irreduziblen Darstellungen von $SU(2)$ ungerader
    Dimension.
\end{bemerkung}

\subsection{Harmonische Polynome und Kugelfunktionen}

\begin{definition}[$H_l$]
    Sei $H_l$ der raum der homogenen Polynome von Grad $l$ in drei
    Unbekannten $x_1,x_2,x_3$:
    \begin{align*}
        H_l = \geschwungeneklammer{\sum_{\stackrel{\abs{\alpha} = l}{\alpha \in \N^3}} c_\alpha x^\alpha \ | \ c_\alpha \in \C}
        = \C[x_1,x_2,x_3]_l
    \end{align*}
\end{definition}

\begin{korollar}
    Der VR $H_l$ hat Dimension $\dim(H_l) = \frac{1}{2} (l+1)(l+2)$.
    Ist $P(x) \in H_l$ so ist es auch $P(R^{-1} x)$ für alle $R \in SO(3)$.
    Wir haben also eine Darstellung von $SO(3)$ auf $H_l$
    \begin{align*}
        \klammer{\rho(R) f}(x) = f \klammer{R^{-1} x}
    \end{align*}
\end{korollar}

\begin{lemma}
    \begin{align*}
        (f,g) = \int_{\abs{x}=1} \overline{f(x)} g(x) d \Omega(x)
    \end{align*}
    ist ein Skalarprodukt auf $H_l$. Die Darstellung $\rho$ ist unitär
    bezüglich $( \ , \ )$.
\end{lemma}

\begin{bemerkung}
    Der Laplaceoperator $\Delta = \sum_{i=1}^3 \frac{\partial^2}{\partial x_i^2}$
    bildet $H_l$ ab nach $H_{l-2}$.
\end{bemerkung}

\begin{definition}[$V_l$]
    Definiere den Raum $V_l$ der harmonischen Polynome in $H_l$.
    \begin{align*}
        V_l = \geschwungeneklammer{f \in H_l \ | \ \Delta f = 0}
    \end{align*}
    Die Dimension von $V_l$ erfüllt
    \begin{align*}
        \dim(V_l) \geq \dim(H_l) - \dim(H_{l-2}) = 2l+1
    \end{align*}
\end{definition}

\begin{bemerkung}
    Für $r^2 = x_1^2 + x_2^2 + x_3^2$ gilt:
    \begin{align*}
        H_l = r^2 H_{l-1} \dsumme V_l
    \end{align*}
\end{bemerkung}

\begin{satz}
    Es gilt die orthogonale Summenzerlegung
    \begin{align*}
        H_l = \bigoplus_{k=0}^{\floor{l/2}} r^{2k} V_{l-2k}
    \end{align*}
    in paarweise orthogonale, $SO(3)$-invariante Unterräume, und dim
    $V_l = 2l+1$.
\end{satz}

\paragraph{Darstellungen von $SO(3)$ auf $V_l$}

Diese Darstellung definiert eine Darstellung $\rho$ von $SU(2)$:
\begin{align*}
    \klammer{\rho(A)u}(x) = \mathfrak{u} \klammer{\varphi(A)^{-1} a}
    \hspace{10pt} \mathfrak{u} \in V_l \hspace{2pt} , \hspace{2pt}
    A \in SU(2) \hspace{2pt} , \hspace{2pt} \varphi: SU(2) \rightarrow SO(3)
\end{align*}
und $\varphi \klammer{\exp \klammer{-i \sum_{j=0}^3 \sigma_j n_j \vartheta/2}}
= R(n,\vartheta)$, $\abs{n} = 1$. Berechne die entsprechende Lie-Algebra Darstellung
$\tau$. Sei
\begin{align*}
    X = \sum_j \alpha_j (-i \sigma_j)
    = \begin{pmatrix}
        -i \alpha_3 & -i \alpha_1 - \alpha_2 \\
        -i \alpha_1 + \alpha_2 + i \alpha_3
    \end{pmatrix}
    \in \su(2)
    \hspace{10pt} , \hspace{5pt} \alpha \in \R^3
\end{align*}
Sei $\alpha = n \vartheta /2$ mit $\abs{n} = 1$. Dann ist
\begin{align*}
    \klammer{\tau(X)u}(x) &= \frac{d}{dt} \Big|_{t=0} u(R(n,t \vartheta)^{-1} x)
    \\
    R(n,\vartheta)^{-1} x &= R(n,-\vartheta)x = (x \cdot n) n + \klammer{x - (x \cdot n) n} \cos(\vartheta)
    \\ &\hspace{10pt} - n \wedge x \sin(\vartheta)
    \\
    \frac{d}{dt} \Big|_{t=0} R(n,t \vartheta)^{-1} x &= - n \wedge x \vartheta = - 2 \alpha \wedge x
\end{align*}
Aus der Kettenregel folgt: \small
\begin{align*}
    &\klammer{\tau(X)u}(x) = -2 \sum_{\beta=1}^3 (\alpha \wedge x)_{\beta} \frac{\partial u}{\partial x_\beta} (x)
    \\
    &= 2 \klammer{\klammer{\alpha_3 x_2 - \alpha_2 x_3} \frac{\partial}{\partial x_1}
        + \klammer{\alpha_1 x_3 - \alpha_3 x_1} \frac{\partial}{\partial x_2}
        + \klammer{\alpha_2 x_1 - \alpha_1 x_2} \frac{\partial}{\partial x_3}} u
\end{align*}
\normalsize
Wir rechnen $\tau_\C$ aus: $H,E,F$ entsprechen $\alpha(0,0,i), \ \alpha
= \klammer{\frac{i}{2},-\frac{1}{2},0}, \ \alpha = \klammer{\frac{i}{2},
\frac{1}{2},0}$. Also:
\begin{align*}
    \tau_\C (h) = H &= - 2i \klammer{x_i \frac{\partial}{\partial x_2} - x_2 \frac{\partial}{\partial x_1}}
    \\
    \tau_\C (e) = E &= x_3 \klammer{\frac{\partial}{\partial x_1} + i \frac{\partial}{\partial x_2}} - (x_1 + i x_2) \frac{\partial}{\partial x_3}
    \\
    \tau_\C (f) = F &= x \klammer{- \frac{\partial}{\partial x_1} + i \frac{\partial}{\partial x_2}} + (x_1 - i x_2) \frac{\partial}{\partial x_3}
\end{align*}
In $V_l$ kennen wir das harmonische Polynom $v_0 = (x_1 + i x_2)^l$. Es erfüllt
$H v_0 = 2 l v_0$ und $E v_0 = 0$. Die Vektoren $v_m = F^m v_0$ spannen eine
irreduzible Darstellung von $\sl(2,\C)$ der Dimension $2l+1$ auf. Da $\dim(V_l)
= 2l+1$ gilt, ist $v_m$ eine Basis von $V_l$. Es folgt, dass $V_l$ eine $2l+1$
dimensionale unitäre Darstellung von $SU(2)$ ist. Eine orthonormierte Basis
finden wir wie folgt: Die Norm im Quadrat von $(x_1 + i x_2)^l$ is:
\begin{align*}
    \Norm{(x_1 + i x_2)^l}^2 &= \int_{S^2} (x_1^2 + x_2^2)^l d \Omega(x)
    = \int_0^\pi \int_0^{2 \pi} \klammer{\sin(\vartheta)}^{2l+1} \ d \vartheta d \varphi
    \\
    &= 2 \pi \int_{-1}^1 (1-x^2)^l \ dx
    = 4 \pi \frac{2^{2l} l!^2}{(2l+1)!}
\end{align*}
Also hat $u_ll(x_1,x_2,x_3)$ Norm eins und die rekursiv definierten Polynome
$u_{l,l-j} (x_1,x_2,x_3)$ sind orthonormiert.
\begin{align}
    u_{ll} (x_1,x_2,x_3) &= \sqrt{\frac{(2l+1)!}{4 \pi}} \frac{(-1/2)^l}{l!} (x_1 + i x_2)^l
    \label{ull}
    \\
    u_{l,l-j} (x_1,x_2,x_3) &= \frac{F u_{l,l-j+1} (x_1,x_2,x_3)}{\sqrt{j (2l+1-j)}}
    \label{ullj}
\end{align}

\begin{satz}
    Die Darstellung von $SU(2)$ auf dem Raum $V_l$ der harmonischen, homogenen
    Polynomen vom Grad $l$ in drei Unbekannten ist irreduzibel und unitär
    bezüglich $(f,g) = \int_{S^2} \overline{f} g \ d \Omega$. (\ref{ull}),
    (\ref{ullj}) definiert eine orthonormierte Basis und es gilt
    \begin{align*}
        H u_{lm} &= 2 m u_{lm}
        \\
        E u_{lm} &= \sqrt{(l-m)(l+m+1)} u_{l,m+1}
        \\
        F u_{lm} &= \sqrt{(l-m+1)(l+m)} u_{l,m-1}
    \end{align*}
\end{satz}

\begin{definition}[Kugelfunktion]
    Eine Kugelfunktion $Y: S^2 \rightarrow \C$ von Index $l$ ist die
    Einschränkung auf $S^2 \subset \R^3$ eines homogenen harmonischen
    Polynoms vom Grad $l$.
\end{definition}

Es bezeichne $\hat{V}_l$ den VR der Kugelfunktionen von Index $l$. Also ist
$Y = Y(\vartheta,\varphi)$ genau dann in $\hat{V}_l$ wenn $r^l Y(\vartheta,
\varphi) \in V_l$. Eine orthonormierte Basis von $\hat{V}_l$ ist also durch
$Y_{lm}(\vartheta,\varphi)$ gegeben.
\begin{align*}
    Y_{lm} (\vartheta,\varphi) &= r^{-l} u_{lm}(r,\vartheta,\varphi)
    \\
    &:=
    u_{lm} (r \sin(\vartheta),\cos(\varphi),r \sin(\vartheta) \sin(\varphi),
    r \cos(\varphi))
\end{align*}
Insbesondere haben wir
\begin{align*}
    Y_ll(\vartheta,\varphi) &= \sqrt{\frac{(2l+1)!}{4 \pi}} \frac{(-2)^l}{l!} \klammer{\sin(\vartheta)}^l e^{il\varphi}
    \\
    H Y_{lm} &= 2 m Y_{lm}
    \\
    E Y_{lm} &= \sqrt{(l-m)(l+m+1)} Y_{l,m+1}
    \\
    F Y_{lm} &= \sqrt{(l-m+1)(l+m)} Y_{l,m-1}
\end{align*}
wobei in die Operatoren $H,E,F$ Kugelkoordinaten einzusetzen sind:
\begin{align*}
    H &= \frac{2}{i} \frac{\partial}{\partial \varphi}
    \hspace{10pt} , \hspace{10pt}
    E = e^{i \varphi} \klammer{\frac{\partial}{\partial \vartheta} + i \cot(\vartheta) \frac{\partial}{\partial \varphi}}
    \\
    F &= e^{-i \varphi} \klammer{- \frac{\partial}{\partial \vartheta} + i \cot(\vartheta) \frac{\partial}{\partial \varphi}}
\end{align*}
Es folgt, dass $Y_{lm}$ die Form $Y_{lm}(\vartheta,\varphi) = F_{lm}(\vartheta) e^{i m \varphi}$
hat. Die Orthonormalitätsrelation ist
\begin{align*}
    \int_0^\pi \int_0^{2 \pi} \overline{Y_{lm}(\vartheta,\varphi)} Y_{l'm'} (\vartheta,\varphi) \sin(\vartheta) \ d \vartheta d \varphi
    = \delta_{ll'} \delta_{mm'}
\end{align*}
Der sphärische Laplace Operator $\Delta_{S^2}$ auf $C^\infty (S^2)$ ist durch
die Formel für den Laplace Operator in Kugelkoordinaten definiert.
\begin{align*}
    \Delta = \frac{\partial^2}{\partial r^2} + \frac{2}{r} \frac{\partial}{\partial r} + \frac{1}{r^2} \Delta_{S^2}
    \hspace{10pt} , \hspace{10pt}
    \Delta_{S^2} = \frac{\partial^2}{\partial \theta^2} + \cot(\vartheta) \frac{\partial}{\partial \theta} + \frac{1}{\sin^2(\theta)} \frac{\partial^2}{\partial \varphi^2}
\end{align*}

\begin{satz}
    Die Funktionen $Y_{lm} (\vartheta,\varphi)$ bilden für $l=0,1,2,\dots$
    und $m = -l,-l+1,\dots,l$ eine orthonormierte Basis von $L^2 (S^2,d \Omega)$.
\end{satz}

\subsection{Tensorprodukte von $SU(2)$ Darstellungen}

\begin{definition}[ONB im Hilbertraum]
    Eine ONB im Hilbertraum ist ein orthonormales System $\geschwungeneklammer{e_i}_{i \in I}$
    welches vollständig ist, d.h. $\forall f$ mit $\langle f, e_i \rangle = 0$
    folgt $f=0$.
\end{definition}

\begin{definition}[Tensorprodukt]
    Das Tensorprodukt von zwei endlichdimensionalen Darstellungen $(\rho,V),
    (\rho',V')$ einer Gruppe $G$ ist die Darstellung $\rho \otimes \rho'$
    auf dem Tensorprodukt $V \otimes V'$, die durch die Formel
    \begin{align*}
        (\rho \otimes \rho')(g) &= \rho(g) \otimes \rho'(g)
        \\ \Rightarrow \
        \klammer{\rho(g) \otimes \rho'(g)}(v \otimes v')
        &= \klammer{\rho(g)v} \otimes \klammer{\rho'(g) v'}
    \end{align*}
    gegeben wird. Es folgt aus den Tensorprodukteigenschaften, dass diese Formel
    eine Darstellung defineirt, und dass die Assoziativität $(\rho \otimes \rho')
    = \rho'' = \rho \otimes \klammer{\rho' \otimes \rho''}$ gilt, wenn die
    Darstellungsräume $(V \otimes V') \otimes V''$, $V \otimes (V' \otimes V'')$
    durch $(v \otimes v') \otimes v'' = v \otimes (v' \otimes v'')$ identifiziert
    wird.
\end{definition}


\begin{definition}[Tensorprodukt]
    Das Tensorprodukt von zwei Darstellungen $(\tau,V), \ (\tau',V')$ eine
    Lie-Algebra $\g$ ist die Darstellung $\tau \otimes \tau'$ von $\g$ auf
    $V \otimes V'$ so dass
    \begin{align*}
        \klammer{\tau \otimes \tau'}(x) &= \tau(x) \otimes 1_{V'} + 1_V \otimes \tau'(x)
        \\
        \klammer{(\tau \otimes \tau')(x)}(v \otimes v') &= \tau(x)(v) \otimes v' + v \otimes \klammer{\tau'(x)v'}
    \end{align*}
    Dies ist wie folgt motiviert: Ist $G$ eine Lie Gruppe mit Lie-Algebra $\g$,
    so wird die Daratellung $(\rho \otimes \rho')_\ast$ durch
    \begin{align*}
        (\rho \otimes \rho')_\ast (X) = \rho_\ast (X) \otimes 1_{V'} + 1_V \otimes \rho_\ast' (X)
        \ \forall X \in G
    \end{align*}
    gegeben. Es ist nämlich nach der Produktregel:
    \begin{align*}
        &\frac{d}{dt} \Big|_{t=0} \klammer{\exp(t \rho_\ast (X)) \otimes
            \exp(t \rho_\ast' (X))}
        \\
        &= \frac{d}{dt} \Big|_{t=0} \exp(t \rho_\ast (X)) \otimes 1_{V'}
        + 1_V \otimes \frac{d}{dt} \Big|_{t=0} \exp(t \rho_\ast' (X))
    \end{align*}
\end{definition}

Die Grundsätliche Frage ist: Für $2$ irreduzible Darstellungen $\rho,\rho'$:
Wie spaltet man $\rho \otimes \rho'$ in eine Summe irreduzibler Darstellungen
auf?

Wir betrachten den Fall $G = SU(2)$. Sei $\rho = \rho_{n'} \otimes \rho_{n''}$.
Dann ist $v_0',\dots,v_{n'}'$ eine Basis vom Darstellungsraum von $\rho_{n'}$,
und $v_0'',\dots,v_{n''}''$ eine Basis vom Darstellungsraum von $\rho_{n''}$
mit
\begin{align*}
    H v_j' = (n' - 2j) v_j'
    \hspace{10pt} , \hspace{10pt}
    E v_j' = j(n' + 1 - j) v_{j-1}'
    \hspace{10pt} , \hspace{10pt}
    F v_j' = v_{j+1}'
\end{align*}
und analog für $\rho_{n''}$. Eine Basis des Darstellungsraumes von $\rho_{n'}
\otimes \rho_{n''}$ ist also $W := (v_j' \otimes v_k'')_{\stackrel{j=0,\dots,n'}{k=0,\dots,n''}}$.
Es gilt:
\begin{align*}
    H(v_j' \otimes v_k'')
    &= H(v_j') \otimes v_k'' + v_j' \otimes H(v_k'')
    \\
    &= \klammer{n' + n'' - 2(j+k)} (v_j' \otimes v_k'')
\end{align*}
Also ist $W$ eine Basis aus Eigenvektoren von $H$. Die Eigenwerte sind $n' +
n'' - 2l$ mit $l=0,\dots,n'+n''$ und $v_j' \otimes v_k''$ mit $j+k=l$,
$j \in \geschwungeneklammer{0,\dots,n'}$, $k \in \geschwungeneklammer{0,\dots,n''}$.
Wir möchten nun schreiben: $\rho_{n'} \otimes \rho_{n''} \cong \rho_{n_1}
\oplus \dotsb \oplus \rho_{n_r}$ In jeder dieser Darstellungen gibt es einen
EV von $H$ der zusätzlich im Kern von $E$ liegt. Dieser erfüllt jeweils
$H w = n_j w$ und $E w = 0$. Wir suchen also Vektoren
\begin{align*}
    w=\sum_{j=0}^l a_j v_j' \otimes v_{l-j}''
\end{align*}
s.d. $E w = 0$. Wir nehmen zunächst an, dass $l \leq \min(n',n'')$, s.d. alle
$v_j',v_{l-j}''$ die in $w$ vorkommen wohldefiniert sind. Man findet durch
Korffizientenvergleich:
\begin{align*}
    a_j = (-1)^j \frac{(n'-j)! (n'' - l + j)!}{j! (l-j)!}
\end{align*}
Die Dimension des Lösungsraumes des lin GLS $H w = n_j w$, $E w = 0$ ist
aber die Vielfachheit von der irreduziblen Darstellung $\rho_n$, in der
Zerlegung $\rho_{n'} \otimes \rho_{n''} = \rho_{n_1} \oplus \dotsb \oplus
\rho_{r_n}$. Wir haben gefunden: $\forall l = 0,\dots,\min(n',n'')$
und $n=n'+n''-2l$ ist diese Vielfachheit also gleich $1$. Betrachte aber die
Dimension der gefundenen Summanden
\begin{align*}
    \sum_{l=0}^{\min(n',n'')}
    &\stackrel{\text{o.B.d.A} \ n' \geq n''}{=}
    (n' + n'' + 1)(n'' + 1) - 2 \frac{n'' (n'' + 1)}{2}
    \\
    &= (n'' + 1) (n' + 1)
    = \dim(\rho_{n'} \otimes \rho_{n''})
\end{align*}

\begin{satz}[Cebsch-Gordon Zerlegung]
    Die Zerlegung eines Tensorproduktes von irreduziblen Darstellungen
    von $SU(2)$ (bzw. von $\su(2),\sl(2,\C)$) der Dimensionen $n'+1$, $n''+1$
    ist:
    \begin{align*}
        \rho_{n'} \otimes \rho_{n''} \cong
        \rho_{n' + n''} \oplus \rho_{n' + n'' - 1} \oplus \dotsb \oplus
        \rho_{\abs{n' - n''}}
    \end{align*}    
    Die irreduzible Unterdarstellung der Dimension $n'+n''+1-2l$ ist aufgespannt
    durch $w_l,Fw_l,\dots,F^{n'+n''-2l} w_l$ wobei, bezüglich der oben definierten
    Basen,
    \begin{align*}
        w_l = \sum_{j=0}^l (-1)^j \frac{(n' - j)! (n'' - l + j)!}{j! (l-j)!} v_j' \otimes v_{l-j}''
    \end{align*}
\end{satz}


\end{document}