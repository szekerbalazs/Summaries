\section{Nullstellensuche}

\vspace{1\baselineskip}

Geg.: $F: U \subset \R^n \rightarrow \R^n$ eine beliebige Funktion

Ges.: $x^{\star} \in \R^n$ s.d. $F(x^{\star}) = 0$

\vspace{1\baselineskip}

\Definition{ (Iteratives Verfahren)

    Ein Iteratives Verfahren ist ein Algorithmus definiert durch:

    - Einen Startwert $x_0$

    - Eine Iterationsvorschrift $x_{k+1} = \phi (x_k)$

    - Eine Abbruchbedingung, wie zB.

        \hspace{10pt} - Max. Anzahl Schritte

        \hspace{10pt} - Absolute Toleranz: $\Norm{x_{k+1} - x_k} <$ abstol

        \hspace{10pt} - Relative Toleranz: $\frac{\Norm{x_{k+1} - x_k}}{\Norm{x_{k+1}}} <$ reltol

    Somit erzeugt es eine Folge $x_1 , \dots , x_N$ von approximierten Lösungen zu einem
    Problem.
}

\vspace{1\baselineskip}

\Definition{

    Sei $\limes{k \rightarrow \infty} x_k = x^\star$ für ein $x^\star$.
    Dann ist der \fat{Fehler} definiert als: $e_k := \Norm{x^\star - x_k}$.
    In der Regel ist $x^\star$ nicht bekannt. Dann: $x^\star \approx x_N$ für
    $N>>1$, $e_k \approx \Norm{x_N - x_k}$.
}

\vspace{1\baselineskip}

\Definition{ (Konvergenzordnung $p$)

    Sei $c>0$ sodass $e_{k+1} \leq c e_k^p$. Dann folgt die Berechnung:
    \begin{align*}
        \begin{rcases}
            e_{k+1} \approx c e_{k}^p \\
            e_k \approx c e_{k-1}^p
        \end{rcases}
        \Rightarrow
        p \approx \frac{\log \klammer{\frac{e_{k+1}}{e_k}}}{\log \klammer{\frac{e_k}{e_{k-1}}}}
        =: p_k
    \end{align*}
    \underline{Achtung}! $p$ ist ein Array mit einträgen $[p_1,\dots,p_{N-2}]$.
    Wähle für $p$ einfach den letzten "vernünftigen" Eintrag.

    \vspace{1\baselineskip}

    \underline{Wichtig}! Dies ist nicht die selbe Konvergenzordnung wie bei DGL oder
    Quadratur.
}

\vspace{1\baselineskip}

\Definition{

    $x^{(k+1)} = \Phi (x^{\star})$ heisst \fat{linear konvergent} nach $x^\star$, falls
    es ein $L < 1$ gibt, sodass:
    \begin{align*}
        \Norm{x^{(k+1)} - x^{\star}} \leq L \Norm{x^{(k)} - x^\star}
        \ \ \ \forall k \in \N
    \end{align*}
}

\vspace{1\baselineskip}

\Bemerkung{
    \begin{align*}
        \Norm{x^{(k+1)} - x^{\star}} \leq L \Norm{x^{(k)} - x^\star} \leq L^{k+1} \Norm{x^{(0)} - x^\star}
    \end{align*}
}

\vspace{1\baselineskip}

\Definition{

    \fat{Konvergenzordnung $p$} des Iterativen Verfahrens heisst, es gibt ein $C>0$, sodass
    \begin{align*}
        \Norm{x^{(k+1)} - x^\star} \leq C \cdot \Norm{x^{(k)} - x^\star}^p
    \end{align*}
}

\vspace{1\baselineskip}

\Bemerkung{

    Bei linearer Konvergenz und bekanntem $L$ kann man folgende Abschätzung machen:
    \begin{align*}
        \Norm{x^{(k+1)} - x^\star} \leq \frac{1}{1-L} \Norm{x^{(k+1)} - x^{(k)}}
    \end{align*}
}

\pagebreak

\underline{\fat{Fixpunktiteration}}

\vspace{1\baselineskip}

\fat{Vorgehen}

Wähle ein $\phi(x)$. Iteratives Verfahren mit Iterationsvorschrift:
$x_{k+1} = \phi (x)$

\vspace{1\baselineskip}

\fat{Konvergenz}

\vspace{1\baselineskip}

\Definition{

    Eine Funktion $\phi$ heisst \fat{Kontraktion}, falls es ein $L>1$ gibt, so dass
    $\Norm{\phi(x) - \phi(y)} \leq L \Norm{x-y}$ für alle $x,y$.
}

\vspace{1\baselineskip}

\Bemerkung{

    Wenn $x^\star$ ein Fixpunkt der Kontraktion $\phi$ ist, dann ist
    \begin{align*}
        \Norm{x^{(k+1)} - x^\star} = \Norm{\phi (x^{(k)}) - \phi(x^\star)}
        \leq L \Norm{x^{(k)} - x^\star}
    \end{align*}
    Das heisst, dass das iterative Verfahren $x^{k+1} = \phi (x^k)$ mindestens linear
    konvergiert.
}

\vspace{1\baselineskip}

\Satz{

    Sei $D \subset \K^n$ ($\K = \R , \C$), mit $D$ abgeschlossen, und $\phi:D \rightarrow D$
    einer Kontraktion. Dann existiert ein eindeutiger Fixpunkt $x^\star$, also
    $\phi(x^\star) = x^\star$. Dieser ist der Grenzwert der Folge $x^{(k+1)} = \phi(x^{(k)})$.
}

\vspace{1\baselineskip}

\Satz{ (Hinreichende Bedingung für lokale lineare Konvergenz einer Fixpunktiteration)

    Es Sei $U$ konvex und $\phi:U \subset \R^n \rightarrow \R^n$ stetig differenzierbar
    mit $L := \sup_{x \in U} \Norm{D \phi(x)} < 1$ wobei ($D \phi (x)$ die Jacobi-Matrix
    von $phi$ ist). Wenn $\phi(x^\star) = x^\star$ für $x^\star \in U$, dann konvergiert
    die Fixpunktiteration $x^{(k+1)} = \phi(x^{(k)})$ gegen $x^\star$ lokal mindestens
    linear.
}

\vspace{1\baselineskip}

\Satz{

    Sei $\phi: \R^n \rightarrow \R^n$ mit $\phi(x^\star) = x^\star$ und $\phi$ stetig
    differenzierbar in $x^\star$. Ist $\Norm{D\phi(x^\star)} < 1$, dann konvergiert die
    Fixpunktiteration $x^{(k+1)} = \phi (x^{(k)})$ lokal (mindestens) linear, mit
    $L = \Norm{D \phi (x^\star)}$.
}

\vspace{1\baselineskip}

\Satz{

    Sei $U \subset \R$ ein Intervall und $\phi: U \rightarrow \R$ $(m+1)$-mal differenzierbar
    mit $\phi(x^\star) = x^\star \in U$. Sei weiterhin $\phi^{(l)} (x^{\star}) = 0$ für
    $l=1,\dots,m$ ($m \geq 1$). Dann konvergiert die Fixpunktiteration
    $x^{(k+1)} = \phi(x^{(k)})$ gegen $x^\star$ lokal der Ordnung $p \geq m+1$.
}

\vspace{1\baselineskip}

\Lemma{

    Konvergiert die Kontraktion $\phi$ linear mit dem Faktor $L<1$, dann gilt die
    folgende Abschätzung:
    \begin{align*}
        \Norm{x^\star - x^{(k)}} \leq \frac{L^{k-l}}{1-L} \Norm{x^{(l+1)} - x^{(l)}}
    \end{align*}
}

\vspace{1\baselineskip}

\underline{\fat{Bisektionsverfahren (Intervallhalbierungsverfahren)}}

\vspace{1\baselineskip}

\underline{Idee}: Zwischenwertsatz:
$F:[a,b] \rightarrow \R$ stetig mit $F(a) F(b) < 0 \Rightarrow \exists x^\star \in [a,b]$
sodass $F(x^\star) = 0$

\underline{Vorgehen}:
Iteratives Verfahren mit der Iterationsvorschrift:

- Intervall halbieren $m=\frac{a+b}{2}$

- Suche Intervall mit unterschiedlichen Vorzeichen:
$F(a) F(m) < 0$ oder $F(m) F(b) < 0$

- Weiter mit Intervall "$<0$"

\vspace{1\baselineskip}

Konvergiert immer. Lineare Konvergenz. Keine verallgemeinerung in mehreren
Dimensionen.

\vspace{1\baselineskip}

\underline{\fat{Newtonverfahren}}

\vspace{1\baselineskip}

\underline{Idee}
Approximation von $F$ durch Tangente bei $x_k$ (Taylor):

$F(x) \approx \tilde{F} (x) := F(x_k) + F'(x) \cdot (x-x_k) \stackrel{!}{=} 0$

\underline{Vorgehen}:
(Newtonverfahren)

Iteratives Verfahren mit Iterationsvorschrift:
$x_{k+1} = x_k - \frac{F(x_k)}{F'(x_k)}$

\vspace{1\baselineskip}

\underline{Mehrdimensional}

Gleiche Idee: $x_{k+1} = x_k - DF(x_k)^{-1} F(x_k) := x_k - s_k$
Aber $DF(x_k) := (\frac{\partial F_i}{\partial x_j})_{ij} \in \Mat(n \times n;\R)$
"Berechne nie das Inverse einer Matrix!"

\underline{Vorgehen}
(Mehrdimensionales Newtonverfahren)

Iteratives Verfahren mit Iterationsvorschrift:

- $\vec{s}_k =$ Lösung des linearen GLS $DF(\vec{x}_k) \vec{s}_k = F (\vec{x}_k)$

- $\vec{x}_{k+1} = \vec{x}_k - \vec{s}_k$

Newtonverfahren = Fixpunktiteration mit $\phi(x) = x- \frac{F(x)}{F'(x)}$
$\Rightarrow$ lokal mindestens quadratische Konvergenz.

\vspace{1\baselineskip}

\underline{Sekantenverfahren}

Falls $F'$ unbekannt: $F'(x) \approx \frac{F(x_k) - F(x_{k-1})}{x_k - x_{k-1}}$

Konvergenzordnung: $p \approx 1.62$

Beachte: 2. Startwert benötigt. Mehrdimensional nicht möchlich.

\vspace{1\baselineskip}

\underline{\fat{Gedämpftes Newtonverfahren}}

\vspace{1\baselineskip}

\underline{Idee}:
Dämfung von $s_k$ mit einem Dämpfungsparameter $\lambda_k \in (0,1]$

Es wird das maximale $\lambda_k$ gewählt, so dass
\begin{align*}
    \Norm{\frac{F \klammer{x^{(k)} - \lambda^{(k)} \frac{F(x^{(k)})}{DF(x^{(k)})}}}{DF(x^{k})}}_2
    \leq
    \klammer{1-\frac{\lambda^{(k)}}{2}} \Norm{\frac{F(x^{(k)})}{DF(x^{(k)})}}_2
\end{align*}
Praxis: $\lambda_k = 1 \rightarrow \frac{1}{2} \rightarrow \frac{1}{4} \rightarrow \dots$
bis die obere Bedingung erfüllt ist.

\vspace{1\baselineskip}

\underline{Vorgehen}:
(Gedämpftes Newtonverfahren)

Iteratives Verfahren mit Iterationsvorschrift:

- $\vec{s}_k =$ Lösung von $DF(\vec{x}_k) \vec{s}_k = F(\vec{x}_k)$ oder in 1D: $\frac{F(x_k)}{F'(x_k)}$

- $\lambda_k = \max \geschwungeneklammer{\frac{1}{2^n} \ | \ n \in \N_0 \text{ und erfüllt obige Bedingung}}$
(in jedem Iterstionsschritt neu berechnen!)

- $\vec{x}_{k+1} = \vec{x}_k - \lambda_k \vec{s}_k$

\vspace{1\baselineskip}

\underline{\fat{Quasi-Newtonverfahren}}

\vspace{1\baselineskip}

\underline{Broyden Verfahren}

Setze $J_0 = DF(x_0) \in M(n \times n , \R)$ und löse für $k=1,2,\dots$
\begin{align*}
    \begin{cases}
        J_k \cdot s_k = F(x_k) \ \Leftrightarrow s_k = \text{solve} (J_k , F(x_k))
        \\
        x_{k+1} = x_k - s_k
        \\
        J_{k+1} = J_k + \frac{1}{\Norm{s_k}^2} F(x_{k+1}) (-s_k)^T
    \end{cases}
\end{align*}

\vspace{1\baselineskip}

\underline{Sherman-Morrison-Formel}
\begin{align*}
    J_{k+1}^{-1} = J_k^{-1} + \frac{J_k^{-1} F(x_{k+1}) s_k^T J_k^{-1}}{\Norm{s_k}^2 - s_k^T J_k^{-1} F(x_{k+1})}
\end{align*}

\vspace{1\baselineskip}

Daraus entsteht folgende \underline{Iteration}:
\begin{enumerate}
    \item Schritt:
        \begin{align*}
            \begin{cases}
                J_0 = DF(x_0) \\
                s_0 = \text{solve} (J_0 , F(x_0)) \\
                x_1 = x_0 - s_0
            \end{cases}
        \end{align*}
    \item Schritt:
        \begin{align*}
            \begin{cases}
                J_1^{-1} = J_0^{-1} + \frac{J_0^{-1} F(x_{1}) s_0^T J_0^{-1}}{\Norm{s_0}^2 - s_0^T J_0^{-1} F(x_{1})}
                \\
                s_1 = J_1^{-1} F(x_1)
                \\
                x_2 = x_1 - s_1
                \\
                \vdots
            \end{cases}
        \end{align*}
\end{enumerate}

\underline{Alternativ}
\begin{align*}
    s_{k+1} = J_{k+1}^{-1} F(x_{k+1})= J_{k}^{-1} F(x_{k+1}) + \frac{J_k^{-1} F(x_{k+1}) s_k^T J_k^{-1} F(x_{k+1})}{\Norm{s_k}^2 - s_k^T J_k^{-1} F(x_{k+1})}
\end{align*}
Wir definieren nun $w_k := J_{k}^{-1} F(k+1)$ (Lsg. des GLS $J_k w_k = F(x_{k+1})$)
und $z_k := s_k^T w_k$.
Dann folgt:
\begin{align*}
    S_{k+1} = \klammer{1 + \frac{z_k}{\Norm{s_k}^2 - z_k}} w_k
\end{align*}

