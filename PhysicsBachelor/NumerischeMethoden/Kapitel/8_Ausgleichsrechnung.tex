\section{Ausgleichsrechnung}

\Bemerkung{

    cond$(A^T A) =$ cond$(A^2)$
}

\vspace{1\baselineskip}

\Bemerkung{

    Norm minimieren: finde $x$ sodass $A^T A x = A^T b$
}

\vspace{1\baselineskip}

\fat{Ausgleichsrechnung}

Geg.: Daten $(t_i,y_i) \in \R^2$, $i \in \geschwungeneklammer{1,\dots,m}$, Modell
$f_{\vec{x}}: \R \rightarrow \R$, $f_{\vec{x}} (t) = \vec{y}$

Ges. Parameter $\vec{x} = (x_1,\dots,x_n)$, so dass die Daten und das Modell am besten passen.
\begin{align*}
    \text{Wir wollen}:
    \begin{pmatrix}
        f_{\vec{x}} (t_1) \\ \vdots \\ f_{\vec{x}} (t_m)
    \end{pmatrix} = \begin{pmatrix}
        y_1 \\ \vdots \\ y_m
    \end{pmatrix} \Leftrightarrow
    \underbrace{
    \begin{pmatrix}
        \abs{f_{\vec{x}} (t_1) - y_1} \\ \vdots \\ \abs{f_{\vec{x}} (t_m) - y_m}
    \end{pmatrix}
    }_{:=\vec{R} = \text{Residuum / Residuenvektor}}
    = \vec{0}
\end{align*}
Bei der Ausgleichsrechnung: $m > n$. Das erste GLS ist somit überbestimmt und es gibt keine
Lösung. Stattdessen erhalten wir das Minimierungsproblem:

\fat{Least Squares Problem}: $\vec{x^\star} = \text{argmin}_{\vec{x}} \sum_{i=1}^{m}
    \abs{f_{\vec{x}} (t_i) - y_i}^2 = \text{argmin}_{\vec{x}} \Norm{\vec{R}}_2^2$

\vspace{1\baselineskip}

\underline{\fat{Lineare Ausgleichsrechnung}}

Falls $f_{\vec{x}}$ linear im Parameter $\vec{x}$, also $f_{\vec{x}} (t) = x_1 b_1 (t) +
\dots + x_n b_n (t)$ für Basisfunktionen $b_j(t)$ (nicht notwendigerweise linear in $t$),
dann handelt es sich um \fat{lineare Ausgleichsrechnung}. Für diesen Spezialfall lässt sich
das Problem umschreiben zu:
\begin{align*}
    \underbrace{\begin{pmatrix}
        b_1(t_1) & \dots & b_n(t_1) \\
        \vdots& \ddots & \vdots \\
        b_1(t_m) & \dots & b_n(t_n)
    \end{pmatrix}}_{:= A \in M(m \times n;K)}
    \underbrace{\begin{pmatrix}
        x_1 \\ \vdots \\ x_n
    \end{pmatrix}}_{:= \vec{x}}
    = \underbrace{\begin{pmatrix}
        y_1 \\ \vdots \\ x_m
    \end{pmatrix}}_{:= \vec{b}}
    \Leftrightarrow
    A \vec{x} = \vec{b}
    \Leftrightarrow
    \Norm{A \vec{x} - \vec{b}} = 0
\end{align*}
Da $m>n$ ist GLS $A \vec{x} = \vec{b}$ wie im allgemeinen Fall überbestimmt. Wir lösen also
Stattdessen das für den linearen Fall umschriebene \fat{Least Squares Problem}:

$x^\star = \text{argmin}_{\vec{x}} \Norm{A \vec{x} - \vec{b}}_2^2$

Code: numpy.linalg.lstsq(A,b)[0]

\vspace{1\baselineskip}

\underline{Normalengleichung}

Idee: Bestimme Minimum durch: $\grad (\Norm{A \vec{x} - \vec{b}}_2^2) \stackrel{!}{=} 0$

$\vec{x^\star} = \text{argmin} \Norm{A \vec{x} - \vec{b}}_2^2 \Leftrightarrow A^T A \vec{x^\star} = A^T \vec{b}$

Code: ATA = np.dot(A.T,A)

\hspace{27pt} ATb = np.dot(A.T,b)

\hspace{27pt} xstar = numpy.linalg.solve(ATA,ATb)

\vspace{1\baselineskip}

\underline{QR-Zerlegung}

Idee: $A = QR = \hat{Q} \hat{R}$

$\Norm{A \vec{x} - \vec{b}}_2^2 = \Norm{Q R \vec{x} - \vec{b}}_2^2 = \Norm{R \vec{x} - Q^T \vec{b}}_2^2$

$= \Norm{\hat{R} \vec{x} - \hat{Q}^T \vec{b}}_2^2 + \Norm{\begin{pmatrix}
    q_{n+1} \\ \vdots \\ q_m
\end{pmatrix} \vec{b}}_2^2$

Minimiere den von $\vec{x}$ abhängigen Teil exakt:

$\vec{x^\star} = \text{argmin} \Norm{A \vec{x} - \vec{b}}_2^2
\Leftrightarrow \hat{R} \vec{x^\star} = \hat{Q}^T \vec{b}$

Code: Qhat,Rhat = numpy.linalg.qr(A)

\hspace{27pt} xstar = numpy.linalg.solve(Rhat , dot(Qhat.T , b))

\vspace{1\baselineskip}

\underline{Singulärwertzerlegung}

Idee: $A=U \Sigma V^T = \begin{pmatrix} U_1 & | & U_2 \end{pmatrix} \begin{pmatrix}
    \Sigma_r  & 0 \\ 0 & 0 \end{pmatrix} \begin{pmatrix} V_1^T \\ \hline V_2^T \end{pmatrix}$

Dabei ist $r = \rang A =$ Anzahl Singulärwerte $\neq 0$, $u_1 \in \R^{m \times r}$,
$V_1 \in \R^{n \times r}$. Analoge Überlungen wie bei QR liefert:

$\vec{x^\star} = \text{argmin} \Norm{A \vec{x} - \vec{b}}_2^2 \Leftrightarrow \vec{x^\star} =
V_1 \Sigma_r^{-1} U_1^T \vec{b}$

Code: psinvA = numpy.linalg.pinv(A)

\hspace{27pt} xstar = dot(psinvA,b)

\vspace{1\baselineskip}

\underline{Allgemeines Vorgehen}
\begin{enumerate}
    \item Daten implementieren: t = np.array$([t_1,\dots,t_m])$, y = np.array$([y_1,\dots,y_m])$
    \item Basisfunktionen und $A$ bestimmmen:
    
            b = lambda s: np.array$([b_1(s),\dots,b_n(s)])$,
    
            A = np.array$([b(s) \text{ for $s$ in $t$}])$
    \item Ausgleichsrechnung lösen: lstsq, red. QR oder SVD
    \item Lösung plotten:

            tnew = np.linspace(t[0],t[-1],1000)

            f = lambda t : np.dot(x,b(t))

            plt.plot(t,y)

            plt.plot(tnew,f(tnew))
\end{enumerate}

\vspace{1\baselineskip}

\underline{\fat{Nichtlineare-Ausgleichsrechnung}}

$f_{\vec{x}}$ ist nicht linear in $\vec{x}$. Ziel: Minimierung der Quadrate:
$\vec{x^\star} = \text{argmin}_{\vec{x}} \sum_{i=1}^{m} \abs{f_{\vec{x}}(t_i) - y_i}^2$

Definiere:
\begin{align*}
    F(\vec{x}) &:= \begin{pmatrix}
        f_{\vec{x}} (t_1) - y_1 \\ \vdots \\ f_{\vec{x}} (t_m) - y_m
    \end{pmatrix}
    \\
    \phi(\vec{x}) &:= \frac{1}{2} \Norm{F(\vec{x})}_2^2 \rightarrow
    \vec{x^\star} = \text{argmin}_{\vec{x}} \sum_{i=1}^{m} \abs{f_{\vec{x}}(t_i) - y_i}^2
    \\ &\Leftrightarrow \vec{x^\star} = \text{argmin} \phi(\vec{x})
\end{align*}
Code: xstar = scipy.optimize.leastsq$(F,x_0)$[0]

mit $x_0$ einem guten Startwert, oftmals reicht ein Einervektor: x0 = np.ones(n)

\vspace{1\baselineskip}

\underline{Newton Verfahren}

Idee: $\phi(\vec{x})$ minimal $\Leftrightarrow \grad (\phi(\vec{x})) = 0 \Rightarrow$
bestimme NST mit Newtonverfahren. Wir brauchen dazu auch die Ableitung von
$\grad (\phi(\vec{x}))$, also die Hessematrix.

Vorgehen:

- $\grad (\phi(\vec{x})) := (DF( \vec{x}))^T F(\vec{x})$

- $Hess (\phi(\vec{x})) := (DF(\vec{x}))^T DF(\vec{x}) + \sum_{j=1}^{m} F_j (\vec{x}) Hess(F_j(\vec{x}))$

- $x^\star =$ Newton$(\grad(\phi) , Hess(\phi) , \vec{x}_0 , \text{tol} , \text{maxit})$

\vspace{1\baselineskip}

\underline{Gauss-Newton Verfahren}

Idee: Taylor: $F(\vec{x}) \approx F(\vec{x}) + DF(\vec{x})(\vec{x} - \vec{x}_k) =:
F(\vec{x}_k) + DF(\vec{x}_k) \vec{s}$

Minimiere also stattdessen $\Norm{F(\vec{x})} \approx \Norm{F(\vec{x}_k) + DF(\vec{x}_k) \vec{s}}
= \Norm{DF(\vec{x}_k) \vec{s} - (- F(\vec{x}_k))}$

Lineares Ausgleichsproblem lösen

Vorgehen:
Iteratives Verfahren mit Iterationsvorschrift:

- $\vec{s}_k =$ Lösung des linearen Ausgleichsproblems

$\vec{s} = \text{argmin}_{\vec{s}} \Norm{DF(\vec{x}_k) - (- F(\vec{x}))}$

- $\vec{x}_{k+1} = \vec{x}_k + \vec{s}_k$





