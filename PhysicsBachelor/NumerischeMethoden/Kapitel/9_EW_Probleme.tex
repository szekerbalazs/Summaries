\section{Eigenwertprobleme}

Code: w,V = scipy.linalg.eig(A)

eigh für hermitische Matrizen

w Vektor mit eigenwerten

V Matrix mit V[:,i] (=i-te Spalte) EV zum EW w[i]

\vspace{1\baselineskip}

\Definition{

    Der \fat{Reyleigh-Quotient} von $x$ ist definiert als
    \begin{align*}
        \rho_A = \frac{x^H A x}{x^H x}
    \end{align*}
}

\vspace{1\baselineskip}

\Bemerkung{

    Wenn $x$ ein Eigenvektor ist, dann ist $A x=\lambda x$ und $\rho_A (x) = \lambda$.
    Wenn $x$ allgemein ist, dann ist
    \begin{align*}
        \rho_A = \text{argmin}_{\alpha} \Norm{Ax-\alpha x}_2
    \end{align*}
}

\pagebreak

\Bemerkung{

    Für $A \in \R^{n \times n}$, $A^T = A$ und $\lambda_1,\dots,\lambda_n \in \R$,
    $q_1,\dots,q_n \in \R^n$ sind orthogonale Eigenvektoren. Wir berechnen die Ableitung
    von $\rho_A: \R^n \rightarrow \R^n$:
    \begin{align*}
        \frac{\partial \rho_A}{\partial x_j} = \frac{1}{x^T x} (Ax-\rho_A (x) x)
    \end{align*}
    sodass $\nabla \rho_A (x) = \frac{2}{\Norm{x}^2} (Ax - \rho_A (x) x)$.
    Somit sind die Eigenwerte von $A$ die Stationärpunkte von $\rho_A$. Der Satz von
    Taylor für $\rho_A$ gibt dann
    \begin{align*}
        \rho_A (x) - \rho_A (q_j) = O(\Norm{x-q_j}^2) \quad \quad \text{ für } x \rightarrow q_j
    \end{align*}
    was bedeutet, dass der Rayleigh-Quotient quadratisch akkurat ist.
}

\vspace{1\baselineskip}

\underline{\fat{Direkte Potenzmethode}}

Ziel: Finde dan Betragsgrössten EW von $A$ und ein EV dazu.

Annahme: $A$ ist diagonalisierbar: $S^{-1} A S = \text{diag} (\lambda_1,\dots,\lambda_n)$
mit 

\hspace{45pt} $\abs{\lambda_1} > \abs{\lambda_1} \dots \geq \abs{\lambda_n}$
($>$ und $\geq$ sind extra so)
\begin{align*}
    \frac{A^k x}{\Norm{A^k x}} \rightarrow S_1
    \quad \text{ für } k \rightarrow \infty
\end{align*}
Idee: notiere $x_k = A^k x$ und berechne $\rho_A (x_k)$:
\begin{align*}
    \rho_A (x_k) = \frac{x_k^H A x_k}{x_k^H x_k}
    = \frac{1}{x_k^H x_k} \klammer{x_k^H \sum_{j=1}^n a_j \lambda_j^{k+1} s_j}
    = \lambda_1 + \mathcal{O} \klammer{\abs{\frac{\lambda_2}{\lambda_1}}^k}
\end{align*}

Code:

Wähle $x_0$ zufällig, $\Norm{x_0} = 1$

für $k=1,2,\dots$

\hspace{12pt} $w=Ax_{k-1}$

\hspace{12pt} $x_k = \frac{w}{\Norm{w}}$

\hspace{12pt} $\lambda_k = x_k^H A x_k$

\vspace{1\baselineskip}

\Theorem{

    Die Potenzmethode liefert eine Iteration die linear gegen $\lambda_1$ konvergiert
    mit der Rate $\abs{\frac{\lambda_2}{\lambda_1}}$
}

\vspace{1\baselineskip}

\Bemerkung{

    Falls $A$ normal $\Rightarrow$ EV orthogonal $\Rightarrow$ Fehler
    $\mathcal{O} \klammer{\abs{\frac{\lambda_2}{\lambda_1}}^{2k}}$
    $\Rightarrow$ quadratische Konvergenz
}

\vspace{1\baselineskip}

\underline{\fat{Inverse Potenzmethode}}

Ziel: Finde den kleinsten EW von $A$

Annahme: $A$ ist invertierbar

$Ax=\lambda x \Rightarrow x = \lambda A^{-1} x \Rightarrow \frac{1}{\lambda} x = A^{-1} x$

$\Rightarrow$ betragskleinste EW $\lambda_n$ $\Rightarrow \frac{1}{\lambda_n} x = A^{-1} x$

$\frac{1}{\lambda_n}$ ist der betragsgrösste EW von $A^{-1}$

\vspace{1\baselineskip}

\Bemerkung{

    Wir berechnen nicht $A^{-1}$ sondern nur einmal eine LU-Zerlegung von $A$
    (strukturerhaltend). Dann löse GLS mit Matrizen L,U um $A^{-1} x$ zu
    implementieren.
}

\vspace{1\baselineskip}

\underline{\fat{Shifted Inverse Iteration}}

Ziel: Finde den EW von $A$ am nächsten bei $\alpha$

$\abs{\alpha - \lambda} = \min \geschwungeneklammer{\abs{\alpha - \mu} \text{ mit $\mu$ EW von $A$}}$

$Ax-\lambda x \Leftrightarrow Ax-\alpha I x = (\lambda - \alpha) x \Leftrightarrow
(A-\alpha I) x = (\lambda-\alpha)x \Leftrightarrow \frac{1}{\lambda - \alpha} x = 
(A-\alpha I)^{-1} x$

$\Rightarrow$ Potenzmethode für $(A-\alpha I)^{-1} \Rightarrow \frac{1}{\lambda-\alpha}
\Rightarrow \lambda$

\pagebreak

\Bemerkung{

    Die Potenzmethode ist schneller wenn $\alpha \approx \lambda_j$

    Idee: wähle $\alpha$ adaptiv, zB. $\alpha = \rho_A (x_{k-1})$ im $k$-ten Schritt.

    $\Rightarrow$ beschleunigte Konvergenz: \fat{Rayleigh-Quotienten-Iteration}
}

\vspace{1\baselineskip}

\Bemerkung{

    Wir brauchen immer einen guten Startwert.

    Konvergenzordnung $3$!
}

\vspace{1\baselineskip}

\Theorem{ (Courant-Fischer)
    \begin{align*}
        \lambda_k = \min_{\dim U = k} \max_{x \in U} \rho_A (x)
        = \max_{\dim U = n-k+1} \min_{x \in U} \rho_A (x)
    \end{align*}
}

\vspace{1\baselineskip}

\underline{\fat{Krylov-Verfahren}}

gut für kleine Matrizen.

\vspace{1\baselineskip}

\Definition{

    Der $l$-te Krylov-Raum ist definiert als:
    \begin{align*}
        \mathcal{K}_{l} (A,z) = \text{span} \geschwungeneklammer{z,Az,\dots,A^{l-1}z}
        = \geschwungeneklammer{p(A) z ; p = \text{ Polynom vom Grad } \leq l-1}
    \end{align*}
}
Finde ONB von Krylov-Raum mittels Gram-Schmidt-Verfahren.

\vspace{1\baselineskip}

\underline{\fat{Arnoldi-Verfahren}}
(Code: Seite 284)

\vspace{1\baselineskip}

\begin{enumerate}[{1)}]
    \item Wähle $l<n$ selber
    \item Sei $\geschwungeneklammer{v_1,\dots,v_{l-1}}$ eine ONB von $K_{l-1} (A,z)$,
            dann:
            
            $\tilde{v}_l = A v_{l-1} - \sum_{j=1}^{l-1} \scalprod{v_j}{A v_{l-1}}$

            $v_l = \frac{\tilde{v}_l}{\Norm{\tilde{v}_l}}$
    \item Eigentlich: (\fat{Vorgehen})
    
            $z$ beliebig

            $v_1 = \frac{z}{\Norm{z}}$

            für $l=1,2,...,k-1$

            \hspace{12pt} $\tilde{v} = A v_l$

            \hspace{12pt} $h_{jl} = v_j^H \tilde{v}$

            \hspace{12pt} $\tilde{v} = \tilde{v} - h_{jl} v_j$

            $h_{l+1,l} = \Norm{\tilde{v}}$

            $v_{l+1} = \frac{\tilde{v}}{h_{l+1,l}}$
\end{enumerate}

\Bemerkung{

    Falls $h_{l+1,l} = 0 \Rightarrow$ Abbruch der Iteration.
    $Av_l \in \mathcal{K}_l (A,z)$
}

\vspace{1\baselineskip}

Daraus definieren wir: $V_l := (v_1 \ | \ \dots \ | \ v_l)$ mit $\geschwungeneklammer{v_1,\dots,v_l}$
ONB von $\mathcal{K}_l$ und Hessenbergmatrizen:
\begin{align*}
    &H_l := \begin{pmatrix}
        h_{1,1} & h_{1,2} & \dots & h_{1,l} \\
        h_{2,1} & \ddots & & \vdots \\
        & \ddots & \ddots & \vdots \\
        0 & & h_{l,l-1} & h_{l,l}
    \end{pmatrix} \in \R^{l \times l}
    \\
    &\tilde{H} := \begin{pmatrix}
        & & & \\
        & & H_l & \\
        & & & \\
        0 & \dots & 0 & h_{l+1,l}
    \end{pmatrix}
    \quad \text{ mit }
    \tilde{h}_{ij} = \begin{cases}
        h_{ij} = v_i^H A v_j \text{ für } i \leq j \\
        \Norm{\tilde{v}_j}_2 \text{ für } i = j+1 \\
        0 \text{ sonst}
    \end{cases}
\end{align*}

\pagebreak

\Bemerkung{
\begin{enumerate}[{1)}]
    \item $V_l^H V_l = i_l$, da $v_1,\dots,v_l$ orthogonal sind.
    \item $H_l = V_l^H A V_l$
    \item Arnolli-Verfahren bricht ab, falls $h_{l+1,l} = 0$ und dann: $AV_l = V_l H_l$ und $\mathcal{K}_{l+1} = 0$
    \item Falls $A$ Hermit-symmetrisch ist ($A^H = A$), dann gilt: $H_l^H = H_l$. Somit ist
            $H_l$ eine tridiagonale Matrix.
\end{enumerate}
}

\vspace{1\baselineskip}

Code: Lanczos-Iteration: Seite 286

\hspace{25pt} Arnoldi EWapproximation: Seite 288

\vspace{1\baselineskip}

\Theorem{ (Falls $V_l$ quadratisch)

    Falls $h_{l+1,l} = 0$ und $h_{j+1,j} \neq 0$, dann:
    \begin{enumerate}[{1)}]
        \item Jeder Eigenwert von $H_l$ ist auh EW von $A$
        \item Falls $A$ regulär ist, dann gibt es $y \in \C^l$, sodass $Ax=b$ mit
                $x=V_l y$, wobei $V_l$ normal für $\mathcal{K}_l (A,b)$ konstruirt wurde.
    \end{enumerate}
}

\Theorem{ (Falls $V_l$ nicht quadratisch)
    \begin{enumerate}[{1)}]
        \item EW von $H_l$ finden, Code: $ew,ev = eig(H_l)$
        \item Achtung! ev $\neq$ EV von $A$. Man kann sie aber berechnen
        \item Sei $u$ ein EV von $H_l$, dann ist der zugehörige EV von $A$: $V_l u$
    \end{enumerate}
}

\vspace{1\baselineskip}

\Theorem{

    Seien $\lambda_1 \geq \lambda_2 \geq \dots \geq \lambda_n$ und
    $\mu_1^{(l)} \geq \dots \geq \mu_l^{(l)}$ die EW der Hermit-symmetrischen
    Matrix $A \in \C^{n \times n}$, bzw von $H_l = V_l^H A V_l$ für
    $l=1,2,\dots$. Dann gelten für $1 \leq j \leq l$ die Ungleichungsketten:
    \begin{align*}
        \lambda_{n-j+1} \leq \mu_{l+1-j+1}^{(l+1)} \leq \mu_{l-j+1}^{(l)}
        \quad \quad \text{ und } \quad \quad
        \mu_j^{(l)} \leq \mu_j^{l+1} \leq \lambda_j
    \end{align*}
}

\vspace{1\baselineskip}

\underline{\fat{Eigenwertprobleme in Zusammenhang mit DGL}}

Approximation der Ableitungsoperatoren durch Differenzialquotienten. Zwei mögliche Varianten
für die erste Ableitung:

Vorwärts: $\frac{df}{dx} (x_i) \approx \frac{f(x_{i+1} - f(x_i))}{h}$

Rückwärts: $\frac{df}{dx} (x_i) \approx \frac{f(x_i) - f(x_{i-1})}{h}$

Mögliche Approximation der zweiten Ableitung

$\frac{d^2 f}{dx^2} (x_i) \approx \frac{f(x_{i+1}) - 2 f(x_i) + f(x_{i+1})}{h^2}$

\vspace{1\baselineskip}

Bsp: $\frac{d^2}{dx^2} \Psi(x) = \lambda \Psi(x)$ mit unbekanntem $\lambda \in \R$

Randbedingung: $\Psi(a) = \Psi(b) = 0$

Idee: Partition $x_0,\dots,x_N$ von $[a,b] \Rightarrow \Psi(x_0) = \Psi(x_N) = 0$

Noch zu bestimmen: $\Psi(x_1),\dots,\Psi(x_{N-1})$ und $\lambda$. Dazu:
\begin{align*}
    \frac{d^2}{dx^2} \begin{pmatrix}
        \Psi(x_1) \\ \vdots \\ \vdots \\ \vdots \\ \Psi(x_{N-1})
    \end{pmatrix}
    \approx \frac{1}{h^2} \begin{pmatrix}
        -2 & 1 & & & \\
        1 & -2 & 1 & & \\
        & \ddots & \ddots & \ddots & \\
        & & 1 & -2 & 1 \\
        & & & 1 & -2
    \end{pmatrix} \begin{pmatrix}
        \Psi(x_1) \\ \vdots \\ \vdots \\ \vdots \\ \Psi(x_{N-1})
    \end{pmatrix}
    = \lambda \begin{pmatrix}
        \Psi(x_1) \\ \vdots \\ \vdots \\ \vdots \\ \Psi(x_{N-1})
    \end{pmatrix}
\end{align*}
Die gesuchten Lösungen $\Psi(x_1),\dots,\Psi(x_{N-1})$ sind somit die EV der Matrix
$A$ und die unbestimmten Werte $\lambda$ dia dazugehörigen EW.

Bem: Dieses $A$ ist dünnbesetzt $\rightarrow$ Krylov für grosses $N$ nützlich.




