\section{Elektrische und magnetische Felder im Materie}

\vspace{1\baselineskip}

\fat{Dielektrika}:

In einem Material gilt für die Kapazität eines Plattenkondensators: $C = \epsilon C_{\text{Vak}}$
mit $\epsilon$ der \fat{Dielektrizitätskonstante}. Falls Ladungs $Q_0$ konstant ist, gilt:
$V = \frac{1}{\epsilon} V_{\text{Vak}}$. Falls die Spannung $V_0$ konstant ist, gilt:
$Q = \epsilon Q_{\text{Vak}}$.

\vspace{1\baselineskip}

Materialien, welche unabhängig von externen Feldern magnetisch sind, heissen \fat{ferromagnetisch}.
Während \fat{paramegnetische} Materialien das vorhandene Magnetfeld verstärken, wird es durch
\fat{diamagnetische} Materialien abgeschwächt.

\vspace{1\baselineskip}

\fat{Nettokraft in einem inhomogenen Feld}:
\begin{align*}
    \vec{F} = - \vec{\nabla} U = \vec{\nabla} (\vec{p} \cdot \vec{E})
    = \klammer{\vec{p} \cdot \vec{\nabla}} \cdot \vec{E}
    = \begin{pmatrix}
        p_x \ \partial_x E_x \\ p_y \ \partial_y E_y \\ p_z \ \partial_z E_z
    \end{pmatrix}
    = \vec{p} \cdot \begin{pmatrix}
        \nabla E_x \\ \nabla E_y \\ \nabla E_z
    \end{pmatrix}
\end{align*}
\begin{align*}
    &F_x = \vec{p} \cdot \nabla E_x \\
    \Longrightarrow \hspace{20pt} &F_y = \vec{p} \cdot \nabla E_y \\
    &F_z = \vec{p} \cdot \nabla E_z
\end{align*}

\vspace{1\baselineskip}

\fat{Die elektrische Flussdichte und das Gauss'sche Gesetz im Dielektrikum}:

\begin{align*}
    \vec{\nabla} \cdot \vec{E} &= \frac{\rho_{\text{ges}}}{\epsilonnull} =
    \frac{\rho_{\text{frei}} + \rho_{\text{geb}}}{\epsilonnull} =
    \frac{1}{\epsilon} \vec{\nabla} \cdot \vec{E}_{\text{vak}} =
    \frac{\rho_{\text{frei}}}{\epsilon \epsilonnull} =
    \epsilon \vec{\nabla} \cdot \vec{E} + \frac{\rho_{\text{geb}}}{\epsilonnull}
    \\ &=
    \frac{- \rho_{\text{geb}}}{\epsilonnull \cdot (\epsilon -1)}
    \\ \\
    \Longrightarrow \vec{P} &= \epsilonnull (\epsilon -1) \vec{E}
    \quad \quad \Longrightarrow \quad
    \rho_{\text{geb}} = - \vec{\nabla} \cdot \vec{P}
\end{align*}
Wir definieren die \fat{elektrische Flussdichte} oder \fat{dielektrische Verschiebung $D$} durch:
\begin{align*}
    \vec{D} = \epsilonnull \vec{E} + \vec{P}
\end{align*}
Damit folgt das \fat{modifizierte Gauss'sche Gesetz}
\begin{align*}
    \vec{\nabla} \cdot \vec{D} = \rho_{\text{frei}}
\end{align*}
und $\vec{D} = \epsilon \epsilonnull \vec{E}$.

\vspace{1\baselineskip}

Die elektrischen Dipole ($\vec{p}$) erzeugen eine Polarisation $\vec{P}$ welche das externe
$E$-Feld im Material schwächen: $\mathcal{E} = \frac{E_{\text{vak}}}{E_{\text{result}}} \geq 1$.

\vspace{1\baselineskip}

Mit der Teilchendichte $N$ und dem magnetischen Dipolmoment $\mu$ definieren wir die
\fat{Magnetisierung} als $\vec{M} = N \cdot \vec{\mu}$.

\vspace{1\baselineskip}

Wir definieren das \fat{$H$-Feld} als $\vec{H} = \frac{\vec{B}}{\munull} - \vec{M}$.
Es gilt:
\begin{align*}
    \vec{\nabla} \times \vec{H} &= \frac{\partial \vec{D}}{\partial t} + \vec{J}_{\text{frei}} \quad \quad \text{(Ampèr'sches Gesetz in Materie)}
    \\
    \vec{B} &= \munull \mu \vec{H}
\end{align*}
