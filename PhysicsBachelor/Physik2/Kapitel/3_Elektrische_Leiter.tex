\section{Elektrische Leiter}

\vspace{1\baselineskip}

\fat{Leiter}:
Einzelnen Ladungen sind sehr mobil. Daher ist das elektrische Feld überall im Leiter Null.
Die Ladungen passen sich also an. Des Weiteren verschwindet die Ladungsdichte überall im
Inneren der Leiters.

\vspace{1\baselineskip}

\fat{Isolator}: Es existieren keine freien Ladungen. Daher können wir das elektrische
Feld und Potential berechnen.

\vspace{1\baselineskip}

\fat{Bedingungen an einen Leiter}:
\begin{itemize}
    \item Das elektrostatische Potential ist überall im Inneren des Leiters auf dessen
            Oberfläche konstant. Insbesondere ist die Oberfläche eine Äquipotentialfläche.
    \item Das elektrische Feld in der Nähe der Oberfläche verschwindet im Inneren und die
            Feldlinien stehen orthogonal auf die Oberfläche.
            $E = \frac{\sigma}{\epsilonnull} \vec{n}$ mit $\sigma =$ lokale Flächenladungsdichte
            auf der Oberfläche und $\vec{n} = $ Normalenvektor.
    \item Die Flächenladungsdichte $\sigma$ ist abhängig von der Form des Leiters.
    \item Die Gesammtladung eines Leiters ist gegeben durch die Integration der Flächenladungsdichte
            über die Oberfläche
            \begin{align*}
                Q = \int_S \sigma da = \epsilonnull \int_S E da
            \end{align*}
\end{itemize}

\vspace{1\baselineskip}

\fat{Das allgemeine elektrostatische Problem}

Wir betrachten einen Leiter im Vakuum, sodass die \fat{Laplacegleichung} zu
$\delta \phi = 0$ wird. Es gibt verschiedene Randbedingungen, die man betrachten kann:
\begin{enumerate}
    \item \fat{Dirichlet-Randbedingung}: Das Potential $\phi$ ist für alle Leiter definiert.
    \item \fat{Neumann-Randbedingung}: Die Ladung $Q$ ist für alle Leiter definiert.
    \item Eine Mischung der beiden.
\end{enumerate}

\vspace{1\baselineskip}

\fat{Eindeutigkeitssatz}:
Falls für eine gegebene Menge an Randbedingungen eine Lösung des elektrostatischen
Problems existiert, so ist diese eindeutig.

\vspace{1\baselineskip}

\fat{Influenz}
Verschiebung von Ladungen in einem Leiter durch eine externe Ladungsverteilung.

\vspace{1\baselineskip}

\fat{Faraday'sche Käfig}
\begin{itemize}
    \item \fat{Ein leerer von einen Leiter umgebener Raum}: Wir wissen, dass das Potential
            auf der Leiteroberfläche konstant sein muss. Damit ist es konstant auf dem ganzen
            Hohlraum, weil er ja keine Ladung einschliesst. Das dabei entstehende elektrische
            Feld ist Null.
    \item \fat{Eine von einem Leiter umgebene Ladung}: Man stelle sich vor, eine positive
            Punktladung $Q$ sitze im Hohlraum eines Leiters. Wegen des Gauss'schen Gesetzes
            muss die Ladung der inneren Oberfläche des Leiters gerade $-Q$ sein
            (Damit das elektrische Feld im Inneren des Leiters wieder Null wird.)
            Die äussere Oberfläche des Leiters trägt dann ebenfalls wieder eine Ladung $Q$,
            damit die Gesammtladung des leiters verschwindet.
    \item \fat{Allgemein}: Durch einen Leiter, der einen Hohlraum umschliesst, wird das
            Innere von allen äussseren Einflüssen abgeschirmt und die Umgebung wird von
            jeglicher Information über die Bewegung der Ladungen im Inneren abgeschnitten.
\end{itemize}
