\section{Kondensatoren}

\vspace{1\baselineskip}

\fat{Kapazität} $C$: $[C] = 1 \text{F} = 1 \frac{\text{C}}{\text{V}}$

Es gilt: $Q = C \cdot \phi$, wobei $Q$ die Ladung ist, $C$ die Kapazität und $\phi$ das
Potential (oft auch $U$ oder $V$).

\vspace{1\baselineskip}

\fat{Plattenkondensator}:

\vspace{1\baselineskip}

\begin{minipage}{0.2\textwidth}
    \begin{center}
        \fat{Elektrisches Feld}:
        \begin{align*}
            E_{\text{Plattenkondensator}} = \frac{V}{d}
        \end{align*}
    \end{center}    
\end{minipage}
\begin{minipage}{0.2\textwidth}
    \begin{center}
        \fat{Kapazität}:
        \begin{align*}
            C_{\text{Plattenkondensator}} = \frac{A \epsilonnull}{d}
        \end{align*}
    \end{center}    
\end{minipage}

\vspace{1\baselineskip}

\fat{Gespeicherte Energie}:
\begin{align*}
    dW = \phi dQ = \frac{Q}{C} dQ
    \ \Rightarrow \
    W = \frac{Q^2}{2 C} = \frac{C V^2}{2} = \frac{Q V}{2}
\end{align*}

\vspace{1\baselineskip}

\fat{Ersatzkapazitäten}:
\begin{itemize}
    \item parallel: $C = \sum C_i$
    \item seriell: $\frac{1}{C} = \sum \frac{1}{C_i}$
\end{itemize}
\underline{Bemerkung}: Werden Kondensatoren einfach verbunden, ohne dass es eine
Spannungsquelle gibt, so gelten die Formeln für die Parallelschaltung, da einer der
Kondensatoren als Spannungsquelle fungiert, der andere als Kondensator.
