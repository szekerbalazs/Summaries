\section{Elektrische Ströme}

\vspace{1\baselineskip}

Ein \fat{Elektrischer Strom} entsteht, wenn sich eine Nettoladung in Bewegung befindet.

\vspace{1\baselineskip}

\fat{Stromstärke}:
\[
    I = \frac{dQ}{dt} = \dot{Q} = n A v q
\]
$I$ bezeichnet die Ladung, welche pro Zeit durch einen Leiter fliesst.

\vspace{1\baselineskip}

\fat{Anzahl Ladungsträger}:
\begin{align*}
    \Delta N = n A \Delta x = n A v \Delta t
\end{align*}
$n$ bezeichnet Anzahldichte in $\frac{1}{m^3}$, $A$ Oberfläche des Leiters, $\Delta x$
dessen Länge. Weiter ist $v$ die Driftgeschwindigkeit.

\vspace{1\baselineskip}

\fat{Stromrichtung}
\begin{itemize}
    \item Die \textit{physikalische} Stromrichtung ist die Bewegungsrichtung der Elektronen.
    \item Die \textit{technische / konventionelle} Stromrichtung ist diejenige der positiven Ladungsträger.
\end{itemize}

\pagebreak

\fat{Stromdichte}: $\vec{J} = n q \vec{v}$ \
Allgemeiner:
\begin{align*}
    \vec{J} = \sum_i n_i q_i \vec{v}_i
    \quad \quad \text{  und  } \quad \quad
    I_A = \int \vec{J} \cdot d \vec{A}
\end{align*}
Die Stromdichte gibt uns an, wie viel Strom pro Oberfläche durch den Leiter fliesst.

\vspace{1\baselineskip}

\fat{Ladugserhaltung}:
\begin{align*}
    I = \int_{\partial V} \vec{J} \cdot d \vec{A} = - \int_V \frac{d \rho}{d t} dV
\end{align*}
\fat{Kondinuitätsgleichung}:
\begin{align*}
    \vec{\nabla} \cdot \vec{J} = - \frac{d \rho}{d t}
\end{align*}

\vspace{1\baselineskip}

\fat{Ohm'sches Gesetz}:
\begin{itemize}
    \item \fat{mikroskopisch}: $\vec{J} = \sigma \vec{E}$ mit $\sigma$ der Leitfähigkeit des Materials.
    \item \fat{idealisiert}: $U = R \cdot I$ mit $U$ der Spannung, $R$ dem Widerstand und $I$ dem Strom.
\end{itemize}
