\section{Spezielle Relativitätstheorie}

\vspace{1\baselineskip}

\fat{Einstein'sche Postulate}:
\begin{enumerate}
    \item Absolute, gleichförmige Bewegung kann man nicht messen.
    \item Die Geschwindigkeit des Lichts ist unabhängig von Bewegungszustand der Lichtquelle
            Folglich misst jeder Beobachter die gleiche Lichtgeschwindigkeit.
\end{enumerate}

\vspace{1\baselineskip}

\fat{Galilei-Transformation}

Wir verwenden zwei Bezugssysteme $S_A$ und $S_B$ mit den jeweiligen Koordinaten
$x^A , y^A , z^A$ resp. $x^B , y^B , z^B$. $S_B$ bewege sich entlang der positiven
$x$-Achse mit $v_B^A$ relativ zu $S_A$. $S_A$ bewegt sich demnach mit $-v_B^A$ relativ zu
$S_B$. Falls nun die Ursprünge zusammenfallen, so erhalten wir die klassischen Beziehungen:

\begin{minipage}{0.25\textwidth}
    \begin{align*}
        x^A &= x^B + v_B^A t^B \\
        y^A &= y^B \\
        z^A &= z^B \\
        t^A &= t^B
    \end{align*}
\end{minipage}
\begin{minipage}{0.25\textwidth}
    \begin{align*}
        x^B &= x^A - v_B^A t^A \\
        y^B &= y^A \\
        z^B &= z^A \\
        t^B &= t^A
    \end{align*}
\end{minipage}

\vspace{1\baselineskip}

Diese Transformationen verstossen aber gegen die Einstein'schen Postulate.

\pagebreak

\fat{Lorentztransformation}

Die relativistische Transformationsvorschrift (entlang der $x$-Achste) lautet:
\begin{align*}
    x^A &= \gamma (x^B + v_B^{A} t^B) \\
    y^{A} &= y^B \\
    z^{A} &= z^B \\
    t^{A} &= \gamma (t^B + \frac{v_B^{A} x^B}{c^2})
\end{align*}
Dabei ist $\gamma = \frac{1}{\sqrt{1- \beta^2}} > 1$ mit $\beta = \frac{v_B^{A}}{c}  =
\frac{-v_A^B}{c}$

In Matrizenschreibweise heisst das für ein ruhendes System $S$ und ein bewegtes
(in $x$-Richtung) System $S'$:
\begin{align*}
    \begin{pmatrix}
        ct' \\ x' \\ y' \\ z'
    \end{pmatrix} = \begin{pmatrix}
        \gamma (ct - \beta x) \\ \gamma (x- \beta c t) \\ y \\ z
    \end{pmatrix} = \begin{pmatrix}
        \gamma & - \beta \gamma & 0 & 0 \\
        - \beta \gamma & \gamma & 0 & 0 \\
        0 & 0 & 1 & 0 \\
        0 & 0 & 0 & 1
    \end{pmatrix} \begin{pmatrix}
        ct \\ x \\ y \\ z
    \end{pmatrix}
\end{align*}

\vspace{1\baselineskip}

\fat{Zeitdilatation}:
\begin{align*}
    \Delta t^A = \gamma \Delta t^B
    \quad \quad \Longrightarrow \quad \quad
    \Delta t = \gamma \Delta t_{\text{eigen}}
\end{align*}
Das Zeitintervall $\Delta t$ in einem beliebigen Bezugssystem ist stets grösser als die
Eigenzeit.

\vspace{1\baselineskip}

\fat{Längenkontraktion}:
\begin{align*}
    \Delta x^A = \frac{1}{\gamma} \Delta x^B
    \quad \quad \Longrightarrow \quad \quad
    \Delta x = \frac{1}{\gamma} \Delta x_{\text{eigen}}
\end{align*}
Die Länge $l$ in einem beliebigen Bezugssystem ist stets kleiner als die Eigenlänge $l_0$.

\vspace{1\baselineskip}

\fat{Gleichzeitigkeit}

Zwei Ereignisse finden gleichzeitig statt, wenn die von den Ereignissen ausgesandten
Lichtsignale einen Beobachter, der sich in der Mitte der Ereignisse befindet, zus selben
Zeit erreichen.

\vspace{1\baselineskip}

\fat{Invarianz}

Wir definieren das \fat{Raumzeit-Intervall $\Delta s$} als:
\begin{align*}
    \Delta s^2 = c^2 \Delta t^2 - \Delta x^2 - \Delta y^2 - \Delta z^2
\end{align*}
welches eine Invariante der Lorentztransformation ist.
\begin{itemize}
    \item \fat{Zeitartig}: $\Delta s^2 > 0$: Zeitintervalle am kleinsten, wo Ereignisse
            am selben Ort stattfinden
    \item \fat{Raumartig}: $\Delta s^2 < 0$: Länge am kleinsten, wo Raumkoordinate zur
            selben Zeit gemessen
    \item \fat{Lichtkegel}: $\Delta s^2 = 0$: Die Gleichung beschreibt die Ausbreitung
            des Lichtes als Kugelwelle.
\end{itemize}

\vspace{1\baselineskip}

\fat{Geschwindigkeitstransformationen}

Es sei ein Teilchen in $S_B$ mit der Geschwindigkeit $\vec{v} = (v_x , v_y , v_z)^T$.
Die Geschwindigkeit in $S_A$ ist:
\begin{align*}
    v_x^{A} &= \frac{dx^{A}}{dt^{A}} = \frac{\gamma (dx^B + v_B^{A} dt^B)}{\gamma (dt^B + \frac{v_B^{A}}{c^2} dx^B)}
        = \frac{v_x^B + v_B^{A}}{1+\frac{v_B^{A} v_x}{c^2}}
    \\
    v_y^{A} &= \frac{dy^{A}}{dt^{A}} = \frac{v_y^B}{\gamma (1+ \frac{v_B^{A} v_x^B}{c^2})}
    \\
    v_z^{A} &= \frac{dz^{A}}{dt^{A}} = \frac{v_z^B}{\gamma (1+ \frac{v_B^{A} v_x^B}{c^2})}
\end{align*}
wobei $v_B^{A}$ die Geschw. von $B$ im Bezugssystem von $A$ ist.

\vspace{1\baselineskip}

\fat{Relativistischer Dopplereffekt}

Der (longitudale) relativistische Dopplereffekt ist gegeben durch:
\begin{align*}
    \nu^{A} &= \sqrt{\frac{1+\beta}{1-\beta}} \nu^{B}
    \quad \quad \text{für kleiner werdenden Abstand}
    \\
    \nu^{A} &= \sqrt{\frac{1-\beta}{1+\beta}} \nu^{B}
    \quad \quad \text{für grösser werdenden Abstand}
\end{align*}
Der transversale Dopplereffekt ist geg. durch $\nu^{A} = \sqrt{1-\beta^2} \nu^B$

\vspace{1\baselineskip}

\fat{Relativistischer Impuls und relativistische Energie}
\begin{itemize}
    \item \fat{Impuls}: $\vec{p} = \gamma m \vec{v}$
    \item \fat{Energie}: $\Etot = \Ekin + E_0 = \gamma m c^2$ mit $E_0 = mc^2$ der Ruheenergie.
\end{itemize}
Die Masse eines Teilchens ist Lorentz-invariant und es gilt:
\begin{align*}
    E^2 - \vec{p}^2 c^2 = m^2 c^4
\end{align*}

\vspace{1\baselineskip}

\fat{Relativistische Kraft}
\begin{align*}
    m \gamma \vec{v} &= \vec{F} - \frac{1}{c^2} (\vec{F} \cdot \vec{v}) \cdot \vec{v}
    \\
    m \gamma^3 \vec{a} &= (\vec{F} \cdot \hat{v}) \cdot \hat{v}
\end{align*}

\vspace{1\baselineskip}

\fat{Vierer-Impuls}

Energie-Impuls Vektor in $4$er-Koordinaten
\begin{align*}
    P^{\nu} = \begin{pmatrix}
        \nicefrac{\Etot}{c} \\ p_1 \\ p_2 \\ p_3
    \end{pmatrix} = \begin{pmatrix}
        \Etot \\ \vec{p}
    \end{pmatrix}
\end{align*}
\begin{tcolorbox}
    Für einen Vierer-Vektor gilt allgemein:    
    \begin{align*}
        x^\mu = \begin{pmatrix}
        x^0 \\ x^1 \\ x^2\\ x^3
        \end{pmatrix} = \begin{pmatrix}
        x^0 \\ \Vec{x}
        \end{pmatrix}
        \quad \quad \quad
        x_\mu = \begin{pmatrix}
        x_0 \\ x_1 \\ x_2\\ x_3
        \end{pmatrix} = \begin{pmatrix}
        x^0 \\ -x^1 \\ -x^2\\ -x^3
        \end{pmatrix} = \begin{pmatrix}
        x^0 \\ -\Vec{x}
        \end{pmatrix}
    \end{align*}    
    Für das 4er-Skalarprodukt gilt    
    \begin{equation*}
        x^\mu y_\mu = \begin{pmatrix}
        x^0 \\ x^1 \\ x^2\\ x^3
        \end{pmatrix} \cdot \begin{pmatrix}
        y^0 \\ -y^1 \\ -y^2\\ -y^3
        \end{pmatrix} = x^0y^0 - \Vec{x}\cdot \Vec{y}
    \end{equation*}
    Wir nennen $x^{\mu}$ kontravariant und $x_{\mu}$ kovariant.
\end{tcolorbox}
