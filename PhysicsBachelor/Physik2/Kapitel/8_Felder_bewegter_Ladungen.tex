\section{Felder bewegter Ladungen}

\vspace{1\baselineskip}

\fat{Transformationen von elektrischen Feldern an einem Plattenkondensator}
\begin{itemize}
    \item \fat{Elektrisches Feld, senkrecht}: $E'_{\perp} = \frac{\sigma'}{\epsilonnull} =
            \frac{\gamma \sigma}{\epsilonnull} = \gamma E_{\perp}$
    \item \fat{Elektrisches Feld, parallel}: $E'_{\parallel} = \frac{\sigma}{\epsilonnull}
            = E_{\parallel}$
\end{itemize}

\pagebreak

\fat{Transformation von elektrischen Felder einer bewegten Punktladung}

Angenommen wir haben eine Punktladung, die sich mit $v_x$ entlang der $x$-Achse bewegt.
Das elektrische Feld sei in der $xz$-Ebene, also parallel, bzw senkrecht zur Bewegung.
Im mitbewegten Inertialsystem, in dem sich die Ladung im Ursprung und in Ruhe befindet
sind die Komponenten des elektrischen Feldes:
\begin{align*}
    E_x (x,0,z) = \frac{q}{4 \pi \epsilonnull} \frac{x}{(x^2 + z^2)^{\nicefrac{3}{2}}}
    \\
    E_z (x,0,z) = \frac{q}{4 \pi \epsilonnull} \frac{z}{(x^2 + z^2)^{\nicefrac{3}{2}}}
\end{align*}
Mit $r'^2 = x'^2 + z'^2$ und $z' = r' \sin(\theta')$ erhalten wir:
\begin{align*}
    E' = \sqrt{{E'}_x^2 + {E'}_z^2} = \frac{q}{4 \pi \epsilonnull r'^2} \frac{1-\beta^2}{\klammer{1-\beta^2 \sin^2(\theta')}^{\nicefrac{3}{2}}}
\end{align*}
wobei die Striche die Koordinaten im Laborsystem kennzeichnen, welches sich mit
Relativgeschwindigkeit $-v_x$ bewegt.

\vspace{1\baselineskip}

\fat{Kräfte auf bewegte Ladungen}:
\begin{align*}
    F_{\parallel} = F'_{\parallel} = q E'_{\parallel} = q E_{\parallel}
    \quad \quad \quad \quad
    F_{\perp} = \frac{1}{\gamma} F'{\perp} = \frac{1}{\gamma} q E'_{\perp} = q E_{\perp}
\end{align*}
