\section{Atoms}

\paragraph{Elements}
${^A_Z}G$: $A$ is the mass of the atom, $G$ is the symbol of the element and $Z$
is the atomic number (number of protons).

\paragraph{Mass spectrometer}
$\vec{E} = E \vec{e}_y$ , $\vec{B} = B \vec{e}_y$ , $\vec{v} = v \vec{e}_z$ ,
$\vec{F} = q \klammer{\vec{E} + \vec{v} \times \vec{B}}$. Thus
$m \frac{d^2 y}{d t^2} = q E \Rightarrow y = \frac{q E t^2}{2 m}$ and
$m \frac{d^2 x}{d t^2} = - q B v \Rightarrow x = - \frac{q B v t^2}{2 m}$.
Since $t = l/v$ we obtain $y = \frac{m}{q} \frac{2 E x^2}{B^2 l^2}$. Where $l$ is
the length of the condensator (interaction length). This is a parabola. Depending
on the ratio $m/q$ the particle will land somewhere else.

\paragraph{Scattering}
Assume we scatter atoms with radius $r_1$ and $r_2$, then they can't get
closer than $r_1 + r_2$. Then the \fat{scattering cross section} is
$\sigma = \pi (r_1 + r_2)^2$. The \fat{flux} is defines as
$F = \frac{\Delta N_p}{A \Delta t}$. Rate of hitting objects is
$R = F \sigma$. Using a \textit{collection} of atoms as target:
probability of two atoms lining up is small, rate of interactions is then
$R = F \sigma N_A$ with $N_A$ the number of atoms interacting with the beam.
Assume atoms have a uniform density $n$ within the volume. Probability that
a single incident atom will get scattered: $P_S = n \sigma d x$.
Change in number of atoms in the beam is $d N = - N \sigma n dx$. After a
distance $x$, the number of remaining particles is $N(x) = N(0) e^{-n \sigma x}$.
Number of particles over a time $T$ through the entire cylinder is then
$N(L) = N(0) e^{- \sigma n L}$, thus: $\sigma = \frac{\ln \klammer{\frac{N(p)}{N(L)}}}{n L}$.

\paragraph{Rutherford Scattering}
Shooting $\alpha$ particles at atom. Coulomb force acting:
$\vec{F} = \frac{k}{r^2} \hat{r}$ with $k = \frac{2 Z e^2}{4 \pi \epsilon_0}$
with charge electron charge $e$ and atomic number $Z$. Since force is central,
angular momentum needs to be conserved, resulting in $m v_0 b = m \dot{\phi} r^2
\Rightarrow \frac{1}{r^2} = \frac{\dot{\phi}}{v_0 b} \Rightarrow \vec{F} =
\frac{k \dot{\phi}}{v_0 b} \hat{r} \Rightarrow F_{\perp} = \frac{k \dot{\phi}}{v_0 b}
\sin(\phi) \Rightarrow m \dot{v}_{\perp} = \frac{k \dot{\phi}}{v_0 b} \sin(\phi)$
Integrate this over a long period of time, $t_1$ long before interaction,
$t_2$ long after interaction. At $t_1$: $v_\perp = 0$. At $t_2$:
$v_\perp (t_2) = v_0 \sin(\theta)$, $\phi = \phi_2 = \pi - \theta$, so:
$\int_{t_1}^{t_2} \frac{d v_\perp}{dt} dt = \int_{t_1}^{t_2} \frac{k}{m v_0 b} \sin(\phi) \frac{d \phi}{dt} dt$
resulting in $\int_0^{v_0 \sin(\theta)} d v_\perp = \int_0^{\pi-\theta} \frac{k \sin(\phi)}{m v_0 b} d \phi$
resulting in $v_0 \sin(\theta) = \frac{k}{m v_0 b} \klammer{1 + \cos(\theta)}$
meaning $b = \frac{k}{m v_0^2} \cot \klammer{\frac{\theta}{2}}$.
Now the goal is to find the distribution of particles over deflection angle.
If we have a flux of particles $F = R/A$, then $d R = F \cdot 2 \pi b \ db
= \pi F \klammer{\frac{k}{m v_0^2}}^2 \frac{\cos(\theta/2)}{\sin^3(\theta/2)} d \theta$.
The differential \fat{solid angle} is defined as $d \Omega = d A_{det} / r_{det}^2$,
with $d A_{det}$ the projection area of the detector and $r_{det}$ the distance
to the detector. $d \Omega = \sin(\theta) d \theta d \gamma$ with $d \gamma$ the
azimuthal angle interval giving the size along the direction orthogonal to $d \theta$.
So we obtain $d R = \frac{F}{4} \klammer{\frac{k}{m v_0^2}}^2 \frac{1}{\sin^4(\theta/2)} d \Omega$.
Differential cross section: $d R = F d \sigma \Rightarrow d \sigma =
\frac{1}{4} \klammer{\frac{k}{m v_0^2}}^2 \frac{1}{\sin^4(\theta/2)} d \Omega$.
$d \sigma$ is called \textit{Rutherford's differential cross section scattering
formula}.

\vspace{1\baselineskip}

\fat{$d \dot{N} = F d \sigma$} with $dN$ the change in the rate of particles
through some differential area $d \sigma$ and $F$ the flux.

\vspace{1\baselineskip}

$b = \frac{k}{m v_0^2} \cot(\theta/2)$ for $ \sim 1 \mathrm{fm} \ll b \ll r_a$

\vspace{1\baselineskip}

$d \Omega = 2 \pi \sin(\theta) d \theta$
