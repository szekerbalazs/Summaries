\section{Quantum Mechanics}

\paragraph{Heisenberg Uncertainty Principle}
% If $\Delta p_x$ and $\Delta x$ are standard deviations of the momentum and position,
% then $\Delta p_x \Delta x \geq \frac{h}{2}$. This means, position and momentum
% uncertainties of a particle along one axis are never known simultaneously to
% arbitrary precision.
$\Delta p_x \Delta x \geq \frac{h}{2}$

% \subsection{Wavefunction}
% The state of every quantum system can be completely described by a complex valued
% function $\psi(x,t)$ called the wavefunction. The probability of finding a particle
% between $x$ and $x+dx$ is $P(x,t) dx = \abs{\psi(x,t)}^2 dx = \overline{\psi(x,t)}
% \psi(x,t) dx$. Normalization $\intii \abs{\psi(x,t)}^2 dx = 1$.

% \subsection{Basis Sets of Functions}

% \paragraph{Fourier Series}
% $\psi(x) = \sum_{n=-\infty}^\infty \psi_n e^{i 2 \pi n x / L}$ with
% $x \in (-L/2,L/2)$ and $\psi_n = \frac{1}{L} \int_{-L/2}^{L/2} \psi(x) e^{-2\pi i n x / L} dx$

% \paragraph{Discrete Bases}
% Suppose we have a set of orthogonal functions $\phi_n(x)$. Meaning:
% $\int_D \overline{\phi_m(x)} \phi_n(x) dx = \delta_{mn}$ with $D$ the domain
% of the set of functions. Such a set is called a \fat{complete orthonormal basis}
% if any function $\psi(x)$ defined on $D$ can be written as
% $\psi(x) = \sum_n \psi_n \phi_n(x)$ with $\psi_n = \int_D \overline{\phi_n(x)}
% \psi(x) dx$ inserting back yields $\sum_n \overline{\phi_n(x')} \phi_n(x) = \delta(x-x')$.
% This is called the \fat{completeness condition} and must be sstisfied for the $\phi_n(x)$
% in order to be a complete basis.
% Properties of the delta function: $\delta(x) = 0 \ \forall x \neq 0$, $\int_{-a}^a
% \delta(x) dx = 1$ for any $a>0$, $\delta(ax) = \frac{1}{\abs{a}} \delta(x) \
% \forall a \in \R$ and $\int \delta(x-x') g(x') dx' = g(x)$.

% \paragraph{Continuous Bases}
% Continuous variable $\alpha \in \R$, basis vectors $\phi(\alpha,x)$. Orthogonality:
% $\int \overline{\phi(\alpha,x)} \phi(\beta,x) dx = \delta(\alpha-\beta)$.
% Completeness condition: $\int \overline{\phi(\alpha,x)} \phi(\alpha,x) d \alpha 
% = \delta(x-x')$. Then $\psi(x) = \int \tilde{\psi}(\alpha) \phi(\alpha,x) d \alpha$
% with $\tilde{\psi}(\alpha) = \int \overline{\phi(\alpha,x)} \psi(x) dx$.

\subsection{Operators}
\fat{Observables}: observable property of a system. Observables are represented by
\fat{operators}. Suppose we have a wavefunction $\psi(x,t) = A e^{i(kx-\omega t)}$
with $p = \hbar k$ and $E = \hbar \omega$. The \fat{Eigenstate} is the physical
state represented by the eigenfunction.
\begin{enumerate}[]
    \item \fat{Position Oberator}: $\hat{x} = x$ or $-i \hbar \frac{\partial}{\partial p}$
    \item \fat{Momentum operator}: $\hat{p} = -i \hbar \frac{\partial}{\partial x}$ of $p$
    \item \fat{Energy operator}: $\hat{E} = i \hbar \frac{\partial}{\partial t}$
\end{enumerate}
% All physical observables are associated with Hermetian operators. The eigenvalue
% of these operators give the result of all possible measurements of the
% observable.

\subsection{Operator Representation}
% Any wavefunction can be expressed as the sum of eigenfunctionf of an observable
% operator. If in addition the eigenvalue of all the eigenfunctions are distinct,
% then it is possible to represent the wavefunction as a function of the eigenvalues
% of the operator. Eigenfunctions of position operator are $\delta(x-\alpha)$
% with eigenvalue $\alpha$. Thus, the wavefunction is representable as
% $\psi(\alpha,t) = \int \delta(x-\alpha) \psi(x,t) dx$ and hence $\psi(x,t)$ is
% the position operator representation of the wavefunction.

\paragraph{Momentum Basis}
We can write wave functions $\psi(x,t)$ in the momentum basis $\phi_p(x) =
\frac{1}{\sqrt{2 \pi \hbar}} e^{i p x / \hbar}$ with the following relation.
$\tilde{\psi}(p,t) = \frac{1}{\sqrt{2 \pi \hbar}} \int \psi(x,t) e^{-i p x / \hbar} dx$.
% Probability of measuring the momentum of the system between $p$ and $p+dp$:
% $P(p) dp = \abs{\tilde{\psi}(p,t)}^2 dp$.
Backtransformation:
$\psi(x,t) = \frac{1}{\sqrt{2 \pi \hbar}} \int \psi(p,t) e^{i p x / \hbar} dp$.

% \paragraph{Spectral Theorem}
% The eigenfunction of an operator $\hat{O}$ representing an observable form a
% complete, orthonormal basis.

\subsection{Hilbert Space and Dirac Notation}
% Column vector: $| S(t) \rangle$, row vector: $\langle S(t) |$.
% We find the representation of a state vector in a particular basis if we take the
% inner product with the basis states. In \fat{Dirac notation}:
% $\psi(x,t) = \langle x | S(t) \rangle$ where $\langle x |$ are the eigenstates
% of the position operator $\hat{x}$. We call $\langle \ | \ \rangle$ a
% \fat{bracket}, $\langle \ |$ the bra and $| \ \rangle$ the ket. Similarly
% $\tilde{\psi}(p,t) = \langle p | S(t) \rangle$. We can write $| S(t) \rangle$
% in different ways:
% \begin{align*}
%     | S(t) \rangle &= \int \psi(x,t) | x \rangle dx
%         = \int \psi(x,t) \delta(x-y) dx
%     \\
%     &= \int \tilde{\psi}(p,t) | p \rangle dp
%         = \int \tilde{\psi}(p,t) \frac{e^{i p x / \hbar}}{\sqrt{2 \pi \hbar}} dp
%     \\
%     &= \sum_n c_n (t) | \phi_n \rangle 
%         = \sum_n c_n (t) \phi_n (x)
%     \\
%     c_n (t) &= \langle e_n | S(t) \rangle 
%         = \intii \overline{\phi_n(x)} \psi(x,t) dx
% \end{align*}
% Orthogonality condition: $\langle \phi_n | \phi_m \rangle = \delta_{mn}$.
% Operators can be written as $\hat{O} = | \alpha \rangle \langle \beta |$.
\fat{Identity operator}: For a discrete basis: $\hat{I} = \sum_n | \phi_n \rangle \langle \phi_n |$,
for a continuous basis: $\int | \phi(\alpha) \rangle \langle \phi(\alpha) | d \alpha = 1$.
For continuous basis: $\langle \phi(\alpha) | \phi(\beta) \rangle = \delta (\alpha - \beta)$.

\subsection{Quantum Measurements}

% \paragraph{Spectral Theorem}
% The eigenstates of an observable operator $\hat{O}$ form a complete basis
% $\geschwungeneklammer{| o_n \rangle}$ (for discrete bases) or
% $\geschwungeneklammer{| o(\alpha) \rangle}$ (for a continuous basis).
% So any state $| \psi \rangle$ can be written as
% $| \psi \rangle = \sum_n c_n (t) | o_n \rangle$ or
% $| \psi \rangle = \int \psi(\alpha,t) | o(\alpha) \rangle d \alpha$

% \vspace{1\baselineskip}

If we make a measurement of the observable, we will get one eigenvalue as a result.
For a discrete basis, the probability of getting the result $o_n$ is
$\abs{\langle o_n | \psi \rangle}^2 = \abs{c_n}^2$.
After getting this result, the state of the system becomes $| o_n \rangle$.
This is also called \fat{collapse} of \fat{projection} of $| \psi \rangle$
into $| o_n \rangle$.
For a continuous basis, the probability of getting a result between $o(\alpha)$
and $o(\alpha + d \alpha)$ is $\abs{\langle o(\alpha) | \psi \rangle}^2 d \alpha
= \abs{\psi(\alpha,t)}^2 d \alpha$

\subsection{Operator Commutation}
If two different operators share the same eigenstate, then particles in those states
have simultanously definite values for both of those observables.
Suppose we have two observables that share the same eigenstate $\psi$, then
$\hat{O} \psi = O \psi$ and $\hat{Q} \psi = Q \psi$ with $O$ and $Q$ the corresponding
eigenvalues. We define the \fat{commutator} $\eckigeklammer{\hat{Q},\hat{O}} \equiv
\hat{Q} \hat{O} - \hat{O} \hat{Q}$. \textit{Two observable operators can share the
same eigenstates if and only if $\eckigeklammer{\hat{Q},\hat{O}} =
\eckigeklammer{\hat{O},\hat{Q}} = 0$.}
The \fat{anticommutator} is defined as
$\geschwungeneklammer{\hat{O},\hat{Q}} = \hat{O} \hat{Q} + \hat{Q} \hat{O}$

\paragraph{Identities}
$\eckigeklammer{\hat{A},c} = 0$ , $\eckigeklammer{\hat{A},\hat{A}} = 0$ ,
$\eckigeklammer{\hat{A},\hat{B}} = - \eckigeklammer{\hat{B},\hat{A}}$ ,
$\eckigeklammer{\hat{A},\sum_n c_n \hat{A}^n} = 0$ ,
$\eckigeklammer{c \hat{A},\hat{B}} = \eckigeklammer{\hat{A},c B} = c \eckigeklammer{\hat{A},\hat{B}}$ ,
$\eckigeklammer{\hat{A},\hat{B} \pm \hat{C}} = \eckigeklammer{\hat{A},\hat{B}} \pm \eckigeklammer{\hat{A},\hat{C}}$ ,
$\eckigeklammer{\hat{A} \hat{B} , \hat{C}} = \hat{A} \eckigeklammer{\hat{B},\hat{C}} + \eckigeklammer{\hat{A},\hat{C}} \hat{B}$ ,
$\eckigeklammer{\hat{A} , \hat{B} \hat{C}} = \hat{B} \eckigeklammer{\hat{A},\hat{C}} + \eckigeklammer{\hat{A},\hat{B}} \hat{C}$

\paragraph{Important commutators}
$\eckigeklammer{\hat{x},\hat{p}} = i \hbar$, in 3D: $\eckigeklammer{x_i,p_j} = i \hbar \delta_{ij}$
and $\eckigeklammer{x_i,x_j} = 0$

\subsection{Measurement for degenerate observables}
\fat{Degeneracy} is the property that  an observable operator has orthonormal
eigenfunctions with the same eigenstate.

Suppose we have a wavefunction $\psi(x)$ that describes a quantum state. This
wavefunction can be decomposed into a basis of orthonormal eigenfunctions $\phi_n(x)$
of an observables $\hat{O}$ each with eigenvalues $O_n$.
$\psi(x) = \sum_n a_n \phi_n (x)$ where $a_n$ are the coefficients of the expansion.
\begin{enumerate}
    \item The probability of a measurement of $\hat{O}$ yielding a value $O_m$ is
        $P(O_m) = \sum_j \abs{a_j}^2$ where the sum runs over all values of $j$
        such that $O_m = O_j$.    
    \item Immediately after a measurement resulting in $O_m$, the wavefunction
        collapses to $\psi'(x) = \frac{\sum_j a_j \phi_j (x)}{\sqrt{\sum_j \abs{a_j}^2}}$
        where again the sum runs over all values of $j$ such that $O_m = O_j$.
\end{enumerate}

\paragraph{Expectation value}
The expectation value for an observable $\hat{O}$ for a state $| \psi \rangle$
is given by
$\langle \hat{O} \rangle = \langle \psi | \hat{O} | \psi \rangle
= \int \overline{\psi(x,t)} \hat{O} \psi(x,t) dx$


\subsection{Schrödinger Equation in 1D}
Full Schrödinger equation in one spatial dimension is
\begin{align*}
    \hat{H} \psi(x,t)
    = \klammer{\frac{\hat{p}^2}{2m} + V (\hat{x})} \psi(x,t)
    = i \hbar \frac{\partial}{\partial t} \psi(x,t)
\end{align*}
With $V(\hat{x})$ the potential energy and $\hat{H}$ the hamilton operator.
The time-\underline{indepentent} Schrödinger equation is
\begin{align*}
    \hat{H} \phi_E (x) = E \phi_E (x)
\end{align*}
This is just the eigenvalue equation of $\hat{H}$. If $\psi(x,0) = \phi_E (x)$,
then we have the following solution to the full Schrödinger equation.
\begin{align*}
    \psi(x,t) = e^{-i \frac{E}{\hbar} t} \phi_E (x)
\end{align*}
For a system in arbitrary initial states, we can always express that initial state
as a superposition of energy eigenstates. We can solve the S.E. for each of them
separately and so we obtain a formula for the evolution of an arbitrary state.
\begin{align*}
    \psi(x,t) = \sum_n c_n e^{-i \frac{E_n}{\hbar} t} \phi_n(x)
\end{align*}

\paragraph{Phase velocity} $v = \frac{\omega}{k}$ velocity at which the phasefronts move.

\paragraph{Group velocity} $v_{group} = \frac{d \omega}{d k}$ velocity of the envelope
of the wavepackets.


\subsection{Examples}

\subsubsection{Free particle in 1-D}
$V(\hat{x}) = 0$, result: $\psi_E(x,t) = A e^{i\klammer{kx-\frac{E}{\hbar} t}} + B e^{-i \klammer{k x + \frac{E}{\hbar} t}}$
with $k = \sqrt{\frac{2 m E}{\hbar^2}}$

\subsubsection{1-D infinite square well}
$V = \begin{cases}
    0 \hspace{5pt} &0 \leq x \leq a
    \\
    \infty \hspace{5pt} &\text{elsewhere}
\end{cases}$
, result: $\psi(x) = A \cos(k x) + B \sin(k x)$ with $k = \sqrt{\frac{2 m E}{\hbar^2}}$

\paragraph{Continuity conditions on the wavefunction}
\begin{enumerate}
    \item The wavefunction $\psi(x)$ is a continuous function of $x$.
    \item The derivative $\frac{d \psi}{d x}$ is also continuous, exept where the
        potential is infinite.
\end{enumerate}
With the first condition we obtain for $\psi$ between $0 \leq x \leq a$:
$\psi(x) = \sqrt{\frac{2}{a}} \sin(k_n a)$ with $k_n = \frac{n \pi}{a}$ and
$n \in \N$. Thus $E_n = \frac{\hbar^2 k_n^2}{2m} = \frac{n^2 \pi^2 \hbar^2}{2 m a^2}$ 

If we shift the coordinate system to $y = x - \frac{a}{2}$ such that it is centered
at $y=0$, we obtain
\begin{align*}
    \psi_n (y) = \sqrt{\frac{2}{a}} \begin{cases}
        \cos \klammer{ \frac{n \pi}{a} y} \hspace{5pt} &\text{for } n=1,3,5,\dots \text{ called \textit{even functions}}
        \\
        - \sin \klammer{\frac{n \pi}{a}} \hspace{5pt} &\text{for } n=2,4,6,\dots \text{ called \textit{odd functions}}
    \end{cases} 
\end{align*}

\paragraph{Symmetry}
Here, in shifted hamiltonian $V(-y) = V(y)$ and kinetic energy also does not change.
Define \fat{parity operator} $\hat{\Pi}$: $\hat{\Pi} \psi(y) = \psi(-y)$. The
parity operator commutes with the hamiltonian, thus $\hat{H}$ and $\hat{\Pi}$
must share a set of orthogonal eigenfunctions $\phi_{\pi}$. The eigenvalues
of $\hat{\Pi}$ are $-1$ and $1$. The eigenvalues $+1$ corresponds to \fat{even functions},
and the eigenvalue $-1$ corresponds to \fat{odd functions}. Hence, since $\hat{H}$
and $\Pi$ commute, the eigenfunctions of $\hat{H}$ can be broken up into two groups:
one group with \fat{odd parity} and one with \fat{even parity}.

\subsubsection{1-D finite square well}
$V(x) = \begin{cases}
    0 \hspace{5pt} &\abs{x} \leq a/2
    \\
    V_0 \hspace{5pt} &\text{elsewhere}
\end{cases}$
, we are looking for solutions with even parity.
We can split the domain up into three regions:
\begin{enumerate}[]
    \item \fat{Region \uproman{1}}: $x < - a/2$, solution:
        $A_{\uproman{1}} e^{-\sqrt{k_0^2 - k^2} x} + B_{\uproman{1}} e^{\sqrt{k_0^2 - k^2} x}$
        with $k_0 = \sqrt{\frac{2 m V_0}{\hbar^2}}$ and $k = \sqrt{\frac{2 m E}{\hbar^2}}$
    \item \fat{Region \uproman{2}}: $-a/2 \leq x \leq a/2$, solution: $A_{\uproman{2}} \cos(k x) + B_{\uproman{2}} \sin(k x)$
        with the same $k$ as above.
    \item \fat{Region \uproman{3}}: $x > a/2$, solution: $A_{\uproman{3}} e^{-\sqrt{k_0^2 - k^2} x} + B_{\uproman{3}} e^{\sqrt{k_0^2 - k^2} x}$
\end{enumerate}

Different solution possibilities:
\begin{enumerate}
    \item \underline{$E < V_0$}
        \begin{enumerate}
            \item \underline{Even solutions}:
                We must have $B_{\uproman{2}} = A_{\uproman{1}} = B_{\uproman{3}} = 0$
                and $B_{\uproman{1}} = A_{\uproman{3}}$. Further restriction:
                $\tan(k a / 2) = \frac{\sqrt{k_0^2 - k^2}}{k}$. Find $A_{\uproman{2}}$ by
                $\intii \abs{\psi_E (x)}^2 dx = 1$.
            \item \underline{Odd solution}
                We must have $A_{\uproman{2}} = 0$, $A_{\uproman{1}} = - B_{\uproman{3}}$
                and $A_{\uproman{3}} = - B_{\uproman{1}}$. Further condition:
                $- \cot(k a / 2) = \frac{\sqrt{k_0^2 - k^2}}{k}$
        \end{enumerate}
    \item \underline{$E > V_0$}
        In this case $\sqrt{k_0^2 - k^2}$ is imaginary. For even parity we have the
        following conditions.
        \footnotesize
        \begin{align*}
            A_{\uproman{1}} e^{i \sqrt{k^2 - k_0^2} a / 2} + B_{\uproman{1}} e^{-i \sqrt{k^2 - k_0^2} a / 2} &= A_{\uproman{2}} \cos(k a / 2)
            \\
            -i A_{\uproman{1}} \sqrt{k^2 - k_0^2} e^{i \sqrt{k^2 - k_0^2} a / 2} + i B_{\uproman{1}} \sqrt{k^2 - k_0^2} e^{-i \sqrt{k^2 - k_0^2} a / 2} &= - A_{\uproman{2}} k \cos(k a / 2)
        \end{align*}
        \normalsize
        For $E>V_0$ there are no restrictions on the allowed energies!
        These wavefunctions are not normalizable. They are called \fat{unbounded
        states}.
\end{enumerate}
In fact, bound states (for which $E<V$ at $x = \pm \infty$) we have discrete
energy eigenvalues. Unbound states have continuous energy eigenvalues.

\subsubsection{Tunneling}
$V(x) = \begin{cases}
    V_0 \hspace{5pt} &0 \leq x \leq a \\
    0 \hspace{5pt} &\text{elsewhere}
\end{cases}$
\
With $k = \sqrt{\frac{2 m E}{\hbar^2}}$, $k_0 = \sqrt{\frac{2 m V_0}{\hbar^2}}$ and
$k' = \sqrt{k^2 - k_0^2}$ wo obtain:
$\phi_E = \begin{cases}
    A_{\uproman{1}} e^{i k x} + B_{\uproman{1}} e^{-i k x} \hspace{5pt} &x < 0
    \\
    A_{\uproman{2}} e^{i k' x} + B_{\uproman{2}} e^{-i k' x} \hspace{5pt} &0 \leq x \leq a
    \\
    A_{\uproman{3}} e^{i k x} + B_{\uproman{3}} e^{-i k x} \hspace{5pt} &x > a
\end{cases}$

And $\phi_E(x,t) = \phi_E(x) e^{-i \frac{E}{\hbar} t}$.
Further conditions
\begin{align*}
    A_{\uproman{1}} + B_{\uproman{1}} &= A_{\uproman{2}} + B_{\uproman{2}}
    \\
    A_{\uproman{2}} e^{i k' a} + B_{\uproman{2}} e^{-i k' a} &= A_{\uproman{3}} e^{i k a}
    \\
    A_{\uproman{1}} k - B_{\uproman{1}} k &= A_{\uproman{2}} k' - B_{\uproman{2}} k'
    \\
    A_{\uproman{2}} k' e^{i k' a} - B_{\uproman{2}} k' e^{-i k' a} &= A_{\uproman{3}} k e^{i k a}
\end{align*}
The transmission coefficient is $T = \abs{\frac{A_{\uproman{3}}}{A_{\uproman{1}}}}^2$.
In the case $E \ll V_0$ there is a non-zero probability that a particle goes through
the wall.

\subsubsection{Potential step}
$V(x) = \begin{cases}
    0 \hspace{5pt} &x<0
    \\
    V_0 \hspace{5pt} &x>0
\end{cases}$,
General solution:
Region \uproman{1}: $\psi(x) = A_{\uproman{1}} e^{i k x} + B_{\uproman{1}} e^{-i k x}$,
Region \uproman{2}: $\psi(x) = A_{\uproman{2}} e^{\sqrt{k_0^2 - k^2} x} + B_{\uproman{2}} e^{- \sqrt{k_0^2 - k^2} x}$,
with $k = \sqrt{\frac{2 m E}{\hbar^2}}$ and $k = \sqrt{\frac{2 m V_0}{\hbar^2}}$
\begin{enumerate}
    \item \underline{$E>V_0$}:
        Conditions: $B_{\uproman{2}} = 0$ and $i k A_{\uproman{1}} - i k B_{\uproman{1}} = i \sqrt{k^2 - k_0^2}$
        \ yield: $A_{\uproman{2}} = \frac{2 k}{k + \sqrt{k^2 - k_0^2}} A_{\uproman{1}}$ and
        $B_{\uproman{1}} = \frac{k - \sqrt{k^2 - k_0^2}}{k + \sqrt{k^2 - k_0^2}} A_{\uproman{1}}$.
        Thus the reflection coefficient is $\abs{r}^2 = \abs{\frac{B_{\uproman{1}}}{A_{\uproman{1}}}}^2$
        and the transmission coefficient is $\abs{t}^2 = \abs{\frac{A_{\uproman{2}}}{A_{\uproman{1}}}}^2$
    \item \underline{$E<V_0$}:
        Conditions: $A_{\uproman{2}} = 0$, $A_{\uproman{1}} + B_{\uproman{1}} = B_{\uproman{2}}$
        and $i k A_{\uproman{1}} - i k B_{\uproman{1}} = - \sqrt{k_0^2 - k^2} B_{\uproman{2}}$.
        Thus: $B_{\uproman{1}} = - \frac{\sqrt{k_0^2 - k^2} + i k}{\sqrt{k_0^2 - k^2} - i k} A_{\uproman{1}}$
        and $B_{\uproman{2}} = - \frac{2 i k}{\sqrt{k_0^2 - k^2} - i k} A_{\uproman{1}}$
\end{enumerate}
As we can see, there is a non-zero probability to find the particle at $x>0$.

\subsubsection{Harmonic Oscillator}
$V(x) = \frac{1}{2} k x^2 =  \frac{1}{2} m \omega^2 x^2$. The \fat{resonance
frequency} is $\omega = \sqrt{k / m}$. Hence $\hat{H} = \frac{\hat{p}^2}{2 m}
+ \frac{1}{2} m \omega^2 \hat{x}^2$. Introduce $\hat{X} = \sqrt{\frac{m \omega}{\hbar}} \hat{x}$
and $\hat{P} = \sqrt{\frac{1}{\hbar m \omega}} \hat{p}$. Now:
$\hat{H} = \frac{\hbar \omega}{2} \klammer{\hat{X}^2 + \hat{P}^2}$. If we factor this
out: $\hat{H} = \frac{\hbar \omega}{2} \klammer{\hat{X}^2 + i \hat{X} \hat{P} - i \hat{P} \hat{X} + \hat{P}^2 - i \klammer{\hat{X} \hat{P} - \hat{P} \hat{X}}}$
with $-i \klammer{\hat{X} \hat{P} - \hat{P} \hat{X}} = 1$. Further define:
$\hat{a} \equiv \frac{1}{\sqrt{2}} \klammer{\hat{X} + i \hat{P}}$ and
$\hat{a}^\dagger \equiv \frac{1}{\sqrt{2}} \klammer{\hat{X} - i \hat{P}}$. Thus:
$\hat{H} = h \omega \klammer{\hat{a}^\dagger \hat{a} + \frac{1}{2}}$. Important
property: $\eckigeklammer{\hat{a},\hat{a}^\dagger} = 1$. In conclusion:
$c_{n-} \hat{a} | n \rangle = | n - 1 \rangle$ and
$c_{n+} \hat{a}^\dagger | n \rangle = | n + 1 \rangle$.
The energies of the eigenstates $| n \rangle$ are $E_n = \hbar \omega \klammer{n + \frac{1}{2}}$
and the eigenvalues of $\hat{a}^\dagger \hat{a}$ are $n=0,1,2,\dots$

\paragraph{Ground states}
$| 0 \rangle$ is a ground state because it has the lowest energy. For
$\psi_0(x) = \langle x | 0 \rangle$ it has to be true that $\hat{a} \psi_0(x) = 0$.
Thus we obtain: $\psi_0 (x) = \klammer{\frac{m \omega}{\pi \hbar}}^{1/4} e^{- \frac{m \omega}{2 \hbar} x^2}$

\paragraph{Excited state eigenfunctions}
Obtain by applying $\hat{a}^\dagger$. We find $c_{n+} = \frac{1}{\sqrt{n+1}}$
and $c_{n-} = \frac{1}{\sqrt{n}}$. Therefore:
$\hat{a}^\dagger | n \rangle = \sqrt{n + 1} | n + 1 \rangle$ and
$\hat{a} | n \rangle = \sqrt{n} | n -1 \rangle$. So in general we obtain:
$\psi_n (x) = \klammer{\frac{m \omega}{\pi \hbar}}^{1/4} \frac{1}{\sqrt{2^n n!}} H_n(X) e^{-X^2 / 2}$
with $X = \sqrt{\frac{m \omega}{\hbar}} x$ and $H_n(X)$ are the Hermite polynomials.
$\eckigeklammer{\hat{a}^\dagger \hat{a} , \hat{a}^\dagger} = \hat{a}^\dagger$,
$\eckigeklammer{\hat{a}^\dagger \hat{a} , a} = - a$.

% \paragraph{Ladder operators}
% Lowering operator $\hat{a}$ and raising operator $\hat{a}^\dagger$.
% $\eckigeklammer{\hat{a},\hat{a}^\dagger} = 1$. $\hat{a} |n\rangle = \sqrt{n} |n-1\rangle$,
% $\hat{a}^\dagger |n\rangle = \sqrt{n+1} |n+1\rangle$.
% $\eckigeklammer{\hat{a}^\dagger \hat{a} , \hat{a}^\dagger} = \hat{a}^\dagger$,
% $\eckigeklammer{\hat{a}^\dagger \hat{a} , a} = - a$.

\paragraph{Expectation value of the harmonic oscillator}
We can redefine the position and momentum operators:
$\hat{x} = \sqrt{\frac{\hbar}{2 m \omega}} (\hat{a}^\dagger + \hat{a})$
and $\hat{p} = i \sqrt{\frac{m \omega \hbar}{2}} (\hat{a}^\dagger - \hat{a})$.
Since the eigenstates of $\hat{a}$ and $\hat{a}^\dagger$ are orthogonal, we obtain:
$\langle \hat{x} \rangle = \langle n | \hat{x} | n \rangle = 0$,
$\langle \hat{x}^2 \rangle = \langle n | \hat{x}^2 | n \rangle = \frac{\hbar}{2 m \omega} (2n+1)$
and $\langle \hat{p} \rangle = \langle n | \hat{p} | n \rangle = 0$,
$\langle \hat{p}^2 \rangle = \langle n | \hat{p}^2 | n \rangle = \frac{\hbar m \omega}{2} (2n+1)$.
Further:
$\Delta x = \sqrt{\langle \hat{x}^2 \rangle - \langle \hat{x} \rangle^2} = \sqrt{\frac{\hbar}{2 m \omega}} \sqrt{2n+1}$
and
$\Delta p = \sqrt{\langle \hat{p^2} \rangle - \langle \hat{p} \rangle^2} = \sqrt{\frac{\hbar m \omega}{2}} \sqrt{2n+1}$
Thus: $\Delta x \Delta p = \frac{\hbar}{2} (2n+1)$

\paragraph{Coherent states}
Ground state is not the only minimum uncertainty state. A coherent state is a
superposition of eigenstates of the Hamiltonian. At time $t=0$:
$\psi_\alpha = | \alpha \rangle = e^{- \abs{\alpha}^2 / 2} \sum_n^\infty \frac{\alpha^n}{\sqrt{n!}} \phi_n(x)
= \sum_n^\infty \frac{\alpha^n}{\sqrt{n!}} | n \rangle$ with $\alpha = \abs{\alpha} e^{i \phi}$.
We can find: $\langle \hat{x} \rangle = \langle \alpha | \hat{x} | \alpha \rangle =
\sqrt{\frac{\hbar}{2 m \omega}} \klammer{\alpha e^{i \omega t} + \overline{\alpha} e^{-i \omega t}}
= \sqrt{\frac{2 \hbar}{m \omega}} \abs{\alpha} \cos(\omega t + \phi)$


\subsection{Schrödinger Equation in 3D}

$i \hbar \frac{\partial \Psi}{\partial t} = \hat{H} \Psi$ with
$\hat{H} = \frac{1}{2 m} \abs{\hat{\vec{p}}}^2 + V(\hat{\vec{r}})$ where
$\hat{\vec{r}} = (\hat{x},\hat{y},\hat{z})$ and
$\hat{\vec{p}} = (\hat{p}_x , \hat{p}_y , \hat{p}_z)$.
In the position representation: $\hat{p} = - i \hbar \nabla$. So the Schrödinger
equation becomes: $i \hbar \frac{\partial \Psi}{\partial t} = - \frac{\hbar^2}{2 m}
\nabla^2 \Psi + V \Psi$.

\paragraph{Operator commutation}
All operators for different cartesian coordinates commute.
$\eckigeklammer{\hat{x},\hat{p}_y} = \eckigeklammer{\hat{x},\hat{p}_z} = 0$.
\underline{But}:
$\eckigeklammer{\hat{x},\hat{p}_x} = \eckigeklammer{\hat{y},\hat{p}_y} =
\eckigeklammer{\hat{z},\hat{p}_z} = i \hbar$.

% \paragraph{Normalization condition}
% $\int \abs{\Psi(\vec{r},t)}^2 dx dy dz = 1$


\subsection{Symmetry}

\paragraph{Tranformations of wavefunctions and operators in space}
Take wavefunction $\psi(\vec{r})$. If we act a transformation on it, we get a new
wavefunction $\psi'(\vec{r})$. This transformation can be represented with an operator
$\hat{Q}$: $\psi' = \hat{Q} \psi$. So if we have an operator $\hat{O}$ in the $\psi$
basis, we can transform it into the $\psi'$ basis: $\hat{O}' = \hat{Q}^\dagger \hat{O} \hat{Q}$.

\paragraph{Parity operator in $3D$}
$\hat{\Pi} \psi(\vec{r}) = \psi'(\vec{r}) = \psi(-\vec{r})$. Cartesian coordinates:
$\hat{\Pi} \psi(x,y,z) = \psi(-x,-y,-z)$, spherical coordinates:
$\hat{\Pi} \psi(r,\theta,\phi) = \psi(r,\pi - \theta, \phi + \pi)$.
Position operator: $\hat{x}' = - \hat{x}$, $\hat{y}' = - \hat{y}$, $\hat{z}' = - \hat{z}$.
$\hat{\vec{r}}' = - \hat{\vec{r}}$. $\hat{\vec{p}}' = -\hat{\vec{p}}$.
$\hat{\vec{L}}' = -\hat{\vec{L}}$. Thus: $\eckigeklammer{\hat{\Pi},\hat{\vec{L}}} = 0$
and $\eckigeklammer{\hat{\Pi},\hat{L}_x} = 0 \eckigeklammer{\hat{\Pi},\hat{L}_y} =
\eckigeklammer{\hat{\Pi},\hat{L}_z} = 0$. For spherical harmonics:
$\hat{\Pi} Y_{lm} = (-1)^l Y_{lm}$

\paragraph{Rotation operator}
A rotation about the $z$-axis by an angle $\varphi$ in position representation:
$\hat{R}_z(\delta) 1 - \delta \frac{\partial}{\partial \phi} = 1 - \frac{i \delta}{\hbar} \hat{L}_z$.
Rotation about an arbitrary axis $\hat{n}$ is given by $\hat{R}_{\hat{n}}(\varphi)
= e^{- \frac{i \varphi}{\hbar} \hat{n} \cdot \vec{L}}$

\paragraph{Selection rules}
Suppose $\hat{\Pi}^\dagger V(x) \hat{\Pi} = V(x)$, then
$\eckigeklammer{V,\hat{\Pi}} = 0$ and $\eckigeklammer{\hat{H},\hat{\Pi}} = 0$.
Then the non-degenerate eigenstates $| \psi_n \rangle$ of $\hat{H}$ satisfy
$\hat{\Pi} | \psi_n \rangle = \pm | \psi_n \rangle$.
If $H_{i,f}(t) = \langle \phi_i | \hat{H}(t) | \phi_f \rangle
= \int \phi_i(\vec{r}) \hat{H}(t,\vec{r}) \phi_f(\vec{r}) d^3 r = 0$, then
the transition isn't allowed. If the integrand is odd under parity, the integral is
is zero. \fat{First selection rule}: $\langle \psi_{n'l'm'} | \hat{\vec{r}} | \psi_{nlm} \rangle = 0$
if $l+l'$ is even.
$\eckigeklammer{L_z , z} = 0$, $\eckigeklammer{L_z ,y} = -i \hbar x$,
$\eckigeklammer{L_z ,x } = i \hbar $,
$\eckigeklammer{L_z , x + iy} = \hbar (x+iy)$,
$\eckigeklammer{L_z , x - iy} = - \hbar (x-iy)$
Thus, selection rule will depend on polatization of light.
$l' - l = \pm 1$ has to be fulfilled.

\paragraph{Linearly polarized light in the $z$ direction}
% $\langle \psi_{n'l'm'} | \hat{z} | \psi_{nlm} \rangle = 0$ unless $m' = m$.
For linearly polarized light along the $z$ direction, transitions cannot happen
unless the $m$ quantum numbers of the initial and final state are the same.

\paragraph{Circularly polarized light in the plane perpendicular to the $z$ direction}
% $\langle \psi_{n'l'm'} | \hat{x} + i \hat{y} | \psi_{nlm} \rangle = 0$ unless
% $m' = m+1$ and $\langle \psi_{n'l'm'} | \hat{x} - i \hat{y} | \psi_{nlm} \rangle = 0$
% unless $m' = m-1$. Thus, 
For circularly polarized light, transitions are forbidden
unless they are between states that defer by $1$ in their $m$ quantum number.


\subsection{quantum statistical Physics}
$N$ particles with total energy $E$. Collection of single-particle states.
$g_j$ of these have energy $\epsilon_j$. What are the number of atoms $n_j$ with
energy $\epsilon_j$? If we have $g_j$ states with the same energy,
$n_j = g_j \cdot N \cdot P_j = g_j N \frac{e^{-\beta \epsilon_j}}{Z}$.
How many ways are there to put $n_1$ particles into $g_1$ states and
$n_2$ particles into $g_2$ states, etc? If the number of ways to put $n_j$ particles
into $g_j$ states is $\omega_j(n_j)$, then we have:
$\Omega_{\geschwungeneklammer{n_j}} = \prod_j \omega_j(n_j)$

\paragraph{Fermions}
Each of the $g_j$ states can only accomodate one particle. So:
$\omega_j(n_j) = \frac{g_j !}{n_j ! (g_j - n_j)!}$. Important:
particles are indistinguishable but the states are distinguishable. So:
$\Omega_{\geschwungeneklammer{n_j}} = \prod_j \frac{g_j !}{n_j ! (g_j - n_j)!}$.
Assume $n_j , g_j , g_j - n_j \gg 1$, then:
$\ln \Omega_{\geschwungeneklammer{n_j}} = \sum_j \eckigeklammer{\ln(g_j !) - \ln(n_j !) - \ln((g_j - n_j)!)}
\approx \sum_j \eckigeklammer{g_j \ln(g_j) - n_j \ln(n_j) - (g_j - n_j) \ln(g_j - n_j)}$.
Constraints: $\sum_j n_j = N$ and $\sum_j n_j \epsilon_j = E$. With lagrange multipliers
we obtain: $n_j = \frac{g_j}{e^{\beta (\epsilon_j - \mu)} + 1}$ with
$\beta = \frac{1}{k_B T}$ and $\mu$ the \fat{chemical potential}.
$n_j$ is called \fat{Fermi-Dirac Distribution}. $\mu(T=0)$ is the \fat{Fermi energy}.

\paragraph{Bosons}
How many ways are there to devie $n_j$ identical particles into $g_j$ bins.
$\omega_j (n_j) = \frac{(n_j + g_j - 1)!}{n_j ! (g_j - 1)!}$,
$\Omega_{\geschwungeneklammer{n_j}} = \prod_j \frac{(n_j + g_j - 1)!}{n_j ! (g_j - 1)!}$,
hence $n_j = \frac{g_j -1}{e^{\beta(\epsilon_j - \mu)}-1} \approx
\frac{g_j}{e^{\beta(\epsilon_j - \mu)} -1}$ since we assumed $g_j \gg 1$.

\paragraph{Average energy in a mode with frequency $\omega$}
$\langle \epsilon \rangle = \frac{\hbar \omega}{e^{\beta \hbar \omega} - 1}$, so
the number of photons in that mode is just $n = \frac{\langle \epsilon \rangle}{\hbar \omega}
= \frac{1}{e^{\beta \hbar \omega} -1}$

\paragraph{Bose-Einstein Distribution}
$n_\omega = \frac{V \omega^2}{\pi^2 c^3} \frac{d \omega}{e^{\beta \hbar \omega} -1}
= \frac{\rho(\omega) d \omega}{\hbar \omega} \cdot V$

\paragraph{Classical limit}
$n_j \ll g_j$: $n_j \approx \frac{g_j}{e^{\beta (\epsilon_j - \mu)}}$
With this, we already lost the information wheter it is a fermion or a boson.
$N = \sum_j n_j \approx \sum_j \frac{g_j}{e^{\beta(\epsilon_j - \mu)}} \Rightarrow
\frac{1}{e^{-\beta \mu}} = \frac{N}{\sum_j g_j e^{-\beta \epsilon_j}} = \frac{N}{Z}
\Rightarrow n_j = N g_j \frac{e^{-\beta \epsilon_j}}{Z}$ which agrees with the
classical result.



