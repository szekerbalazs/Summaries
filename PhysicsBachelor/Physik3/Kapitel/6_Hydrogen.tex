
\section{Hydrogen}

\subsection{Hydrogen atom Schrödinger Equation}
% If we have a central potential, meaning $V(\vec{r}) = V(\Norm{r}) = V(r)$,
% then we can use separation of variables to solve SE: $\psi(r,\theta,\phi) =
% R(r) Y(\theta,\phi)$. We obtain:
% \begin{enumerate}[]
%     \item $\frac{1}{R} \frac{\partial}{\partial r} \klammer{r^2 \frac{\partial R}{\partial r}} - \frac{2 m r^2}{\hbar} \klammer{V(r) - E} = l(l+1)$
%     \item $\frac{1}{Y} \eckigeklammer{\frac{1}{\sin(\theta)} \frac{\partial}{\partial \theta} \klammer{\sin(\theta) \frac{\partial Y}{\partial \theta}} + \frac{1}{\sin^2(\theta)} \frac{\partial^2 Y}{\partial \phi^2}} = -l(l+1)$
% \end{enumerate}
% Thus
% \begin{enumerate}[]
%     \item $Y_l^m (\theta,\phi) = \sqrt{\frac{2l+1}{4 \pi} \frac{(l-m)!}{(l+m)!}} e^{i m \phi} P_l^m (\cos(\theta))$
%     \item $R(r)$ has to be solved ad hoc because it is dependent on $V(r)$. But with
%         $u(r) = r R(r)$, we can simplify to: $- \frac{\hbar^2}{2m} \frac{\partial^2 u}{\partial r^2} + \eckigeklammer{V(r) + \frac{\hbar^2}{2m} \frac{l(l+1)}{r^2}} u = E u$
% \end{enumerate}

First few spherical harmonics:
\begin{align*}
    Y_0^0 = \frac{1}{\sqrt{4 \pi}}
    \hspace{5pt} , \hspace{5pt}
    Y_1^0 = \sqrt{\frac{3}{4 \pi}} \cos(\theta)
    \hspace{5pt} , \hspace{5pt}
    Y_1^{\pm 1} = \mp \sqrt{\frac{3}{8 \pi}} \sin(\theta) e^{\pm i \phi}
    \\
    Y_2^0 = \sqrt{\frac{5}{16 \pi}} \klammer{3 \cos^2(\theta) - 1}
    \hspace{5pt} , \hspace{5pt}
    Y_2^{\pm 1} = \mp \sqrt{\frac{15}{8 \pi}} \sin(\theta) \cos(\theta) e^{\pm i \phi}
    \\
    Y_2^{\pm 2} = \sqrt{\frac{15}{32 \pi}} \sin^2(\theta) e^{\pm 2 i \phi}
\end{align*}

First few radial solutions:
\begin{align*}
    R_{10}(r) &= 2 \klammer{\frac{1}{a_0}}^{3/2} e^{-r/a_0}
    \hspace{5pt} , \hspace{5pt}
    R_{20}(r) = 2 \klammer{\frac{1}{2 a_0}}^{3/2} \klammer{1 - \frac{r}{2 a_0}} e^{-r/(2a_0)}
    \\
    R_{21}(r) &= \frac{2}{\sqrt{3}} \klammer{\frac{1}{2a_0}}^{3/2} \klammer{\frac{r}{2 a_0}} e^{-r/(2 a_0)} 
\end{align*}

% \paragraph{Normalization condition}
% $\int_0^\pi \int_0^{2 \pi} \overline{Y_l^m} Y_{l'}^{m'} \sin(\theta) d \theta d \phi = \delta_{ll'} \delta_{mm'}$


% \subsubsection{Radial equation for Coulomb potential}
% Assume proton remains motionless. $V(r) = \frac{-1}{4 \pi \epsilonnull} \frac{1}{r}$
% \begin{align*}
%     - \frac{\hbar^2}{2 m_e} \frac{\partial^2 u}{\partial r^2} + \eckigeklammer{
%     - \frac{e^2}{4 \pi \epsilonnull} \frac{1}{r} + \frac{\hbar^2}{2 m_e} \frac{l(l+1)}{r^2}} u
%     = E u
% \end{align*}
% $V(r) \rightarrow 0$ for $r \rightarrow \infty$. We only have bound states if
% $E<0$. Only consider those cases. Define $\kappa = \sqrt{\frac{-2 m_e E}{\hbar}} \in \R$.
% Then:
% \begin{align*}
%     \frac{1}{\kappa^2} \frac{\partial^2 u}{\partial r^2} = \eckigeklammer{
%         1 - \frac{m_e e^2}{2 \pi \epsilonnull \hbar^2 \kappa} \frac{1}{\kappa r} + \frac{l(l+1)}{(\kappa r)^2}}
%         u
% \end{align*}
% Define $\rho = \kappa r$ and $\rho_0 = \frac{m_e e^2}{2 \pi \epsilonnull \hbar^2 \kappa}$
% Thus the ODE becomes:
% \begin{align*}
%     \frac{\partial^2 u}{\partial \rho^2} = \eckigeklammer{1 - \frac{\rho_0}{\rho} + \frac{l(l+1)}{\rho^2}} u
% \end{align*}
% For $\rho \rightarrow \infty$: $\frac{\partial^2 u}{\partial \rho^2} = u$. Thus
% $u = A e^{-\rho}$. ($B=0$ because would not be bound.) As $\rho \rightarrow 0$:
% $\frac{\partial^2 u}{\partial \rho^2} = \frac{l(l+1)}{\rho^2} u$. Thus
% $u = C \rho^{l+1}$. Thus we obtain: $u(\rho) = \rho^{l+1} e^{-\rho} V(\rho)$.
% We can write $V(\rho) = \sum_{j=0}^\infty c_j \rho^j$. With some inserting we obtain:
% $c_{j+1} = \eckigeklammer{\frac{2(j+l+1) - \rho_0}{(j+1)(j+2l+2)}} c_j$. Further
% calculations yield:

\paragraph{Energy}
$E_n = - \eckigeklammer{\frac{m_e}{2 \hbar^2} \klammer{\frac{e^2}{4 \pi \epsilonnull}}^2} \frac{1}{n^2}
= \frac{E_1}{n^2}$. With $E_1 = -13.6 \mathrm{eV} = - R_e$ the ground state energy.

\paragraph{Bohr radius}
$\kappa = \klammer{\frac{m_e e^2}{4 \pi \epsilonnull \hbar^2}} \frac{1}{n} = \frac{1}{a n}$
with $a_0 = \frac{4 \pi \epsilonnull \hbar^2}{m_e e^2}$ the Bohr radius.

\paragraph{Full spatial wavefunction}
$\Psi_{nlm} (r,\theta,\phi) = R_{nl}(r) Y_l^m (\theta,\phi)$
where $n,l,m$ are \fat{quantum numbers}. $n$ is called \fat{principle quantum number}.
\underline{Energy only depends on $n$}. $l = 0,1,2,\dots,n-1$ and $m = -l,-l+1,\dots,
0,\dots,l-1,l$. Number of degenerate eigenstates: $n^2$.
$n$ is the orbital energy level, $l$ the shape and $m$ the number of spatial orientations
that the orbital can assume. The larger $n$ is, the larger is $E_n$ (because of
the negative sign).

% \paragraph{Ground state energy}
% $E_1 = - \eckigeklammer{\frac{m}{2 \hbar^2} \klammer{\frac{e^2}{4 \pi \epsilonnull}}^2}
% = -13.6 \mathrm{eV} = - \mathrm{R_e}$

% \paragraph{Wavefunction including normalization}
% $Y_{nlm} (r,\theta,\phi) = \sqrt{\klammer{\frac{2}{n a}}^3 \frac{(n-l-1)!}{2n(n+l)!}} e^{-\frac{r}{n a}} \klammer{\frac{2r}{n a}}^l
% L_{n-l-1}^{2l+1} \klammer{\frac{2 r}{n a}} Y_l^m (\theta,\phi)$

\subsection{Angular Momentum}
$\vec{L} = \vec{x} \times \vec{p}$, $\eckigeklammer{L_x , L_y} = i \hbar L_z$,
$\eckigeklammer{L_y , L_z} = i \hbar L_x$, $\eckigeklammer{L_z , L_y} = i \hbar = L_y$,
$\eckigeklammer{L^2 , L_x} = 0$, $\eckigeklammer{L^2 , L_y} = 0$,
$\eckigeklammer{L^2 , L_z} = 0$, $L^2 | l,m \rangle = \hbar^2 l (l+1) | l,m \rangle$,
$L_z | l,m \rangle = \hbar m | l,m \rangle$.
Construct new operator:
$L_{\pm} \equiv L_x \pm i L_y$ for which:
$\eckigeklammer{L_z , L_{\pm}} = \pm \hbar L_{\pm}$ and
$\eckigeklammer{L^2 , L_{\pm}} = 0$. In general:
$L_{\pm} |l,m \rangle = A_{l,\pm}^m |l,m \pm 1 \rangle$ with
$A_{l,\pm}^m = \hbar \sqrt{l(l+1) - m(m \pm 1)}$.
Due to causality constraints we find
$m_{max} = \mu$ s.t. $L_+ |l,\mu \rangle = 0$ and
$m_{min} = \mu'$ s.t. $L_- |l,\mu' \rangle = 0$. Also:
$L_{\pm} L_{\mp} = L^2 - L_z^2 \pm \hbar L_z$ and
$L^2 |l,\mu \rangle = \hbar^2 \mu (\mu + 1) |l,\mu \rangle$
so we find $\mu = l$ and $\mu' = - l$. Possible values for $l$ and $m$:
$l = 0,\frac{1}{2},1,\frac{3}{2},2,\dots$ and $m=-l,-l+1,\dots,l-1,l$.
In position representation we define the \fat{angular momentum operator}
$\vec{L} = -i \hbar (\vec{r} \times \vec{\nabla})$ where
$\vec{\nabla} = \hat{r} \frac{\partial}{\partial r} + \hat{\theta} \frac{1}{r}
\frac{\partial}{\partial \theta} + \hat{\phi} \frac{1}{r \sin(\theta)} \frac{\partial}{\partial \phi}$
Hamiltonian can be rewritten as
$H = \frac{1}{2 m r^2} \eckigeklammer{- \hbar^2 \frac{\partial}{\partial r} \klammer{r^2 \frac{\partial}{\partial r}} + L^2} + V$
Thus: $H \psi = E \psi$ , $L^2 \psi = \hbar^2 l(l+1) \psi$ , $L_z \psi = \hbar m \psi$.


\subsection{Time-independent Perturbation Theory}
Write Hamiltonian as $\hat{H} = \hat{H}_0 + \delta \hat{H}$ where $\hat{H}_0$
can be solved exactly and $\delta \hat{H}$ is a small perturbation, meaning:
If $| \psi_n^0 \rangle$ is an eigenstate of $\hat{H}_0$ s.t.
$\hat{H}_0 | \psi_n^0 \rangle = E_n^0 | \psi_n^0 \rangle$, then the time independent
S.E. can be written as $\hat{H} | \psi_n \rangle = E_n | \psi_n \rangle$
wher $E_n = E_n^0 + \delta E_n$, $\abs{\delta E_n} \ll E_n^0 \ \forall n$
and $| \psi_n \rangle = | \psi_n^0 \rangle + | \delta \psi_n \rangle$,
$\abs{\delta \psi_n} \ll \abs{\psi_n^0}$. Hence
$\hat{H}_0 | \delta \psi_n \rangle + \delta \hat{H} | \psi_n^0 \rangle =
E_n^0 | \delta \psi_n \rangle + \delta E_n | \psi_n^0 \rangle$. Since
$\geschwungeneklammer{| \psi_n^0 \rangle}$ is a complete basis we can write
$| \delta \psi_n \rangle = \sum_l c_l | \psi_l^0 \rangle$. Thus
$E_m^0 + \langle \psi_m^0 | \delta \hat{H} | \psi_n^0 \rangle = E_n^0 c_m +
\delta E_n \langle \psi_n^0 | \psi_m^0 \rangle$.
\begin{enumerate}[]
    \item \underline{If $n=m$}: $\langle \psi_n^0 | \delta H | \psi_n^0 \rangle = \delta E_n$.
        This means, the correction to the energy of the eigenstate caused by the perturbation
        is just the expectation value of $\delta \hat{H}$ for the original eigenstate.
    \item \underline{if $n \neq m$}: $\langle \psi_m^0 | \delta \hat{H} | \psi_n^0 \rangle
        = c_m (E_n^0 - E_m^0) \ \Rightarrow \ c_m = \frac{\langle \psi_m^0 | \delta \hat{H} | \psi_n^0 \rangle}{E_n^0 - E_m^0}$
        With these coefficients we can then write $| \delta \psi_n  \rangle$.
        Problem if $E_n^0 = E_m^0$. Could be if there is degeneracy.
\end{enumerate}

\subsection{Degenerate Perturbation Theory}
The above problem can be resolved if $\langle \psi_m^0 | \delta \hat{H} | \psi_n^0 \rangle = 0$
The case if $| \psi_n^0 \rangle$ and $| \psi_m^0 \rangle$ were also eigenstates
of $\delta \hat{H}$. Then: $\delta \hat{H} | \psi_n^0 \rangle = \delta E_n | \psi_n^0 \rangle$
and $\delta \hat{H} | \psi_m^0 \rangle = \delta E_m | \psi_m^0 \rangle$. This means,
perturbed eigenstates $| \psi_n \rangle$ and $| \psi_m \rangle$ are just
$| \psi_n^0 \rangle$ and $| \psi_m^0 \rangle$. If we have a set of $N$ degenerate
eigenstates $\geschwungeneklammer{| \psi_{kl}^0 \rangle}$ of $\hat{H}_0$ with
eigenenergy $E_k^0$, then any linear combination of these states is also an
eigenstate of $\hat{H}_0$ with energy $E_k$. That means, $\geschwungeneklammer{| \psi_{kl}^0 \rangle}$
forms a basis for a so called \fat{$N$-dimensional degenerate subspace}.
$\hat{p}^2 = 2 m (\hat{H}_0 - \hat{V})$

\subsection{Time-depentent Perturbation Theory}
Suppose we can write $\hat{H} = \hat{H}_0 + \hat{H}_1 (t)$ with $\hat{H}_0$ a
time-independent Hamiltonian, and $\hat{H}_1 (t)$ is a time-dependent perturbation.
If $\hat{H}_1 (t) = 0$, then $| \psi(t) \rangle = \sum_n c_n e^{-i \omega_n t} | \phi_n \rangle$
with $\omega_n = \frac{E_n}{\hbar}$ and $\geschwungeneklammer{| \phi_n \rangle}$
energy eigenstates. If $\hat{H}_1 (t) \neq 0$, then
$| \psi(t) \rangle = \sum_n c_n (t) e^{-i \omega_n t} | \phi_n \rangle$, such that
$\frac{\partial c_m}{\partial t} = \frac{1}{i \hbar} \sum_n c_n e^{-i(\omega_n - \omega_m) t} \langle \phi_m | \hat{H}_1 (t) | \phi_n \rangle$.
The system cannot transition from the state $j$ to the state $n$ if
$\langle \phi_j | H | \phi_m \rangle = 0$.

Suppose an atom encounters an electromagnetic wave. The electric field of the wave
has a sinusiodal time dependence, while the atom has an electric dipole moment
$\vec{p}_0$: $\vec{E} = E_0 \cos(\omega t) \hat{k}$ and $\vec{p}_0 = - e \vec{r}$.
Then: $\hat{H}_1 (t) = - \vec{p_0} \cdot \vec{E} = e E_0 \hat{\vec{r}} \cdot \hat{k}
\cos(\omega t)$

If an atom starts in the state $| \phi_a \rangle$, the probability that it ends
up in the state $| \phi_b \rangle$ at time $t$ is:
$P_{a \rightarrow b} (t) = \abs{c_b (t)}^2 \approx \frac{\abs{V_{ab}}^2}{\hbar^2}
\klammer{\frac{t}{2}}^2 \text{sinc}^2 \eckigeklammer{\klammer{\omega_0 - \omega} \frac{t}{2}}$
with $V_{ab} = \langle \phi_a | e E_0 \vec{r} \cdot \hat{k} | \phi_b \rangle$ and
$\omega_0 = \omega_a - \omega_b$

\subsection{Zeeman Effect}
$\vec{\mu}_p = \frac{g_p e}{2 m_p} \vec{s}_p \ll \vec{\mu} = \frac{g_s e}{2 m_e} \vec{s}$.
The perturbed hamiltonian is: $\delta \hat{H} = - \vec{\mu} \cdot \vec{B}$. If the
magnetic field points along the $z$-axis: $\delta \hat{H} = \frac{g_s e B}{2 m_e} \hat{s}_z$.
Spin eigenstates $| \frac{1}{2} , \pm \frac{1}{2} \rangle$ are the eigenstates of
$s_z$, so: $\delta E_+ = \frac{g_s e \hbar B}{4 m_e}$ and
$\delta E_- = \frac{g_s e \hbar B}{4 m_e} = - \frac{g_s}{2} \mu_B B$.
In general: For any particle with magnetic moment $\vec{\mu} = g_j \frac{\mu_B}{\hbar} \vec{J}$
where $\vec{J}$ is the angular momentum operator, the energy shift is given by
$\delta E = \mu_B g_j m_j B$.

For a magnetic field along the $x$-direction:
$\delta \hat{H}_{zm} = \frac{g_s e B}{2 m_e} \hat{s}_x$. Use eigenstates of
$\hat{s}_x$ to calculate $\delta E$. ($|+\rangle = \frac{1}{\sqrt{2}} (1,1)$ and
$|-\rangle = \frac{1}{\sqrt{2}} (1,-1)$) $\delta E_+ = \frac{g_s}{2} \mu_B B$ and
$\delta E_- = - \frac{g_s}{2} \mu_B B$.

\subsection{Dipole allowed transitions}
\begin{itemize}
    \item $\Delta l = \pm 1$ and $\Delta m = 0$ for linearly ($\pi$) polarized light.
    \item $\Delta l = \pm 1$ and $\Delta m = +1$ for right handed ($\sigma^+$) circularly polarized light.
    \item $\Delta l = \pm 1$ and $\Delta m = -1$ for left handed ($\sigma^-$) circularly polarized light.
\end{itemize}

\subsection{Hyperfine splitting in Hydrogen}
Magnetic moment of the hydrogen nucleus also generates a magnetic field which
interacts with the electron spin. In classical electrodynamics, a magnetic moment
$\vec{\mu}$ generates a magnetic field: $\vec{B}$
%  = \frac{\mu_0}{4 \pi r^3} \eckigeklammer{3 (\vec{\mu}_p \cdot \hat{r}) \hat{r} - \vec{\mu}_p} + \frac{2}{3}
% \mu_0 \vec{\mu}_p \delta^3 (\vec{r})$
% with $\vec{\mu}_p = \frac{g_p e}{2 m_e} \vec{s}_p$.
Hamiltonian of the electron: $\delta \hat{H}_{hf} = - \vec{\mu} \cdot \vec{B}$
% = \frac{\mu_0 g_p g_s e^2}{16 \pi m_p m_e}
% \frac{3 (\vec{s}_p \cdot \hat{r}) (\vec{s}_e \cdot \hat{r}) - \vec{s}_p \cdot \vec{s}_e}{r^3}
% + \frac{\mu_0 g_p g_s e^2}{6 m_p m_e} \vec{s}_p \cdot \vec{s}_e \delta^3 (r)$.
% Energy shift:
% $E_{hf} = \frac{\mu_0 g_p g_s e^2}{16 \pi m_p m_e} \left\langle \psi_{100} , m_s \Big|
% \frac{3 (\vec{s}_p \cdot \hat{r}) (\vec{s}_e \cdot \hat{r}) - \vec{s}_p \cdot \vec{s}_e}{r^3}
% \Big| \psi_{100} , m_s \right\rangle + \frac{\mu_0 g_p g_s e^2}{6 m_p m_e} \langle
% m_s | \vec{s}_p \cdot \vec{s}_e | m_s \rangle \abs{\psi_{100} (r=0)}^2$.
% For $l=0$: $E_{hf} = \frac{\mu_0 g_p g_s e^2}{6 \pi m_p m_e a^3}
% \langle \vec{s}_p \cdot \vec{s}_e \rangle$. We have $\vec{s}_p \cdot \vec{s}_e =
% \frac{1}{2} \klammer{s^2 - s_p^2 - s_e^2}$, since both electron and proton have spin
% $\frac{1}{2}$ we have $s_p^2 = s_e^2 = \frac{3}{4} \hbar^2$. Possible values of
% $S^2$ are: $S^2 = 2 \hbar^2$ (\textit{triplet}, $m_s = -1,0,2$) and
% $S^2 = 0$ (\textit{singlet},$m_s = 0$). Their energies are:
% $E_{hf} = \underbrace{\frac{\mu_0 g_p g_s e^2 \hbar^2}{6 \pi m_p m_e a^3}}_{:= \Delta E} \cdot \begin{cases}
%     \frac{1}{4} \hspace{5pt} &\text{triplet} \\
%     - \frac{3}{4} \hspace{5pt} &\text{singlet}
% \end{cases}$
% A photon is emitted when the electron transition from the triplet to the singlet
% state has frequency and wavelength $\nu = \frac{\Delta E}{h}$, $\lambda = \frac{c}{\nu}
% = 21 \mathrm{cm}$.


