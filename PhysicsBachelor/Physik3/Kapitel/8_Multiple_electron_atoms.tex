\section{Multiple-electron atoms}

\subsection{Helium}

\paragraph{Hamiltonian for a neutral atom with atomic number $Z$}
$\hat{H} = \sum_{j=1}^Z \eckigeklammer{- \frac{\hbar^2}{2m} \nabla_j^2 -
\klammer{\frac{1}{4 \pi \epsilonnull}} \frac{Z e^2}{r_j}} + \frac{1}{2}
\klammer{\frac{1}{4 \pi \epsilonnull} \sum_{k \neq j}^Z \frac{e^2}{\abs{\vec{r}_j - \vec{r}_k}}}$

\paragraph{Hydtogen like-atoms}
effective Bohr-radius: $a/Z$ and energies become $Z^2 E_n$.

\paragraph{Complete spatial wavefunction of the helium atom}
$\psi(\vec{r}_1 , \vec{r}_2) = \psi_{nlm} (\vec{r}) \psi_{n' l' m'} (\vec{r}_2)$

\paragraph{Exchange operator}
$\hat{X} \psi(\vec{r}_1 , \vec{r}_2) = \psi (\vec{r}_2 , \vec{r}_1)$,
$\hat{X}$ has eigenvalues $\pm 1$. Eigenstates with eigenvalue $+1$ are
\fat{symmetric} and eigenstates with eigenvalue $-1$ are \fat{antisymmetric}.

Construct symmetric and anti-symmetric eigenstates:
$\psi_{ab}^\pm = (\vec{r}_1 , \vec{r}_2) = A \eckigeklammer{\psi_a(\vec{r}_1)
\psi_b(\vec{r}_2) \pm \psi_a (\vec{r}_2) \psi_b (\vec{r}_1)}$ with
$A = \begin{cases}
    \frac{1}{\sqrt{2}} \hspace{5pt} &\text{for } a \neq b
    \\
    \frac{1}{2} \hspace{5pt} &\text{for } a = b
\end{cases}$

% \paragraph{Ground state of helium}
% $| \psi_{100} , \chi_- \rangle$

\subsection{Bosons and Fermions}

\paragraph{Spin-statistics theorem}
Particles whose states are symmetric under exchange have integer spin values.
They are called \fat{bosons}. Particles whose states are antisymmetric under
exchange have half-integer spin value. They are called \fat{fermions}.

\paragraph{Pauli exclusion principle}
No two identical fermions can be simultaneously in the same single-particle state.

% \paragraph{General state condition}
% $| \psi \rangle = \sum_{\sigma \in \stackrel{\text{Spin}}{\text{states}}}
% | \psi_{\sigma} , \sigma \rangle$ The state $| \psi \rangle$ for a single particlecan
% can be factorized into spatial and spin components. Thus, for two particles we can
% write $| \psi \rangle = | \psi_{ab} , \chi \rangle$ with $\psi_{ab}$ a state with
% wavefunction $\psi_{ab} (\vec{r}_1 , \vec{r}_2)$ and $\chi$ the spin state of the
% two particles. $| \psi_{ab} \rangle$ and $| \chi \rangle$ have to be eigenstates
% of $\hat{X}$. Triplet states are symmetric and singlet states are antisymmetric.

\subsection{Many electron atoms}

\paragraph{Central field approximation}
Write $\hat{H}$ as $\hat{H}_c + \hat{H}_{nc}$ with $\hat{H}_c$ the spherically
symmetric part of $\hat{H}$ for each electron:
$\hat{H}_c = \sum_{j=1}^Z \eckigeklammer{- \frac{\hbar^2}{2m} \nabla_j^2 + U(r_j)}
= \sum_{j=1}^Z \hat{H}_{cj}$ with $U(r_j)$ the central potential that is identical
for each electron. $\hat{H}_{nc}$ is the rest of the Hamiltonian:
$\hat{H}_{nc} = \sum_{j=1}^Z \eckigeklammer{- \frac{Z e^2}{4 \pi \epsilonnull r_j} - U(r_j)}
+ \frac{1}{2} \klammer{\frac{1}{4 \pi \epsilonnull} \sum_{k \neq j}^Z \frac{e^2}{\abs{\vec{r}_j - \vec{r}_k}}}$.
We have $U(r \rightarrow 0) = \frac{- Z e^2}{4 \pi \epsilonnull r}$ and
$U(r \rightarrow \infty) = \frac{-e^2}{4 \pi \epsilonnull r}$.
Energy increases with $l$. A lower value of $n$ gives a lower energy.

\paragraph{Electron Configurations}
The number of single electron states if given by $2(2l+1)$

\begin{tabular}{|c|c|c|}
    \hline
    Value of $l$ & Letter symbol & Number of single electron states \\
    \hline
    0 & s & 2 \\ \hline
    1 & p & 6 \\ \hline
    2 & d & 10 \\ \hline
    3 & f & 14 \\ \hline
    4 & g & 18 \\ \hline
\end{tabular}

Total Spin: $\hat{S} = \sum_{j=1}^N \hat{S}_j$ with eigenvalues $\hbar^2 S(S+1)$.
Total angular momenta: $\hat{L} = \sum_{j=1}^N \hat{L}_j$ with eigenvalues
$\hbar^1 L(L+1)$. In general, since states with different $L$ and $S$ quantum numbers
can have different energies, it's also useful to specify what they are in the form
of an additional label called a term symbol, which is written as $^{2S+1}L$.

\paragraph{L-S coupling}
Energy depends on the value of $L$ and $S$.

\paragraph{Spin-orbit coupling}
Total angular momentum $\hat{L}$ couples to the total spin $\hat{S}$ to produce
a perturbation Hamiltonian: $\hat{H}_{so} = \xi \vec{L} \cdot \vec{S}$ with $\xi$
a constant. Perturbation requires us to work with total angular momentum
$\vec{J} = \vec{L} + \vec{S}$. Eigenvalues of $J^2$ are $\hbar^2 J(J+1)$ and the
eigenvalues of $J_z$ are $\hbar M_j$. Term symbol can be modified to specify
$J$: $^{2S+1}L_J$






