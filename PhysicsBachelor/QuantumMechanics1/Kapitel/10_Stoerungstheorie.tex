\section{Störungstheorie}

\subsection{Nicht-entartete zeit-unabhängige ST}

Ausgangslage:
\begin{align*}
    H \Psi = E \Psi
    \hspace{10pt} , \hspace{10pt}
    H = H_0 + H'
    \hspace{10pt} , \hspace{10pt}
    H_0 \varphi_n = \mathcal{E}_n \varphi_n
\end{align*}
Hamiltonoperator $H$ mit einem Teil $H_0$ der exakt lösbar ist. Die $\varphi_n$ für
$n \in \N$ sind ein vollständiger Satz von orthonormierten Eigenfunktionen von
$H_0$. Es gilt $\mathcal{E}_m \neq \mathcal{E}_n$ für $m \neq n$. Wir fassen
$H'$ als kleine Störung auf und wir schreiben
\begin{align*}
    H = H_0 + \lambda H'
\end{align*}
und machen den Störungsansatz
\begin{align*}
    \Psi = \Psi_0 + \lambda \Psi_1 + \lambda^2 \Psi_2 + \dotsb
    \hspace{10pt} , \hspace{10pt}
    E = E_0 + \lambda E_1 + \lambda^2 E_2 + \lambda^3 E_3 + \dotsb
\end{align*}
Eingesetzt erhlaten wir für die $s$-te Ordnung von $\lambda$:
\begin{align*}
    (H_0 - E_0) \Psi_s = - H' \Psi_{s-1} + \sum_{j=1}^s E_j \Psi_{s-j}
    \ \ \forall s \geq 1
    \hspace{10pt} , \hspace{10pt}
    (H_0 - E_0) \Psi_0 = 0
\end{align*}
In $0$-ter Ordnung:
\begin{align*}
    \Psi_0 = \varphi_n
    \hspace{10pt} , \hspace{10pt}
    E_0 = \mathcal{E}_n
    \hspace{10pt}
    \text{für geeignetes $n \in \N$}
\end{align*}
Die weiteren $\Psi_k$ für $k \geq 1$ können wir iterativ finden. Diese Gleichungen
bestimmen $\Psi_k$ nur bis auf Vektoren, die im Kern von $(H_0 - E_0)$ liegen.
Da die Eigenwerte von $H_0$ nicht entartet sind, sind solche Vektoren proportional
zu $\varphi_n$. Wir legen diese Freiheit dadurch fest, dass wir verlangen, dass
\begin{align*}
    \left\langle \Psi_0 | \Psi_k \right\rangle = \delta_{0k}
    \hspace{10pt} \Leftrightarrow \hspace{10pt}
    \left\langle \varphi_n | \Psi \right\rangle = 1
\end{align*}
Damit folgt:
\begin{align*}
    \underbrace{\left\langle \Psi_0 | H_0 - E_0 | \Psi_s \right\rangle}_{=0}
    = \underbrace{\left\langle \Psi_0 | E_1 - H' | \Psi_{s-1} \right\rangle}_{\delta_{s1} E_1 - \left\langle \Psi_0 | H' | \Psi_{s-1} \right\rangle}
    + \sum_{j=2}^s E_j \underbrace{\left\langle \Psi_0 | \Psi_{s-j} \right\rangle}_{= \delta_{sj}}
\end{align*}
\begin{align*}
    \Rightarrow \hspace{10pt}
    E_s = \left\langle \varphi_n | H' | \Psi_{s-1} \right\rangle
    \hspace{10pt} s \geq 1
\end{align*}
\begin{align*}
    E = E_0 + \sum_{s=1}^\infty \lambda^s E_s
    = E_0 + \lambda \sum_{s=1}^\infty \lambda^{s-1} \langle \varphi_n | H' | \Psi_{s-1} \rangle
    = \mathcal{E}_n + \lambda \left\langle \varphi_n | H' | \Psi \right\rangle
\end{align*}
Um $\Psi_s$ zu finden, entwickeln wir für $s \geq 1$.
\begin{align*}
    |\Psi_s\rangle &= \sum_{l \neq n} \left\langle \varphi_l | \Psi_s \right\rangle |\varphi_l\rangle
    \\
    \left\langle \varphi_l | \Psi_s \right\rangle &=
    \frac{1}{\E_n - \E_l} \eckigeklammer{\left\langle \varphi_l | H' | \Psi_{s-1} \right\rangle
        - \sum_{j=1}^s E_j \left\langle \varphi_l | \Psi_{s-j} \right\rangle}
\end{align*}
Somit gilt:
\begin{align*}
    E_s &= \left\langle \varphi_n | H' | \Psi_{s-1} \right\rangle
    \\
    | \Psi_s \rangle &= \sum_{l \neq n} \frac{\left\langle \varphi_l | H' | \Psi_{s-1} \right\rangle - \sum_{j=0}^{s-1} E_j \left\langle \varphi_j | \Psi_{s-j} \right\rangle}{\E_n - \E_l} |\varphi_l \rangle
\end{align*}
Für $s \leq 2$ gilt explizit:
\begin{align*}
    E_0 &= \E_n
    \hspace{5pt} , \hspace{5pt}
    E_1 = \langle n | H' | n \rangle
    \hspace{5pt} , \hspace{5pt}
    E_2 = \sum_{l \neq n} \frac{\abs{\langle l | H' | n \rangle}^2}{\E_n - \E_l}
    \\
    % E_0 = \E_n
    % \hspace{10pt} &, \hspace{10pt}
    | \Psi_0 \rangle &= |n\rangle
    % E_1 = \langle n | H' | n \rangle
    \hspace{5pt} , \hspace{5pt}
    | \Psi_1 \rangle = \sum_{l \neq n} \frac{\langle l | H' | n \rangle}{\E_n - \E_l} | l \rangle
    \\
    % E_2 = \sum_{l \neq n} \frac{\abs{\langle l | H' | n \rangle}^2}{\E_n - \E_l}
    % \hspace{10pt} &, \hspace{10pt}
    | \Psi_2 \rangle &= \sum_{l \neq n} \eckigeklammer{
        \sum_{k \neq n} \frac{\langle l | H' | k \rangle \langle k | H' | n \rangle}{\klammer{\E_n - \E_l} \klammer{\E_n - \E_k}}
        - \frac{\langle l | H' | n \rangle \langle n | H' | n \rangle}{\klammer{\E_n - \E_l}^2}
    } |l\rangle
\end{align*}

Die Normierungsbedingung ist
\begin{align*}
    \left\langle \Psi | \Psi \right\rangle = 1 + \lambda^2
    \sum_{l \neq n} \frac{\abs{\left\langle l | H' | n \right\rangle}^2}{\klammer{\E_n - \E_l}^2}
\end{align*}

\subsection{Entartete zeit-unabhängige ST}

\subsubsection{Aufhebung einer zweifachen Entartung zu erster Ordnung}

Angenommen $E_0 = \E_n = \E_m$ und $\E_k \neq E_n$ für $k \neq n,m$. Sei $\varphi_l$
eine Orthonormalbasis der zugehörigen Eigenvektoren. In diesem Fall ist
\begin{align*}
    \Psi_0 = a_m | m \rangle + a_n | n \rangle
    \hspace{5pt} , \hspace{5pt}
    a_m , a_n \in \C
\end{align*}
Aus ST erster Ordnung und durch bilden der Matrixelemente mit $\langle m |$ und
$\langle n |$ folgen die folgenden Gleichungen:
\begin{align*}
    \klammer{\langle m | H' | m \rangle - E_1} a_m + \langle m | H' | n \rangle a_n = 0
    \\
    \langle n | H' | m \rangle a_m + \klammer{ \langle n | H' | n \rangle - E_1 } a_n = 0
\end{align*}
Wir definieren nun
\begin{align*}
    \E_m' = \langle m | H' | m \rangle
    \hspace{5pt} , \hspace{5pt}
    \E_n' = \langle n | H' | n \rangle
    \hspace{5pt} , \hspace{5pt}
    \delta = \langle m | H' | n \rangle
    \hspace{5pt} , \hspace{5pt}
    \delta'^\ast = \langle n | H' | m \rangle
\end{align*}
Somit lässt sich obiges GLS schreiben als
\begin{align*}
    \begin{pmatrix}
        \E_m' - E_1 & \delta' \\
        \delta'^\ast & \E_n' - E_1
    \end{pmatrix} \begin{pmatrix}
        a_m \\ a_n
    \end{pmatrix} = 0
\end{align*}
Eine Lsg $\Psi_0 \neq 0$ existiert falls die Determinante gleich Null ist.
Somit folgt für $E_1$:
\begin{align*}
    E_1^\pm = \frac{\E_m' - \E_n'}{2} \pm
    \sqrt{\klammer{\frac{\E_m' - \E_n'}{2}}^2 + \abs{\delta'}^2}
\end{align*}
Für die zugehörigen EV muss gelten $\frac{a_m^\pm}{a_n^\pm} = \frac{\delta'}{E_1^\pm - \E_m'}$.
In niedrigster Ordnung ist $E_0 = \E$ und $| \Psi_0^\pm \rangle = a_m^\pm |m\rangle
+ a_n^\pm |n\rangle$. In nächster Ordnung ist $E_1 = E_1^\pm$. Falls $\E_m' \neq
\E_n'$ oder $\delta' \neq 0$ ist $E_1^+ \neq E_1^-$ und die Entartung ist aufgehoben.
Nun betrachten wir noch die Korrektur $| \Psi_1^\pm$ für den EV von $H$. Wir
berechnen die Matrixelemente mit $\langle l |$, $l \neq m,n$.
\begin{align*}
    \langle l | H_0 - E_0 | \Psi_1^\pm \rangle &= \langle l | E_1^\pm - H' | \Psi_0^\pm \rangle
    \\
    \klammer{\E_l - \E} \langle l | \Psi_1^\pm \rangle &=
        \underbrace{E_1^\pm \langle \Psi_0^\pm \rangle}_{=0}
        - \langle l | H' | \Psi_0^\pm \rangle
\end{align*}
Wobei $\E \equiv \E_m = \E_n$. Die Komponenten von $\Psi_1^\pm$ im orthogonalen
Komplement von span$\geschwungeneklammer{ |m\rangle , |n\rangle}$ ist
\begin{align*}
    | P \Psi_1^\pm \rangle = \sum_{l \neq m,n} \frac{\langle l | H' | \Psi_0^\pm}{\E - \E_l} | l \rangle
    \hspace{10pt} \text{wobei} \hspace{10pt}
    P = \mathds{1} - |m\rangle\langle m| - |n\rangle\langle n|
\end{align*}
$P$ ist der Projektor auf diesen Unterraum. Es gilt $\langle \Psi_n^\pm |
\Psi_0^\pm \rangle = 0$ für $n \geq 1$. Wir wollen nun $\langle \psi_1^\pm |
\Psi_0^\mp \rangle$ bestimmen. Wir betrachten zweite Ordnung ST.
\begin{align*}
    \underbrace{\langle \Psi_0^- | H_0 - E_0 | \Psi_2^+ \rangle}_{=0}
    &= \langle \Psi_0^- | E_1^+ - H' | \Psi_1^+ \rangle +
    E_2^+ \underbrace{\langle \Psi_0^- | \Psi_0^+ \rangle}_{=0}
    \\
    &\Rightarrow
    \langle \Psi_1^+ | E_1^+ - H' | \Psi_0^- \rangle = 0
\end{align*}
Andererseits
\begin{align*}
    \klammer{E_1^+ - H'} | \Psi_0^- \rangle &=
    \klammer{E_1^+ - P H'} | \Psi_0^- \rangle - \klammer{\mathds{1} - P} H' |\Psi_0^- \rangle
    \\
    &= \klammer{E_1^+ - E_1^-} | \Psi_0^- \rangle - P H' |\Psi_0' \rangle
\end{align*}
Wir multiplizieren mit $\langle \Psi_1^+ |$ und finden
\begin{align*}
    0 &= \klammer{E_1^+ - E_1^-} \langle \Psi_0^- | \Psi_1^+ \rangle
    - \langle \Psi_0^- | H' P | \Psi_1^+ \rangle
    \\
    \langle \Psi_0^- | \Psi_1^+ \rangle &=
    \sum_{l \neq m,n} \frac{\langle \Psi_0^- | H' | l \rangle \langle | H' | \Psi_0^+ \rangle}{\klammer{\E - \E_l} \klammer{E_1^+ - E_1^-}}
\end{align*}
Somit folgt:
\begin{align*}
    E_0 = \E
    \hspace{10pt} , \hspace{10pt}
    |\Psi_0^\pm \rangle = a_m^\pm |m\rangle + a_n^\pm |n\rangle
    \hspace{10pt} , \hspace{10pt}
    E_1 = E_1^\pm
    \\
    |\Psi_1^\pm \rangle = \sum_{l \neq m,n} \eckigeklammer{
        \frac{\langle \Psi_0^\mp | H' | l \rangle \langle l | H' | \Psi_0^\pm \rangle}{\klammer{E_1^+ - E_1^-} \klammer{\E - \E_l}} | \Psi_0^\mp \rangle
        + \frac{\langle l | H' | \Psi_0^\pm \rangle}{\E - \E_l} |l\rangle
    }
\end{align*}

\subsubsection{Der allgemeine Fall}

Betrachte den Eigenraum zu $\E_n$ der Dimension $k$. Sei $\geschwungeneklammer{
\varphi_n^1 , \dots , \varphi_n^k}$ eine Orthonormalbasis dieses Eigenraumes.
Für $\Psi_0$ machen wir den Ansatz
\begin{align*}
    \Psi_0 = \sum_{j=1}^k a_n^j \varphi_n^j
\end{align*}
Nun ist das EW-Problem gegeben durch
\begin{align*}
    \begin{pmatrix}
        \langle \varphi_n^1 | H' | \varphi_n^1 \rangle - E_1 & \langle \varphi_n^1 | H' | \varphi_n^2 \rangle & \dotsb & \langle \varphi_n^1 | H' | \varphi_n^k \rangle \\
        \langle \varphi_n^2 | H' | \varphi_n^1 \rangle & \klammer{\varphi_n^2 | H' | \varphi_n^2 \rangle - E_1} & \dotsb & \langle \varphi_n^2 | H' | \varphi_n^k \rangle \\
        \vdots & \ddots & \ddots & \vdots \\
        \langle \varphi_n^k | H' | \varphi_n' \rangle & \dotsb & \dotsb & \langle \varphi_n^k | H' | \varphi_n^k \rangle - E_1
    \end{pmatrix}
    \begin{pmatrix}
        a_n^1 \\ a_n^2 \\ \vdots \\ a_n^k
    \end{pmatrix}
    =0
\end{align*}
Wir bezeichnen die Eigenwerte mit $E_1^\alpha = \langle n^\alpha | H' | n^\alpha \rangle$,
für $\alpha = 1,\dots,k$ wobei der zugehörige EV gegeben ist durch
\begin{align*}
    |n^\alpha \rangle = \sum_{l=1}^k a_n^l (\alpha) |\varphi_n^l \rangle
\end{align*}
Zu zweiter Ordnung für $E$ findet man
\begin{align*}
    E = \E_n + \langle n^\alpha | H' | n^\alpha \rangle
    + \sum_{\E_l \neq E_n} \frac{\abs{\langle l | H' | n^\alpha \rangle}^2}{\E_n - \E_l}
\end{align*}
und in erster Ordnung für $\Psi$
\begin{align*}
    |\Psi^\alpha \rangle = |n^\alpha \rangle + \sum_{\alpha' \neq \alpha}
    \sum_{\E_l \neq \E_n} \frac{\langle n^\alpha | H' | l \rangle \langle | H' | n^\alpha \rangle}{\klammer{\E_n - \E_l} \klammer{E_1^\alpha - E_1^{\alpha'}}}
    + \sum_{\E_l \neq \E_n} \frac{\langle l | H' | n^\alpha \rangle}{\E_n - \E_l} |l\rangle
\end{align*}
Hier wurde angenommen, dass die Entartung in erster Ordnung aufgehoben wurde, d.h.
$E_1^\alpha \neq E_1^\beta$ für $\alpha \neq \beta$.

\subsubsection{Aufhebung der Entartung zu zweiter Ordnung}

Die Entartung für den Fall der zweifachen Entartung ist nicht aufgehoben, falls
\begin{align*}
    \langle m | H' | m \rangle = \langle n | H' | n \rangle
    \hspace{10pt} , \hspace{10pt}
    \langle m | H' | n \rangle = 0
\end{align*}
Wir betrachten ST in zweiter Ordnung und nehmen Matrixelemetne mit $\langle m |$
und $\langle n |$.
\begin{align*}
    \langle m | H_0 - E_0 | \Psi_2 \rangle = \langle m | E_1 - H' | \Psi_1 \rangle
    + E_2 \langle m | \Psi_0 \rangle
\end{align*}
Wir benutzen die Resultate aus erster Ordnung
\begin{align*}
    |\Psi_0\rangle = a_m |m\rangle + a_n|n\rangle
    \hspace{10pt} &, \hspace{10pt} E_0 = \E = \E_n = \E_m
    \hspace{10pt} , \hspace{10pt} \langle n | H' | m \rangle = \E' \delta_{nm}
    \\
    P | \Psi_1 \rangle &= \sum_{l \neq m,n}
    \frac{a_m \langle l | H' | m \rangle + a_n \langle l | H' | n \rangle}{\E - \E_l} |l\rangle
\end{align*}
Daraus folgt:
\begin{align*}
    0 = \langle m | E_1 - H' | P \Psi_1 \rangle +
    \underbrace{\langle m | E_1 - H' | \klammer{\mathds{1} - P} \Psi_1 \rangle}_{=0}
    + E_2 \langle m | \Psi_0 \rangle
\end{align*}
Somit $E_2 a_m = \langle m | H' | P \Psi_1 \rangle$ und schliesslich
\begin{align*}
    \Biggl[\underbrace{\sum_{l \neq m,n} \frac{\abs{\langle l | H' | m \rangle}^2}{\E - \E_l}}_{=\E_m'} - E_2 \Biggr] a_m
    + \underbrace{\sum_{l \neq m,n} \frac{\langle m | H' | l \rangle \langle l | H' | n \rangle}{\E - \E_l}}_{= \delta'} a_n = 0
\end{align*}
Analog folgt für $\langle n |$:
\begin{align*}
    \klammer{\E_n' - E_2} a_n + \delta'^\ast a_m = 0
    \hspace{10pt} , \hspace{10pt}
    \E_n' = \sum_{l \neq m,n} \frac{\abs{\langle l | H' | n \rangle}^2}{\E - \E_l}
\end{align*}
Die Entartung wird zu zweiter Ordnung aufgehoben, falls entweder $\E_n' \neq \E_m'$
oder $\delta' \neq 0$.

\subsubsection{Der Stark Effekt}
Betrachte ein elektrisches Feld $\vec{\E} = (0,0,\E)$ und die Störung
$H' = e \E z$. Betrachte die Aufhebung der Entartung im $n=2$ Niveau
($|n,l,m\rangle = |2s_0\rangle,|2p_1\rangle,|2p_0\rangle,|2p_{-1}\rangle$).
Im Prinzip müsste man eine $4 \times 4$ Matrix betrachten, da aber $[L_z,z] =0$
vertauscht $H'$ mit $L_z$ und es gilt:
\begin{align*}
    0 = \langle m' | [H',L_z] | m \rangle = (m-m') \langle m' | H' | m \rangle
\end{align*}
Somit $\langle m' | H' | m \rangle = 0$ falls $m \neq m'$. Somit bleiben nur
die Matrixelemente $\langle 2s_0 | H' | 2p_0 \rangle = \delta'$ übrig und die
Diagonaleinträge. Die Kugelfkt. sind Eigenfkt. des Paritätsoperators $P$ mit
Parität $(-1)^l$. Andererseits hat der Operator $z$ Parität $-1$, und somit
\begin{align*}
    (-1)^l \langle Y_{lm} | H' | Y_{lm} \rangle
    &= \langle Y_{lm} | H' P | Y_{lm} \rangle
    = - \langle Y_{lm} | P H' | Y_{lm} \rangle
    \\
    &= - (-1)^l \langle Y_{lm} | H' | Y_{lm} \rangle
\end{align*}
Daher verschwinden alle Diagonaleinträge der Matrix $H'$. Die Matrixelemente
von $H'$ auf dem entarteten Unterraum sind also durch folgende Matrix
beschrieben:
\begin{align*}
    \begin{pmatrix}
        0 & 0 & \delta & 0 \\
        0 & 0 & 0 & 0 \\
        \delta^\ast & 0 & 0 & 0 \\
        0 & 0 & 0 & 0
    \end{pmatrix}
    \hspace{10pt} , \hspace{10pt}
    \delta = \langle 2s_0 | H' | 2p_0 \rangle = - 3 a_0 e \E
\end{align*}
Es bleibt die $2 \times 2$ Matrix zu diagonalisieren
\begin{align*}
    3 a_0e \E \begin{pmatrix}
        0 & -1 \\ -1 & 0
    \end{pmatrix}
    \hspace{10pt} \Rightarrow \hspace{10pt}
    \text{EW} = \mp 3 a_0 e \E
    \hspace{5pt} , \hspace{5pt}
    \text{EV} = \frac{1}{\sqrt{2}} \begin{pmatrix}
        1 \\ \pm 1
    \end{pmatrix}
\end{align*}

\subsection{Zeit-abhängige Störungstheorie}
Wir nehmen im folgenden an, dass $H' = H'(t)$ zeit-abhängig ist, aber der
ungestörte Hamiltonoperator $H_0$ immernoch zeit-unabhängig ist.

\subsubsection{Der Propagator der Quantenmechanik}

Im Schrödingerbild sind Operatoren zeit-unabhängig. Die Dynamik des Systems
wird durch die Zeitabhängigkeit der Zustandsvektoren $\Psi(t)$ beschrieben.
Dies wird durch die SG $i \hbar \partial_t \Psi(t) = H \Psi(t)$ bestimmt. Wir
können die Lsg der SG durch einen Propagator beschreiben. Dazu definieren wir
einen Operator $U(t,s)$ (den Propagator), der folgende Bedingungen erfüllt.
\begin{enumerate}[(i)]
    \item $U(t,t) = \mathds{1}$
    \item Additivität: es gilt $U(t,s) U(s,r) = U(t,r)$
    \item Der Operator $U(t,s)$ erfüllt die DGL $i \hbar \partial_t U(t,s) = H U(t,s)$
\end{enumerate}
Es gilt $\Psi(t) = U(t,s) \Psi(s)$. Falls $H$ zeit-unabhängig ist, so kann man
$U(t,s)$ schreiben als
\begin{align*}
    U(t,s) = \exp \klammer{-i H \frac{(t-s)}{\hbar}}
\end{align*}
Da $H$ selbst-adjungiert ist, folgt dass $U(t,s)$ unitär ist. Im allgemeinen
Fall lässt sich $U(t,s)$ wie folgt definieren.
\begin{align*}
    U(t,s) &= \mathds{1} \sum_{n=1}^\infty U^{(n)}(t,s)
    \\
    U^{(n)} (t,s) &= \klammer{\frac{1}{i \hbar}}^n \int_s^t dt_1
        \int_s^{t_1} dt_2 \dotsb \int_{s}^{t_{n-1}} dt_n
        H(t_1) H(t_2) \dotsb H(t_n)
\end{align*}
Es gilt $t \geq t_1 \geq t_2 \geq \dotsb \geq s$. Wir definieren den
Zeitordnungs-Operator $\T$:
\begin{align*}
    \T \klammer{A(t_1) B(t_2)} = \begin{cases}
        A(t_1) B(t_2) \hspace{10pt} &t_1 > t_2
        \\
        B(t_2) A(t_1) \hspace{10pt} &t_2 > t_1
    \end{cases}
\end{align*}
Die Verallgemeinerung auf $n$ Operatoren ist
\begin{align*}
    \T \klammer{\prod_{i=1}^n A_i (t_i)} = A_{\pi_1} (t_{\pi_1}) \dotsb A_{\pi_n} (t_{\pi_n})
\end{align*}
wobei $\pi$ jene Permutation ist, welche die Zeiten ordnet $t_{\pi_1} \geq t_{\pi_2}
\geq \dotsb \geq t_{\pi_n}$.
Mit dieser Notation können wir schreiben:
\begin{align*}
    U(t,s) = \T \exp \eckigeklammer{- \frac{i}{\hbar} \int_s^t dt' H(t')}
\end{align*}

\subsubsection{Das Heisenbergbild}

Im Heisenbergbild sind die Operatoren zeit-abhängig und nicht die Zustände. Die
beiden Beschreibungen sind äquivalent und durch den unitären Operator $U(t,s)$
miteinander verbunden. Konkret definieren wir die Zustände im Heisenbergbild durch
\begin{align*}
    \Psi_H \equiv \Psi(t_0)
    \hspace{10pt} , \hspace{10pt}
    \Psi_H = U(t_0,t) \Psi(t)
\end{align*}
Die Observablen im Schrödingerbild werden wie folgt zu den Observablen im
Heisenbergbild transformiert.
\begin{align*}
    A_H = U(t_0,t) A U(t,t_0)
\end{align*}
Diese definition garantiert $\klammer{A \Psi(t)}_H = A_H \Psi_H$. Insbesondere
\begin{align*}
    \langle A \rangle = \langle \Psi(t) | A | \Psi(t) \rangle
    = \langle \Psi_H | A_H (t) | \Psi_H \rangle
\end{align*}
Weiter gilt
\begin{align*}
    U(t,s)^\dagger = U(t,s{-1} = U(s,t)
\end{align*}
Die Bew.gl. im Heisenbergbild ist eine Bew.gl. für die Operatoren:
\begin{align*}
    i \hbar \frac{d}{dt} = \eckigeklammer{A_H,H_H} + i \hbar \partial_t A_H
\end{align*}
Falls $\partial_t H = 0$, so ist $[H,U(t_0,t)] = 0$ und damit
$H_H(t) = H_H = H$. Dann ist $H_H$ zeit-unabhängig. Für die Operatoren gilt aber
\begin{align*}
    A_H (t) = e^{i H (t-t_0) / \hbar} A e^{-i H (t-t_0) / \hbar}
\end{align*}
$A_H$ ist eine Erhaltungsgrösse, falls $\partial_t A = 0$ und $[H,A] = 0$.
Der Kommutator ist also genau das quantenmechanische Analogon von der
Poissonklammer. Für zwei Operatoren gilt
\begin{align*}
    \geschwungeneklammer{A,B}
    \hspace{2pt} \Rightarrow \hspace{2pt}
    - \frac{i}{\hbar} \eckigeklammer{A,B}
    \hspace{10pt} \text{somit} \hspace{10pt}
    \geschwungeneklammer{q,p} = 1
    \hspace{2pt} \Rightarrow \hspace{2pt}
    - \frac{i}{\hbar} \eckigeklammer{q,p} = 1
\end{align*}

\subsubsection{Das Wechselwirkungsbild}

Wir betrachten den Fall der zeit-abhängigen ST, also $H = H_0 + H'(t)$. Hierbei
ist $H_0$ zeit-unabhängig und exakt lösbar und $H'(t)$ ist eine kleine Störung.
Im Wechselwirkungsbild sind sowohl die Zustände als auch die Opservablen
zeit-abhängig. Wir betrachten die unitäre Transformation mit $U_0 (t,t_0)$,
wobei $U_0$ der Propagator des ungestörten Problems ist. Dann definieren wir
\begin{align*}
    \Psi_D (t) = U_0 (t_0,t) \Psi(t) = U_0 (t_0,t) U(t,t_0) \Psi(t_0)
\end{align*}
wobei $U$ der Propagator für $H = H_0 + H'$ ist. Entsprechend:
\begin{align*}
    A_D (t) = U_0(t_0,t) A U_0(t,t_0)
    \hspace{10pt} \text{wobei} \hspace{10pt}
    U_0 (t_0,t) = e^{i H (t-t_0) / \hbar}
\end{align*}
Die Zeitentwicklung der Zustände $\Psi_D$ ist dann gegeben durch
\begin{align*}
    \Psi_D (t) = U_D (t,t_0) \Psi_D (t_0)
    \hspace{10pt} \text{wobei} \hspace{10pt}
    U_D(t,t_0) = U_0 (t_0,t) U(t,t_0)
\end{align*}
Der zugehörige Hamiltonoperator ist
\begin{align*}
    i \hbar \partial_t U_D = H_D' U_D
    \hspace{10pt} \text{wobei} \hspace{10pt}
    H_D' = U_0(t_0,t) H' U_0(t,t_0)
\end{align*}
Als Neumann-Reihe gilt daher
\begin{align*}
    U_D (t,t_0) = \T \exp \eckigeklammer{- \frac{i}{\hbar} \int_{t_0}^t dt' H_D'(t')}
\end{align*}

\subsubsection{Erste Ordnung ST}

Wir betrachten nun den Fall wo $H(t) = H_0 + H'(t)$ und $H'(t) = 0$ für
$t < t_0$. Wir nehmen oBdA an, dass zur Zeit $t = t_0$, das System im
Eigenzustand $| i \rangle$ von $H_0$ mit EW $\E_i$ ist. Wir wollen die Wsk
berechnen, das das System für $t>t_0$ im Zustand $|f\rangle$ ist. Wir nehmen
weiter an, dass $\langle f | i \rangle = 0$. Da $\Psi(t) = U_D (t,t_0) |i\rangle$,
folgt
\begin{align*}
    \langle f | \Psi (t) \rangle &=
    \langle f | U_D (t,t_0) | i \rangle
    \simeq \langle f | i \rangle - \frac{i}{\hbar} \langle f | \int_{t_0}^t dt' H_D' (t') | i \rangle
    \\
    &= - \frac{i}{\hbar} \int_{t_0}^t dt' \langle f | e^{i \E_f (t'-t_0) / \hbar} H'(t') e^{-i \E_i (t' - t_0) / \hbar} | i \rangle
\end{align*}
Für die Übergangswahrscheinlichkeit erhalten wir zu erster Ordnung also
\begin{align*}
    P_{i \rightarrow f} = \frac{1}{\hbar^2} \abs{
        \int_{t_0}^t dt' e^{i \klammer{\E_f - \E_i} t' / \hbar} \langle f | H'(t') | i \rangle
    }^2
\end{align*}
Wir betrachten das Bsp $H'(t) = V \theta(t - t_0)$. Wir wählen oBdA $t_0 = 0$ und
definieren $\omega_{if} = \klammer{\E_f - \E_i} / \hbar$. Dann gilt:
\begin{align*}
    P_{i \rightarrow f} = \abs{\frac{e^{i \omega_{if} t} - 1}{\hbar \omega_{if}} \langle f | V | i \rangle}
    = \klammer{\frac{\sin(\omega_{if} t / 2)}{\hbar \omega_{if} / 2}}^2 \abs{\langle f | V | i \rangle}^2
\end{align*}
Diese Übergangswahrscheinlichkeit hat Nullstellen bei $\omega_{if} t = 2 \pi n$.
Für $\omega_{if} \rightarrow 0$ und somit für $t \rightarrow \infty$ verhällt sich
der zeitabhängige Vorfaktor wie
\begin{align*}
    \frac{\klammer{\omega_{if} t / 2}^2}{\klammer{\hbar \omega_{if} / 2}^2}
    \approx \klammer{\frac{t}{\hbar}}^2
\end{align*}
Für $t \rightarrow \infty$ erhalten wir einen scharfen Peak bei $\omega_{if} = 0$
mit Gewicht proportional zu $t$. Die Interpretation ist die folgende: Nach
Einschalten der Störung können prinzipiell Übergänge zu allen Energien erfolgen.
Die relevanten Übergänge, jene welche eine grosse Wsk haben und nicht in der Zeit
oszillieren, sind diejenigen welche die Energie erhalten, $\omega_{if} = 0$,
d.h. $\E_i = \E_f$. Das Gewicht dieser energieerhaltenden Prozesse nimmt linear
mit der Zeit zu.
\begin{align*}
    \int_{- \frac{2 \pi}{t}}^{\frac{2 \pi}{t}} P_{i \rightarrow f} (\omega) d \omega
    = \frac{2 t}{\hbar} \int_{-\pi}^\pi du \frac{\sin^2(u)}{u^2} \abs{\langle f | V | i \rangle}^2
    \sim t
\end{align*}
Wir betrachtenn nun ein Kontinuum von Zuständen oder eine Gruppe von Finalzuständen
welche dicht liegen. Die Dichte dieser Zustände im Energieraum sei beschrieben durch
$\rho(\E_f)$. D.h. wir können die Summe über Zustände $|f\rangle$ ersetzen durch
Integrale über $\E_f$: $\sum_f \rightarrow \int d \E_f \rho(\E_f)$. Die Wsk für
einen Übergang in einen Zustand um $|f\rangle$ herum ist dann
\begin{align*}
    \Gamma t = \sum_f P_{i \rightarrow f} (t)
    = \int d \E_f \rho(\E_f) \frac{\sin^2(\omega_{if} t / 2)}{\klammer{\hbar \omega_{if} / 2}^2} \abs{\langle f | V | i \rangle}^2
    = \frac{2 \pi}{\hbar} t \rho(\E_f) \abs{\langle f | V | i \rangle}^2 \Big|_{\E_i = \E_f}
\end{align*}
Hierbei wurde benutzt, dass $\sin^2(\omega t) / \pi t \omega^2 \rightarrow
\delta(\omega)$ für $t \rightarrow \infty$, sowie $d \E_f = \hbar d \omega_{if}$.
Somit finden wir die Übergangsrate $dP_{i \rightarrow f} / dt = \Gamma$
\begin{align*}
    \Gamma = \frac{2 \pi}{\hbar} \abs{\langle f | V | i \rangle}^2 \rho(\E_f)
\end{align*}
Diese Relation nennt man Fermis Goldene Regel. Hierbei ist $2 \pi \hbar / \Delta \E_f
< t \ll 2 \pi \hbar / \delta \epsilon$. Die Zeitrestriktionen haben folgenden Ursprung:
\begin{itemize}
    \item Der Peak in $P_{i \rightarrow f}$ muss innerhalb der Gruppe von Zuständen
        um $f$ liegen. Wir bezeichnen die Breite dieser Gruppe mit $\Delta \E_f$,
        insbesondere sei das Matrixelement $\abs{\langle i | V | f \rangle}$
        innerhalb von $\Delta \E_f$ in etwa konstant. Dann muss $\Delta \E_f > 2 \pi
        \hbar / t$ sein.
    \item Die Dichte unter dem Peak muss genügend gross sein, d.h., die Energieseparation
        $\delta \epsilon$ zwischen Zuständen klein, s.d. viele Zustände innerhalb des
        Zentralen Peaks zu liegen kommen, $\delta \epsilon \ll 2 \pi \hbar / t$.
\end{itemize}