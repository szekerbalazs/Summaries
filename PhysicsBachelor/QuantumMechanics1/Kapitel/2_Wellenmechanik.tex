\section{Wellenmechanik}

\subsection{Die Schrödinger-Gleichung}

Nach de Broglie kann man einem Teilchen mit Impuls $p^\mu = (E,\vec{p})$ eine
Welle zuordnen deren Wellenlänge $\lambda$ und Kreisfrequenz $\omega$ gemäss
(\ref{deBroglie}) gegeben sind. Eine Welle mit diesen Parametern ist die
Funktion
\begin{align*}
    \Psi(x,t) = \Psi(0,0) e^{\frac{2 \pi i}{\lambda} x} e^{-i \omega t}
\end{align*}
wobei $\Psi(0,0)$ eine Konstante ist. Es gilt:
\begin{align*}
    \frac{\partial}{\partial x} \Psi(x,t)
        &= \frac{2 \pi i}{\lambda} \Psi(x,t)
        = i \frac{p}{\hbar} \Psi(x,t)
    \\
    \frac{\partial}{\partial t} \Psi(x,t)
        &= -i \omega \Psi(x,t)
        = -i \frac{E}{\hbar} \Psi(x,t)
\end{align*}
Somit auch:
\begin{align*}
    p \Psi (x,t) = -i \hbar \frac{\partial}{\partial x} \Psi(x,t)
    \hspace{10pt} , \hspace{10pt}
    E \Psi(x,t) = i \hbar \frac{\partial}{\partial t} \Psi(x,t)
\end{align*}
Für ein freies Teilchen mit Energie $E$ und und Impuls $p$ gilt $E = \frac{p^2}{2m}$.
Es folgt:
\begin{align*}
    i \hbar \frac{\partial}{\partial t} \Psi(x,t)
    = - \frac{\hbar^2}{2m} \frac{\partial^2}{\partial x^2} \Psi(x,t)
\end{align*}
Falls das Teilchen nicht frei ist, sondern sich im Einfluss des Potentials
$V(x,t)$ bewegt, folgt die (zeit-abhängige) Schrödinger-Gleichung:
\begin{align}\label{ZeitunabhaengigeSG}
    i \hbar \frac{\partial}{\partial t} \Psi(x,t) =
    \klammer{- \frac{\hbar^2}{2 m} \frac{\partial^2}{\partial x^2} + V(x,t)} \Psi(x,t)
    \equiv H \Psi(x,t)
\end{align}
$H$ ist der Hamiltonoperator. $\Psi(x,t)$ ist eine komplexe Funktion von
$(x,t)$. $\abs{\Psi(x,t)}^2$ ist die Wahrscheinlichkeit mit der sich das
Teilchen zur Zeit $t$ am Punkt $x$ befindet. Es gilt die
Normalisierungsbedingung:
\begin{align}\label{NormalisierungPsi}
    \intii dx \ \abs{\Psi(x,t)}^2 = 1 \ \forall t
\end{align}
Die wichtigsten Postulate im Überblick:
\begin{enumerate}[(i)]
    \item Die Wahrscheinlichkeit das Teilchen zur Zeit $t$ am Ort $x$
        anzutreffen, ist durch das Normquadrat $\abs{\Psi(x,t)}^2$ der
        Wellenfunktion gegeben.
    \item Die Wellenfunktion $\Psi(x,t)$ ist für jedes $t$ quadrat-integrabel
        in $x$; insbesondere ist sie gemässt (\ref{NormalisierungPsi}) normalisiert.
    \item Die Wellenfunktion $\Psi(x,t)$ ist eine komplexe Funktion, die der
        zeit-unabhängigen SG (\ref{ZeitunabhaengigeSG}) genügt.
\end{enumerate}

Klassische Observable werden durch Operatoren dargestellt. Sie sind im
Allgemeinen nicht vertauschbar. Der Impulsoperator ist gegeben als:
\begin{align*}
    p = -i \hbar \frac{\partial}{\partial x}
\end{align*}

\subsection{Wahrscheinlichkeitsstrom und die Kontinuitätsgleichung}

Wie bereits erwähnt beschreibt $\abs{\Psi(x,t)}^2 \equiv \rho(x,t)$ eine
Wahrscheinlichkeitsdichte. Durch ableiten erhalten wir eine Kontinuitätsgleichung:
\begin{align*}
    \frac{d}{dt} \rho(x,t) + \frac{\partial}{\partial x} j(x,t) = 0
\end{align*}
wobei $j(x,t)$ der Wahrscheinlichkeitsstrom ist.
\begin{align*}
    j(x,t) = \frac{i \hbar}{2m} \klammer{
        \Psi(x,t) \frac{\partial}{\partial x} \Psi(x,t)^\ast
        - \Psi(x,t)^\ast \frac{\partial}{\partial x} \Psi(x,t)
    }
\end{align*}
Wir integrieren die Wahrscheinlichkeitsdichte über ein kleines Intervall
\begin{align*}
    P(x_0;\delta) &= \int_{x_0 - \delta}^{x_0 + \delta} dx \ \rho(x,t)
    \\ \Rightarrow
    \frac{d}{dt} P(x_0;\delta)
    &= j(x_0 - \delta,t) - j(x_0 + \delta,t)
\end{align*}
Insbesondere folgt wegen $j \rightarrow 0$ für $x \rightarrow \pm \infty$:
\begin{align*}
    \frac{d}{dt} \intii dx \ \rho(x,t)
    = \eckigeklammer{- j(x,t)}_{-\infty}^\infty
    = 0
\end{align*}

\subsection{Das Ehrenfest'sche Theorem}

Der Erwartungswert einer Ortsmessung ist
\begin{align*}
    \langle x \rangle = \intii dx \ x \ \abs{\Psi(x,t)}^2
\end{align*}
Mit einigem Rechenaufwand bekommt man:
\begin{align*}
    \frac{d}{dt} \langle x \rangle = \frac{1}{m} \langle p \rangle
    \hspace{10pt} , \hspace{10pt}
    m \frac{d^2}{dt^2} \langle x \rangle = - \langle \partial_x V(x,t) \rangle
\end{align*}
Dies wird Ehrenfest'sches Theorem genannt.

\subsection{Realität der physikalischen Observablen}

Es gilt $\langle p \rangle^\ast = \langle p \rangle$ und $\langle H \rangle^\ast
= \langle H \rangle$ und somit $\langle p \rangle \in \R$ und $\langle H \rangle
\in \R$.

\subsection{Zeit-unabhängige Schrödinger-Gleichung}

Wir betrachten nun den Fall wo das Potential $V(x,t) = V(x)$ zeitunabhängig ist.
In diesem Fall können wir einen Separationsansatz machen: $\Psi(x,t) = \psi(x) \chi(t)$.
Die Schrödinger-Gleichung impliziert dann:
\begin{align*}
    \psi(x) i \hbar \dot{\chi}(t) &=
    \chi(t) \klammer{-\frac{\hbar^2}{2 m} \psi''(x) + V(x) \psi(x)}
    \\
    \Rightarrow
    i \hbar \frac{\dot{\chi}(t)}{\chi(t)} &= \frac{1}{\psi(x)}
    \klammer{-\frac{\hbar^2}{2m} \psi''(x) + V(x) \psi(x)} = E
\end{align*}
Wobei $E$ eine Konstante ist und der Energie entspricht. Es folgt:
\begin{align}\label{EWGl}
    i \hbar \dot{\chi}(t) = E \chi(t)
    \hspace{10pt} , \hspace{10pt}
    H \psi(x) = E \psi(x) 
\end{align}
wobei $H$ der Hamiltonoperator ist. Es folgt:
\begin{align*}
    \chi(t) = \exp \klammer{- \frac{i E}{\hbar} t}
\end{align*}
Falls $\psi(x)$ die Eigenwertgleichung in (\ref{EWGl}) erfüllt, dann löst
\begin{align*}
    \Psi(x,t) = \psi(x) \exp \klammer{- i \frac{E}{\hbar} t}
\end{align*}
die (zeit-abhängige) Schrödinger-Gleichung. Die Eigenvektorgleichung
\begin{align*}
    H \psi(x) = E \psi(x)
    \hspace{10pt} \text{ mit } \hspace{10pt}
    H = \klammer{- \frac{\hbar^2}{2m} \frac{\partial^2}{\partial x^2} + V(x)}
\end{align*}
wird zeit-unabhängige Schrödinger-Gleichung genannt. Da $\abs{\Psi(x,t)} =
\abs{\psi(x,t)}$ muss gelten:
\begin{align*}
    \intii dx \ \abs{\psi(x)}^2 = 1
\end{align*}
Benutze die zeit-unabhängige Hamilton-Jakobi Gleichung und den folgenden
Ansatz für 3 Dimensionen:
\begin{align*}
    \Psi(\vec{x},t) = \psi(\vec{x}) e^{- i \frac{E}{\hbar} t}
    \hspace{10pt} , \hspace{10pt}
    \psi(\vec{x}) = A(\vec{x}) e^{i \frac{S(\vec{x})}{\hbar}}
\end{align*}
wobei $A(\vec{x})$ und $S(\vec{x})$ reelle Funktionen sind. Die 3D zeit-unabhängige
SG lautet:
\begin{align*}
    \eckigeklammer{- \frac{\hbar^2}{2 m} \Laplace + V(\vec{x})}\psi(\vec{x}) = E \psi(\vec{x})
\end{align*}
Wenn wir nach Real- und Imaginärteil zerlegen, erhalten wir:
\begin{align*}
    \frac{\klammer{\vec{\nabla} S}^2}{2 m} + \klammer{V(\vec{x}) - E}
    = \frac{\hbar^2}{2m} \frac{\Laplace A}{A}
    \hspace{15pt} , \hspace{15pt}
    (\vec{\nabla} A) \cdot (\vec{\nabla} S) + \frac{A}{2} \Laplace S = 0
\end{align*}
Im klassischen Limes, wo $\hbar \rightarrow 0$ wird aus der ersten Gleichung:
\begin{align*}
    \frac{\klammer{\vec{\nabla} S}^2}{2 m} + V(\vec{x}) = E
\end{align*}
was gerade die zeit-unabhängige HJ-Gl. ist. Die Phase der Welle $\Psi$ kann
mit der Grüsse $S/\hbar$ identifiziert werden wobei $S$ die erzeugende
Funktion der kanonishcen Transformation ist. Ausserdem gilt: $\vec{p} = m
\dot{\vec{x}} = \vec{\nabla} S$. Die physikalische Bahn ist also immer
Senkrecht zu der Fläche $S =$const. Diese Flächen sind jedoch gerade die
Wellenfronten der Welle $\Psi$, da $S$ die Phase der Welle ist.

\subsection{Energie Eigenzustände}
Bezeichne mit $\psi_n$, $n \in \N$ einen vollständigen Satz normalisierter
Eigenfunktionen des Hamiltonoperators mit Eigenwerten $H \psi_n = E_n \psi_n$.
Nehme an, dass jede quadrat-integrierbare Funktion sich als Lin.komb dieser
$psi_n$ schreiben lässt. Angenommen wir kennen die Wellenfunktion zur Zeit
$t=0$, $\Psi(x,0)$. Wir wollen die Lösung $\forall t \geq 0$ bestimmen. Da
die $\psi_n$ eine Basis des VR bilden gilt:
\begin{align*}
    \Psi(x,0) = \sum_{n=0}^\infty a_n \psi_n (x)
    \hspace{15pt} , \hspace{10pt} a_n \in \C
\end{align*}
Somit folgt für die Zeitentwicklung die allgemeine Lösung:
\begin{align}\label{SolSGgeneral}
    \Psi(x,t) = \sum_{n=0}^\infty a_n \psi_n (x) \exp \klammer{-i \frac{E_n}{\hbar} t}
\end{align}
Dies ist eine Lösung der zeit-abhängigen SG da jeder einzelne Term eine Lsg ist.

\subsection{Energiemessung}
Es gelten die selben Annahmen für die $\psi_n$ wie oben und es seien alle
Eigenwerte diskret, d.h. $E_n \neq E_m$ für $n \neq m$. Es gilt:
\begin{align}\label{deltamn}
    \intii dx \ \psi_n^\ast (x) \psi_m (x) = \delta_{mn}
\end{align}
Denn es gilt:
\begin{align*}
    E_m \int_\R dx \ \psi_n^\ast (x) \psi_m(x)
    = E_n^\ast \int_\R dx \ \psi_n^\ast (x) \psi_m (x)
\end{align*}
Für $m=n$ folgt: $E_n = E_n^\ast$ und somit $E_n \in \R$. Und falls $n \neq m$
folgt aus $E_n - E_m \neq 0$ die Gl. (\ref{deltamn}).
Wenn man die allgemeine Lösung (\ref{SolSGgeneral}) einsetzt im Integral
erhält man für den Erwartungswert des Hamiltonians das folgende:
\begin{align*}
    \langle H \rangle = \int_\R \Psi(x,t)^\ast H \Psi(x,t)
    = \sum_{n=0}^\infty \abs{a_n}^2 E_n
\end{align*}
Wir sehen, dass der Erwartungswert unabhängig ist von der Zeit.
Dieses Resultat lässt sich wie folgt interpretieren: die möglichen Messergebnisse
einer Energiemessung sind die Eigenwerte $E_n$, und die Wahrscheinlichkeit,
mit der der Wert $E_n$ gemessen wird, ist gleich $\abs{a_n}^2$.

Weitere Annahmen/Postulate der Wellenmechanik im Überblick:
\begin{enumerate}[(i)]
    \setcounter{enumi}{3}
    \item Die möglichen Messergebnisse einer Energiemessung eines Systems
        sind die Eigenwerte des Hamilton-Operators $H$.
    \item Falls die Eigenwerte alle voneinander verschieden sind, so ist die
        Wahrscheinlichkeit, den Eigenwert $E_n$ zu messen, gerade
        \begin{align*}
            \abs{a_n}^2 = \abs{\int_{\R} dx \ \psi_n (x)^\ast \Psi(x,t)}^2
        \end{align*}
\end{enumerate}
Es muss weiter gelten: $\sum_{n=0}^\infty \abs{a_n}^2 = 1$. Wir sehen, dass
nur diskrete Energiewerte zulässig sind. Ein weiteres Postulat ist:
\begin{enumerate}[(i)]
    \setcounter{enumi}{5}
    \item Die Energie eines Quantensystems sei zur Zeit $t=t_0$ mit dem
        Ergebnis $E_n$ gemessen worden. Dann 'kollabiert' die Wellenfunktion
        des Systems zu der Eigenfunktion $\psi_n$ mit $H \psi_n = E_n$, d.h.
        es gilt
        \begin{align*}
            \Psi(x,t) = A \psi_n (x) \exp \klammer{-i \frac{E_n}{\hbar} t}
            \hspace{15pt} \forall t > t_0
        \end{align*}
        Hierbei ist $A$ eine Konstante mit $\abs{A} = 1$, und wir haben
        wiederum angenommen, dass die EW von $H$ alle unterschiedlich sind.
\end{enumerate}
Die physikalische Interpretation dieser Tatsache ist, dass es unmöglich ist,
eine Messung durchzuführen, ohne dabei das System zu beeinflussen. Äquivalent:
Nachdem die Energie des Systems einmal als $E_n$ gemessen wurde, wird jede
zukünftige Energiemessung immer wieder $E_n$ finden und zwar mit
Wahrscheinlichkeit $1$. Hierbei ist aber nicht klar wie man eine 'Messung'
präzise definieren soll.