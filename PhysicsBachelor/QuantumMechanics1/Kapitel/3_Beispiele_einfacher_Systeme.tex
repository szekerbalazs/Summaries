\section{Beispiele einfacher Systeme}

Wir betrachten 1-dimensionale Systeme, in denen das Teilchen sich nur in
einer Richtung bewegen kann.

\subsection{Das Teilchen in der 'Box'}
Ein Teilchen, das sich im Intervall $[0,a]$ frei bewegen kann und das Intervall
nicht verlassen kann, wird durch folgendes Potential beschrieben.
\begin{align*}
    V(x) = \begin{cases}
        0 \hspace{10pt} &0 \leq x \leq a \\
        \infty \hspace{10pt} &\text{sonst}
    \end{cases}
\end{align*}
Es folgt, dass $\psi(x) = 0$ für $x \notin [0,a]$. Für $x \in [0,a]$ muss
gelten:
\begin{align}\label{zeitunabhSGohnePot}
    - \frac{\hbar^2}{2m} \frac{d^2 \psi}{d x^2} = E \psi
\end{align}
Weiterhin muss $\psi$ überall stetig sein. Falls dies nicht der Fall wäre,
dann wäre $\psi'' \sim \delta'$ und somit wäre $\psi$ nicht eine Lösung der
SG. Wegen der Stetigkeit muss gelten: $\psi(0) = \psi(a) = 0$. Es folgt die
allgemeine Lsg.
\begin{align*}
    \psi (x) = \begin{cases}
        A \cosh \klammer{\sqrt{2 m \abs{E}} \frac{x}{\hbar}} +
            B \sinh \klammer{\sqrt{2 m \abs{E}} \frac{x}{\hbar}}
            \hspace{10pt} &\text{falls } E < 0
        \\
        A + B x \hspace{10pt} &\text{falls } E = 0
        \\
        A \cos \klammer{\sqrt{2 m \abs{E}} \frac{x}{\hbar}} +
            B \sin \klammer{\sqrt{2m \abs{E}} \frac{x}{\hbar}}
            \hspace{10pt} &\text{falls } E > 0
    \end{cases}
\end{align*}
mit $A$ und $B$ Konstanten. Die Randbedingungen implizieren $A=0$ und für
$E \leq 0$ auch $B=0$; alle Lösungen mit $E \leq 0$ sind also trivial. Für
$E > 0$ gibt es nicht triviale Lösungen mit
\begin{align*}
    E_n = \frac{n^2 \pi^2 \hbar^2}{2 m a^2}
    \hspace{10pt} , \hspace{10pt}
    \psi_n = B_n \sin \klammer{\frac{n \pi x}{a}}
    = \sqrt{\frac{2}{a}} \sin \klammer{\frac{n \pi x}{a}}
\end{align*}
wobei $B_n$ durch die Normalisierung $1 = \int_0^a dx \ \psi_n^\ast \psi_n$
bestimmt wird.

Wir sehen, dass die möglichen Energieeigenwerte diskret sind. Diese Diskretheit
folgt aus den Randbedingungen. Weiterhin sind alle Energieeigenwerte positiv,
und insbesondere ist $E=0$ nicht zulässig. Dies folgt aus der Heisenbergschen
Unschärferelation. Weiter hat der $n$-te angeregte Zustand $\psi_{n+1}(x)$
gerade $n$ Nullstellen in $(0,a)$.

\subsection{Teilchen im Topf}\label{Teilchen_im_Topf}
\begin{align*}
    V(x) = \begin{cases}
        - V \hspace{10pt} &\text{falls } -a \leq x \leq a
        \\
        0 \hspace{10pt} &\text{sonst}
    \end{cases}
\end{align*}
Wir interessieren uns für gebundene Zustände, d.h. $\abs{\psi(x)} \rightarrow
0$ für $x \rightarrow \pm \infty$. Falls $x<-a$ und $x>a$, dann lautet die
zeit-unabhängige SG wie in Gl. (\ref{zeitunabhSGohnePot}). Eine Allgemeine
Lsg ist:
\begin{align*}
    \psi(x) = A e^{\kappa x} + D e^{- \kappa x}
    \hspace{10pt} \text{wobei} \hspace{10pt}
    E = - \frac{\hbar^2 \kappa^2}{2 m}
    \hspace{5pt} , \hspace{5pt}
    \kappa = \frac{\sqrt{-2mE}}{\hbar}
\end{align*}
Damit $\psi \rightarrow 0$ für $x \rightarrow \pm \infty$, muss $\kappa \in \R$
sein und somit $E\stackrel{!}{<}0$. Im folgenden sei $\kappa >0$. Die Lsg
ist somit:
\begin{align*}
    \psi(x) = \begin{cases}
        A e^{\kappa x} \hspace{10pt} &\text{falls } x < -a
        \\
        D e^{-\kappa x} \hspace{10pt} &\text{falls } x > a
    \end{cases}
\end{align*}
Für $-a < x < a$ ist die zeit-unabhängige SG:
\begin{align*}
    - \frac{\hbar^2}{2m} \psi'' (x) - V \psi(x) = E \psi(x)
\end{align*}
Die allgemeine Lsg ist von der folgenden Form:
\begin{align*}
    \psi(x) = B e^{i k x} + C e^{-i k x}
    \hspace{10pt} \text{wobei} \hspace{10pt}
    k = \frac{\sqrt{2m(V+E)}}{\hbar}
\end{align*}
Falls $\psi(x)$ eine gerade Fkt ist, d.h. $\psi(-x) = \psi(x)$, dann ist es
$H \psi(x)$ auch. Selbiges gilt für ungerade Funktionen. Wir können das EW-Problem
separat für gerade und ungerade Fkt lösen.

Im folgenden betrachten wir gerade Lösungen. Für diese gilt: $A=D$ und $B=C$.
Somit folgt:
\begin{align*}
    \psi(x) = \begin{cases}
        A e^{\kappa x} \hspace{10pt} &\text{falls } x < -a
        \\
        2 B \cos(kx) \hspace{10pt} &\text{falls } -a < x < a
        \\
        A e^{- \kappa x} \hspace{10pt} &\text{falls } x > a
    \end{cases}
\end{align*}
Die stetigkeit von $\psi$ und $\psi'$ bei $x = \pm a$ folgt aus:
\begin{align*}
    &- \frac{\hbar^2}{2m} \klammer{\psi_k' (a+\epsilon) - \psi_k'(a-\epsilon)}
    = - \frac{\hbar^2}{2m} \int_{a-\epsilon}^{a+\epsilon} dx \ \psi_k''(x)
    \\
    &= \int_{a-\epsilon}^a dx \ (E+V) \psi_k (x) + \int_a^{a+\epsilon} dx \ E \psi_k(x)
    \hspace{5pt} \stackrel{\epsilon \rightarrow 0}{\longrightarrow} 0
\end{align*}
Es folgt dann:
\begin{align*}
    A e^{-\kappa a} = 2B \cos(k a)
    \hspace{10pt} &\text{und} \hspace{10pt}
    \kappa A e^{-\kappa a} = 2 B k \sin(k a)
    \\
    \Rightarrow \kappa &= k \tan(k a)
\end{align*}
Weiter gilt:
\begin{align*}
    \kappa^2 + k^2 = \frac{2 m V}{\hbar^2}
\end{align*}
Definiere $\eta := \kappa a$ und $\xi := k a$. Dann vereinfachen sich die
Bedingungen zu:
\begin{align*}
    \eta = \xi \tan(\xi)
    \hspace{10pt} , \hspace{10pt}
    \eta^2 + \xi^2 = \frac{2 m a^2 V}{\hbar^2}
\end{align*}
Die Anzahl der geraden gebundenen Lösungen ist gerade $N$, falls
\begin{align*}
    (N-1)^2 \pi^2 < \frac{2ma^2 V}{\hbar^2} \leq N^2 \pi^2
\end{align*}
Im Fall der ungeraden Fkt gilt $A=-D$ und $B=-C$ und somit folgt:
\begin{align*}
    \psi(x) = \begin{cases}
        A e^{\kappa x} \hspace{10pt} &\text{falls } x < -a
        \\
        2 i B \sin(k x) \hspace{10pt} &\text{falls } -a < x < a
        \\
        -A e^{- \kappa x} \hspace{10pt} &\text{falls } x > a
    \end{cases}
\end{align*}
Stetigkeitsbedingungen ergeben:
\begin{align*}
    A e^{- \kappa a} = -2 i B \sin(k a)
    \hspace{10pt} &\text{und} \hspace{10pt}
    \kappa A e^{- \kappa a} = 2 i B k \cos(k a)
    \\
    \Rightarrow \kappa &= -k \cot(k a)
\end{align*}
Durch $\eta$ und $\xi$ ausgedrückt:
\begin{align*}
    \eta = - \xi \cot(\xi)
    \hspace{10pt} , \hspace{10pt}
    \eta^2 + \xi^2 = \frac{2 m a^2 V}{\hbar^2}
\end{align*}
Es gibt keine ungerade gebundene Lösung, falls
\begin{align*}
    \frac{2ma^2V}{\hbar^2} \leq \frac{\pi^2}{4}
\end{align*}

\subsection{Das Stufenpotential}

\begin{align*}
    V(x) = \begin{cases}
        0 \hspace{10pt} &\text{falls } x < 0
        \\
        V_0\hspace{10pt} &\text{falls } x \geq 0
    \end{cases}
    \hspace{20pt} \text{wobei } V_0 > 0
\end{align*}
Wir betrachten den Fall wo das Teilchen von links kommt und an der Stufe
gestreut wird. Ziel: Wsk bestimmen mit der das Teilchen reflektiert bzw
transmittiert wird.

Für $x < 0$ lautet die zeit-unabhängige SG wie folgt:
\begin{align*}
    - \frac{\hbar^2}{2m} \frac{d^2 \psi}{d x^2} = E \psi
    \hspace{10pt} \Rightarrow \hspace{10pt}
    \frac{d^2 \psi}{d x^2} \psi = - \frac{2mE}{\hbar^2} \psi
\end{align*}
Die allgemeinste Lsg ist:
\begin{align*}
    \psi_k (x) = A e^{\frac{i k x}{\hbar}} + B e^{- \frac{i k x}{\hbar}}
\end{align*}
mit $A$ und $B$ Konstanten und $k=\sqrt{2mE}$. Wir benötigen $k \in \R$
und somit auch $E>0$. Der Term $A e^{\frac{i k x}{\hbar}}$ beschreibt eine
von links einlaufende Welle mit Impuls $k$, wohingegen der Term $B e^{\frac{-i k x}{\hbar}}$
die reflektierte Welle beschreibt, die mit entgegengesetztem Impuls von
rechts nach links fliegt. Die Reflektionswahrscheinlichkeit ist dann
$R = \abs{B/A}^2$.

Für die Region $x \geq 0$ lautet die zeit-unabhängige SG wie folgt:
\begin{align*}
    - \frac{\hbar^2}{2m} \frac{d^2 \psi}{dx^2} \psi + V_0 \psi = E \psi
    \hspace{10pt} \Rightarrow \hspace{10pt}
    \frac{d^2 \psi}{d x^2} = - \frac{2m}{\hbar^2} \klammer{E-V_0} \psi
\end{align*}
Es gibt nun zwei Fälle: $E>V_0$ und $E<V_0$.

\paragraph{Fall 1: $E>V_0$.} In diesem Fall ist die allgemeine Lsg:
\begin{align*}
    \psi_k (x) = C e^{\frac{i l x}{\hbar}} + D e^{-\frac{i l x}{\hbar}}
\end{align*}
wobei $C$ und $D$ Konstanten sind und $l = \sqrt{2m(E-V_0)}$.
Da kein Teilchen von rechts aus dem Unendlichen kommt gilt $D=0$.
Aus der Stetigkeit folgt:
\begin{align*}
    A + B = C
    \hspace{10pt} &\text{und} \hspace{10pt}
    (A-B)k = C l
    \\
    \Rightarrow \hspace{10pt}
    2A = C \klammer{1+\frac{l}{k}}
    \hspace{10pt} &\text{und} \hspace{10pt}
    2B = C \klammer{1-\frac{l}{k}}
    \\
    \Rightarrow \hspace{10pt}
    \frac{C}{A} = \frac{2}{1+\frac{l}{k}}
    \hspace{10pt} &\text{und} \hspace{10pt}
    \frac{B}{A} = \frac{1-\frac{l}{k}}{1+\frac{l}{k}}
\end{align*}
Somit folgt für die Reflexions- und Transmissionswahrscheinlichkeit
\begin{align*}
    R = \abs{\frac{B}{A}}^2 = \frac{\klammer{1-\frac{l}{k}}^2}{\klammer{1+\frac{l}{k}}^2}
    \hspace{10pt} , \hspace{10pt}
    T = \frac{l}{k} \abs{\frac{C}{A}}^2 = \frac{4 \frac{l}{k}}{\klammer{1+\frac{l}{k}}^2}
\end{align*}
Falls $V_0$ relativ zu $E$ klein ist, dann ist $T \gg R$ und falls $V_0$
relativ zu $E$ grösser wird, nimmt die Reflektionswahrscheinlichkeit zu.

\paragraph{Fall 2: $E<V_0$.} Für $x<0$ ist die Lsg wie zuvor, aber für
$x>0$ haben wir
\begin{align*}
    \psi_k (x) = C' e^{- \frac{\kappa x}{\hbar}} + D' e^{\frac{\kappa x}{\hbar}}
    \hspace{15pt} \text{wobei } \kappa = \sqrt{2m(V_0-E)} \in \R_{>0}
\end{align*}
Damit diese Lsg nicht explodiert für $x \rightarrow \infty$, muss $D' = 0$
gelten. Aus den Stetigkeitsbedingungen folgt:
\begin{align*}
    A + B = C'
    \hspace{10pt} &\text{und} \hspace{10pt}
    i (A-B) k = - C' \kappa
    \\
    \Rightarrow \hspace{10pt}
    \frac{C'}{A} = \frac{2}{1+\frac{i \kappa}{k}}
    \hspace{10pt} &\text{und} \hspace{10pt}
    \frac{B}{A} = \frac{1-\frac{i \kappa}{k}}{1+\frac{i \kappa}{k}}
\end{align*}
Insbesondere ist $C' \neq 0$ und somit die Wahrscheinlichkeit dass sich das
Teilchen bei $x>0$ befindet nicht gleich Null! Es ist aber $T=0$ da
der Wahrscheinlichkeitsstrom $0$ ist weil die Wellenfunktionen rein reell ist.

\subsection{Normierung und Wellenpaket}

Streng genommen sind die obigen Wellenfunktionen nicht normierbar, da z.B.
$\psi_k (x) = e^{i k x}$ nicht quadrad-integrabel ist. Dieses Problem tritt auf,
falls die Menge der Lösungen der zeit-unabhängigen SG durch kontinuierliche
Parameter (hier $k$) parametrisiert wird. Wir machen neu den folgenden Ansatz
für die zeit-abhängige SG:
\begin{align*}
    \Psi(x,t) = \frac{1}{\sqrt{2 \pi}} \int_\R dk \ f(k) e^{i x k} e^{-i \frac{E_k}{\hbar} t}
\end{align*}
wobei $f(k)$ die Superposition der fundamentalen Lösungen ist der zeit-unabhängigen
SG. Wir suchen nun solche $f(k)$ sodass $\Psi(x,t)$ quadrat-integrabel ist.
Wir erinnern uns:
\begin{align*}
    &\int_\R dx \ e^{i k x} = 2 \pi \delta(k)
    \hspace{10pt} , \hspace{10pt}
    f \mapsto \int_\R dx \ f(x) \delta(x) = f(0)
    \\
    &\int_\R f(x) \delta'(x) = - f'(0)
\end{align*}
Weiter ist die Fouriertransformation und die inverse Fouriertransformation
gegeben als:
\begin{align*}
    \tilde{f}(p) = \frac{1}{\sqrt{2 \pi}} \int dx \ e^{-i p x} f(x)
    \hspace{10pt} , \hspace{10pt}
    f(x) = \frac{1}{\sqrt{2 \pi}} \int dp \ e^{i p x} \tilde{f}(p)
\end{align*}
Mit diesen tools kann man berechnen:
\begin{align}\label{contcond}
    \int_\R dx \ \abs{\Psi(x,t)}^2 = \int_\R dk \ \abs{f(k)}^2
    \stackrel{!}{=} 1
\end{align}
(\ref{contcond}) ist das kontinuierliche Analogon zu $\sum_n \abs{a_n}^2 = 1$.
Des weiteren:
\begin{align*}
    \int dx \ \psi_k^\ast \psi_{k'} = 2 \pi \delta(k-k')
    \hspace{10pt} \Leftrightarrow \hspace{10pt}
    \int dx \ \psi_n^\ast (x) \psi_m (x) = \delta_{n m}
\end{align*}

\subsection{Teilchen im Topf: positive Energielösung}

Wir betrachten wie in Kapitel \ref{Teilchen_im_Topf} das folgende Potential:
\begin{align*}
    V(x) = \begin{cases}
        \ -V \hspace{10pt} &\text{falls } -a \leq x \leq a
        \\
        \ 0 \hspace{10pt} &\text{sonst}
    \end{cases}
\end{align*}
aber diesmal mit positiver Energie. Klassisch erwartet man, dass das Teilchen
nicht beeinflusst wird vom Potential. Wir nehmen an, dass das Teilchen von
links einfliegt. Somit machen wir den folgenden Ansatz:
\begin{align*}
    \psi(x) = \begin{cases}
        e^{i k x} + r e^{-i k x} \hspace{10pt} &x<-a
        \\
        A e^{i l x} + B e^{-i l x} \hspace{10pt} &-a<x<a
        \\
        t e^{i k x} \hspace{10pt} &x>a
    \end{cases}
\end{align*}
wobei $R = \abs{r}^2$, $T = \abs{t}^2$ and
\begin{align*}
    k = \frac{\sqrt{2 m E}}{\hbar} > 0
    \hspace{10pt} , \hspace{10pt}
    l = \frac{\sqrt{2 m (E+V)}}{\hbar} > 0
\end{align*}
Durch die Klebebedingungen bei $a$ und $-a$ erhalten wir:
\begin{align*}
    A = \frac{1}{2} e^{-i(l-k) a} t \klammer{1 + \frac{k}{l}}
    \hspace{10pt} &, \hspace{10pt}
    B = \frac{1}{2} e^{i(l+k)a} t \klammer{1 - \frac{k}{l}}
    \\
    e^{-i k a} + r e^{i k a} = A e^{-i l a} + B e^{i l a}
    \hspace{10pt} &, \hspace{10pt}
    e^{-i k a} - r e^{i k a} = \frac{l}{k} \klammer{A e^{-i l a} - B e^{i l a}}
\end{align*}
Somit erhalten wir:
\begin{align}\label{transmission}
    t = \frac{e^{-2 i k a}}{\cos(2 l a) - i \frac{l^2 + k^2}{2 l k} \sin(2 l a)}
\end{align}
Wenn wir $l$ und $k$ einsetzen erhalten wir:
\begin{align*}
    T = \abs{t}^2 = \frac{1}{1 + \frac{\sin^2(2 l a) V^2}{4 E (E + V)}}
\end{align*}
Im Allgemeinen ist $0 \leq T \leq 1$ und somit wird das Teilchen mit $E>0$
vom Potential beeinflusst.

\subsubsection{Tunneleffekt}

Der Tunneleffekt tritt dann auf, falls $V$ negativ ist (s.d. $-V > 0$) und
$0 < E < -V$. Da $E+V < 0$, ist $l$ rein imaginär und somit:
\begin{align*}
    T = \frac{1}{1 - \frac{\sinh^2 \klammer{1 \abs{l} a} V^2}{4 E (E+V)}}
    \hspace{10pt} , \hspace{10pt}
    \abs{l} = \frac{\sqrt{2m(-E-V)}}{\hbar} \ \in \R^+
\end{align*}
Wir sehen dass $T$ nicht null ist für $x>a$ aber exponentiell abfällt mit
$a$ und $-(E+V)$.

\subsubsection{Resonanz}

Resonanz ist der Fall der perfekten Transmission, d.h. $T=1$. Dies tritt genau
dann auf falls $\sin^2(2 l a) = 0$, also für $l=\frac{n \pi}{2 a}$ mit $n \in \N$.
Die zugehörigen Energien sind:
\begin{align*}
    E_{res} = \frac{\hbar^2 \pi^2}{8 m a^2} n^2 - V = E_0 (n \pi)^2 - V
\end{align*}
Die resonanten Energien können selbst für $n>0$ negativ sein. Dies ist aber
nur eine formale Lösung da $k$ imaginär ist und somit i.A. nicht die gewünschten
Abfalleigenschaften aufweist. Es gibt genau so viele resonante Lösungen negativer
Energie wie gebundene Zustände. Es gibt $N$ gebundene (gerade und ungerade)
Lösungen falls:
\begin{align*}
    (N-1)^2 \frac{\pi^2}{4} \leq \frac{2 m a^2 V}{\hbar^2} \leq N^2 \frac{\pi^2}{4}
\end{align*}
Die entsprechenden Resonanzen mit negativer Energie treten daher für $n=0,1,\dots,N-1$
auf. Gebundene Lösungen existieren genau dann, wenn $t$ einen Pol besitzt. Die
Bedingung dafür ist:
\begin{align*}
    2 \cot(2 l a) = \cot(l a) - \tan(l a) = \frac{i k}{l} - \frac{l}{i k}
\end{align*}
Diese Gleichung hat zwei Lösungen:
\begin{align*}
    \tan(l a) = - \frac{i k}{l}
    \hspace{10pt} \text{ oder } \hspace{10pt}
    \cot(l a) = \frac{i k}{l}
\end{align*}
Diese sind gerade die Bedingungen für die Existenz von geraden, bzw. ungeraden
gebundenen Lösungen. Wir haben in Kapitel \ref{Teilchen_im_Topf}
bereits $k$ und $\kappa$ definiert. Wir wollen nun die Beziehung zwischen jenen
Konstanten und denen hier aufstellen: $k_{\text{dort}} = l_{\text{hier}}$ und
$\kappa_{\text{dort}} = -i k_{\text{hier}}$. $k_{\text{hier}}$ ist imaginär und somit
ist $\kappa_{\text{dort}}$ reell.

Wir betrachten nun noch wie sich die Lösung in der Nähe der Resonanz verhält.
Es gilt:
\begin{align*}
    e^{-2 i k a} \frac{1}{t} = \cos(2 l a) \eckigeklammer{1 - \frac{i}{2} \klammer{\frac{k}{l} + \frac{l}{k}} \tan(2 l a)} \approx 1
\end{align*}
Bei Resonanz gilt: $E=E_n$, $\cos(2 l a) = 1$ und $\tan(2 l a) = 0$. Wir entwickeln
in $E$ und erhalten:
\begin{align*}
    1 -& \frac{i}{2} \frac{d}{dE} \eckigeklammer{\klammer{\frac{k}{l} + \frac{l}{k}} \tan(2 a l)}_{E_n} (E-E_n)
    \equiv 1 - \frac{2i}{\Gamma} (E-E_n)
    \\
    &\text{wobei} \hspace{10pt} \frac{4}{\Gamma} = \klammer{\frac{k}{l} + \frac{l}{k}} 2 a \frac{dl}{dE} \Big|_{E_n}
\end{align*}
Als Funktion von $E$ verhält sich also $t(E)$ in der Nähe einer Resonanz wie
\begin{align*}
    e^{2 i k a} t(E) \simeq \frac{i \Gamma/2}{E - E_n + i \Gamma/2}
\end{align*}
somit folgt:
\begin{align*}
    T = \abs{t(E)}^2 \simeq \frac{\Gamma^2/4}{(E-E_n)^2 + \Gamma^2/4}
\end{align*}
Neu schreiben wir $t(E) = \abs{t} e^{i \delta(E)}$ wobei $\delta(E)$ eine Phase ist.
Jedes Mal wenn eine Resonanz durchlaufen wird, wächst diese Phase um $\pi$.