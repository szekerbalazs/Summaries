\section{Der Formalismus der Quantenmechanik}


Da die SG eine lineare DGL ist, gilt das Superpositionsprinzip. Somit ist
der Lösungsraum ein komplexer VR. Wichtig ist, dass die Elemente des VR
normierbar sind, d.h. $\langle \Psi(x,t) | \Psi(x,t) \rangle = 1 \ \forall t$.
Formal bedeutet dies, dass der komplexe VR der Lösungen ein inneres Produkt
besitzen muss. Einen solchen VR nennt man Hilbertraum.

\subsection{Der Hilbertraum}

Sei $\H$ ein komplexer VR. $\H$ ist ein Hilbertraum, falls $\H$ eine positiv-definite
sesquilineare Form (Skalarprodukt) besitzt, bzgl deren Norm $\H$ vollständig ist.
Das Skalarprodukt $\langle \cdot | \cdot \rangle$ ist eine Abbildung
\begin{align*}
    \langle \cdot | \cdot \rangle : \H \times \H \rightarrow \C
\end{align*}
mit den folgenden Eigenschaften für $\psi, \chi, \chi_i \in \H$ und $\alpha,\beta \in \C$:
\begin{enumerate}[(i)]
    \item Linearität im 2. Argument: $\langle \psi | \alpha \chi_1 + \beta \chi_2 \rangle
        = \alpha \langle \psi | \chi_1 \rangle + \beta \langle \psi | \chi_2 \rangle$
    \item Komplexe Konjugation: $\overline{\langle \psi | \chi \rangle} = \langle \chi | \psi \rangle$
    \item Anti-Linearität im 1. Argument: $\langle \chi_1 + \beta_2 \chi_2 | \psi \rangle
        = \alpha^\ast \langle \chi_1 | \psi \rangle + \beta^\ast \langle \chi_2 | \psi \rangle$
    \item Positive Definitheit: $\langle \psi | \psi \rangle \geq 0$ und
        $\langle \psi | \psi \rangle = 0 \ \Rightarrow \ \psi = 0$.
\end{enumerate}
Wir definieren die Norm $\Norm{\psi} = \sqrt{\langle \psi | \psi \rangle}$.
Es gilt:
\begin{align*}
    \abs{\langle \phi | \psi \rangle}^2 &\leq \langle \phi | \phi \rangle \cdot \langle \psi | \psi \rangle
    \hspace{10pt} \text{(Cauchy-Schwarz Ungleichung)}
    \\
    \Norm{\psi + \phi} &\leq \Norm{\psi} + \Norm{\phi}
    \hspace{20pt} \text{(Dreiecksungleichung)}
\end{align*}
Ein komplexer VR $\H$ mit einem Skalarprodukt ist ein Hilbertraum, falls $\H$ bzgl.
der obigen Norm vollständig ist: dies bedeutet, dass jede Cauchy Folge von
Vektoren $\phi_n \in \H$ zu einem Element in $\H$ konvergiert:
$\limes{n \rightarrow \infty} \phi_n = \phi \in \H$. $\eckigeklammer{\text{Def. Cauchy-Folge: }
\forall \epsilon > 0 \ \exists N \in \N : \ \forall n,m > N: \Norm{\phi_n - \phi_m}
< \epsilon}$

\subsection{Der $L^2$ Raum}

Der Hilbertraum, der der Wellenmechanik zu Grunde liegt, ist der Raum der
quadratintegrablen Funktionen, der $L^2$ Raum.
\begin{align*}
    L^2 \klammer{\R^f} = \geschwungeneklammer{f: \R^f \rightarrow \C \ \ \Big| \ \int dx \ \abs{f(x)}^2 < \infty}
\end{align*}
Wir definieren das Skalarprodukt durch:
\begin{align*}
    \langle f | g \rangle = \int_\R dx \ f^\ast (x) g(x)
\end{align*}
Es gelten die VR-Operationen:
\begin{align*}
    (f+g)(x) = f(x) + g(x)
    \hspace{10pt} , \hspace{10pt}
    (\lambda f)(x) = \lambda f(x)
\end{align*}

\subsection{Separable Hilberträume}

$L^2 (\R)$ ist ein unendlich-dimensionaler und separabler Hilbertraum. D.h.
er besitzt eine abzählbar unendliche Basis $f_n$. D.h. $\forall f \in L^2(\R)$
gilt
\begin{align*}
    f = \sum_{n=1}^\infty c_n f_n \hspace{5pt} , \hspace{10pt} c_n \in \C
\end{align*}
Wann immer ein Hilbertraum eine abzählbare Basis besitzt, lässt sich eine orthonormale
Basis konstruieren s.d. $\langle f_n | f_m \rangle = \delta_{mn}$.
ONB kann mit Schmidt'schem Orthogonalisierungsverfahren konstruiert werden:
\begin{align*}
    h_1 = f_1 / \Norm{f_1}
    \hspace{10pt} , \hspace{10pt}
    g_2 = f_2 - \langle h_1 | f_2 \rangle h_1
    \hspace{10pt} , \hspace{10pt}
    h_2 = g_2 / \Norm{g_2}
    \\
    g_3 = f_3 - \langle h_1 | f_3 \rangle h_1 - \langle h_2 | f_3 \rangle h_2
    \hspace{10pt} , \hspace{10pt}
    h_3 = g_3 / \Norm{g_3}
\end{align*}

\subsection{Operatoren und Observable}

Physikalische Observablen entsprechen Differentialoperatoren, die auf
Wellenfunktionen wirken. Der Impuls- und Ortsoperator sind gegeben als:
\begin{align}
    \hat{p} : L^2(\R) \rightarrow L^2(\R)
    \hspace{10pt} &, \hspace{10pt}
    f(x) \mapsto \hat{p} f(x) = -i \hbar \frac{\partial f(x)}{\partial x} \label{Impulsoperator}
    \\
    \hat{x} : L^2(\R) \rightarrow L^2(\R)
    \hspace{10pt} &, \hspace{10pt}
    f(x) \mapsto \hat{x} f(x) = x \cdot f(x) \label{Ortsoperator}
\end{align}
Beides sind lineare Operatoren, d.h. $\hat{p}(\lambda f + \mu g) = \lambda \hat{p}(f)
+ \mu \hat{p}(g)$.

\vspace{1\baselineskip}

Der Erwartungswert einer physikalischen Observablen $A$ ist gegeben durch
$\langle A \rangle_{\Psi} \equiv \langle \Psi | A \Psi \rangle =
\langle A \Psi | \Psi \rangle \in \R$.

\vspace{1\baselineskip}

Für die Fälle $\Psi = \Psi_1 + \Psi_2$ und $\Psi = \Psi_1 + i \Psi_2$ gilt
$\langle \Psi_1 | A \Psi_2 \rangle = \langle A \Psi_1 | \Psi_2 \rangle$.
Solche Operatoren nennt man \textit{symmetrisch}. Wir definieren den
\textit{adjungierten Operator} $B^\dagger$ zu $B$ durch:
$\langle f | B g \rangle = \langle B^\dagger f | g \rangle \ \forall f,g \in \H$.
Falls $A$ \textit{selbst-adjungiert} ist, also $A^\dagger = A$, dann ist $A$ auch
symmetrisch. Im allgemeinen gilt die Umkehrung nicht, da im unendlich dimensionalen
Selbst-Adjungiertheit eine stärkere Bedingung ist als Symmetrie. Selbst-Adjungiertheit
impliziert auch, dass der UR, auf dem $A$ definiert ist, mit demjenigen übereinstimmt,
auf dem $A^\dagger$ definiert ist.

Im endlichdimensionalen Fall, d.h. $\H = \C^n$, können wir die Operatoren $A$ durch
eine Matrix $M(A)$ beschreiben. Wir wählen hierzu eine ONB. Die Matrix des
adjungierten Operators $A^\dagger$ ist dann gegeben durch:
$M(A^\dagger) = \overline{M(A)^t} = M(A)^\dagger$. Die Observablen entsprechen
also gerade den hermiteschen Matrizen.

\subsection{Messung, Erwartungswert und Dirac Notation}

Nehme an, der Hilbertraum $\H$ sei endlichdimensional mit Dimension $N$. Jeder
selbst-adjungierte Operator kann dann diagonalisiert werden; somit können wir
eine Basis von $\H$ aus EV von $A$ finden. Bezeichne die EV von $A$ mit
$\psi_n$ und die EW mit $\lambda_n$ für $n = 1,\dots,N$.
\begin{align*}
    A \psi_n = \lambda_n \psi_n
\end{align*}
OBdA sind $\psi_n$ normiert: $\langle \psi_n | \psi_n \rangle$. Des weiteren
müssen die EW $\lambda_n \in \R$ reell sein und es gilt:
\begin{align*}
    \langle \psi_n | \psi_m \rangle = \delta_{mn}
\end{align*}
Wir führen die \textit{Dirac-Notation} ein. Die Vektoren im Hilbertraum $ | a \rangle
\in \H$ werden als 'kets' bezeichnet. Entsprechend sind die 'bras'
$\langle a | \in \H^\ast$ Elemente des Dualsraumes. Somit ist deren Komposition,
auch 'bracket' genannt $\langle a | (|b \rangle) \equiv \langle a | b \rangle$
eine Zahl.

Wir definieren die \textit{Spektralprojektoren}. Betrachte
\begin{align*}
    P_n = | \psi_n \rangle \langle \psi_n | \ : \ \H \rightarrow \H
\end{align*}
Da $P_n^2 = P_n$, ist $P_n$ ein \textit{Projektor}. Ausserdem projiziert
$P_n$ auf den Eigenraum $| \psi_n \rangle$. Des weiteren ist die
Spektralzerlegung von $A$ gegeben durch:
\begin{align*}
    A = \sum_n \lambda_n P_n = \sum_n \lambda_n | \psi_n \rangle \langle \psi_n |
\end{align*}
Somit ist der Erwartungswert von $A$ in dem Zustand $\psi$ gegeben als:
\begin{align*}
    \langle A \rangle_{\psi}
    = \sum_n \lambda_n \langle \psi | \psi_n \rangle \langle \psi_n | \psi \rangle
    = \sum_n \lambda_n \abs{\langle \psi | \psi_n \rangle}^2
\end{align*}
Mit dieser Erkenntniss, können wir die Postulate der QM allgemeiner formulieren zu:
\begin{enumerate}[(i')]
    \item Der Raum der Zustände ist der Hilbertraum $\H$, auf dem ein selbst-adjungierter
        Hamiltonoperator $H$ definiert ist. Das System wird zu jeder Zeit $t$ durch
        einen Strahl im Hilbertraum beschrieben; ein Strahl $\psi(t)$ ist die
        Äquivalenzklasse von normierten Vektoren $\chi$ mit $\langle \chi | \chi
        \rangle = 1$, wobei $\chi_1 \sim \chi_2$ falls $\chi_1 = e^{i \alpha} \chi_2$.
    \item Die Zeitentwicklung wird durch die SG beschrieben:
        \begin{align*}
            i \hbar \frac{\partial}{\partial t} \psi(t) = H \psi(t)
        \end{align*}
    \item Observable werden durch selbst-adjungierte Operatoren $A$ beschrieben.
    \item Die möglichen Ergebnisse einer Messung von $A$ sind die verschiedenen
        EW $\lambda_n$.
    \item Die Wsk., dass als Messergebnis $\lambda$ auftritt ist:
        \begin{align*}
            W(\lambda) = \sum_{\lambda_m = \lambda} \abs{\langle \psi_m | \psi \rangle}^2
        \end{align*}
        Der Erwartungswert von $A$ im Zustand $\psi$ ist $\langle \psi | A | \psi \rangle$.
\end{enumerate}

\subsection{Verallgemeinerung auf $\infty$-dimensionale Hilberträume}

Das \textit{Spektrum} $\sigma(A)$ eines selbst-adjungierten Operators
$A: \H \rightarrow \H$ enthält $\lambda \in \sigma(A)$, falls $\forall \epsilon > 0
\ \exists \psi_\epsilon \in \H$ normierbar ($\langle \psi_\epsilon | \psi_\epsilon \rangle$),
s.d. $\Norm{(A - \lambda) \psi_\epsilon} \leq \epsilon$. Falls $\lambda$ ein EW ist
von $A$, dann ist natürlich $\lambda \in \sigma(A)$. Das Spektrum ist die
Verallgemeinerung des Konzeptes der Eigenwerte für den $\infty$-dim. Fall.
Das Spektrum $\sigma(A)$ eines selbst-adjungierten Operators ist immer eine
Teilmenge der reellen Zahlen.
\begin{enumerate}[]
    \item Postulat: Die möglichen Resultate einer Messung der Observablen $A$
        ist das Spektrum von $A$.
\end{enumerate}
Das Spektrum kann diskrete oder kontinuierlich sein, oder beide Teile enthalten.
Beispielsweise:
\begin{enumerate}[]
    \item Freies Teilchen: $\sigma(x) = \sigma(p) = \R$, $\sigma(H) = \R_{\geq 0}$
    \item Potentialtopf: $\sigma(H) = \geschwungeneklammer{E_1,\dots,E_N} \cup \R_{\geq}$
    \item Harmonischer Oszillator: $\sigma(H) = \geschwungeneklammer{\hbar \omega \klammer{\frac{1}{2} + n} \ : \ n=1,2,\dots}$
\end{enumerate}

Falls das Spektrum distret ist, dann gilt:
\begin{align*}
    f(A) := \sum_{a \in \sigma(A)} f(a) P_a
\end{align*}
wobei $f: \R\rightarrow \C$ eine Funktion und $f(A): \H \rightarrow \H$ ein
Operator ist. Diese Zuordnung hat folgende Eigenschaften:
\begin{align*}
    (\alpha_1 f_1 + \alpha_2 f_2)(A) &= \alpha_1 f_1 (A) + \alpha_2 f_2(A)
    \hspace{5pt} , \hspace{10pt} \alpha_1 , \alpha_2 \in \C
    \\
    (f_1 f_2)(A) &= f_1(A) f_2(A)
    \\
    \overline{f}(A) &= f(A)^\dagger
    \\
    f(A) &= \begin{cases}
        \mathds{1} \hspace{10pt} &\text{für } f(x) = 1
        \\
        A \hspace{10pt} &\text{für } f(x) = x
    \end{cases}
\end{align*}
Der \textit{Spektralsatz} besagt, dass es auch im $\infty$-dim. Fall eine eindeutige
Zuordnung $f \mapsto f(A)$ gibt die die obigen Eigenschaften erfüllt. Und
falls $A$ selbst-adjungiert ist gibt es solch eine Zuordnung auch falls das
Spektrum nicht diskret ist.

Sei $I \subset \R$ ein Intervall und $P_I(x)$ dessen charakteristische Fkt.
Dann ist $P_I (A)$ ein orthogonaler Projektor: $P_I (A) = P_I(A)^\dagger
= P_I (A)^2$. Für disjunkte Intervalle $I_1$ und $I_2$ gilt:
$P_{I_1 \bigcup I_2} (A) = P_{I_1} (A) + P_{I_2}(A)$. Für jeden Zustand $| \psi \rangle
\in \H$ ist dann $W_{\psi} (I) = \langle \psi | P_I (A) | \psi \rangle$ ein
Wahrscheinlichkeitsmass auf $\R$, d.h.
\begin{align*}
    W_{\psi}(I) &= \Norm{P_I(A) \psi}^2 \geq 0
    \\
    W_{\psi} (I_1 \cup I_2) &= W_{\psi} (I_1) + W_\psi (I_2)
    \hspace{10pt} \text{für } I_1 \cap I_2 = \emptyset
    \\
    W_{\psi} (\R) &= 1
\end{align*}
Interpretation: $W_\psi(I)$ ist die Wahrscheinlichkeit, dass $A$ im Zustand
$| \psi \rangle$ einen Messwert $a \in I$ annimmt.

Weiter definieren wir den Erwartungswert von $A$ im Zustand $\psi$ durch:
\begin{align*}
    \langle A \rangle_\psi = \int \lambda \ dW_\psi ((-\infty,\lambda])
    = \langle \psi | A | \psi \rangle
\end{align*}

\subsection{Andere Darstllung der Quantenmechanik}

Im 'Ortsraum' sind die Operatoren gegeben wie in \cref{Impulsoperator} und
\cref{Ortsoperator}. Wir können aber auch eine Definition im 'Impulsraum' geben.
Die Impulsdarstellung einer Wellenfunktion ist gegeben durch die
Fourier-Transformation:
\begin{align*}
    \tilde{\Psi} (p,t) = \frac{1}{\sqrt{2 \pi \hbar}} \intii dx \ \Psi(x,t) e^{-i p x / \hbar}
\end{align*}
Hier sind Impulsoperator und Ortsoperator wie folgt gegeben:
\begin{align*}
    \hat{p}: L^2(\R) \rightarrow L^2(\R)
    \hspace{10pt} , \hspace{10pt}
    f(q) \mapsto [\hat{p}(f)](p) = p f(p)
    \\
    \hat{x}: L^2(\R) \rightarrow L^2 (\R)
    \hspace{10pt} , \hspace{10pt}
    f(p) \mapsto [\hat{x} f](p) = i \hbar \frac{\partial f(p)}{\partial p}
\end{align*}
Der Kommutator zweier Operatoren $A$ und $B$ ist gegeben als
\begin{align*}
    [A,B] = A B - B A = i \hbar \geschwungeneklammer{A,B}
\end{align*}
Wobei $\geschwungeneklammer{.,.}$ die Poissonklammer ist.
Für den Fall $A=x$ und $B=p$ gilt: $[x,p] = i \hbar$.
