\section{Symmetrien in der Quantenmechanik}

\subsection{Unitäre Darstellungen}

Eine unitäre Darstellung einer Gruppe $G$ auf dem Hilbertraum $\H$ ist ein
(Gruppen-) Homomorphismus von $G$ auf die Gruppe der unitären Operatoren auf $\H$.
D.h. $\forall g \in G$ haben wir einen unitären Operator $U(g)$, s.d.
\begin{align*}
    U(g_1) \cdot U(g_2) = U(g_1 \cdot g_2)
\end{align*}
wobei der unitäre Operator $U(g): \H \rightarrow \H$ eine lineare Abbildung ist,
die das Skalarprodukt erhält: $\left\langle U(g) \psi | U(g) \chi \right\rangle
= \left\langle \psi | \chi \right\rangle \ \forall \psi,\chi \in \H$ und $g \in G$.
Es folgt: $U(g)^\dagger = U(g)^{-1} = U(g^{-1})$.

\vspace{1\baselineskip}

Jede Gruppe besitzt die triviale Darstellung: $U(g) = \mathds{1} \ \forall g \in G$.
Wir betrachten die Darstellung der Rotationsgruppe auf dem Raum der Wellenfunktionen
$L^2(\R^3)$. Die Gruppe der Rotationen im $\R^3$ ist die spezielle orthogonale
Gruppe $SO(3)$ ($\forall R \in SO(3): \ R^T R = \mathds{1}$ und $\det(R) = 1$).
Eine unitäre Darstellung auf $L^2(\R^3)$ ist dann definiert durch
\begin{align*}
    (U(R) \psi)(\vec{x}) = \psi (R^{-1} x)
\end{align*}
Wir betrachten die Untergruppe von $O(3)$ welche nur $\pm \mathds{1}$ enthält.
Diese Gruppe wird durch den Paritätsoperator $\mathcal{P}$ erzeugt, der zu der
orthogonalen Transformation $-\mathds{1}$ gehört. Da $\mathcal{P}^2 = 1$, folgt
dass die EW des Paritätsoperators gerade $\pm 1$ sind. Die EV erfüllen:
\begin{align*}
    \psi(-\vec{x}) = \klammer{\mathcal{P} \psi} (\vec{x}) = \pm \psi(\vec{x})
\end{align*}
D.h. sie entsprechen gerade den geraden, bzw. ungeraden Funktionen. In vielen
Fällen ist der Hamiltonoperator $H$ eine gerade Fkt. von $x$. D.h. die beiden
Operatoren $H$ und $\mathcal{P}$ vertauschen miteinander. Somit lassen sich gemeinsame
Eigenfunktionen zu $H$ und $\mathcal{P}$ finden.

\subsection{Die Drehgruppe $SO(3)$ und ihre Lie Algebra.}

Kontinuierliche Gruppen werden statt Gruppentransformationen durch infinitesimale
Transformationen betrachtet. Eine kontinuierliche Gruppe ist eine Lie-Gruppe; die
infinitesimalen Transformationen einer Lie Gruppe bilden eine Lie Algebra. Sei
$R(t)$ eine differenzierbare Kurve von Rotationen in $SO(3)$ durch $R(0) = \mathds{1}$.
Die infinitesimale Rotation is dann:
\begin{align*}
    \Omega = \frac{d}{dt} \Big|_{t=0} R(t)
\end{align*}
Die Menge dieser infinitesimalen Rotationen bilden einen (reellen) VR, da
für $\alpha_i \in \R$
\begin{align*}
    \alpha_1 \Omega_1 + \alpha_2 \Omega_2 =
    \frac{d}{dt} \Big|_{t=0} R_1 (\alpha_1 t) R_2 (\alpha_2 t)
\end{align*}
Dieser VR ist der Lie Algebra zugrunde liegende VR, und wird mit $\so(3)$
bezeichnet. Er bildet auch eine Darstellung der Lie Gruppe, da $\forall R \in SO(3)$
\begin{align*}
    R \Omega_1 R^{-1} = \frac{d}{dt} \Big|_{t=0} R R_1 (t) R^{-1}
    \hspace{5pt} \in \so(3)
\end{align*}
Da die Lie Algebra $\so(3)$ ein VR ist, folgt:
\begin{align*}
    \eckigeklammer{\Omega_1,\Omega_2} = \frac{d}{dt} \Big|_{t=0}
    R_1 (t) \Omega_2 R_1 (t)^{-1}
    \hspace{5pt} \in \so(3)
\end{align*}
Die Lie Algebra $\so(3)$ ist somit nicht nur ein VR, sondern trägt eine Struktur,
nämlich die Lie-Klammer $\eckigeklammer{\cdot,\cdot}$. Für die Matrix-Gruppe gilt:
\begin{align*}
    \eckigeklammer{\Omega_1,\Omega_2} = \Omega_1 \Omega_2 - \Omega_2 \Omega_1
\end{align*}
Die Lie-Klammer beschreibt also einfach den Kommutator, welcher anti-symmetrisch
ist und die Jacobi Identität erfüllt:
\begin{align*}
    &\eckigeklammer{\Omega_1,\Omega_2} = - \eckigeklammer{\Omega_2,\Omega_1}
    \\
    &0 =
    \eckigeklammer{\Omega_1,\eckigeklammer{\Omega_2,\Omega_3}} +
    \eckigeklammer{\Omega_2,\eckigeklammer{\Omega_3,\Omega_1}} +
    \eckigeklammer{\Omega_3,\eckigeklammer{\Omega_1,\Omega_2}}
\end{align*}
Für den Fall $R(t) \in SO(3)$ ist $R(t)^T R(t) = \mathds{1}$ und es folgt
\begin{align*}
    \Omega^T + \Omega = 0
\end{align*}
Somit besteht die Lie Algebra $\so(3)$ aus den anti-symmetrischen reellen
$3 \times 3$ Matrizen. Jede solche Matrix ist von der Form
\begin{align*}
    \Omega(\vec{\omega}) = \begin{pmatrix}
        0 & -\omega_3 & \omega_2 \\
        \omega_3 & 0 & -\omega_1 \\
        -\omega_2 & \omega_1 & 0
    \end{pmatrix}
\end{align*}
D.h. $\Omega(\vec{\omega}) \vec{x} = \vec{\omega} \wedge \vec{x}$. Somit
ist $\dim_{R} \so(3) = 3$. Bezeichne mit $\Omega_i := \Omega(\vec{e}_i)$.
Für $\vec{\omega} = \omega \vec{e}$ mit $\abs{\vec{e}}=1$
ist $e^{\Omega(\vec{\omega}) t} = R(e,\omega t)$ gerade die Drehung um die Achse
$\vec{e}$ um den Winkel $\omega t$. Es folgt:
\begin{align*}
    R \Omega(\vec{\omega}) R^{-1} = \Omega(R \vec{\omega})
    \hspace{10pt}
    R \in SO(3)
\end{align*}
Weiter folgt:
\begin{align*}
    \eckigeklammer{\Omega(\omega_1),\Omega(\vec{\omega_2})} =
    \Omega(\vec{\omega}_1 \wedge \vec{\omega}_2)
    \hspace{10pt} , \hspace{10pt}
    \eckigeklammer{\Omega_1,\Omega_2} = \Omega_3 \hspace{5pt}
    \text{(und zyklisch)}
\end{align*}
Sei $U(R)$ eine unitäre Darstellung von $SO(3)$ auf einem Hilbertraum $\H$, dann
erhält man daraus auch eine Darstellung der Lie Algebra $\so(3)$
\begin{align*}
    U(\Omega) = \frac{d}{dt} \Big|_{t=0} U(R(t))
\end{align*}
Diese Darstellung von $\so(3)$ ist auch ein VR-Homomorphismus von $\so(3)$ in den
VR der Operatoren $\H$ die die Lie Algebra Klammer erhält:
\begin{align*}
    U(\eckigeklammer{\Omega_1,\Omega_2}) = \eckigeklammer{U(\Omega_1),U(\Omega_2)}
\end{align*}
Die Bedingung dass $U(R)$ unitär ist impliziert $U(\Omega)^\dagger = - U(\Omega)$.
Falls dies erfüllt ist, nennen wir eine Darstellung der Lie Algebra unitär. 

\vspace{1\baselineskip}

Für jedes $\omega \in \R^3$ definieren wir den selbstadjungierten Drehumpulsoperator
\begin{align*}
    M(\vec{\omega}) := i \Omega(\omega) = \sum_{i=1}^3 M_i \omega_i
\end{align*}
wobei $M_i = M(e_i)$. Weiter gilt: $\eckigeklammer{M_1,M_2} = i M_3$

\vspace{1\baselineskip}

Die unitäre Darstellung der Lie Gruppe $SO(3)$ ist eindeutig durch die Darstellung
ihrer Lie Algebra $\so(3)$ bestimmt. Man kann die Lie Gruppe durch 'Exponenzieren'
aus der Lie Algebra rekonstruieren ($\exp : \so(3) \rightarrow SO(3)$, $\Omega
\mapsto e^\Omega$). Weiterhin ist $\Omega$ gerade das Lie Algebra Element, das
zu der Kurve $R(t) = e^{t \Omega}$ assoziiert ist. Die Produktstruktur der Lie
Gruppe kann auch rekonstruiert werden:
\begin{align*}
    \exp(\Omega_1) \cdot \exp(\Omega_2) = \exp \klammer{\Omega_1 \star \Omega_2}
\end{align*}
Wobei $\Omega_1 \star \Omega_2$ durch die CBH Formel berechnet werden kann:
\begin{align*}
    \Omega_1 \star \Omega_2 = \Omega_1 &+ \Omega_2 + \frac{1}{2} \klammer{\Omega_1,\Omega_2}
    + \frac{1}{12} \eckigeklammer{\Omega_1,\eckigeklammer{\Omega_1,\Omega_2}}
    \\
    &+ \frac{1}{12} \eckigeklammer{\Omega_2,\eckigeklammer{\Omega_2,\Omega_1}}
    + \dotsb
\end{align*}

\subsection{Reduzible und irreduzible Darstellungen}

Sei $\H$ ein Hilbertraum, auf dem eine unitäre Darstellung der Gruppe $G$
definiert ist. (D.h. $\forall g \in G$ $U(g) \in V$ wobei $V \subseteq \H$).
Wir nennen diese Darstellung reduzibel falls eine nicht triviale Unterdaratellung
$V$ existiert und irreduzibel falls die einzigen Unterräume $V \subseteq \H$,
die unter der Wirkung von $U(g)$ $\forall g \in G$ auf sich selbst abgebildet
werden, $U(g) V \subseteq V$, einfach $V = \geschwungeneklammer{0}$ und $V = \H$
sind. Für unitäre Darstellungen kann man jede reduzible Darstellung in
irreduzible Darstellungen zerlegen. Falls die Darstellung unitär ist, kann die
reduzible Darstellung geschrieben werden als direkte Summe:
\begin{align*}
    \exists V^\perp : \ V \oplus V^\perp = \H
    \hspace{10pt} , \hspace{10pt}
    V^\perp = \geschwungeneklammer{\psi \in \H \ | \ \scalprod{\chi}{\psi} = 0 \ \forall \chi \in V}
\end{align*}
Wegen Linearität des Skalarproduktes ist auch $V^\perp$ ein Unterraum von $\H$.
Die Unitarität von $U$ impliziert, dass $V^\perp$ invariant ist unter der
Wirkung von $G$. Mit diesem Verfahren können wir also die Darstellung $\H$ in
kleinere Unterdarstellungen zerlegen. Jede reduzible unitäre Darstellung kann als
Summe von irreduziblen Darstellungen geschrieben werden.

\subsection{Irreduzible Darstellungen von $\so(3)$}

Die Lie Algebra von $SO(3)$ wird durch die Generatoren $M_i, \ i = 1,2,3$
aufgespannt, wobei die Kommutatoren durch
\begin{align*}
    \eckigeklammer{M_1,M_2} = i M_3
    \hspace{10pt} \text{(und zyklisch)}
\end{align*}
gegeben sind. Wir definieren nun Auf- und Absteigeoperatoren
\begin{align*}
    M_{\pm} = M_1 \pm i M_2
\end{align*}
Es folgt:
\begin{align*}
    \eckigeklammer{M_3 , M_{\pm}} = \pm M_\pm
    \hspace{10pt} , \hspace{10pt}
    \eckigeklammer{M_+ , M_-} = 2 M_3
\end{align*}
Sei $\psi$ ein EV von $U(M_3) = M_3$ (wir unterscheiden im Folgenden nicht
zwischen $U(M_i)$, der Wirkung von $M_i$ in der Darstellung, und $M_i$
selbst)
\begin{align*}
    M_3 \psi = z \psi \ \ \ , \ z \in \C
    \hspace{10pt} \Rightarrow \hspace{10pt}
    M_3 M_{\pm} \psi = (z \pm 1) M_{\pm} \psi
\end{align*}

Da $\dim (\H) < \infty$, existiert ein EW $j \in \C$ mit EV $\psi_j$, derart dass
\begin{align*}
    M_3 \psi_j = j \psi_j
    \hspace{10pt} , \hspace{10pt}
    M_+ \psi_j = 0
\end{align*}
Wir setzten induktiv
\begin{align*}
    M_- \psi_m =: \psi_{m-1}
    \hspace{10pt} \text{für } m = j,j-1,\dots
    \hspace{10pt} \Rightarrow \hspace{10pt}
    M_3 \psi_m = m \psi_m
\end{align*}
Da auch diese Folge abbrechen muss $\exists k \in \N$, s.d.
\begin{align*}
    \psi_{j-k} \neq 0
    \hspace{10pt} , \hspace{10pt}
    M_- \psi_{j-k} = 0
\end{align*}
Aus
\begin{align*}
    M_+ \psi_m &= \mu_0 \psi_{m+1}
    \\
    \Rightarrow \
    M_+ \psi_{m-1} &= (2m+\mu_m) \psi_m
    \equiv \mu_{m-1} \psi_m
\end{align*}
folgt
\begin{align*}
    \mu_m = (j-m)(j+m+1)
    \hspace{10pt} , \hspace{10pt}
    j = 0 , \frac{1}{2} , 1 , \frac{3}{2}
\end{align*}
Jede irreduzible Darstellung entspricht damit ein solches $j$. Aus folgender
Zusammenfassung folgt, dass die $M_3,M_{\pm}$ eine Darstellung $\D_j$ der $\so(3)$
liefern.
\begin{align*}
    M_3 \psi_m = m \psi_m
    \hspace{10pt} , \hspace{10pt}
    M_+ \psi_m = \mu_m \psi_{m+1}
    \hspace{10pt} , \hspace{10pt}
    M_- \psi_m = \psi_{m-1}
\end{align*}
Die Darstellung kann auch durch den Wert des Casimiroperators charakterisiert
werden.
\begin{align*}
    C = \vec{M}^2 &= M_1^2 + M_2^2 + M_3^2
    = M_3^2 + \frac{1}{2} \klammer{M_+ M_- + M_- M_+}
    \\
    \eckigeklammer{M_i,C} &= 0 \ \forall M_i \so(3)
    \\
    \vec{M}^2 \psi_j &= j(j+1) \psi_j
    \hspace{10pt} , \hspace{10pt} C \psi = j(j+1) \psi \ \forall \psi \in \D_j
\end{align*}
Letzteres gilt weil $C$ mit allen $M_i$ vertauscht. Da dies auch für $\psi_{j-k}$
gelten muss, folgt:
\begin{align*}
    j(j+1) = (j-k)(j-k-1)
    \hspace{10pt} \Rightarrow \hspace{10pt}
    k = -1 \text{ oder } k = 2j
\end{align*}
Die erste Lösung macht keinen Sinn da $k$ positiv sein muss. Aus der zweiten
Lösung erhalten wir wieder $j = \frac{k}{2}$ für $k \in \N$. Es gilt
$\dim(\D_j) = 2j+1$, $M_i = M_i^\dagger$ und $M_\pm^\dagger = M_{\mp}$.
So ist eine ONB von $D_j$ gegeben durch
\begin{align*}
    \geschwungeneklammer{|j,m \rangle}_{m=-j}^j
    \hspace{10pt} , \hspace{10pt}
    |j,j \rangle := \frac{\psi_j}{\Norm{\psi_j}}
    \hspace{10pt} , \hspace{10pt}
    \sqrt{\mu_m} |j,m \rangle := M_- |j,m+1 \rangle
\end{align*}
Es folgt:
\begin{align*}
    C |j,m \rangle &= \vec{M}^2 |j,m \rangle = j(j+1) |j,m \rangle
    \hspace{10pt} , \hspace{10pt}
    M_3 | j,m \rangle = m |j,m \rangle
    \\
    M_{\pm} | j,m \rangle &= \sqrt{j(j+1) - m(m \pm 1)} | j,m \pm 1 \rangle
\end{align*}
Die triviale Darstellung entspricht $\D_0$. Die fundamentale Darstellung
$U(R) = R$, $U(\Omega) = \Omega$ ist auf $\H = \R^3$ definiert. Sie ist
irreduzibel und hat Dimension 3, und somit ist sie isomorph zu $\D_1$.
Selbiges gilt für die adjungierte Darstellung auf $\H = \so(3)$ welche definiert
ist durch
\begin{align*}
    U(R) \Omega = R \Omega R^{-1}
    \hspace{10pt} \text{bzw.} \hspace{10pt}
    U(\Omega_1) \Omega_2 = \eckigeklammer{\Omega_1,\Omega_2}
\end{align*}
Eine andere Klasse von Darstellungen sind Unterräume der Fkt. auf der Kugel,
die durch die Kugelfunktionen erzeugt werden. Auf dem VR dieser Fkt. wirkt
die Drehgruppe in Kugelkoordinaten wie folgt:
\begin{align*}
    M_3 = -i \frac{\partial}{\partial \phi}
    \hspace{10pt} , \hspace{10pt}
    M_\pm = e^{\pm i \phi} \klammer{\pm \frac{\partial}{\partial \theta} + i \cot(\theta) \frac{\partial}{\partial \phi}}
\end{align*}
Um die invarianten UR zu beschreiben führen wir die Kugelfkt.
\begin{align*}
    Y_{l,m} (\theta,\phi) = \sqrt{\frac{2l+1}{4 \pi} \frac{(l-m)!}{(l+m)!}} P_l^m (\cos(\theta)) e^{i m \phi}
\end{align*}
ein, wobei $m=-l,\dots,l$. $P_l^m$ sind die assoziierten Legendre-Fkt., die die
folgende verallgemeinerte Legendre-Gleichung lösen.
\begin{align*}
    \frac{d}{dz} \eckigeklammer{(1-z^2) \frac{d P(z)}{dz}} +
    \eckigeklammer{l(l+1) - \frac{m^2}{1-z^2}} P(z) = 0
\end{align*}
Die Explizite Lösung ist:
\begin{align*}
    P_l^m (z) = \frac{(-1)^m}{2^l l!} (1-z^2)^{m/2} \frac{d^{l+m}}{dz^{l+m}} (z^2 -1)^l
\end{align*}
Man sieht nun:
\begin{align*}
    M_3 Y_{l,m} = m Y_{l,m}
    \hspace{10pt} , \hspace{10pt}
    M_\pm Y_{l,m} = C_{l,m,\pm} Y_{l,m \pm 1}
\end{align*}
wobei $C_{l,m,\pm}$ eine Konstante ist. Wir sehen dass die Kugelfunktionen
$\geschwungeneklammer{Y_{l,m} \ | \ m = -l,\dots,l}$
eine irreduzible Darstellung der Lie Gruppe $SO(3)$ definieren.
Sie stimmt gerade mit $D_j$ überein.
$D_j \cong \geschwungeneklammer{Y_{l,m} \ | \ m = -l,\dots,l}$


\subsection{SO(3) vs. SU(2)}

Die Kugelfkt. beschreiben alle Darstellungen von $\so(3)$, für die $j \in \N$.
Die übrigen Darstellungen von $\so(3)$, also jene für $j \in \N + \frac{1}{2}$,
führen nicht zu Darstellungen von $SO(3)$ sondern ihrer Überlagerungsgruppe
$SU(2)$.
\begin{align*}
    SU(2) &= \geschwungeneklammer{A \in U(2) \ | \ \det(A) = 1}
    \\
    &= \geschwungeneklammer{g = \begin{pmatrix}
        a & b \\ -\overline{b} & \overline{a}
    \end{pmatrix} \ \Big| \ \abs{a}^2 + \abs{b}^2 = 1}
    \\
    \su(2) &= \geschwungeneklammer{A \in \Mat(2,\C) \ | \ A + A^\dagger = 0 \ , \ \tr(A) = 0}
\end{align*}
Eine Basis von $\su(2)$ ist gegeben durch die Pauli Matrizen
\begin{align*}
    \sigma_1 = \begin{pmatrix}
        0 & 1 \\ 1 & 0
    \end{pmatrix}
    \hspace{10pt} , \hspace{10pt}
    \sigma_2 = \begin{pmatrix}
        0 & -i \\ i & 0
    \end{pmatrix}
    \hspace{10pt} , \hspace{10pt}
    \sigma_3 = \begin{pmatrix}
        1 & 0 \\ 0 & -1
    \end{pmatrix}
\end{align*}
Jedes $A \in \su(2)$ kann geschrieben werden als
\begin{align*}
    A \equiv A(\vec{a}) = -\frac{i}{2} \begin{pmatrix}
        a_3 & a_1 - i a_2 \\ a_1 + i a_2 & -a_3
    \end{pmatrix}
    \equiv - \frac{i}{2} \vec{\sigma} \cdot \vec{a}
\end{align*}
Somit hat $\su(2)$ eine (reelle) 3 dim. Basis
\begin{align*}
    A_j = -\frac{i}{2} \sigma_j
    \hspace{10pt} j=1,2,3
\end{align*}
Es gelten die folgenden Eigenschaften für die Pauli Matrizen:
\begin{align*}
    \sigma_i \sigma_j = \delta_{ij} + i \epsilon_{ijk} \sigma_k
    \hspace{10pt} &, \hspace{10pt}
    \klammer{\vec{\sigma} \cdot \vec{a}} \klammer{\vec{\sigma} \cdot \vec{b}}
    = \klammer{\vec{a} \cdot \vec{b}} \mathds{1} + i \vec{\sigma} \cdot \klammer{\vec{a} \wedge \vec{b}}
    \\
    \eckigeklammer{A(\vec{a}) , A(\vec{b})} = A(\vec{a} \wedge \vec{b})
    \hspace{10pt} &, \hspace{10pt}
    [A_1 , A_2] = A_3 \hspace{5pt} \text{(und zyklisch)}
\end{align*}
Die Lie Algebren $\so(3)$ und $\su(2)$ sind also isomorph, wobei der Isomorphismus
gegeben ist durch
\begin{align*}
    \su(2) \rightarrow \so(3)
    \hspace{5pt} , \hspace{5pt}
    A(\vec{\omega}) \mapsto \Omega(\vec{\omega})
\end{align*}
Die irreduziblen Darstellungen der $\su(2)$ sind damit die $D_j$.

Wir definieren für jeden Vektor $\vec{x} \in \R^3$ die Matrix
\begin{align*}
    \tilde{x} = \vec{x} \cdot \vec{\sigma} = \begin{pmatrix}
        x_3 & x_1 - i x_2 \\ x_1 + i x_2 & -x_3
    \end{pmatrix}
\end{align*}
Diese Abbildung ist invertierbar da $x_j = \frac{1}{2} \tr(\tilde{x} \cdot \sigma_j)$.
Falls $\vec{x}$ ein reeller Vektor ist, dann ist $\tilde{x}$ eine hermitesche Matrix.
Die Umkehrung gilt auch. Es gilt
\begin{align*}
    \det(\tilde{x}) = - \vec{x} \cdot \vec{x}
\end{align*}
Für jedes Element $A \in SU(2)$ betrachten wir die folgende Abbildung:
\begin{align*}
    \tilde{x} \mapsto \tilde{x}' = A \tilde{x} A^\dagger
\end{align*}
Dies ist eine lineare Abbildung. Sie definiert eine Drehung und nach
Konstruktion einen Gruppenhomomorphismus
\begin{align*}
    SU(2) \rightarrow SO(3)
\end{align*}
Der Kern dieser Transformation ist gerade $\pm \mathds{1} \in SU(2)$.
Da der Homomorphismus surjektiv ist, folgt
\begin{align*}
    SU(2) / \geschwungeneklammer{\pm \mathds{1}} \cong SO(3)
\end{align*}
Zu jedem Element in $SO(3)$ gibt es also zwei Elemente in der Überlagerungsgruppe
$SU(2)$, die sich gerade um ein Vorzeichen unterscheidet. Jede Darstellung von
$SO(3)$ definiert eine Darstellung von $SU(2)$ (indem man nämlich die Wirkung
von $\pm \mathds{1} \in SU(2)$ trivial definiert), aber die Umkehrung ist nicht
richtig: eine Darstellung von $SU(2)$ ist nur dann eine Darstellung von $SO(3)$,
falls $\pm \mathds{1} \in SU(2)$ trivial wirkt.

Im Allgemeinen gilt in $SO(3)$
\begin{align*}
    R(\vec{e}_3 , \varphi) =
    \begin{pmatrix}
        \cos(\varphi) & -\sin(\varphi) & 0 \\
        \sin(\varphi) & \cos(\varphi) & 0 \\
        0 & 0 & 1
    \end{pmatrix}
    = \exp \eckigeklammer{\varphi \begin{pmatrix}
        0 & -1 & 0 \\ 1 & 0 & 0 \\ 0 & 0 & 0
    \end{pmatrix}} = e^{-i M_3 \varphi}
\end{align*}
Insbesondere gilt für jeden Darstellungsvektor $|j,m\rangle$
\begin{align*}
    U(R(\vec{e}_3 , 2 \pi)) |j,m\rangle = e^{-2 \pi i M_3} |j,m\rangle
    = e^{-2\pi i m} |j,m\rangle = e^{-2 \pi i j} |j,m\rangle
\end{align*}
da $m=j-n$ für $n \in \N$. Wir sehen dass die Rotation um $2 \pi$ nur für
ganzzahlige $j$ trivial dargestellt wird. Diese Darstellungen werden durch die
Kugelfunktionen realisiert und sind somit tatsächlich alle Darstellungen von $SO(3)$.
Falls $j \in \N + \frac{1}{2}$ dann wird die Rotation um $2 \pi$ nicht trivial
dargestellt: diese Darstellungen der Lie Algebra $\su(2) \equiv \so(3)$ definieren
nur Darstellungen von $SU(2)$, aber nicht von $SO(3)$. Wir finden somit nicht auf
den Raum der Kugelfunktionen.


\subsection{Projektive Darstellungen}

Bis jetzt haben wir nur unitäre Darstellungen der Symmetriegruppe studiert.
Es gibt aber eine mögliche Abschwächung. Wir wissen, dass ein physikalischer
Zustand nicht durch einen Vektor im Hilbertraum beschrieben wird, sondern lediglich
durch einen Strahl. Alle Relationen auf dem Hilbertraum müssen deshalb nur bis auf
Phasen richtig sein. Es genügt also wenn die Symmetriegruppe projektiv dargestellt
wird, d.h. falls gilt
\begin{align*}
    U(g_1) U(g_2) = c(g_1,g_2)U(g_1 g_2)
\end{align*}
wobei $c(g_1 , g_2)$ eine Phase ist. Wegen der Assoziativität gilt die
Kozyklenbedingung
\begin{align*}
    c(g_1 , g_2 g_3) c(g_2 , g_3) = c(g_1 g_2 , g_3) c(g_1 , g_2)
\end{align*}
Sei $U(g)$ eine echte Darstellung von $g$, dann definiere
\begin{align*}
    \hat{U}(g) = U(g) c(g)
\end{align*}
wobei $c(g) \ \forall g \in G$ eine Phase ist. $\hat{U} (g)$ definiert eine
projektive Darstellung, wobei
\begin{align*}
    c(g,h) = \frac{c(g) c(h)}{c(g h)}
\end{align*}
Die projektiven Darstellungen einer Lie Gruppe $G$ stehen in eins zu eins
Korrespondenz zu den echten Darstellungen der universellen Überlagerungsgruppe
$\tilde{G}$ von $G$. Die Darstellungen von $SU(2)$ werden durch $\D_j$ parametrisiert.
Falls $j \in \N$, definiert diese Darstellung tatsächlich eine echte Darstellung von
$SO(3)$; andernfalls ist die Darstellung lediglich eine projektive Darstellung
von $SO(3)$.

\subsection{Der Spin des Elektrons}

Die Möglichkeit dass $j \in \N + \frac{1}{2}$, wird vom Elektron realisiert.
Evidenz dafür ist, dass der Eigenraum eines rotationssymmetrischen Hamiltonoperators
es ebenfalls ist une es somit die projektive Darstellung der $SO(3)$ trägt.
Die Vielfachheit (Entartung) des Eigenwertes ist $2j+1$; sie wird durch eine nicht
rotationsinvariante Störung des Hamiltonoperators, wie z.B. infolge eines äusseren
Magnetfeldes $\vec{B}$, aufgehoben.
Um die Entartung zu erklären, führt man einen zusätzlichen Freiheitsgrad ein: den
Spin. Der Hilbertraum eines einzelnen Elektrons ist dann
\begin{align*}
    \H = L^2 (\R^3) \otimes \C^2
\end{align*}
Auf diesem Raum wirkt $V \in SU(2)$ gemäss
\begin{align*}
    U(V) = U_0 (R(V)) \otimes V
\end{align*}
wobei $U_0(R)$ die Darstellung von $R \in SO(3)$ auf $L^2(\R^3)$ ist. Für den Fall
des Elektrons transformiert sich der interne Freiheitsgrad also in der Darstellung
$F_{\frac{1}{2}}$. In dieser Darstellung $\D_{1/2}$ ist $M_j = i U(A_j)$ gegeben
durch
\begin{align*}
    M_j = \frac{\sigma_j}{2}
\end{align*}
Damit ist
\begin{align*}
    M_+ = \begin{pmatrix}
        0 & 1 \\ 0 & 0
    \end{pmatrix}
    \hspace{10pt} , \hspace{10pt}
    M_- = \begin{pmatrix}
        0 & 0 \\ 1 & 0
    \end{pmatrix}
\end{align*}
und die Basis ist gerade gegeben durch die Standardbasis für $\C^2$
\begin{align*}
    | \frac{1}{2} \rangle = \begin{pmatrix}
        1 \\ 0
    \end{pmatrix} \equiv | e_3 \rangle
    \hspace{10pt} , \hspace{10pt}
    | \frac{1}{2} \rangle = \begin{pmatrix}
        0 \\ 1
    \end{pmatrix} \equiv |-e_3 \rangle
\end{align*}
Spin nach oben, bzw. unten bzgl der Quantisierungsrichtung $e_3$. Eigenbasen für
$M_1$ bzw. $M_2$ sind
\begin{align*}
    &M_1 : \hspace{5pt} |e_1\rangle = \frac{e^{-i \pi /4}}{\sqrt{2}} \begin{pmatrix}
        1 \\ 1
    \end{pmatrix} \hspace{10pt} , \hspace{10pt}
    |-e_1 \rangle = \frac{e^{i \pi /4}}{\sqrt{2}} \begin{pmatrix}
        -1 \\ 1
    \end{pmatrix}
    \\
    &M_2 : \hspace{5pt} |e_2\rangle = \frac{e^{i \pi / 4}}{\sqrt{2}} \begin{pmatrix}
        1 \\ i
    \end{pmatrix} \hspace{10pt} , \hspace{10pt}
    |-e_2\rangle = \frac{e^{-i \pi /4}}{\sqrt{2}} \begin{pmatrix}
        i \\ 1
    \end{pmatrix}
\end{align*}

\subsection{Wigners Theorem}

Wir haben gesehen, dass es genügt dass Darstellung der Symmetriegruppe projektiv ist.
Im Allgemeinen genügt es, wenn die Symmetrie durch anti-unitäre Operatoren $U(g)$
dargestellt wird. Ein anti-unitärer Operator had die Eigenschaft, dass
\begin{align*}
    \left\langle U(g) \phi | U(g) \psi \right\rangle =
    \overline{\left\langle \phi | \right\rangle}
    = \left\langle \psi | \phi \right\rangle
\end{align*}
Das Wignersche Theorem besagt: 
\begin{theorem}[Wigners Theorem]
    Sei $G$ eine Gruppe die auf den Zuständen des
    Hilbertraumes $\H$ so wirkt, dass die Wsk
    $\frac{\abs{\left\langle \phi | \psi \right\rangle}^2}{\left\langle \phi | \phi \right\rangle \left\langle \psi | \psi \right\rangle}$
    erhalten bleibt. Dann kann man die Wirkung von $g \in G$ als $\psi \mapsto U(g) \psi$
    schreiben, wobei jedes $U(g)$ ein unitärer oder anti-unitärer Operator ist, und
    \begin{align*}
        U(g h) = c(g,h) U(g) U(h)
    \end{align*}
    gilt, wobei $c(g,h)$ die Kozyklenphasen sind.
\end{theorem}

Ein Beispiel von Symmetrien, die durch anti-unitäre Operatoren dargestellt wird,
ist die Zeitumkehr. Wir definieren die Wirkung des Zeitumkehroperators $\T$ auf
den Wellenfunktionen durch
\begin{align*}
    \klammer{\T \Psi} (t,\vec{x}) = \overline{\Psi} (-t,\vec{x})
\end{align*}
Die meisten Symmetrien werden aber durch unitäre Operatoren dargestellt.
Es gilt:
\begin{lemma}[Weyl'sches Lemma]
    Der Operator $U(g^2)$ ist immer unitär $\forall g \in G$.
\end{lemma}
Aus diesem Resultat folgt, dass jede Rotation durch unitäre Operatoren dargestellt
werden muss.
