\section{Drehimpulsaddition}

Wir wollen verstehen, wie sich der Zustandsraum eines Quantensystems mehrerer
Teilchen beschreiben lässt durch den Zustandsraum der einzelnen Teilsysteme.

\subsection{Tensorprodukt}

Der Hilbertraum eines Systems, das aus zwei Teilsystemen zusammengesetzt ist, ist
das Tensorprodukt der Hilberträume der Teilsysteme
\begin{align*}
    \H = \H^{(1)} \otimes \H^{(2)}
\end{align*}
Ein Beispiel ist das 2-Körperproblem. Hier gilt $L^2(\R^6) = L^2(\R^3) \otimes
L^2(\R^3)$. D.h. der Hilbertraum der 2-Teilchen-Wellenfkten $\psi(\vec{x}_1,\vec{x}_2)$
wird aufgespannt durch
\begin{align*}
    \psi^{(1)} (\vec{x}_1) \psi^{(2)}(\vec{x}_2) = \klammer{\psi^{(1)} \otimes
    \psi^{(2)}} \klammer{\vec{x}_1 , \vec{x}_2}
\end{align*}
D.h. jede Fkt $\psi(x_1,x_2)$ lässt sich als Summe von Produktfunktionen schreiben.
Wenn $e_i^{(1)}$ bzw $e_i^{(2)}$ eine Basis von $\H^{(1)}$ bzw $\H^{(2)}$ ist,
dann ist die Basis von $\H = \H^{(1)} \otimes \H^{(2)}$ gegeben durch
\begin{align*}
    e_i^{(1)} \otimes e_j^{(2)}
\end{align*}
Somit gilt auch $\dim(\H) = \dim \klammer{\H^{(1)}} \cdot \dim \klammer{\H^{(2)}}$.
Das Skalarprodukt ist definiert als
\begin{align*}
    \left\langle v_1 \otimes w_1 | v_2 \otimes w_2 \right\rangle
    = \left\langle v_1 | v_2 \right\rangle^{(1)} \cdot
        \left\langle w_1 | w_2 \right\rangle^{(2)}
\end{align*}
Ein allgemeiner Vektor in $\H^{(1)} \otimes \H^{(2)}$ ist gegeben als
\begin{align*}
    \sum_{m,n} v_m \otimes w_n
    \hspace{10pt} \text{für } v_m \in \H^{(1)} \ , \ w_n \in \H^{(2)}
\end{align*}

\subsection{Addition von Drehimpulsen}

Betrachten den Fall wo beide separaten Hilberträume eine (projektive) Darstellung
der Drehgruppe tragen. Wir betrachten das Beispiel des $2$-Körperproblems. Hier
ist der Hamiltonoperator des Gesamtsystems nicht invariant unter den separaten
Rotationen, sondern lediglich under den gemeinsamen Rotationen. D.h. der
Hamiltonoperator vertauscht nur mit den Generatoren der gemeinsamen Rotationen.
Betrachte denn Fall wo die Beiden Hilberträume $\H^{(1)}$ und $\H^{(2)}$ die
Darstellungen $U^{(i)}$ der Rotationsgruppe $SO(3)$ bzw. $SU(2)$ tragen.
$\forall g \in SU(2)$ gilt
\begin{align*}
    U^{(i)} (g) : \H^{(i)} \rightarrow \H^{(i)}
    \hspace{10pt} , \hspace{10pt}
    U^{(i)}(g) U^{(i)}(h) = U^{(i)} (gh)
\end{align*}
Auf dem Tensorprodukt $\H^{(1)} \otimes \H^{(2)}$ haben wir die Darstellung der
Produktgruppe $SU(2) \times SU(2)$ gegeben durch
\begin{align*}
    (g,h) &\mapsto U^{(1)} (g) \otimes U^{(2)} (h)
    \\
    \klammer{U^{(1)} (g) \otimes U^{(2)} (h)} (\psi_1 \otimes \psi_2)
    &= \klammer{\klammer{U^{(1)} (g) \psi_1} \otimes \klammer{U^{(2)} (h) \psi_2}}
\end{align*}
Wir interessieren uns für gemeinsame Rotationen, also:
\begin{align*}
    g \mapsto U^{(1)} (g) \otimes U^{(2)} (g)
\end{align*}
Betrachte nun eine Kurve $g(t)$ mit $g(0) = \text{id}$ und den Lie-Algebra Generator
\begin{align*}
    \Omega = \frac{d}{dt}\Big|_{t=0} g(t) 
\end{align*}
Dann gilt:
\begin{align*}
    \klammer{U^{(1)} \otimes U^{(2)}}(\Omega)
    = U^{(1)} (\Omega) \otimes \mathds{1} + \mathds{1} \otimes U^{(2)} (\Omega)
    = U(\Delta (g)) (v \otimes w)
\end{align*}
Wobei wir die Komultiplikation definiert haben als
\begin{align*}
    \Delta : g \mapsto g \otimes g
    \hspace{10pt} , \hspace{10pt}
    \Delta(\Omega) = \Omega \otimes \mathds{1} + \mathds{1} \otimes \Omega
\end{align*}
Es gilt:
\begin{align*}
    \eckigeklammer{\Delta(\Omega_1),\Delta(\Omega_2)} &= \Delta(\eckigeklammer{\Omega_1,\Omega_2})
    \\
    \eckigeklammer{\klammer{U^{(1)} \otimes U^{(2)}} (\Omega_1) , \klammer{U^{(1)} \otimes U^{(2)}} (\Omega_2)}
    &= \klammer{U^{(1)} \otimes U^{(2)}} \klammer{\eckigeklammer{\Omega_1 , \Omega_2}}
\end{align*}

% \subsection{Addition zweier $j=\frac{1}{2}$ Darstellungen}

% Viellleicht noch hinzufügen für Beispiele

\subsection{Die Clebsch-Gordan Reihe im allgemeinen Fall}

Allgemein gilt die Zerlegung (Clebsch-Gordan Reihe):
\begin{align*}
    D_{j_1} \otimes D_{j_2} = D_{j_1 + j_2} \oplus D_{j_1 + j_2 - 1} \oplus \dotsb \oplus D_{\abs{j_1 - j_2}}
\end{align*}


\subsection{Clebsch-Gordan Koeffizienten}

Wie lassen sich die verschiedenen Vektoren $|j,m\rangle$ der Darstellungen $D_j$
durch die Vektoren des Tensorproduktes ausdrücken. Dazu bezeichnen die Vektoren
des Tensorproduktes durch
\begin{align*}
    |j_1,j_2,m_1,m_2\rangle \equiv |j_1,m_1\rangle \otimes |j_2,m_2\rangle
    \in D_{j_1} \otimes D_{j_2}
\end{align*}
und jene der Darstellung $D_j$ die in der Clebsch-Gordan Reihe vorkommen durch
\begin{align*}
    |j_1,j_1,j,m\rangle = \sum_{\stackrel{m_1,m_2}{m_1 + m_2 = m}}
    \underbrace{\left\langle j_1,j_2,m_1,m_2 | j_1,j_2,j,m \right\rangle}_{\text{C-G Koeff.}}
    |j_1,j_2,m_1,m_2\rangle
\end{align*}
Für die Clebsch-Gordan Koeffizienten gelten die Dreiecksrekursionen
\begin{align*}
    &\sqrt{j(j+1) - m(m+1)} \left\langle j_1 j_2 m_1 m_2 | j_1 j_2 j(m+1) \right\rangle
    \\
    &\hspace{5pt} = \sqrt{j_1(j_1 + 1) - m_1(m_1 -1)} \left\langle j_1 j_2 (m_1 -1) m_2 | j_1 j_2 j m \right\rangle
    \\
    &\hspace{10pt} + \sqrt{j_2(j_2 + 1) - m_2(m_2 - 1)} \left\langle j_1 j_2 m_1 (m_2 - 1) | j_1 j_2 j m \right\rangle
\end{align*}
\begin{align*}
    &\sqrt{j(j+1) - m(m-1)} \left\langle j_1 j_2 m_1 m_2 | j_1 j_2 j(m-1) \right\rangle
    \\
    &\hspace{5pt} = \sqrt{j_1(j_1 + 1) - m_1(m_1 + 1)} \left\langle j_1 j_2 (m_1 + 1) m_2 | j_1 j_2 j m \right\rangle
    \\
    &\hspace{10pt} + \sqrt{j_2(j_2 + 1) - m_2(m_2 + 1)} \left\langle j_1 j_2 m_1 (m_2 + 1) | j_1 j_2 j m \right\rangle
\end{align*}
Unser Ziel ist es bei festem aber beliebigem $j$ mit $\abs{j_1 - j_2} \leq j
\leq j_1 + j_2$ die Koeffizienten für alle $m_1$, $m_2$ zu finden.
Wir finden die Rekursion:
\begin{align*}
    &\sqrt{j(j+1) - m(m-1)} \langle j_1 , j_2 , j_1 , m_2 | j_1 , j_2 , j , m-1 \rangle
    \\
    &= \sqrt{j_2(j_2 - 1) - m_2 (m_2 + 1)} \langle j_1 , j_2 , j_1 , m_2 + 1 | j_1 , j_2 , j , m \rangle
\end{align*}

\subsection{Physikalische Beispiele}

\subsubsection{Magnetisches Moment}

Ein Teilchen der Ladung $q$ ($=-e$ für Elektron) und Masse $m$, welches sich auf
einem Orbit mit Drehimpuls $\vec{L}$ bewegt, produziert ein magnetisches Moment
$\vec{M}$. Wir leiten dies nun her.

\begin{align*}
    \vec{E} = - \frac{1}{c} \frac{\partial \vec{A}}{\partial t} - \vec{\nabla} \Phi
    \hspace{10pt} , \hspace{10pt}
    \vec{B} = \vec{\nabla} \wedge \vec{A}
    \hspace{10pt} , \hspace{10pt}
    H = \frac{1}{2m} \klammer{\vec{p} - \frac{q}{c} \vec{A}}^2 + q \Phi
\end{align*}
Wobei $\vec{p} = m \dot{\vec{x}} + \frac{q}{c} \vec{A}$ der zu $\vec{x}$
konjugierte Impuls ist. Gemäss des Korrespondenzprinzips wird $\vec{p}$ zu
$-i \hbar \vec{\nabla}$. Somit folgt für den Hamiltonoperator:
\begin{align*}
    H &= - \frac{\hbar^2}{2m} \Delta + i \frac{\hbar q}{2 m c}
    \klammer{\vec{\nabla} \cdot \vec{A} + \vec{A} \cdot \vec{\nabla}}
    + \klammer{\frac{q^2}{2mc^2} \vec{A}^2 + q \Phi}
    \\
    &\stackrel{(\ast)}{=} - \frac{\hbar^2}{2m} \Delta + i \frac{\hbar q}{m c}
    \vec{A} \cdot \vec{\nabla} + \klammer{\frac{q^2}{2mc^2} \vec{A}^2 + q \Phi}
\end{align*}
Wobei wir in $(\ast)$ die Coulomb-Eichung $\vec{\nabla} \cdot \vec{A} = 0$ gewählt
haben. Das Vektorpotential eines konstanten Magnetfeldes $\vec{B}$ ist gegeben
durch
\begin{align*}
    \vec{A} = - \frac{1}{2} \vec{x} \wedge \vec{B}
\end{align*}
Wir finden:
\begin{align*}
    i \frac{\hbar q}{m c} \vec{A} \cdot \vec{\nabla}
    = i \frac{\hbar q}{2 m c} \klammer{\vec{x} \wedge \vec{\nabla}} \cdot \vec{B}
    = - \frac{q}{2mc} \vec{L} \cdot \vec{B}
\end{align*}
wobei wir $\vec{L} = -i \hbar \klammer{\vec{x} \wedge \vec{\nabla}}$ verwendet haben.
Das magnetische Moment des Teilchens ist also
\begin{align*}
    \vec{M} = \frac{q}{2mc} \vec{L}
\end{align*}
Den Zusatzterm zum Hamiltonian
\begin{align*}
    H_Z = - \vec{M} \cdot \vec{B}
\end{align*}
nennt man den Zeeman Term. Wir definieren weiter
\begin{align*}
    \Delta H = \klammer{\frac{q^2}{2mc^2} \vec{A}^2 + q \Phi}
\end{align*}
Der Zeeman Term führt dazu, dass die für rotationssymmetrische Probleme
typische Entartung der Energieniveaus aufgehoben wird. Betrachte das Beispiel
des Wasserstoffatoms. Die Energiezustände ohne einen Einfluss eines Magnetfeldes
sind durch die Quantenzahlen $(n,l,m)$ wie folgt parametrisiert:
\begin{align*}
    E_n = - \frac{me^4}{2\hbar^2} \cdot \frac{1}{n^2}
\end{align*}
Wenn wir ein homogenes Magnetfeld einschalten erhalten wir
\begin{align*}
    E_{n,m} = E_n + \Delta E_Z
    \hspace{10pt} , \hspace{10pt}
    \Delta E_Z = \mu_B B_z m
    \hspace{10pt} , \hspace{10pt}
    \mu_B = \frac{e \hbar}{2 m_e c}
\end{align*}
wobei $\Delta E_Z$ ein Korrekturterm ist und $\mu_B$ das Bohr'sche Magneton
ist.

\subsection{Der anormale Zeeman Effekt}

Der Spin eines Teilchens erzeugt auch ein magnetisches Moment, sogar für
ungeladene Teilchen. Für ein Elektron:
\begin{align*}
    \vec{M}_{\text{el}} = -g \frac{e}{2 m c} \vec{S}
    \hspace{10pt} , \hspace{10pt}
    g = 2 + \frac{\alpha}{\pi} + \mathcal{O} (\alpha^2)
    \hspace{10pt} , \hspace{10pt}
    \alpha = \frac{e^2}{\hbar c} = \frac{1}{137.04}
\end{align*}
Wobei $g$ der Landé Faktor und $\alpha$ die Feinstrukturkonstante sind.
In diesem Fall sind $\vec{M}$ und $\vec{S}$ antiparallel, da $q=-e<0$.
Für ein Proton und ein Neutron gilt
\begin{align*}
    \vec{M}_{\text{prot}} &= g_p \frac{e}{2 M_p c} \vec{S}
    \hspace{10pt} , \hspace{10pt}
    g_p \approx 5.59
    \\
    \vec{M}_{\text{neut}} &= - g_n \frac{e}{2 M_n c} \vec{S}
    \hspace{10pt} , \hspace{10pt}
    g_n \approx 3.83
\end{align*}
Bewegt sich das Teilchen in einem externen Magnetfeld $\vec{B} = B e_z$, so
erhält man den gesamten Zeemann Term.
\begin{align*}
    H_Z = \frac{e B_z}{2 m c} \klammer{L_z + g S_z}
\end{align*}
Der 'anormale Zeemann Effekt' beschreibt den Effekt, der vom Spin erzeugt wird.
Die Quantenzahlen des Elektrons sind durch $(n,l,m,m_s)$ gegeben, wobei
$m_s = \pm \frac{1}{2}$ die Spinquantenzahl ist. Mit $g = 2$ erhält man:
\begin{align*}
    \Delta E_Z = \mu_B B_z (m + 2 m_s)
\end{align*}

\subsubsection{Spin-Bahn-Kopplung}

Das sich im $\vec{E}$-Feld des Kernes bewegende Elektron spürt ein Magnetfeld
$\vec{B} = - \vec{v} \wedge \vec{E} /c$, woran das Spin-Moment koppelt. Die
resultierende Energie ist
\begin{align}\label{eq:SpinBahnKopplung}
    - \frac{e}{m c^2} \vec{S} \cdot \klammer{\vec{v} \wedge \vec{E}}
\end{align}
Die Thomas-Präzession führt eine weitere Korrektur ein. Diese ist gerade
das $-1/2$-fache von \cref{eq:SpinBahnKopplung}. Wir erhalten folgenden
Ausdruck für die Spin-Bahn Kopplung:
\begin{align*}
    H_{\text{SO}} &=
    \frac{-e}{2mc^2} \vec{S} \cdot \klammer{\vec{v} \wedge \vec{E}}
    \stackrel{(\ast)}{=}
    \frac{1}{2mc^2} \vec{S} \cdot \klammer{\vec{r} \wedge \vec{v}} \frac{1}{r} \partial_r V
    \\
    &= \frac{1}{2 m^2 c^2} \frac{1}{r} \frac{dV}{dr} \vec{L} \cdot \vec{S}
\end{align*}
Wobei wir in $(\ast)$ verwendet haben, dass $e \vec{E} = \vec{\nabla} V
= \hat{r} \partial_r V(r)$. Der Term $\vec{L} \cdot \vec{S}$ ist under der
gemeinsamen Rotation von $\vec{L}$ und $\vec{S}$ invariant. Es gilt:
\begin{align*}
    \vec{L}^2 = \hbar^2 l(l+1)
    \hspace{10pt} &, \hspace{10pt}
    \vec{S}^2 = \hbar^2 \frac{1}{2} \klammer{\frac{1}{2} + 1} = \hbar^2 \frac{3}{4}
    \\
    \klammer{\vec{L} + \vec{S}}^2 = \hbar^2 J (J+1)
    \hspace{10pt} &, \hspace{10pt} J = l \pm \frac{1}{2}
\end{align*}
Hierbei ist $J$ der Spin der Darstellung, in die sich das Tensorprodukt $D_l \otimes
D_{\frac{1}{2}}$ zerlegt. $\vec{L}$ und $\vec{S}$ stimmen bis auf einen Faktor von
$\hbar$ gerade mit den infinitesimalen Rotationsgeneratoren überein. $\vec{L} +
\vec{S}$ ist derade die Wirkung dieser Generatoren auf dem Tensorprodukt. Es gilt:
\begin{align*}
    \vec{L} \cdot \vec{S} = \frac{1}{2} \klammer{\klammer{\vec{L} + \vec{S}}^2 - \vec{L}^2 - \vec{S}^2}
    = \begin{cases}
        \frac{\hbar^2}{2} l \hspace{5pt} &J=l+\frac{1}{2}
        \\
        - \frac{\hbar^2}{2} (l+1) \hspace{5pt} &J=l-\frac{1}{2}
    \end{cases}
\end{align*}
Weiter gilt:
\begin{align*}
    \frac{1}{r} \partial_r V = \frac{e^2}{r^3}
    \hspace{10pt} \Rightarrow \hspace{10pt}
    \Delta E_{SB} = \frac{e^2}{2 m^2 c^2} \frac{\hbar^2}{2}
        \left\langle \frac{1}{r^3} \right\rangle_{nl}
    \begin{cases}
        l \hspace{5pt} &J=l+\frac{1}{2}
        \\
        -(l+1) \hspace{5pt} &J=l-\frac{1}{2}
    \end{cases}
\end{align*}
Da folgendes gilt:
\begin{align*}
    \left\langle \frac{1}{r^3} \right\rangle_{nl}
    = \frac{1}{a_0^3} \frac{1}{n^3} \frac{1}{l \klammer{l+\frac{1}{2}} \klammer{l+1}}
    \hspace{10pt} &, \hspace{10pt}
    a_0 = \frac{\hbar^2}{m e^2}
    \\
    \frac{e^2 \hbar^2}{m^2 c^2} = \frac{e^4}{\hbar^2 c^2} \frac{\hbar^4}{e^2 m^2}
    = \alpha^2 e^2 a_0^2
    \hspace{10pt} &, \hspace{10pt}
    \frac{e^4 m}{2 \hbar^2} = \frac{e^2}{2 a_0} = E_R = 13.6 \ \mathrm{eV}
\end{align*}
folgt:
\begin{align*}
    \Delta E_{SB} = \frac{\alpha^2 E_R}{2 n^3} \frac{1}{l \klammer{l+\frac{1}{2}} \klammer{l+1}}
    \begin{cases}
        l \hspace{5pt} &J = l+\frac{1}{2}
        \\
        -(l+1) \hspace{5pt} & J=l-\frac{1}{2}
    \end{cases}
\end{align*}
Für kleine externe Magnetfelder ist der Spin-Bahn-Kopplungseffekt relativ
zum Zeemann-Effekt dominant.
